\documentclass[a4paper,10pt]{report}

\usepackage[in]{fullpage}
\usepackage{parskip}
\usepackage{hyperref}
\usepackage{tikz}
\usepackage{amsmath}
\usepackage[backend=bibtex,style=numeric-comp,sorting=nyt,sortcites=true,maxnames=4]{biblatex}
\usepackage[section]{placeins}
\usepackage[color]{circus}
\usepackage{fixltx2e}

\title{A Framework for Verifying Safety-Critical Java Virtual Machines}
\author{James Baxter}
\date{}

\bibliography{../Qualifying-Dissertation/literature} 

%TC:group zed 0 displaymath
%TC:group axdef 0 displaymath
%TC:group schema 1 displaymath
%TC:group circus 0 displaymath
%TC:group circusaction 0 displaymath
%TC:macroword \Circus 1 

\begin{document}
\maketitle

\begin{abstract}
  Abstract goes here
\end{abstract}

\tableofcontents

\chapter{Introduction}

% short explanation of chapter: motivation, objectives, document
% structure

\section{Motivation}

The Java programming language~\cite{gosling2013} has achieved great
popularity and is now used in a wide variety of areas.
One particular area of interest in which Java is used is that of
embedded systems, where it was realised that the features of
portability, modularity, safety and security that Java offers could be
of use in those areas~\cite{mulchandani1998}.
As Java programs are compiled to be run by the Java Virtual Machine
(JVM), this required creating JVMs for embedded devices and, indeed,
research has gone into making smaller and smaller JVMs to accommodate
a wider range of embedded devices~\cite{caska2011,thomm2010}.

However, many embedded systems have real-time requirements, meaning
that strict timing properties are required to be adhered to.
Features of Java such as the garbage collector and its concurrency
model make it difficult to ensure such properties, thus making
standard Java unsuitable for real-time systems.
To address this problem, the Real-Time Specification for Java
(RTSJ)~\cite{gosling2000} was created.
The RTSJ extends Java with some additional features that allow for
greater time predictability.
One feature the RTSJ adds is scoped memory, allowing memory to
allocated for use in a particular scope and deallocated when that
scope in left, thus allowing for prediction of when memory
deallocations will occur.
Another feature of the RTSJ is a more predictable priority scheduler
that better allows the execution time of higher priority threads to be
computed.

Though the RTSJ provides for real-time considerations, many real-time
embedded systems require certification under safety standards such as
\mbox{DO-178C} and ISO~26262.
To better facilitate the certification of such systems,
Safety-Critical Java (SCJ)~\cite{locke2013} was created.
SCJ is based on the RTSJ but removes the features of the RTSJ that
would make it diffcult to certify programs written using it.
Some particular differences of SCJ from standard Java are in the areas
of scheduling and memory management.

SCJ's scheduling model is based around organising a program into a
series of missions, which are executed sequentially in an order chosen
by a programmer-supplied mission sequencer.
Each mission may contain multiple asynchronous event handler threads,
which are created when the mission is initialised and run throughout
the duration of the mission.
The event handlers within a mission are scheduled according to a
priority scheduling system, in which the running handler is always the
highest priority relesed handler.
Once the mission is requested to terminate, the event handler threads
are stopped, the mission's resources are cleaned up, and the mission
sequencer selects the next mission.

The memory management of SCJ makes use of the scoped memory areas of
RTSJ, adapted for the SCJ mission model.
Memory is allocated in memory areas, which have varying lifetimes.
Each mission has a memory area associated with it that persists for
the duration of the mission, while each event handler within the
mission has a memory area that persists only during releases of the
handler.
Additional nested private memory areas can be created, which can be
entered and left as needed, being cleared when they are left.
There is also an immortal memory area, in which memory is never freed
and persists for the entire duration of the program.
A system of annotations allows for static checking that dangling
pointers cannot arise from misuse of the scoped memory
model~\cite{tang2010}.

Due to the safety-critical nature of the software SCJ is designed to
create, it is necessary to show that they are correct.
While the design of SCJ allows for it to be more easily shown that
programs will execute correctly, greater certainty and precision can
be obtained through the use of formal methods.
There has already been work on generating correct SCJ programs from
formal specifications~\cite{cavalcanti2011, cavalcanti2013}, which
ensures that an SCJ program will have the behaviour intended.
Some work has also been done on formalising the SCJ memory
model~\cite{cavalcanti2011a}, allowing it's correctness to be ensured.

However, although SCJ can be compiled by a standard Java compiler, SCJ
requires a specialised virtual machine due to the differences between
SCJ and standard Java.
There have been several SCJ virtual machines (SCJVMs) created,
including the icecap HVM~\cite{sondergaard2012}, Fiji
VM~\cite{pizlo2009}, OVM~\cite{armbruster2007},
HVM\textsubscript{TP}~\cite{luckow2014} and PERC Pico~\cite{atego2015,
  richard2010}.
Of these, only Fiji VM and icecap HVM appear to be maintained and,
while icecap provides support for most of SCJ, Fiji VM does not
necessariy provide proper support for all aspects of SCJ.
Since the correct execution of an SCJ program depends on the
correctness of the underlying virtual machine, which must be ensured
in addition to checking the correctness of the program.
While there has been work on using formal methods to show SCJ program
correctness, no SCJVM has been verified.

Additionally, having to execute a program via a virtual machine
presents a cost in terms of memory and time resources, which may not
be available on the embedded real-time systems SCJ is targetting.
For this reason, most SCJVMs, including the ones listed above compile
the Java bytecode to native code ahead-of-time.
This can be particularly seen in the examples of Fiji VM and the
icecap HVM, which compile Java bytecode to C code.
The correctness of this compilation to native code must also be
ensured, in addition to the correctness of the infrastructure
supporting the running program, and so the problem of ensuring the
correctness of an SCJVM is, in part, a compiler verification problem.

There has already been much research in the area of compiler
correctness, with most of the research following one of two
approaches.
The majority of the literature on compiler correctness follows a
commuting-diagram approach, in which the compilation function is shown
to commute with functions defining the semantics of the source and
target languages.
This approach was first identified by Lockwood
Morris~\cite{morris1973} and later refined by Thatcher \emph{et
  al.}~\cite{thatcher1979}, but can be seen in much of the earlier
work, including the earliest work by McCarthy and
Painter~\cite{mccarthy1967}.
The commuting-diagram approach has also been used in more recent work,
including some very comprehensive work as part of the CompCert
project~\cite{leroy2009a, leroy2009b}.
As this approach is based on the use of ordinary functions to describe
compiler correctness, it can be readily encoded in an automated
theorem prover and there have been several works that have made use of
automated theorem provers~\cite{klein2006, milner1972, nipkow2000}.

The other main approach to compiler verification is the algebraic
approach proposed by Hoare~\cite{hoare1991} and implemented by
Sampaio~\cite{hoare1993, sampaio1993}.
The algebraic approach is based around the notion of refinement, which
formally captures the idea of a program being an implementation of a
less deterministic specification.
In the algebraic approach, the source and target languages are defined
in the same semantic space and proven refinement laws are used to
refine the source program to a normal form representing the target
machine running the target code.
This approach has the advantage that the rules used to perform the
compilation are known to be correct, thus making the resultant
compiler correct by construction.
It also allows for passing seamlessly from the source language to the
target language using algebraic laws since the source and target
languages are in the same semantic space.
The algebraic approach has not been used particularly widely but, in
addition to Sampaio's work, there has been work using the algebraic
approach for compilation of object-oriented languages~\cite{duran2005,
  duran2010} and for hardware compilation~\cite{perna2010, perna2011}.


\section{Objectives}

Since there appears to be no formally verified SCJVM and formal
verification of an SCJVM is desirable, we propose to create a
framework for formal verification of an SCJVM.
This framework will consist of the following parts:
\begin{itemize}
\item A specification of the services required to be provided by an
  SCJVM,
\item A compilation strategy from Java bytecode to C,
\item A formal model of the specification and compilation strategy,
\item Proofs of the correctness of the formal model, and
\item A mechanisation of the proofs in an automated theorem prover.
\end{itemize}
Each of these parts will be discussed in the following paragraphs.

First, it is necessary to have an informal specification of what is
required of an SCJVM.
Even though the ultimate aim is one of formal verification, an
informal specification is still needed to guide the development of the
formal model since there is no existing clear specification for SCJVMs
and some VM implementors may find the formal model difficult to
understand.
Concerning the content of the specification, it should be noted that
the role of an SCJVM is not merely to execute bytecode instructions.
An SCJVM must also provide services to support the SCJ API in areas
such as scheduling and memory management.
An informal specification of all aspects of the SCJVM is required for
proper creation of a formal model, but the semantics of SCJ bytecode
does not differ much from that of standard Java bytecode so much of
the Java Virtual Machine specification~\cite{lindholm2014} also
applies to SCJVMs.

Secondly, the need for a compilation strategy arises from the fact
that most exising SCJVMs compile Java bytecode to some native code in
order to improve performance on embedded system so it seems wise to
include it in our specification of an SCJVM.
We will focus on compilation from Java bytecode to C for much the same
reasons as the icecap HVM and Fiji VM: that C is a language already
widely used in embedded systems and is suffciently low level to enable
its use as an efficient target language while retaining enough
abstraction to be interacted with by the programmer.

The specification and compilation strategy must then be formalised in
some formal notation.
As noted previously, there are two main approaches to formalising
compilers and showing their correctness: the commuting-diagram
approach and the algebraic approach.
In the commuting diagram approach, the compiler is viewed as a
function from some representation of the source language to some
representation of the target language.
The semantics of the source and target lanaguages are then defined as
functions into some space of meanings, with a function to relate the
meanings of the source and target languages if necessary.
The correctness of the compiler is then checked by showing that the
functions are all consistent with one another, i.e.\ that a diagram
with the functions as edges commutes.

The algebraic approach is based on defining the source and target
languages using the same specification language.
The notion of refinement can then be used to prove algebraic laws
that can be applied to transform the source program into a normal form
representing the target machine running the target code.
Because the algebraic laws are proved correct and the compilation is
made up of applying these laws in a systematic way, the compilation
can then be known to be correct in the sense that the target code
running on the target machine is a correct implementation of the
program specified by the source code.

We will follow the algebraic approach in our work as it relates the
source and target languages in a clearer way than the
commuting-diagram approach.
Additionally, the commuting-diagram approach requires the definition
of additional functions to relate the source and target semantics,
thus relying more on definitions, which may be incorrect, than on
mathematically derived results.
However, there has been some work on deriving a compilation function
given an operational sematics of the source language and a way of
relating the semantics of the source and target
languages~\cite{bahr2015}.
We do not believe this work will be of use in the case we wish to
apply it as it relies on knowledge of the state of the target machine
and a way of identifying the semantics of the target code within the
compilation function, which is not easily done with a complex language
like C, though it may work better for machine languages.

We require a formal language in which to specify the source and target
languages, as well as for specifying the virtual machine services.
For this we will use the \Circus{} specification
language~\cite{oliveira2009}, which is based on Z
notation~\cite{woodcock1996} and CSP~\cite{roscoe2011}.
\Circus{} uses CSP to specify processes that communicate over channels
and uses Z notation to specify state and data operations that can be
encapsulated in processes.
The reason for the choice of \Circus{} as our formal notation is due
to the fact that \Circus{} is designed as a notation for refinement
and so works well with the algebraic approach.
The combination of Z notation and CSP also gives far greater
expressive power than using those languages separately.
Finally, the use of \Circus{} helps to tie this work into the existing
specification work surrounding SCJ~\cite{cavalcanti2011,
  cavalcanti2011a, cavalcanti2013, zeyda2011}.

The consistency of the specification and the correctness of the
compilation strategy must be proved by formal mathematical proof from
algebraic laws known to be correct.
\Circus{} has many such laws already developed and has an underlying
semantics given using the model of Hoare and He's Unifying Thories of
Programming (UTP)~\cite{hoare1998} that can be used to verify the
correctness of the laws.

Finally, to ensure the laws are understood and applied correctly, it
is often helpful to mechanise the proofs in an automated theorem
prover so that the proofs can then be machine checked.
There are a variety of tools for machine checking of \Circus{}
specifications.
The \Circus{} parser and typechecker included as part of the Community
Z Tools~\cite{malik2011, xavier2008, malik2005, miller2005} can be
used to perform basic checking of a \Circus{} specification and
integration with the Z/Eves theorem prover~\cite{saaltink1997} can be
used to prove the Z portions of the \Circus{} specification.
Process properties such as deadlock-freedom can be checked by
translating the \Circus{} specification to CSP and checking with a
tool such as FDR~\cite{gibson-robinson2014}.
For more comprehensive proving of properties about \Circus
specifications, recent work on Isabelle/UTP~\cite{foster2015} can be
used to permit reasoning about \Circus{} in the Isabelle theorem
prover~\cite{nipkow2002}.

\printbibliography


\end{document}