\section{Elimination of Program Counter}

\subsection{Expand Bytecode}

\begin{minipage}{\textwidth}
  \PCExpansionRule*
\end{minipage}

\begin{minipage}{\textwidth}
  \HandleInstructionRefinementRule*
\end{minipage}

\begin{minipage}{\textwidth}
  \CheckSynchronizedReturnSynchronizedRefinementRule*
\end{minipage}

\begin{minipage}{\textwidth}
\begin{restatable}[$CheckSynchronizedReturn$-nonsync-refinement]{crule}{CheckSynchronizedReturnNonsynchronizedRefinementRule}
  \label{CheckSynchronizedReturn-nonsynchronized-refinement-rule}
  Given $i : ProgramAddress$,
  \setlength{\zedindent}{0.5cm}
  \setlength{\zedtab}{0.5cm}
  \begin{circus}
    \begin{array}{l}
      \circif {} \cdots {} \\
      {} \circelse pc = i \circthen CheckSynchronizedReturn \circseq A \\
      {} \cdots {} \\
      \circfi
    \end{array}
    \circrefines_A
    \begin{array}{l}
      \circif {} \cdots {} \\
      {} \circelse pc = i \circthen A \\
      {} \cdots {} \\
      \circfi
    \end{array}
  \end{circus}
  provided
  \begin{displaymath}
    \exists c : Class; m : MethodID | \\
    \t1 c \in \ran cs \land m \in \dom c.methodEntry @ \\
    \t1 i \in c.methodEntry~m \upto c.methodEnd~m \land \\
    \t1 m \notin c.synchronisedMethods \lor m \in c.staticMethods
  \end{displaymath}
\end{restatable}
\end{minipage}

\subsection{Introduce Sequential Composition}

\begin{minipage}{\textwidth}
\SequenceIntroductionRule*
\end{minipage}

\subsection{Introduce Loops and Conditionals}

\begin{minipage}{\textwidth}
  \IfConditionalIntroductionRule*
\end{minipage}  

\begin{minipage}{\textwidth}
\begin{restatable}[\texttt{if}-\texttt{else}-conditional-intro]{crule}{IfElseConditionalIntroductionRule}
  \label{if-else-introduction-rule}
  \setlength{\zedindent}{0.25cm}
  \setlength{\zedtab}{0.57cm}
  % \setlength{\abovedisplayskip}{0.1cm}
  % \setlength{\belowdisplayskip}{0.1cm}
  Given $i : ProgramAddress$, if $i \neq j$, $i \neq k$, and 
  \begin{circus}
    \{frameStack \neq \emptyset\} \circseq A \circseq P \\
    {} = {} \\
    \{frameStack \neq \emptyset\} \circseq A \circseq P \circseq \{frameStack \neq \emptyset\}
  \end{circus}
  then
  \begin{circus}
    \begin{array}{l}
      \circmu X \circspot \\
      \t1 \circif frameStack = \emptyset \circthen \Skip \\
      \t1 {} \circelse frameStack \neq \emptyset \circthen {} \\
      \t2 \circif \cdots \\
      \t2 {} \circelse pc = i \circthen A \circseq \\
      \t3 (\circvar value1, value2 : Word \circspot P \circseq \\
      \t3 pc := \IF b \THEN j \ELSE k) \\
      \t2 {} \cdots {} \\
      \t2 {} \circelse pc = j \circthen B \circseq pc := x \\
      \t2 {} \cdots {} \\
      \t2 {} \circelse pc = k \circthen C \circseq pc := x \\
      \t2 {} \cdots {} \\
      \t2 \circfi \circseq Poll \circseq X \\
      \t1 \circfi
    \end{array}
    \circrefines_A
    \begin{array}{l}
      \circmu X \circspot \\
      \t1 \circif frameStack = \emptyset \circthen \Skip \\
      \t1 {} \circelse frameStack \neq \emptyset \circthen {} \\
      \t2 \circif \cdots \\
      \t2 {} \circelse pc = i \circthen A \circseq \\
      \t3 (\circvar value1, value2 : Word \circspot P \circseq \\
      \t3 \circif b \circthen pc := j \circseq Poll \circseq B \\
      \t3 {} \circelse \lnot b \circthen pc := k \circseq Poll \circseq C \\
      \t3 \circfi) \circseq pc := x \\
      \t2 {} \cdots {} \\
      \t2 {} \circelse pc = j \circthen B \circseq pc := x \\
      \t2 {} \cdots {} \\
      \t2 {} \circelse pc = k \circthen C \circseq pc := x \\
      \t2 {} \cdots {} \\
      \t2 \circfi \circseq Poll \circseq X \\
      \t1 \circfi 
    \end{array}
  \end{circus}
\end{restatable}
\end{minipage}

\begin{minipage}{\textwidth}
\begin{restatable}[conditional-intro]{crule}{ConditionalIntroductionRule}
  \label{conditional-introduction-rule}
  \setlength{\zedindent}{0.25cm}
  \setlength{\zedtab}{0.57cm}
  % \setlength{\abovedisplayskip}{0.1cm}
  % \setlength{\belowdisplayskip}{0.1cm}
  Given $i : ProgramAddress$, if $i \neq j$, $i \neq k$, and
  \begin{circus}
    \{frameStack \neq \emptyset\} \circseq A \circseq P \\
    {} = {} \\
    \{frameStack \neq \emptyset\} \circseq A \circseq P \circseq \{frameStack \neq \emptyset\}
  \end{circus}
  then
  \begin{circus}
    \begin{array}{l}
      \circmu X \circspot \\
      \t1 \circif frameStack = \emptyset \circthen \Skip \\
      \t1 {} \circelse frameStack \neq \emptyset \circthen {} \\
      \t2 \circif \cdots \\
      \t2 {} \circelse pc = i \circthen A \circseq \\
      \t3 (\circvar value1, value2 : Word \circspot P \circseq \\
      \t3 pc := \IF b \THEN j \ELSE k) \\
      \t2 {} \cdots {} \\
      \t2 {} \circelse pc = j \circthen B \\
      \t2 {} \cdots {} \\
      \t2 {} \circelse pc = k \circthen C \\
      \t2 {} \cdots {} \\
      \t2 \circfi \circseq Poll \circseq X \\
      \t1 \circfi
    \end{array}
    \circrefines_A
    \begin{array}{l}
      \circmu X \circspot \\
      \t1 \circif frameStack = \emptyset \circthen \Skip \\
      \t1 {} \circelse frameStack \neq \emptyset \circthen {} \\
      \t2 \circif \cdots \\
      \t2 {} \circelse pc = i \circthen A \circseq \\
      \t3 (\circvar value1, value2 : Word \circspot P \circseq \\
      \t3 \circif b \circthen pc := j \circseq Poll \circseq B \\
      \t3 {} \circelse \lnot b \circthen pc := k \circseq Poll \circseq C \\
      \t3 \circfi) \\
      \t2 {} \cdots {} \\
      \t2 {} \circelse pc = j \circthen B \\
      \t2 {} \cdots {} \\
      \t2 {} \circelse pc = k \circthen C \\
      \t2 {} \cdots {} \\
      \t2 \circfi \circseq Poll \circseq X \\
      \t1 \circfi 
    \end{array}
  \end{circus}
\end{restatable}
\end{minipage}

\begin{minipage}{\textwidth}
  \WhileLoopIntroductionRuleA*
\end{minipage}

\begin{minipage}{\textwidth}
\begin{crule}[\texttt{while}-loop-intro2]
  \label{while-introduction-rule2}
  \setlength{\zedindent}{0.2cm}
  \setlength{\zedtab}{0.58cm}
  Given $i : ProgramAddress$, if $i \neq j$,
  \begin{circus}
    \{frameStack \neq \emptyset\} \circseq A \circseq P \\
    {} = {} \\
    \{frameStack \neq \emptyset\} \circseq A \circseq P \circseq \{frameStack \neq \emptyset\}
  \end{circus}
  and
  \begin{circus}
    \{frameStack \neq \emptyset\} \circseq B \circseq \\
    {} = {} \\
    \{frameStack \neq \emptyset\} \circseq B \circseq \{frameStack \neq \emptyset\}
  \end{circus}
  then
  \begin{circus}
    \begin{array}{l}
      \circmu X \circspot \\
      \t1 \circif frameStack = \emptyset \circthen \Skip \\
      \t1 {} \circelse frameStack \neq \emptyset \circthen {} \\
      \t2 \circif \cdots \\
      \t2 {} \circelse pc = i \circthen A \circseq \\
      \t3 (\circvar value1, value2 : Word \circspot P \circseq \\
      \t3 pc := \IF b \THEN j \ELSE k) \\
      \t2 \cdots \\
      \t2 {} \circelse pc = j \circthen B \circseq pc := i \\
      \t2 \cdots \\
      \t2 {} \circelse pc = k \circthen C \\
      \t2 \cdots \\
      \t2 \circfi \circseq Poll \circseq X \\
      \t1 \circfi 
    \end{array}
    \circrefines_A
    \begin{array}{l}
      \circmu X \circspot \\
      \t1 \circif frameStack = \emptyset \circthen \Skip \\
      \t1 {} \circelse frameStack \neq \emptyset \circthen {} \\
      \t2 \circif \cdots \\
      \t2 {} \circelse pc = i \circthen (\circmu Y \circspot A \circseq \\
      \t3 (\circvar value1, value2 : Word \circspot P \circseq \\
      \t3 \circif b \circthen \\
      \t4 pc := j \circseq Poll \circseq B \circseq \\
      \t4 pc := i \circseq Poll \circseq Y \\
      \t3 {} \circelse \lnot b \circthen C \\
      \t3 \circfi)) \circseq pc := k \\
      \t2 \cdots \\
      \t2 {} \circelse pc = j \circthen B \circseq pc := i \\
      \t2 \cdots \\
      \t2 {} \circelse pc = k \circthen C \\
      \t2 \cdots \\
      \t2 \circfi \circseq Poll \circseq X \\
      \t1 \circfi 
    \end{array}
  \end{circus}
\end{crule}
\end{minipage}

\begin{minipage}{\textwidth}
\begin{restatable}[\texttt{do}-\texttt{while}-loop-intro]{crule}{DoWhileLoopIntroductionRule}
  \label{do-while-introduction-rule}
  \setlength{\zedindent}{0.2cm}
  \setlength{\zedtab}{0.58cm}
  Given $i : ProgramAddress$, if $i \neq j$,
  \begin{circus}
    \{frameStack \neq \emptyset\} \circseq A \circseq P \\
    {} = {} \\
    \{frameStack \neq \emptyset\} \circseq A \circseq P \circseq \{frameStack \neq \emptyset\}
  \end{circus}
  then
  \begin{circus}
    \begin{array}{l}
      \circmu X \circspot \\
      \t1 \circif frameStack = \emptyset \circthen \Skip \\
      \t1 {} \circelse frameStack \neq \emptyset \circthen {} \\
      \t2 \circif \cdots \\
      \t2 {} \circelse pc = i \circthen A \circseq \\
      \t3 (\circvar value1, value2 : Word \circspot P \circseq \\
      \t3 pc := \IF b \THEN i \ELSE j) \\
      \t2 \cdots \\
      \t2 {} \circelse pc = j \circthen B \\
      \t2 \cdots \\
      \t2 \circfi \circseq Poll \circseq X \\
      \t1 \circfi 
    \end{array}
    \circrefines_A
    \begin{array}{l}
      \circmu X \circspot \\
      \t1 \circif frameStack = \emptyset \circthen \Skip \\
      \t1 {} \circelse frameStack \neq \emptyset \circthen {} \\
      \t2 \circif \cdots \\
      \t2 {} \circelse pc = i \circthen (\circmu Y \circspot A \\
      \t3 (\circvar value1, value2 : Word \circspot P \circseq \\
      \t3 \circif b \circthen pc := i \circseq Poll \circseq Y \\
      \t3 {} \circelse \lnot b \circthen \Skip \\
      \t3 \circfi)) \circseq pc := j \\
      \t2 \cdots \\
      \t2 \circfi \circseq Poll \circseq X \\
      \t1 \circfi 
    \end{array}
  \end{circus}
\end{restatable}
\end{minipage}

\begin{minipage}{\textwidth}
\begin{restatable}[infinite-loop-intro]{crule}{InfiniteLoopIntroductionRule}
  \label{infinite-loop-introduction-rule}
  Given $i : ProgramAddress$, if
  \begin{circus}
    \{frameStack \neq \emptyset\} \circseq A \\
    {} = {} \\
    \{frameStack \neq \emptyset\} \circseq A \circseq \{frameStack \neq \emptyset\}
  \end{circus}
  then
  \def\zedindent{0.25cm}
  \begin{circus}
    \begin{array}{l}
      \circmu X \circspot \\
      \t1 \circif frameStack = \emptyset \circthen \Skip \\
      \t1 {} \circelse frameStack \neq \emptyset \circthen {} \\
      \t2 \circif \cdots \\
      \t2 {} \circelse pc = i \circthen {} \\
      \t3 A \circseq pc := i \\
      \t2 {} \cdots {} \\
      \t2 \circfi \circseq Poll \circseq X \\
      \t1 \circfi
    \end{array}
    \circrefines_A
    \begin{array}{l}
      \circmu X \circspot \\
      \t1 \circif frameStack = \emptyset \circthen \Skip \\
      \t1 {} \circelse frameStack \neq \emptyset \circthen {} \\
      \t2 \circif \cdots \\
      \t2 {} \circelse pc = i \circthen {} \\
      \t3 \circmu Y \circspot A \circseq pc := i \circseq Poll \circseq Y \\
      \t2 {} \cdots {} \\
      \t2 \circfi \circseq Poll \circseq X \\
      \t1 \circfi
    \end{array}
  \end{circus}
\end{restatable}
\end{minipage}

\subsection{Resolve Method Calls}

\begin{minipage}{\textwidth}
\begin{restatable}[refine-invokespecial]{crule}{RefineInvokespecialRule}
  \label{refine-invokespecial-rule}
  \setlength{\zedindent}{0.25cm}
  \begin{circus}
    \begin{array}{l}
      \{ pc = i \} \circseq \\
      HandleInvokespecialEPC(cpi)
    \end{array}
    \circrefines_A
    \begin{array}{l}
      \{ pc = i \} \circseq \circvar poppedArgs : \seq Word \circspot \\
      \t1 \lschexpract \exists argsToPop? == methodArguments~m + 1 @ \\
      \t2 InterpreterStackFrameInvoke \rschexpract \circseq \\
      \t1 Invoke(c, m, poppedArgs)
    \end{array}
  \end{circus}
  where $m : MethodID$ and $c : ClassID$ are such that
  \begin{displaymath}
    \exists c_0 : Class; m_0 : MethodID | c_0 \in \ran cs \land m_0 \in \dom c_0.methodEntry @ \\
    \t1 cpi \in methodRefIndices~c_0 \land \\
    \t1 (\exists c_1 : ClassID | c_0.constantPool~cpi = MethodRef~(c_1,m) @ \\
    \t2 (((thisClassID~c_0,c_1) \in subclassRel~cs \land c_1 \neq thisClassID~c_0) \\
    \t3 {} \implies c = superClassID~c_0) \land \\
    \t2 (((thisClassID~c_0,c_1) \notin subclassRel~cs \lor c_1 = thisClassID~c_0) \\
    \t3 {} \implies c = c_1)) \land \\
    \t1 i \in c_0.methodEntry~m_0 \upto c_0.methodEnd~m_0.
  \end{displaymath}
\end{restatable}
\end{minipage}

\begin{minipage}{\textwidth}
  \RefineInvokestaticRule*
\end{minipage}

\begin{minipage}{\textwidth}
  \RefineInvokeVirtualMultiRule*
\end{minipage}

\begin{minipage}{\textwidth}
  \ResolveSpecialMethodRule*
\end{minipage}

\begin{minipage}{\textwidth}
  \ResolveNormalMethodRule*
\end{minipage}

\begin{minipage}{\textwidth}
\begin{restatable}[$CheckSynchronizedInvoke$-sync-refinement]{crule}{CheckSynchronizedInvokeSynchronizedRefinementRule}
  \label{CheckSynchronizedInvoke-synchronized-refinement-rule}
  \hfill \\
  If $m \in c.synchronizedMethods \land m \notin c.staticMethods$, then
  \begin{circus}
    \begin{array}{l}
      CheckSynchronizedInvoke(c, m, args)
    \end{array}
    \circrefines_A
    \begin{array}{l}
      takeLock!(head~args) \\
      {} \then takeLockRet \then \Skip
    \end{array}
  \end{circus}
\end{restatable}

\begin{restatable}[$CheckSynchronizedInvoke$-nonsync-refinement]{crule}{CheckSynchronizedInvokeNonsynchronizedRefinementRule}
  \label{CheckSynchronizedInvoke-nonsynchronized-refinement-rule}
  \hfill \\
  If $m \notin c.synchronizedMethods \lor m \in c.staticMethods$, then
  \begin{circus}
    \begin{array}{l}
      CheckSynchronizedInvoke(c, m, args)
    \end{array}
    \circrefines_A
    \begin{array}{l}
      \Skip
    \end{array}
  \end{circus}
\end{restatable}
\end{minipage}

% \begin{restatable}[resolve-special-method-branch]{crule}{ResolveSpecialMethodBranchRule}
%   \label{resolve-special-method-branch-rule}
%   Given $c_\ell : ClassID$, if $c = c_\ell$, $m$ and $static = \false$
%   match one of the rows of Table~\ref{special-method-action-table},
%   then
%   \setlength{\zedindent}{0.25cm}
%   \setlength{\zedtab}{0.5cm}
%   \begin{circus}
%     \begin{array}{l}
%       \circvar poppedArgs : \seq Word \circspot \\
%       \lschexpract \exists argsToPop? == e @ \\
%       \t1 InterpreterStackFrameInvoke \rschexpract \circseq \\
%       \circif cid = c_1 \circthen A_1 \\
%       {} \cdots {} \\
%       {} \circelse cid = c_\ell \circthen {} \\
%       \t1 \{ (last~frameStack).storedPC = j + 1 \} \circseq \\
%       \t1 Invoke(c_\ell, m, poppedArgs, \false) \\
%       {} \cdots {} \\
%       {} \circelse cid = c_n \circthen A_n \\
%       \circfi
%     \end{array}
%     \circrefines_A
%     \begin{array}{l}
%       \{ pc = i \} \circseq \circvar poppedArgs : \seq Word \circspot \\
%       \lschexpract \exists argsToPop? == e @ \\
%       InterpreterStackFrameInvoke \rschexpract \circseq \\
%       \circif cid = c_1 \circthen A_1 \\
%       {} \cdots {} \\
%       {} \circelse cid = c_\ell \circthen {} \\
%       \t1 specialMethodAction(c_\ell, m, \false) \circseq pc := i + 1 \\
%       {} \cdots {} \\
%       {} \circelse cid = c_n \circthen A_n \\
%       \circfi
%     \end{array} 
%   \end{circus}
%   where $specialMethodAction$ is the syntactic function defined by
%   Table~\ref{special-method-action-table}.
% \end{restatable}

\begin{minipage}{\textwidth}
\begin{restatable}[resolve-normal-method-branch]{crule}{ResolveNormalMethodBranchRule}
  \label{resolve-normal-method-branch-rule}
  Given $i : ProgramAddress$ and $c_{\ell} : ClassID$, if
  \begin{itemize}
  \item \hfill{\vspace*{-\baselineskip}\setlength{\abovedisplayskip}{0cm}\setlength{\abovedisplayshortskip}{0cm}
      \begin{circus}
      \{frameStack \neq \emptyset\} \circseq A \\
      {} = {} \\
      \{frameStack \neq \emptyset\} \circseq A \circseq \{frameStack \neq \emptyset\},
    \end{circus}}
  \item $methodID = m \land classID = c_{\ell} \implies \pre ResolveMethod$ and there is $classInfo : Class$ such that
    \begin{circus}
      \{ methodID = m \land classID = c_{\ell} \} \circseq \lschexpract ResolveMethod \rschexpract \\
      {} = {} \\
      \{ methodID = m \land classID = c_{\ell} \} \circseq \lschexpract ResolveMethod \rschexpract \circseq \\
      \t1 \{ class = classInfo \land class.methodEntry~m = k \},
    \end{circus}
  \item for any $x : ProgramAddress$,
    \begin{circus}
      \{ (last~(front~frameStack)).storedPC = x \} \circseq M \\
      {} = {} \\
      \{ (last~(front~frameStack)).storedPC = x \} \circseq M \circseq \{ pc = x \},
    \end{circus}
  \item $m$ and $c$ do not match any of the conditions in
    Table~\ref{special-method-action-table},
  \end{itemize}
  then,
  \setlength{\zedtab}{0.3cm}
  \setlength{\zedindent}{0cm}
  \begin{circus}
    \begin{array}{l}
      \circmu X \circspot \\
      \t1 \circif frameStack = \emptyset \circthen \Skip \\
      \t1 {} \circelse frameStack \neq \emptyset \circthen {} \\
      \t2 \circif \cdots \\
      \t2 {} \circelse pc = i \circthen A \circseq \\
      \t3 \circvar poppedArgs : \seq Word \circspot \\
      \t3 \lschexpract \exists argsToPop? == e @ \\
      \t4 InterpreterStackFrameInvoke \rschexpract \circseq \\
      \t3 \circif cid = c_1 \circthen A_1 \\
      \t3 {} \cdots {} \\
      \t3 {} \circelse cid = c_\ell \circthen {} \\
      \t4 \{ (last~frameStack).storedPC = j + 1 \} \circseq \\
      \t4 Invoke(c_\ell, m, poppedArgs, \false) \\
      \t3 {} \cdots {} \\
      \t3 {} \circelse cid = c_n \circthen A_n \\
      \t3 \circfi \\
      \t2 {} \circelse pc = k \circthen M \\
      \t2 \cdots \\
      \t2 \circfi \circseq Poll \circseq X \\
      \t1 \circfi 
    \end{array}
    \circrefines_A
    \begin{array}{l}
      \circmu X \circspot \\
      \t1 \circif frameStack = \emptyset \circthen \Skip \\
      \t1 {} \circelse frameStack \neq \emptyset \circthen {} \\
      \t2 \circif \cdots \\
      \t2 {} \circelse pc = i \circthen A \circseq \{ pc = j \} \circseq  \\
      \t3 \circvar poppedArgs : \seq Word \circspot \\
      \t3 \lschexpract \exists argsToPop? == e @ \\
      \t4 InterpreterStackFrameInvoke \rschexpract \circseq \\
      \t3 \circif cid = c_1 \circthen A_1 \\
      \t3 {} \cdots {} \\
      \t3 {} \circelse cid = c_\ell \circthen {} \\
      \t4 CheckSynchronizedInvoke( \\
      \t5 classInfo, m, poppedArgs) \circseq \\
      \t4 \lschexpract InterpreterNewStackFrame[ \\
      \t5 classInfo/class?, \\
      \t5 m/methodID?, \\
      \t5 poppedArgs/methodArgs?] \rschexpract \circseq \\
      \t4 Poll \circseq M \circseq pc := j + 1 \\
      \t3 {} \cdots {} \\
      \t3 {} \circelse cid = c_n \circthen A_n \\
      \t3 \circfi) \\
      \t2 {} \circelse pc = k \circthen M \\
      \t2 \cdots \\
      \t2 \circfi \circseq Poll \circseq X \\
      \t1 \circfi 
    \end{array}
  \end{circus}
\end{restatable}
\end{minipage}

\begin{minipage}{\textwidth}
\begin{restatable}[virtual-method-call-dist]{crule}{VirtualMethodCallDist}
  \label{virtual-method-call-dist-rule}
  \setlength{\zedtab}{0.3cm}
  \setlength{\zedindent}{0cm}
  \begin{circus}
    \begin{array}{l}
      (\circvar poppedArgs : \seq Word \circspot P \\
      getClassIDOf!(head~poppedArgs)?cid \then {} \\
      \circif cid = c_1 \circthen A_1  \circseq pc := x \\
      {} \cdots {} \\
      {} \circelse cid = c_n \circthen A_\ell \circseq pc := x \\
      \circfi)
    \end{array}
    \circrefines_A
    \begin{array}{l}
      (\circvar poppedArgs : \seq Word \circspot P \\
      getClassIDOf!(head~poppedArgs)?cid \then {} \\
      \circif cid = c_1 \circthen A_1 \\
      {} \cdots {} \\
      {} \circelse cid = c_n \circthen A_\ell \\
      \circfi) \circseq pc := x
    \end{array}
  \end{circus}
\end{restatable}
\end{minipage}

\subsection{Refine Main Actions}

\begin{minipage}{\textwidth}
  \RunningRefinementRule*
\end{minipage}

\section{Elimination of Frame Stack}

\subsection{Remove Launcher Returns}
\label{remove-launcher-returns-appendix-subsection}

\begin{minipage}{\textwidth}
\begin{restatable}[conditional-dist]{crule}{ReturnActionConditionalDist}
  \label{conditional-dist-rule}
  Given an action $X$,
  \begin{circus}
    \begin{array}{l}
      \circvar value1, value2 : Word \circspot A \circseq \\
      \t1 \circif b \circthen B \circseq X \\
      \t1 {} \circelse c \circthen C \circseq X \\
      \t1 \circfi
    \end{array}
    \circrefines_A
    \begin{array}{l}
      (\circvar value1, value2 : Word \circspot A \circseq \\
      \t1 \circif b \circthen B \\
      \t1 {} \circelse c \circthen C \\
      \t1 \circfi) \circseq X
    \end{array}
  \end{circus}
\end{restatable}
\end{minipage}

\begin{algorithm}[H]
  \begin{algorithmic}[1]
    \State $methodBody \gets$ \Call{ActionBody}{$methodName$}
    \Match{$methodBody$}{($A \circseq returnAction$)}
    \State \ApplyFor{Law~[\nameref{action-intro-law}]}{$methodName'$, $A$}
    \State \ApplyReverseToFor{Law~[\nameref{copy-rule-law}]}{$methodBody$}{$methodName'$}
    \State \ExhaustivelyApplyFor{Law~[\nameref{copy-rule-law}]}{$methodName$}
    \State \ApplyReverseFor{Law~[\nameref{action-intro-law}]}{$methodName$, $methodBody$}
    \State \ApplyFor{Law~[\nameref{action-rename-law}]}{$methodName$, $methodName'$}
  \end{algorithmic}
  \caption{RedefineMethodExcludingReturn($methodName$,$returnAction$)}
  \label{redefine-method-action-excluding-return-action-algorithm}
\end{algorithm}

\begin{algorithm}[H]
  \begin{algorithmic}[1]
    \State \Apply{Rule~[\nameref{InterpreterInitEPC-frameStack-assump-intro-rule}]} 
    \State {\bf exhaustively apply} \\
    $\t1$ Rule~[\nameref{frameStack-assump-non-return-dist-rule}] \\
    $\t1$ Rule~[\nameref{frameStack-assump-return-dist-rule}] \\
    $\t1$ Rule~[\nameref{frameStack-assump-NewStackFrame-dist-rule}] \\
    $\t1$ Rule~[\nameref{restricted-assump-alt-distl-rule}] \\
    $\t1$ Rule~[\nameref{restricted-assump-alt-distr-rule}] \\
    $\t1$ Rule~[\nameref{restricted-assump-var-distl-rule}] \\
    $\t1$ Rule~[\nameref{restricted-assump-var-distr-rule}] \\
    $\t1$ Rule~[\nameref{restricted-assump-output-prefix-distl-rule}] \\
    $\t1$ Rule~[\nameref{restricted-assump-output-prefix-distr-rule}] \\
    $\t1$ Rule~[\nameref{restricted-assump-input-prefix-distl-rule}] \\
    $\t1$ Rule~[\nameref{restricted-assump-input-prefix-distr-rule}] \\
    $\t1$ Rule~[\nameref{restricted-assump-infinite-loop-distl-rule}] \\
    $\t1$ Rule~[\nameref{restricted-assump-infinite-loop-distr-rule}] \\
    $\t1$ Rule~[\nameref{restricted-assump-while-loop-distl-rule}] \\
    $\t1$ Rule~[\nameref{restricted-assump-while-loop-distr-rule}] \\
    $\t1$ Rule~[\nameref{restricted-assump-do-while-loop-distl-rule}] \\
    $\t1$ Rule~[\nameref{restricted-assump-do-while-loop-distr-rule}] \\
    $\t1$ Rule~[\nameref{restricted-assump-mid-while-loop-distl-rule}] \\
    $\t1$ Rule~[\nameref{restricted-assump-mid-while-loop-distr-rule}] \\
    $\t1$ Rule~[\nameref{restricted-assump-extchoice-distl-rule}] \\
    $\t1$ Rule~[\nameref{restricted-assump-extchoice-distr-rule}] \\
    $\t1$ Rule~[\nameref{restricted-assump-guard-dist-rule}] \\
    $\t1$ Rule~[\nameref{restricted-assump-assign-dist-rule}]
  \end{algorithmic}
  \caption{IntroduceFrameStackAssumptions}
  \label{introduce-frameStack-assumptions-algorithm}
\end{algorithm}

\begin{minipage}{\textwidth}
\begin{restatable}[$InterpreterInitEPC$-$frameStack$-assump-intro]{crule}{InterpreterInitEPCFrameStackAssumpIntro}
  \label{InterpreterInitEPC-frameStack-assump-intro-rule}
  \begin{circus}
    \begin{array}{l}
      \lschexpract InterpreterInitEPC \rschexpract
    \end{array}
    \circrefines_A
    \begin{array}{l}
      \lschexpract InterpreterInitEPC \rschexpract \circseq
      \{ \# frameStack = 0 \}
    \end{array}
  \end{circus}
\end{restatable}
\end{minipage}

\begin{minipage}{\textwidth}
\begin{restatable}[$frameStack$-assump-NewStackFrame-dist]{crule}{FrameStackAssumpNewStackFrameDist}
  \label{frameStack-assump-NewStackFrame-dist-rule}
  \begin{circus}
    \begin{array}{l}
      \{\# frameStack = k\} \circseq \\
      \lschexpract InterpreterNewStackFrame[ \\
      \t1 c/class?,
      \t1 m/methodID?, \\
      \t1 args/methodArgs?]\rschexpract
    \end{array}
    \circrefines_A
    \begin{array}{l}
      \{\# frameStack = k\} \circseq \\
      \lschexpract InterpreterNewStackFrame[ \\
      \t1 c/class?,
      \t1 m/methodID?, \\
      \t1 args/methodArgs?]\rschexpract \circseq \\
      \{\# frameStack = k + 1 \}
    \end{array}
  \end{circus}
\end{restatable}
\end{minipage}

\begin{minipage}{\textwidth}
\begin{restatable}[$frameStack$-assump-non-return-dist]{crule}{FrameStackAssumpNonReturnDist}
  \label{frameStack-assump-non-return-dist-rule}
  If $A$ is one of
  \begin{itemize}
  \item $\Skip$
  \item $Poll$,
  \item $HandleAconst\_nullEPC$,
  \item $HandleDupEPC$,
  \item $HandleAloadEPC(lvi)$,
  \item $HandleAstoreEPC(lvi)$,
  \item $HandleIaddEPC$,
  \item $HandleIconstEPC(n)$,
  \item $HandleInegEPC$,
  \item $\lschexpract InterpreterPopEPC \rschexpract$,
  \item $\lschexpract InterpreterPushEPC \rschexpract$,
  \item $\lschexpract \exists argsToPop? == m @ InterpreterStackFrameInvoke \rschexpract$,
  \item $HandleNewEPC(cpi)$,
  \item $HandleGetfieldEPC(cpi)$,
  \item $HandlePutfieldEPC(cpi)$,
  \item $HandleGetstaticEPC(cpi)$, or
  \item $HandlePutstaticEPC(cpi)$,
  \end{itemize}
  and $B$ does not begin with $\{\# frameStack = k\}$, then
  \begin{circus}
    \begin{array}{l}
      \{\# frameStack = k\} \circseq A \circseq B
    \end{array}
    \circrefines_A
    \begin{array}{l}
      \{\# frameStack = k\} \circseq A \circseq \{\# frameStack = k \} \circseq B
    \end{array}
  \end{circus}
\end{restatable}
\end{minipage}

\begin{minipage}{\textwidth}
\begin{restatable}[$frameStack$-assump-return-dist-rule]{crule}{FrameStackAssumpReturnDist}
  \label{frameStack-assump-return-dist-rule}
  If $A$ is $HandleAreturnEPC$ or $HandleReturnEPC$, $B$ does not begin with
  $\{\# frameStack = k\}$, and $k > 0$ then
  \begin{circus}
    \begin{array}{l}
      \{\# frameStack = k\} \circseq A \circseq B
    \end{array}
    \circrefines_A
    \begin{array}{l}
      \{\# frameStack = k\} \circseq A \circseq \\
      \t1 \{\# frameStack = k - 1 \} \circseq B
    \end{array}
  \end{circus}
\end{restatable}
\end{minipage}

\begin{minipage}{\textwidth}
\begin{restatable}[restricted-assump-alt-distl]{crule}{RestrictedAssumpAltDistl}
  \label{restricted-assump-alt-distl-rule}
  If no $A_i$ begins with $\{h\}$ then
  \begin{circus}
    \{h\} \circseq \circif {} \circelse_{i} g_i \circthen A_i \circfi
    =
    \{h\} \circseq \circif {} \circelse_{i} g_i \circthen \{h\} \circseq A_i \circfi
  \end{circus}
\end{restatable}
\end{minipage}

\begin{minipage}{\textwidth}
\begin{restatable}[restricted-assump-alt-distr]{crule}{RestrictedAssumpAltDistr}
  \label{restricted-assump-alt-distr-rule}
  If no $A_i$ begins with $\{h\}$ then
  \begin{circus}
    \circif {} \circelse_{i} g_i \circthen A_i \circseq \{h\} \circfi
    =
    \{h\} \circseq \circif {} \circelse_{i} g_i \circthen A_i \circfi \circseq \{h\}
  \end{circus}
\end{restatable}
\end{minipage}

\begin{minipage}{\textwidth}
\begin{restatable}[restricted-assump-var-distl]{crule}{RestrictedAssumpVarDistl}
  \label{restricted-assump-var-distl-rule}
  If $A$ does not begin with $\{h\}$ then
  \begin{circus}
    \{h\} \circseq (\circvar x : T \circspot A)
    =
    \{h\} \circseq (\circvar x : T \circspot \{h\} \circseq A)
  \end{circus}
\end{restatable}
\end{minipage}

\begin{minipage}{\textwidth}
\begin{restatable}[restricted-assump-var-distr]{crule}{RestrictedAssumpVarDistr}
  \label{restricted-assump-var-distr-rule}
  If $B$ does not begin with $\{h\}$ then
  \begin{circus}
    (\circvar x : T \circspot A \circseq \{h\}) \circseq B
    =
    (\circvar x : T \circspot A \circseq \{h\}) \circseq \{h\} \circseq B
  \end{circus}
\end{restatable}
\end{minipage}

\begin{minipage}{\textwidth}
\begin{restatable}[restricted-assump-output-prefix-distl]{crule}{RestrictedAssumpOutputPrefixDistl}
  \label{restricted-assump-output-prefix-distl-rule}
  If $A$ does not begin with $\{g\}$  then
  \begin{circus}
    \{ g \} \circseq c!x \then A = \{ g \} \circseq c!x \then \{ g \} \circseq A
  \end{circus}
\end{restatable}
\end{minipage}

\begin{minipage}{\textwidth}
\begin{restatable}[restricted-assump-output-prefix-distr]{crule}{RestrictedAssumpOutputPrefixDistr}
  \label{restricted-assump-output-prefix-distr-rule}
  If $B$ does not begin with $\{g\}$ then
  \begin{circus}
    (c!x \then A \circseq \{ g \}) \circseq B
    =
    (c!x \then A \circseq \{ g \}) \circseq \{ g \} \circseq B
  \end{circus}
\end{restatable}
\end{minipage}

\begin{minipage}{\textwidth}
\begin{restatable}[restricted-assump-input-prefix-distl]{crule}{RestrictedAssumpInputPrefixDistl}
  \label{restricted-assump-input-prefix-distl-rule}
  If $A$ does not begin with $\{g\}$ and $x$ is not free in $\{g\}$
  then
  \begin{circus}
    \{ g \} \circseq c?x \then A
    =
    \{ g \} \circseq c?x \then \{ g \} \circseq A
  \end{circus}
\end{restatable}
\end{minipage}

\begin{minipage}{\textwidth}
\begin{restatable}[restricted-assump-input-prefix-distr]{crule}{RestrictedAssumpInputPrefixDistr}
  \label{restricted-assump-input-prefix-distr-rule}
  If $B$ does not begin with $\{g\}$ and $x$ is not free in $\{g\}$ then
  \begin{circus}
    (c?x \then A \circseq \{ g \}) \circseq B
    =
    (c?x \then A \circseq \{ g \}) \circseq \{ g \} \circseq B
  \end{circus}
\end{restatable}
\end{minipage}

\begin{minipage}{\textwidth}
\begin{restatable}[restricted-assump-infinite-loop-distl]{crule}{RestrictedAssumpInfiniteLoopDistl}
  \label{restricted-assump-infinite-loop-distl-rule}
  If $A$ does not begin with $\{g\}$ and
  $\{g\} \circseq A \circrefines_A A \circseq \{g\}$ then
  \begin{circus}
    \{g\} \circseq (\circmu X \circspot A \circseq X)
    \circrefines_A
    \{g\} \circseq (\circmu X \circspot \{g\} \circseq A \circseq X)
  \end{circus}
\end{restatable}
\end{minipage}

\begin{minipage}{\textwidth}
\begin{restatable}[restricted-assump-infinite-loop-distr]{crule}{RestrictedAssumpInfiniteLoopDistr}
  \label{restricted-assump-infinite-loop-distr-rule}
  If $B$ does not begin with $\{g\}$ then
  \begin{circus}
    (\circmu X \circspot A \circseq \{g\} \circseq X) \circseq B
    =
    (\circmu X \circspot A \circseq \{g\} \circseq X) \circseq \{g\} \circseq B
  \end{circus}
\end{restatable}
\end{minipage}

\begin{minipage}{\textwidth}
\begin{restatable}[restricted-assump-mid-while-loop-distl]{crule}{RestrictedAssumpMidWhileLoopDistl}
  \label{restricted-assump-mid-while-loop-distl-rule}
  If $A$ does not begin with $\{g\}$,
  $\{g\} \circseq A \circrefines_A A \circseq \{g\}$,
  $\{g\} \circseq B \circrefines_A B \circseq \{g\}$,
  then
  \begin{circus}
    \begin{array}{l}
      \{g\} \circseq (\circmu X \circspot A \circseq \\
      \t1 \circif h \circthen B \circseq X \\
      \t1 {} \circelse \lnot h \circthen \Skip \\
      \t1 \circfi)
    \end{array}
    \circrefines_A
    \begin{array}{l}
      \{g\} \circseq (\circmu X \circspot \{g\} \circseq A \circseq \\
      \t1 \circif h \circthen B \circseq X \\
      \t1 {} \circelse \lnot h \circthen \Skip \\
      \t1 \circfi)
    \end{array}
  \end{circus}
\end{restatable}
\end{minipage}

\begin{minipage}{\textwidth}
\begin{restatable}[restricted-assump-mid-while-loop-distr]{crule}{RestrictedAssumpMidWhileLoopDistr}
  \label{restricted-assump-mid-while-loop-distr-rule}
  If $C$ does not begin with $\{g\}$,
  $\{g\} \circseq A \circrefines_A A \circseq \{g\}$,
  $\{g\} \circseq B \circrefines_A B \circseq \{g\}$,
  then
  \begin{circus}
    \begin{array}{l}
      (\circmu X \circspot A \circseq \\
      \t1 \circif h \circthen B \circseq \{g\} \circseq X \\
      \t1 {} \circelse \lnot h \circthen \Skip \circseq \{g\} \\
      \t1 \circfi) \circseq C
    \end{array}
    \circrefines_A
    \begin{array}{l}
      (\circmu X \circspot A \circseq \\
      \t1 \circif h \circthen B \circseq \{g\} \circseq X \\
      \t1 {} \circelse \lnot h \circthen \Skip \circseq \{g\}  \\
      \t1 \circfi) \circseq \{g\} \circseq C
    \end{array}
  \end{circus}
\end{restatable}
\end{minipage}

\begin{minipage}{\textwidth}
\begin{restatable}[restricted-assump-do-while-loop-distl]{crule}{RestrictedAssumpDoWhileLoopDistl}
  \label{restricted-assump-do-while-loop-distl-rule}
  If $A$ does not begin with $\{g\}$ and
  $\{g\} \circseq A \circrefines_A A \circseq \{g\}$,
  then
  \begin{circus}
    \begin{array}{l}
      \{g\} \circseq (\circmu X \circspot A \circseq \\
      \t1 \circif h \circthen X \\
      \t1 {} \circelse \lnot h \circthen \Skip \\
      \t1 \circfi)
    \end{array}
    \circrefines_A
    \begin{array}{l}
      \{g\} \circseq (\circmu X \circspot \{g\} \circseq A \circseq \\
      \t1 \circif h \circthen X \\
      \t1 {} \circelse \lnot h \circthen \Skip \\
      \t1 \circfi)
    \end{array}
  \end{circus}
\end{restatable}
\end{minipage}

\begin{minipage}{\textwidth}
\begin{restatable}[restricted-assump-do-while-loop-distr]{crule}{RestrictedAssumpDoWhileLoopDistr}
  \label{restricted-assump-do-while-loop-distr-rule}
  If $B$ does not begin with $\{g\}$ and
  $\{g\} \circseq A \circrefines_A A \circseq \{g\}$,
  then
  \begin{circus}
    \begin{array}{l}
      (\circmu X \circspot A \circseq \\
      \t1 \circif h \then \{g\} \circseq X \\
      \t1 {} \circelse \lnot h \circseq \Skip \circseq \{g\} \\
      \t1 \circfi) \circseq B
    \end{array}
    \circrefines_A
    \begin{array}{l}
      (\circmu X \circspot A \circseq \\
      \t1 \circif h \then \{g\} \circseq X \\
      \t1 {} \circelse \lnot h \circseq \Skip \circseq \{g\} \\
      \t1 \circfi) \circseq \{g\} \circseq B
    \end{array}
  \end{circus}
\end{restatable}
\end{minipage}

\begin{minipage}{\textwidth}
\begin{restatable}[restricted-assump-while-loop-distl]{crule}{RestrictedAssumpWhileLoopDistl}
  \label{restricted-assump-while-loop-distl-rule}
  If $A$ does not begin with $\{g\}$ and
  $\{g\} \circseq A \circrefines_A A \circseq \{g\}$,
  then
  \begin{circus}
    \begin{array}{l}
      \{g\} \circseq (\circmu X \circspot \\
      \t1 \circif h \circthen A \circseq X \\
      \t1 {} \circelse \lnot h \circthen \Skip \\
      \t1 \circfi)
    \end{array}
    \circrefines_A
    \begin{array}{l}
      \{g\} \circseq (\circmu X \circspot \{g\} \circseq \\
      \t1 \circif h \circthen A \circseq X \\
      \t1 {} \circelse \lnot h \circthen \Skip \\
      \t1 \circfi)
    \end{array}
  \end{circus}
\end{restatable}
\end{minipage}

\begin{minipage}{\textwidth}
\begin{restatable}[restricted-assump-while-loop-distr]{crule}{RestrictedAssumpWhileLoopDistr}
  \label{restricted-assump-while-loop-distr-rule}
  If $B$ does not begin with $\{g\}$ and
  $\{g\} \circseq A \circrefines_A A \circseq \{g\}$,
  then
  \begin{circus}
    \begin{array}{l}
      (\circmu X \circspot \\
      \t1 \circif h \circthen A \circseq \{g\} \circseq X \\
      \t1 {} \circelse \lnot h \circthen \Skip \circseq \{g\}  \\
      \t1 \circfi) \circseq B
    \end{array}
    \circrefines_A
    \begin{array}{l}
      (\circmu X \circspot \\
      \t1 \circif h \circthen A \circseq \{g\} \circseq X \\
      \t1 {} \circelse \lnot h \circthen \Skip \circseq \{g\} \\
      \t1 \circfi) \circseq \{g\} \circseq B
    \end{array}
  \end{circus}
\end{restatable}
\end{minipage}

\begin{minipage}{\textwidth}
\begin{restatable}[restricted-assump-extchoice-distl]{crule}{RestrictedAssumpExtchoiceDistl}
  \label{restricted-assump-extchoice-distl-rule}
  If $A$ and $B$ do not begin with $\{g\}$ then
  \begin{circus}
    \begin{array}{l}
      \{g\} \circseq (A \extchoice B)
    \end{array}
    \circrefines_A
    \begin{array}{l}
      \{g\} \circseq ((\{g\} \circseq A) \extchoice (\{g\} \circseq B))
    \end{array}
  \end{circus}
\end{restatable}
\end{minipage}

\begin{minipage}{\textwidth}
\begin{restatable}[restricted-assump-extchoice-distr]{crule}{RestrictedAssumpExtchoiceDistr}
  \label{restricted-assump-extchoice-distr-rule}
  If $C$ does not begin with $\{g\}$ then
  \begin{circus}
    \begin{array}{l}
      ((A \circseq \{g\}) \extchoice (\{g\} \circseq B)) \circseq C
    \end{array}
    \circrefines_A
    \begin{array}{l}
      ((A \circseq \{g\}) \extchoice (\{g\} \circseq B)) \circseq \{g\} \circseq C
    \end{array}
  \end{circus}
\end{restatable}
\end{minipage}

\begin{minipage}{\textwidth}
\begin{restatable}[restricted-assump-guard-dist]{crule}{RestrictedAssumpGuardDist}
  \label{restricted-assump-guard-dist-rule}
  If $A$ does not begin with $\{g\}$ then
  \begin{circus}
    \begin{array}{l}
      \{g\} \circseq \lcircguard h \rcircguard \circguard A
    \end{array}
    =
    \begin{array}{l}
      \{g\} \circseq \lcircguard h \rcircguard \circguard \{g\} \circseq A
    \end{array}
  \end{circus}
\end{restatable}
\end{minipage}

\begin{minipage}{\textwidth}
\begin{restatable}[restricted-assump-assign-dist]{crule}{RestrictedAssumpAssignDist}
  \label{restricted-assump-assign-dist-rule}
  If $B$ does not begin with $\{g\}$ and $x$ is not free in $g$ then
  \begin{circus}
    \begin{array}{l}
      \{g\} \circseq x := e \circseq B
    \end{array}
    =
    \begin{array}{l}
      \{g\} \circseq x := e \circseq \{g\} \circseq B
    \end{array}
  \end{circus}
\end{restatable}
\end{minipage}

\begin{minipage}{\textwidth}
\RefineHandleReturnEPCEmptyFrameStackRule*
\end{minipage}

\begin{minipage}{\textwidth}
\begin{restatable}[refine-$HandleReturnEPC$-nonempty-$frameStack$]{crule}{RefineHandleReturnEPCNonemptyFrameStackRule}
  \label{refine-HandleReturnEPC-nonempty-frameStack-rule}
  If $k > 1$ then
  \begin{circus}
    \begin{array}{l}
      \{\# frameStack = k\} \circseq \\
      HandleReturnEPC
    \end{array}
    \circrefines_A
    \begin{array}{l}
      \lschexpract InterpreterReturnEPC \rschexpract
    \end{array}
  \end{circus}
\end{restatable}
\end{minipage}

\begin{minipage}{\textwidth}
\begin{restatable}[refine-$HandleAreturnEPC$-empty-$frameStack$]{crule}{RefineHandleAreturnEPCEmptyFrameStackRule}
  \label{refine-HandleAreturnEPC-empty-frameStack-rule}
  \begin{circus}
    \begin{array}{l}
      \{\# frameStack = 1\} \circseq \\
      HandleAreturnEPC
    \end{array}
    \circrefines_A
    \begin{array}{l}
      \circvar returnValue : Word \circspot \\
      \lschexpract InterpreterAreturn2EPC \rschexpract \circseq \\
      executeMethodRet!thread!returnValue \then \Skip
    \end{array}
  \end{circus}
\end{restatable}
\end{minipage}

\begin{minipage}{\textwidth}
\begin{restatable}[refine-$HandleAreturnEPC$-nonempty-$frameStack$]{crule}{RefineHandleAreturnEPCNonemptyFrameStackRule}
  \label{refine-HandleAreturnEPC-nonempty-frameStack-rule}
  If $k > 1$ then
  \begin{circus}
    \begin{array}{l}
      \{\# frameStack = k\} \circseq \\
      HandleAreturnEPC
    \end{array}
    \circrefines_A
    \begin{array}{l}
      \lschexpract InterpreterAreturn1EPC \rschexpract
    \end{array}
  \end{circus}
\end{restatable}
\end{minipage}

\subsection{Localise Stack Frames}
\label{localise-stack-frames-appendix-subsection}

\begin{minipage}{\textwidth}
  \ArgumentsIntroductionRule*
\end{minipage}

\begin{algorithm}[H]
  \begin{algorithmic}[1]
    \MatchThen{%
      $\begin{array}[t]{l}
         (\circval arg1, \ldots, arg{<}n{>} : Word \circspot \\
         \t1 \lschexpract \exists methodArgs? == \langle arg1, \ldots, arg{<}n{>} \rangle @ \\
         \t2 InterpreterNewStackFrame[c/class?, m/methodID?] \rschexpract \circseq \\
         \t1 Poll \circseq methodName \circseq \lschexpract InterpreterReturn \rschexpract)(args~1, \ldots, args~n)
       \end{array}$}
     \State \ApplyFor{Law~[\nameref{action-intro-law}]}{$methodName'$, %
       $\begin{array}[t]{l}
          (\circval arg1, \ldots, arg{<}n{>} : Word \circspot \\
          \t1 \lschexpract \exists methodArgs? \\
          \t3 {} == \langle arg1, \ldots, arg{<}n{>} \rangle @ \\
          \t2 InterpreterNewStackFrame[ \\
          \t3 c/class?, m/methodID?] \rschexpract \circseq \\
          \t1 Poll \circseq methodName \circseq \\
          \t1 \lschexpract InterpreterReturn \rschexpract)
        \end{array}$}
      \State \ExhaustivelyApplyReverseFor{Law~[\nameref{copy-rule-law}]}{$methodName'$}
      \State \ApplyToFor{Law~[\nameref{copy-rule-law}]}{\Call{ActionBody}{$methodName'$}}{$methodName$}
      \State \ApplyReverseFor{Law~[\nameref{action-intro-law}]}{$methodName$, $methodBody$}
      \State \ApplyFor{Law~[\nameref{action-rename-law}]}{$methodName'$, $methodName$}
  \end{algorithmic}
  \caption{RedefineMethodToIncludeParameters($nethodName$)}
  \label{redefine-method-action-to-include-parameters-algorithm}
\end{algorithm}

\begin{minipage}{\textwidth}
  \HandleReturnEPCStackFrameIntroductionRule*
\end{minipage}

% \begin{restatable}[$InterpreterAreturn$-$stackFrame$-introduction]{crule}{HandleAreturnEPCStackFrameIntroductionRule}
%   \label{HandleAreturnEPC-stackFrame-introduction-rule}
%   %\setlength{\zedtab}{0.4cm}
%   %\setlength{\zedindent}{0pt}
%   %\setlength{\zedleftsep}{0pt}
%   If the actions in $A$ operate solely on $last~frameStack$ and do not
%   change the length of $frameStack$, then
%   \begin{circus}
%     \begin{array}{l}
%       InterpreterNewStackFrame[ \\
%       \t1 c/class?, \\
%       \t1 m/methodID?, \\
%       \t1 args/methodArgs?] \circseq \\
%       A \circseq \lschexpract InterpreterAreturn1 \rschexpract
%     \end{array}
%     \circrefines_A
%     \begin{array}{l}
%       (\circvar retVal : Word \circspot \\
%       \t1 (\circvar stackFrame : StackFrameEPC \circspot \\
%       \t2 \lschexpract [stackFrame' : StackFrameEPC | \\
%       \t3 args \subseteq stackFrame'.localVariables \land \\
%       \t3 \# stackFrame'.localVariables = \ell \land \\
%       \t3 stackFrame'.operandStack = \langle\rangle \land \\
%       \t3 stackFrame'.frameClass = c \land \\
%       \t3 stackFrame'.stackSize = s] \rschexpract \circseq \\
%       \t2 A[stackFrame/last~frameStack, \\
%       \t3 stackFrame'/last~frameStack'] \circseq \\
%       \t2 \lschexpract InterpreterPopEPC[ \\
%       \t3 stackFrame/last~frameStack, \\
%       \t3 stackFrame'/last~frameStack', \\
%       \t3 retVal!/value!] \rschexpract) \\
%       \t1 \lschexpract InterpreterPushEPC[retVal?/value?]  \rschexpract)
%     \end{array}
%   \end{circus}
%   where $\ell = c.methodLocals~m$ and $s = c.methodStackSize~m$.
% \end{restatable}

% TODO: rules for the areturn cases here?

\subsection{Introduce Variables}
\label{introduce-variables-appendix-subsection}

\begin{algorithm}[H]
  \begin{algorithmic}[1]
    \State \ApplyTo{Rule~[\nameref{stackFrame-init-frameClass-assump-intro-rule}]}{$A$}
    \State {\bf exhaustively apply to }{$A$}  \\
    $\t1$ Rule~[\nameref{frameClass-assump-dist-rule}] \\
    $\t1$ Rule~[\nameref{restricted-assump-alt-distl-rule}] \\
    $\t1$ Rule~[\nameref{restricted-assump-alt-distr-rule}] \\
    $\t1$ Rule~[\nameref{restricted-assump-var-distl-rule}] \\
    $\t1$ Rule~[\nameref{restricted-assump-var-distr-rule}] \\
    $\t1$ Rule~[\nameref{restricted-assump-output-prefix-distl-rule}] \\
    $\t1$ Rule~[\nameref{restricted-assump-output-prefix-distr-rule}] \\
    $\t1$ Rule~[\nameref{restricted-assump-input-prefix-distl-rule}] \\
    $\t1$ Rule~[\nameref{restricted-assump-input-prefix-distr-rule}] \\
    $\t1$ Rule~[\nameref{restricted-assump-infinite-loop-distl-rule}] \\
    $\t1$ Rule~[\nameref{restricted-assump-infinite-loop-distr-rule}] \\
    $\t1$ Rule~[\nameref{restricted-assump-while-loop-distl-rule}] \\
    $\t1$ Rule~[\nameref{restricted-assump-while-loop-distr-rule}] \\
    $\t1$ Rule~[\nameref{restricted-assump-do-while-loop-distl-rule}] \\
    $\t1$ Rule~[\nameref{restricted-assump-do-while-loop-distr-rule}] \\
    $\t1$ Rule~[\nameref{restricted-assump-mid-while-loop-distl-rule}] \\
    $\t1$ Rule~[\nameref{restricted-assump-mid-while-loop-distr-rule}]
    % $\t1$ Rule~[\nameref{restricted-assump-extchoice-distl-rule}] \\
    % $\t1$ Rule~[\nameref{restricted-assump-extchoice-distr-rule}] \\
    % $\t1$ Rule~[\nameref{restricted-assump-guard-dist-rule}] \\
    % $\t1$ Rule~[\nameref{restricted-assump-assign-dist-rule}]
  \end{algorithmic}
  \caption{IntroduceFrameClassAssumptions(A)}
  \label{introduce-frameClass-assumptions-algorithm}
\end{algorithm}

\begin{minipage}{\textwidth}
\begin{restatable}[$stackFrame$-init-$frameClass$-assump-intro]{crule}{StackFrameInitFrameClassAssumpIntroRule}
  \label{stackFrame-init-frameClass-assump-intro-rule}
  \setlength{\zedindent}{0.5cm}
  \begin{circus}
    \begin{array}{l}
      \lschexpract [arg1?, \ldots, arg{<}n{>}? : Word; \\
      \t1 stackFrame' : StackFrameEPC  | \\
      \t1 \langle arg1?, \ldots, arg{<}n{>}? \rangle \\
      \t2 {} \subseteq stackFrame'.localVariables \land \\
      \t1 \# stackFrame'.localVariables = \ell \land \\
      \t1 stackFrame'.operandStack = \langle\rangle \land \\
      \t1 stackFrame'.frameClass = c \land \\
      \t1 stackFrame'.stackSize = s] \rschexpract
    \end{array}
    \circrefines_A
    \begin{array}{l}
      \lschexpract [arg1?, \ldots, arg{<}n{>}? : Word; \\
      \t1 stackFrame' : StackFrameEPC  | \\
      \t1 \langle arg1?, \ldots, arg{<}n{>}? \rangle \\
      \t2 {} \subseteq stackFrame'.localVariables \land \\
      \t1 \# stackFrame'.localVariables = \ell \land \\
      \t1 stackFrame'.operandStack = \langle\rangle \land \\
      \t1 stackFrame'.frameClass = c \land \\
      \t1 stackFrame'.stackSize = s] \rschexpract \circseq \\
      \{ stackFrame.frameClass = c\}
    \end{array}
  \end{circus}
\end{restatable}
\end{minipage}

\begin{minipage}{\textwidth}
\begin{restatable}[$frameClass$-assump-dist]{crule}{FrameClassAssumpDistRule}
  \label{frameClass-assump-dist-rule}
  If $A$ is one of
  \begin{itemize}
  \item $\Skip$,
  \item $Poll$,
  \item $HandleAconst\_nullSF$,
  \item $HandleDupSF$,
  \item $HandleAloadSF(lvi)$,
  \item $HandleAstoreSF(lvi)$,
  \item $HandleIaddSF$,
  \item $HandleIconstSF(n)$,
  \item $HandleInegSF$,
  \item $\lschexpract InterpreterPopSF \rschexpract$,
  \item $\lschexpract InterpreterPushSF \rschexpract$,
  \item $\lschexpract \exists argsToPop? == m @ InvokeSF \rschexpract$
  \end{itemize}
  and $B$ does not begin with $\{ stackFrame.frameClass = c \}$, then
  \begin{circus}
    \begin{array}{l}
      \{ stackFrame.frameClass = c \} \circseq A \circseq B
    \end{array}
    \circrefines_A
    \begin{array}{l}
      \{ stackFrame.frameClass = c \} \circseq A \circseq \\
      \{ stackFrame.frameClass = c \} \circseq B
    \end{array}
  \end{circus}
\end{restatable}
\end{minipage}

\begin{minipage}{\textwidth}
  \RefinePutfieldSFRule*
\end{minipage}

\begin{minipage}{\textwidth}
\begin{restatable}[refine-$GetfieldSF$]{crule}{RefineGetfieldSFRule}
  \label{refine-GetfieldSF-rule}
  \begin{circus}
    \begin{array}{l}
      \{stackFrame.frameClass = c\} \circseq \\
      GetfieldSF(cpi)
    \end{array}
    \circrefines_A
    \begin{array}{l}
      (\circvar oid : ObjectID \circspot \\
      \t1 \lschexpract InterpreterPop[ \\
      \t2 oid!/value!, \\
      \t2 stackFrame/last~frameStack, \\
      \t2 stackFrame'/last~frameStack'] \rschexpract \circseq \\
      \t1 getField!oid!cid!fid! \\
      \t1 {} \then getFieldRet?value \\
      \t1 {} \then \lschexpract InterpreterPush[ \\
      \t2 stackFrame/last~frameStack, \\
      \t2 stackFrame'/last~frameStack'] \rschexpract)
    \end{array}
  \end{circus}
  where
  \begin{circus}
    cpi \in fieldRefIndices~c \land \\
    c.constantPool~cpi = FieldRef~(cid, fid)
  \end{circus}
\end{restatable}
\end{minipage}

\begin{minipage}{\textwidth}
\begin{restatable}[refine-$PutstaticSF$]{crule}{RefinePutstaticSFRule}
  \label{refine-PutstaticSF-rule}
  \begin{circus}
    \begin{array}{l}
      \{stackFrame.frameClass = c\} \circseq \\
      PutstaticSF(cpi)
    \end{array}
    \circrefines_A
    \begin{array}{l}
      (\circvar value : Word \circspot \\
      \t1 \lschexpract InterpreterPop[ \\
      \t2 stackFrame/last~frameStack, \\
      \t2 stackFrame'/last~frameStack'] \rschexpract \circseq \\
      \t1 putStatic!cid!fid!value \then \Skip)
    \end{array}
  \end{circus}
  where
  \begin{circus}
    cpi \in fieldRefIndices~c \land \\
    c.constantPool~cpi = FieldRef~(cid, fid)
  \end{circus}
\end{restatable}
\end{minipage}

\begin{minipage}{\textwidth}
\begin{restatable}[refine-$GetstaticSF$]{crule}{RefineGetstaticSFRule}
  \label{refine-GetstaticSF-rule}
  \begin{circus}
    \begin{array}{l}
      \{stackFrame.frameClass = c\} \circseq \\
      GetstaticSF(cpi)
    \end{array}
    \circrefines_A
    \begin{array}{l}
      getStatic!cid!fid \\
      {} \then getStaticRet?value \\
      {} \then \lschexpract InterpreterPush[ \\
      \t1 stackFrame/last~frameStack, \\
      \t1 stackFrame'/last~frameStack'] \rschexpract)
    \end{array}
  \end{circus}
  where
  \begin{circus}
    cpi \in fieldRefIndices~c \land \\
    c.constantPool~cpi = FieldRef~(cid, fid)
  \end{circus}
\end{restatable}
\end{minipage}

\begin{minipage}{\textwidth}
\begin{restatable}[refine-$NewSF$]{crule}{RefineNewSFRule}
  \label{refine-NewSF-rule}
  \begin{circus}
    \begin{array}{l}
      \{stackFrame.frameClass = c\} \circseq \\
      NewSF(cpi)
    \end{array}
    \circrefines_A
    \begin{array}{l}
      newObject!thread!cid \\
      {} \then newObjectRet?oid \\
      {} \then \lschexpract InterpreterPush[ \\
      \t1 oid/value?, \\
      \t1 stackFrame/last~frameStack, \\
      \t1 stackFrame'/last~frameStack'] \rschexpract)
    \end{array}
  \end{circus}
  where
  \begin{circus}
    cpi \in ClassRefIndices~c \land \\
    c.constantPool~cpi = ClassRef~cid
  \end{circus}
\end{restatable}
\end{minipage}

\begin{algorithm}[H]
  \begin{algorithmic}[1]
    \State \ApplyTo{Rule~[\nameref{stackFrame-init-operandStack-assump-intro-rule}]}{$A$}
    \State {\bf exhaustively apply to }{$A$}  \\
    $\t1$ Rule~[\nameref{operandStack-assump-unchanged-dist-rule}] \\
    $\t1$ Rule~[\nameref{operandStack-assump-increment-dist-rule}] \\
    $\t1$ Rule~[\nameref{operandStack-assump-decrement-dist-rule}] \\
    $\t1$ Rule~[\nameref{operandStack-assump-InvokeSF-dist-rule}] \\
    $\t1$ Rule~[\nameref{restricted-assump-alt-distl-rule}] \\
    $\t1$ Rule~[\nameref{restricted-assump-alt-distr-rule}] \\
    $\t1$ Rule~[\nameref{restricted-assump-var-distl-rule}] \\
    $\t1$ Rule~[\nameref{restricted-assump-var-distr-rule}] \\
    $\t1$ Rule~[\nameref{restricted-assump-output-prefix-distl-rule}] \\
    $\t1$ Rule~[\nameref{restricted-assump-output-prefix-distr-rule}] \\
    $\t1$ Rule~[\nameref{restricted-assump-input-prefix-distl-rule}] \\
    $\t1$ Rule~[\nameref{restricted-assump-input-prefix-distr-rule}] \\
    $\t1$ Rule~[\nameref{restricted-assump-infinite-loop-distl-rule}] \\
    $\t1$ Rule~[\nameref{restricted-assump-infinite-loop-distr-rule}] \\
    $\t1$ Rule~[\nameref{restricted-assump-while-loop-distl-rule}] \\
    $\t1$ Rule~[\nameref{restricted-assump-while-loop-distr-rule}] \\
    $\t1$ Rule~[\nameref{restricted-assump-do-while-loop-distl-rule}] \\
    $\t1$ Rule~[\nameref{restricted-assump-do-while-loop-distr-rule}] \\
    $\t1$ Rule~[\nameref{restricted-assump-mid-while-loop-distl-rule}] \\
    $\t1$ Rule~[\nameref{restricted-assump-mid-while-loop-distr-rule}]
    % $\t1$ Rule~[\nameref{restricted-assump-extchoice-distl-rule}] \\
    % $\t1$ Rule~[\nameref{restricted-assump-extchoice-distr-rule}] \\
    % $\t1$ Rule~[\nameref{restricted-assump-guard-dist-rule}] \\
    % $\t1$ Rule~[\nameref{restricted-assump-assign-dist-rule}]
  \end{algorithmic}
  \caption{IntroduceOperandStackAssumptions(A)}
  \label{introduce-operandStack-assumptions-algorithm}
\end{algorithm}

\begin{minipage}{\textwidth}
\begin{restatable}[$operandStack$-init-$frameClass$-assump-intro]{crule}{StackFrameInitOperandStackAssumpIntroRule}
  \label{stackFrame-init-operandStack-assump-intro-rule}
  \setlength{\zedindent}{0.1cm}
  \begin{circus}
    \begin{array}{l}
      \lschexpract [arg1?, \ldots, arg{<}n{>}? : Word; \\
      \t1 stackFrame' : StackFrameEPC  | \\
      \t1 \langle arg1?, \ldots, arg{<}n{>}? \rangle \\
      \t2 {} \subseteq stackFrame'.localVariables \land \\
      \t1 \# stackFrame'.localVariables = \ell \land \\
      \t1 stackFrame'.operandStack = \langle\rangle \land \\
      \t1 stackFrame'.frameClass = c \land \\
      \t1 stackFrame'.stackSize = s] \rschexpract
    \end{array}
    \circrefines_A
    \begin{array}{l}
      \lschexpract [arg1?, \ldots, arg{<}n{>}? : Word; \\
      \t1 stackFrame' : StackFrameEPC  | \\
      \t1 \langle arg1?, \ldots, arg{<}n{>}? \rangle \\
      \t2 {} \subseteq stackFrame'.localVariables \land \\
      \t1 \# stackFrame'.localVariables = \ell \land \\
      \t1 stackFrame'.operandStack = \langle\rangle \land \\
      \t1 stackFrame'.frameClass = c \land \\
      \t1 stackFrame'.stackSize = s] \rschexpract \circseq \\
      \{ \# stackFrame.operandStack = 0 \}
    \end{array}
  \end{circus}
\end{restatable}
\end{minipage}

\begin{minipage}{\textwidth}
\begin{restatable}[$operandStack$-assump-unchanged-dist]{crule}{OperandStackAssumpUnchangedDistRule}
  \label{operandStack-assump-unchanged-dist-rule}
  \setlength{\zedindent}{0.1cm}
  If $A$ is one of
  \begin{itemize}
  \item $\Skip$,
  \item $Poll$,
  % \item $HandleAconst\_nullSF$,
  % \item $HandleDupSF$,
  % \item $HandleAloadSF(lvi)$,
  % \item $HandleAstoreSF(lvi)$,
  % \item $HandleIaddSF$,
  % \item $HandleIconstSF(n)$,
  \item $HandleInegSF$,
  % \item $\lschexpract InterpreterPopSF \rschexpract$,
  % \item $\lschexpract InterpreterPushSF \rschexpract$,
  % \item $\lschexpract \exists argsToPop? == m @ InvokeSF \rschexpract$
  \end{itemize}
  and $B$ does not begin with $\{ \# stackFrame.operandStack = k \}$, then
  \begin{circus}
    \begin{array}{l}
      \{ \# stackFrame.operandStack = k \} \circseq A \circseq B
    \end{array}
    \circrefines_A
    \begin{array}{l}
      \{ \# stackFrame.operandStack = k \} \circseq A \circseq \\ \{ \# stackFrame.operandStack = k \} \circseq B
    \end{array}
  \end{circus}
\end{restatable}
\end{minipage}

\begin{minipage}{\textwidth}
\begin{restatable}[$operandStack$-assump-increment-dist]{crule}{OperandStackAssumpIncrementDistRule}
  \label{operandStack-assump-increment-dist-rule}
  \setlength{\zedindent}{0.1cm}
  If $A$ is one of
  \begin{itemize}
  % \item $\Skip$,
  % \item $Poll$,
  \item $HandleAconst\_nullSF$,
  \item $HandleDupSF$,
  \item $HandleAloadSF(lvi)$,
  % \item $HandleAstoreSF(lvi)$,
  % \item $HandleIaddSF$,
  \item $HandleIconstSF(n)$,
  % \item $HandleInegSF$,
  % \item $\lschexpract InterpreterPopSF \rschexpract$,
  \item $\lschexpract InterpreterPushSF \rschexpract$,
  % \item $\lschexpract \exists argsToPop? == m @ InvokeSF \rschexpract$
  \end{itemize}
  and $B$ does not begin with $\{ \# stackFrame.operandStack = k+1 \}$, then
  \begin{circus}
    \begin{array}{l}
      \{ \# stackFrame.operandStack = k \} \circseq A \circseq B
    \end{array}
    \circrefines_A
    \begin{array}{l}
      \{ \# stackFrame.operandStack = k \} \circseq A \circseq \\
      \{ \# stackFrame.operandStack = k+1 \} \circseq B
    \end{array}
  \end{circus}
\end{restatable}
\end{minipage}

\begin{minipage}{\textwidth}
\begin{restatable}[$operandStack$-assump-decrement-dist]{crule}{OperandStackAssumpDecrementDistRule}
  \label{operandStack-assump-decrement-dist-rule}
  \setlength{\zedindent}{0.1cm}
  If $A$ is one of
  \begin{itemize}
  % \item $\Skip$,
  % \item $Poll$,
  % \item $HandleAconst\_nullSF$,
  % \item $HandleDupSF$,
  % \item $HandleAloadSF(lvi)$,
  \item $HandleAstoreSF(lvi)$,
  \item $HandleIaddSF$,
  \item $HandleIconstSF(n)$,
  % \item $HandleInegSF$,
  \item $\lschexpract InterpreterPopSF \rschexpract$,
  % \item $\lschexpract InterpreterPushSF \rschexpract$,
  % \item $\lschexpract \exists argsToPop? == m @ InvokeSF \rschexpract$
  \end{itemize}
  and $B$ does not begin with $\{ \# stackFrame.operandStack = k-1 \}$, then
  \begin{circus}
    \begin{array}{l}
      \{ \# stackFrame.operandStack = k \} \circseq A \circseq B
    \end{array}
    \circrefines_A
    \begin{array}{l}
      \{ \# stackFrame.operandStack = k \} \circseq A \circseq \\ \{ \# stackFrame.operandStack = k-1 \} \circseq B
    \end{array}
  \end{circus}
\end{restatable}
\end{minipage}

\begin{minipage}{\textwidth}
\begin{restatable}[$operandStack$-assump-$InvokeSF$-dist]{crule}{OperandStackAssumpInvokeSFDistRule}
  \label{operandStack-assump-InvokeSF-dist-rule}
  \setlength{\zedindent}{0.1cm}
  If $B$ does not begin with \[\{ \# stackFrame.operandStack = k-m \},\] then
  \begin{circus}
    \begin{array}{l}
      \{ \# stackFrame.operandStack = k \} \circseq \\
      \lschexpract \exists argsToPop? == m @ InvokeSF \rschexpract \circseq B
    \end{array}
    \circrefines_A
    \begin{array}{l}
      \{ \# stackFrame.operandStack = k \} \circseq \\
      \lschexpract \exists argsToPop? == m @ InvokeSF \rschexpract \circseq \\
      \{ \# stackFrame.operandStack = k-m \} \circseq B
    \end{array}
  \end{circus}
\end{restatable}
\end{minipage}

\begin{minipage}{\textwidth}
\begin{restatable}[$HandleAconst\_nullSF$-simulation]{crule}{HandleAconstNullSFSimulationRule}
  \label{HandleAconst_nullSF-simulation-rule}
  \begin{circus}
    \begin{array}{l}
      \{\# stackFrame.operandStack = k\} \circseq \\
      HandleAconst\_nullSF
    \end{array}
    \circsimulates
    \begin{array}{l}
      stack{<}k+1{>} := null
    \end{array}
  \end{circus}
\end{restatable}
\end{minipage}

\begin{minipage}{\textwidth}
\begin{restatable}[$HandleDupSF$-simulation]{crule}{HandleDupSFSimulationRule}
  \label{HandleDupSF-simulation-rule}
  \begin{circus}
    \begin{array}{l}
      \{\# stackFrame.operandStack = k\} \circseq \\
      HandleDupSF
    \end{array}
    \circsimulates
    \begin{array}{l}
      stack{<}k+1{>} := stack{<}k{>}
    \end{array}
  \end{circus}
\end{restatable}
\end{minipage}

\begin{minipage}{\textwidth}
  \HandleAloadSFSimulationRule*
\end{minipage}

\begin{minipage}{\textwidth}
\begin{restatable}[$HandleAstoreSF$-simulation]{crule}{HandleAstoreSFSimulationRule}
  \label{HandleAstoreSF-simulation-rule}
  \begin{circus}
    \begin{array}{l}
      \{\# stackFrame.operandStack = k\} \circseq \\
      HandleAstoreSF(lvi)
    \end{array}
    \circsimulates
    \begin{array}{l}
      var{<}lvi+1{>} := stack{<}k{>} 
    \end{array}
  \end{circus}
\end{restatable}
\end{minipage}

\begin{minipage}{\textwidth}
\begin{restatable}[$HandleIaddSF$-simulation]{crule}{HandleIaddSFSimulationRule}
  \label{HandleIaddSF-simulation-rule}
  \begin{circus}
    \begin{array}{l}
      \{\# stackFrame.operandStack = k\} \circseq \\
      HandleIaddSF
    \end{array}
    \circsimulates
    \begin{array}{l}
      stack{<}k-1{>} := stack{<}k-1{>} + stack{<}k{>} 
    \end{array}
  \end{circus}
\end{restatable}
\end{minipage}

\begin{minipage}{\textwidth}
\begin{restatable}[$HandleIconstSF$-simulation]{crule}{HandleIconstSFSimulationRule}
  \label{HandleIconstSF-simulation-rule}
  \begin{circus}
    \begin{array}{l}
      \{\# stackFrame.operandStack = k\} \circseq \\
      HandleIconstSF(n)
    \end{array}
    \circsimulates
    \begin{array}{l}
      stack{<}k+1{>} := n 
    \end{array}
  \end{circus}
\end{restatable}
\end{minipage}

\begin{minipage}{\textwidth}
\begin{restatable}[$HandleInegSF$-simulation]{crule}{HandleInegSFSimulationRule}
  \label{HandleInegSF-simulation-rule}
  \begin{circus}
    \begin{array}{l}
      \{\# stackFrame.operandStack = k\} \circseq \\
      HandleInegSF
    \end{array}
    \circsimulates
    \begin{array}{l}
      stack{<}k{>} := \negate stack{<}k{>} 
    \end{array}
  \end{circus}
\end{restatable}
\end{minipage}

\begin{minipage}{\textwidth}
  \InterpreterPopEPCSimulationRule*
\end{minipage}

\begin{minipage}{\textwidth}
\begin{restatable}[$InterpreterPushEPC$-simulation]{crule}{InterpreterPushEPCSimulationRule}
  \label{InterpreterPushEPC-simulation-rule}
  \begin{circus}
    \begin{array}{l}
      \{\# stackFrame.operandStack = k\} \circseq \\
      \lschexpract InterpreterPushEPC[ \\
      \t1 stackFrame/last~frameStack, \\
      \t1 stackFrame'/last~frameStack']\rschexpract
    \end{array}
    \circsimulates
    \begin{array}{l}
      stack{<}k+1{>} := value
    \end{array}
  \end{circus}
\end{restatable}
\end{minipage}

\begin{minipage}{\textwidth}
\begin{restatable}[$InterpreterPop2EPC$-simulation]{crule}{InterpreterPop2EPCSimulationRule}
  \label{InterpreterPop2EPC-simulation-rule}
  \begin{circus}
    \begin{array}{l}
      \{\# stackFrame.operandStack = k\} \circseq \\
      \lschexpract InterpreterPopEPC[ \\
      \t1 stackFrame/last~frameStack, \\
      \t1 stackFrame'/last~frameStack']\rschexpract
    \end{array}
    \circsimulates
    \begin{array}{l}
      value1 := stack{<}k-1{>} \circseq \\
      value2 := stack{<}k{>}      
    \end{array}
  \end{circus}
\end{restatable}
\end{minipage}

\begin{minipage}{\textwidth}
  \InvokeSFSimulationRule*
\end{minipage}

\begin{minipage}{\textwidth}
  \StackFrameInitSimulationRule*
\end{minipage}

\begin{minipage}{\textwidth}
  \EliminateValueOneValueTwoConditional*
\end{minipage}

\begin{minipage}{\textwidth}
\begin{restatable}[$getField$-$oid$-elim]{crule}{EliminateOidGetField}
  \label{eliminate-oid-getField-rule}
  \begin{circus}
    \begin{array}{l}
      (\circvar oid : ObjectID \circspot \\
      \t1 oid := stack{<}k{>} \circseq \\
      \t1 getfield!oid!cid!fid \\
      \t1 {} \then getFieldRet?value \\
      \t1 {} \then stack{<}k{>} := value)
    \end{array}
    \circrefines_A
    \begin{array}{l}
      getfield!stack{<}k{>}!cid!fid \\
      \t1 {} \then getFieldRet?value \\
      \t1 {} \then stack{<}k{>} := value
    \end{array}
  \end{circus}
\end{restatable}
\end{minipage}

\begin{minipage}{\textwidth}
\begin{restatable}[$putField$-$oid$-$value$-elim]{crule}{EliminateOidValuePutField}
  \label{eliminate-oid-value-putField-rule}
  \begin{circus}
    \begin{array}{l}
      (\circvar oid : ObjectID; value : Word \circspot \\
      \t1 value := stack{<}k{>} \circseq \\
      \t1 oid := stack{<}k-1{>} \circseq \\
      \t1 putField!oid!cid!fid!value \\
      \t1 {} \then \Skip)
    \end{array}
    \circrefines_A
    \begin{array}{l}
      putField!stack{<}k-1{>}!cid!fid!stack{<}k{>} \\
      {} \then \Skip
    \end{array}
  \end{circus}
\end{restatable}
\end{minipage}

\begin{minipage}{\textwidth}
\begin{restatable}[$putStatic$-$value$-elim]{crule}{EliminateValuePutStatic}
  \label{eliminate-value-putStatic-rule}
  \begin{circus}
    \begin{array}{l}
      (\circvar value : Word \circspot \\
      \t1 value := stack{<}k{>} \circseq \\
      \t1 putStatic!cid!fid!value \then \Skip)
    \end{array}
    \circrefines_A
    \begin{array}{l}
      putStatic!cid!fid!stack{<}k{>} \then \Skip
    \end{array}
  \end{circus}
\end{restatable}
\end{minipage}

\begin{minipage}{\textwidth}
  \MethodParameterIntroductionRule*
\end{minipage}

\begin{minipage}{\textwidth}
\begin{restatable}[$poppedArgs$-sync-elim]{crule}{PoppedArgsSyncElimRule}
  \label{poppedArgs-sync-elim-rule}
  \begin{circus}
    \begin{array}{l}
      (\circvar poppedArgs : \seq Word \circspot \\
      poppedArgs := \langle arg_1, \ldots, arg_n \rangle \circseq \\
      takeLock!(head~methodArgs) \\
      {} \then takeLockRet \then \Skip \circseq \\
      M(poppedArgs~1, \ldots, poppedArgs~n))
    \end{array}
    \circrefines_A
    \begin{array}{l}
      takeLock!arg_1 \\
      {} \then takeLockRet \then \Skip \circseq \\
      M(arg_1, \ldots, arg_n)
    \end{array}
  \end{circus}
\end{restatable}
\end{minipage}

% not necessary if we don't eliminate the choice for invokevirtual
% \begin{restatable}[$getClassIDOf$-$poppedArgs$-elim]{crule}{GetClassIDOfMethodParameterIntroductionRule}
%   \label{getClassIDOf-method-parameter-introduction-rule}
%   \begin{circus}
%     \begin{array}{l}
%       (\circvar poppedArgs : \seq Word \circspot \\
%       poppedArgs := \langle arg_1, \ldots, arg_n \rangle \circseq \\
%       getClassIDOf!(head~poppedArgs)!cid \\
%       {} \then M(poppedArgs~1, \ldots, poppedArgs~n))
%       % Remember: Poll starts the \cdots
%     \end{array}
%     \circrefines_A
%     \begin{array}{l}
%       getClassIDOf!arg_1!cid \\
%       {} \then M(arg_1, \ldots, arg_n))
%     \end{array}
%   \end{circus}
% \end{restatable}

\begin{minipage}{\textwidth}
\begin{restatable}[invokevirtual-$poppedArgs$-elim]{crule}{GetClassIDOfMultiMethodParameterIntroductionRule}
  \label{getClassIDOf-multi-method-parameter-introduction-rule}
  If, for each $i \in 1 \upto m$, $A_i$ matches
  \begin{circus}
    M_i(poppedArgs~1, \ldots, poppedArgs~n)
  \end{circus}
  or
  \begin{circus}
    takeLock!(head~methodArgs) \then takeLockRet \then \Skip \circseq \\
    M_i(poppedArgs~1, \ldots, poppedArgs~n)
  \end{circus}
  then,
  \begin{circus}
    \begin{array}{l}
      (\circvar poppedArgs : \seq Word \circspot \\
      poppedArgs := \langle arg_1, \ldots, arg_n \rangle \circseq \\
      getClassIDOf!(head~poppedArgs)!cid \then {} \\
      \circif cid = c_1 \circthen A_1 \\
      %\t1 M_1(poppedArgs~1, \ldots, poppedArgs~n) \\
      {} \cdots {} \\
      \circelse cid = c_m \circthen A_m \\
      %\t1 M_m(poppedArgs~1, \ldots, poppedArgs~n) \\
      \circfi
    \end{array}
    \circrefines_A
    \begin{array}{l}
      getClassIDOf!arg_1!cid \then {} \\
      \circif cid = c_1 \circthen instantiateArgs(A_1) \\
      %\t1 M_1(arg_1, \ldots, arg_n) \\
      {} \cdots {} \\
      \circelse cid = c_m \circthen instantiateArgs(A_m) \\
      %\t1 M_m(arg_1, \ldots, arg_n) \\
      \circfi
      % \circif cid = c_1 \circthen {} \\
      % \t1 (\circval var1, \ldots, var{<}n{>} : Word \circspot \\
      % \t1 \circvar var{<}n+1{>}, \ldots, var{<}\ell_1{>}) : Word \circspot \\
      % \t1 \circvar stack1, \ldots, stack{<}s_1{>} : Word \circspot \\
      % \t2 {} \cdots {})(arg_1, \ldots, arg_n)) \\
      % {} \circelse cid = c_m \circthen {} \\
      % \t1 (\circval var1, \ldots, var{<}n{>} : Word \circspot \\
      % \t1 \circvar var{<}n+1{>}, \ldots, var{<}\ell_m{>}) : Word \circspot \\
      % \t1 \circvar stack1, \ldots, stack{<}s_m{>} : Word \circspot \\
      % \t2 {} \cdots {})(arg_1, \ldots, arg_n)) \\
      % {} \cdots {} \\
      % \circfi
    \end{array}
  \end{circus}
  where, for each $i \in 1 \upto m$,
  \begin{circus}
    instantiateArgs(M_i(poppedArgs~1, \ldots, poppedArgs~n)) \\
    {} = {} \\
    instantiateArgs(M_i(arg_1, \ldots, arg_n))
  \end{circus}
  and
  \begin{circus}
    instantiateArgs(takeLock!(head~methodArgs) \then takeLockRet \then \Skip \circseq \\
    \t1 M_i(poppedArgs~1, \ldots, poppedArgs~n)) \\
    {} = {} \\
    instantiateArgs(takeLock!arg_1 \then takeLockRet \then \Skip \circseq \\
    \t1 M_i(arg_1, \ldots, arg_n)) \\
  \end{circus}
\end{restatable}
\end{minipage}

% \begin{restatable}[toplevel-$poppedArgs$-elimination]{crule}{ToplevelMethodParameterIntroductionRule}
%   \label{toplevel-method-parameter-introduction-rule}
%   \begin{circus}
%     \begin{array}{l}
%       (\circvar var1, \ldots, var{<}\ell{>} : Word \circspot \\
%       \circvar stack1, \ldots, stack{<}s{>} : Word \circspot \\
%       var1 := methodArgs~1 \circseq \\
%       \t1 \vdots \\
%       var{<}n{>} := methodArgs~n \circseq \\
%       \t1 {} \cdots {})) \\
%       % Remember: Poll starts the \cdots
%     \end{array}
%     \circrefines_A
%     \begin{array}{l}
%       (\circval var1, \ldots, var{<}n{>} : Word \circspot \\
%       \circvar var{<}n+1{>}, \ldots, var{<}\ell{>}) : Word \circspot \\
%       \circvar stack1, \ldots, stack{<}s{>} : Word \circspot \\
%       \t1 {} \cdots {})(methodArgs~1, \ldots, methodArgs~n) \\
%     \end{array}
%   \end{circus}
% \end{restatable}

\begin{minipage}{\textwidth}
  \VarParameterConversionRule*
\end{minipage}

\begin{algorithm}[H]
  \begin{algorithmic}[1]
     % introduce new method action
     \MatchIn{$%
       \begin{array}[t]{l}
         (\circval var1, \ldots, var{<}n{>} : Word \circspot \\
         \circvar var{<}n+1{>}, \ldots, var{<}\ell{>}) : Word \circspot \\
         \circvar stack1, \ldots, stack{<}s{>} : Word \circspot \\
         \t1 A)(arg_1, \ldots, arg_n)
       \end{array}$
     }{\Call{ActionBody}{$methodName$}}
     \State \ApplyFor{Law~[\nameref{action-intro-law}]}{$methodName'$, $%
       \begin{array}[t]{l}
         (\circval var1, \ldots, var{<}n{>} : Word \circspot \\
         \circvar var{<}n+1{>}, \ldots, var{<}\ell{>}) : Word \circspot \\
         \circvar stack1, \ldots, stack{<}s{>} : Word \circspot \\
         \t1 A)(arg_1, \ldots, arg_n)
       \end{array}$
     }
     % copy out new method action
     \State \ApplyReverseToFor{Law~[\nameref{copy-rule-law}]}{\Call{ActionBody}{$methodName$}}{$methodName'$}
     % expand old method action
     \State \ExhaustivelyApplyFor{Law~[\nameref{copy-rule-law}]}{$methodName$}
     % remove old method method action
     \State \ApplyReverseFor{Law~[\nameref{action-intro-law}]}{$methodName$, \Call{ActionBody}{$methodName$}}
     % rename new method action
     \State \ApplyFor{Law~[\nameref{action-rename-law}]}{$methodName'$, $methodName$}
  \end{algorithmic}
  \caption{RedefineMethodActionToExcludeParameters($methodName$)}
  \label{redefine-method-action-to-exclude-parameters-algorithm}
\end{algorithm}

\begin{minipage}{\textwidth}
  \ArgumentVariableEliminationRule*
\end{minipage}

\section{Data Refinement of Objects}
\label{data-refinement-of-objects-appendix-section}

\begin{minipage}{\textwidth}
  \RefineNewObjectRule*
\end{minipage}

\begin{minipage}{\textwidth}
\begin{restatable}[refine-$GetField$]{crule}{RefineGetFieldRule}
  \label{refine-GetField-rule}
  \begin{circus}
    \begin{array}{l}
      \circvar value : Word \circspot \\
      getField?objectID?classID?field \then {}\\
      \circif (objectID \in \dom objects \\ 
      \t1 {} \land (classIDOf~(objects~objectID), classID) \in subclassRel~cs) \circthen {} \\
      \t1 \lschexpract StructManGetField \rschexpract \circseq getFieldRet!value \then \Skip \\
      {} \circelse (objectID \notin \dom objects \\ 
      \t1 {} \lor (classIDOf~(objects~objectID), classID) \notin subclassRel~cs) \circthen \Chaos \\
      \circfi
    \end{array}\\
    \\
    \t2 {} \circrefines_A {} \\
    \\
    \begin{array}{l}
      getField?oid?cid?fid \then {} \\
      \circif oid \in \dom objects \circthen {} \\
      \t1 \circif cid = {<}classID_1{>} \land objects~oid \in \dom cast{<}classID_1{>} \circthen {} \\
      \t2 \circif fid = {<}fieldID_{1,1}{>} \circthen {} \\
      \t3 getFieldRet!((cast{<}classID_1{>}~(objects~oid)).{<}fieldID_{1,1}{>}) \then \Skip \\
      \t3 {} \cdots {} \\
      \t3 {} \circelse fid = {<}fieldID_{1,m_1}{>} \circthen {} \\
      \t3 getFieldRet!((cast{<}classID_1{>}~(objects~oid)).{<}fieldID_{1,m_1}{>}) \then \Skip \\
      \t2 \circfi \\
      \t2 {} \cdots {} \\
      \t1 {} \circelse cid = {<}classID_n{>} \land objects~oid \in \dom cast{<}classID_n{>} \circthen {} \\
      \t2 \circif fid = {<}fieldID_{n,1}{>} \circthen {} \\
      \t3 getFieldRet!((cast{<}classID_n{>}~(objects~oid)).{<}fieldID_{n,1}{>}) \then \Skip \\
      \t3 {} \cdots {} \\
      \t2 {} \circelse fid = {<}fieldID_{n,m_n}{>} \circthen {} \\
      \t3 getFieldRet!((cast{<}classID_n{>}~(objects~oid)).{<}fieldID_{n,m_n}{>}) \then \Skip \\
      \t2 \circfi \\
      \t1 \circfi \\
      {} \circelse oid \notin \dom objects \circthen \Chaos \\
      \circfi
    \end{array}
  \end{circus}
\end{restatable}
\end{minipage}

\begin{minipage}{\textwidth}
\begin{restatable}[refine-$PutField$]{crule}{RefinePutFieldRule}
  \label{refine-PutField-rule}
  \setlength{\zedindent}{0.5cm}
  \setlength{\zedtab}{0.6cm}
  \begin{circus}
    \begin{array}{l}
      putField?objectID?classID?field?value \then {} \\
      \circif (objectID \in \dom objects \\ 
      \t1 {} \land (classIDOf~(objects~objectID), classID) \in subclassRel~cs) \circthen \lschexpract StructManPutField \rschexpract \\
      {} \circelse (objectID \notin \dom objects \\ 
      \t1 {} \lor (classIDOf~(objects~objectID), classID) \notin subclassRel~cs) \circthen \Chaos \\
      \circfi
    \end{array}\\
    \\
    \t2 {} \circrefines_A {} \\
    \\
    \begin{array}{l}
      putField?oid?cid?fid?value \then {} \\
      \circif oid \in \dom objects \circthen {} \\
      \t1 \circif cid = {<}classID_1{>} \land objects~oid \in \dom cast{<}classID_1{>} \circthen {} \\
      \t2 \circif fid = {<}fieldID_{1,1}{>} \circthen {} \\
      \t3 objects := \\
      \t4 objects \oplus \{ oid \mapsto update{<}classID_1{>}\_{<}fieldID_{1,1}{>}~(objects~oid)~value \} \\
      \t3 {} \cdots {} \\
      \t2 {} \circelse fid = {<}fieldID_{1,m_1}{>} \circthen {} \\
      \t3 objects := \\
      \t4 objects \oplus \{ oid \mapsto update{<}classID_1{>}\_{<}fieldID_{1,m_1}{>}~(objects~oid)~value \} \\
      \t2 \circfi \\
      \t2 {} \cdots {} \\
      \t1 {} \circelse cid = {<}classID_n{>} \land objects~oid \in \dom cast{<}classID_n{>}\circthen {} \\
      \t2 \circif fid = {<}fieldID_{n,1}{>} \circthen {} \\
      \t3 objects := \\
      \t4 objects \oplus \{ oid \mapsto update{<}classID_n{>}\_{<}fieldID_{n,1}{>}~(objects~oid)~value \} \\
      \t3 {} \cdots {} \\
      \t2 {} \circelse fid = {<}fieldID_{n,m_n}{>} \circthen {} \\
      \t3 objects := \\
      \t4 objects \oplus \{ oid \mapsto update{<}classID_n{>}\_{<}fieldID_{n,m_n}{>}~(objects~oid)~value \} \\
      \t2 \circfi \\
      \t1 \circfi \\
      {} \circelse oid \notin \dom objects \circthen \Chaos \\
      \circfi
    \end{array}
  \end{circus}
\end{restatable}
\end{minipage}

\begin{minipage}{\textwidth}
\begin{restatable}[refine-$GetStatic$]{crule}{RefineGetStaticRule}
  \label{refine-GetStatic-rule}
  \setlength{\zedindent}{0.5cm}
  \begin{circus}
    \begin{array}{l}
      getStatic?cid?fid \then {} \\
      \circif (cid,fid) \in \dom staticClassFields \circthen {} \\
      \t2 \circvar value : Word \circspot \lschexpract ObjManGetStatic \rschexpract \circseq \\
      \t2 getStaticRet!value \then \Skip \\
      {} \circelse (cid,fid) \notin \dom staticClassFields \circthen \Chaos \\
      \circfi
    \end{array}\\
    \\
    \t2 {} \circrefines_A {} \\
    \\
    \begin{array}{l}
      getStatic?cid?fid \then {} \\
      \circif cid = {<}classID_1{>} \land fid = {<}staticFieldID_{1,1}{>} \circthen {} \\
      \t1 getStaticRet!(staticClassFields.{<}classID_1{>}\_{<}staticFieldID_{1,1}{>}) \then \Skip \\
      \t1 {} \cdots {} \\
      {} \circelse cid = {<}classID_1{>} \land fid = {<}staticFieldID_{1,\ell_1}{>} \circthen {} \\
      \t1 getStaticRet!(staticClassFields.{<}classID_1{>}\_{<}staticFieldID_{1,\ell_1}{>}) \then \Skip \\
      \t1 {} \cdots {} \\
      {} \circelse cid = {<}classID_n{>} \land fid = {<}staticFieldID_{n,1}{>} \circthen {} \\
      \t1 getStaticRet!(staticClassFields.{<}classID_n{>}\_{<}staticFieldID_{n,1}{>}) \then \Skip \\
      \t1 {} \cdots {} \\
      {} \circelse cid = {<}classID_n{>} \land fid = {<}staticFieldID_{n,\ell_n}{>} \circthen {} \\
      \t1 getStaticRet!(staticClassFields.{<}classID_n{>}\_{<}staticFieldID_{n,\ell_n}{>}) \then \Skip \\
      \circfi
    \end{array}
  \end{circus}
\end{restatable}
\end{minipage}

\begin{minipage}{\textwidth}
\begin{restatable}[refine-$PutStatic$]{crule}{RefinePutStaticRule}
  \label{refine-PutStatic-rule}
  \setlength{\zedindent}{0.5cm}
  \setlength{\zedtab}{0.55cm}
  \begin{circus}
    \begin{array}{l}
      putStatic?cid?fid?value \then {} \\
      \circif (cid,fid) \in \dom staticClassFields \circthen \lschexpract StructManPutStatic \rschexpract \\
      {} \circelse (cid,fid) \in \dom staticClassFields \circthen \Chaos \\
      \circfi
    \end{array}\\
    \\
    \t2 {} \circrefines_A {} \\
    \\
    \begin{array}{l}
      putStatic?cid?fid?value \then {} \\
      \circvar staticFieldsID : ObjectID; staticFields : StaticFields \circspot \\
      staticFieldsID := (Initialised\inv)~staticClassFieldsID \circseq \\
      staticFields := staticClassFields~staticFieldsID \circseq \\
      \circif cid = {<}classID_1{>} \land fid = {<}staticFieldID_{1,1}{>} \circthen {} \\
      \t1 staticClassFields := staticClassFields \oplus {} \\
      \t2 \{ staticFieldsID \mapsto updateStatic{<}classID_1{>}\_{<}staticFieldID_{1,1}{>}~staticFields~value \} \\
      \t1 {} \cdots {} \\
      {} \circelse cid = {<}classID_1{>} \land fid = {<}staticFieldID_{1,\ell_1}{>} \circthen {} \\
      \t1 staticClassFields := staticClassFields \oplus {} \\
      \t2 \{ staticFieldsID \mapsto updateStatic{<}classID_1{>}\_{<}staticFieldID_{1,\ell_1}{>}~staticFields~value \} \\
      \t1 {} \cdots {} \\
      {} \circelse cid = {<}classID_n{>} \land fid = {<}staticFieldID_{n,1}{>} \circthen {} \\
      \t1 staticClassFields := staticClassFields \oplus {} \\
      \t2 \{ staticFieldsID \mapsto updateStatic{<}classID_n{>}\_{<}staticFieldID_{n,1}{>}~staticFields~value \} \\
      \t1 {} \cdots {} \\
      {} \circelse cid = {<}classID_n{>} \land fid = {<}staticFieldID_{n,\ell_n}{>} \circthen {} \\
      \t1 staticClassFields := staticClassFields \oplus {} \\
      \t2 \{ staticFieldsID \mapsto updateStatic{<}classID_n{>}\_{<}staticFieldID_{n,\ell_n}{>}~staticFields~value \} \\
      \circfi
    \end{array}
  \end{circus}
\end{restatable}
\end{minipage}

\section{Algebraic Laws Used in the Compilation Strategy}
\label{compilation-strategy-algebraic-laws-section}

\begin{minipage}{\textwidth}
\begin{restatable}[action-intro]{law}{ActionIntroLaw}
  \label{action-intro-law}
  Given an action name $N$ and action body $B$, if $N$ is not
  referenced in the body of $P$ then,
  \begin{circus}
    \begin{array}{l}
      \circprocess P \circdef \circbegin \\
      \t1 {} \cdots {} \\
      \t1 \circstate S \\
      \t1 {} \cdots {} \\
      \t1 \circspot A \\
      \t1 \circend
    \end{array}
    =
    \begin{array}{l}
      \circprocess P \circdef \circbegin \\
      \t1 {} \cdots {} \\
      \t1 \circstate S \\
      \t1 {} \cdots {} \\
      \t1 N \circdef B \\
      \t1 {} \cdots {} \\
      \t1 \circspot A \\
      \t1 \circend
    \end{array}
  \end{circus}
\end{restatable}
\end{minipage}

\begin{minipage}{\textwidth}
\begin{restatable}[action-rename]{law}{ActionRenameLaw}
  \label{action-rename-law}
  Given action names $M$ and $N$, if $N$ is not referenced in the body
  of $P$ then,
  \begin{circus}
    \begin{array}{l}
      \circprocess P \circdef \circbegin \\
      \t1 {} \cdots {} \\
      \t1 \circstate S \\
      \t1 {} \cdots {} \\
      \t1 M \circdef B \\
      \t1 {} \cdots {} \\
      \t1 PPars \\
      \t1 {} \cdots {} \\
      \t1 \circspot A \\
      \t1 \circend
    \end{array}
    =
    \begin{array}{l}
      \circprocess P \circdef \circbegin \\
      \t1 {} \cdots {} \\
      \t1 \circstate S \\
      \t1 {} \cdots {} \\
      \t1 N \circdef B \\
      \t1 {} \cdots {} \\
      \t1 PPars[N/M] \\
      \t1 {} \cdots {} \\
      \t1 \circspot A[N/M] \\
      \t1 \circend
    \end{array}
  \end{circus}
\end{restatable}
\end{minipage}

% \begin{minipage}{\textwidth}
% \begin{restatable}[alt-seq-dist]{law}{AltSeqDistLaw}
%   \label{alt-seq-dist-law}
%   \begin{circus}
%     \circif {} \circelse_{i} g_i \circthen A_i \circfi \circseq B
%     =
%     \circif {} \circelse_{i} g_i \circthen A_i \circseq B \circfi
%   \end{circus}
% \end{restatable}
% \end{minipage}

\begin{minipage}{\textwidth}
\begin{restatable}[assump-elim]{law}{AssumpElimLaw}
  \label{assump-elim-law}
  \begin{circus}
    \{g\} \circrefines_A \Skip
  \end{circus}
\end{restatable}
\end{minipage}

\begin{minipage}{\textwidth}
\begin{restatable}[copy-rule]{law}{CopyRuleLaw}
  \label{copy-rule-law}
  Give an action name $N$, if $N$ names an action in the current process then,
  \begin{circus}
    N(e) = B(N)(e)
  \end{circus}
  where $B$ is a function that returns the body of an action given its
  name.
\end{restatable}
\end{minipage}

\begin{minipage}{\textwidth}
  \begin{restatable}[forwards-data-refinement]{law}{ForwardsDataRefinementLaw}
    \label{forwards-data-refinement-law}
    Given a new process state $S_2$ and a relation $CI$, if $CI$
    relates a process state $S_1$ to $S_2$, with action local state
    $L$, and, for actions $A_1$ and $A_2$,
  \begin{circus}
    \forall S_2; L @ (\exists S_1 @ CI),
  \end{circus}
  and
  \begin{circus}
    \forall S_1; S_2; S_2'; L @ CI \land A_2 \implies (\exists S_1'; L' @ A_1 \land CI' ),
  \end{circus}
  then,
  \begin{circus}
    \begin{array}{l}
      \circprocess P1 \circdef \circbegin \\
      \t1 {} \cdots {} \\
      \t1 \circstate S_1 \\
      \t1 {} \cdots {} \\
      \t1 \circspot A_1 \\
      \t1 \circend
    \end{array}
    \circrefines_P
    \begin{array}{l}
      \circprocess P_2 \circdef \circbegin \\
      \t1 {} \cdots {} \\
      \t1 \circstate S_2 \\
      \t1 {} \cdots {} \\
      \t1 \circspot A_2 \\
      \t1 \circend
    \end{array}
  \end{circus}
  where $A_2$ is such that $A_1 \circsimulates A_2$
\end{restatable}
\end{minipage}

\begin{minipage}{\textwidth}
\begin{restatable}[process-param-elim]{law}{ProcessParamElimLaw}
  \label{process-param-elim-law}
  If $x$ is not referenced in the body of $P$, then
  \begin{circus}
    \begin{array}{l}
      \circprocess P \circdef x : T \circspot \circbegin \\
      \t1 {} \cdots {} \\
      \t1 \circstate S \\
      \t1 {} \cdots {} \\
      \t1 \circspot A \\
      \t1 \circend
    \end{array}
    =
    \begin{array}{l}
      \circprocess P \circdef \circbegin \\
      \t1 {} \cdots {} \\
      \t1 \circstate S \\
      \t1 {} \cdots {} \\
      \t1 \circspot A \\
      \t1 \circend
    \end{array}
  \end{circus}
\end{restatable}
\end{minipage}

\begin{minipage}{\textwidth}
\begin{restatable}[rec-action-intro]{law}{RecActionIntroLaw}
  \label{rec-action-intro-law}
  Given an action $B$,
  \begin{circus}
    (\circmu X \circspot A \circseq X) \circrefines_A (\circmu X \circspot A \circseq X) \circseq B
  \end{circus}
\end{restatable}
\end{minipage}

\begin{minipage}{\textwidth}
\begin{restatable}[rec-rolling-rule]{law}{RecRollingRuleLaw}
  \label{rec-rolling-rule-law}
  Given action functions $F$ and $G$,
  \begin{circus}
    (\circmu X \circspot F(G(X))) = F(\circmu X \circspot G(F(X)))
  \end{circus}
\end{restatable}
\end{minipage}

\begin{minipage}{\textwidth}
\begin{restatable}[seq-unitl]{law}{SeqUnitlLaw}
  \label{seq-unitl-law}
  \begin{circus}
    \Skip \circseq A = A
  \end{circus}
\end{restatable}
\end{minipage}