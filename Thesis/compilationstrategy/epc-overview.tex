The first stage eliminates $pc$ from the state of each thread's
process, $Thr(bc,cs,t)$, introducing the control flow constructs of C
as a result. 
It is summarised by the following theorem.
%
\begin{thm}[Elimination of Program Counter]\label{epc-thm}
  \begin{circus}
    Thr(bc,cs,t) \circrefines ThrCF_{bc,cs}(cs,t)
  \end{circus}%
\end{thm}
%
We act mainly upon the $Running$ action of $Thr$; its loop is unrolled
to introduce the control flow that follows each bytecode instruction.
The aim is to get each method's bytecode instructions into a form in
which the control flow, but not the data operations, are described
using C constructs and, moreover, each path of execution (including
every branch of the conditionals) ends in a return instruction or a
loop.
We refer to a method in this form as a \emph{complete} method.

It is important to observe that it is possible to transform the
bytecode instructions of every method so that they become complete.
If we consider the control flow of a method beginning from that
method's entry point, each bytecode instruction reached must either be
a return instruction, or followed by another bytecode.
If another bytecode follows the bytecode's execution, then it must be
either a bytecode already considered, resulting in a loop, or one not
already considered.
Since there are finitely many bytecode instructions in a method, a
loop or return must eventually be reached.
Failure to do so would lead to an instruction beyond the end of the
method, which is forbidden by the structural restrictions on Java
bytecode that are checked during bytecode verification. 
% We assume bytecode input to our strategy will have undergone bytecode
% verification so this cannot happen.

When a method is complete, it can be defined by a separate \Circus{}
action.
When the code for all the methods has been separated out in this way,
the choice of bytecode instruction using the program counter value can
be removed and replaced with a choice over method identifiers.
Thus dependency on the program counter can be completely removed,
allowing it to be eliminated from the state of $Thr$.

The detailed description of the strategy for transforming $Thr$ in
this stage and achieving this elimination is provided by
Algorithm~\ref{epc-algorithm}.
It begins at line~\ref{algorithm-expand-bytecode} by expanding the
\Circus{} definitions of the bytecode instructions from the $bc$ map
into the $Running$ action, pulling out the program counter updates so
that they can be more easily manipulated.
In line~\ref{algorithm-introduce-forward-sequence}, simple sequential
compositions, that is, those that do not involve handling loops or
conditionals, are introduced.
% instructions that
% are followed by the execution of a bytecode instruction that is not
% part of the start or end of a conditional or loop are sequenced with
% the instructions following them.
After that, for each method, its loops and conditionals are introduced
in line~\ref{algorithm-introduce-loops-and-conditionals}. 
Afterwards, any complete methods are separated out, in
line~\ref{algorithm-separate-complete-methods}, and any method calls
involving completed methods are resolved by sequencing the method call
with the \Circus{} action representing the method, in
line~\ref{algorithm-resolve-method-calls}.

This is repeated until all methods have been separated out, as
indicated by the while loop in line~\ref{algorithm-method-loop}.
The $MainThread$ and $NotStarted$ actions are then refined in
line~\ref{algorithm-refine-main-actions} to provide a choice over
method identifiers, rather than $pc$ values, thus removing all uses of
$pc$ from the interpreter.
The $pc$ component is then removed from the state in
line~\ref{algorithm-remove-pc-from-state} of the algorithm.

\begin{algorithm}[tp!]
  \begin{algorithmic}[1]
    \State \Call{ExpandBytecode}{} \label{algorithm-expand-bytecode}
    \State \Call{IntroduceSequentialComposition}{} \label{algorithm-introduce-forward-sequence}
    \While{$\lnot$\Call{AllMethodsSeparated}{}} \label{algorithm-method-loop}
    \State \Call{IntroduceLoopsAndConditionals}{} \label{algorithm-introduce-loops-and-conditionals}
    \State \Call{SeparateCompleteMethods}{} \label{algorithm-separate-complete-methods}
    \State \Call{ResolveMethodCalls}{} \label{algorithm-resolve-method-calls}
    \EndWhile
    \State \Call{RefineMainActions}{} \label{algorithm-refine-main-actions}
    \State \Call{RemovePCFromState}{} \label{algorithm-remove-pc-from-state}
  \end{algorithmic}
  \caption{Elimination of Program Counter}
  \label{epc-algorithm}
\end{algorithm}

Each of the procedures used in Algorithm~\ref{epc-algorithm} is
defined in a separate section in the sequel.
Beforehand, we give a more detailed overview of the strategy using an
example.