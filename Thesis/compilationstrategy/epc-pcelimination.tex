After $MainThread$ and $Started$ have been refined, $pc$ is no longer
used by $Thr$, and so we can remove it from the state of $Thr$, as
specified in Algorithm~\ref{pc-elimination-algorithm}.
This algorithm operates over the $Thr$ process as a whole, since $pc$
must be removed from every action in $Thr$ simultaneously.

\begin{algorithm}
  \begin{algorithmic}[1]
    \State \ApplyFor{Law~[\nameref{forwards-data-refinement-law}]}{$InterpreterStateEPC$, $CI$}
    \label{algorithm-pc-data-refinement}
    \State \Apply{Law~[\nameref{seq-unitl-law}]}
    \label{algorithm-eliminate-Skips}
    \State \ApplyFor{Law~[\nameref{process-param-elim-law}]}{$bc$}
    \label{algorithm-eliminate-bc-parameter}
  \end{algorithmic}
  \caption{RemovePCFromState}
  \label{pc-elimination-algorithm}
\end{algorithm}

Algorithm~\ref{pc-elimination-algorithm} begins with the application
of Law~[\nameref{forwards-data-refinement-law}] at
line~\ref{algorithm-pc-data-refinement}.
This law describes a standard \Circus{} data refinement between
processes, in which a coupling invariant is defined to describe the
relationship between the old state of the process and the new state of
the process.
We characterise the refinement by providing the new process state and
the coupling invariant.
In this case, the relation defined by the coupling invariant is a
function, so the actions of the new process can be calculated from the
actions of the old process.
Thus, the new state and coupling invariant are sufficient to uniquely
characterise the data refinement.

For our refinement, the new state is $InterpreterStateEPC$, shown
below.
It is similar to $InterpreterState$, but the $pc$ component is
removed.
The $frameStack$ also has a different type, being a sequence of
$StackFrameEPC$ structures, which are similar to $StackFrame$, but
without the $storedPC$ component, since that is only used for storing
a value from $pc$.
The invariant of $InterpreterStateEPC$ is the same as for
$InterpreterState$, but without the requirement that the
$currentClass$ and stack frame $frameClass$ values be consistent with
the $pc$ value.
\begin{schema}{InterpreterStateEPC}
  frameStackID : Stack \\
  frameStack : \seq StackFrameEPC \\
  % pc : ProgramAddress \\
  currentClass : Class
\where
  frameStackID = Uninitialised \implies frameStack = \emptyset \\
  % definition of currentClass (only important if the frame stack is nonempty)
  frameStack \neq \emptyset \implies currentClass = (last~frameStack).frameClass
  % need to ensure pc is consistent with currentClass
  % currentClass = (\mu c : \ran cs | \\
  % \t1 (\exists_1 m : MethodID | m \in \dom c.methodEntry @  \\
  % \t2 pc \in c.methodEntry~m \upto c.methodEnd~m)) \\
  % \forall f : \ran frameStack @ f.frameClass = (\mu c :\ran cs | \\
  % \t1 (\exists_1 m : MethodID | m \in \dom c.methodEntry @  \\
  % \t2 f.storedPC \in c.methodEntry~m \upto c.methodEnd~m))
\end{schema}

% \begin{schema}{StackFrameEPC}
%   localVariables : \seq Word \\
%   operandStack : \seq Word \\
%   % storedPC : ProgramAddress \\
%   frameClass : Class \\
%   stackSize : \nat
% \where
%   \# operandStack \leq stackSize
% \end{schema}

The coupling invariant, $CI$, for the refinement from
$InterpreterState$ to $InterpreterStateEPC$ is shown below, with
$InterpreterStateEPC$ decorated with ${}_1$ to distinguish its
components.
$CI$ equates the $frameStackID$ and $currentClass$ components of the
two schemas, since they are unaffected.
The $frameStack$ components are declared to have the same domain, and
each $StackFrame$ in $frameStack$ is mapped onto a $StackFrameEPC$
with the same $localVariables$, $operandStack$, $frameClass$, and
$stackSize$ values.
The $pc$ and $storedPC$ values of $InterpreterState$ are discarded,
since they are not present in $InterpreterStateEPC$.
\begin{schema}{CI}
  InterpreterState \\
  InterpreterStateEPC_1
\where
  frameStackID = frameStackID_1 \\
  currentClass = currentClass_1 \\
  \dom frameStack = \dom frameStack_1 \\
  \forall i : \dom frameStack @ \\
  \t1 (frameStack~i).localVariables = (frameStack_1~i).localVariables \land \\
  \t1 (frameStack~i).operandStack = (frameStack_1~i).operandStack \land \\
  \t1 (frameStack~i).frameClass = (frameStack_1~i).frameClass \land \\
  \t1 (frameStack~i).stackSize = (frameStack_1~i).stackSize
\end{schema}

This data refinement has the effect of removing $pc$ from each of the
data operations in $Thr$.
This effect is minimal for most data operations, since their $pc$
updates have already been extracted. 
However, $InterpreterStackFrameInvoke$ no longer stores the current
$pc$ value in the $storedPC$ component of the topmost stack frame, and
$InterpreterNewStackFrame$ does not set the $pc$ value.
Additionally, the method return operations $InterpreterAreturn$ and
$IntepreterReturn$ do not set set the value of $pc$ using the
$storedPC$ value of the previous stack frame.

The $pc$ assignments introduced between bytecode instructions during
the strategy are also affected by the data refinement.
The data refinement removes $pc$ from the assignments, leaving only
their effect on the other components of the state.
Since the assignments leave all other state components unchanged, the
data-refined $pc$ assignments have no effect, making them equivalent
to $\Skip$.
These $\Skip$ actions are then eliminated by applying
Law~[\nameref{seq-unitl-law}] wherever possible, at
line~\ref{algorithm-eliminate-Skips} of
Algorithm~\ref{pc-elimination-algorithm}.

Finally, we eliminate the $bc$ parameter to the process, since it is
also no longer needed, using an application of
Law~[\nameref{process-param-elim-law}], at
line~\ref{algorithm-eliminate-bc-parameter}.
This completes the refinement of $Thr(bc,cs,t)$ into
$ThrCF_{bc,cs}(cs,t)$, referenced in
Theorem~\ref{epc-thm}, which has its control flow
introduced and does not include $pc$ in its state.
The next stage of the strategy operates on $ThrCF_{bc,cs}(cs,t)$ to
eliminate the $frameStack$.
