Following the introduction of local $stackFrame$ variables, we perform
local data refinements to introduce the local variables and stack
slots for each $stackFrame$, on
line~\ref{algorithm-introduce-variables} of
Algorithm~\ref{efs-algorithm}.
This is performed as described in
Algorithm~\ref{introduce-variables-algorithm}, which defines the
\Call{IntroduceVariables}{} procedure.

\begin{algorithm}
  \begin{algorithmic}[1]
    \For{$methodName \gets$ \Call{MethodNames}{$cs$}}
    \label{algortihm-introduce-variables-loop}
    \State \Call{IntroduceFrameClassAssumptions}{{\Call{ActionBody}{$methodName$}}}
    \label{algorithm-introduce-frameClass-assumptions}
    \State \ExhaustivelyApplyTo{Rule~[\nameref{refine-PutfieldSF-rule}]}{\Call{ActionBody}{$methodName$}}
    \label{algorithm-apply-refine-PutfieldSF}
    \State \ExhaustivelyApplyTo{Rule~[\nameref{refine-GetfieldSF-rule}]}{\Call{ActionBody}{$methodName$}}
    \State \ExhaustivelyApplyTo{Rule~[\nameref{refine-PutstaticSF-rule}]}{\Call{ActionBody}{$methodName$}}
    \State \ExhaustivelyApplyTo{Rule~[\nameref{refine-GetstaticSF-rule}]}{\Call{ActionBody}{$methodName$}}
    \State \ExhaustivelyApplyTo{Rule~[\nameref{refine-NewSF-rule}]}{\Call{ActionBody}{$methodName$}}
    \label{algorithm-apply-refine-NewSF}
    \State \ExhaustivelyApplyTo{Law~[\nameref{assump-elim-law}]}{\Call{ActionBody}{$methodName$}}
    \label{algorithm-frameClass-assump-elim}
    \State \ExhaustivelyApplyTo{Law~[\nameref{seq-unitl-law}]}{\Call{ActionBody}{$methodName$}}
    \label{algorithm-frameClass-seq-unitl}
    % introduce operand stack assumptions
    \State \Call{IntroduceOperandStackAssumptions}{\Call{ActionBody}{$methodName$}}
    \label{algorithm-introduce-operandStack-assumptions}
    \State $\ell \gets$ \Call{MethodLocals}{$methodName$}
    \label{algorithm-get-method-locals}
    \State $s \gets$ \Call{MethodStackSize}{$methodName$}
    \label{algorithm-get-method-stack-size}
    \State {\bf apply}
    \label{algorithm-local-data-refinement}
    {Law~[\nameref{forwards-data-refinement-law}]}
    ({\arraycolsep=0cm
      $\begin{array}[t]{l}
         [var1, \ldots, var{<}\ell{>} : Word;\\
          stack1, \ldots, stack{<}s{>} : Word]
       \end{array}$,
       $V{<}\ell{>}S{<}s{>}CI$})
     \Statex $\t1${\bf to} {\Call{ActionBody}{$methodName$}}
    \State \ExhaustivelyApplyTo{Law~[\nameref{assump-elim-law}]}{\Call{ActionBody}{$methodName$}}
    \label{algorithm-operandStack-assump-elim}
    \State \ExhaustivelyApplyTo{Law~[\nameref{seq-unitl-law}]}{\Call{ActionBody}{$methodName$}}
    \label{algorithm-operandStack-seq-unitl}
    \State \ExhaustivelyApplyTo{Rule~[\nameref{eliminate-value1-value2-conditional-rule}]}{\Call{ActionBody}{$methodName$}}
    \label{algorithm-apply-eliminate-value1-value2-conditional-rule}
    \State \ExhaustivelyApplyTo{Rule~[\nameref{eliminate-oid-getField-rule}]}{\Call{ActionBody}{$methodName$}}
    \State \ExhaustivelyApplyTo{Rule~[\nameref{eliminate-oid-value-putField-rule}]}{\Call{ActionBody}{$methodName$}}
    \State \ExhaustivelyApplyTo{Rule~[\nameref{eliminate-value-putStatic-rule}]}{\Call{ActionBody}{$methodName$}}
    \label{algorithm-apply-eliminate-value-putStatic-rule}
    \State \ExhaustivelyApply{Rule~[\nameref{method-parameter-introduction-rule}]}
    \label{algorithm-poppedArgs-elimination}
    \State \ExhaustivelyApply{Rule~[\nameref{poppedArgs-sync-elim-rule}]}
    %\State \ExhaustivelyApply{Rule~[\nameref{getClassIDOf-method-parameter-introduction-rule}]}
    \State \ExhaustivelyApply{Rule~[\nameref{getClassIDOf-multi-method-parameter-introduction-rule}]}
    \label{algorithm-multi-poppedArgs-elimination}
    \MatchIn{
      $\circmu X \circspot A \circseq
      \begin{array}[t]{l}
        \circif b \circthen B \circseq X \\
        {} \circelse \lnot b \circthen \Skip \\
        \circfi
      \end{array}$
     }{\Call{ActionBody}{$methodName$}}
     \State \ExhaustivelyApplyFor{Law~[\nameref{rec-rolling-rule-law}]}{%
       \arraycolsep=0cm
       $(\lambda X \circspot A \circseq X)$,
       $(\lambda X \circspot
       \begin{array}[t]{l}
         \circif b \circthen B \circseq X \\
         {} \circelse \lnot b \circthen \Skip \\
         \circfi
       \end{array})$%
     }
     \label{algorithm-move-loop-start}
     % convert local vars to parameters
     \State \ApplyTo{Rule~[\nameref{var-parameter-conversion-rule}]}{\Call{ActionBody}{$methodName$}}
     \label{algorithm-var-parameter-conversion}
     \State \Call{RedefineMethodActionToExcludeParameters}{$methodName$}
     \label{algorithm-redefine-method-action-to-exclude-parameters}
     % remove unnecessary parameters
     \State \ApplyFor{Rule~[\nameref{argument-variable-elimination-rule}]}{$methodName$}
     \label{algorithm-argument-variable-elimination}     
    \EndFor
  \end{algorithmic}
  \caption{\Call{IntroduceVariables}{}}
  \label{introduce-variables-algorithm}
\end{algorithm}

Algorithm~\ref{introduce-variables-algorithm} operates upon each of
the method actions in turn on
line~\ref{algortihm-introduce-variables-loop}, determining the names
of the actions from $cs$ via the function \Call{MethodNames}{}.
Within this loop we refer to the name of the method action under
consideration as $methodName$.
Unlike the introduction of the $stackFrame$ variables, the order in
which the methods are iterated over does not matter, since each has
its own $stackFrame$ variable that undergoes local refinement.

We first refine field access operations to remove their reliance on
the $frameClass$ component of the $stackFrame$, which is removed in
the data refinement later in this section.
This is done by first introducing and distributing assumptions stating
the value of $frameClass$, using the procedure
\Call{IntroduceFrameClassAssumptions}{} on
line~\ref{algorithm-introduce-frameClass-assumptions}, which is
applied to the body of $methodName$.
This procedure is similar to \Call{IntroduceFrameStackAssumptions}{}
in that it introduces an assumption and distributes it with restricted
forms of standard algebraic laws.
However, \Call{IntroduceFrameClassAssumptions}{} acts on the body of a
single method and introduces an assumption about the value of the
$frameClass$ component of $stackFrame$ from the schema action
initialising $stackFrame$.
The \Call{IntroduceFrameClassAssumptions}{} procedure is defined by
Algorithm~\ref{introduce-frameClass-assumptions-algorithm}, which we
omit here, but is included in
Appendix~\ref{introduce-variables-appendix-subsection}.

We can then apply Rules~[\nameref{refine-PutfieldSF-rule}],
[\nameref{refine-GetfieldSF-rule}],
[\nameref{refine-PutstaticSF-rule}],
[\nameref{refine-GetstaticSF-rule}] and~[\nameref{refine-NewSF-rule}]
wherever possible to refine the field accesses and object creation
actions, on lines~\ref{algorithm-apply-refine-PutfieldSF}
to~\ref{algorithm-apply-refine-NewSF}.
As an example of one of these rules, we show
Rule~[\nameref{refine-PutfieldSF-rule}] in
Figure~\ref{refine-PutfieldSF-rule-figure}.
It refines a $PutfieldSF(cpi)$ instruction preceded by an assumption
stating the value of the $frameClass$ component of $stackFrame$.
With the application of the rule, the definition of $PutfieldSF$ is
expanded and the class identifier, $cid$, and field identifier, $fid$,
at the constant pool index $cpi$ are substituted in place of the
accesses to the constant pool.
This removes the reference to the $constantPool$ of the $frameClass$,
and hence the reference to the $frameClass$.
Rule~[\nameref{refine-GetfieldSF-rule}],
Rule~[\nameref{refine-PutstaticSF-rule}],
Rule~[\nameref{refine-GetstaticSF-rule}] and
Rule~[\nameref{refine-NewSF-rule}] are similar so we omit them here.
They can be found, along with the other compilation rules, in
Appendix~\ref{introduce-variables-appendix-subsection}.
After these laws have been applied, we eliminate any remaining
$frameClass$ assumptions by applying Law~[\nameref{assump-elim-law}]
and Law~[\nameref{seq-unitl-law}] on
lines~\ref{algorithm-frameClass-assump-elim}
and~\ref{algorithm-frameClass-seq-unitl}.

\begin{figure}
  \centering
  % \setlength{\zedtab}{0.4cm}
  % \setlength{\zedindent}{0pt}
  % \setlength{\zedleftsep}{0pt}
  % \setlength{\abovedisplayskip}{0pt}
  % \setlength{\belowdisplayskip}{0pt}
  % \setlength{\abovedisplayshortskip}{0pt}
  % \setlength{\belowdisplayshortskip}{0pt}
  \begin{restatable}[refine-$PutfieldSF$]{crule}{RefinePutfieldSFRule}
    \label{refine-PutfieldSF-rule}
    \begin{circus}
      \begin{array}{l}
        \{stackFrame.frameClass = c\} \circseq \\
        PutfieldSF(cpi)
      \end{array}
      \circrefines_A
      \begin{array}{l}
        (\circvar oid : ObjectID; value : Word \circspot \\
        \t1 \lschexpract InterpreterPop[ \\
        \t2 stackFrame/last~frameStack, \\
        \t2 stackFrame'/last~frameStack'] \rschexpract \circseq \\
        \t1 \lschexpract InterpreterPop[ \\
        \t2 oid!/value!, \\
        \t2 stackFrame/last~frameStack, \\
        \t2 stackFrame'/last~frameStack'] \rschexpract \circseq \\
        \t1 putField!oid!cid!fid!value \then \Skip)
      \end{array}
    \end{circus}
    where
    \begin{circus}
      cpi \in fieldRefIndices~c \land \\
      c.constantPool~cpi = FieldRef~(cid, fid)
    \end{circus}
  \end{restatable}
  \caption{Rule~[\nameref{refine-PutfieldSF-rule}]}
  \label{refine-PutfieldSF-rule-figure}
\end{figure}

After references to the $frameClass$ have been removed, we can perform
a local data refinement on the body of the method to convert the
$stackFrame$ variable to separate variables for the local variables
and operand stack slots.
Since we are converting the $operandStack$ component of $stackFrame$
from a sequence to a fixed set of variables representing an array of
stack slots, we must know the length of $operandStack$ before each
operation in order to determine which variable corresponds to the top
of the stack, and hence should be affected by the operation.
We ensure that this information is available, by introducing and
distributing assumptions on the size of $operandStack$.
This is performed by the \Call{IntroduceOperandStackAssumptions}{}
procedure, called on
line~\ref{algorithm-introduce-operandStack-assumptions}.
It is similar to \Call{IntroduceFrameClassAssumptions}{} and is
defined by
Algorithm~\ref{introduce-operandStack-assumptions-algorithm}, which is
included in Appendix~\ref{introduce-variables-appendix-subsection}.

After an $operandStack$ assumption has been introduced before each
data operation, the local data refinement is performed on
line~\ref{algorithm-local-data-refinement}.
The new state for the data refinement contains the local variables,
which are all of type $Word$, and are named $var$ followed by an
integer beginning at $1$ and going up to the total number of local
variables for the method, $\ell$.
It also contains operand stack slots, named $stack$ followed by an
integer from $1$ up to the maximum stack size for the method, $s$.
These values $\ell$ and $s$ are obtained from the $methodLocals$ and
$methodStackSize$ information for the method in $cs$, on
lines~\ref{algorithm-get-method-locals}
and~\ref{algorithm-get-method-stack-size} of
Algorithm~\ref{introduce-variables-algorithm}.
For example, the values associated with $handleAsyncEvent$ in the
$TPK$ class in Figure~\ref{example-model-figure} are $6$ for $\ell$
and $3$ for $s$. 
The coupling invariant for the data refinement of a method $m$ is then
given by the template $V{<}\ell{>}S{<}s{>}CI$, shown below.

% I don't think this is functional...does that matter?
\begin{schema}{V{<}\ell{>}S{<}s{>}CI}
  stackFrame : StackFrameEPC \\
  var1, \ldots, var{<}\ell{>} : Word \\
  stack1, \ldots, stack{<}s{>} : Word
\where
  \# stackFrame.localVariables = {<}\ell{>} \\
  stackFrame.localVariables~1 = var1 \\
  \t1 \vdots \\
  stackFrame.localVariables~{<}\ell{>} = var{<}\ell{>} \\
  stackFrame.stackSize = {<}s{>} \\
  \# stackFrame.operandStack \geq 1 \implies \\
  \t1 stackFrame.operandStack~1 = stack1 \\
  \t1 \vdots \\
  \# stackFrame.operandStack \geq {<}s{>} \implies \\
  \t1 stackFrame.operandStack~{<}s{>} = stack{<}s{>}
\end{schema}

$V{<}\ell{>}S{<}s{>}CI$ requires the number of local variables to be
equal to $\ell$, relates each of the values in the $localVariables$
sequence in $stackFrame$ to the corresponding local variables, $var1$
to $var{<}\ell{>}$.
It also requires the maximum operand stack size, $stackSize$, to be
$s$ and relates each value in $operandStack$ to the corresponding
stack slots, $stack1$ to $stack{<}s{>}$, but only if $operandStack$ is
long enough to contain such a value.
The values of the stack slots outside the length of the $operandStack$
at each point in the program are not specified, and so are chosen
nondeterministically, since they are not used until they have been
initialised with the correct value.
This nondeterminism allows us to avoid introducing unnecessary
assignments to initialise the stack slots and return them to a default
value when they are no longer used, which would be required if we
specified a value for unused stack slots in the coupling invariant.

However, the nondeterminism in $V{<}\ell{>}S{<}s{>}CI$ means that it
does not define a function, since there are multiple possible states
for the operand stack slots that correspond to a non-full
$operandStack$.
This means that we cannot directly compute the actions resulting from
the refinement (since there are multiple possibilities). 
So we must specify how each of the data operations is refined.
We, therefore, state compilation rules in terms of \Circus{}
simulations between actions.

Most of the bytecode instructions at this stage in the strategy have
their semantics stated in terms of a data operation over $stackFrame$,
in the form of a $Handle*SF$ action.
We state the simulations for such instructions as simulations of a
$Handle*SF$ action preceded by an $operandStack$ size assumption,
which can be viewed as adding an extra precondition to the action.
An example of such a simulation is
Rule~[\nameref{HandleAloadSF-simulation-rule}], shown in
Figure~\ref{HandleAloadSF-simulation-rule-figure}.
It states that $HandleAloadSF(lvi)$, with an assumption that the size
of $operandStack$ is $k$, is simulated by an assignment
$stack{<}k+1{>} := var{<}lvi{>}$.
This rule applies, for example, to the $HandleAloadSF(1)$ action
deriving from the $aload~1$ instruction at $pc = 12$ in
$TPK\_handleAsyncEvent$, which is refined to $stack1 := var1$, since
the stack is empty at that point.

\begin{figure}[thp]
  \begin{restatable}[$HandleAloadSF$-simulation]{crule}{HandleAloadSFSimulationRule}
    \label{HandleAloadSF-simulation-rule}
    \begin{circus}
      \begin{array}{l}
        \{\# stackFrame.operandStack = k\} \circseq \\
        HandleAloadSF(lvi)
      \end{array}
      \circsimulates
      \begin{array}{l}
        stack{<}k+1{>} := var{<}lvi{>}
      \end{array}
    \end{circus}
  \end{restatable}
  \caption{Rule~[\nameref{HandleAloadSF-simulation-rule}]}
  \label{HandleAloadSF-simulation-rule-figure}
\end{figure}

The instructions that manipulate objects by communicating with the
object manager have already had their definitions expanded earlier in
this stage.
Their communication with the object manager need not be changed by the
data refinement.
However, the data operations used by these operations to pop or push
the values communicated from or to the operand stack must be refined.
An example of the simulation for such an operation is
Rule~[\nameref{InterpreterPopEPC-simulation-rule}], shown in
Figure~\ref{InterpreterPopEPC-simulation-rule-figure}.
This establishes a simulation between $InterpreterPopEPC$, modified to
act over $stackFrame$, and an assignment of a stack slot value to the
variable $value$, which is in scope in the contexts where
$InterpreterPopEPC$ is used.

\begin{figure}[thp]
  \begin{restatable}[$InterpreterPopEPC$-simulation]{crule}{InterpreterPopEPCSimulationRule}
    \label{InterpreterPopEPC-simulation-rule}
    \begin{circus}
      \begin{array}{l}
        \{\# stackFrame.operandStack = k\} \circseq \\
        \lschexpract InterpreterPopEPC[ \\
        \t1 stackFrame/last~frameStack, \\
        \t1 stackFrame'/last~frameStack']\rschexpract
      \end{array}
      \circsimulates
      \begin{array}{l}
        value := stack{<}k{>}
      \end{array}
    \end{circus}
  \end{restatable}
  \caption{Rule~[\nameref{InterpreterPopEPC-simulation-rule}]}
  \label{InterpreterPopEPC-simulation-rule-figure}
\end{figure}

Method invocations also use data operations to pop the method's
arguments from the stack and pass them to the method, which must be
refined in the data refinement.
The $InvokeSF$ operation, which pops the method's arguments from the
stack, is simulated by an assignment of a sequence of operand stack
values to the $poppedArgs$ variables, as stated in
Rule~[\nameref{InvokeSF-simulation-rule}], shown in
Figure~\ref{InvokeSF-simulation-rule-figure}.

\begin{figure}[thp]
  \begin{restatable}[$InvokeSF$-simulation]{crule}{InvokeSFSimulationRule}
    \label{InvokeSF-simulation-rule}
    \begin{circus}
      \begin{array}{l}
        \{\# stackFrame.operandStack = k\} \circseq \\
        \lschexpract \exists argsToPop? == m @ InvokeSF \rschexpract
      \end{array}
      \circsimulates
      \begin{array}{l}
        poppedArgs := \\
        \t1 \langle stack{<}k-m+1{>}, \ldots , stack{<}k{>} \rangle
      \end{array}
    \end{circus}
  \end{restatable}
  \caption{Rule~[\nameref{InvokeSF-simulation-rule}]}
  \label{InvokeSF-simulation-rule-figure}
\end{figure}

The passing of the arguments to the invoked method has already been
refined in the introduction of the $stackFrame$ variable, but the
schema initialising $stackFrame$ must be further refined to initialise
the local variables.
The simulation for the $stackFrame$ initialisation schema is stated by
Rule~[\nameref{stackFrame-init-simulation-rule}], shown in
Figure~\ref{stackFrame-init-simulation-rule-figure}.
It is simulated by a sequence of assignments setting the local
variables to the values of the arguments.
The initialisation sets $operandStack$ to be empty, so there is no
need to assign values to the stack slot variables; they can be left
arbitrary.

\begin{figure}[thp]
  \begin{restatable}[$stackFrame$-init-simulation]{crule}{StackFrameInitSimulationRule}
    \label{stackFrame-init-simulation-rule}
    \begin{circus}
      \begin{array}{l}
        \lschexpract [arg1?, \ldots, arg{<}n{>}? : Word; \\
        \t1 stackFrame' : StackFrameEPC | \\
        \t1 \langle arg_1, \ldots, arg_n \rangle \subseteq stackFrame'.localVariables \land \\
        \t1 \# stackFrame'.localVariables = \ell \land \\
        \t1 stackFrame'.operandStack = \langle\rangle \land \\
        \t1 stackFrame'.frameClass = c \land \\
        \t1 stackFrame'.stackSize = s] \rschexpract
      \end{array}
      \circsimulates
      \begin{array}{l}
        var{<}1{>} := arg_1 \circseq \\
        \t1 \vdots \\
        var{<}n{>} := arg_n
      \end{array}
    \end{circus}
  \end{restatable}
  \caption{Rule~[\nameref{stackFrame-init-simulation-rule}]}
  \label{stackFrame-init-simulation-rule-figure}
\end{figure}

The simulation rules we have omitted here can be found with the rest
of the compilation rules in
Appendix~\ref{introduce-variables-appendix-subsection}.
These, together with the standard laws for distributing simulations
through \Circus{} constructs, are sufficient to unambiguously define
the local data refinement to be applied to the method.
After the data refinement, we eliminate any remaining assumptions by
applying Law~[\nameref{assump-elim-law}] and
Law~[\nameref{seq-unitl-law}] on
lines~\ref{algorithm-operandStack-assump-elim}
and~\ref{algorithm-operandStack-seq-unitl}.

We then eliminate the additional variables used in the data operations
that pop values from the stack.
Those that push values to the stack are pushing values received from a
channel, which require a separate assignment operation and so cannot
be eliminated.
The additional variables are eliminated using the rules applied on
lines~\ref{algorithm-apply-eliminate-value1-value2-conditional-rule}
to~\ref{algorithm-apply-eliminate-value-putStatic-rule} of
Algorithm~\ref{introduce-variables-algorithm}.
In particular,
Rule~[\nameref{eliminate-value1-value2-conditional-rule}], shown in
Figure~\ref{eliminate-value1-value2-conditional-rule-figure}, applies
to the $TPK\_handleAsyncEvent$ action in our example.
It removes the need for additional $value1$ and $value2$ variables,
replacing the references to them in the conditional with the stack
slot variables whose values they store.

\begin{figure}
  \begin{restatable}[cond-$value1$-$value2$-elim]{crule}{EliminateValueOneValueTwoConditional}
    \label{eliminate-value1-value2-conditional-rule}
    \begin{circus}
      \begin{array}{l}
        (\circvar value1, value2 : Word \circspot \\
        \t1 value1 := stack{<}k{>} \circseq \\
        \t1 value2 := stack{<}k+1{>} \circseq \\
        \t1 \circif value1 \leq value2 \circthen {} \\
        \t2 {} \cdots {} \\
        \t1 {} \circelse value1 > value2 \circthen {} \\
        \t2 {} \cdots {} \\
        \t1 \circfi)
      \end{array}
      \circrefines_A
      \begin{array}{l}
        \circif stack{<}k{>} \leq stack{<}k+1{>} \circthen {} \\
        \t1 {} \cdots {} \\
        {} \circelse stack{<}k{>} > stack{<}k+1{>} \circthen {} \\
        \t1 {} \cdots {} \\
        \circfi)
      \end{array}
    \end{circus}
  \end{restatable}
  \caption{Rule~[\nameref{eliminate-value1-value2-conditional-rule}]}
  \label{eliminate-value1-value2-conditional-rule-figure}
\end{figure}

We also eliminate the intermediate $poppedArgs$ variable used when
passing variables to method calls in the body of a method.
This is performed by the rules applied on
lines~\ref{algorithm-poppedArgs-elimination}
to~\ref{algorithm-multi-poppedArgs-elimination}, which are applied to
every method call in the body of $methodName$.
These rules eliminate $poppedArgs$ and copy the values stored in it
into the values passed to the value parameters of the called method.
This can be seen in
Rule~[\nameref{method-parameter-introduction-rule}], shown in
Figure~\ref{method-parameter-introduction-rule-figure}, which
eliminates $poppedArgs$ when associated with method calls arising from
a \texttt{invokespecial} or \texttt{invokestatic} instruction.
Rule~[\nameref{poppedArgs-sync-elim-rule}] and
Rule~[\nameref{getClassIDOf-multi-method-parameter-introduction-rule}]
are similar, but account for the extra communication before
synchronized methods, and the extra $getClassIDOf$ communication and
multiple targets arising from \texttt{invokevirtual} instructions,
respectively.

\begin{figure}[thp]
  \begin{restatable}[$poppedArgs$-elim]{crule}{MethodParameterIntroductionRule}
    \label{method-parameter-introduction-rule}
    \begin{circus}
      \begin{array}{l}
        (\circvar poppedArgs : \seq Word \circspot \\
        poppedArgs := \langle arg_1, \ldots, arg_n \rangle \circseq \\
        M(poppedArgs~1, \ldots, poppedArgs~n))
      \end{array}
      \circrefines_A
      \begin{array}{l}
        M(arg_1, \ldots, arg_n)
      \end{array}
    \end{circus}
  \end{restatable}
  \caption{Rule~[\nameref{method-parameter-introduction-rule}]}
  \label{method-parameter-introduction-rule-figure}
\end{figure}

We also move any actions that are at the start of a loop before the
loop condition is checked.
Such actions cannot be represented in C without the use of the comma
operator, which is not allowed in MISRA-C.
The actions are moved, using Law~[\nameref{rec-rolling-rule-law}], so
that they are before the start of the loop and at the end of the loop
body.
This is performed on line~\ref{algorithm-move-loop-start}.

After these rules have been applied to the body of
$TPK\_handleAsyncEvent$, it has the form shown in
Figure~\ref{efs-introduce-variables-mid-example-figure}.
The effect of these rules can be seen in the fact that values stored
in stack slots such as $stack1$ are passed directly to the arguments
of called functions.
The conditionals also compare stack slots directly and the assignments
to those stack slots ($stack1 := var4$ and $stack2 := 10$) have been
moved to before the start of the loop and just before the end of the
loop, rather than just after the beginning of the loop.
The argument to the function is passed in via the $arg1$ variable and
assigned to the local variable $var1$. 
This indirection is unnecessary, and we wish instead to have the
argument passed directly into $var1$.
We thus perform some final transformations to turn the local variables
corresponding to the methods arguments into parameters and eliminate
the $arg1, \ldots, arg{<}n{>}$ parameters for the method.

\begin{figure}[tp!]
  \centering
  \setlength{\zedtab}{0.5cm}
  \setlength{\zedindent}{0pt}
  \setlength{\zedleftsep}{0pt}
  \setlength{\abovedisplayskip}{0pt}
  \setlength{\belowdisplayskip}{0pt}
  \setlength{\abovedisplayshortskip}{0pt}
  \setlength{\belowdisplayshortskip}{0pt}
  \begin{circusaction}
    TPK\_handleAsyncEvent \circdef \circval arg1 : Word \circspot \\
    \t1 \circvar var1 var2, var3, var4, var5, var6 : Word \circspot \circvar stack1, stack2, stack3 : Word \circspot Poll \circseq \\
    \t1 var1 := arg1
    \t1 newObject!thread!ConsoleConnectionClassID \\
    \t2 {} \then  newObjectRet!oid \then stack1 := oid \circseq Poll \circseq \\
    \t1 stack2 := stack1 \circseq Poll \circseq \\
    \t1 stack3 := null \circseq Poll \circseq \\
    \t1 {} \cdots {} \\
    % \t1 ConsoleConnection\_CCinit(stack2, stack3) \circseq Poll \circseq \\
    % \t1 var1 := stack1 \circseq Poll \circseq \\
    % \t1 stack1 := var1 \circseq Poll \circseq \\
    % \t1 getClassIDOf!stack1?cid \\
    % \t2 {} \then ConsoleConnection\_openInputStream(stack1, stack1) \circseq Poll \circseq \\
    % \t1 var2 := stack1 \circseq Poll \circseq \\
    % \t1 stack1 := var1 \circseq Poll \circseq \\
    % \t1 getClassIDOf!stack1?cid \\
    % \t2 {} \then ConsoleConnection\_openOutputStream(stack1, stack1) \circseq Poll \circseq \\
    % \t1 var3 := stack1 \circseq Poll \circseq \\
    % \t1 stack1 := 0 \circseq Poll \circseq \\
    \t1 var4 := stack1 \circseq Poll \circseq Poll \circseq \\
    \t1 stack1 := var4 \circseq Poll \circseq \\
    \t1 stack2 := 10 \circseq Poll \circseq \\
    \t1 \circmu Y \circspot \\
    \t2 \circif stack1 \leq stack2 \circthen Poll \circseq \\
    \t3 stack1 := var2 \circseq Poll \circseq \\
    \t3 getClassIDOf!stack1?cid \then ConsoleInput\_read(stack1, stack1) \circseq Poll \circseq \\
    \t3 TPK\_f(stack1, stack1) \circseq Poll \circseq \\
    \t3 var5 := stack1 \circseq Poll \circseq \\
    \t3 {} \cdots {} \\
    % \t3 stack1 := var5 \circseq Poll \circseq \\
    % \t3 stack2 := 400 \circseq Poll \circseq \\
    % \t3 \circif stack1 \leq stack2 \circthen Poll \circseq \\
    % \t4 stack1 := var3 \circseq Poll \circseq \\
    % \t4 stack2 := var5 \circseq Poll \circseq \\
    % \t4 getClassIDOf!stack1?cid \then ConsoleOutput\_write(stack1, stack2) \\
    % \t3 {} \circelse stack1 > stack2 \circthen Poll \circseq \\
    % \t4 stack1 := var3 \circseq Poll \circseq \\
    % \t4 stack2 := 0 \circseq Poll \circseq \\
    % \t4 getClassIDOf!stack1?cid \then ConsoleOutput\_write(stack1, stack2) \\
    % \t3 \circfi \circseq Poll \circseq \\
    % \t3 stack1 := var4 \circseq Poll \circseq \\
    % \t3 stack2 := 1 \circseq Poll \circseq \\
    % \t3 stack1 := stack1 + stack2 \circseq Poll \circseq \\
    % \t3 var4 := stack1 \circseq Poll \\
    \t3 stack1 := var4 \circseq Poll \circseq \\
    \t3 stack2 := 10 \circseq Poll \circseq Y \\
    \t2 {} \circelse stack1 > stack2 \circthen \Skip \\
    \t2 \circfi \circseq Poll
  \end{circusaction}
  \caption{$TPK\_handleAsyncEvent$ after its variables have been introduced}
  \label{efs-introduce-variables-mid-example-figure}
\end{figure}

First, we make the first $n$ local variables into parameters using
Rule~[\nameref{var-parameter-conversion-rule}], shown in
Figure~\ref{var-parameter-conversion-rule-figure}.
This matches the \Circus{} variable blocks representing local
variables and stack slots, along with the assignments initialising the
first $n$ local variables.
The rule moves these assignments and, using the definition of value
parameter, converts them into instantiations of value parameters.
This is applied to the method's action on
line~\ref{algorithm-var-parameter-conversion}.

\begin{figure}[thp]
  \begin{restatable}[$var$-parameter-conversion]{crule}{VarParameterConversionRule}
    \label{var-parameter-conversion-rule}
    \begin{circus}
      \begin{array}{l}
        (\circvar var1, \ldots, var{<}\ell{>} : Word \circspot \\
        \circvar stack1, \ldots, stack{<}s{>} : Word \circspot \\
        \t1 var1 := arg1 \circseq \\
        \t1 \cdots \\
        \t1 var{<}n{>} := arg{<}n{>} \circseq \\
        \t1 A)
      \end{array}
      \circrefines_A
      \begin{array}{l}
        (\circval var1, \ldots var{<}n{>} : Word \circspot \\
        \circvar var{<}n+1{>}, \ldots, var{<}\ell{>} : Word \circspot \\
        \circvar stack1, \ldots, stack{<}s{>} : Word \circspot \\
        \t1 A)(arg1, \ldots, arg{<}n{>})
      \end{array}
    \end{circus}
  \end{restatable}  
  \caption{Rule~[\nameref{var-parameter-conversion-rule}]}
  \label{var-parameter-conversion-rule-figure}
\end{figure}

After local variables have been converted into arguments, we redefine
the method action to exclude the parametrised block for the
$arg1, \ldots, arg{<}n{>}$ parameters, so that the, now redundant,
parameters can be eliminated.
This is performed, as with previous redefinitions of method actions,
in a separate procedure,
\Call{RedefineMethodActionToExcludeParameters}{}, defined by
Algorithm~\ref{redefine-method-action-to-exclude-parameters-algorithm}
in Appendix~\ref{introduce-variables-appendix-subsection}.
This procedure is called on
line~\ref{algorithm-redefine-method-action-to-exclude-parameters} of
Algorithm~\ref{introduce-variables-algorithm}.

After this, the argument parameters $arg1, \ldots, arg{<}n{>}$ are
outside the method action and can be eliminated by application of
Rule~[\nameref{argument-variable-elimination-rule}], shown in
Figure~\ref{argument-variable-elimination-rule-figure}.
This rule takes the name of the method action as a parameter and
eliminates the argument parameters around the call to the method
action, passing their values directly to the method action.
It is applied on line~\ref{algorithm-argument-variable-elimination}

\begin{figure}[thp]
  \begin{restatable}[argument-variable-elimination]{crule}{ArgumentVariableEliminationRule}
    \label{argument-variable-elimination-rule}
    Given an action name $M$,
    \begin{circus}
      \begin{array}{l}
        (\circval arg1, \ldots, arg{<}n{>} : Word \circspot \\
        \t1 M(arg1, \ldots, arg{<}n{>}))(arg_1, \ldots, arg_n)
      \end{array}
      \circrefines_A
      \begin{array}{l}
        M(arg_1, \ldots, arg_n)
      \end{array}
    \end{circus}
  \end{restatable}
  \caption{Rule~[\nameref{argument-variable-elimination-rule}]}
  \label{argument-variable-elimination-rule-figure}
\end{figure}

As in the previous section, we only handle methods that do not return
a value.
Handling methods that do return a value requires additional
compilation rules, similar to
Rule~[\nameref{var-parameter-conversion-rule}] and
Rule~[\nameref{argument-variable-elimination-rule}], to account for
the additional result parameter present in such methods.
The result parameter is not replaced with a local variable, since it
is only used for returning the value and, as such, does not map onto a
specific local variable.
Instead, it is simply moved to be grouped with the local variable
parameters.
The rules on lines~\ref{algorithm-poppedArgs-elimination}
to~\ref{algorithm-multi-poppedArgs-elimination} also require separate
versions to handle the storing of the returned value in the calling
method.

\begin{figure}[tp!]
  \centering
  \setlength{\zedtab}{0.5cm}
  \setlength{\zedindent}{0pt}
  \setlength{\zedleftsep}{0pt}
  \setlength{\abovedisplayskip}{0pt}
  \setlength{\belowdisplayskip}{0pt}
  \setlength{\abovedisplayshortskip}{0pt}
  \setlength{\belowdisplayshortskip}{0pt}
  \begin{circusaction}
    TPK\_handleAsyncEvent \circdef \circval var1 : Word \circspot \\
    \t1 \circvar var2, var3, var4, var5, var6 : Word \circspot \circvar stack1, stack2, stack3 : Word \circspot Poll \circseq \\
    \t1 newObject!thread!ConsoleConnectionClassID \\
    \t2 {} \then  newObjectRet!oid \then stack1 := oid \circseq Poll \circseq \\
    \t1 stack2 := stack1 \circseq Poll \circseq \\
    \t1 stack3 := null \circseq Poll \circseq \\
    \t1 ConsoleConnection\_CCinit(stack2, stack3) \circseq Poll \circseq \\
    \t1 var1 := stack1 \circseq Poll \circseq \\
    \t1 {} \cdots {} \\
    % \t1 stack1 := var1 \circseq Poll \circseq \\
    % \t1 getClassIDOf!stack1?cid \\
    % \t2 {} \then ConsoleConnection\_openInputStream(stack1, stack1) \circseq Poll \circseq \\
    % \t1 var2 := stack1 \circseq Poll \circseq \\
    % \t1 stack1 := var1 \circseq Poll \circseq \\
    % \t1 getClassIDOf!stack1?cid \\
    % \t2 {} \then ConsoleConnection\_openOutputStream(stack1, stack1) \circseq Poll \circseq \\
    % \t1 var3 := stack1 \circseq Poll \circseq \\
    \t1 stack1 := 0 \circseq Poll \circseq \\
    \t1 var4 := stack1 \circseq Poll \circseq Poll \circseq \\
    \t1 stack1 := var4 \circseq Poll \circseq \\
    \t1 stack2 := 10 \circseq Poll \circseq \\
    \t1 \circmu Y \circspot \\
    \t2 \circif stack1 \leq stack2 \circthen Poll \circseq \\
    \t3 stack1 := var2 \circseq Poll \circseq \\
    \t3 getClassIDOf!stack1?cid \then {} \\
    \t4 \circif cid = ConsoleInputClassID \circthen ConsoleInput\_read(stack1, stack1) \circseq \\
    \t4 \circfi \circseq Poll \circseq \\
    \t3 TPK\_f(stack1, stack1) \circseq Poll \circseq \\
    \t3 var5 := stack1 \circseq Poll \circseq \\
    \t3 stack1 := var5 \circseq Poll \circseq \\
    \t3 stack2 := 400 \circseq Poll \circseq \\
    \t3 \circif stack1 \leq stack2 \circthen Poll \circseq \\
    \t4 stack1 := var3 \circseq Poll \circseq \\
    \t4 stack2 := var5 \circseq Poll \circseq \\
    \t4 getClassIDOf!stack1?cid \then {} \\
    \t5 \circif cid = ConsoleOutputClassID \circthen ConsoleOutput\_write(stack1, stack2) \\
    \t5 \circfi
    \t3 {} \circelse stack1 > stack2 \circthen Poll \circseq \\
    \t4 {} \cdots {} \\
    % \t4 stack1 := var3 \circseq Poll \circseq \\
    % \t4 stack2 := 0 \circseq Poll \circseq \\
    % \t4 getClassIDOf!stack1?cid \then {} \\
    % \t5 \circif cid = ConsoleOutputClassID \circthen ConsoleOutput\_write(stack1, stack2) \\
    % \t5 \circfi
    \t3 \circfi \circseq Poll \circseq \\
    \t3 stack1 := var4 \circseq Poll \circseq \\
    \t3 stack2 := 1 \circseq Poll \circseq \\
    \t3 stack1 := stack1 + stack2 \circseq Poll \circseq \\
    \t3 var4 := stack1 \circseq Poll \\
    \t3 stack1 := var4 \circseq Poll \circseq \\
    \t3 stack2 := 10 \circseq Poll \circseq Y \\
    \t2 {} \circelse stack1 > stack2 \circthen \Skip \\
    \t2 \circfi \circseq Poll
  \end{circusaction}
  \caption{$TPK\_handleAsyncEvent$ at the end of the Introduce
    Variables step}
  \label{efs-introduce-variables-example-figure}
\end{figure}

\begin{figure}[tp!]
  \centering
  \setlength{\zedtab}{0.5cm}
  \setlength{\zedindent}{0pt}
  \setlength{\zedleftsep}{0pt}
  \setlength{\abovedisplayskip}{0pt}
  \setlength{\belowdisplayskip}{0pt}
  \setlength{\abovedisplayshortskip}{0pt}
  \setlength{\belowdisplayshortskip}{0pt}
  \begin{lstlisting}[basicstyle=\ttfamily,keywordstyle=\bf,language=C,
    numbers=left,numberstyle=\scriptsize,stepnumber=1, numbersep=5pt,
    escapeinside={(*@}{@*)}]
void TPK_handleAsyncEvent(int32_t var1) {
    int32_t var2, var3, var4, var5, var6;
    int32_t stack1, stack2, stack3;

    stack1 = newObject(ConsoleConnectionClassID);
    stack2 = stack1;
    stack3 = 0;
    ConsoleConnection_CCinit(stack2, stack3);
    var1 = stack1;
    
    (*@
    % stack1 = var1;
    % stack1 = ConsoleConnection_openInputStream(stack1);
    % var2 = stack1;
    
    % stack1 = var1;
    % stack1 = ConsoleConnection_openOutputStream(stack1);
    % var3 = stack1;
    @*)...

    stack1 = 0; (*@ \label{C-code-unnecessary-assignment} @*)
    var4 = stack1;
    stack1 = var4; (*@ \label{C-code-unnecessary-assignment-end} @*)
    stack2 = 10;
    while (stack1 <= stack2) {
        stack1 = var2;
        if (stack1->classID = ConsoleInputClassID) {
            stack1 = ConsoleInput_read(stack1);
        }
        stack1 = TPK_f(stack1);
        var5 = stack1;
        
        stack1 = var5;
        stack2 = 400;
        if (stack1 <= stack2) {  
            stack1 = var3;
            stack2 = var5;
            if (stack1->classID = ConsoleOutputClassID) {
                ConsoleOutput_write(stack1, stack2);
            }
        } else {
            (*@
            % stack1 = var3;
            % stack2 = 0;
            % if (stack1->classID = ConsoleOutputClassID) {
            %     ConsoleOutput_write(stack1, stack2);
            % }
            @*)...
        }
          
        stack1 = var4;
        stack2 = 1;
        stack1 = stack1 + stack2;
        var4 = stack1;

        stack1 = var4;
        stack2 = 10;
    }
}
\end{lstlisting}
  \caption{The C code corresponding to $TPK\_handleAsyncEvent$}
  \label{efs-introduce-variables-c-code-figure}
\end{figure}

When the variables have been introduced, the model that we obtain has
a form that corresponds directly to the C code for each method.
This can be seen from
Figures~\ref{efs-introduce-variables-example-figure}
and~\ref{efs-introduce-variables-c-code-figure}, which show the
\Circus{} code for $TPK\_handleAsyncEvent$ and its corresponding C
code.
Note that the $getClassIDOf$ communications with the object manager
correspond to accesses to C structs representing objects.
These structs are introduced in the final stage the strategy, in
Section~\ref{data-refinement-of-objects-section}.
