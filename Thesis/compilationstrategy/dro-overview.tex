The final stage of the compilation strategy introduces the
representation of objects in C.
Unlike the previous stages of the strategy, this stage operates on the
object manager, $ObjMan$, refining it to the $StructMan_{cs}$ process
described in Section~\ref{cee-struct-manager-subsection}.
Thus, this stage may be summarised by the following theorem.
\begin{thm}[Data Refinement of Objects]\label{dro-thm}
  \begin{circus}
    ObjMan(cs) \circrefines StructMan_{cs}
  \end{circus}
\end{thm}

\begin{algorithm}[b]
  \begin{algorithmic}[1]
    \State \ApplyFor{Law~[\nameref{forwards-data-refinement-law}]}{$StructManState$, $ObjectCI$}
    \label{algorithm-dro-data-refinement}
    \State \ApplyTo{Rule~[\nameref{refine-NewObject-rule}]}{$NewObject$}
    \label{algorithm-dro-refine-NewObject}
    \State \ApplyTo{Rule~[\nameref{refine-GetField-rule}]}{$GetField$}
    \State \ApplyTo{Rule~[\nameref{refine-PutField-rule}]}{$PutField$}
    \State \ApplyTo{Rule~[\nameref{refine-GetStatic-rule}]}{$GetStatic$}
    \State \ApplyTo{Rule~[\nameref{refine-PutStatic-rule}]}{$PutStatic$}
    \label{algorithm-dro-refine-PutStatic}
    \State \ApplyFor{Law~[\nameref{process-param-elim-law}]}{$cs$}
    \label{algorithm-dro-cs-elimination}
  \end{algorithmic}
  \caption{Data Refinement of Objects}
  \label{dro-algorithm}
\end{algorithm}

The process of refining $ObjMan$ is described by
Algorithm~\ref{dro-algorithm}. 
It begins on line~\ref{algorithm-dro-data-refinement} with a data
refinement to change the representation of objects used in the
interpreter to the C structs used in the final code.
% This data refinement uses new types and functions to represent struct
% types for each class in $cs$, having the form described in
% Section~\ref{cee-struct-manager-subsection}.
The fields in all superclasses of a class are collected together to
form the fields used to define its struct.
Note that an interface cannot have non-static fields and so, since we
require that the inputs to the compilation strategy pass the standard
checks performed on Java class files, a class's objects only inherit
fields from its true superclasses, even though our superclass relation
includes implemented interfaces.

As an example, we present the struct for objects of the \texttt{TPK}
class, $TPKObj$, below.
Since \texttt{TPK} declares no fields of its own (as indicated by the
empty $fields$ set for $TPK$ in Figure~\ref{example-model-figure} from
Section~\ref{overview-subsection}), it only includes fields from the
superclasses of \texttt{TPK} (shown in
Figure~\ref{example-superclasses-model-figure}).
Its fields are thus those of \texttt{ManagedEventHandler}, since
\texttt{AperiodicEventHandler} does not add any information to the
base information for an event handler.
These fields are in addition to the $classID$ field, which is
contained in every object's struct and identifies the class of the
object.
The other fields are the $threadID$, $backingStoreSize$,
$allocAreaSize$ and $stackSize$ fields from the $ManagedEventHandler$
class information structure.
These provide space in which the information about the event handler
may be stored by the $Launcher$ during mission startup.
\begin{schema}{TPKObj}
  classID : ClassID \\
  threadID : Word \\
  backingStoreSize : Word \\
  allocAreaSize : Word \\
  stackSize : Word
\end{schema}

Similar types are created for $ManagedEventHandler$ and
$AperiodicEventHandler$, with the same fields, and named
$ManagedEventHandlerObj$ and $AperiodicEventHandlerObj$ repectively.
This means that $TPKObj$ values can be converted to those types, since
they contain fields of the same name and type.
Other aperiodic event handlers may have additional fields to store
data specific to them.
Their object types can be converted to $AperiodicEventHandlerObj$ by
simply discarding the additional fields.
The struct types for each class are collected together into an
$ObjectStruct$ type, shown below.
\begin{zed}
  ObjectStruct ::= \\
  \t1 TPKCon \ldata TPKObj \rdata | \\
  \t1 AperiodicEventHandlerCon \ldata AperiodicEventHandlerObj \rdata | \\
  \t1 ManagedEventHandlerCon \ldata ManagedEventHandlerObj \rdata | \cdots {} 
\end{zed}

The values of $ObjectStruct$ that can be converted to
$ManagedEventHandlerObj$ can be cast to it by the function
$castManagedEventHandler$.
We also define functions for performing a combined cast and field
update and collect the static fields from all classes into a
$StaticFields$ structure, as described in
Section~\ref{cee-struct-manager-subsection}.

% TODO: does this need to be said?
These types and functions are all used in the struct manager, so we
introduce them before applying the data refinement.
Note that, since the types are Z data types, they do not need to be
declared in a process and so we do not need to introduce them by
refinement of $ObjMan$.

The state resulting from the data refinement on
line~\ref{algorithm-dro-data-refinement} is $StructManState$, shown
below, which uses the object struct types described above.
\begin{schema}{StructManState}
  BackingStoreManager \\
  objects : ObjectID \pfun ObjectStruct \\
  staticClassFields : StaticFields
\where
  backingStoreMap \partition \dom objects
\end{schema}
As mentioned in Section~\ref{cee-struct-manager-subsection},
$StructManState$ is very similar to $ObjManState$.
It contains $BackingStoreManager$, which is the same as in
$ObjManState$ since the management of backing stores is unaffected by
the compilation strategy.
The $objects$ map is similar to that in $ObjManState$, but it maps to
the $ObjectStruct$ type, rather than the $Object$ type.
The component $staticClassFields$, in $StructManState$, is of the
$StaticFields$ type, rather than the map from fields to their values
in $ObjManState$, since we have a known set of static fields, having
been supplied with the $cs$ map.

The data refinement is described by the coupling invariant $ObjectCI$,
shown in Figure~\ref{ObjectCI-figure}, which relates $ObjManState$ to
$StructManState$.
It is presented as a template to be instantiated by the identifiers of
the classes in $cs$ and their corresponding fields, in much the same
format as for the schemas in
Section~\ref{cee-struct-manager-subsection}.
This equates the $BackingStoreManager$ fields in $ObjManState$ with
those in $StructManState$, since management of backing stores is
unaffected by the data refinement.

\begin{figure}[tp!]
\begin{schema}{ObjectCI}
  ObjManState \\
  StructManState_1
\where
  backingStoreMap = backingStoreMap_1 \\
  backingStoreStacks = backingStoreStacks_1 \\
  rootBS = rootBS_1 \\
  \dom~objects = \dom objects_1 \\
  \forall x : \dom objects @ \\
  % \t1 thisClassID~((objects~x).class) \in \{{<}classID_1{>}, \ldots, {<}classID_n{>}\} \land \\
  \t1 (thisClassID~((objects~x).class) = {<}classID_1{>} \implies \\
  % \t2 (\dom (objects~x).fields = \{{<}fieldID_{1,1}{>}, \ldots, {<}fieldID_{1,m_1}{>}\} \land \\
  \t2 objects_1~x = {<}classID_1{>}Con~\lblot \\
  \t3 classID == {<}classID_1{>}, \\
  \t3 {<}fieldID_{1,1}{>} == (objects~x).fields~{<}fieldID_{1,1}{>}, \\
  \t4 \cdots \\
  \t3 {<}fieldID_{1,m_1}{>} == (objects~x).fields~{<}fieldID_{1,m_1}{>} \\
  \t2 \rblot) \land \\
  \t2 \cdots \\
  \t1 (thisClassID~((objects~x).class) = {<}classID_n{>} \implies \\
  % \t2 (\dom (objects~x).fields = \{{<}fieldID_{n,1}{>}, \ldots, {<}fieldID_{1,m_n}{>}\} \land \\
  \t2 objects_1~x = {<}classID_n{>}Con~\lblot \\
  \t3 classID == {<}classID_n{>}, \\
  \t3 {<}fieldID_{n,1}{>} == (objects~x).fields~{<}fieldID_{n,1}{>}, \\
  \t4 \cdots \\
  \t3 {<}fieldID_{n,m_n}{>} == (objects~x).fields~{<}fieldID_{n,m_n}{>} \\
  \t2 \rblot) \\
  staticClassFields_1.{<}classID_1{>}\_{<}staticfieldID_{1,1}{>} = \\
  \t1 staticClassFields~({<}classID_1{>}, {<}staticfieldID_{1,1}{>}) \\
  \t1 \cdots \\
  staticClassFields_1.{<}classID_1{>}\_{<}staticfieldID_{1,\ell_1}{>} = \\
  \t1 staticClassFields~({<}classID_1{>}, {<}staticfieldID_{1,\ell_1}{>}) \\
  \t1 \cdots \\
  staticClassFields_1.{<}classID_n{>}\_{<}staticfieldID_{n,1}{>} = \\
  \t1 staticClassFields~({<}classID_n{>}, {<}staticfieldID_{n,1}{>}) \\
  \t1 \cdots \\
  staticClassFields_1.{<}classID_n{>}\_{<}staticfieldID_{n,\ell_n}{>} = \\
  \t1 staticClassFields~({<}classID_n{>}, {<}staticfieldID_{n,\ell_n}{>}) \\
\end{schema}
\caption{The $ObjectCI$ schema, which is the coupling invariant between $ObjManState$ and $StructManState$}
\label{ObjectCI-figure}
\end{figure}

The functions $objects$ for both the abstract and concrete states have
the same domain, since the set of objects does not change, merely
their representation.
The representation of each object in $objects$ after the data
refinement is determined by the class identifier in its $class$
information before the refinement.
The object is of the struct type for the class identifier.
For example, $TPKClassID$ is the class identifier corresponding to the
class information of \texttt{TPK}, since that is the identifier in the
$constantPool$ of the $TPK$ structure
(Figure~\ref{example-model-figure}) that corresponds to its $this$
value.
Thus, any object of that type will have the correponding class
information stored and so is refined to an $ObjectStruct$ value using
the $TPKCon$ constructor and containing a value of the $TPKObj$ type
shown above.

The fields of the structure are taken from the values corresponding to
each field identifier in the object's information before the data
refinement.
These fields are guaranteed to correspond to the fields listed in the
object's class information (including the fields from its
superclasses) by the invariant of $ObjManState$.
Similarly, the fields in $staticClassFields$ map directly onto fields
in the $StaticFields$ structure, with the set of fields guaranteed to
be the same by the invariant of $StaticFieldsInfo$.

$ObjectCI$ is based on equating the information in the old model with
the fields of the object structs in the new model, and so it describes
a functional data refinement from which we can calculate the form of
the actions in the resultant model.
However, the direct application of this refinement yields actions in a
form that does not directly correspond to the representation of the
semantics of C structs that we desire.
We thus apply some additional compilation rules on
lines~\ref{algorithm-dro-refine-NewObject}
to~\ref{algorithm-dro-refine-PutStatic}, to refine the actions of the
process to the correct form.

An example of such a rule is Rule~[\nameref{refine-NewObject-rule}],
shown in Figure~\ref{refine-NewObject-rule-figure}.
It operates on the body of the $NewObject$ action, which has the form
on the left-hand side of the rule after the application of the data
refinement.
In this rule, we represent by the schema $StructManObjectInit$ the
result of applying the data refinement to the $ObjManObjectInit$
schema.
The rule splits this data operation into separate operations defined
by simpler schemas, which can be found in Appendix~B of the extended
version of this thesis~\cite{baxter2018-extended}, initialising each
of the object struct types, offering a choice over each class
identifier in $cs$ to determine which one should be used.
The $AllocateObject$ action that communicates with the memory manager
to allocate space for the object is also supplied with constants
indicating the size of each object struct type, rather than
determining it from the class information in $cs$.
This means the $GetObjectClassInfo$ schema, which determines the class
information, can also be removed, eliminating reliance on $cs$ in the
action.

\begin{figure}[t!]
  \begin{restatable}[refine-$NewObject$]{crule}{RefineNewObjectRule}
    \label{refine-NewObject-rule}
    \setlength{\zedtab}{0.6cm}
    \setlength{\zedindent}{0.6cm}
    \begin{circus}
      \begin{array}{l}
        \circvar thread : ThreadID; classID : ClassID \circspot \\
        \circvar objectID : ObjectID; class : Class \circspot \\
        newObject?t?c \then thread, classID := t, c \circseq \\
        \lschexpract GetObjectClassInfo \rschexpract \circseq \\
        AllocateObject(\\
        \t1 thread, sizeOfObject~class, objectID) \circseq \\
        \lschexpract StructManObjectInit \rschexpract \circseq \\
        newObjectRet!objectID \then \Skip
      \end{array}
      \circrefines_A
      \begin{array}{l}
        \circvar objectID : ObjectID \circspot \\
        newObject?thread?classID \then {} \\
        \circif classID = {<}classID_1{>} \circthen {} \\
        \t1 AllocateObject(\\
        \t2 thread, \\
        \t2 sizeof{<}classID_1{>}Obj, \\
        \t2 objectID) \circseq \\
        \t1 \lschexpract StructMan{<}classID_1{>}ObjInit \rschexpract \circseq \\
        \t1 {} \vdots {} \\
        {} \circelse classID = {<}classID_n{>} \circthen {} \\
        \t1 AllocateObject(\\
        \t2 thread, \\
        \t2 sizeof{<}classID_n{>}Obj, \\
        \t2 objectID) \circseq \\
        \t1 \lschexpract StructMan{<}classID_n{>}ObjInit \rschexpract \circseq \\
        \circfi \circseq newObjectRet!objectID \then \Skip
      \end{array}
    \end{circus}
    where, for all $k \in 1 \upto n$,
    \begin{circus} 
        \exists \Delta Class | \Xi Class \hide (fields, fields') @ \\
        \t1 \theta Class = cs~{<}classID_k{>} \land \\
        \t1 fields' = \bigcup \{ cid : \dom cs | ({<}classID_k{>},cid) \in subclassRel~cs @ (cs~cid).fields \} \land  \\
        %\t1 \exists class? == \theta Class~' @ ObjManObjectInit \\
        \t1 sizeof{<}classID_k{>}Obj = sizeOfObject~(\theta Class~') \\
    \end{circus}
  \end{restatable}
  \caption{Rule~[\nameref{refine-NewObject-rule}]}
  \label{refine-NewObject-rule-figure}
\end{figure}

Rule~[\nameref{refine-GetField-rule}],
Rule~[\nameref{refine-PutField-rule}],
Rule~[\nameref{refine-GetStatic-rule}] and
Rule~[\nameref{refine-PutStatic-rule}], which refine the $GetField$,
$PutField$, $GetStatic$ and $PutStatic$ actions respectively, are
similar to Rule~[\nameref{refine-NewObject-rule}], refining the data
operations that result from the data refinement to choices over class
and field identifiers received on the channels for the operations.
These rules can be found in Appendix~\ref{compilation-rules-appendix}.
Note that, while the $Init$ action of $ObjMan$ is refined in this
stage, it does not require a separate rule, since the necessary
refinement is performed by the data refinement alone.

After the refinement has been performed, we can eliminate the $cs$
parameter of the process via an application of
Law~[\nameref{process-param-elim-law}] on
line~\ref{algorithm-dro-cs-elimination}.
This completes the transformation of $ObjMan(cs)$ into
$StructMan_{cs}$.
After this, the model corresponds completely to the C code.
