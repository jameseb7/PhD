\begin{algorithm}
  \begin{algorithmic}[1]
    \State $cfg \gets$ \Call{MakeControlFlowGraph}{}
    \label{algorithm-make-control-flow-graph}
    \For{$node \gets$ \Call{Nodes}{$cfg$}}
    \label{algorithm-sequence-cfg-loop}
    \While{\Call{HasSimpleSequence}{$node$}}
    \label{algorithm-forward-sequence-condition}
    \State \ApplyFor{Rule~[\nameref{sequence-introduction-rule}]}{$node$}
    \label{algorithm-forward-sequence-application}
    \EndWhile
    \EndFor
  \end{algorithmic}
  \caption{\Call{IntroduceSequentialComposition}{}}
  \label{introduce-forward-sequence-algorithm}
\end{algorithm}
The simplest control flows to introduce are those of instructions
where execution continues at the next program counter value.
These control flows are introduced as shown in
Algorithm~\ref{introduce-forward-sequence-algorithm}, which defines
the \Call{IntroduceSequentialComposition}{} procedure.
The algorithm constructs a control flow graph for each method in the
program, as specified on line~\ref{algorithm-make-control-flow-graph}.
Since the introduction of sequential composition does not depend on
the relationships between methods, the control flow graph is
constructed as a disconnected graph containing the control flow of
each method in the program.
The nodes in this graph correspond to the branches in the choice over
$pc$ values introduced in the previous section.

We construct the control flow graph by starting at the entry point for
each method and following the $pc$ update at the end of each node,
introducing an edge in the process.
For method call instructions, we introduce an edge to the node for the
next $pc$ value, as if the instruction were replaced with
$pc := pc + 1$.
This is consistent with how method calls are handled later in the
strategy, since execution resumes at the next instruction after the
called method returns.
We do not add an edge from a return instruction, since no further
instructions are executed in the method after a return instruction.
Construction of the control graph for a method terminates when there
are no further edges to add.
Since there are only finitely many instructions in a method, edges for
all the reachable nodes are eventually added.

The control flow graph for our example is shown in
Figure~\ref{example-control-flow-graph-figure}.
We label the nodes of the graph with the $pc$ values of the
instructions at the nodes.
Due to our assumptions about the source bytecode, the subgraph
corresponding to each method's control flow is a structured graph as
defined in Section~\ref{compilation-assumptions-section}.

\begin{figure}
  \begin{center}
    \footnotesize
    \begin{tikzpicture}[every new ->/.style={-latex}]
      % \node (start) at (0,0) {};
      \foreach \x in {0,...,6}  { \node (\x) at (\x, 1cm) {\x}; }
      \foreach \x in {7,...,20}  { \node (\x) at (0.85*\x - 0.85*7,0) {\x}; }
      \foreach \x in {21,...,27} { \node (\x) at (0.85*\x - 0.85*21, -2cm) {\x}; }
      \foreach \x in {28,...,31} { \node (\x) at (0.85*\x - 0.85*21, -1cm) {\x}; }
      \foreach \x in {32,...,34} { \node (\x) at (0.85*\x - 0.85*25, -3cm) {\x}; }
      \foreach \x in {35,...,42} { \node (\x) at (0.85*\x - 0.85*24.3, -2cm) {\x}; }
      \foreach \x in {43,...,50} { \node (\x) at (-43+\x, -4.5cm) {\x}; }

      \graph{ (7) -> (8) -> (9) -> (10) -> (11) -> (12) ->
        (13) -> (14) -> (15) -> (16) -> (17) -> (18) -> (19) -> (20)
        -> (39) -> (40) -> (41) -> (42); (21) -> (22) -> (23) -> (24)
        -> (25) -> (26) -> (27) -> {
          (28) -> (29) -> (30) -> (31);
          (32) -> (33) -> (34);
        } -> (35) -> (36) -> (37) -> (38) -> (39);
      };

      \graph{(0) -> (1) -> (2) -> (3) -> (4) -> (5) -> (6)};
      \graph{(43) -> (44) -> (45) -> (46) -> (47) -> (48) -> (49) -> (50)};

      \draw[-latex] (41) edge[out=270,in=270,looseness=0.35] (21);
    \end{tikzpicture}
  \end{center}
  \caption{Control flow graph for our example program}
  \label{example-control-flow-graph-figure}
\end{figure}

After the control flow graph is constructed, we consider each node in
turn, as specified by the for loop on
line~\ref{algorithm-sequence-cfg-loop}.
As mentioned earlier, we require a node to have only a single outgoing
edge and its target to have only a single incoming edge in order for
it to be considered for the introduction of sequential composition.
The reason for this is that nodes with two outgoing edges are points
at which conditionals should be introduced.
Such nodes in our example are the nodes for $pc$ values $27$ and $41$,
which represent the start of conditionals.
Likewise, nodes with multiple incoming edges represent points at which
a more complex control flows occur.
For our example, such nodes include $39$, which is the start of a
loop, and $35$, which is the end oxf a conditional.
These prevent introduction of sequential composition for the $pc$
values $20$, $31$, $34$, and $38$, since the targets of those nodes
are nodes $35$ and $39$.

The procedure \Call{HasSimpleSequence}{$node$} checks this requirement
for introducing sequential composition.
It returns a true value if $node$ has only a single outgoing edge and
its target has only a single incoming edge, and otherwise returns a
false value.
This check is performed on
line~\ref{algorithm-forward-sequence-condition}, where it defines the
condition of a while loop.

For a node that meets the above requirement and is not a method call,
we can introduce sequential composition at that node by applying
Rule~[\nameref{sequence-introduction-rule}]
(Figure~\ref{sequence-introduction-rule-figure}), on
line~\ref{algorithm-forward-sequence-application} of the algorithm.
This rule works by unrolling the loop in $Running$ to sequence an
instruction at $pc$ value $i$ with the instruction that is executed
after it, inserting $Poll$ inbetween.
It is required that the $pc$ value of the node's target, $j$, not be
the same as $i$, since that would introduce a loop, rather than a
sequential composition.
Also, the sequence of instructions at the node, $A$, must not affect
the non-emptiness of the $frameStack$ to ensure that the choice at the
start of the main loop in $Running$ can be resolved.
\begin{figure}[thp]
\begin{restatable}[sequence-intro]{crule}{SequenceIntroductionRule}
  \label{sequence-introduction-rule}
  Given $i : ProgramAddress$, if $i \neq j$ and \def\zedindent{0.25cm}
  \begin{circus}
    \{frameStack \neq \emptyset\} \circseq A \\
    {} = {} \\
    \{frameStack \neq \emptyset\} \circseq A \circseq \{frameStack \neq \emptyset\}
  \end{circus}
  then,
  \begin{circus}
    \begin{array}{l}
      \circmu X \circspot \\
      \t1 \circif frameStack = \emptyset \circthen \Skip \\
      \t1 {} \circelse frameStack \neq \emptyset \circthen {} \\
      \t2 \circif {} \cdots {} \\
      \t2 {} \circelse pc = i \circthen A \circseq pc := j \\
      \t2 {} \cdots {} \\
      \t2 {} \circelse pc = j \circthen B \\
      \t2 {} \cdots {} \\
      \t2 \circfi \circseq Poll \circseq X \\
      \t1 \circfi
    \end{array}
    \circrefines_A
    \begin{array}{l}
      \circmu X \circspot \\
      \t1 \circif frameStack = \emptyset \circthen \Skip \\
      \t1 {} \circelse frameStack \neq \emptyset \circthen {} \\
      \t2 \circif {} \cdots {} \\
      \t2 {} \circelse pc = i \circthen {} \\
      \t3 A \circseq pc := j \circseq Poll \circseq B \\
      \t2 {} \cdots {} \\
      \t2 {} \circelse pc = j \circthen B \\
      \t2 {} \cdots {} \\
      \t2 \circfi \circseq Poll \circseq X \\
      \t1 \circfi
    \end{array}
  \end{circus}
\end{restatable}
\caption{Rule~[\nameref{sequence-introduction-rule}]}
\label{sequence-introduction-rule-figure}
\end{figure}

Since Rule~[\nameref{sequence-introduction-rule}] pulls two nodes
together, we can continue to introduce sequential composition at a
node after the first application of
Rule~[\nameref{sequence-introduction-rule}], until that node no longer
satisfies the conditions for introducing sequential composition.
This is specified by the while loop at
line~\ref{algorithm-forward-sequence-condition} of the algorithm.
The control flow graph is updated as
Rule~[\nameref{sequence-introduction-rule}] is applied, to take into
account the merging of nodes.
Since there are finitely many nodes, the merging of nodes eventually
results in a graph in which no further sequential compositions can be
introduced and so the loop terminates.

The resulting control flow graph after introduction of sequential
composition has been performed at every point is shown in
Figure~\ref{example-control-flow-graph-after-sequence-introduction-figure}.
We note that this graph is still a union of structured graphs since
merging sequentially composed nodes does not affect whether a graph is
structured.
This is due to the fact that sequential composition is one of the
constructs used to define structured control flow graphs
(Figure~\ref{sequence-figure}), and merging the nodes may be seen as
performing the reverse of node replacement for it.
\begin{figure}
  \begin{center}
    \begin{tikzpicture}[every new ->/.style={-latex}]
      \path (0,0) node (0) {0} -- ++(1,0) node (6) {6};
      \path (0,-1cm) node (7) {7} -- ++(1,0) node (11) {11} -- ++(1,0) node (14) {14} -- ++(1,0) node (17) {17};
      \path (0,-3cm) node (21) {21} -- ++(1,0) node (23) {23} -- ++(1,0) node (24) {24};
      \path (3,-2.3cm) node (28) {28} -- ++(1,0) node (31) {31};
      \node at (3,-3.7cm) (32) {32};
      \path (5,-3cm) node (35) {35} -- ++(1,0) node (39) {39} -- ++(1,0) node (42) {42};
      
      \graph{
        (7) -> (11) -> (14) -> (17) -> (39) -> (42);
        (21) -> (23) -> (24) -> {
          (28) -> (31);
          (32);
        } -> (35) -> (39);
      };

      \graph[grow down]{(0) -> (6)};
      \node at (0,-5cm) {43};

      \draw[-latex] (39) edge[out=270,in=270,looseness=0.6] (21);
    \end{tikzpicture}
  \end{center}
  \caption{Control flow graph for our example after sequential composition introduction}
  \label{example-control-flow-graph-after-sequence-introduction-figure}
\end{figure}

The only remaining nodes in this graph are those where the sequence of
instructions ends with a method call or return, or which represent a
more complex control flow.
In particular, the instructions for the \texttt{f()} method of
\texttt{TPK}, which begin at $pc = 43$, have been completely sequenced
together into a single node.
The code that corresponds to this control flow graph is that shown
earlier in Figure~\ref{forward-sequence-introduction-example-figure}.