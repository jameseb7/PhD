After sequential composition has been introduced for all methods, we
must consider each method separately to handle method calls.
This means the strategy must loop, introducing loops and conditionals
to those methods that have no unresolved method calls and resolving
calls of methods that are then complete, until every method is
complete and has been separated into its own action.

Introducing loops and conditionals is performed as described by
Algorithm~\ref{introduce-loops-and-conditionals-algorithm}.
This considers each method individually, collecting the list of method
entries from the $cs$ information on
line~\ref{algorithm-loops-and-conditionals-get-method-entries}, and
iterating over them in the for loop on
line~\ref{algorithm-introduce-loops-and-conditionals-method-loop}.
The program address that forms the entry point of the method under
consideration in a given iteration of the loop is referred to as
$methodEntry$.
The condition on line~\ref{algorithm-no-unresolved-calls-condition}
ensures that only those methods where all method calls have already
been resolved undergo loop and conditional introduction.
Since we do not allow recursion, there is always at least one method
that does not depend on another method in the program.
It may be the case that a method depends only on special methods, in
which case this stage has no effect on that method until the special
method calls have been resolved.
Special method calls can always be resolved as they do not depend on
other methods in the program.
% TODO: discussion of accuracy of static recursion detection in
% conclusions

The \Call{HasNoUresolvedCalls}{$methodEntry$} procedure, used in the
condition on line~\ref{algorithm-no-unresolved-calls-condition},
checks that no node in the control flow graph beginning at
$methodEntry$ ends in a method call, as a way of determining whether
the method has unresolved calls.
Since method resolution sequences a method call with the instructions
following it, a method call with nothing following it is a call that
has not yet been resolved.

%TODO: possibly use try here
\begin{algorithm}
  \begin{algorithmic}[1]
    \State $methodEntries \gets$ \Call{MethodEntries}{$cs$}
    \label{algorithm-loops-and-conditionals-get-method-entries}
    \For{$methodEntry \gets methodEntries$}
    \label{algorithm-introduce-loops-and-conditionals-method-loop}
    \If{\Call{HasNoUresolvedCalls}{$methodEntry$}}
    \label{algorithm-no-unresolved-calls-condition}
    \State $cfg \gets$ \Call{MakeMethodControlFlowGraph}{$methodEntry$}
    \label{algorithm-make-control-flow-graph2}
    \State $iterationOrder \gets$ \Call{ReverseNodes}{$cfg$}
    \label{algorithm-compute-iteration-order}
    \For{$node \gets iterationOrder$}
    \label{algorithm-node-checking-loop}
    \State \ApplyFor{Rule~[\nameref{if-introduction-rule}]}{$node$}
    \label{algorithm-introduce-if}
    \State \ApplyFor{Rule~[\nameref{if-else-introduction-rule}]}{$node$}
    \label{algorithm-introduce-if-else}
    \If{\Call{IsSimpleConditional}{$node$}}
    \label{algorithm-conditional-check}
    \State \ApplyFor{Rule~[\nameref{conditional-introduction-rule}]}{$node$}
    \label{algorithm-introduce-conditional}
    \EndIf
    \State \ApplyFor{Rule~[\nameref{while-introduction-rule1}]}{$node$}
    \label{algorithm-introduce-while1}
    \State \ApplyFor{Rule~[\nameref{while-introduction-rule2}]}{$node$}
    \label{algorithm-introduce-while2}
    \State \ApplyFor{Rule~[\nameref{do-while-introduction-rule}]}{$node$}
    \label{algorithm-introduce-do-while}
    \State \ApplyFor{Rule~[\nameref{infinite-loop-introduction-rule}]}{$node$}
    \label{algorithm-introduce-infinite-loop}
    \If{\Call{HasSimpleSequence}{$node$}}
    \label{algorithm-lci-sequence-check}
    \State \ApplyFor{Rule~[\nameref{sequence-introduction-rule}]}{$node$}
    \label{algorithm-lci-sequence-introduction}
    \EndIf
    \EndFor
    \EndIf
    \EndFor
  \end{algorithmic}
  \caption{\Call{IntroduceLoopsAndConditionals}{}}
  \label{introduce-loops-and-conditionals-algorithm}
\end{algorithm}

For each method that undergoes loop and conditional introduction, we
consider again its control-flow graph to ensure the loops and
conditionals are introduced in the correct order to properly form
their bodies.
This involves constructing a control-flow graph for the method, at
line~\ref{algorithm-make-control-flow-graph2}.
The control-flow graph for a method beginning at $methodEntry$ is
created by the procedure
\Call{MakeMethodControlFlowGraph}{$methodEntry$}.
This is similar to the \Call{MakeControlFlowGraph}{} procedure used in
the previous section, but it just constructs the graph for a single
method, starting at\deleted{ the}\added{ its} $methodEntry$.

The graph for the \texttt{handleAsyncEvent()} method in our example
(beginning at $pc=7$, its entry point) is shown in
Figure~\ref{example-simplified-control-flow-graph-figure}, alongside
the \Circus{} code obtained at the beginning of this stage for the
method.
The edge that forms a loop from $pc=35$ to $pc=39$ is shown as a
dashed line since looping edges are ignored at certain points in this
part of the strategy.
\begin{figure}
  \begin{center}
    % \begin{multicols}{4}
    \begin{minipage}{0.3\linewidth}
      \begin{tikzpicture}
        \node (7)  at (0,0)  {7};
        \node (39)  at (0,-1) {39};
        \node (42) at (-1,-2) {42};
        \node (21) at (1,-2) {21};
        \node (28) at (0.5,-3) {28};
        \node (32) at (1.5,-3) {32};
        \node (35) at (1,-4) {35};
        \draw[-latex] (7) to (39);
        \draw[-latex] (39) to (42);
        \draw[-latex] (39) to (21);
        \draw[-latex] (21) to (28);
        \draw[-latex] (21) to (32);
        \draw[-latex] (28) to (35);
        \draw[-latex] (32) to (35);
        % \draw[-latex,red!70!black,dashed,out=0,in=0,looseness=1.1] (35) to (39);
        \draw[-latex,dashed,out=0,in=0,looseness=1.1] (35) to (39);
      \end{tikzpicture}
    \end{minipage}
    % \columnbreak
    \begin{minipage}{0.65\linewidth}
      \scriptsize
      \setlength{\zedindent}{0cm}
      \begin{circus}
        Running \circdef \\
        \t1 \circif frameStack = \emptyset \circthen \Skip \\
        \t1 {} \circelse frameStack \neq \emptyset \circthen {} \\
        \t2 \circif pc = 0 \circthen {} \cdots {} \\
        \t2 {} \circelse pc = 7 \circthen HandleNewEPC(27) \circseq pc := 8 \circseq Poll \circseq \cdots \circseq \\
        \t3 pc := 39 \\
        \t2 {} \cdots {} \\
        \t2 {} \circelse pc = 21 \circthen HandleAloadEPC(2) \circseq pc := 22 \circseq Poll \circseq \cdots \circseq \\
        \t3 pc := \IF value1 \leq value2 \THEN 32 \ELSE 28 \\
        \t2 {} \cdots {} \\
        \t2 {} \circelse pc = 28 \circthen HandleAloadEPC(3) \circseq pc := 29 \circseq Poll \circseq \cdots \circseq \\
        \t3 pc := 35 \\
        \t2 {} \cdots {} \\
        \t2 {} \circelse pc = 32 \circthen HandleAloadEPC(3) \circseq pc := 33 \circseq Poll \circseq \cdots \circseq \\
        \t3 pc := 35 \\
        \t2 {} \cdots {} \\
        \t2 {} \circelse pc = 35 \circthen HandleAloadEPC(4) \circseq pc := 36 \circseq Poll \circseq \cdots \circseq \\
        \t3 pc := 39 \\
        \t2 {} \cdots {} \\
        \t2 {} \circelse pc = 39 \circthen HandleAloadEPC(4) \circseq pc := 36 \circseq Poll \circseq \cdots \circseq \\
        \t3 pc := \IF value1 \leq value2 \THEN 21 \ELSE 42 \\
        \t2 {} \cdots {} \\
        \t2 {} \circelse pc = 42 \circthen HandleReturnEPC \\
        \t2 \circfi \circseq Poll \circseq Running \\
        \t1 \circfi
      \end{circus}
    \end{minipage}
    %\end{multicols}
  \end{center}
  \caption{Simplified control flow graph and corresponding code for our example
    program}
  \label{example-simplified-control-flow-graph-figure}
\end{figure}

%TODO: explain how our definition of structure differs from that of
% MISRA - leave for final considerations?

The control-flow graph of each method is structured since the
transformations of the graph up to this point consist solely of
collapsing sequential compositions, which, as explained in the
previous section, does not cause a structured graph to become
unstructured.
Since we have defined the desired program structure in terms of a
small number of standard structures (shown in
Figure~\ref{structured-cfg-figures}), we can identify each of these
structures in the graph and introduce them into the program,
collapsing the graph in the process.

We iterate over the nodes in the method\added{'}s control flow graph,
identifying the control flow structures at each node.
This is specified by the loop beginning on
line~\ref{algorithm-node-checking-loop} of
Algorithm~\ref{introduce-loops-and-conditionals-algorithm}.

In order to identify a structure, we must first introduce any
structures embedded in it.
This can be seen by considering a graph such as that shown in
Figure~\ref{internal-replacement-figure}, where the graph starting at
node $1$ does not have the form of an \texttt{if}-\texttt{else}
conditional (Figure~\ref{if-else-figure}), although it is constructed
from such a graph.
The subgraph starting at $a$, however, does have the form of a
\texttt{while} loop, so it can be introduced.
Once that has been introduced, the graph has the form of an
\texttt{if}-\texttt{else} conditional.
To introduce such embedded structures first, we consider the
successors of each node (ignoring loops) before the node itself.
This ensures that we consider the internal nodes of a structure first
and introduce any structures that may have been inserted at those
nodes via internal-node or branch-end replacement.

The iteration order is specified using a procedure
\Call{ReverseNodes}{$cfg$}, which constructs a sequence indicating the
order in which the nodes of $cfg$ should be iterated over.
The sequence is constructed ensuring that, where a node occurs in the
sequence, the successors of that node (ignoring loops) occur earlier.
This means that the sequence begins with an end node (ignoring loops).
The order, $iterationOrder$, is constructed on
line~\ref{algorithm-compute-iteration-order} of the algorithm and used
for the range of the for loop\added{ starting} on
line~\ref{algorithm-node-checking-loop}.

In our example, we may consider the $pc=42$ and $pc=35$ nodes first,
then $pc=28$ and $pc=32$, then $pc=21$, $pc=39$, and finally $pc=7$,
as can be seen from the graph in
Figure~\ref{example-simplified-control-flow-graph-figure}.
Other valid orders may be used in an implementation of the strategy.

For each node, we check each type of structure to see if the
control-flow graph starting at that point matches the structure, and
introduce the structure if it does.
Some of the structures (Figure~\ref{if-figure},
\subref{if-else-figure}, \subref{while-figure} and
\subref{do-while-figure}) are followed by further instructions.
In these cases, a sequential composition must be introduced with the
instructions following the structure.
\addchange{Changed explanation of why sequential composition
  introduction is separated out for loops and conditionals}%

However, in programs with graphs such as the one shown\added{ on the
  left} below, the sequential composition\deleted{ can only be
  introduced after the outer conditional has been introduced}\added{
  cannot be introduced after the inner conditionals have been
  introduced}.
\added{Introducing the inner conditionals yields the graph shown on
  the right below, which has the form of an \texttt{if-else}
  conditional.
  This graph would be broken up by the introduction of a sequential
  composition to the final node, since it is part of the outer
  conditional.}
Thus, the introduction of the sequential composition cannot be made
part of the rule for introducing the\added{ inner} conditional.
\added{We instead perform sequential compositions for such structures
  separately, rather than as part of the loop and conditional
  introduction rules.}
\begin{center}
  \begin{tikzpicture}
    \node (1) at (0.0, 0.0) {$\bullet$};
    \node (2) at (1.5, 0.5) {$\bullet$};
    \node (3) at (1.5,-0.5) {$\bullet$};
    \node (4) at (2.7, 1.0) {$\bullet$};
    \node (5) at (2.7,-1.0) {$\bullet$};
    \node (6) at (2.7, 0.2) {$\bullet$};
    \node (7) at (2.7,-0.2) {$\bullet$};
    \node (8) at (4.0, 0.0) {$\bullet$};
    
    \draw[-latex] (-1,0) to (1);
    
    \draw[-latex] (1) to (2);
    \draw[-latex] (1) to (3);
    \draw[-latex] (2) to (4);
    \draw[-latex] (3) to (5);
    \draw[-latex] (2) to (6);
    \draw[-latex] (3) to (7);
    \draw[-latex] (4) to (8);
    \draw[-latex] (5) to (8);
    \draw[-latex] (6) to (8);
    \draw[-latex] (7) to (8);

    \node at (5.5, 0.0) {$\implies$};

    \node (A) at (8.0, 0.0) {$\bullet$};
    \node (B) at (9.5, 0.5) {$\bullet$};
    \node (C) at (9.5,-0.5) {$\bullet$};
    \node (D) at (12.0, 0.0) {$\bullet$};
    
    \draw[-latex] (7,0) to (A);
    
    \draw[-latex] (A) to (B);
    \draw[-latex] (A) to (C);
    \draw[-latex] (B) to (D);
    \draw[-latex] (C) to (D);
  \end{tikzpicture}
\end{center}

The first type of structure we check for are conditionals.
There are three conditional structures:~\texttt{if} conditionals
(Figure~\ref{if-figure}), \texttt{if}-\texttt{else} conditionals
(Figure~\ref{if-else-figure}), and divergent conditionals
(Figure~\ref{divergent-figure}).
We introduce each with a separate rule, specialised to the form of the
conditional.

An \texttt{if} conditional with no else branch is introduced using
Rule~[\nameref{if-introduction-rule}], shown in
Figure~\ref{if-introduction-rule-figure}.
Such a structure can be recognised from the form of the \Circus{} code
in the $Running$ action, which is that of a node whose sequence of
instructions ends with a variable block, ending with an assignment of
the form $pc := \IF b \THEN x \ELSE y$, and for which the $pc = y$
node ends in an assignment $pc := x$.
Note that the branches cannot be the other way round (i.e.\ the
$pc = x$ branch cannot be the body of the conditional) since the
conditional branches come from Java's branching instructions, which
branch to the specified address if the condition is true and go to the
next instruction if it is false.

\begin{figure}[thp]
\begin{restatable}[\texttt{if}-conditional-intro]{crule}{IfConditionalIntroductionRule}
  \label{if-introduction-rule}
  \setlength{\zedindent}{0.25cm}
  \setlength{\zedtab}{0.57cm}
  % \setlength{\abovedisplayskip}{0.1cm}
  % \setlength{\belowdisplayskip}{0.1cm}
  Given $i : ProgramAddress$, if $i \neq j$, $i \neq k$, and 
  \begin{circus}
    \{frameStack \neq \emptyset\} \circseq A \circseq P \\
    {} = {} \\
    \{frameStack \neq \emptyset\} \circseq A \circseq P \circseq \{frameStack \neq \emptyset\}
  \end{circus}
  then
  \begin{circus}
    \begin{array}{l}
      \circmu X \circspot \\
      \t1 \circif frameStack = \emptyset \circthen \Skip \\
      \t1 {} \circelse frameStack \neq \emptyset \circthen {} \\
      \t2 \circif \cdots \\
      \t2 {} \circelse pc = i \circthen A \circseq \\
      \t3 (\circvar value1, value2 : Word \circspot P \circseq \\
      \t3 pc := \IF b \THEN j \ELSE k) \\
      \t2 {} \cdots {} \\
      \t2 {} \circelse pc = k \circthen B \circseq pc := j \\
      \t2 {} \cdots {} \\
      \t2 \circfi \circseq Poll \circseq X \\
      \t1 \circfi
    \end{array}
    \circrefines_A
    \begin{array}{l}
      \circmu X \circspot \\
      \t1 \circif frameStack = \emptyset \circthen \Skip \\
      \t1 {} \circelse frameStack \neq \emptyset \circthen {} \\
      \t2 \circif \cdots \\
      \t2 {} \circelse pc = i \circthen A \circseq \\
      \t3 (\circvar value1, value2 : Word \circspot P \circseq \\
      \t3 \circif b \circthen \Skip \\
      \t3 {} \circelse \lnot b \circthen pc := k \circseq Poll \circseq B \\
      \t3 \circfi) \circseq pc := j \\
      \t2 {} \cdots {} \\
      \t2 {} \circelse pc = k \circthen B \circseq pc := j \\
      \t2 {} \cdots {} \\
      \t2 \circfi \circseq Poll \circseq X \\
      \t1 \circfi 
    \end{array}
  \end{circus}
\end{restatable}
\caption{Rule~[\nameref{if-introduction-rule}]}
\label{if-introduction-rule-figure}
\end{figure}
Rule~[\nameref{if-introduction-rule}] introduces a conditional for
nodes that match the form described above, which in the rule is the
$pc = i$ node.
The conditional is introduced with the true branch empty (represented
by $\Skip$) and the false branch containing the instructions in the
body of the conditional.
The assignment $pc := j$ is moved outside the conditional from both
the true a\deleted{m}\added{n}d false branches.

As in Rule~[\nameref{sequence-introduction-rule}], the sequence of
actions for the node must not affect the nonemptiness of the
$frameStack$.
A similar condition is required for all the rules in this section.
We also require that the targets of the conditional are different from
the node at which the conditional is introduced, since that would
introduce a loop, which is not the purpose of this rule.
Rule~[\nameref{if-introduction-rule}] is applied on
line~\ref{algorithm-introduce-if} of
Algorithm~\ref{introduce-loops-and-conditionals-algorithm}.
Note that, since the structure can be identified from the form of the
\Circus{} code alone, it is not necessary to guard the application of
the rule with a condition on the control-flow graph.

We introduce \texttt{if}-\texttt{else} conditionals using
Rule~[\nameref{if-else-introduction-rule}] and divergent conditionals
using Rule~[\nameref{conditional-introduction-rule}].
Since these are similar to Rule~[\nameref{if-introduction-rule}], we
omit them here.
They can be found in Appendix~\ref{compilation-rules-appendix}.
We apply these rules on lines~\ref{algorithm-introduce-if-else}
and~\ref{algorithm-introduce-conditional}.

Rule~[\nameref{conditional-introduction-rule}] introduces a
conditional with no restrictions on its form.
To ensure it is only applied to nodes that match the form of
Figure~\ref{divergent-figure}, we guard its application by the
condition \Call{IsSimpleConditional}{$node$} on
line~\ref{algorithm-conditional-check}.
The procedure \Call{IsSimpleConditional}{$node$} checks if the targets
of $node$ have no outgoing nodes.
This is a condition on the control-flow graph that cannot be expressed
in the statement of the rule.

After attempting to introduce conditionals, we attempt to introduce
loops.
There are three types of loop to consider, as shown earlier:
\texttt{while} loops (Figure~\ref{while-figure}),
\texttt{do}-\texttt{while} loops (Figure~\ref{do-while-figure}), and
infinite loops (Figure~\ref{infinite-loop-figure}).
A \texttt{while} loop has a form similar to that of a conditional,
except that one of the branches ends with a jump back to the beginning
of the node with the conditional.
This structure may be introduced using
Rule~[\nameref{while-introduction-rule1}], shown in
Figure~\ref{while-introduction-rule1-figure}.
This rule introduces a conditional at a node $pc=i$ with its false
branch ending in an assignment of $i$ to $pc$, and introduces a
recursion to the beginning of the $pc=i$ node in that branch of the
conditional, representing a loop.
\begin{figure}[th]
\begin{restatable}[\texttt{while}-loop-intro1]{crule}{WhileLoopIntroductionRuleA}
  \label{while-introduction-rule1}
  \setlength{\zedindent}{0.2cm}
  \setlength{\zedtab}{0.55cm}
  Given $i : ProgramAddress$, if $i \neq j$,
  \begin{circus}
    \{frameStack \neq \emptyset\} \circseq A \circseq P \\
    {} = {} \\
    \{frameStack \neq \emptyset\} \circseq A \circseq P \circseq \{frameStack \neq \emptyset\}
  \end{circus}
  and
  \begin{circus}
    \{frameStack \neq \emptyset\} \circseq C \\
    {} = {} \\
    \{frameStack \neq \emptyset\} \circseq C \circseq \{frameStack \neq \emptyset\}
  \end{circus}
  then
  \begin{circus}
    \begin{array}{l}
      \circmu X \circspot \\
      \t1 \circif frameStack = \emptyset \circthen \Skip \\
      \t1 {} \circelse frameStack \neq \emptyset \circthen {} \\
      \t2 \circif \cdots \\
      \t2 {} \circelse pc = i \circthen A \circseq \\
      \t3 (\circvar value1, value2 : Word \circspot P \circseq \\
      \t3 pc := \IF b \THEN j \ELSE k) \\
      \t2 \cdots \\
      \t2 {} \circelse pc = j \circthen B \\
      \t2 \cdots \\
      \t2 {} \circelse pc = k \circthen C \circseq pc := i \\
      \t2 \cdots \\
      \t2 \circfi \circseq Poll \circseq X \\
      \t1 \circfi 
    \end{array}
    \circrefines_A
    \begin{array}{l}
      \circmu X \circspot \\
      \t1 \circif frameStack = \emptyset \circthen \Skip \\
      \t1 {} \circelse frameStack \neq \emptyset \circthen {} \\
      \t2 \circif \cdots \\
      \t2 {} \circelse pc = i \circthen (\circmu Y \circspot A \circseq \\
      \t3 (\circvar value1, value2 : Word \circspot P \circseq \\
      \t3 \circif b \circthen \Skip \\
      \t3 {} \circelse \lnot b \circthen {} \\
      \t4 pc := k \circseq Poll \circseq C \circseq \\
      \t4 pc := i \circseq Poll \circseq Y \\
      \t3 \circfi)) \circseq pc := j \\
      \t2 \cdots \\
      \t2 {} \circelse pc = j \circthen B \\
      \t2 \cdots \\
      \t2 {} \circelse pc = k \circthen C \circseq pc := i \\
      \t2 \cdots \\
      \t2 \circfi \circseq Poll \circseq X \\
      \t1 \circfi 
    \end{array}
  \end{circus}
\end{restatable}%
\caption{Rule~[\nameref{while-introduction-rule1}]}
\label{while-introduction-rule1-figure}
\end{figure}

As a \texttt{while} loop may occur with the loop at the end of either
condition\added{al} branch (since the loop may be created by a
\texttt{goto} instruction in the Java bytecode), we also provide a
similar rule, Rule~[\nameref{while-introduction-rule2}], which
introduces the loop in the true branch of the conditional.
These two rules are applied on lines~\ref{algorithm-introduce-while1}
and~\ref{algorithm-introduce-while2} of the algorithm.
Rule~[\nameref{while-introduction-rule2}] is presented in
Appendix~\ref{compilation-rules-appendix}.

The second type of loop we introduce is the \texttt{do}-\texttt{while}
loop.
A \texttt{do}-\texttt{while} loop is similar to a \texttt{while} loop,
but is distinguished by the fact that the conditional $pc$ assignment
that causes the loop is at the end of the loop, rather than at the
beginning or in the middle.
We introduce these loops using
Rule~[\nameref{do-while-introduction-rule}], which we omit due to its
similarity with Rule~[\nameref{while-introduction-rule1}]; it is also
presented in Appendix~\ref{compilation-rules-appendix}.
This rule is applied on line~\ref{algorithm-introduce-do-while} of the
algorithm.
Note that the false branch can never cause the loop in this case,
since it will just go to the next instruction.
Attempting to redirect it and create the loop with a \texttt{goto}
instruction would add an instruction within the loop after the
conditional, so it would be dealt with as a \texttt{while} loop. 
Therefore, it is not necessary to provide two compilation rules for
\texttt{do}-\texttt{while} loops.

The final loop structure that we attempt to introduce is that of an
infinite loop.
An infinite loop may be identified as a block of instructions that
ends with a $pc$ assignment that causes a jump back to the beginning
of the block of instructions.
We introduce these loops using
Rule~[\nameref{infinite-loop-introduction-rule}], presented in
Appendix~\ref{compilation-rules-appendix}.
This rule is applied on line~\ref{algorithm-introduce-infinite-loop}
of the algorithm.

After we have attempted to introduce each of the structures for a
particular node, we attempt to introduce a sequential composition.
This ensures that \texttt{if}, \texttt{if}-\texttt{else},
\texttt{while} and \texttt{do}-\texttt{while} structures that occur
within conditionals are sequentially composed with the node following
them if possible.
It also handles cases where sequential compositions occur before
loops, preventing them from being introduced in
Section~\ref{introduce-forward-sequence-subsection} without
interfering with the introduction of the loop.
Such a case occurs at the $pc=7$ node in our example.

The requirement for sequential composition to be introduced is the
same as in Section~\ref{introduce-forward-sequence-subsection}:~it
must be a simple sequential composition from a node with a single
outgoing edge to a node with a single incoming edge.
Thus we check for a simple sequence on
line~\ref{algorithm-lci-sequence-check} of
Algorithm~\ref{introduce-loops-and-conditionals-algorithm}.
The sequential composition is then introduced on
line~\ref{algorithm-lci-sequence-introduction} if it is a simple
sequential composition.

As mentioned earlier, these steps are repeated for each node, working
backwards through the control-flow graph of each method.
Each of the rules for introducing control flow structures reduces the
graph to either a sequential composition graph
(Figure~\ref{sequence-figure}) or a single node.

Divergent conditionals and infinite loops are the structures whose
control-flow graphs are reduced to a single node.
The remaining structures are reduced to sequential composition graphs.
The reduction of the sequential composition graph depends on which
form of node replacement is used to embed the structure in the
control-flow graph of the method.
There are four cases to consider:~root-node replacement
(Figure~\ref{root-replacement-figure}), end-node replacement
(Figure~\ref{end-replacement-figure}), internal-node replacement
(Figure~\ref{internal-replacement-figure}) and branch-end replacement
(Figure~\ref{branch-end-replacement-figure}).

Replacing the root node of a graph $G$ with a graph $H$ can be viewed
as replacing the end node of $H$ with $G$.
Since we are considering the nodes moving backwards through the
control flow graph, we always treat this as an end node replacement.

In the cases of end-node replacement and internal-node replacement we
can introduce the sequential composition immediately, reducing the
graph to a single node.

In the case of branch-end replacement, if some graph $H$ is embedded
in a graph $G$, then reducing $H$ to a sequential composition results
in the overall graph having the form of $G$.
This can be seen from the example shown in
Figure~\ref{branch-end-replacement-figure}, where reducing the
\texttt{while} loop structure formed by nodes $a$, $b$ and $4$ to a
sequential composition yields the graph shown in
Figure~\ref{branch-end-reduction-figure}.
This has the form of a \texttt{if}-\texttt{else} conditional
(Figure~\ref{if-else-figure}).
Such structures are introduced on further iterations of the loop over
the nodes.
Thus, given a structured control-flow graph at the beginning of this
stage, the control-flow graph is reduced to a single node, with all
the control-flow structures in the method introduced.

\begin{figure}
  \begin{center}
    \begin{tikzpicture}
      \useasboundingbox (-1.5,-2.5) rectangle (1.5,2);
      \node at (0,2) (start) {};
      \node at ( 0, 1)    (A) {$1$};
      \node at (-1,-0.5)  (B) {$a$};
      \node at ( 1,-0.5)  (C) {$3$};
      %\node at (-2,-1.5)  (D) {$b$};
      \node at ( 0,-2)    (G) {$4$};
      \draw[-latex] (start) -- (A);
      \draw[-latex] (A) -- (B);
      \draw[-latex] (A) -- (C);
      %\draw[-latex] (B) to (D);
      \draw[-latex] (B) to (G);
      %\draw[-latex] (D) to[in=180,out=90] (B);
      \draw[-latex] (C) -- (G);
    \end{tikzpicture}
  \end{center}
  \caption{The graph of Figure~\ref{branch-end-replacement-figure} after loop introduction}
  \label{branch-end-reduction-figure}
\end{figure}

In our example, we begin at the $pc=35$ node, where there are no
structures to introduce. 
The same holds true of the $pc=28$ and $pc=32$ nodes (note that the
edges coming from them are not simple sequential compositions).
An \texttt{if}-\texttt{else} conditional is introduced at $pc=21$,
absorbing the $pc=28$ and $pc=32$ nodes.
The sequential composition from the $pc=21$ node to the $pc=35$ node
can then be introduced immediately as it is now a simple sequential
composition (because it is not at the end of an outer conditional).
We then introduce a \texttt{while} loop at the $pc=39$\deleted{
  loop}\added{ node} (using
Rule~[\nameref{while-introduction-rule2}]), and the sequential
composition with the $pc=42$ node is introduced afterwards.
Finally, a sequential composition from the $pc=7$ to the $pc=39$ node
is introduced, collapsing the control flow graph to a single node.
The code at $pc=7$ is then that shown earlier in
Figure~\ref{loop-and-conditional-introduction-example-figure}.
