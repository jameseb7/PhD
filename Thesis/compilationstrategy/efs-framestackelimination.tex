After all methods have been refined to use individual variables, the
$frameStack$ is no longer used.
We can thus eliminate the $frameStack$ from the state. 
This is performed as described in
Algorithm~\ref{remove-frameStack-from-state-algorithm}, which defines
the \Call{RemoveFrameStackFromState}{} procedure.

\begin{algorithm}
  \begin{algorithmic}
    \State \ApplyFor{Law~[\nameref{forwards-data-refinement-law}]}{$[]$, $FrameStackEliminationCI$}
    \label{algorithm-remove-frameStack-data-refinement}
    \State \ApplyFor{Law~[\nameref{process-param-elim-law}]}{$cs$}
    \label{algorithm-cs-elimination}
  \end{algorithmic}
  \caption{\Call{RemoveFrameStackFromState}{}}
  \label{remove-frameStack-from-state-algorithm}
\end{algorithm}

First, on line~\ref{algorithm-remove-frameStack-data-refinement}, we
perform a data refinement to remove the $frameStack$ from the state.
The new state after the data refinement is the empty schema, and the
coupling invariant, $FrameStackEliminationCI$, maps all $frameStack$
values onto the empty state.
Since $frameStack$ is no longer used in the process, the only action
affected is the state initialisation, which becomes $Skip$.
After this, on line~\ref{algorithm-cs-elimination}, we eliminate the
$cs$ parameter from the process, since it is no longer used.
The result is the process $CThr_{bc,cs}(t)$, as shown in
Theorem~\ref{efs-thm}.
The only thing remaining to be done is to refine the representation of
objects, which is performed in the next stage of the strategy.
