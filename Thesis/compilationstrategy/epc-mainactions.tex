After the control flow of each method has been introduced and each
method has been separated into its own method action, the only
remaining uses of $pc$ are to select a method action when a method is
executed in response to a request from the $Launcher$.
This occurs in the $MainThread$ and $Started$ actions, where a call to
$Running$ follows a call to $StartInterpreter$.
To remove these final uses of $pc$, we replace $Running$ with a call
to a new action, $ExecuteMethod$, which chooses a method action based
on a class and method identifier.
This performed as specified in
Algorithm~\ref{refine-main-actions-algorithm}.

\begin{algorithm}
  \begin{algorithmic}[1]
    \State \ApplyToFor{Law~[\nameref{copy-rule-law}]}{\Call{ActionBody}{$MainThread$}}{$Running$}
    \label{algorithm-MainThread-expand-Running}
    \State \ApplyToFor{Law~[\nameref{copy-rule-law}]}{\Call{ActionBody}{$Started$}}{$Running$}
    \label{algorithm-Started-expand-Running}
    \State \ApplyTo{Rule~[\nameref{Running-refinement-rule}]}{\Call{ActionBody}{$MainThread$}}
    \label{algorithm-MainThread-Running-refinement}
    \State \ApplyTo{Rule~[\nameref{Running-refinement-rule}]}{\Call{ActionBody}{$Started$}}
    \label{algorithm-Started-Running-refinement}
    \MatchIn{$\begin{array}[t]{l}
                executeMethod?t \prefixcolon (t = thread) ?c?m?a \\
                \t1 {} \then (A)(c, m, a)
              \end{array}$}{\Call{ActionBody}{$Started$}}
    \State \ApplyFor{Law~[\nameref{action-intro-law}]}{$ExecuteMethod$, $A$}
    \label{algorithm-ExecuteMethod-introduction}
    \State \ApplyToFor{Law~[\nameref{copy-rule-law}]}{\Call{ActionBody}{$MainThread$}}{$ExecuteMethod$}
    \label{algorithm-MainThread-copy-ExecuteMethod}
    \State \ApplyToFor{Law~[\nameref{copy-rule-law}]}{\Call{ActionBody}{$Started$}}{$ExecuteMethod$}
    \label{algorithm-Started-copy-ExecuteMethod}
  \end{algorithmic}
  \caption{\Call{RefineMainActions}{}}
  \label{refine-main-actions-algorithm}
\end{algorithm}

Algorithm~\ref{refine-main-actions-algorithm} differs from previous
algorithms in that it does not operate purely upon the $Running$
action.
We instead refine the composition of $StartInterpreter$ and $Running$
in $MainThread$ and $Started$.
First, Law~[\nameref{copy-rule-law}] is applied on
lines~\ref{algorithm-MainThread-expand-Running}
and~\ref{algorithm-Started-expand-Running} to replace the call to
$Running$ with its body in $MainThread$ and $Started$.
Then, we apply Rule~[\nameref{Running-refinement-rule}], shown in
Figure~\ref{Running-refinement-rule-figure}, on
lines~\ref{algorithm-MainThread-Running-refinement}
and~\ref{algorithm-Started-Running-refinement}.
This refines the composition of $Started$ with the body of $Running$
in $MainThread$ and $Started$.

\begin{figure}[thp]
\begin{restatable}[$StartInterpreter$-$Running$-refinement]{crule}{RunningRefinementRule}
  \label{Running-refinement-rule}
  If $(c_1,m_1), \ldots , (c_n,m_n)$ are the only 
  $ClassID \cross MethodID$ values such that
  $classID = c_i \land methodID = m_i \implies \pre ResolveMethod$,
  and for each $i \in \{1 \upto n\}$, there exists
  $classInfo_i : Class$ and $entry_i : ProgramAddress$ such that,
  \begin{circus}
    \{ classID = c_i \land methodID = m_i \} \circseq ResolveMethod \\
    {} = {} \\
    \{ classID = c_i \land methodID = m_i \} \circseq ResolveMethod \circseq \\
    \t1 \{ class = classInfo_i \land classInfo_i.methodEntry~m_i = entry_i \},
  \end{circus}
  and, for each $i \in \{1 \upto n\}$,
  \begin{circus}
    \{ \# frameStack = 1\} \circseq M_i \\
    {} = {} \\
    \{ \# frameStack = 1\} \circseq M_i \circseq \{ framestack = \emptyset\},
  \end{circus}
  then,
  \begin{circus}
    \begin{array}{l}
      \{ frameStack = \emptyset \} \circseq \\
      StartInterpreter \circseq Poll \circseq \circmu X \circspot \\
      \t1 \circif frameStack = \emptyset \circthen \Skip \\
      \t1 {} \circelse framestack \neq \emptyset \circthen {}  \\
      \t2 \circif pc = entry_1 \circthen M_1 \\
      \t2 {} \cdots {} \\
      \t2 {} \circelse pc = entry_n \circthen M_n \\
      \t2 \circfi \circseq Poll \circseq X \\
      \t1 \circfi
    \end{array}
    \circrefines_A
    \begin{array}{l}
      executeMethod?t \prefixcolon (t = thread) ?c?m?a \then {} \\
      (\circval classID : ClassID; \\
      \circval methodID : MethodID; \\
      \circval methodArgs : \seq Word \circspot \\
      \circif {(classID, methodID) = (c_1,m_1)} \circthen {} \\
      \t1 InterpreterNewStackFrame[\\
      \t2 classInfo_1/class?, \\
      \t2 m_1/methodID?] \circseq Poll \circseq M_1 \\
      {} \cdots {} \\
      {} \circelse (classID, methodID) = (c_n,m_n) \circthen {} \\
      \t1 InterpreterNewStackFrame[\\
      \t2 classInfo_n/class?, \\
      \t2 m_n/methodID?] \circseq Poll \circseq M_n \\ 
      \circfi)(c, m, a) \circseq Poll
    \end{array}
  \end{circus}
\end{restatable}
\caption{Rule~[\nameref{Running-refinement-rule}]}
\label{Running-refinement-rule-figure}
\end{figure}

When we apply Rule~[\nameref{Running-refinement-rule}], we first
introduce an assumption stating that the $frameStack$ is nonempty
before execution of $StartInterpreter$.
This is true in both the places that the rule is applied, since the
$frameStack$ is initially empty and no stack frames have been created
at the point when $MainThread$ and $Started$ occur.
The $Running$ action causes the $stackFrame$ to be empty after its
execution, so the $stackFrame$ is also empty when the $MainThread$ and
$Started$ actions loop to allow $StartInterpreter$ to be executed
again.

Rule~[\nameref{Running-refinement-rule}] collects all the class and
method identifiers that are resolved by the data operation
$ResolveMethod$.
It expands the definiton of $StartInterpreter$, and introduces a
choice over these class and method identifiers, comparing them to the
identifiers communicated on the $executeMethod$ channel.

Within each branch of the choice for a class identifier $c_i$ and
method identifier $m_i$, a new stack frame is first created by the
$InterpreterNewStackFrame$ operation, using the class information,
$classInfo_i$, provided by $ResolveMethod$ for $c_i$ and $m_i$.
The branch then behaves as the method action in $Running$ that
corresponds to the method entry point, $entry_i$, associated with
$m_i$ in $classInfo_i$.
Note that the mapping from class and method identifiers to class
information and method entries is not necessarily injective, since
inherited methods\deleted{ will} share the same class information and
bytecode instructions.
The choice over class and method identifiers is wrapped in a value
parameter block, since it forms the body of the $ExecuteMethod$
action.

After Rule~[\nameref{Running-refinement-rule}] has been applied, the
$ExecuteMethod$ action is introduced in the $Thr$ process using
Law~[\nameref{action-intro-law}], on
line~\ref{algorithm-ExecuteMethod-introduction} of
Algorithm~\ref{refine-main-actions-algorithm}.
The body of $ExecuteMethod$, introduced in $MainThread$ and
$NotStarted$ by Rule~[\nameref{Running-refinement-rule}], is then
replaced with a call to $ExecuteMethod$ by application of
Law~[\nameref{copy-rule-law}], on
lines~\ref{algorithm-MainThread-copy-ExecuteMethod}
and~\ref{algorithm-Started-copy-ExecuteMethod}.
This results in $MainThread$ and $Started$ having the form shown
previously in Figure~\ref{refine-main-actions-example-figure}.
