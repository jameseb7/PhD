In this chapter, we have presented our compilation strategy from an
interpreting SCJVM to our model of C code.
While our compilation strategy proves the correctness of the
compilation from an interpreting SCJVM to our model of C code, there
are further optimisations that may be performed on the output of the
strategy.

One example of such an optimisation is the removal of the unnecessary
choice offered in virtual method calls with only one possible target.
Such choices are made using the class identifier of the object, which,
in our model, is obtained via communication with the struct manager on
the $getClassIDOf$ channel.
The removal of the choice requires the removal of this communication,
which is a refinement all the processes that participate in this
communication.
To remove the communication requires collapsing the parallelism
between these processes and using the fact that the communication is
hidden to remove it.
This is not performed in our strategy, since each stage of the
strategy operates only on a single process, but is a relatively
straightforward optimisation that could be added as an extension of the
strategy in future work.

A further consideration is that Z schemas bindings represent an
unordered collection of fields, whereas C structs define the order in
which their fields are stored in memory.
This means that, while our struct manager model defines what fields
must be in each object's struct type, it does not specify the order of
those fields and so is still somewhat more abstract than the C code
itself.
This can be rectified by a further data refinement to a representation
using Z sequences to represent the field data stored in memory.
Field names for each struct type would then be associated with offsets
into these sequences, as field names in C are associated with offsets
in the struct's memory.
We have not performed such a data refinement as part of the strategy,
since we believe the form of the struct manager is sufficiently clear
to implementers, although they will have to choose an ordering for the
C structs when implementing.

We expect other optimisations to be performed by the C compiler that
compiles the output of the strategy.
The correctness of such optimisations is part of verification of the C
compiler and thus outside the scope of our work.
However, some optimisations could be integrated into the strategy as
part of future work.
An example is the elimination of unnecessary assignments, such as on
line~\ref{C-code-unnecessary-assignment} of the code in
Figure~\ref{efs-introduce-variables-c-code-figure}.
There, \texttt{stack1} is used as an intermediate variable to set
\texttt{var4} and is not otherwise used before it is overwritten on
line~\ref{C-code-unnecessary-assignment-end}.
These assignments are removed by optimising C compilers, and so are
not removed by our strategy, or the icecap HVM, but could be removed
in order to produce clearer C code.

Other possible directions for future work extending the strategy
include weakening the assumptions described in
Section~\ref{compilation-assumptions-section}.
Our definition of a structured program is slightly stronger than the
structural requirements imposed by MISRA-C, which permits a single
exit from the middle of a loop in addition to the condition at the
start or end of the loop.
This means loops may have two exits in MISRA-C, whereas our strategy
only accounts for loops with a single exit point.
The strategy could be modified to allow for loops with two exit points
by adding new rules, similar to
Rule~[\nameref{while-introduction-rule1}], to introduce such loops,
having two conditionals to allow for exit from the loop.

We do not model and handle integer overflow in the strategy due to the
fact that it is not handled in icecap, instead requiring the SCJ
programmer to ensure that their code does not include overflows.
A possible extension of the strategy would be to model the overflow
behaviour of the JVM in the bytecode interpreter and refine it C code
that enforces that overflow behaviour by checking if overflow would
occur before performing an operation.
This would create extra checks in many places where they would not be
necessary, but such checks could be removed in places where overflow
can be proved not to occur.

While we do not allow recursion due to its potential for stack
overflow, there may be a few cases in which recursion can be shown to
be bounded and hence safe.
The strategy could be extended to handle such cases, requiring rules
for introducing loops caused by method calls, and separating the
resulting recursions in recursive actions to represent recursive
methods.
Recursion with more than one method involves a similar approach,
introducing nested loops and creating mutually recursive actions.


In the next chapter, we discuss in more detail how the correctness of
the strategy is assured and evaluate it through consideration of some
examples.

