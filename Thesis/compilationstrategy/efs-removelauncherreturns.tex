After the previous stage, each conditional branch in a method ends
with a return instruction or an infinite loop.
This can be seen in
Figure~\ref{pc-elimination-HandleAsyncEvent-example-figure}, presented
earlier, where the method $TPK\_handleAsyncEvent$ ends with a
$HandleReturnEPC$ action.
In the first step of this stage, at
line~\ref{algorithm-remove-launcher-returns} of
Algorithm~\ref{efs-algorithm}, such actions are moved outside the
method and their definitions are expanded so that their communication
with the $Launcher$ can be handled.
This is performed as described in
Algorithm~\ref{remove-launcher-returns-algorithm}, which defines the
procedure \Call{RemoveLauncherReturns}{}.
\begin{algorithm}[ht]
  \begin{algorithmic}[1]
    \For{$methodName \gets $ \Call{MethodActionNames}{$cs$}}
    \label{algorithm-method-single-return-loop}
    \State $methodBody \gets$ \Call{ActionBody}{$methodName$}
    \State $returnAction \gets$ \Call{ReturnAction}{$methodBody$}
    \label{algorithm-determine-return-action}   
    \State \ExhaustivelyApplyToFor{Law~[\nameref{rec-action-intro-law}]}{$methodBody$}{$returnAction$}
    \label{algorithm-introduce-infinite-loop-returns}
    \State \ExhaustivelyApplyTo{Law~[\nameref{alt-seq-dist-law}]}{$methodBody$}
    \label{algorithm-distribute-returns}
    \State \Call{RedefineMethodExcludingReturn}{$methodName$,$returnAction$}
    \label{algorithm-redefine-method-action-excluding-return-action}
    \EndFor
    \State \Call{IntroduceFrameStackAssumptions}{}
    \label{algorithm-introduce-frameStack-assumptions}
    \State \ExhaustivelyApply{Rule~[\nameref{refine-HandleReturnEPC-empty-frameStack-rule}]}
    \label{algorithm-eliminate-returns-try-begin}
    \State \ExhaustivelyApply{Rule~[\nameref{refine-HandleReturnEPC-nonempty-frameStack-rule}]}
    \State \ExhaustivelyApply{Rule~[\nameref{refine-HandleAreturnEPC-empty-frameStack-rule}]}
    \State \ExhaustivelyApply{Rule~[\nameref{refine-HandleAreturnEPC-nonempty-frameStack-rule}]}
    \label{algorithm-eliminate-returns-try-end}
    \State \ExhaustivelyApply{Law~[\nameref{assump-elim-law}]}
    \label{algorithm-remove-launcher-returns-assump-elim}
    \State \ExhaustivelyApply{Law~[\nameref{seq-unitl-law}]}
    \label{algorithm-remove-launcher-returns-seq-unitl}
    \State \ApplyToFor{Law~[\nameref{copy-rule-law}]}{\Call{ActionBody}{$MainThread$}}{$ExecuteMethod$}
    \label{algorithm-copy-ExecuteMethod-in-MainThread}
    \State \ApplyToFor{Law~[\nameref{copy-rule-law}]}{\Call{ActionBody}{$Started$}}{$ExecuteMethod$}
    \label{algorithm-copy-ExecuteMethod-in-Started}
    \State \ApplyTo{Rule~[\nameref{ExecuteMethod-refinement-rule}]}{\Call{ActionBody}{$MainThread$}}
    \label{algorithm-ExecuteMethod-refinement-MainThread}
    \State \ApplyTo{Rule~[\nameref{ExecuteMethod-refinement-rule}]}{\Call{ActionBody}{$Started$}}
    \label{algorithm-ExecuteMethod-refinement-Started}
    \State \ApplyReverseFor{Law~[\nameref{action-intro-law}]}{$ExecuteMethod$, \Call{ActionBody}{$ExecuteMethod$}}
    \label{algorithm-remove-ExecuteMethod}
    \MatchIn{$\begin{array}[t]{l}(\circvar retVal : Word \circspot (A)(c, m, a, retVal) \circseq \\\t1 executeMethodRet!thread!retVal \then \Skip)\end{array}$\\$\t2$}{\Call{ActionBody}{$Started$}}
  \State \ApplyFor{Law~[\nameref{action-intro-law}]}{$ExecuteMethod$, $A$}
    \label{algorithm-reintroduce-ExecuteMethod}
    \State \ApplyReverseToFor{Law~[\nameref{copy-rule-law}]}{$MainThread$}{$ExecuteMethod$}
    \label{algorithm-copy-ExecuteMethod-out-MainThread}
    \State \ApplyReverseToFor{Law~[\nameref{copy-rule-law}]}{$Started$}{$ExecuteMethod$}
    \label{algorithm-copy-ExecuteMethod-out-Started}
  \end{algorithmic}
  \caption{\Call{RemoveLauncherReturns}{}}
  \label{remove-launcher-returns-algorithm}
\end{algorithm}
  
Algorithm~\ref{remove-launcher-returns-algorithm} begins by iterating
over each of the method actions, in the for loop beginning on
line~\ref{algorithm-method-single-return-loop}.
This determines the name, $methodName$, for each method's action from
the class information, $cs$, via a function
\Call{MethodActionNames}{}.
We take the method's body, $methodBody$, as the body of the action
corresponding to $methodName$.

The return actions that may occur at the end of method branches are
either $HandleAreturnEPC$ or $HandleReturnEPC$.
$HandleAreturnEPC$ occurs only in methods that return a value and,
conversely, $HandleReturnEPC$ occurs only in methods that do not
return a value.
We can thus determine which return action a method uses by examining
$methodBody$ to see which action occurs at the end of the branches.
A method in which all branches end in infinite loops is treated as
using the return action $HandleReturnEPC$, since it does not produce a
value.
We determine the return action type, $returnAction$, for $methodBody$
on line~\ref{algorithm-determine-return-action}, using a syntactic
function \Call{ReturnAction}{}.

With $returnAction$ identified, we convert the method to a form in
which it has one occurrence of that action at the end of its body.
This is achieved by introducing occurrences of $returnAction$ after
infinite loops in $methodBody$ using
Law~[\nameref{rec-action-intro-law}] and distributing occurrences of
$returnAction$ outside conditionals using
Law~[\nameref{alt-seq-dist-law}].
These laws are applied on
lines~\ref{algorithm-introduce-infinite-loop-returns}
and~\ref{algorithm-distribute-returns} of
Algorithm~\ref{remove-launcher-returns-algorithm}.

When the method has a single return instruction at the end, it is
redefined to exclude the return action.
This is performed using Law~[\nameref{copy-rule-law}] and
Law~[\nameref{action-intro-law}], but since the use of these laws to
redefine an action in this way is standard, it is specified in a
separate procedure
\Call{RedefineMethodExcludingReturn}{$methodName$,$returnAction$},
called on
line~\ref{algorithm-redefine-method-action-excluding-return-action} of
Algorithm~\ref{remove-launcher-returns-algorithm}.
This procedure is defined in
Algorithm~\ref{redefine-method-action-excluding-return-action-algorithm},
which is included in
Appendix~\ref{remove-launcher-returns-appendix-subsection}.
% This is performed by introducing an intermediate action named
% $methodName'$ on
% line~\ref{algorithm-remove-returns-introduce-method-action}, using
% Law~[\nameref{action-intro-law}], containing the actions of
% $methodBody$ before $returnAction$.
% Those actions are then replaced with a reference to $methodName'$ on
% line~\ref{algorithm-eliminate-returns-copy-method-out} via an
% application of Law~[\nameref{copy-rule-law}].
% The definition of $methodName$ is then expanded everywhere on
% line~\ref{algorithm-eliminate-returns-copy-method-in}, replacing
% references to it with a reference to $methodName'$ and $returnAction$.
% The $methodName$ action is then removed on
% line~\ref{algorithm-remove-returns-eliminate-method-action} and the
% $methodName'$ action is renamed to $methodName$ on
% line~\ref{algorithm-rename-method-action}, using
% Law~[\nameref{action-rename-law}], which allows renaming an action to
% a fresh name.

We then introduce assumptions that state the depth of the
$frameStack$, so that we can determine whether the $frameStack$ is
empty or not at each return instruction.
This introduction of assumptions is performed by the call to the
\Call{IntroduceFrameStackAssumptions}{} procedure on
line~\ref{algorithm-introduce-frameStack-assumptions}.
It is defined by
Algorithm~\ref{introduce-frameStack-assumptions-algorithm}, which is
included in Appendix~\ref{remove-launcher-returns-appendix-subsection}
along with the rules used by it.

This procedure introduces an assumption $\{\# frameStack = 0 \}$ from
the $InterpreterInitEPC$ schema (which is the result of applying the
data refinement in Section~\ref{remove-pc-from-state-subsection} to
$InterpreterInit$) at the start of the main action of $ThrCF_{bc,cs}$.
The assumption is then distributed throughout the process by
exhaustive application of restricted versions of standard algebraic
assumption-distribution laws, and rules stating how the size of
$frameStack$ is affected by the operations that appear in the code
resulting from the elimination of program counter.

The restrictions added to these laws, in the form of extra provisos,
guarantee that the assumption is not distributed if an identical
assumption is already in place, thus preventing unbounded distribution
of the assumptions and ensuring the procedure terminates.
% As the procedure is a straightforward application of these rules and
% laws, we omit its definition here.
The result is that the return instructions following the method
actions in $ExecuteMethod$ have an assumption $\# frameStack = 1$
before them, and the return instructions occurring in the middle of
other methods have an assumption $\# frameStack = k$ for some $k > 1$.

% \begin{figure}[t!]
%   \centering
%   \setlength{\zedtab}{0.4cm}
%   \setlength{\zedindent}{0pt}
%   \setlength{\zedleftsep}{0pt}
%   \setlength{\abovedisplayskip}{0pt}
%   \setlength{\belowdisplayskip}{0pt}
%   \setlength{\abovedisplayshortskip}{0pt}
%   \setlength{\belowdisplayshortskip}{0pt}
%   \begin{circusaction}
%     ExecuteMethod \circdef \\
%     \t1 \circval classID : ClassID; \circval methodID : MethodID; \circval methodArgs : \seq Word \circspot \\
%     \t1 \circif (classID, methodID) = (TPKClassID, APEHinit) \circthen {} \\
%     \t2 InterpreterNewStackFrame[TPK/class?] \circseq \\
%     \t2 TPK\_APEHinit \circseq HandleReturnEPC \circseq \{ framestack = \emptyset \} \\
%     \t1 {} \circelse (classID, methodID) = (TPKClassID, handleAsyncEvent) \circthen {} \\
%     \t2 InterpreterNewStackFrame[TPK/class?] \circseq \\
%     \t2 TPK\_handleAsyncEvent \circseq HandleReturnEPC \circseq \{ framestack = \emptyset \} \\
%     \t1 {} \circelse (classID, methodID) = (TPKClassID, f) \circthen {} \\
%     \t2 InterpreterNewStackFrame[TPK/class?] \circseq \\
%     \t2 TPK\_f \circseq HandleAreturnEPC \circseq \{ framestack = \emptyset \} \\
%     \t1 {} \cdots {} \\
%     \t1 \circfi
%   \end{circusaction}

%   \begin{circusaction}
%     TPK\_handleAsyncEvent \circdef \\
%     \t1 HandleNewEPC(27) \circseq Poll \circseq HandleDupEPC \circseq Poll \circseq  HandleAconst\_nullEPC \circseq Poll \circseq \\
%     \t1 (\circvar poppedArgs : \seq Word \circspot \\
%     \t2 \lschexpract \exists argsToPop? == m + 1 @ InterpreterStackFrameInvoke \rschexpract \circseq \\
%     \t2 \lschexpract InterpreterNewStackFrame[\\
%     \t3 ConsoleConnection/class?, CCinit/methodID?, poppedArgs/methodArgs?] \rschexpract) \circseq Poll \circseq \\
%     \t1 ConsoleConnection\_CCinit \circseq HandleReturnEPC \circseq \{ frameStack \neq \emptyset \} \circseq Poll \circseq \\
%     %\t1 HandleAstoreEPC(1) \circseq Poll \circseq HandleAloadEPC(1) \circseq \\
%     \t1 {} \cdots {} \\
%     % \t1 Poll \circseq (\circvar poppedArgs : \seq Word \circspot \lschexpract \exists argsToPop? == m + 1 @ InterpreterStackFrameInvoke \rschexpract \circseq \\
%     % \t1 getClassIDOf!(head~poppedArgs)?cid \then \lschexpract InterpreterNewStackFrame[ \\
%     % \t2 ConsoleConnection/class?, openInputStream/methodID?, poppedArgs/methodArgs?] \rschexpract) \circseq \\
%     % \t1 Poll \circseq ConsoleConnection\_openInputStream \circseq Poll \circseq  HandleAstoreEPC(2) \circseq Poll \circseq \\
%     % \t1 HandleAloadEPC(1) \circseq Poll \circseq (\circvar poppedArgs : \seq Word \circspot \\
%     % \t1 \lschexpract \exists argsToPop? == m + 1 @ InterpreterStackFrameInvoke \rschexpract \circseq \\
%     % \t1 getClassIDOf!(head~poppedArgs)?cid \then \lschexpract InterpreterNewStackFrame[\\
%     % \t2 ConsoleConnection/class?, openOutputStream/methodID?, poppedArgs/methodArgs?] \rschexpract) \circseq \\
%     % \t1 Poll \circseq ConsoleConnection\_openOutputStream \circseq Poll \circseq HandleAstoreEPC(3) \circseq \\
%     %\t1 {} \cdots {} \\
%     % \t1 Poll \circseq HandleIconstEPC(0) \circseq Poll \circseq HandleAstoreEPC(4) \circseq Poll \circseq Poll \circseq \circmu Y \circspot \\
%     % \t2 HandleAloadEPC(4) \circseq Poll \circseq HandleIconstEPC(10) \circseq Poll \circseq \\
%     % \t2 \circvar value1, value2 : Word \circspot InterpreterPop2 \circseq \\
%     % \t2 \circif value1 \leq value2 \circthen \\
%     % \t3 {} \cdots {}  \\
%     % Poll \circseq HandleAloadEPC(2) \circseq Poll \circseq \\
%     % \t3 (\circvar poppedArgs : \seq Word \circspot \\
%     % \t4 \lschexpract \exists argsToPop? == m + 1 @ InterpreterStackFrameInvoke \rschexpract \circseq \\
%     % \t4 getClassIDOf!(head~poppedArgs)?cid \then \lschexpract InterpreterNewStackFrame[ \\
%     % \t5 ConsoleInput/class?, read/methodID?, poppedArgs/methodArgs?] \rschexpract) \circseq \\
%     % \t3 Poll \circseq ConsoleInput\_read \circseq Poll \circseq \\
%     \t3 (\circvar poppedArgs : \seq Word \circspot \\
%     \t4 \lschexpract \exists argsToPop? == m @ InterpreterStackFrameInvoke \rschexpract \circseq \\
%     \t4 \lschexpract InterpreterNewStackFrame[\\
%     \t5 TPK/class?, f/methodID?, poppedArgs/methodArgs?] \rschexpract) \circseq \\
%     \t3 Poll \circseq TPK\_f \circseq HandleAreturnEPC \circseq \{ frameStack \neq \emptyset \} \circseq Poll \circseq \\
%     % \t3 HandleAstoreEPC(5) \circseq Poll \circseq HandleAloadEPC(5) \circseq \\
%     \t3 {} \cdots {} \\
%     % \t3 Poll \circseq HandleIconstEPC(400) \circseq Poll \circseq \circvar value1, value2 : Word \circspot InterpreterPop2 \circseq \\
%     % \t3 \circif value1 \leq value2 \circthen HandleAloadEPC(3) \circseq Poll \circseq HandleAloadEPC(5) \circseq Poll \circseq \\
%     % \t4 (\circvar poppedArgs : \seq Word \circspot \lschexpract \exists argsToPop? == m + 1 @ InterpreterStackFrameInvoke \rschexpract \circseq \\
%     % \t4 getClassIDOf!(head~poppedArgs)?cid \then \lschexpract InterpreterNewStackFrame[ \\
%     % \t5 ConsoleOutput/class?, write/methodID?, poppedArgs/methodArgs?] \rschexpract)) \circseq \\
%     % \t4 Poll \circseq ConsoleOutput\_write \\
%     % \t3 {} \circelse value1 > value2 \circthen HandleAloadEPC(3) \circseq Poll \circseq HandleIconstEPC(0) \circseq Poll \circseq \\
%     % \t4 (\circvar poppedArgs : \seq Word \circspot \lschexpract \exists argsToPop? == m + 1 @ InterpreterStackFrameInvoke \rschexpract \circseq \\
%     % \t4 getClassIDOf!(head~poppedArgs)?cid \then \lschexpract InterpreterNewStackFrame[ \\
%     % \t5 ConsoleOutput/class?, write/methodID?, poppedArgs/methodArgs?] \rschexpract)) \circseq \\
%     % \t4 Poll \circseq ConsoleOutput\_write \\
%     % \t3 \circfi \circseq Poll \circseq HandleAloadEPC(4) \circseq Poll \circseq HandleIconstEPC(1) \circseq Poll \circseq HandleIaddEPC \circseq \\
%     \t3 Poll \circseq HandleAstoreEPC(4) \circseq Poll \circseq Y \\
%     \t2 {} \circelse value1 > value2 \circthen \Skip \\
%     \t2 \circfi \circseq Poll
%   \end{circusaction}
%   \caption{The $ExecuteMethod$ and $TPK\_handleAsyncEvent$ actions
%     after introduction of $frameStack$ assumptions}
%   \label{efs-return-assumption-distribution-figure}
% \end{figure}

After assumptions on the state of the $frameStack$ have been
introduced, we can handle the return actions at each point where they
occur, applying the rules on
lines~\ref{algorithm-eliminate-returns-try-begin}
to~\ref{algorithm-eliminate-returns-try-end} of
Algorithm~\ref{remove-launcher-returns-algorithm} wherever possible.
An example is
Rule~[\nameref{refine-HandleReturnEPC-empty-frameStack-rule}], shown
in Figure~\ref{refine-HandleReturnEPC-empty-frameStack-rule-figure}.
This rule replaces an occurrence of $HandleReturnEPC$ where the
$frameStack$ has a size of $1$, with a call to the data operation
$InterpreterReturnEPC$ followed by a communication with the $Launcher$
on the $executeMethodRet$ channel.
The value communicated on $executeMethodRet$ is $returnValue$,
introduced in a variable block.

Rule~[\nameref{refine-HandleReturnEPC-empty-frameStack-rule}]
essentially expands the definition of the $HandleReturnEPC$ action,
shown below, and resolves the choice in $CheckLauncherReturn$
(presented earlier in Section~\ref{cee-interpreter-subsection}) over
whether $frameStack$ is empty.
This involves distributing the assumption over the data operation
$InterpreterReturnEPC$, which removes the last stack frame from the
$frameStack$, causing it to be empty when $\# frameStack = 1$. 
\begin{circus}
  HandleReturnEPC \circdef \circvar returnValue : Word \circspot \\
  \t1 \lschexpract InterpreterReturnEPC \rschexpract \circseq CheckLauncherReturn(returnValue)
\end{circus}

\begin{figure}[tp!]
  \begin{restatable}[refine-$HandleReturnEPC$-empty-$frameStack$]{crule}{RefineHandleReturnEPCEmptyFrameStackRule}
  \label{refine-HandleReturnEPC-empty-frameStack-rule}
  \begin{circus}
    \begin{array}{l}
      \{\# frameStack = 1\} \circseq \\
      HandleReturnEPC
    \end{array}
    \circrefines_A
    \begin{array}{l}
      \circvar returnValue : Word \circspot \\
      \lschexpract InterpreterReturnEPC \rschexpract \circseq \\
      executeMethodRet!thread!returnValue \then \Skip
    \end{array}
  \end{circus}
\end{restatable}
\caption{Rule~[\nameref{refine-HandleReturnEPC-empty-frameStack-rule}]}
\label{refine-HandleReturnEPC-empty-frameStack-rule-figure}
\end{figure}

The other rules used on lines~\ref{algorithm-eliminate-returns-try-begin}
to~\ref{algorithm-eliminate-returns-try-end} are similar, handling the
cases for the $frameStack$ being left nonempty and the
$HandleAreturnEPC$ action.
They can be found in Appendix~\ref{compilation-rules-appendix}.

In the cases when the $frameStack$ is not empty after execution of the
return instruction, which occurs when a method is called from within
another method, the resolution of the choice in $CheckLauncherReturn$
collapses it to the branch where there is no communication on
$executeMethodRet$.
The variable block for $returnValue$ is thus removed as part of the
rules handling those cases, since it is not used.

After application of these rules, any remaining assumptions are
eliminated by application of Law~[\nameref{assump-elim-law}] and
Law~[\nameref{seq-unitl-law}] on
lines~\ref{algorithm-remove-launcher-returns-assump-elim}
and~\ref{algorithm-remove-launcher-returns-seq-unitl}, since they are
no longer necessary.

\begin{figure}[tp!]
  \centering
  % \setlength{\zedtab}{0.4cm}
  % \setlength{\zedindent}{0pt}
  % \setlength{\zedleftsep}{0pt}
  % \setlength{\abovedisplayskip}{0pt}
  % \setlength{\belowdisplayskip}{0pt}
  % \setlength{\abovedisplayshortskip}{0pt}
  % \setlength{\belowdisplayshortskip}{0pt}
  \begin{circusaction}
    ExecuteMethod \circdef \\
    \t1 \circval classID : ClassID; \circval methodID : MethodID; \circval methodArgs : \seq Word \circspot \\
    \t1 \circif (classID, methodID) = (TPKClassID, APEHinit) \circthen {} \\
    \t2 InterpreterNewStackFrame[TPK/class?, APEHinit/methodID?] \circseq Poll \circseq \\
    \t2 TPK\_APEHinit \circseq \\
    \t2 (\circvar returnValue : Word \circspot \lschexpract InterpreterReturnEPC \rschexpract \circseq \\
    \t3 executeMethodRet!thread!returnValue \then \Skip) \\
    \t1 {} \circelse (classID, methodID) = (TPKClassID, handleAsyncEvent) \circthen {} \\
    \t2 InterpreterNewStackFrame[TPK/class?, handleAsyncEvent/methodID?] \circseq Poll \circseq \\
    \t2 TPK\_handleAsyncEvent \circseq \\
    \t2 (\circvar returnValue : Word \circspot \lschexpract InterpreterReturnEPC \rschexpract \circseq \\
    \t3 executeMethodRet!thread!returnValue \then \Skip) \\
    \t1 {} \circelse (classID, methodID) = (TPKClassID, f) \circthen {} \\
    \t2 InterpreterNewStackFrame[TPK/class?, f/methodID?] \circseq Poll \circseq \\
    \t2 TPK\_f \circseq \\
    \t2 (\circvar returnValue : Word \circspot \lschexpract InterpreterAreturn2EPC \rschexpract \circseq \\
    \t3 executeMethodRet!thread!returnValue \then \Skip) \\
    \t1 {} \cdots {} \\
    \t1 \circfi
  \end{circusaction}
  \caption{$ExecuteMethod$ after refining return instructions}
  \label{efs-ExecutMethod-refined-return-instructions-figure}
\end{figure}

Since the return action at the end of each method in $ExecuteMethod$
causes $frameStack$ to be empty, the application of these
transformations to our running example results in the $ExecuteMethod$
action shown in
Figure~\ref{efs-ExecutMethod-refined-return-instructions-figure}.
Each method action is followed by a $returnValue$ variable block with
a data operation followed by an $executeMethodRet$ communication,
which result from the refinement of the return actions.
While the data operations differ depending on whether a value is
returned from the method or not, the variable block and
$executeMethodRet$ communication are the same for each method.
We thus distribute them outside $ExecuteMethod$, to avoid their
unecessary duplication in each of the branches of $ExecuteMethod$.
This is performed by first replacing $ExecuteMethod$ with its
definition, via an application of Law~[\nameref{copy-rule-law}] on
lines~\ref{algorithm-copy-ExecuteMethod-in-MainThread}
and~\ref{algorithm-copy-ExecuteMethod-in-Started}, then applying
Rule~[\nameref{ExecuteMethod-refinement-rule}], shown in
Figure~\ref{ExecuteMethod-refinement-rule-figure}, on
lines~\ref{algorithm-ExecuteMethod-refinement-MainThread}
and~\ref{algorithm-ExecuteMethod-refinement-Started}, to distribute
the variable block and communication.

\begin{figure}[tbhp!]
  \begin{restatable}[$ExecuteMethod$-refinement]{crule}{ExecuteMethodRefinementRule}
    \label{ExecuteMethod-refinement-rule}
    If, for all $i$, $returnValue$ is not free in $A_i$, then
    \setlength{\zedindent}{0.5cm}
    \setlength{\abovedisplayskip}{0pt}
    \setlength{\belowdisplayskip}{0pt}
    \setlength{\abovedisplayshortskip}{0pt}
    \setlength{\belowdisplayshortskip}{0pt}
  \begin{circus}
    \begin{array}{l}
      (\circval classID : ClassID; \\
      \circval methodID : MethodID; \\
      \circval methodArgs : \seq Word \circspot \\
      \circif {(classID, methodID) = (c_1,m_1)} \circthen {} \\
      \t1 A_1 \circseq \circvar returnValue : Word \circspot  B_1 \circseq \\
      \t1 executeMethodRet!thread!returnValue \\
      \t1 {} \then \Skip \\
      {} \cdots {} \\
      {} \circelse (classID, methodID) = (c_n,m_n) \circthen {} \\
      \t1 A_n \circseq \circvar returnValue : Word \circspot B_n \circseq \\
      \t1 executeMethodRet!thread!returnValue \\
      \t1 {} \then \Skip \\
      \circfi)(c, m, a)
    \end{array}
    \circrefines_A
    \begin{array}{l}
      \circvar retVal : Word \circspot \\
       (\circval classID : ClassID; \\
      \circval methodID : MethodID; \\
      \circval methodArgs : \seq Word; \\
      \circres returnValue : Word \circspot \\
      \circif {(classID, methodID) = (c_1,m_1)} \circthen {} \\
      \t1 A_1 \circseq B_1 \\
      {} \cdots {} \\
      {} \circelse (classID, methodID) = (c_n,m_n) \circthen {} \\
      \t1 A_n \circseq B_n \\
      \circfi)(c, m, a, retVal) \circseq \\
      executeMethodRet!thread!retVal \\
      {} \then \Skip 
    \end{array}
  \end{circus}
\end{restatable}
\caption{Rule~[\nameref{ExecuteMethod-refinement-rule}]}
\label{ExecuteMethod-refinement-rule-figure}
\end{figure}

After Rule~[\nameref{ExecuteMethod-refinement-rule}] has been applied,
$ExecuteMethod$ is redefined as the actions inside the parametrised
block on the left-hand side of that rule.
This is performed using Law~[\nameref{action-intro-law}] on
lines~\ref{algorithm-remove-ExecuteMethod}
to~\ref{algorithm-reintroduce-ExecuteMethod}, eliminating the existing
definition of $ExecuteMethod$ and introducing a new definition.
The actions are then copied back out using the new definition by
application of Law~[\nameref{copy-rule-law}] on
lines~\ref{algorithm-copy-ExecuteMethod-out-MainThread}
and~\ref{algorithm-copy-ExecuteMethod-out-Started}.
This results in the $MainThread$ and $Started$ actions shown in
Figure~\ref{efs-eliminate-returns-MainThread-Started-figure}, where we
have highlighted the parts of those actions that have changed as a
result of applying the rules in this section.
Note that, since these actions have a very specific format, we can be
sure that the application of Law~[\nameref{copy-rule-law}] replaces
only the intended components of those actions.
Within the body of a method, the return actions have been refined to
simple data operations with no $executeMethodRet$ communication, as
can be seen in
Figure~\ref{efs-eliminate-returns-handleAsyncEvent-example-figure},
which shows the form of $TPK\_handleAsyncEvent$, with those data
operations highlighted.

\begin{figure}[tp!]
  \centering
  \setlength{\zedtab}{0.4cm}
  \setlength{\zedindent}{0pt}
  \setlength{\zedleftsep}{0pt}
  \setlength{\abovedisplayskip}{0pt}
  \setlength{\belowdisplayskip}{0pt}
  \setlength{\abovedisplayshortskip}{0pt}
  \setlength{\belowdisplayshortskip}{0pt}
  \begin{circusaction}
    MainThread \circdef \\
    \t1 setStack?t \prefixcolon (t = thread) ?stack \then frameStackID := Initialised~stack \circseq \circmu X \circspot \\
    \t1 \circblockbegin
    \colorbox{lightgray}{$\circvar retVal : Word \circspot$}
    executeMethod? t \prefixcolon (t = thread)?c?m?a \then {} \\
    \t1 \colorbox{lightgray}{$ExecuteMethod(c, m, a, retVal) \circseq
    executeMethodRet!thread!retVal \then Poll$} \circseq  X \\
    {} \extchoice {} \\
    CEEswitchThread?from?to \prefixcolon (from = thread) \then Blocked \circseq X
    \circblockend
  \end{circusaction}
  
  \begin{circusaction}
    Started \circdef \\
    \t1 \circblockbegin
    \colorbox{lightgray}{$\circvar retVal : Word \circspot$} executeMethod? t \prefixcolon (t = thread)?c?m?a \then {} \circseq \\
    \t1 \colorbox{lightgray}{$ExecuteMethod(c, m, a, retVal) \circseq executeMethodRet!thread!retVal \then Poll$} \circseq \\
    \t1 \circblockbegin
    continue?t \prefixcolon (t = thread) \then Started \\
    {} \extchoice {} \\
    endThread?t \prefixcolon (t = thread) \then \Skip
    \circblockend \\
    {} \extchoice {} \\
    CEEswitchThread?from?to \prefixcolon (from = thread) \then Blocked \circseq Started \\
    {} \extchoice {} \\
    endThread?t \prefixcolon (t = thread) \then \Skip
    \circblockend \circseq \\
    \t1 removeThreadMemory!thread \then CEEremoveThread!thread \\
    \t1 {} \then CEEswitchThread?from?to \prefixcolon (from = thread) \then NotStarted
  \end{circusaction}
  \caption{$MainThread$ and $Started$ after $Launcher$ return
    elimination}
  \label{efs-eliminate-returns-MainThread-Started-figure}
\end{figure}

\begin{figure}[tp!]
  \centering
  \setlength{\zedtab}{0.4cm}
  \setlength{\zedindent}{0pt}
  \setlength{\zedleftsep}{0pt}
  \setlength{\abovedisplayskip}{0pt}
  \setlength{\belowdisplayskip}{0pt}
  \setlength{\abovedisplayshortskip}{0pt}
  \setlength{\belowdisplayshortskip}{0pt}
  \begin{circusaction}
    TPK\_handleAsyncEvent \circdef \\
    \t1 HandleNewEPC(27) \circseq Poll \circseq HandleDupEPC \circseq Poll \circseq  HandleAconst\_nullEPC \circseq Poll \circseq \\
    \t1 (\circvar poppedArgs : \seq Word \circspot \\
    \t2 \lschexpract \exists argsToPop? == 2 @ InterpreterStackFrameInvoke \rschexpract \circseq \\
    \t2 \lschexpract InterpreterNewStackFrame[\\
    \t3 ConsoleConnection/class?, CCinit/methodID?, poppedArgs/methodArgs?] \rschexpract) \circseq Poll \circseq \\
    \t1 ConsoleConnection\_CCinit \circseq \colorbox{lightgray}{$\lschexpract InterpreterReturnEPC \rschexpract$} \circseq Poll \circseq \\
    %\t1 HandleAstoreEPC(1) \circseq Poll \circseq HandleAloadEPC(1) \circseq \\
    \t1 {} \cdots {} \\
    % \t1 Poll \circseq (\circvar poppedArgs : \seq Word \circspot \lschexpract \exists argsToPop? == m + 1 @ InterpreterStackFrameInvoke \rschexpract \circseq \\
    % \t1 getClassIDOf!(head~poppedArgs)?cid \then \lschexpract InterpreterNewStackFrame[ \\
    % \t2 ConsoleConnection/class?, openInputStream/methodID?, poppedArgs/methodArgs?] \rschexpract) \circseq \\
    % \t1 Poll \circseq ConsoleConnection\_openInputStream \circseq Poll \circseq  HandleAstoreEPC(2) \circseq Poll \circseq \\
    % \t1 HandleAloadEPC(1) \circseq Poll \circseq (\circvar poppedArgs : \seq Word \circspot \\
    % \t1 \lschexpract \exists argsToPop? == m + 1 @ InterpreterStackFrameInvoke \rschexpract \circseq \\
    % \t1 getClassIDOf!(head~poppedArgs)?cid \then \lschexpract InterpreterNewStackFrame[\\
    % \t2 ConsoleConnection/class?, openOutputStream/methodID?, poppedArgs/methodArgs?] \rschexpract) \circseq \\
    % \t1 Poll \circseq ConsoleConnection\_openOutputStream \circseq Poll \circseq HandleAstoreEPC(3) \circseq \\
    %\t1 {} \cdots {} \\
    % \t1 Poll \circseq HandleIconstEPC(0) \circseq Poll \circseq HandleAstoreEPC(4) \circseq Poll \circseq Poll \circseq \circmu Y \circspot \\
    % \t2 HandleAloadEPC(4) \circseq Poll \circseq HandleIconstEPC(10) \circseq Poll \circseq \\
    % \t2 \circvar value1, value2 : Word \circspot InterpreterPop2 \circseq \\
    % \t2 \circif value1 \leq value2 \circthen \\
    % \t3 {} \cdots {}  \\
    % Poll \circseq HandleAloadEPC(2) \circseq Poll \circseq \\
    % \t3 (\circvar poppedArgs : \seq Word \circspot \\
    % \t4 \lschexpract \exists argsToPop? == m + 1 @ InterpreterStackFrameInvoke \rschexpract \circseq \\
    % \t4 getClassIDOf!(head~poppedArgs)?cid \then \lschexpract InterpreterNewStackFrame[ \\
    % \t5 ConsoleInput/class?, read/methodID?, poppedArgs/methodArgs?] \rschexpract) \circseq \\
    % \t3 Poll \circseq ConsoleInput\_read \circseq Poll \circseq \\
    \t3 (\circvar poppedArgs : \seq Word \circspot \\
    \t4 \lschexpract \exists argsToPop? == 1 @ InterpreterStackFrameInvoke \rschexpract \circseq \\
    \t4 \lschexpract InterpreterNewStackFrame[\\
    \t5 TPK/class?, f/methodID?, poppedArgs/methodArgs?] \rschexpract) \circseq \\
    \t3 Poll \circseq TPK\_f \circseq \colorbox{lightgray}{$\lschexpract InterpreterAreturn1EPC \rschexpract$} \circseq Poll \circseq \\
    % \t3 HandleAstoreEPC(5) \circseq Poll \circseq HandleAloadEPC(5) \circseq \\
    \t3 {} \cdots {} \\
    % \t3 Poll \circseq HandleIconstEPC(400) \circseq Poll \circseq \circvar value1, value2 : Word \circspot InterpreterPop2 \circseq \\
    % \t3 \circif value1 \leq value2 \circthen HandleAloadEPC(3) \circseq Poll \circseq HandleAloadEPC(5) \circseq Poll \circseq \\
    % \t4 (\circvar poppedArgs : \seq Word \circspot \lschexpract \exists argsToPop? == m + 1 @ InterpreterStackFrameInvoke \rschexpract \circseq \\
    % \t4 getClassIDOf!(head~poppedArgs)?cid \then \lschexpract InterpreterNewStackFrame[ \\
    % \t5 ConsoleOutput/class?, write/methodID?, poppedArgs/methodArgs?] \rschexpract)) \circseq \\
    % \t4 Poll \circseq ConsoleOutput\_write \\
    % \t3 {} \circelse value1 > value2 \circthen HandleAloadEPC(3) \circseq Poll \circseq HandleIconstEPC(0) \circseq Poll \circseq \\
    % \t4 (\circvar poppedArgs : \seq Word \circspot \lschexpract \exists argsToPop? == m + 1 @ InterpreterStackFrameInvoke \rschexpract \circseq \\
    % \t4 getClassIDOf!(head~poppedArgs)?cid \then \lschexpract InterpreterNewStackFrame[ \\
    % \t5 ConsoleOutput/class?, write/methodID?, poppedArgs/methodArgs?] \rschexpract)) \circseq \\
    % \t4 Poll \circseq ConsoleOutput\_write \\
    % \t3 \circfi \circseq Poll \circseq HandleAloadEPC(4) \circseq Poll \circseq HandleIconstEPC(1) \circseq Poll \circseq HandleIaddEPC \circseq \\
    \t3 Poll \circseq HandleAstoreEPC(4) \circseq Poll \circseq Y \\
    \t2 {} \circelse value1 > value2 \circthen \Skip \\
    \t2 \circfi \circseq Poll 
  \end{circusaction}
  \caption{$TPK\_handleAsyncEvent$ after $Launcher$ return
    elimination}
  \label{efs-eliminate-returns-handleAsyncEvent-example-figure}
\end{figure}