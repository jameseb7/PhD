The second stage of the compilation strategy eliminates the
$frameStack$ from the state of each thread's process,
$ThrCF_{bc,cs}(cs,t)$. 
The information stored in the stack frames on $frameStack$\deleted{
  are}\added{ is} transferred into variables representing the local
variables and operand stack slots for each method.
The operations representing the bytecode instructions are refined to
operations over these variables.
This refines $ThrCF_{bc,cs}(cs,t)$ to the $CThr_{bc,cs}(t)$ process
described in Section~\ref{cee-c-program-subsection}, so this stage may
be summarised by the following theorem.
%
\begin{thm}[Elimination of Frame Stack]\label{efs-thm}
  \begin{circus}
    ThrCF_{bc,cs}(cs,t) \circrefines CThr_{bc,cs}(t)
  \end{circus}
\end{thm}
%

In this stage, we operate mainly on the method actions introduced in
the previous stage.
Algorithm~\ref{efs-algorithm} describes the strategy for transforming
those actions to introduce variables and eliminate the $frameStack$.
\begin{algorithm}[tp!]
  \begin{algorithmic}[1]
    \State \Call{RemoveLauncherReturns}{}
    \label{algorithm-remove-launcher-returns}
    % \State \Call{RemoveCurrentClassAndFrameStackIDFromState}{}
    % \label{algorithm-remove-currentClass}
    \State \Call{LocaliseStackFrames}{}
    \label{algorithm-localise-stack-frames}
    \State \Call{IntroduceVariables}{}
    \label{algorithm-introduce-variables}
    \State \Call{RemoveFrameStackFromState}{}
    \label{algorithm-remove-frameStack}
  \end{algorithmic}
  \caption{Elimination of Frame Stack}
  \label{efs-algorithm}
\end{algorithm}
It begins on line~\ref{algorithm-remove-launcher-returns}\deleted{,}
by refining the return instructions that occur at the end of each
method action to remove the $CheckLauncherReturn$ actions that occur
in those instructions, resolving the check of whether $frameStack$ is
empty.
This removes the only remaining use of $frameStack$ as a whole,
enabling us to consider the stack frames for each method individually.
We introduce a variable in each method that contains its stack
frame\added{,} on line~\ref{algorithm-localise-stack-frames} of the
algorithm, and convert the operations of the method to operate over
the new variable rather than the global $frameStack$.
Afterwards, on line~\ref{algorithm-introduce-variables}, we perform
local data refinements to convert the stack frame for each method into
variables representing the local variables and operand stack slots of
the method.
Finally, we eliminate the, now unused, $frameStack$ from the state of
the process, on line~\ref{algorithm-remove-frameStack}.

We discuss each of these steps in more detail in separate sections,
explaining them with reference to the running example introduced in
Section~\ref{overview-subsection}.
The removal of launcher returns is discussed first, in
Section~\ref{remove-launcher-returns-subsection}.
Afterwards, the localisation of stack frames is discussed in
Section~\ref{localise-stack-frames-subsection}, followed by variable
introduction in Section~\ref{introduce-variables-subsection}.
Finally, we discuss the removal of $frameStack$ from the state of the
process, in Section~\ref{remove-frameStack-from-state-subsection}.
