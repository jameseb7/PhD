The second stage of the compilation strategy eliminates the
$frameStack$ from the state of each thread's process,
$ThrCF_{bc,cs}(cs,t)$. 
The information stored in the stack frames on $frameStack$ are
transferred into variables representing the local variables and
operand stack slots for each method.
The operations representing the bytecode instructions are refined to
operations over these variables.
This stage may be summarised by the following theorem
%
\begin{thm}
  \begin{circus}
    ThrCF_{bc,cs}(cs,t) \circrefines ThrLV_{bc,cs}(t)
  \end{circus}
\end{thm}
%

In this stage, we operate mainly on the method actions introduced in
the previous stage.
Algorithm~\ref{efs-algorithm} describes the strategy for transforming
the method actions to introduce variables and eliminate the
$frameStack$.
\begin{algorithm}[tp!]
  \begin{algorithmic}[1]
    % do we need to remove currentClass (and frameStackID) first?
    %\State \Call{RemoveCurrentClassFromState}
    %\label{algorithm-remove-currentClass}
    \State \Call{RemoveLauncherReturns}{}
    \label{algorithm-remove-launcher-returns}
    \State \Call{LocaliseStackFrames}{}
    \label{algorithm-localise-stack-frames}
    \State \Call{IntroduceVariables}{}
    \label{algorithm-introduce-variables}
    \State \Call{RemoveFrameStackFromState}{}
    \label{algorithm-remove-frameStack}
  \end{algorithmic}
  \caption{Elimination of Frame Stack}
  \label{efs-algorithm}
\end{algorithm}
It begins on line~\ref{algorithm-remove-launcher-returns}, by refining
the return instructions that occur at the end of each method to remove
the $CheckLauncherReturn$ actions that occur in those instructions,
resolving the check of whether $frameStack$ is empty.
This removes the only remaining use of $frameStack$ as a whole,
enabling us to consider the stack frames for each method individually.
We introduce a variable in each method that contains its stack frame
on line~\ref{algorithm-localise-stack-frames} of the algorithm, and
convert the operations of the method to operate over the new variable
rather than the global $frameStack$.
Then, on line~\ref{algorithm-introduce-variables}, we perform local
data refinements to convert the stack frame for each method into
variables representing the local variables and operand stack slots of
the method.
Finally, we eliminate the, now unused, $frameStack$ from the state of
the process, on line~\ref{algorithm-remove-frameStack}.

Next, we discuss each of these steps in more detail, explaining them
with reference to the running example introduced in
Section~\ref{overview-subsection}.
