When a method is complete, calls to that method can then be resolved.
This is performed after introduction of loops and conditionals,
ensuring methods with loops and conditionals are complete so that this
step can be applied.
% As mentioned previously, since this requires all the method calls in a
% given method to be resolved first, we do not allow recursion.

This step begins with the copying of the method into a separate
action, so that it can be referenced elsewhere.
This is performed by as described by
Algorithm~\ref{separate-complete-methods-algorithm}.
\begin{algorithm}
  \begin{algorithmic}[1]
    \Procedure{SeparateCompleteMethods}{}
    \For{$m \gets methods$} \label{algorithm-method-separation-loop}
    % \For{$mc \gets$ \Call{MethodsCalls}{$m$}}
    % \EndFor
    \If{\Call{MethodIsComplete}{$m$}} \label{algorithm-check-method-completeness}
    \State \Call{ApplyCopyRule}{$m$}
    \EndIf
    \EndFor
    \EndProcedure
  \end{algorithmic}
  \caption{Separate Complete Methods}
  \label{separate-complete-methods-algorithm}
\end{algorithm}

Algorithm~\ref{separate-complete-methods-algorithm} looks at each
method separately, as specified by the loop on
line~\ref{algorithm-method-separation-loop}, and determines if it is
complete, on line~\ref{algorithm-check-method-completeness}.
This involves a simple syntactic check that each conditional branch
ends in a return instruction or a recursion.
Those methods that are complete are moved into a separate action by an
application of the copy rule.

In our example, the method \texttt{f()} of the \texttt{TPK} class,
which starts at $pc = 43$, is complete on the first iteration of the
loop on line~\ref{algorithm-method-loop} of
Algorithm~\ref{epc-algorithm}, with the $Running$ action as shown
below.
The method is complete in this case because it consists of a straight
sequence of instructions ending with $HandleAreturnEPC$, which
represents the \texttt{areturn} instruction.
\begin{circus}
  Running \circdef \\
  \t1 \circif frameStack = \emptyset \circthen \Skip \\
  \t1 {} \circelse frameStack \neq \emptyset \circthen {} \\
  \t2 \circif pc = 0 \circthen {} \cdots {} \\
  \t2 {} \cdots {} \\
  \t2 {} \circelse pc = 43 \circthen HandleAloadEPC(0) \circseq Poll \circseq HandleAloadEPC(0) \circseq \\
  \t3 Poll \circseq HandleIaddEPC \circseq Poll \circseq HandleAloadEPC(0) \circseq Poll \circseq \\
  \t3 HandleIaddEPC \circseq Poll \circseq HandleIconstEPC(5) \circseq Poll \circseq \\
  \t3 HandleIaddEPC \circseq Poll \circseq HandleAreturnEPC \\
  \t2 {} \cdots {} \\
  \t2 \circfi \circseq Poll \circseq Running \\
  \t1 \circfi
\end{circus}
The sequence of instructions at $pc = 43$ can then be copied into a
separate action, shown below.
The name of this action contains the name of the class and method
identifier of the method it represents.
\begin{circus}
  TPK\_f \circdef HandleAloadEPC(0) \circseq Poll \circseq HandleAloadEPC(0) \circseq Poll \circseq \\
  \t1 HandleIaddEPC \circseq Poll \circseq HandleAloadEPC(0) \circseq Poll \circseq HandleIaddEPC \circseq \\
  \t1 Poll \circseq HandleIconstEPC(5) \circseq Poll \circseq  HandleIaddEPC \circseq Poll \circseq \\
  \t1 HandleAreturnEPC 
\end{circus}

After all the complete methods have been copied into separate actions,
calls to those methods are resolved.
This is performed as described by
Algorithm~\ref{resolve-method-calls-algorithm}.
\begin{algorithm}
  \begin{algorithmic}[1]
    \Procedure{ResolveMethodCalls}{}
    \For{$m \gets methods$}
    \For{$mc \gets$ \Call{UnresolvedMethodsCalls}{$m$}}
    \State $targets \gets$ \Call{DetermineMethodCallTargets}{$mc$}
    \If{$\# targets = 1$}
    \State \Call{ApplyRule[\nameref{method-call-resolution-rule}]}{}
    \Else
    \State \Call{ApplyRule[\nameref{dynamic-method-call-resolution-rule}]}{}
    \EndIf
    \EndFor
    \EndFor
    \EndProcedure
  \end{algorithmic}
  \caption{Resolve Method Calls}
  \label{resolve-method-calls-algorithm}
\end{algorithm}


%%%%

To resolve a method call, the type of method call must be considered.
Some methods are handled directly by the SCJVM as they relate to the
SCJVM services.
Such methods are treated specially by the interpreter, communicating
with the launcher to perform the behaviour of the method.
In cases where the method invocation is simply handled by
communication with the launcher and then followed by execution of the
next instruction, the control flow can be introduced using
Rule [\nameref{sequence-introduction-rule}] as for other instructions with
simple control flow.

% TODO: discuss special methods with nested calls here

For method calls that do not require special handling, the control
flow is that of the corresponding method's action, followed by
execution of the next instruction.
If the method is called with static dispatch (as is the case with the
\texttt{invokespecial} and \texttt{invokestatic} instructions), the
correct method, and hence the corresponding action can be easily
determined.
The method call is then resolved using
Rule [\nameref{method-call-resolution-rule}].
We require as a precondition of the rule that the method action
returns to the return address stored on the stack, to ensure that the
control flow is resolved correctly.
This will be the case for a method where every path of execution ends
in a return or loop, since returns establish the condition and loops
will either lead to a return eventually or form an infinite loop that
may be followed by any action (including the assumption we require).
%
\begin{restatable}[Method call resolution]{crule}{MethodCallResolutionRule}
  \label{method-call-resolution-rule}
  If an action $M$ is such that
  \[\{ (head~frameStack).storedPC = i \land frameStack = fs \} \circseq M \\
    {} = {} \\
    \{ (head~frameStack).storedPC = i \land frameStack = fs \}
    \circseq M \circseq \\
    \t1 \{ pc = i \land frameStack = tail~fs \}\]
  and $i \neq j$,
  \setlength{\zedindent}{0.25cm}
  \begin{circus}
    \begin{array}{l}
      \circmu X \circspot \\
      \t1 \circif frameStack = \emptyset \circthen \Skip \\
      \t1 {} \circelse frameStack \neq \emptyset \circthen {} \\
      \t2 \circif \cdots \\
      \t2 {} \circelse pc = i \circthen A \circseq \\
      \t3 \{ pc = k \} \circseq \\
      \t3 \lschexpract \exists retAddr? == pc+1 @ \\
      \t4 SetReturnAddr \rschexpract \circseq \\
      \t3 pc := j \\
      \t2 {} \circelse pc = k+1 \circthen B \\
      \t2 {} \circelse pc = j \circthen M \\
      \t2 \cdots \\
      \t2 \circfi \circseq Poll \circseq X \\
      \t1 \circfi 
    \end{array}
    \circrefines_A
    \begin{array}{l}
      \circmu X \circspot \\
      \t1 \circif frameStack = \emptyset \circthen \Skip \\
      \t1 {} \circelse frameStack \neq \emptyset \circthen {} \\
      \t2 \circif \cdots \\
      \t2 {} \circelse pc = i \circthen A \circseq \\
      \t3 \{ pc = k \} \circseq \\
      \t3 \lschexpract \exists retAddr? == pc+1 @ \\
      \t4 SetReturnAddr \rschexpract \circseq \\
      \t3 pc := j \circseq Poll \circseq M \circseq Poll \circseq B \\
      \t2 {} \circelse pc = k+1 \circthen B \\
      \t2 {} \circelse pc = j \circthen M \\
      \t2 \cdots \\
      \t2 \circfi \circseq Poll \circseq X \\
      \t1 \circfi 
    \end{array}
  \end{circus}
  %TODO: elimination of pc assignment
\end{restatable}
%
For method calls that have dynamic dispatch (as in the case of
\texttt{invokevirtual} instructions), we must determine what classes
the method may be called on.
The control flow is then a choice over the class of the object the
method is called on, with the method action corresponding to that
class chosen.
As with the static case, we require the method action to return to the
return address stored on the stack.
In the dynamic case, this requirement must be met by all the method
actions so that the next instruction can be executed after the choice
of methods.
Rule [\nameref{dynamic-method-call-resolution-rule}] is used to
introduce this choice.
%
\begin{restatable}[Dynamic method call resolution]{crule}{DynamicMethodCallResolutionRule}
  \label{dynamic-method-call-resolution-rule}
  If actions $M_1, \dots, M_n$ are such that
  \[\{ returnAddress = i \land frameStack = fs \} \circseq M_k \\
    {} = {} \\
    \{ returnAddress = i \land frameStack = fs \}
    \circseq M_k \circseq \{ pc = i \land frameStack = tail~fs \}\]
   for $k in \{ 1, \dots, n \}$ and $i \neq j$,
  \setlength{\zedindent}{0.25cm}
  \begin{circus}
    \begin{array}{l}
      \circmu X \circspot \\
      \t1 \circif frameStack = \emptyset \circthen \Skip \\
      \t1 {} \circelse frameStack \neq \emptyset \circthen {} \\
      \t2 \circif \cdots \\
      \t2 {} \circelse pc = i \circthen A(class, method) \circseq \\
      \t3 \{ class \in \{ c_1, \dots, c_n \}\} \circseq \\
      \t3 \lschexpract \exists retAddr? == pc+1 @ \\
      \t4 SetReturnAddr \rschexpract \circseq \\
      \t3 pc := entry(class, method) \\
      \t2 {} \circelse pc = k+1 \circthen B \\
      \t2 {} \circelse pc = j_1 \circthen M_1 \\
      \t2 \cdots \\
      \t2 {} \circelse pc = j_n \circthen M_n \\
      \t2 \circfi \circseq Poll \circseq X \\
      \t1 \circfi 
    \end{array}
    \circrefines_A
    \begin{array}{l}
      \circmu X \circspot \\
      \t1 \circif frameStack = \emptyset \circthen \Skip \\
      \t1 {} \circelse frameStack \neq \emptyset \circthen {} \\
      \t2 \circif \cdots \\
      \t2 {} \circelse pc = i \circthen A(class, method) \circseq \\
      \t3 \{ class \in \{ c_1, \dots, c_n \} \circseq \\
      \t3 \lschexpract \exists retAddr? == pc+1 @ \\
      \t4 SetReturnAddr \rschexpract \circseq \\
      \t3 pc := entry(class, method) \circseq Poll \circseq \\
      \t3 \circif class = c_1 \circthen M_1 \\
      \t3 \cdots \\
      \t3 {} \circelse class = c_n \circthen M_n \\
      \t3 \circfi \circseq Poll \circseq B \\
      \t2 {} \circelse pc = k+1 \circthen B \\
      \t2 {} \circelse pc = j_1 \circthen M_1 \\
      \t2 \cdots \\
      \t2 {} \circelse pc = j_n \circthen M_n \\
      \t2 \circfi \circseq Poll \circseq X \\
      \t1 \circfi
    \end{array}
  \end{circus}
  %TODO: elimination of pc assignment
\end{restatable}

The resolution of loops, conditionals and methods is performed in a
loop until all the methods have been separated into their own action.
The remaining use of the program counter in the main actions of $Thr$
can then be eliminated as described in the next section.