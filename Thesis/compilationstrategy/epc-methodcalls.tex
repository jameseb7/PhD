When a method is complete, calls to that method can then be resolved.
This is performed after introduction of loops and conditionals,
ensuring methods with loops and conditionals are complete so that this
step can be applied.
% As mentioned previously, since this requires all the method calls in a
% given method to be resolved first, we do not allow recursion.

This step begins with the copying of the method into a separate
action, so that it can be referenced elsewhere.
This is performed by as described by
Algorithm~\ref{separate-complete-methods-algorithm}.
\begin{algorithm}
  \begin{algorithmic}[1]
    \For{$m \gets methods$} \label{algorithm-method-separation-loop}
    \If{\Call{MethodIsComplete}{$m$}} \label{algorithm-check-method-completeness}
    \State \ApplyFor{Law~[\nameref{action-intro-law}]}{$m$} \label{algorithm-introduce-method-action}
    \State \ApplyFor{Law~[\nameref{copy-rule-law}]}{$m$} \label{algorithm-copy-method-action}
    \EndIf
    \EndFor
  \end{algorithmic}
  \caption{SeparateCompleteMethods}
  \label{separate-complete-methods-algorithm}
\end{algorithm}

Algorithm~\ref{separate-complete-methods-algorithm} looks at each
method separately, as specified by the loop on
line~\ref{algorithm-method-separation-loop}, and determines if it is
complete, on line~\ref{algorithm-check-method-completeness}.
This involves a simple syntactic check that each conditional branch
ends in a return instruction or a recursion.

% TODO: move a predicate concerning class information across?

For the methods that are complete, the sequence of actions for the
method are placed in a separate action, which is introduced using
Law~[\nameref{action-intro-law}] on
line~\ref{algorithm-introduce-method-action}.
We form the name of this action from the name of the class to which
the method belongs and the name of the method, concatenated together
with an underscore.
The name used for the action does not have an effect upon the
correctness of the strategy, provided it is unique.
Once the method action has been introduced, the sequence of actions at
the method's entry point in the $Running$ action is replaced with a
reference to the newly introduced action by applying
Law~[\nameref{copy-rule-law}] on
line~\ref{algorithm-copy-method-action}.

In our example, the method \texttt{f()} of the \texttt{TPK} class,
which starts at $pc = 43$, is complete on the first iteration of the
loop on line~\ref{algorithm-method-loop} of
Algorithm~\ref{epc-algorithm}.
This can be seen in
Figure~\ref{forward-sequence-introduction-example-figure}.
The method is complete in this case because it consists of a straight
sequence of instructions ending with $HandleAreturnEPC$, which
represents the \texttt{areturn} instruction.
The sequence of instructions at $pc = 43$ is copied into the action $TPK\_f$,
which can be seen in
Figure~\ref{method-call-resolution-example-figure}.
The $pc = 43$ branch is replaced with a call to $TPK\_f$.

After all the complete methods have been copied into separate actions,
calls to those methods are resolved.
This is performed as described by
Algorithm~\ref{resolve-method-calls-algorithm}.
\begin{algorithm}
  \begin{algorithmic}[1]
    \For{$m \gets methods$}
    \For{$mc \gets$ \Call{UnresolvedMethodsCalls}{$m$}}
    \If{\Call{HasMultipleTargets}{$mc$}}
    \State \ApplyFor{Rule~[\nameref{refine-invokevirtual-multi-rule}]}{$mc$}
    \For{$target \gets $\Call{Targets}{$mc$}}
    \Try
    \State \ApplyFor{Rule~[\nameref{resolve-special-method-branch-rule}]}{$mc$}
    \State \ApplyFor{Rule~[\nameref{resolve-normal-method-branch-rule}]}{$mc$}
    \EndTry
    \EndFor
    \State \ApplyFor{Law~[\nameref{alt-seq-dist-law}]}{$mc$}
    \Else
    \Try
    \State \ApplyFor{Rule~[\nameref{refine-invokespecial-rule}]}{$mc$}
    \State \ApplyFor{Rule~[\nameref{refine-invokestatic-rule}]}{$mc$}
    \State \ApplyFor{Rule~[\nameref{refine-invokevirtual-single-rule}]}{$mc$}
    \EndTry
    \Try
    \State \ApplyFor{Rule~[\nameref{resolve-special-method-rule}]}{$mc$}
    \State \ApplyFor{Rule~[\nameref{resolve-normal-method-rule}]}{$mc$}
    \EndTry
    \EndIf
    \If{\Call{HasSimpleSequence}{$node$}}
    \label{algorithm-lci-sequence-check}
    \State \ApplyFor{Rule~[\nameref{sequence-introduction-rule}]}{$node$}
    \label{algorithm-lci-sequence-introduction}
    \EndIf
    \EndFor
    \EndFor
  \end{algorithmic}
  \caption{ResolveMethodCalls}
  \label{resolve-method-calls-algorithm}
\end{algorithm}

\begin{restatable}[refine-invokespecial]{crule}{RefineInvokespecialRule}
  \label{refine-invokespecial-rule}
  Given $i : ProgramAddress$ and $cs : ClassID \pfun Class$,
  \setlength{\zedindent}{0.25cm}
  \begin{circus}
    \begin{array}{l}
      \circif \cdots \\
      {} \circelse pc = i \circthen \\
      \t1 A \circseq pc := j \circseq Poll \circseq \\
      \t1 HandleInvokespecialEPC(cpi) \\
      \cdots \\
      \circfi
    \end{array}
    \circrefines_A
    \begin{array}{l}
      \circif \cdots \\
      {} \circelse pc = i \circthen \\
      \t1 A \circseq pc := j \circseq Poll \circseq \\
      \t1 \circvar poppedArgs : \seq Word \circspot \\
      \t1 \lschexpract \exists argsToPop? == methodArguments~m + 1 @ \\
      \t2 InterpreterStackFrameInvoke \rschexpract \circseq \\
      \t1 Invoke(c, m, poppedArgs, \false) \\
      \cdots \\
      \circfi
    \end{array}
  \end{circus}
  where $m : MethodID$ and $c : ClassID$ are such that
  \begin{displaymath}
    \exists c_0 : Class; m_0 : MethodID | c_0 \in \ran cs \land m_0 \in \dom c_0.methodEntry @ \\
    \t1 cpi \in methodRefIndices~c_0 \land \\
    \t1 c_0.constantPool~cpi = MethodRef~(c,m) \land \\
    \t1 j \in c_0.methodEntry~m_0 \upto c_0.methodEnd~m_0.
  \end{displaymath}
\end{restatable}

\begin{restatable}[refine-invokestatic]{crule}{RefineInvokestaticRule}
  \label{refine-invokestatic-rule}
  Given $i : ProgramAddress$ and $cs : ClassID \pfun Class$,
  \setlength{\zedindent}{0.25cm}
  \begin{circus}
    \begin{array}{l}
      \circif \cdots \\
      {} \circelse pc = i \circthen \\
      \t1 A \circseq pc := j \circseq Poll \circseq \\
      \t1 HandleInvokespecialEPC(cpi) \\
      \cdots \\
      \circfi
    \end{array}
    \circrefines_A
    \begin{array}{l}
      \circif \cdots \\
      {} \circelse pc = i \circthen \\
      \t1 A \circseq pc := j \circseq Poll \circseq \\
      \t1 \circvar poppedArgs : \seq Word \circspot \\
      \t1 \lschexpract \exists argsToPop? == methodArguments~m @ \\
      \t2 InterpreterStackFrameInvoke \rschexpract \circseq \\
      \t1 Invoke(c, m, poppedArgs, \true) \\
      \cdots \\
      \circfi
    \end{array}
  \end{circus}
  where $m : MethodID$ and $c : ClassID$ are such that
  \begin{displaymath}
    \exists c_0 : Class; m_0 : MethodID | c_0 \in \ran cs \land m_0 \in \dom c_0.methodEntry @ \\
    \t1 cpi \in methodRefIndices~c_0 \land \\
    \t1 c_0.constantPool~cpi = MethodRef~(c,m) \land \\
    \t1 j \in c_0.methodEntry~m_0 \upto c_0.methodEnd~m_0.
  \end{displaymath}
\end{restatable}

\begin{restatable}[refine-invokevirtual-single]{crule}{RefineInvokeVirtualSingleRule}
  \label{refine-invokevirtual-single-rule}
  Given $i : ProgramAddress$ and $cs : ClassID \pfun Class$,
  \setlength{\zedindent}{0.25cm}
  \begin{circus}
    \begin{array}{l}
      \circif \cdots \\
      {} \circelse pc = i \circthen \\
      \t1 A \circseq pc := j \circseq Poll \circseq \\
      \t1 HandleInvokevirtualEPC(cpi) \\
      \cdots \\
      \circfi
    \end{array}
    \circrefines_A
    \begin{array}{l}
      \circif \cdots \\
      {} \circelse pc = i \circthen \\
      \t1 A \circseq pc := j \circseq Poll \circseq \\
      \t1 \circvar poppedArgs : \seq Word \circspot \\
      \t1 \lschexpract \exists argsToPop? == methodArguments~m @ \\
      \t2 InterpreterStackFrameInvoke \rschexpract \circseq \\
      \t1 Invoke(c, m, poppedArgs, \false) \\
      \cdots \\
      \circfi
    \end{array}
  \end{circus}
  where $m : MethodID$ and $c : ClassID$ are such that
  \begin{displaymath}
    \exists c_0 : Class; m_0 : MethodID | c_0 \in \ran cs \land m_0 \in \dom c_0.methodEntry @ \\
    \t1 cpi \in methodRefIndices~c_0 \land \\
    \t1 c_0.constantPool~cpi = MethodRef~(c,m) \land \\
    \t1 j \in c_0.methodEntry~m_0 \upto c_0.methodEnd~m_0,
  \end{displaymath}
  and provided $\exists_1 c_1 : ClassID @ (c_1,c) \in subclassRel~cs$.
\end{restatable}

\begin{restatable}[refine-invokevirtual-multi]{crule}{RefineInvokeVirtualMultiRule}
  \label{refine-invokevirtual-multi-rule}
  Given $i : ProgramAddress$ and $cs : ClassID \pfun Class$, if
  \begin{itemize}
  \item $c_0 \in \ran cs$,
  \item $m_0 \in \dom c_0.methodEntry $
  \item $j \in c_0.methodEntry~m_0 \upto c_0.methodEnd~m_0$,
  \item $cpi \in methodRefIndices~c_0$,
  \item $c_0.constantPool~cpi = MethodRef~(c,m)$, and
  \item $\{ x | (x,c) \in subclassRel~cs \} = \{c_1, \ldots , c_n\}$, then
  \end{itemize}
  \setlength{\zedindent}{0.25cm}
  \begin{circus}
    \begin{array}{l}
      \circif \cdots \\
      {} \circelse pc = i \circthen \\
      \t1 A \circseq pc := j \circseq Poll \circseq \\
      \t1 HandleInvokevirtualEPC(cpi) \\
      \cdots \\
      \circfi
    \end{array}
    \circrefines_A
    \begin{array}{l}
      \circif \cdots \\
      {} \circelse pc = i \circthen \\
      \t1 A \circseq pc := j \circseq Poll \circseq \\
      \t1 \circvar poppedArgs : \seq Word \circspot \\
      \t1 \lschexpract \exists argsToPop? == methodArguments~m @ \\
      \t2 InterpreterStackFrameInvoke \rschexpract \circseq \\
      \t1 getClassIDOf!(head~poppedArgs)?cid \then {} \\
      \t1 \circif cid = c_1 \circthen Invoke(c_1, m, poppedArgs, \false) \\
      \t1 {} \cdots {} \\
      \t1 {} \circelse cid = c_n \circthen Invoke(c_n, m, poppedArgs, \false) \\
      \t1 \circfi \\
      \cdots \\
      \circfi
    \end{array}
  \end{circus}
  where $m : MethodID$ and $c_1, \ldots, c_n : ClassID$ are such that
  \begin{displaymath}
    \exists c_0 : Class; m_0 : MethodID | c_0 \in \ran cs \land m_0 \in \dom c_0.methodEntry @ \\
    \t1 cpi \in methodRefIndices~c_0 \land \\
    \t1 j \in c_0.methodEntry~m_0 \upto c_0.methodEnd~m_0 \land \\
    \t1 \exists c : ClassID @ c_0.constantPool~cpi = MethodRef~(c,m) \land \\
    \t1 \{ x : ClassID | (x,c) \in subclassRel~cs \} = \{c_1, \ldots , c_n\}
  \end{displaymath}
  and provided $n > 1$.
\end{restatable}

\begin{restatable}[resolve-special-method]{crule}{ResolveSpecialMethodRule}
  \label{resolve-special-method-rule}
  If
  \setlength{\zedindent}{0.25cm}
  \setlength{\zedtab}{0.5cm}
  \begin{circus}
    \begin{array}{l}
      pc := j \circseq Poll \circseq \\
      \circvar poppedArgs : \seq Word \circspot \\
      \lschexpract \exists argsToPop? == e @ \\
      \t1 InterpreterStackFrameInvoke \rschexpract \circseq \\
      Invoke(c, m, poppedArgs, static)
    \end{array}
    \circrefines_A
    \begin{array}{l}
      pc := j \circseq Poll \circseq \\
      \circvar poppedArgs : \seq Word \circspot \\
      \lschexpract \exists argsToPop? == e @ \\
      \t1 InterpreterStackFrameInvoke \rschexpract \circseq \\
      specialMethodAction(c, m, static) \circseq \\
      pc := j + 1
      \end{array}
  \end{circus}
\end{restatable}

\begin{table}
  \centering
  \setlength{\abovedisplayskip}{0pt}
  \setlength{\belowdisplayskip}{0pt}
  \setlength{\abovedisplayshortskip}{0pt}
  \setlength{\belowdisplayshortskip}{0pt}
  \setlength{\zedindent}{0cm}
  \renewcommand{\arraystretch}{1}
  \begin{tabular}{p{4.5cm}p{8cm}}
    \hline
    Conditions on $c$, $m$ and $static$ & $specialMethodAction(c, m, static)$ \\
    \hline
    \begin{circus}
      (c,resumeThreadClass) \\
      \t1 {} \in subclassRel~cs \\
      \land method = resumeThreadID \\
      \land static = \true
    \end{circus} &
                   \begin{circus}
                     resumeThread!(WordToThreadID~(methodArgs~1)) \\
                     {} \then resumeThreadRet \then \Skip
                   \end{circus}\\
    \begin{circus}
      (c,suspendClass) \\
      \t1 {} \in subclassRel~cs \\
      \land method = suspendID \\
      \land static = \true
    \end{circus} &
                   \begin{circus}
                     suspend \then suspendRet \then \Skip
                   \end{circus}\\
    \begin{circus}
      (c,writeClass) \\
      \t1 {} \in subclassRel~cs \\
      \land method = writeID \\
      \land static = \true
    \end{circus} &
                   \begin{circus}
                     output!(methodArgs~1) \then \Skip
                   \end{circus}\\
    \begin{circus}
      (c,readClass) \\
      \t1 {} \in subclassRel~cs \\
      \land method = readID \\
      \land static = \true
    \end{circus} &
                   \begin{circus}
                     input?value \then \lschexpract InterpreterPush \hide (pc,pc') \rschexpract
                   \end{circus}\\
    \hline

  \end{tabular}
  \caption{The syntactic function $specialMethodAction(c, m, static)$}
\end{table}

\begin{restatable}[resolve-nested-special-method]{crule}{ResolveNestedSpecialMethodRule}
  \label{resolve-nested-special-method-rule}
  \setlength{\zedindent}{0.25cm}
  \setlength{\zedtab}{0.45cm}
  \begin{circus}
    \begin{array}{l}
      \circmu X \circspot \\
      \t1 \circif frameStack = \emptyset \circthen \Skip \\
      \t1 {} \circelse frameStack \neq \emptyset \circthen \circmu Y \circspot \\
      \t2 \circif last~frameStack = \emptyset \circthen \Skip \\
      \t2 {} \circelse last~frameStack \neq \emptyset \circthen {} \\
      \t3 \circif \cdots \\
      \t3 {} \circelse pc = i \circthen A \circseq pc := j \circseq Poll \circseq \\
      \t4 \circvar poppedArgs : \seq Word \circspot \\
      \t4 \lschexpract \exists argsToPop? == e @ \\
      \t5 InterpreterStackFrameInvoke \rschexpract \circseq \\
      \t4 Invoke(c, m, poppedArgs, static) \\
      \t3 {} \cdots {} \\
      \t3 \circfi \circseq Poll \circseq Y \\
      \t2 \circfi \circseq \lschexpract PopSubFrameStack \rschexpract \circseq X \\
      \t1 \circif
    \end{array}
    \circrefines_A
    \begin{array}{l}
      \circmu X \circspot \\
      \t1 \circif frameStack = \emptyset \circthen \Skip \\
      \t1 {} \circelse frameStack \neq \emptyset \circthen \circmu Y \circspot \\
      \t2 \circif last~frameStack = \emptyset \circthen \Skip \\
      \t2 {} \circelse last~frameStack \neq \emptyset \circthen {} \\
      \t3 \circif \cdots \\
      \t3 {} \circelse pc = i \circthen A \circseq pc := j \circseq Poll \circseq \\
      \t4 \circvar poppedArgs : \seq Word \circspot \\
      \t4 \lschexpract \exists argsToPop? == e @ \\
      \t5 InterpreterStackFrameInvoke \rschexpract \circseq \\
      \t4 specialMethodAction2(c, m, static) \circseq \\
      \t4 StartInterpreter \circseq Poll \circseq Y \circseq \\
      \t4 \lschexpract PopSubFrameStack \rschexpract \circseq \\
      \t4 pc := j + 1 \\
      \t3 {} \cdots {} \\
      \t3 \circfi \circseq Poll \circseq Y \\
      \t2 \circfi \circseq \lschexpract PopSubFrameStack \rschexpract \circseq X \\
      \t1 \circif
    \end{array}
  \end{circus}
\end{restatable}

\begin{table}[t]
  \centering
  \setlength{\abovedisplayskip}{0pt}
  \setlength{\belowdisplayskip}{0pt}
  \setlength{\abovedisplayshortskip}{0pt}
  \setlength{\belowdisplayshortskip}{0pt}
  \setlength{\zedindent}{0cm}
  \renewcommand{\arraystretch}{1}
  \begin{tabular}{p{5.5cm}p{9cm}}
    \hline
    Conditions on $c$, $m$ and $static$ & $specialMethodAction2(c, m, static)$ \\
    \hline
    \begin{circus}
      (c,managedSchedulableClass) \\
      \t1 {} \in subclassRel~cs \\
      \land method = registerID \\
      \land static = \false
    \end{circus} &
                   \begin{circus}
                     register!thread!(head~methodArgs) \\
                     {} \then registerRet \then \Skip
                   \end{circus}\\
    \begin{circus}
      (c,managedMemoryClass) \\
      \t1 {} \in subclassRel~cs \\
      \land method = enterPrivateMemoryID \\
      \land static = \true
    \end{circus} &
                   \begin{circus}
                     enterPrivateMemory!thread!(methodArgs~1)!(methodArgs~2) \\
                     {} \then \Skip 
                   \end{circus}\\
    \begin{circus}
      (c,managedMemoryClass) \\
      \t1 {} \in subclassRel~cs \\
      \land method = executeInAreaOfID \\
      \land static = \true
    \end{circus} &
                   \begin{circus}
                     executeInAreaOf!thread!(methodArgs~1)!(methodArgs~2) \\
                     {} \then \Skip 
                   \end{circus}\\
    \begin{circus}
      (c,managedMemoryClass) \\
      \t1 {} \in subclassRel~cs \\
      \land method = executeInOuterAreaID \\
      \land static = \true
    \end{circus} &
                   \begin{circus}
                     executeInOuterArea!thread!(methodArgs~1) \\
                     {} \then \Skip 
                   \end{circus}\\
    \begin{circus}
      (c,managedMemoryClass) \\
      \t1 {} \in subclassRel~cs \\
      \land method = \\
      \t1 enterPerReleaseMemoryID \\
      \land static = \true
    \end{circus} &
                   \begin{circus}
                     enterPerReleaseMemory!thread!(methodArgs~1) \\
                     {} \then \Skip 
                   \end{circus}\\
    \hline
  \end{tabular}
  \caption{The syntactic function $specialMethodAction2(c, m, static)$}
\end{table}

\begin{restatable}[resolve-normal-method]{crule}{ResolveNormalMethodRule}
  \label{resolve-normal-method-rule}
  Given $i : ProgramAddress$ and $cs : ClassID \pfun Class$,
  if
  \begin{circus}
    \{ methodID? = m \land class? = c \land cs? = cs \} \circseq \lschexpract ResolveMethod \rschexpract \\
    {} = {} \\
    \{ methodID? = m \land class? = c \land cs? = cs \} \circseq \lschexpract ResolveMethod \rschexpract \circseq
    \{ class?.methodEntry~m = k \},
  \end{circus}
  and
  \begin{circus}
    \{ (last~(front~fs)).storedPC = k \land frameStack = fs \} \circseq M \\
    {} = {} \\
    \{ (last~(front~fs)).storedPC = k \land frameStack = fs \} \circseq M \circseq \{ pc = k \land frameStack = front~fs \},
  \end{circus}
  then,
  \setlength{\zedindent}{0.25cm}
  \begin{circus}
    \begin{array}{l}
      \circmu X \circspot \\
      \t1 \circif frameStack = \emptyset \circthen \Skip \\
      \t1 {} \circelse frameStack \neq \emptyset \circthen {} \\
      \t2 \circif \cdots \\
      \t2 {} \circelse pc = i \circthen A \circseq pc := j \circseq Poll \circseq \\
      \t3 \circvar poppedArgs : \seq Word \circspot \\
      \t3 \lschexpract \exists argsToPop? == e @ \\
      \t4 InterpreterStackFrameInvoke \rschexpract \circseq \\
      \t3 Invoke(c, m, poppedArgs, static) \\
      \t2 {} \circelse pc = k \circthen M \\
      \t2 \cdots \\
      \t2 \circfi \circseq Poll \circseq X \\
      \t1 \circfi 
    \end{array}
    \circrefines_A
    \begin{array}{l}
      \circmu X \circspot \\
      \t1 \circif frameStack = \emptyset \circthen \Skip \\
      \t1 {} \circelse frameStack \neq \emptyset \circthen {} \\
      \t2 \circif \cdots \\
      \t2 {} \circelse pc = i \circthen A \circseq pc := j \circseq Poll \circseq \\
      \t3 \circvar poppedArgs : \seq Word \circspot \\
      \t3 \lschexpract \exists argsToPop? == e @ \\
      \t4 InterpreterStackFrameInvoke \rschexpract \circseq \\
      \t3 InvokeOther(c, m, poppedArgs, static) \circseq \\
      \t3 Poll \circseq M \circseq pc := j + 1 \\
      \t2 {} \circelse pc = k \circthen M \\
      \t2 \cdots \\
      \t2 \circfi \circseq Poll \circseq X \\
      \t1 \circfi 
    \end{array}
  \end{circus}
\end{restatable}

%%%%

To resolve a method call, the type of method call must be considered.
Some methods are handled directly by the SCJVM as they relate to the
SCJVM services.
Such methods are treated specially by the interpreter, communicating
with the launcher to perform the behaviour of the method.
In cases where the method invocation is simply handled by
communication with the launcher and then followed by execution of the
next instruction, the control flow can be introduced using
Rule [\nameref{sequence-introduction-rule}] as for other instructions with
simple control flow.

% TODO: discuss special methods with nested calls here

For method calls that do not require special handling, the control
flow is that of the corresponding method's action, followed by
execution of the next instruction.
If the method is called with static dispatch (as is the case with the
\texttt{invokespecial} and \texttt{invokestatic} instructions), the
correct method, and hence the corresponding action can be easily
determined.
The method call is then resolved using
Rule [\nameref{method-call-resolution-rule}].
We require as a precondition of the rule that the method action
returns to the return address stored on the stack, to ensure that the
control flow is resolved correctly.
This will be the case for a method where every path of execution ends
in a return or loop, since returns establish the condition and loops
will either lead to a return eventually or form an infinite loop that
may be followed by any action (including the assumption we require).
%
\begin{restatable}[Method call resolution]{crule}{MethodCallResolutionRule}
  \label{method-call-resolution-rule}
  If an action $M$ is such that
  \[\{ (head~frameStack).storedPC = i \land frameStack = fs \} \circseq M \\
    {} = {} \\
    \{ (head~frameStack).storedPC = i \land frameStack = fs \}
    \circseq M \circseq \\
    \t1 \{ pc = i \land frameStack = tail~fs \}\]
  and $i \neq j$,
  \setlength{\zedindent}{0.25cm}
  \begin{circus}
    \begin{array}{l}
      \circmu X \circspot \\
      \t1 \circif frameStack = \emptyset \circthen \Skip \\
      \t1 {} \circelse frameStack \neq \emptyset \circthen {} \\
      \t2 \circif \cdots \\
      \t2 {} \circelse pc = i \circthen A \circseq \\
      \t3 \{ pc = k \} \circseq \\
      \t3 \lschexpract \exists retAddr? == pc+1 @ \\
      \t4 SetReturnAddr \rschexpract \circseq \\
      \t3 pc := j \\
      \t2 {} \circelse pc = k+1 \circthen B \\
      \t2 {} \circelse pc = j \circthen M \\
      \t2 \cdots \\
      \t2 \circfi \circseq Poll \circseq X \\
      \t1 \circfi 
    \end{array}
    \circrefines_A
    \begin{array}{l}
      \circmu X \circspot \\
      \t1 \circif frameStack = \emptyset \circthen \Skip \\
      \t1 {} \circelse frameStack \neq \emptyset \circthen {} \\
      \t2 \circif \cdots \\
      \t2 {} \circelse pc = i \circthen A \circseq \\
      \t3 \{ pc = k \} \circseq \\
      \t3 \lschexpract \exists retAddr? == pc+1 @ \\
      \t4 SetReturnAddr \rschexpract \circseq \\
      \t3 pc := j \circseq Poll \circseq M \circseq Poll \circseq B \\
      \t2 {} \circelse pc = k+1 \circthen B \\
      \t2 {} \circelse pc = j \circthen M \\
      \t2 \cdots \\
      \t2 \circfi \circseq Poll \circseq X \\
      \t1 \circfi 
    \end{array}
  \end{circus}
  %TODO: elimination of pc assignment
\end{restatable}
%
For method calls that have dynamic dispatch (as in the case of
\texttt{invokevirtual} instructions), we must determine what classes
the method may be called on.
The control flow is then a choice over the class of the object the
method is called on, with the method action corresponding to that
class chosen.
As with the static case, we require the method action to return to the
return address stored on the stack.
In the dynamic case, this requirement must be met by all the method
actions so that the next instruction can be executed after the choice
of methods.
Rule [\nameref{dynamic-method-call-resolution-rule}] is used to
introduce this choice.
%
\begin{restatable}[Dynamic method call resolution]{crule}{DynamicMethodCallResolutionRule}
  \label{dynamic-method-call-resolution-rule}
  If actions $M_1, \dots, M_n$ are such that
  \[\{ returnAddress = i \land frameStack = fs \} \circseq M_k \\
    {} = {} \\
    \{ returnAddress = i \land frameStack = fs \}
    \circseq M_k \circseq \{ pc = i \land frameStack = tail~fs \}\]
   for $k in \{ 1, \dots, n \}$ and $i \neq j$,
  \setlength{\zedindent}{0.25cm}
  \begin{circus}
    \begin{array}{l}
      \circmu X \circspot \\
      \t1 \circif frameStack = \emptyset \circthen \Skip \\
      \t1 {} \circelse frameStack \neq \emptyset \circthen {} \\
      \t2 \circif \cdots \\
      \t2 {} \circelse pc = i \circthen A(class, method) \circseq \\
      \t3 \{ class \in \{ c_1, \dots, c_n \}\} \circseq \\
      \t3 \lschexpract \exists retAddr? == pc+1 @ \\
      \t4 SetReturnAddr \rschexpract \circseq \\
      \t3 pc := entry(class, method) \\
      \t2 {} \circelse pc = k+1 \circthen B \\
      \t2 {} \circelse pc = j_1 \circthen M_1 \\
      \t2 \cdots \\
      \t2 {} \circelse pc = j_n \circthen M_n \\
      \t2 \circfi \circseq Poll \circseq X \\
      \t1 \circfi 
    \end{array}
    \circrefines_A
    \begin{array}{l}
      \circmu X \circspot \\
      \t1 \circif frameStack = \emptyset \circthen \Skip \\
      \t1 {} \circelse frameStack \neq \emptyset \circthen {} \\
      \t2 \circif \cdots \\
      \t2 {} \circelse pc = i \circthen A(class, method) \circseq \\
      \t3 \{ class \in \{ c_1, \dots, c_n \} \circseq \\
      \t3 \lschexpract \exists retAddr? == pc+1 @ \\
      \t4 SetReturnAddr \rschexpract \circseq \\
      \t3 pc := entry(class, method) \circseq Poll \circseq \\
      \t3 \circif class = c_1 \circthen M_1 \\
      \t3 \cdots \\
      \t3 {} \circelse class = c_n \circthen M_n \\
      \t3 \circfi \circseq Poll \circseq B \\
      \t2 {} \circelse pc = k+1 \circthen B \\
      \t2 {} \circelse pc = j_1 \circthen M_1 \\
      \t2 \cdots \\
      \t2 {} \circelse pc = j_n \circthen M_n \\
      \t2 \circfi \circseq Poll \circseq X \\
      \t1 \circfi
    \end{array}
  \end{circus}
  %TODO: elimination of pc assignment
\end{restatable}

The resolution of loops, conditionals and methods is performed in a
loop until all the methods have been separated into their own action.
The remaining use of the program counter in the main actions of $Thr$
can then be eliminated as described in the next section.