Before the control flow can be introduced, the bytecode instructions
provided in the $bc$ parameter to $Thr$ must be expanded to allow
consideration of their semantics.
This is achieved using the procedure shown in
Algorithm~\ref{expand-bytecode-algorithm}.
\begin{algorithm}[thp]
  \begin{algorithmic}[1]
    \State \ApplyFor{Rule~[\nameref{pc-expansion-rule}]}{$bc$}
    \label{algorithm-introduce-choice-over-pc}
    \State \ApplyFor{Law~[\nameref{action-intro-law}]}{$Handle{*}EPC$ actions}
    \label{algorithm-introduce-handleEPC-actions}
    \For{$i \gets \dom bc$}
    \label{algorithm-expand-bytecode-loop}
    \State \ApplyFor{Rule~[\nameref{HandleInstruction-refinement-rule}]}{$bc$, $i$}
    \label{algorithm-HandleInstruction-refinement}
    \EndFor
  \end{algorithmic}
  \caption{ExpandBytecode}
  \label{expand-bytecode-algorithm}
\end{algorithm}
It begins on line~\ref{algorithm-introduce-choice-over-pc} by applying
Rule~[\nameref{pc-expansion-rule}], shown in
Figure~\ref{pc-expansion-rule-figure}.
It introduces a choice over all the possible values of $pc$ in the
domain of $bc$ at the $HandleInstruction$ action in $Running$.
This does not affect the behaviour of $HandleInstruction$, because it
behaves as $\Chaos$ when $pc$ is outside the domain of $bc$.
We write $HandleInstruction$ with a $bc$ subscript to indicate that it
makes use of $bc$, which is a parameter of the $Thr$ process in which
$HandleInstruction$ occurs.
\begin{figure}[thp]
\begin{restatable}[$pc$-expansion]{crule}{PCExpansionRule}
  \label{pc-expansion-rule}
  Given $bc : ProgramAddress \pfun Bytecode$,
  \begin{circus}
    HandleInstruction_{bc} = \circif {} \circelse_{i\in\dom bc} pc = i \then HandleInstruction_{bc} \circfi
  \end{circus}
\end{restatable}
\caption{Rule~[\nameref{pc-expansion-rule}]}
\label{pc-expansion-rule-figure}
\end{figure}
The proof of this rule and others can be found in
Appendix~\ref{compilation-rules-proofs-appendix}.
After applying Rule~[\nameref{pc-expansion-rule}], we operate on the
occurrence of $HandleInstruction$ at each branch of the conditional at
line~\ref{algorithm-expand-bytecode-loop}.
We apply Rule~[\nameref{HandleInstruction-refinement-rule}], shown in
Figure~\ref{HandleInstruction-refinement-rule-figure}, on
line~\ref{algorithm-HandleInstruction-refinement} to refine each
occurrence to a more specific form that is easier to operate on during
the rest of the strategy.
These new actions are determined from the bytecode instruction in $bc$
at each $pc$ value by applying a syntactic function $handleAction$,
which is defined by Table~\ref{handle-action-table}.
\begin{figure}[thp]
\begin{restatable}[$HandleInstruction$-refinement]{crule}{HandleInstructionRefinementRule}
  \label{HandleInstruction-refinement-rule}
  Given $i : ProgramAddress$, if $i \in \dom bc$ then,
  \begin{circus}
    \begin{array}{l}
      \circif {} \cdots {} \\
      {} \circelse pc = i \then HandleInstruction_{bc} \\
      {} \cdots {} \\
      \circfi
    \end{array}
    \circrefines_A
    \begin{array}{l}
      \circif {} \cdots {} \\
      {} \circelse pc = i \then handleAction(bc~i) \\
      {} \cdots {} \\
      \circfi
    \end{array}
  \end{circus}
  where $handleAction$ is a syntactic function defined by
  Table~\ref{handle-action-table}.
\end{restatable}
\caption{Rule~[\nameref{HandleInstruction-refinement-rule}]}
\label{HandleInstruction-refinement-rule-figure}
\end{figure}
\begin{table}
  \centering
  \begin{tabular}{lp{8.5cm}}
    \hline
    Bytecode ($bc~i$) & Action ($handleAction(bc~i)$) \\
    \hline
    $aconst\_null$ & $HandleAconst\_nullEPC \circseq pc := i+1$ \\
    $dup$ & $HandleDupEPC \circseq pc := i+1$ \\
    $aload~lvi$ & $HandleAloadEPC(lvi) \circseq pc := i+1$ \\
    $astore~lvi$ & $HandleAstoreEPC(lvi) \circseq pc := i+1$ \\
    $iadd$ & $HandleIaddEPC \circseq pc := i+1$ \\
    $iconst~n$ & $HandleIconstEPC(n) \circseq pc := i+1$ \\
    $ineg$ & $HandleInegEPC \circseq pc := i+1$ \\
    $goto~ofst$ & $pc := i+ofst$ \\
    $if\_icmple~ofst$ & $\circvar value1, value2 : Word \circspot$ \endgraf
                        \t1 $\lschexpract InterpreterPopEPC \rschexpract \circseq$ \endgraf
                        \t1 $\lschexpract InterpreterPopEPC \rschexpract \circseq$ \endgraf
                         \t1 $pc := \IF value1 \leq value2 \THEN i+ofst \ELSE i+1$ \\
    $areturn$ & $HandleAreturnEPC$ \\
    $return$ & $HandleReturnEPC$ \\
    $new~cpi$ & $HandleNewEPC(cpi) \circseq pc := i+1$ \\
    $getfield~cpi$ & $HandleGetfieldEPC(cpi) \circseq pc := i+1$ \\
    $putfield~cpi$ & $HandlePutfieldEPC(cpi) \circseq pc := i+1$ \\
    $getstatic~cpi$ & $HandleGetstaticEPC(cpi) \circseq pc := i+1$ \\
    $putstatic~cpi$ & $HandlePutstaticEPC(cpi) \circseq pc := i+1$ \\
    $invokevirtual~cpi$ & $\{pc = i\} \circseq HandleInvokevirtualEPC(cpi)$ \\
    $invokespecial~cpi$ & $\{pc = i\} \circseq HandleInvokespecialEPC(cpi)$ \\
    $invokestatic~cpi$ & $\{pc = i\} \circseq HandleInvokestaticEPC(cpi)$ \\
    \hline
  \end{tabular}
  \caption{The syntactic function $handleAction$}
  \label{handle-action-table}
\end{table}
The actions generated by $handleAction$ use new actions for handling
the individual bytecode instructions.
These are similar to the actions used to define $HandleInstruction$
(e.g.\ $HandleDup$, $HandleAload$ etc.), which we refer to as
$Handle*$ actions.
We name the new actions used by $handleAction$ by appending $EPC$ to
the names of the $Handle*$ actions they are based on, and we refer to
them as $Handle{*}EPC$ actions.
The $Handle{*}EPC$ actions are introduced on
line~\ref{algorithm-introduce-handleEPC-actions} of
Algorithm~\ref{expand-bytecode-algorithm}, before the application of
Rule~[\nameref{HandleInstruction-refinement-rule}], by application of
Law~[\nameref{action-intro-law}], which introduces unused actions to
processes.

In addtion to the $Handle{*}EPC$ actions, the actions output from
$handleAction$ also include $pc$ updates extracted from the $Handle*$
actions.
The output from $handleAction$ for the $goto$ and $if\_icmple$
instructions consists solely of a $pc$ update with no $Handle{*}EPC$
actions, since updating the value of pc is the main effect of those
instructions.

The differences between the $Handle{*}EPC$ actions and the $Handle*$
actions on which they are based are explained using the $HandleAstore$
action as an example.
We recall that it is defined as shown below.
\begin{circusaction}
  HandleAstore \circdef \lcircguard pc \in \dom bc \land bc~pc \in \ran astore \rcircguard \circguard \\
  \t1 \circvar variableIndex : \nat \circspot variableIndex := (astore\inv)~(bc~pc) \circseq \lschexpract InterpreterAstore \rschexpract
\end{circusaction}
Its corresponding $Handle{*}EPC$ action, $HandleAstoreEPC$, is shown
below.
\begin{circusaction}
  HandleAstoreEPC \circdef \circval variableIndex : \nat \circspot \lschexpract InterpreterAstoreEPC \rschexpract
\end{circusaction}
The first difference of $HandleAstoreEPC$ from $HandleAstore$ is that
it is not guarded by the condition on the value of $bc$ at the current
$pc$ value.
The choice that such guards mediate is collapsed by
Rule~[\nameref{HandleInstruction-refinement-rule}], since the value of
$bc$ at a given $pc$ value is determined by the supplied $bc$
parameter of $Thr$.

The second difference is that the parameters of the bytecode
instructions are transferred to become parameters of the
$Handle{*}EPC$ actions, so $HandleAstoreEPC$ has a $variableIndex$
parameter.
This corresponds to the $variableIndex$ variable in $HandleAstore$,
which is used to store the value extracted from the $astore$
instruction.
This transformation is, of course, not performed for instructions that
do not take parameters.
This transformation is standard in the context of a call to a
parametrised action.

Finally, the schema $InterpreterAstore$ is replaced with a schema
$InterpreterAstoreEPC$, which does not affect $pc$, since
Rule~[\nameref{HandleInstruction-refinement-rule}] extracts the
updates to $pc$ from the $Handle{*}$ actions.
The $pc$ updates are not removed in the case of the actions for
handling method invocation and return, where the $pc$ updates are
closely connected to the operations on the stack and require special
handling.
Instead, an assumption on the value of $pc$ is introduced for the
method invocation handling actions, since the $pc$ information is used
in setting the return address.
We discuss how we operate on the method invocation and return handling
actions in Section~\ref{resolve-method-calls-subsection}.

At the end of Algorithm~\ref{expand-bytecode-algorithm}, our example
has the form shown earlier in
Figure~\ref{bytecode-expansion-example-figure}.
After the bytecode semantics is expanded in the $Running$ action by
this step, the control flow that corresponds to each $pc$ update can
be introduced.
This is dicussed in the next section.