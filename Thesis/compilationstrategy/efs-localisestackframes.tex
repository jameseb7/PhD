After the $CheckLauncherReturn$ actions have been handled, the process
no longer has any actions that use the whole $frameStack$.
We can therefore refine each method to only operate on a local stack
frame variable.
This is performed as described in
Algorithm~\ref{localise-stack-frames-algorithm}, which defines the
procedure \Call{LocaliseStackFrames}{}.

\begin{algorithm}[t]
  \begin{algorithmic}[1]
    \arraycolsep=0cm
    \State \ApplyFor{Law~[\nameref{forwards-data-refinement-law}]}{$InterpreterStateFS$, $FrameStackCI$}
    \label{algorithm-remove-currentClass-data-refinement}
    \State $iterationOrder \gets$ \Call{MethodDependencyOrder}{}
    \label{algorithm-method-dependency-order-call}
    \For{$methodName \gets iterationOrder$}
    \label{algorithm-localise-stack-frames-loop}
    \State $numArgs \gets$ \Call{MethodArguments}{$methodName$}
    \label{algorithm-numArgs-declaration}
    \If{$\lnot$ \Call{IsStatic}{$methodName$}}
    \label{algorithm-static-args-check-start}
    \State $numArgs \gets numArgs + 1$
    \EndIf
    \label{algorithm-static-args-check-end}
    \State \ExhaustivelyApplyFor{Rule~[\nameref{arguments-introduction-rule}]}{$methodName$, $numArgs$}
    \label{algorithm-arguments-introduction}
    \State \Call{RedefineMethodToIncludeParameters}{$methodName$}
    \label{algorithm-redefine-method-action-to-include-parameters}
    \State \ApplyFor{Rule~[\nameref{HandleReturnEPC-stackFrame-introduction-rule}]}{$numArgs$}
    \label{algorithm-HandleReturnEPC-stackFrame-introduction-rule}
    \Statex $\t2${\bf to} {\Call{ActionBody}{$methodName$}}
    \EndFor
  \end{algorithmic}
  \caption{\Call{LocaliseStackFrames}{}}
  \label{localise-stack-frames-algorithm}
\end{algorithm}
 
We first apply a data refinement to remove $currentClass$ from the
state.
We have defined $currentClass$ in the model as a convenience when
accessing the $frameClass$ of the topmost stack frame, which is no
longer necessary when we have separate variables for each stack frame.
The data refinement is applied on
line~\ref{algorithm-remove-currentClass-data-refinement} of
Algorithm~\ref{localise-stack-frames-algorithm}, and transforms the
state to $InterpreterStateFS$, below, which only contains
$frameStack$.
\begin{schema}{InterpreterStateFS}
  frameStack : \seq StackFrameEPC
\end{schema}

The relationship between $InterpreterStateEPC$ and
$InterpreterStateFS$ is described by the coupling invariant
$FrameStackCI$, shown below.
It ensures $frameStack$ is unaffected by the refinement and replaces
occurrences of $currentClass$ with $(last~frameStack).frameClass$.
\begin{schema}{FrameStackCI}
  InterpreterStateEPC \\
  InterpreterStateFS_1
\where
  frameStack = frameStack_1 \\
  currentClass = (last~frameStack_1).frameClass
\end{schema}

$FrameStackCI$ describes a functional data refinement, so the new
actions can be calculated in each case.
The effect of this data refinement is, as mentioned above, that
$currentClass$ is replaced with $(last~frameStack).frameClass$,
wherever it occurs in the old actions.
We can then proceed with introducing stack frame variables.

When introducing these variables, we must begin with those methods at
the greatest call depth.
This is necessary due to the approach we take in introducing these
variables, as we explain later in this section.
We therefore introduce stack frame variables to the methods in the
order specified by a procedure \Call{MethodDependencyOrder}{}, which
constructs a sequence of method action names indicating the order in
which the method actions should be handled.
This sequence is constructed by first adding to the sequence any
methods that contain no method calls, then adding any methods that
only call methods already in the sequence, and repeating until all
methods are in the sequence.
Since we do not allow recursion, this always terminates.
We construct this sequence and assign it to $iterationOrder$ on
line~\ref{algorithm-method-dependency-order-call} of
Algorithm~\ref{localise-stack-frames-algorithm}.

We then loop, introducing a stack frame variable for each method in
the order specified by $iterationOrder$, in the for loop on
line~\ref{algorithm-localise-stack-frames-loop}.
Within the for loop, we first introduce value parameters, representing
the arguments to the method, around the call to the method action.
This ensures that the body of the method is completely independent of
the context in which it is called, enabling us to separate the whole
method body (including stack frame creation and return actions) into
its own action.
Introduction of method arguments is performed using
Rule~[\nameref{arguments-introduction-rule}], shown in
Figure~\ref{arguments-introduction-rule-figure}.
This rule is applied to two parameters:~$methodName$, the name of the
method being considered, and $numArgs$, the number of arguments to the
method.
The number of arguments is determined from the name of the method
(since the arguments of a method are encoded in its identifier) and we
add an extra argument for \texttt{this} if the method is not static,
as indicated by if statement on
lines~\ref{algorithm-static-args-check-start}
to~\ref{algorithm-static-args-check-end}.
We apply this rule everywhere it applies on
line~\ref{algorithm-HandleReturnEPC-stackFrame-introduction-rule}.

\begin{figure}[thp]
\begin{restatable}[$InterpreterReturn$-args-intro]{crule}{ArgumentsIntroductionRule}
  \label{arguments-introduction-rule}
  %\setlength{\zedtab}{0.4cm}
  %\setlength{\zedindent}{0pt}
  %\setlength{\zedleftsep}{0pt}
  Given an action name $M$ and $n : \nat$, if
  $arg1$, \ldots, $arg{<}n{>}$ are not free in $M$ and are distinct from
  $c$, $m$ and $args$, then
  \begin{circus}
    \begin{array}{l}
      \lschexpract InterpreterNewStackFrame[ \\
      \t1 c/class?, \\
      \t1 m/methodID?, \\
      \t1 args/methodArgs?] \rschexpract \circseq \\
      Poll \circseq M \circseq \lschexpract InterpreterReturn \rschexpract
    \end{array}
    \circrefines_A
    \begin{array}{l}
      (\circval arg1, \ldots, arg{<}n{>} : Word \circspot \\
      \t1 \lschexpract \exists methodArgs? == \langle arg1, \ldots, arg{<}n{>} \rangle @ \\
      \t2 InterpreterNewStackFrame[ \\
      \t3 c/class?, \\
      \t3 m/methodID?] \rschexpract \circseq \\
      \t1 Poll \circseq M \circseq \lschexpract InterpreterReturn \rschexpract \\
      )(args~1, \ldots, args~n)
    \end{array}
  \end{circus}
\end{restatable}
\caption{Rule~[\nameref{arguments-introduction-rule}]}
\label{arguments-introduction-rule-figure}
\end{figure}

Rule~[\nameref{arguments-introduction-rule}] introduces value
parameters representing method arguments around a method ending with
an $InterpreterReturn$ operation.
The arguments array passed as the $methodArgs$ input of
$InterpreterNewStackFrame$ is split into its individual elements,
which are passed to the parameters and then recombined to be passed into
$InterpreterNewStackFrame$.
This splitting of the array ensures that the individual arguments can
be more easily handled in the next step, where we introduce local
variables.

For example, every call to the method action $TPK\_handleAsyncEvent$
is preceded by an $InterpreterNewStackFrame$ operation and followed by
an $InterpreterReturnEPC$ operation.
The Rule~[\nameref{arguments-introduction-rule}] therefore applies to
it and the number of arguments given as the $n$ parameter to the rule
is $1$, since the SCJ method \texttt{handleAsyncEvent()} takes no
explicit arguments, but is not static.
Calls to $TPK\_handleAsyncEvent$ (the only one of which occurs in
$ExecuteMethod$, since it is only called by the infrastructure), then
have the form shown below.
\begin{circus}
  (\circval arg1 : Word \circspot \\
      \t1 \lschexpract \exists methodArgs? == \langle arg1 \rangle @ \\
      \t2 InterpreterNewStackFrame[TPK/class?, handleAsyncEvent/methodID?] \rschexpract \circseq \\
      \t2 TPK\_handleAsyncEvent \circseq \lschexpract InterpreterReturn \rschexpract \\
      )(methodArgs~1)
\end{circus}

After the method arguments have been introduced, we redefine the
method action to include the contents of the parametrised block
introduced by Rule~[\nameref{arguments-introduction-rule}].
This is performed in a separate procedure,
\Call{RedefineMethodToIncludeParameters}{}, which is similar to
the \Call{RedefineMethodExcludingReturn}{} procedure used
in the previous section.
It is defined by
Algorithm~\ref{redefine-method-action-to-include-parameters-algorithm}
in Appendix~\ref{localise-stack-frames-appendix-subsection}.

Having completely separated the method into its own, independent,
action, we then introduce the stack frame variable for the method
using Rule~[\nameref{HandleReturnEPC-stackFrame-introduction-rule}],
shown in
Figure~\ref{HandleReturnEPC-stackFrame-introduction-rule-figure}.
This is applied to the body of $methodName$ on
line~\ref{algorithm-HandleReturnEPC-stackFrame-introduction-rule},
with the number of arguments, $numArgs$, passed to it.

\begin{figure}[thp]
\begin{restatable}[$InterpreterReturn$-$stackFrame$-intro]{crule}{HandleReturnEPCStackFrameIntroductionRule}
  \label{HandleReturnEPC-stackFrame-introduction-rule}
  \setlength{\zedtab}{0.4cm}
  \setlength{\zedindent}{0.5cm}
  %\setlength{\zedleftsep}{0pt}
  Given $n : \nat$, if the only occurrences of $frameStack$ in $A$ are
  in the expression $last~frameStack$ and the length of $frameStack$
  does not change throughout $A$, then
  \begin{circus}
    \begin{array}{l}
      \lschexpract \exists methodArgs? == \langle arg1, \ldots, arg{<}n{>} \rangle @ \\
      \t2 InterpreterNewStackFrame[ \\
      \t3 c/class?, \\
      \t3 m/methodID? \rschexpract \circseq \\
      A \circseq \lschexpract InterpreterReturn \rschexpract
    \end{array}
    \circrefines_A
    \begin{array}{l}
      \circvar stackFrame : StackFrameEPC \circspot \\
      \t1 \lschexpract Init{<}c{>}\_{<}m{>}SF \rschexpract \circseq \\
      \t1 A[stackFrame/last~frameStack, \\
      \t2 stackFrame'/last~frameStack']
    \end{array}
  \end{circus}
  where $Init{<}c{>}\_{<}m{>}SF$ is defined by
  \begin{schema}{Init{<}c{>}\_{<}m{>}SF}
    arg1?, \ldots, arg{<}n{>}? : Word \\
    stackFrame' : StackFrameEPC
  \where
    \langle arg1?, \ldots, arg{<}n{>}? \rangle \subseteq stackFrame'.localVariables \\
    \# stackFrame'.localVariables = c.methodLocals~m \\
    stackFrame'.operandStack = \langle\rangle \\
    stackFrame'.frameClass = c \\
    stackFrame'.stackSize = c.methodStackSize~m
  \end{schema}
\end{restatable}
\caption{Rule~[\nameref{HandleReturnEPC-stackFrame-introduction-rule}]}
\label{HandleReturnEPC-stackFrame-introduction-rule-figure}
\end{figure}

Rule~[\nameref{HandleReturnEPC-stackFrame-introduction-rule}]
introduces a variable $stackFrame$, which has type $StackFrameEPC$,
over the body of a method that ends with an $InterpreterReturn$
operation.
The $stackFrame$ variable is initialised, in $Init{<}c{>}\_{<}m{>}SF$,
in the same way as for the stack frame created by
$InterpreterNewStackFrame$, and each reference to $last~frameStack$ in
the body of the method is replaced with a reference to $stackFrame$.
Replacing the references to $last~frameStack$ requires that the size
of $frameStack$ does not change during the method.
However, this requirement is met since method calls are the only
operation that changes the size of $frameStack$ and we replace
references to the $frameStack$ in nested methods first, by the
definition of $iterationOrder$.

Iterating over methods beginning with the greatest call depth ensures
that the requirements of
Rule~[\nameref{HandleReturnEPC-stackFrame-introduction-rule}] are met.
Otherwise, each nested method call would have its own
$InterpreterNewStackFrame$ operation, as can be seen in
Figure~\ref{efs-eliminate-returns-handleAsyncEvent-example-figure},
where the call to $TPK\_f$ has such an operation.
This changes the size of $frameStack$, making references to
$last~frameStack$ refer to a different stack frame and so preventing a
direct replacement with the $stackFrame$ variable.
Ensuring the stack frame variable is introduced for $TPK\_f$ first
avoids this issue, since it means that references to $frameStack$
(including the operation to change its size) are already replaced with
references to a separate $stackFrame$ variable by the time we apply
Rule~[\nameref{HandleReturnEPC-stackFrame-introduction-rule}] to
$TPK\_handleAsyncEvent$.

Note that the operations performed on
lines~\ref{algorithm-arguments-introduction}
and~\ref{algorithm-HandleReturnEPC-stackFrame-introduction-rule} of
Algorithm~\ref{localise-stack-frames-algorithm} specifically handle
methods that do not return a value.
We omit the similar handling of methods that do return a value. 
Handling such methods requires rules similar to
Rule~[\nameref{arguments-introduction-rule}] for method bodies
followed by $InterpreterAreturn1$ and $InterpreterAreturn2$, which
introduce a result parameter for the method in addition to the value
parameters representing the method's arguments.
The new method action then has to match the different method
parameters.
We also require a rule similar
Rule~[\nameref{HandleReturnEPC-stackFrame-introduction-rule}] to
handle the slightly different ending of the method action that the
return handling creates.
These rules are applied in a way similar to the existing rules.

In our example, the body of $TPK\_handleAsyncEvent$ (having the form
of the actions in the parametrised block shown above), begins with an
$InterpreterNewStackFrame$ operation having $TPK$ as its $class?$ and
$handleAsyncEvent$ as its $methodID?$.
This operation is thus replaced with an $InitTPK\_handleAsyncEventSF$
operation, of the form shown in
Rule~[\nameref{HandleReturnEPC-stackFrame-introduction-rule}].
% The values for $\ell$ and $s$ can thus be obtained from the
% $methodLocals$ and $methodStackSize$ components of the $TPK$ class
% structure shown in Figure~\ref{example-model-figure}.
% This gives values of $6$ for $\ell$ and $3$ for $s$.
After this rule has been applied, $TPK\_handleAsyncEvent$ is as shown
in Figure~\ref{efs-localise-stack-frames-example-figure}.

For brevity, we define new actions, which we refer to as $Handle*SF$
actions.
These are not formally introduced as actions in the compilation
strategy as they are an abbreviation used for presenting examples and
stating compilation rules.
They are refined to a different form later in the elimination of frame
stack stage.
The $Handle*SF$ actions are similar to the $Handle*EPC$ actions,
except they have every reference to $last~frameStack$ (or
$last~frameStack'$) replaced with a reference to $stackFrame$ (or
$stackFrame'$), and have undergone the data refinement described
above.
We name them by replacing $EPC$ in the names of the $Handle*EPC$
actions with $SF$.
Similarly, we define an $InvokeSF$ operation that performs the
operation of $InterpreterStackFrameInvoke$ over $stackFrame$ instead
of $last~frameStack$.

% TODO: fix this for algorithm changes
\begin{figure}[tp!]
  \centering
  \setlength{\zedtab}{0.4cm}
  \setlength{\zedindent}{0pt}
  \setlength{\zedleftsep}{0pt}
  \setlength{\abovedisplayskip}{0pt}
  \setlength{\belowdisplayskip}{0pt}
  \setlength{\abovedisplayshortskip}{0pt}
  \setlength{\belowdisplayshortskip}{0pt}
  \begin{circusaction}
    TPK\_f \circdef \\
    \t1 \circval arg1 : Word \circspot \\
    \t1 \circvar stackFrame : StackFrameEPC \circspot \\
    \t1 \lschexpract InitTPK\_handleAsyncEventSF \rschexpract \circseq \\
    \t1 Poll \circseq HandleNewSF(27) \circseq Poll \circseq HandleDupSF \circseq Poll \circseq  HandleAconst\_nullSF \circseq Poll \circseq \\
    \t1 (\circvar poppedArgs : \seq Word \circspot \\
    \t2 \lschexpract \exists argsToPop? == 2 @ InvokeSF \rschexpract \circseq \\
    \t2 ConsoleConnection\_CCinit(poppedArgs~1, poppedArgs~2)) \circseq Poll \circseq \\
    \t1 HandleAstoreSF(1) \circseq Poll \circseq HandleAloadSF(1) \circseq \\
    \t1 {} \cdots {} \\
    % \t1 Poll \circseq (\circvar poppedArgs : \seq Word \circspot \lschexpract \exists argsToPop? == m + 1 @ InterpreterStackFrameInvoke \rschexpract \circseq \\
    % \t1 getClassIDOf!(head~poppedArgs)?cid \then \lschexpract InterpreterNewStackFrame[ \\
    % \t2 ConsoleConnection/class?, openInputStream/methodID?, poppedArgs/methodArgs?] \rschexpract) \circseq \\
    % \t1 Poll \circseq ConsoleConnection\_openInputStream \circseq Poll \circseq  HandleAstoreEPC(2) \circseq Poll \circseq \\
    % \t1 HandleAloadEPC(1) \circseq Poll \circseq (\circvar poppedArgs : \seq Word \circspot \\
    % \t1 \lschexpract \exists argsToPop? == m + 1 @ InterpreterStackFrameInvoke \rschexpract \circseq \\
    % \t1 getClassIDOf!(head~poppedArgs)?cid \then \lschexpract InterpreterNewStackFrame[\\
    % \t2 ConsoleConnection/class?, openOutputStream/methodID?, poppedArgs/methodArgs?] \rschexpract) \circseq \\
    % \t1 Poll \circseq ConsoleConnection\_openOutputStream \circseq Poll \circseq HandleAstoreEPC(3) \circseq \\
    %\t1 {} \cdots {} \\
    \t1 Poll \circseq HandleIconstSF(0) \circseq Poll \circseq HandleAstoreSF(4) \circseq Poll \circseq Poll \circseq \circmu Y \circspot \\
    \t2 HandleAloadSF(4) \circseq Poll \circseq HandleIconstSF(10) \circseq Poll \circseq \\
    \t2 \circvar value1, value2 : Word \circspot \\
    \t3 \lschexpract InterpreterPop2[stackFrame/last~frameStack,stackFrame'/last~frameStack'] \rschexpract \circseq \\
    \t2 \circif value1 \leq value2 \circthen {} \\
    \t3 {} \cdots {}  \\
    % Poll \circseq HandleAloadEPC(2) \circseq Poll \circseq \\
    % \t3 (\circvar poppedArgs : \seq Word \circspot \\
    % \t4 \lschexpract \exists argsToPop? == m + 1 @ InterpreterStackFrameInvoke \rschexpract \circseq \\
    % \t4 getClassIDOf!(head~poppedArgs)?cid \then \lschexpract InterpreterNewStackFrame[ \\
    % \t5 ConsoleInput/class?, read/methodID?, poppedArgs/methodArgs?] \rschexpract) \circseq \\
    % \t3 Poll \circseq ConsoleInput\_read \circseq Poll \circseq \\
    % \t3 (\circvar poppedArgs : \seq Word \circspot \\
    % \t4 \lschexpract \exists argsToPop? == 1 @ InvokeSF \rschexpract \circseq \\
    % \t4 TPK\_f()) \circseq \lschexpract InterpreterAreturn1 \rschexpract \circseq Poll \circseq \\
    % \t3 HandleAstoreEPC(5) \circseq Poll \circseq HandleAloadEPC(5) \circseq \\
    % \t3 {} \cdots {} \\
    % \t3 Poll \circseq HandleIconstEPC(400) \circseq Poll \circseq \circvar value1, value2 : Word \circspot InterpreterPop2 \circseq \\
    % \t3 \circif value1 \leq value2 \circthen HandleAloadEPC(3) \circseq Poll \circseq HandleAloadEPC(5) \circseq Poll \circseq \\
    % \t4 (\circvar poppedArgs : \seq Word \circspot \lschexpract \exists argsToPop? == m + 1 @ InterpreterStackFrameInvoke \rschexpract \circseq \\
    % \t4 getClassIDOf!(head~poppedArgs)?cid \then \lschexpract InterpreterNewStackFrame[ \\
    % \t5 ConsoleOutput/class?, write/methodID?, poppedArgs/methodArgs?] \rschexpract)) \circseq \\
    % \t4 Poll \circseq ConsoleOutput\_write \\
    % \t3 {} \circelse value1 > value2 \circthen HandleAloadEPC(3) \circseq Poll \circseq HandleIconstEPC(0) \circseq Poll \circseq \\
    % \t4 (\circvar poppedArgs : \seq Word \circspot \lschexpract \exists argsToPop? == m + 1 @ InterpreterStackFrameInvoke \rschexpract \circseq \\
    % \t4 getClassIDOf!(head~poppedArgs)?cid \then \lschexpract InterpreterNewStackFrame[ \\
    % \t5 ConsoleOutput/class?, write/methodID?, poppedArgs/methodArgs?] \rschexpract)) \circseq \\
    % \t4 Poll \circseq ConsoleOutput\_write \\
    % \t3 \circfi \circseq Poll \circseq HandleAloadEPC(4) \circseq Poll \circseq HandleIconstEPC(1) \circseq Poll \circseq HandleIaddEPC \circseq \\
    \t3 Poll \circseq HandleAstoreSF(4) \circseq Poll \circseq Y \\
    \t2 {} \circelse value1 > value2 \circthen \Skip \\
    \t2 \circfi \circseq Poll
  \end{circusaction}
  \caption{$TPK\_handleAsyncEvent$ after its $stackFrame$ variable is
    introduced}
  \label{efs-localise-stack-frames-example-figure}
\end{figure}