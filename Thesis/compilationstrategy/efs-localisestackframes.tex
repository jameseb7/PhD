After the $CheckLauncherReturn$ actions have been handled, the process
no longer has any actions that use the whole $frameStack$.
We can therefore refine each method to only operate on a local stack
frame variable.
This is performed as described in
Algorithm~\ref{localise-stack-frames-algorithm}, which defines the
procedure \Call{LocaliseStackFrames}{}.

\begin{algorithm}
  \begin{algorithmic}[1]
    \arraycolsep=0cm
    \State \ApplyFor{Law~[\nameref{forwards-data-refinement-law}]}{$InterpreterStateFS$, $FrameStackCI$}
    \label{algorithm-remove-currentClass-data-refinement}
    \State $iterationOrder \gets$ \Call{MethodDependencyOrder}{}
    \label{algorithm-method-dependency-order-call}
    \For{$methodName \gets iterationOrder$}
    \label{algorithm-localise-stack-frames-loop}
    \State $numArgs \gets$ \Call{MethodArguments}{$methodName$}
    \State $methodBody \gets$ \Call{ActionBody}{$methodName$}
    \State \ExhaustivelyApplyFor{Rule~[\nameref{arguments-introduction-rule}]}{$methodName$, $numArgs$}
    \label{algorithm-arguments-introduction}
    \MatchThen{%
      $\begin{array}[t]{l}
         (\circval arg1, \ldots, arg{<}n{>} : Word \circspot \\
         \t1 \lschexpract \exists methodArgs? == \langle arg1, \ldots, arg{<}n{>} \rangle @ \\
         \t2 InterpreterNewStackFrame[c/class?, m/methodID?] \rschexpract \circseq \\
         \t1 methodName \circseq \lschexpract InterpreterReturn \rschexpract)(args~1, \ldots, args~n)
       \end{array}$}
     \State \ApplyFor{Law~[\nameref{action-intro-law}]}{$methodName'$, %
       $\begin{array}[t]{l}
          (\circval arg1, \ldots, arg{<}n{>} : Word \circspot \\
          \t1 \lschexpract \exists methodArgs? \\
          \t3 {} == \langle arg1, \ldots, arg{<}n{>} \rangle @ \\
          \t2 InterpreterNewStackFrame[ \\
          \t3 c/class?, m/methodID?] \rschexpract \circseq \\
          \t1 methodName \circseq \lschexpract InterpreterReturn \rschexpract)
        \end{array}$}
        \label{algorithm-localise-stack-frames-new-method-action}
      \State \ExhaustivelyApplyReverseFor{Law~[\nameref{copy-rule-law}]}{$methodName'$}
      \label{algorithm-localise-stack-frames-copy-out}
      \State \ApplyToFor{Law~[\nameref{copy-rule-law}]}{\Call{ActionBody}{$methodName'$}}{$methodName$}
      \label{algorithm-localise-stack-frames-copy-in}
      \State \ApplyReverseFor{Law~[\nameref{action-intro-law}]}{$methodName$, $methodBody$}
      \label{algorithm-localise-stack-frames-method-elim}
      \State \ApplyFor{Law~[\nameref{action-rename-law}]}{$methodName'$, $methodName$}
      \label{algorithm-localise-stack-frames-method-rename}
      \State \ApplyToFor{Rule~[\nameref{HandleReturnEPC-stackFrame-introduction-rule}]}{\Call{ActionBody}{$methodName$}}{$numArgs$}
      \label{algorithm-HandleReturnEPC-stackFrame-introduction-rule}
    \EndFor
  \end{algorithmic}
  \caption{LocaliseStackFrames}
  \label{localise-stack-frames-algorithm}
\end{algorithm}
 
Since we must be able to operate directly on the $frameStack$ when
introducing stack frame variables, we first apply a data refinement to
remove $currentClass$ from the state.
We defined $currentClass$ in the model as a convenience when accessing
the $frameClass$ of the topmost stack frame, which is no longer
necessary when we have separate variables for each stack frame.
We also remove $frameStackID$ at this stage, since its only purpose is
to guard whether frameStack is permitted to be non-empty, which holds
from the form of the model.
The data refinement is applied on
line~\ref{algorithm-remove-currentClass-data-refinement} of
Algorithm~\ref{localise-stack-frames-algorithm}, and transforms the
state to $InterpreterStateFS$, shown below, which only contains
$frameStack$.
\begin{schema}{InterpreterStateFS}
  frameStack : StackFrameEPC
\end{schema}

The relationship between $InterpreterStateEPC$ and
$InterpreterStateFS$ is described by the coupling invariant
$FrameStackCI$, shown below.
It ensures $frameStack$ is unaffected by the refinement and replaces
occurrences of $currentClass$ with $(last~frameStack).frameClass$.
The $frameStackID$ is hidden from the new state.
\begin{schema}{FrameStackCI}
  InterpreterStateEPC \\
  InterpreterStateFS_1
\where
  frameStack = frameStack_1 \\
  currentClass = (last~frameStack_1).frameClass
\end{schema}

$FrameStackCI$ describes a functional data refinement, so the new
actions can be calculated in each case.
None of the actions in the process read from $frameStackID$, so it can
be safely hidden and the operations to set the $frameStackID$ are
reduced to $\Skip$.
As mentioned above, $(last~frameStack).frameClass$ is inserted
wherever $currentClass$ occurs in the old actions.
We can then proceed with introducing stack frame variables.

When introducing the variables to represent stack frames, we must
begin with those stack frames at the greatest depth on the stack.
This ensures uses of the $frameStack$ within a nested method do not
interfere with replacing uses of the $frameStack$ in an outer method.
The order in which we introduce stack frame variables to the methods
is specified by a procedure \Call{MethodDependencyOrder}{}, which
constructs a sequence of method action names indicating the order in
which the method actions should be handled.
This sequence is constructed by first adding to the sequence any
methods that contain no method calls, then adding any methods that
only call methods already in the sequence, and repeating until all
methods are in the sequence.
Since we do not allow recursion, this will always terminate.
We construct this sequence, $iterationOrder$, on
line~\ref{algorithm-method-dependency-order-call} of
Algorithm~\ref{localise-stack-frames-algorithm}.

We then loop, introducing a stack frame variable for each method in
the order specified by $iterationOrder$, in the for loop on
line~\ref{algorithm-localise-stack-frames-loop}.
Within the for loop, we first introduce value parameters, representing
the arguments to the method, around the call to the method action.
This ensures that the body of the method is completely independent of
the context in which it is called, enabling us to separate the whole
method body (including stack frame creation and return actions) into
its own action.
Introduction of method arguments is performed using
Rule~[\nameref{arguments-introduction-rule}], shown in
Figure~\ref{arguments-introduction-rule-figure}.
This rule is applied to two parameters:~$methodName$, the name of the
method being considered, and $numArgs$, the number of arguments to the
method (which is encoded as part of the method identifier, and so can
be determined during the strategy).
We apply this rule everywhere it applies on
line~\ref{algorithm-HandleReturnEPC-stackFrame-introduction-rule}.

\begin{figure}[thp]
\begin{restatable}[$Return$-arguments-intro]{crule}{ArgumentsIntroductionRule}
  \label{arguments-introduction-rule}
  %\setlength{\zedtab}{0.4cm}
  %\setlength{\zedindent}{0pt}
  %\setlength{\zedleftsep}{0pt}
  Given an action name $M$ and $n : \nat$,
  \begin{circus}
    \begin{array}{l}
      \lschexpract InterpreterNewStackFrame[ \\
      \t1 c/class?, \\
      \t1 m/methodID?, \\
      \t1 args/methodArgs?] \rschexpract \circseq \\
      M \circseq \lschexpract InterpreterReturn \rschexpract
    \end{array}
    \circrefines_A
    \begin{array}{l}
      (\circval arg1, \ldots, arg{<}n{>} : Word \circspot \\
      \t1 \lschexpract \exists methodArgs? == \langle arg1, \ldots, arg{<}n{>} \rangle @ \\
      \t2 InterpreterNewStackFrame[ \\
      \t3 c/class?, \\
      \t3 m/methodID?] \rschexpract \circseq \\
      \t1 M \circseq \lschexpract InterpreterReturn \rschexpract \\
      )(args~1, \ldots, args~n)
    \end{array}
  \end{circus}
  where $\ell = c.methodLocals~m$ and $s = c.methodStackSize~m$.
\end{restatable}
\caption{Rule~[\nameref{arguments-introduction-rule}]}
\label{arguments-introduction-rule-figure}
\end{figure}

Rule~[\nameref{arguments-introduction-rule}] introduces value
parameters representing method arguments around a method ending with
an $InterpreterReturn$ operation.
The arguments array passed into the $methodArgs$ input of
$InterpreterNewStackFrame$ is split into its individual elements,
which are passed into the parameters and recombined to be passed into
$InterpreterNewStackFrame$.
This splitting of the array ensures that the individual arguments can
be more easily handled in the next step, where we introduce local
variables.

After the method arguments have been introduced, we redefine the
method action to include the value parameters representing those
arguments.
That is performed by first introducing a temporary method action,
$methodName'$, matching the parametrised block around the call to
$methodName$, via an application of Law~[\nameref{action-intro-law}]
on line~\ref{algorithm-localise-stack-frames-new-method-action}.
The occurrences of the body of $methodName'$ are replaced with a
reference to the new action using Law~[\nameref{copy-rule-law}] on
line~\ref{algorithm-localise-stack-frames-copy-out}.
The occurrence of $methodName$ in the body of $methodName'$ is then
expanded using Law~[\nameref{copy-rule-law}] and $methodName$ is
eliminated using Law~[\nameref{action-intro-law}], since it no longer
occurs in the process.
Finally, the temporary action $methodName'$ is renamed to $methodName$
using Law~[\nameref{action-rename-law}] on
line~\ref{algorithm-localise-stack-frames-method-rename}.

Having completely separated the method into its own, independent,
action, we then introduce the stack frame variable for the method
using Rule~[\nameref{HandleReturnEPC-stackFrame-introduction-rule}],
shown in
Figure~\ref{HandleReturnEPC-stackFrame-introduction-rule-figure}.
This is applied to the body of $methodName$ on
line~\ref{algorithm-HandleReturnEPC-stackFrame-introduction-rule},
with the number of arguments, $numArgs$, passed to it.

\begin{figure}[thp]
\begin{restatable}[$Return$-$stackFrame$-intro]{crule}{HandleReturnEPCStackFrameIntroductionRule}
  \label{HandleReturnEPC-stackFrame-introduction-rule}
  \setlength{\zedtab}{0.4cm}
  \setlength{\zedindent}{0.5cm}
  %\setlength{\zedleftsep}{0pt}
  Given $n : \nat$, if $A$ operates solely on $last~frameStack$ and do
  not change the length of $frameStack$, then
  \begin{circus}
    \begin{array}{l}
      \lschexpract \exists methodArgs? == \langle arg1, \ldots, arg{<}n{>} \rangle @ \\
      \t2 InterpreterNewStackFrame[ \\
      \t3 c/class?, \\
      \t3 m/methodID? \rschexpract \circseq \\
      A \circseq \lschexpract InterpreterReturn \rschexpract
    \end{array}
    \circrefines_A
    \begin{array}{l}
      \circvar stackFrame : StackFrameEPC \circspot \\
      \t1 \lschexpract [arg1?, \ldots, arg{<}n{>}? : Word; \\
      \t1 stackFrame' : StackFrameEPC  | \\
      \t2 \langle arg1?, \ldots, arg{<}n{>}? \rangle \\
      \t3 {} \subseteq stackFrame'.localVariables \land \\
      \t2 \# stackFrame'.localVariables = \ell \land \\
      \t2 stackFrame'.operandStack = \langle\rangle \land \\
      \t2 stackFrame'.frameClass = c \land \\
      \t2 stackFrame'.stackSize = s] \rschexpract \circseq \\
      \t1 A[stackFrame/last~frameStack, \\
      \t2 stackFrame'/last~frameStack']
    \end{array}
  \end{circus}
  where $\ell = c.methodLocals~m$ and $s = c.methodStackSize~m$.
\end{restatable}
\caption{Rule~[\nameref{HandleReturnEPC-stackFrame-introduction-rule}]}
\label{HandleReturnEPC-stackFrame-introduction-rule-figure}
\end{figure}

Rule~[\nameref{HandleReturnEPC-stackFrame-introduction-rule}]
introduces a variable $stackFrame$, of type $StackFrameEPC$, over the
body of a method that ends with an $InterpreterReturn$ operation.
The $stackFrame$ variable is initialised in the same way as for the
stack frame created by $InterpreterNewStackFrame$, and each reference
to $last~frameStack$ in the body of the method is replaced with a
reference to $stackFrame$.
Replacing the references to $last~frameStack$ requires that the size
of $frameStack$ does not change during the method.
However, this requirement is met since method calls are the only
operation that changes the size of $frameStack$ and we replace
references to the $frameStack$ in nested methods first, by the
definition of $iterationOrder$.

Note that the operations performed on
lines~\ref{algorithm-arguments-introduction}
and~\ref{algorithm-HandleReturnEPC-stackFrame-introduction-rule}
specifically handle methods that do not return a value.
We omit the handling of methods that do return a value. 
Handling such methods would require rules similar to
Rule~[\nameref{arguments-introduction-rule}] for method bodies
followed by $InterpreterAreturn1$ and $InterpreterAreturn2$, which
would introduce a result parameter for the method in addition to the
value parameters representing the method's arguments.
The new method action would then have to match the different method
parameters.
We would also require a rule similar
Rule~[\nameref{HandleReturnEPC-stackFrame-introduction-rule}] to
handle the slightly different ending of the method action that the
return handling would create.
These rules would be applied in a way similar to the existing rules.

In our example, a $stackFrame$ variable will be introduced for
$TPK\_f$ first, since it does not call any other methods.
$TPK\_handleAsyncEvent$ only has its $stackFrame$ variable introduced
after $stackFrame$ variables have been introduced for all methods
called in $TPK\_handleAsyncEvent$.
When the $stackFrame$ variable for $TPK\_handleAsyncEvent$ has been
introduced, it has the form shown in
Figure~\ref{efs-localise-stack-frames-example-figure}.

For brevity, we define new actions, which we refer to as $Handle*SF$
actions.
These are not formally introduced as actions in the compilation
strategy as they are an abbreviation used for presenting examples and
stating compilation rules.
They are refined to a different form later in the elimination of frame
stack stage.
The $Handle*SF$ actions are similar to the $Handle*EPC$ actions,
except they have every reference to $last~frameStack$ (or
$last~frameStack'$) replaced with a reference to $stackFrame$ (or
$stackFrame'$), and have undergone the data refinement described
above.
We name them by replacing $EPC$ in the names of the $Handle*EPC$
actions with $SF$.
Similarly, we define an $InvokeSF$ operation that performs the
operation of $InterpreterStackFrameInvoke$ over $stackFrame$ instead
of $last~frameStack$.

% TODO: fix this for algorithm changes
\begin{figure}[tp!]
  \centering
  \setlength{\zedtab}{0.4cm}
  \setlength{\zedindent}{0pt}
  \setlength{\zedleftsep}{0pt}
  \setlength{\abovedisplayskip}{0pt}
  \setlength{\belowdisplayskip}{0pt}
  \setlength{\abovedisplayshortskip}{0pt}
  \setlength{\belowdisplayshortskip}{0pt}
  \begin{circusaction}
    TPK\_f \circdef \\
    \t1 \circval arg1 : Word \circspot \\
    \t1 \circvar stackFrame : StackFrameEPC \circspot \\
    \t1 \lschexpract [arg1? : Word; stackFrame' : StackFrameEPC | \\
    \t2 \langle arg1? \rangle \subseteq stackFrame'.localVariables \land \\
    \t2 \# stackFrame'.localVariables = 6 \land \\
    \t2 stackFrame'.operandStack = \langle\rangle \land \\
    \t2 stackFrame'.frameClass = TPK \land \\
    \t2 stackFrame'.stackSize = 3] \rschexpract \circseq \\
    \t1 Poll \circseq HandleNewSF(27) \circseq Poll \circseq HandleDupSF \circseq Poll \circseq  HandleAconst\_nullSF \circseq Poll \circseq \\
    \t1 (\circvar poppedArgs : \seq Word \circspot \\
    \t2 \lschexpract \exists argsToPop? == 2 @ InvokeSF \rschexpract \circseq \\
    \t2 ConsoleConnection\_CCinit(poppedArgs~1, poppedArgs~2)) \circseq Poll \circseq \\
    \t1 HandleAstoreSF(1) \circseq Poll \circseq HandleAloadSF(1) \circseq \\
    \t1 {} \cdots {} \\
    % \t1 Poll \circseq (\circvar poppedArgs : \seq Word \circspot \lschexpract \exists argsToPop? == m + 1 @ InterpreterStackFrameInvoke \rschexpract \circseq \\
    % \t1 getClassIDOf!(head~poppedArgs)?cid \then \lschexpract InterpreterNewStackFrame[ \\
    % \t2 ConsoleConnection/class?, openInputStream/methodID?, poppedArgs/methodArgs?] \rschexpract) \circseq \\
    % \t1 Poll \circseq ConsoleConnection\_openInputStream \circseq Poll \circseq  HandleAstoreEPC(2) \circseq Poll \circseq \\
    % \t1 HandleAloadEPC(1) \circseq Poll \circseq (\circvar poppedArgs : \seq Word \circspot \\
    % \t1 \lschexpract \exists argsToPop? == m + 1 @ InterpreterStackFrameInvoke \rschexpract \circseq \\
    % \t1 getClassIDOf!(head~poppedArgs)?cid \then \lschexpract InterpreterNewStackFrame[\\
    % \t2 ConsoleConnection/class?, openOutputStream/methodID?, poppedArgs/methodArgs?] \rschexpract) \circseq \\
    % \t1 Poll \circseq ConsoleConnection\_openOutputStream \circseq Poll \circseq HandleAstoreEPC(3) \circseq \\
    %\t1 {} \cdots {} \\
    \t1 Poll \circseq HandleIconstSF(0) \circseq Poll \circseq HandleAstoreSF(4) \circseq Poll \circseq Poll \circseq \circmu Y \circspot \\
    \t2 HandleAloadSF(4) \circseq Poll \circseq HandleIconstSF(10) \circseq Poll \circseq \\
    \t2 \circvar value1, value2 : Word \circspot \\
    \t3 \lschexpract InterpreterPop2[stackFrame/last~frameStack,stackFrame'/last~frameStack'] \rschexpract \circseq \\
    \t2 \circif value1 \leq value2 \circthen {} \\
    \t3 {} \cdots {}  \\
    % Poll \circseq HandleAloadEPC(2) \circseq Poll \circseq \\
    % \t3 (\circvar poppedArgs : \seq Word \circspot \\
    % \t4 \lschexpract \exists argsToPop? == m + 1 @ InterpreterStackFrameInvoke \rschexpract \circseq \\
    % \t4 getClassIDOf!(head~poppedArgs)?cid \then \lschexpract InterpreterNewStackFrame[ \\
    % \t5 ConsoleInput/class?, read/methodID?, poppedArgs/methodArgs?] \rschexpract) \circseq \\
    % \t3 Poll \circseq ConsoleInput\_read \circseq Poll \circseq \\
    % \t3 (\circvar poppedArgs : \seq Word \circspot \\
    % \t4 \lschexpract \exists argsToPop? == 1 @ InvokeSF \rschexpract \circseq \\
    % \t4 TPK\_f()) \circseq \lschexpract InterpreterAreturn1 \rschexpract \circseq Poll \circseq \\
    % \t3 HandleAstoreEPC(5) \circseq Poll \circseq HandleAloadEPC(5) \circseq \\
    % \t3 {} \cdots {} \\
    % \t3 Poll \circseq HandleIconstEPC(400) \circseq Poll \circseq \circvar value1, value2 : Word \circspot InterpreterPop2 \circseq \\
    % \t3 \circif value1 \leq value2 \circthen HandleAloadEPC(3) \circseq Poll \circseq HandleAloadEPC(5) \circseq Poll \circseq \\
    % \t4 (\circvar poppedArgs : \seq Word \circspot \lschexpract \exists argsToPop? == m + 1 @ InterpreterStackFrameInvoke \rschexpract \circseq \\
    % \t4 getClassIDOf!(head~poppedArgs)?cid \then \lschexpract InterpreterNewStackFrame[ \\
    % \t5 ConsoleOutput/class?, write/methodID?, poppedArgs/methodArgs?] \rschexpract)) \circseq \\
    % \t4 Poll \circseq ConsoleOutput\_write \\
    % \t3 {} \circelse value1 > value2 \circthen HandleAloadEPC(3) \circseq Poll \circseq HandleIconstEPC(0) \circseq Poll \circseq \\
    % \t4 (\circvar poppedArgs : \seq Word \circspot \lschexpract \exists argsToPop? == m + 1 @ InterpreterStackFrameInvoke \rschexpract \circseq \\
    % \t4 getClassIDOf!(head~poppedArgs)?cid \then \lschexpract InterpreterNewStackFrame[ \\
    % \t5 ConsoleOutput/class?, write/methodID?, poppedArgs/methodArgs?] \rschexpract)) \circseq \\
    % \t4 Poll \circseq ConsoleOutput\_write \\
    % \t3 \circfi \circseq Poll \circseq HandleAloadEPC(4) \circseq Poll \circseq HandleIconstEPC(1) \circseq Poll \circseq HandleIaddEPC \circseq \\
    \t3 Poll \circseq HandleAstoreSF(4) \circseq Poll \circseq Y \\
    \t2 {} \circelse value1 > value2 \circthen \Skip \\
    \t2 \circfi \circseq Poll
  \end{circusaction}
  \caption{$TPK\_handleAsyncEvent$ after its $stackFrame$ variable is
    introduced}
  \label{efs-localise-stack-frames-example-figure}
\end{figure}