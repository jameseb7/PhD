% undo some CZT meddling with the hash symbol
\renewcommand\#{\char"0023}

\lstset{basicstyle=\ttfamily\footnotesize,language=C,
  numbers=left, numberstyle=\tiny, stepnumber=1, numbersep=2pt,
  escapeinside={(*@}{@*)}, linewidth=\linewidth, breaklines=true,
  tabsize=2, framexleftmargin=0.3cm, xleftmargin=0.3cm,
  postbreak=\mbox{$\!\hookrightarrow$}, breakindent=2pt}

This appendix contains the C code for the examples considered in
Chapter~\ref{evaluation-chapter}.
We provide the code for each example in a separate
section:~\texttt{PersistentSignal} in
Section~\ref{PersistentSignal-code-section}, \texttt{Buffer} in
Section~\ref{Buffer-code-section}, and \texttt{Barrier} in
Section~\ref{Barrier-code-section}.

For each example, we first provide the Java code used as input to our
prototype (Sections~\ref{PersistentSignal-java-code-subsection},
\ref{Buffer-java-code-subsection}
and~\ref{Barrier-java-code-subsection}).
The code input to icecap is similar, but with the addition of a file
containing a main method that invokes icecap's launcher code, passing
the safelet for the program.
Java arrays are also used in the code input to icecap, rather than the
array classes used in our code.

After the Java code for each example, we present the code generated by
our prototype for each of the program methods of the examples the
examples (Sections~\ref{PersistentSignal-code-comparison-subsection},
\ref{Buffer-code-our-subsection}
and~\ref{Barrier-code-our-subsection}).
For the first example we also present the corresponding icecap code
for each method.
Since the icecap code is quite long and the corresponding code for
each of the constructs in our prototype code is similar, we omit it
here.
It can be found among the online resources that accompany this thesis.
% TODO: insert URL here

Due to the length of some of the identifiers in the code, we have
shortened the the identifiers by omitting type signatures in method
and field identifiers, since there are no places in the code where
that would cause ambiguity.
This brings the identifiers closer to those used in the corresponding
icecap code.
Also, since some of the lines of code are particularly long, 
they are broken across multiple lines in our presentation.
Lines that are the continuation of a line of code in the original file
are marked with a hooked arrow (\mbox{$\hookrightarrow$}) at the start
and are not given a separate line number.

\begin{landscape}
\begin{multicols}{2}

\section{\texorpdfstring{\texttt{PersistentSignal}}{PersistentSignal}}
\label{PersistentSignal-code-section}

\subsection{Java Code}
\label{PersistentSignal-java-code-subsection}

\subsubsection{\texttt{MainMission.java}}
\lstinputlisting{../../workspace2/ModelPersistentSignal/src/main/MainMission.java}

\subsubsection{\texttt{MainSequence.java}}
\lstinputlisting{../../workspace2/ModelPersistentSignal/src/main/MainSequence.java}

\subsubsection{\texttt{MySafelet.java}}
\lstinputlisting{../../workspace2/ModelPersistentSignal/src/main/MySafelet.java}

\subsubsection{\texttt{PersistentSignal.java}}
\lstinputlisting{../../workspace2/ModelPersistentSignal/src/main/PersistentSignal.java}

\subsubsection{\texttt{Producer.java}}
\lstinputlisting{../../workspace2/ModelPersistentSignal/src/main/Producer.java}

\subsubsection{\texttt{Worker.java}}
\lstinputlisting{../../workspace2/ModelPersistentSignal/src/main/Worker.java}

\subsection{Comparison of program code}
\label{PersistentSignal-code-comparison-subsection}

\subsubsection{\texttt{main\_MySafelet\_globalBackingStoreSize}}

\paragraph{Our code}\hfill
\begin{lstlisting}[firstnumber=257]
void main_MySafelet_globalBackingStoreSize(int32_t var1, int32_t * retVal_msb, int32_t * retVal_lsb) {
	int32_t stack1, stack2;
	stack1 = 0;
	stack2 = 0;
	*retVal_lsb = stack2;
	*retVal_msb = stack1;
}
\end{lstlisting}

\paragraph{Corresponding icecap code}\hfill\\
There is no corresponding icecap code for this method.

\subsubsection{\texttt{main\_Producer\_init}}

\paragraph{Our code}\hfill
\begin{lstlisting}[firstnumber=281]
void main_Producer_init(int32_t var1, int32_t var2, int32_t var3, int32_t var4, int32_t var5, int32_t var6, int32_t var7) {
	int32_t stack1, stack2, stack3, stack4, stack5, stack6, stack7, stack8, stack9, stack10, stack11, stack12, stack13;
	stack1 = var1;
	stack2 = newObject(javax_realtime_PriorityParametersID);
	stack3 = stack2;
	javax_safetycritical_PriorityScheduler_instance(&stack4);
	if (((java_lang_Object*)  ((uintptr_t)stack4))->classID == javax_safetycritical_PrioritySchedulerID) {
		javax_safetycritical_PriorityScheduler_getNormPriority(stack4, &stack4);
	}
	javax_realtime_PriorityParameters_init_I_V(stack3, stack4);
	stack3 = newObject(javax_realtime_PeriodicParametersID);
	stack4 = stack3;
	stack5 = newObject(javax_realtime_RelativeTimeID);
	stack6 = stack5;
	stack7 = var6;
	stack8 = var7;
	stack9 = 0;
	javax_realtime_RelativeTime_init(stack6, stack7, stack8, stack9);
	stack6 = newObject(javax_realtime_RelativeTimeID);
	stack7 = stack6;
	stack8 = var4;
	stack9 = var5;
	stack10 = 0;
	javax_realtime_RelativeTime_init(stack7, stack8, stack9, stack10);
	javax_realtime_PeriodicParameters_init(stack4, stack5, stack6);
	stack4 = newObject(javax_realtime_memory_ScopeParametersID);
	stack5 = stack4;
	stack6 = 0;
	stack7 = 40000;
	stack8 = 0;
	stack9 = 20000;
	stack10 = 0;
	stack11 = 0;
	stack12 = 0;
	stack13 = 0;
	javax_realtime_memory_ScopeParameters_init(stack5, stack6, stack7, stack8, stack9, stack10, stack11, stack12, stack13);
	stack5 = newObject(javax_realtime_ConfigurationParametersID);
	stack6 = stack5;
	stack7 = -1;
	stack8 = -1;
	stack9 = newObject(java_lang_LongArray1ID);
	stack10 = stack9;
	stack11 = 0;
	stack12 = 6144;
	java_lang_LongArray1_init_J_V(stack10, stack11, stack12);
	javax_realtime_ConfigurationParameters_init(stack6, stack7, stack8, stack9);
	javax_safetycritical_PeriodicEventHandler_init(stack1, stack2, stack3, stack4, stack5);
	stack1 = var1;
	stack2 = var2;
	((main_Producer *) ((uintptr_t)stack1))->_signal = stack2;
	stack1 = var1;
	stack2 = var3;
	((main_Producer *) ((uintptr_t)stack1))->_worker = stack2;

}
\end{lstlisting}

\paragraph{Corresponding icecap code}\hfill
\begin{lstlisting}[firstnumber=54747]
int16 main_Producer_init_(int32 *fp, int32 this, int32 signal, int32 worker, int32 period_ms, int32 lv_4, int32 offset_ms, int32 lv_6)
{
	int32* sp;
	int32 i_val12;
	int16 rval_m_5;
	int32 i_val11;
	int32 rval_5;
#if defined(JAVA_LANG_THROWABLE_INIT_)
	unsigned short pc;
#endif
	int16 excep;
	unsigned short handler_pc;
	int16 rval_m_9;
	int32 rval_9;
	int16 rval_m_13;
	int32 i_val10;
	int32 i_val9;
	int16 rval_m_28;
	int16 rval_m_38;
	int32 hvm_arg_no_3_42;
	int32 hvm_arg_no_2_42;
	int32 hvm_arg_no_1_42;
	int16 rval_m_42;
	int32 lsb_int32;
	int32 msb_int32;
	int32 i_val8;
	int32 i_val7;
	int32 i_val6;
	int32 i_val5;
	int32 i_val4;
	int16 rval_m_66;
	int16 s_val9;
	Object* narray;
	uint16 _count_;
	int8 b_val7;
	int8 index_int8;
	uint32* cobj_89;
	int16 rval_m_90;
	int32 hvm_arg_no_5_94;
	int32 hvm_arg_no_4_94;
	int32 hvm_arg_no_3_94;
	int32 hvm_arg_no_2_94;
	int32 hvm_arg_no_1_94;
	int16 rval_m_94;
	unsigned char* cobj;
	sp = &fp[9]; /* make room for local VM state on the stack */
	/*		super( */
	i_val12 = this;
	/*				new PriorityParameters(PriorityScheduler.instance().getNormPriority()), */
	*sp = (int32)i_val12;
	sp++;
	if (handleNewClassIndex(sp, 63) == 0) {
		fp[0] = *sp;
		return getClassIndex((Object*) (pointer) *sp);
	}
	sp++;
	/*				new PriorityParameters(PriorityScheduler.instance().getNormPriority()), */
	i_val12 = *(sp - 1);
	/*				new PriorityParameters(PriorityScheduler.instance().getNormPriority()), */
	sp += 1;
	rval_m_5 = javax_safetycritical_PriorityScheduler_instance(sp);
	if (rval_m_5 == -1) {
		rval_5 = *(int32*)sp;
		i_val11 = rval_5;
	}
	else
	{
		fp[0] = *sp;
		return rval_m_5;
	}
	sp -= 1;
	/*				new PriorityParameters(PriorityScheduler.instance().getNormPriority()), */
	if (i_val11 == 0) {
#if defined(JAVA_LANG_THROWABLE_INIT_)
		pc = 9;
#endif
		goto throwNullPointer;
	}
	sp += 1;
	rval_m_9 = javax_realtime_PriorityScheduler_getNormPriority(sp, i_val11);
	if (rval_m_9 == -1) {
		rval_9 = *(int32*)sp;
		i_val11 = rval_9;
	}
	else
	{
		fp[0] = *sp;
		return rval_m_9;
	}
	sp -= 1;
	/*				new PriorityParameters(PriorityScheduler.instance().getNormPriority()), */
	rval_m_13 = javax_realtime_PriorityParameters_init_(sp, i_val12, i_val11);
	if (rval_m_13 == -1) {
		;
	}
	else
	{
		fp[0] = *sp;
		return rval_m_13;
	}
	/*				new PeriodicParameters(new RelativeTime(offset_ms, 0), new RelativeTime(period_ms, 0)), */
	if (handleNewClassIndex(sp, 91) == 0) {
		fp[0] = *sp;
		return getClassIndex((Object*) (pointer) *sp);
	}
	sp++;
	/*				new PeriodicParameters(new RelativeTime(offset_ms, 0), new RelativeTime(period_ms, 0)), */
	i_val12 = *(sp - 1);
	/*				new PeriodicParameters(new RelativeTime(offset_ms, 0), new RelativeTime(period_ms, 0)), */
	*sp = (int32)i_val12;
	sp++;
	if (handleNewClassIndex(sp, 133) == 0) {
		fp[0] = *sp;
		return getClassIndex((Object*) (pointer) *sp);
	}
	sp++;
	/*				new PeriodicParameters(new RelativeTime(offset_ms, 0), new RelativeTime(period_ms, 0)), */
	i_val12 = *(sp - 1);
	/*				new PeriodicParameters(new RelativeTime(offset_ms, 0), new RelativeTime(period_ms, 0)), */
	i_val11 = offset_ms;
	i_val10 = lv_6;
	/*				new PeriodicParameters(new RelativeTime(offset_ms, 0), new RelativeTime(period_ms, 0)), */
	i_val9 = 0;
	/*				new PeriodicParameters(new RelativeTime(offset_ms, 0), new RelativeTime(period_ms, 0)), */
	rval_m_28 = javax_realtime_RelativeTime_init__(sp, i_val12, i_val11, i_val10, i_val9);
	if (rval_m_28 == -1) {
		;
	}
	else
	{
		fp[0] = *sp;
		return rval_m_28;
	}
	/*				new PeriodicParameters(new RelativeTime(offset_ms, 0), new RelativeTime(period_ms, 0)), */
	if (handleNewClassIndex(sp, 133) == 0) {
		fp[0] = *sp;
		return getClassIndex((Object*) (pointer) *sp);
	}
	sp++;
	/*				new PeriodicParameters(new RelativeTime(offset_ms, 0), new RelativeTime(period_ms, 0)), */
	i_val12 = *(sp - 1);
	/*				new PeriodicParameters(new RelativeTime(offset_ms, 0), new RelativeTime(period_ms, 0)), */
	i_val11 = period_ms;
	i_val10 = lv_4;
	/*				new PeriodicParameters(new RelativeTime(offset_ms, 0), new RelativeTime(period_ms, 0)), */
	i_val9 = 0;
	/*				new PeriodicParameters(new RelativeTime(offset_ms, 0), new RelativeTime(period_ms, 0)), */
	rval_m_38 = javax_realtime_RelativeTime_init__(sp, i_val12, i_val11, i_val10, i_val9);
	if (rval_m_38 == -1) {
		;
	}
	else
	{
		fp[0] = *sp;
		return rval_m_38;
	}
	/*				new PeriodicParameters(new RelativeTime(offset_ms, 0), new RelativeTime(period_ms, 0)), */
	sp--;
	hvm_arg_no_3_42 = (int32)(*sp);
	sp--;
	hvm_arg_no_2_42 = (int32)(*sp);
	sp--;
	hvm_arg_no_1_42 = (int32)(*sp);
	rval_m_42 = javax_realtime_PeriodicParameters_init_(sp, hvm_arg_no_1_42, hvm_arg_no_2_42, hvm_arg_no_3_42);
	if (rval_m_42 == -1) {
		;
	}
	else
	{
		fp[0] = *sp;
		return rval_m_42;
	}
	/*				new StorageParameters( */
	if (handleNewClassIndex(sp, 64) == 0) {
		fp[0] = *sp;
		return getClassIndex((Object*) (pointer) *sp);
	}
	sp++;
	/*				new StorageParameters( */
	i_val12 = *(sp - 1);
	/*						Const.PRIVATE_BACKING_STORE, */
	i_val11 = ((struct _staticClassFields_c *)(pointer)HEAP_REF((pointer)classData, staticClassFields_c*)) -> PRIVATE_BACKING_STORE_f;
	/*						Const.PRIVATE_BACKING_STORE, */
	lsb_int32 = i_val11;
	if (lsb_int32 < 0) {
		msb_int32 = -1;
	} else {
		msb_int32 = 0;
	}
	i_val11 = msb_int32;
	i_val10 = lsb_int32;
	/*						Const.PRIVATE_MEM, */
	i_val9 = ((struct _staticClassFields_c *)(pointer)HEAP_REF((pointer)classData, staticClassFields_c*)) -> PRIVATE_MEM_f;
	/*						Const.PRIVATE_MEM, */
	lsb_int32 = i_val9;
	if (lsb_int32 < 0) {
		msb_int32 = -1;
	} else {
		msb_int32 = 0;
	}
	i_val9 = msb_int32;
	i_val8 = lsb_int32;
	/*						0, 0), */
	i_val7 = 0;
	i_val6 = 0;
	/*						0, 0), */
	i_val5 = 0;
	i_val4 = 0;
	/*				new StorageParameters( */
	rval_m_66 = javax_safetycritical_StorageParameters_init_(sp, i_val12, i_val11, i_val10, i_val9, i_val8, i_val7, i_val6, i_val5, i_val4);
	if (rval_m_66 == -1) {
		;
	}
	else
	{
		fp[0] = *sp;
		return rval_m_66;
	}
	/*				new ConfigurationParameters(-1, -1, new long[] {Const.HANDLER_STACK_SIZE})); */
	if (handleNewClassIndex(sp, 14) == 0) {
		fp[0] = *sp;
		return getClassIndex((Object*) (pointer) *sp);
	}
	sp++;
	/*				new ConfigurationParameters(-1, -1, new long[] {Const.HANDLER_STACK_SIZE})); */
	i_val12 = *(sp - 1);
	/*				new ConfigurationParameters(-1, -1, new long[] {Const.HANDLER_STACK_SIZE})); */
	i_val11 = -1;
	/*				new ConfigurationParameters(-1, -1, new long[] {Const.HANDLER_STACK_SIZE})); */
	i_val10 = -1;
	/*				new ConfigurationParameters(-1, -1, new long[] {Const.HANDLER_STACK_SIZE})); */
	s_val9 = 1;
	/*				new ConfigurationParameters(-1, -1, new long[] {Const.HANDLER_STACK_SIZE})); */
	_count_ = s_val9;
	narray = (Object*) createArray(48, (uint16) _count_ FLASHARG((0)));
	if (narray == 0) {
#if defined(JAVA_LANG_THROWABLE_INIT_)
		pc = 77;
#endif
		goto throwOutOfMemory;
	}
	i_val9 = (int32) (pointer) narray;
	/*				new ConfigurationParameters(-1, -1, new long[] {Const.HANDLER_STACK_SIZE})); */
	i_val8 = i_val9;
	/*				new ConfigurationParameters(-1, -1, new long[] {Const.HANDLER_STACK_SIZE})); */
	b_val7 = 0;
	/*				new ConfigurationParameters(-1, -1, new long[] {Const.HANDLER_STACK_SIZE})); */
	i_val6 = ((struct _staticClassFields_c *)(pointer)HEAP_REF((pointer)classData, staticClassFields_c*)) -> HANDLER_STACK_SIZE_f;
	/*				new ConfigurationParameters(-1, -1, new long[] {Const.HANDLER_STACK_SIZE})); */
	lsb_int32 = i_val6;
	if (lsb_int32 < 0) {
		msb_int32 = -1;
	} else {
		msb_int32 = 0;
	}
	i_val6 = msb_int32;
	i_val5 = lsb_int32;
	/*				new ConfigurationParameters(-1, -1, new long[] {Const.HANDLER_STACK_SIZE})); */
	lsb_int32 = i_val5;
	msb_int32 = i_val6;
	index_int8 = b_val7;
	cobj_89 = HEAP_REF((pointer)(i_val8 + sizeof(Object) + 2), uint32*);
	cobj_89[index_int8 << 1] = msb_int32;
	cobj_89[(index_int8 << 1) + 1] = lsb_int32;
	/*				new ConfigurationParameters(-1, -1, new long[] {Const.HANDLER_STACK_SIZE})); */
	rval_m_90 = javax_realtime_ConfigurationParameters_init_(sp, i_val12, i_val11, i_val10, i_val9);
	if (rval_m_90 == -1) {
		;
	}
	else
	{
		fp[0] = *sp;
		return rval_m_90;
	}
	/*				new ConfigurationParameters(-1, -1, new long[] {Const.HANDLER_STACK_SIZE})); */
	sp--;
	hvm_arg_no_5_94 = (int32)(*sp);
	sp--;
	hvm_arg_no_4_94 = (int32)(*sp);
	sp--;
	hvm_arg_no_3_94 = (int32)(*sp);
	sp--;
	hvm_arg_no_2_94 = (int32)(*sp);
	sp--;
	hvm_arg_no_1_94 = (int32)(*sp);
	rval_m_94 = javax_safetycritical_PeriodicEventHandler_init_(sp, hvm_arg_no_1_94, hvm_arg_no_2_94, hvm_arg_no_3_94, hvm_arg_no_4_94, hvm_arg_no_5_94);
	if (rval_m_94 == -1) {
		;
	}
	else
	{
		fp[0] = *sp;
		return rval_m_94;
	}
	/*		this._signal = signal; */
	i_val12 = this;
	/*		this._signal = signal; */
	i_val11 = signal;
	/*		this._signal = signal; */
	lsb_int32 = i_val11;
	cobj = (unsigned char *) (pointer)i_val12;
	((struct _main_Producer_c *)HEAP_REF(cobj, void*)) -> _signal_f = lsb_int32;
	/*		this._worker = worker; */
	i_val12 = this;
	/*		this._worker = worker; */
	i_val11 = worker;
	/*		this._worker = worker; */
	lsb_int32 = i_val11;
	cobj = (unsigned char *) (pointer)i_val12;
	((struct _main_Producer_c *)HEAP_REF(cobj, void*)) -> _worker_f = lsb_int32;
	/*	} */
	return -1;
	throwNullPointer:
	excep = initializeException(sp, JAVA_LANG_NULLPOINTEREXCEPTION, JAVA_LANG_NULLPOINTEREXCEPTION_INIT_);
	goto throwIt;
	throwOutOfMemory:
	excep = initializeException(sp, JAVA_LANG_OUTOFMEMORYERROR, JAVA_LANG_OUTOFMEMORYERROR_INIT_);
	goto throwIt;
	throwIt:
#if defined(JAVA_LANG_THROWABLE_INIT_)
	handler_pc = handleAthrow(&methods[531], excep, pc);
#else
	handler_pc = -1;
#endif
	sp++;
	switch(handler_pc) {
		case (unsigned short)-1: /* Not handled */
		default:
		fp[0] = *(sp - 1);
		return excep;
	}
}
\end{lstlisting}

\subsubsection{\texttt{main\_PersistentSignal\_reset}}

\paragraph{Our code}\hfill
\begin{lstlisting}[firstnumber=514]
void main_PersistentSignal_reset(int32_t var1) {
	int32_t stack1, stack2;
	stack1 = var1;
	stack2 = 0;
	((main_PersistentSignal *) ((uintptr_t)stack1))->_set = stack2;
	releaseLock(var1);
}
\end{lstlisting}

\paragraph{Corresponding icecap code}\hfill
\begin{lstlisting}[firstnumber=54641]
int16 main_PersistentSignal_reset(int32 *fp, int32 this)
{
	int32 i_val1;
	int8 b_val0;
	unsigned char* cobj;
	int8 lsb_int8;
	/*		this._set = false; */
	i_val1 = this;
	/*		this._set = false; */
	b_val0 = 0;
	/*		this._set = false; */
	lsb_int8 = b_val0;
	cobj = (unsigned char *) (pointer)i_val1;
	((struct _main_PersistentSignal_c *)HEAP_REF(cobj, void*)) -> _set_f = lsb_int8;
	/*	} */
	handleMonitorEnterExit((Object*)(pointer)this, 0, fp + 1, "");
	return -1;
}
\end{lstlisting}

\subsubsection{\texttt{main\_MySafelet\_cleanUp}}

\paragraph{Our code}\hfill
\begin{lstlisting}[firstnumber=646]
void main_MySafelet_cleanUp(int32_t var1) {
	
}
\end{lstlisting}

\paragraph{Corresponding icecap code}\hfill\\
There is no corresponding icecap code for this method.

\subsubsection{\texttt{main\_PersistentSignal\_isSet}}

\paragraph{Our code}\hfill
\begin{lstlisting}[firstnumber=814]
void main_PersistentSignal_isSet(int32_t var1, int32_t * retVal) {
	int32_t stack1;
	stack1 = var1;
	stack1 = ((main_PersistentSignal *)  ((uintptr_t)stack1))->_set;
	releaseLock(var1);
	*retVal = stack1;
}
\end{lstlisting}

\paragraph{Corresponding icecap code}\hfill
\begin{lstlisting}[firstnumber=54620]
int16 main_PersistentSignal_isSet(int32 *fp, int32 this)
{
	int32 i_val0;
	unsigned char* cobj;
	int8 b_val0;
	/*		return this._set; */
	i_val0 = this;
	/*		return this._set; */
	cobj = (unsigned char *) (pointer)i_val0;
	b_val0 = ((struct _main_PersistentSignal_c *)HEAP_REF(cobj, void*)) -> _set_f;
	/*		return this._set; */
	handleMonitorEnterExit((Object*)(pointer)this, 0, fp + 1, "");
	return (uint8)b_val0;
}
\end{lstlisting}

\subsubsection{\texttt{main\_MySafelet\_handleStartupError}}

\paragraph{Our code}\hfill
\begin{lstlisting}[firstnumber=1177]
void main_MySafelet_handleStartupError(int32_t var1, int32_t var2, int32_t var3, int32_t var4, int32_t * retVal) {
	int32_t stack1;
	stack1 = 0;
	*retVal = stack1;
}
\end{lstlisting}

\paragraph{Corresponding icecap code}\hfill\\
There is no corresponding icecap code for this method.

\subsubsection{\texttt{main\_MainMission\_missionMemorySize}}

\paragraph{Our code}\hfill
\begin{lstlisting}[firstnumber=1256]
void main_MainMission_missionMemorySize(int32_t var1, int32_t * retVal_msb, int32_t * retVal_lsb) {
	int32_t stack1, stack2;
	stack1 = 0;
	stack2 = 1000000;
	*retVal_lsb = stack2;
	*retVal_msb = stack1;
}
\end{lstlisting}

\paragraph{Corresponding icecap code}\hfill\\
There is no corresponding icecap code for this method.

\subsubsection{\texttt{main\_MainSequence\_init}}

\paragraph{Our code}\hfill
\begin{lstlisting}[firstnumber=1271]
void main_MainSequence_init(int32_t var1) {
	int32_t stack1, stack2, stack3, stack4, stack5, stack6, stack7, stack8, stack9, stack10, stack11, stack12;
	stack1 = var1;
	stack2 = newObject(javax_realtime_PriorityParametersID);
	stack3 = stack2;
	javax_safetycritical_PriorityScheduler_instance(&stack4);
	if (((java_lang_Object*)  ((uintptr_t)stack4))->classID == javax_safetycritical_PrioritySchedulerID) {
		javax_safetycritical_PriorityScheduler_getMaxPriority(stack4, &stack4);
	}
	javax_realtime_PriorityParameters_init(stack3, stack4);
	stack3 = newObject(javax_realtime_memory_ScopeParametersID);
	stack4 = stack3;
	stack5 = 0;
	stack6 = 702000;
	stack7 = 0;
	stack8 = 20000;
	stack9 = 0;
	stack10 = 100000;
	stack11 = 0;
	stack12 = 200000;
	javax_realtime_memory_ScopeParameters_init(stack4, stack5, stack6, stack7, stack8, stack9, stack10, stack11, stack12);
	stack4 = newObject(javax_realtime_ConfigurationParametersID);
	stack5 = stack4;
	stack6 = -1;
	stack7 = -1;
	stack8 = newObject(java_lang_LongArray1ID);
	stack9 = stack8;
	stack10 = 0;
	stack11 = 6144;
	java_lang_LongArray1_init(stack9, stack10, stack11);
	javax_realtime_ConfigurationParameters_init(stack5, stack6, stack7, stack8);
	javax_safetycritical_MissionSequencer_init(stack1, stack2, stack3, stack4);

}
\end{lstlisting}

\paragraph{Corresponding icecap code}\hfill
\begin{lstlisting}[firstnumber=54130]
int16 main_MainSequence_init_(int32 *fp) {
	int32* sp;
	int32 i_val11;
	int16 rval_m_5;
	int32 i_val10;
	int32 rval_5;
#if defined(JAVA_LANG_THROWABLE_INIT_)
	unsigned short pc;
#endif
	int16 excep;
	unsigned short handler_pc;
	int16 rval_m_9;
	int32 rval_9;
	int16 rval_m_13;
	int32 lsb_int32;
	int32 msb_int32;
	int32 i_val9;
	int32 i_val8;
	int32 i_val7;
	int32 i_val6;
	int32 i_val5;
	int32 i_val4;
	int32 i_val3;
	int16 rval_m_49;
	int16 s_val8;
	Object* narray;
	uint16 _count_;
	int8 b_val6;
	int8 index_int8;
	uint32* cobj_72;
	int16 rval_m_73;
	int32 hvm_arg_no_4_77;
	int32 hvm_arg_no_3_77;
	int32 hvm_arg_no_2_77;
	int32 hvm_arg_no_1_77;
	int16 rval_m_77;
	int32
	this;
	this = (int32)(*(fp + 0));
	sp = &fp[3]; /* make room for local VM state on the stack */
	/*		super( */
	i_val11 = this;
	/*				new PriorityParameters(PriorityScheduler.instance().getMaxPriority()), */
	*sp = (int32) i_val11;
	sp++;
	if (handleNewClassIndex(sp, 63) == 0) {
		fp[0] = *sp;
		return getClassIndex((Object*) (pointer) * sp);
	}
	sp++;
	/*				new PriorityParameters(PriorityScheduler.instance().getMaxPriority()), */
	i_val11 = *(sp - 1);
	/*				new PriorityParameters(PriorityScheduler.instance().getMaxPriority()), */
	sp += 1;
	rval_m_5 = javax_safetycritical_PriorityScheduler_instance(sp);
	if (rval_m_5 == -1) {
		rval_5 = *(int32*) sp;
		i_val10 = rval_5;
	} else {
		fp[0] = *sp;
		return rval_m_5;
	}
	sp -= 1;
	/*				new PriorityParameters(PriorityScheduler.instance().getMaxPriority()), */
	if (i_val10 == 0) {
#if defined(JAVA_LANG_THROWABLE_INIT_)
		pc = 9;
#endif
		goto throwNullPointer;
	}
	sp += 1;
	rval_m_9 = javax_realtime_PriorityScheduler_getMaxPriority(sp, i_val10);
	if (rval_m_9 == -1) {
		rval_9 = *(int32*) sp;
		i_val10 = rval_9;
	} else {
		fp[0] = *sp;
		return rval_m_9;
	}
	sp -= 1;
	/*				new PriorityParameters(PriorityScheduler.instance().getMaxPriority()), */
	rval_m_13 = javax_realtime_PriorityParameters_init_(sp, i_val11, i_val10);
	if (rval_m_13 == -1) {
		;
	} else {
		fp[0] = *sp;
		return rval_m_13;
	}
	/*				new StorageParameters( */
	if (handleNewClassIndex(sp, 64) == 0) {
		fp[0] = *sp;
		return getClassIndex((Object*) (pointer) * sp);
	}
	sp++;
	/*				new StorageParameters( */
	i_val11 = *(sp - 1);
	/*						Const.OUTERMOST_SEQ_BACKING_STORE, */
	i_val10 = ((struct _staticClassFields_c *)(pointer)HEAP_REF((pointer)classData, staticClassFields_c*)) -> OUTERMOST_SEQ_BACKING_STORE_f;
	/*						Const.OUTERMOST_SEQ_BACKING_STORE, */
	lsb_int32 = i_val10;
	if (lsb_int32 < 0) {
		msb_int32 = -1;
	} else {
		msb_int32 = 0;
	}
	i_val10 = msb_int32;
	i_val9 = lsb_int32;
	/*						Const.PRIVATE_MEM, */
	i_val8 = ((struct _staticClassFields_c *)(pointer)HEAP_REF((pointer)classData, staticClassFields_c*)) -> PRIVATE_MEM_f;
	/*						Const.PRIVATE_MEM, */
	lsb_int32 = i_val8;
	if (lsb_int32 < 0) {
		msb_int32 = -1;
	} else {
		msb_int32 = 0;
	}
	i_val8 = msb_int32;
	i_val7 = lsb_int32;
	/*						Const.IMMORTAL_MEM, */
	i_val6 = ((struct _staticClassFields_c *)(pointer)HEAP_REF((pointer)classData, staticClassFields_c*)) -> IMMORTAL_MEM_f;
	/*						Const.IMMORTAL_MEM, */
	lsb_int32 = i_val6;
	if (lsb_int32 < 0) {
		msb_int32 = -1;
	} else {
		msb_int32 = 0;
	}
	i_val6 = msb_int32;
	i_val5 = lsb_int32;
	/*						Const.MISSION_MEM), */
	i_val4 = ((struct _staticClassFields_c *)(pointer)HEAP_REF((pointer)classData, staticClassFields_c*)) -> MISSION_MEM_f;
	/*						Const.MISSION_MEM), */
	lsb_int32 = i_val4;
	if (lsb_int32 < 0) {
		msb_int32 = -1;
	} else {
		msb_int32 = 0;
	}
	i_val4 = msb_int32;
	i_val3 = lsb_int32;
	/*				new StorageParameters( */
	rval_m_49 = javax_safetycritical_StorageParameters_init_(sp, i_val11,
			i_val10, i_val9, i_val8, i_val7, i_val6, i_val5, i_val4, i_val3);
	if (rval_m_49 == -1) {
		;
	} else {
		fp[0] = *sp;
		return rval_m_49;
	}
	/*				new ConfigurationParameters(-1, -1, new long[] {Const.HANDLER_STACK_SIZE})); */
	if (handleNewClassIndex(sp, 14) == 0) {
		fp[0] = *sp;
		return getClassIndex((Object*) (pointer) * sp);
	}
	sp++;
	/*				new ConfigurationParameters(-1, -1, new long[] {Const.HANDLER_STACK_SIZE})); */
	i_val11 = *(sp - 1);
	/*				new ConfigurationParameters(-1, -1, new long[] {Const.HANDLER_STACK_SIZE})); */
	i_val10 = -1;
	/*				new ConfigurationParameters(-1, -1, new long[] {Const.HANDLER_STACK_SIZE})); */
	i_val9 = -1;
	/*				new ConfigurationParameters(-1, -1, new long[] {Const.HANDLER_STACK_SIZE})); */
	s_val8 = 1;
	/*				new ConfigurationParameters(-1, -1, new long[] {Const.HANDLER_STACK_SIZE})); */
	_count_ = s_val8;
	narray = (Object*) createArray(48, (uint16) _count_ FLASHARG((0)));
	if (narray == 0) {
#if defined(JAVA_LANG_THROWABLE_INIT_)
		pc = 60;
#endif
		goto throwOutOfMemory;
	}
	i_val8 = (int32) (pointer) narray;
	/*				new ConfigurationParameters(-1, -1, new long[] {Const.HANDLER_STACK_SIZE})); */
	i_val7 = i_val8;
	/*				new ConfigurationParameters(-1, -1, new long[] {Const.HANDLER_STACK_SIZE})); */
	b_val6 = 0;
	/*				new ConfigurationParameters(-1, -1, new long[] {Const.HANDLER_STACK_SIZE})); */
	i_val5 = ((struct _staticClassFields_c *)(pointer)HEAP_REF((pointer)classData, staticClassFields_c*)) -> HANDLER_STACK_SIZE_f;
	/*				new ConfigurationParameters(-1, -1, new long[] {Const.HANDLER_STACK_SIZE})); */
	lsb_int32 = i_val5;
	if (lsb_int32 < 0) {
		msb_int32 = -1;
	} else {
		msb_int32 = 0;
	}
	i_val5 = msb_int32;
	i_val4 = lsb_int32;
	/*				new ConfigurationParameters(-1, -1, new long[] {Const.HANDLER_STACK_SIZE})); */
	lsb_int32 = i_val4;
	msb_int32 = i_val5;
	index_int8 = b_val6;
	cobj_72 = HEAP_REF((pointer)(i_val7 + sizeof(Object) + 2), uint32*);
	cobj_72[index_int8 << 1] = msb_int32;
	cobj_72[(index_int8 << 1) + 1] = lsb_int32;
	/*				new ConfigurationParameters(-1, -1, new long[] {Const.HANDLER_STACK_SIZE})); */
	rval_m_73 = javax_realtime_ConfigurationParameters_init_(sp, i_val11,
			i_val10, i_val9, i_val8);
	if (rval_m_73 == -1) {
		;
	} else {
		fp[0] = *sp;
		return rval_m_73;
	}
	/*				new ConfigurationParameters(-1, -1, new long[] {Const.HANDLER_STACK_SIZE})); */
	sp--;
	hvm_arg_no_4_77 = (int32)(*sp);
	sp--;
	hvm_arg_no_3_77 = (int32)(*sp);
	sp--;
	hvm_arg_no_2_77 = (int32)(*sp);
	sp--;
	hvm_arg_no_1_77 = (int32)(*sp);
	rval_m_77 = javax_safetycritical_MissionSequencer_init_(sp, hvm_arg_no_1_77,
			hvm_arg_no_2_77, hvm_arg_no_3_77, hvm_arg_no_4_77);
	if (rval_m_77 == -1) {
		;
	} else {
		fp[0] = *sp;
		return rval_m_77;
	}
	/*	} */
	return -1;
	throwNullPointer: excep = initializeException(sp,
			JAVA_LANG_NULLPOINTEREXCEPTION,
			JAVA_LANG_NULLPOINTEREXCEPTION_INIT_);
	goto throwIt;
	throwOutOfMemory: excep = initializeException(sp,
			JAVA_LANG_OUTOFMEMORYERROR, JAVA_LANG_OUTOFMEMORYERROR_INIT_);
	goto throwIt;
	throwIt:
#if defined(JAVA_LANG_THROWABLE_INIT_)
	handler_pc = handleAthrow(&methods[520], excep, pc);
#else
	handler_pc = -1;
#endif
	sp++;
	switch (handler_pc) {
	case (unsigned short) -1: /* Not handled */
	default:
		fp[0] = *(sp - 1);
		return excep;
	}
}
\end{lstlisting}

\subsubsection{\texttt{main\_Producer\_handleAsyncEvent}}

\paragraph{Our code}\hfill
\begin{lstlisting}[firstnumber=1525]
void main_Producer_handleAsyncEvent(int32_t var1) {
	int32_t var2;
	int32_t stack1, stack2;
	stack1 = -11;
	devices_Console_write(stack1);
	stack1 = var1;
	stack1 = ((main_Producer *)  ((uintptr_t)stack1))->_signal;
	if (((java_lang_Object*)  ((uintptr_t)stack1))->classID == main_PersistentSignalID) {
		takeLock(stack1);
		main_PersistentSignal_reset(stack1);
	}
	stack1 = var1;
	stack1 = ((main_Producer *)  ((uintptr_t)stack1))->_worker;
	if (((java_lang_Object*)  ((uintptr_t)stack1))->classID == main_WorkerID) {
		javax_safetycritical_AperiodicEventHandler_release(stack1);
	}
	stack1 = -12;
	devices_Console_write(stack1);
	stack1 = 0;
	var2 = stack1;
	stack1 = var2;
	stack2 = 1000000;
	while (stack1 < stack2) {
		var2 = var2 + 1;
		var2 = var2 + -1;
		var2 = var2 + 1;
		stack1 = var2;
		stack2 = 1000000;

	}
	stack1 = -13;
	devices_Console_write(stack1);
	stack1 = var1;
	stack1 = ((main_Producer *)  ((uintptr_t)stack1))->_signal;
	if (((java_lang_Object*)  ((uintptr_t)stack1))->classID == main_PersistentSignalID) {
		takeLock(stack1);
		main_PersistentSignal_isSet(stack1, &stack1);
	}
	if (stack1 == 0) {
		stack1 = -140;
		devices_Console_write(stack1);
	} else {
		stack1 = -141;
		devices_Console_write(stack1);

	}

}
\end{lstlisting}

\paragraph{Corresponding icecap code}\hfill
\begin{lstlisting}[firstnumber=55086]
int16 main_Producer_handleAsyncEvent(int32 *fp, int32 this)
{
	int32* sp;
	int32 i_val1;
	int16 rval_m_2;
	unsigned char* cobj;
#if defined(JAVA_LANG_THROWABLE_INIT_)
	unsigned short pc;
#endif
	int16 excep;
	unsigned short handler_pc;
	int16 rval_m_13;
	int16 rval_m_24;
	int16 rval_m_30;
	int32 i_val0;
	int16 rval_m_57;
	int16 rval_m_68;
	int8 b_val1;
	int16 rval_m_78;
	int16 rval_m_88;
	int32 i;
	sp = &fp[4]; /* make room for local VM state on the stack */
	/*		devices.Console.println(-11); */
	i_val1 = (signed char)-11;
	/*		devices.Console.println(-11); */
	rval_m_2 = devices_Console_println(sp, i_val1);
	if (rval_m_2 == -1) {
		;
	}
	else
	{
		fp[0] = *sp;
		return rval_m_2;
	}
	/*		this._signal.reset(); */
	i_val1 = this;
	/*		this._signal.reset(); */
	cobj = (unsigned char *) (pointer)i_val1;
	i_val1 = ((struct _main_Producer_c *)HEAP_REF(cobj, void*)) -> _signal_f;
	/*		this._signal.reset(); */
	if (i_val1 == 0) {
#if defined(JAVA_LANG_THROWABLE_INIT_)
		pc = 13;
#endif
		goto throwNullPointer;
	}
	handleMonitorEnterExit((Object*)(pointer)i_val1, 1, sp, "");
	rval_m_13 = main_PersistentSignal_reset(sp, i_val1);
	if (rval_m_13 == -1) {
		;
	}
	else
	{
		fp[0] = *sp;
		return rval_m_13;
	}
	/*		this._worker.release(); */
	i_val1 = this;
	/*		this._worker.release(); */
	cobj = (unsigned char *) (pointer)i_val1;
	i_val1 = ((struct _main_Producer_c *)HEAP_REF(cobj, void*)) -> _worker_f;
	/*		this._worker.release(); */
	if (i_val1 == 0) {
#if defined(JAVA_LANG_THROWABLE_INIT_)
		pc = 24;
#endif
		goto throwNullPointer;
	}
	rval_m_24 = javax_safetycritical_AperiodicEventHandler_release(sp, i_val1);
	if (rval_m_24 == -1) {
		;
	}
	else
	{
		fp[0] = *sp;
		return rval_m_24;
	}
	/*		devices.Console.println(-12); */
	i_val1 = (signed char)-12;
	/*		devices.Console.println(-12); */
	rval_m_30 = devices_Console_println(sp, i_val1);
	if (rval_m_30 == -1) {
		;
	}
	else
	{
		fp[0] = *sp;
		return rval_m_30;
	}
	/*		for (int i = 0; i < 1000000; i++) { */
	i_val1 = 0;
	/*		for (int i = 0; i < 1000000; i++) { */
	i = i_val1;
	/*		for (int i = 0; i < 1000000; i++) { */
	goto L48;
	/*			i = i + 1; */
	L39:
	i = (int32)i + 1;
	/*			i = i - 1; */
	i = (int32)i + -1;
	/*		for (int i = 0; i < 1000000; i++) { */
	i = (int32)i + 1;
	/*		for (int i = 0; i < 1000000; i++) { */
	L48:
	i_val1 = (int32)i;
	/*		for (int i = 0; i < 1000000; i++) { */
	i_val0 = 1000000;
	/*		for (int i = 0; i < 1000000; i++) { */
	if (i_val1 < i_val0) {
		yieldToScheduler(sp);
		goto L39;
	}
	/*		devices.Console.println(-13); */
	i_val1 = (signed char)-13;
	/*		devices.Console.println(-13); */
	rval_m_57 = devices_Console_println(sp, i_val1);
	if (rval_m_57 == -1) {
		;
	}
	else
	{
		fp[0] = *sp;
		return rval_m_57;
	}
	/*		if (this._signal.isSet()) { */
	i_val1 = this;
	/*		if (this._signal.isSet()) { */
	cobj = (unsigned char *) (pointer)i_val1;
	i_val1 = ((struct _main_Producer_c *)HEAP_REF(cobj, void*)) -> _signal_f;
	/*		if (this._signal.isSet()) { */
	if (i_val1 == 0) {
#if defined(JAVA_LANG_THROWABLE_INIT_)
		pc = 68;
#endif
		goto throwNullPointer;
	}
	handleMonitorEnterExit((Object*)(pointer)i_val1, 1, sp, "");
	rval_m_68 = main_PersistentSignal_isSet(sp, i_val1);
	if (rval_m_68 >= 0) {
		b_val1 = rval_m_68;
	}
	else
	{
		rval_m_68 = -rval_m_68;
		fp[0] = *sp;
		return rval_m_68;
	}
	/*		if (this._signal.isSet()) { */
	if (b_val1 == 0) {
		goto L85;
	}
	/*			devices.Console.println(-141); */
	i_val1 = -141;
	/*			devices.Console.println(-141); */
	rval_m_78 = devices_Console_println(sp, i_val1);
	if (rval_m_78 == -1) {
		;
	}
	else
	{
		fp[0] = *sp;
		return rval_m_78;
	}
	/*		} else { */
	goto L92;
	/*			devices.Console.println(-140); */
	L85:
	i_val1 = -140;
	/*			devices.Console.println(-140); */
	rval_m_88 = devices_Console_println(sp, i_val1);
	if (rval_m_88 == -1) {
		;
	}
	else
	{
		fp[0] = *sp;
		return rval_m_88;
	}
	/*	} */
	L92:
	return -1;
	throwNullPointer:
	excep = initializeException(sp, JAVA_LANG_NULLPOINTEREXCEPTION, JAVA_LANG_NULLPOINTEREXCEPTION_INIT_);
	goto throwIt;
	throwIt:
#if defined(JAVA_LANG_THROWABLE_INIT_)
	handler_pc = handleAthrow(&methods[532], excep, pc);
#else
	handler_pc = -1;
#endif
	sp++;
	switch(handler_pc) {
		case (unsigned short)-1: /* Not handled */
		default:
		fp[0] = *(sp - 1);
		return excep;
	}
}
\end{lstlisting}

\subsubsection{\texttt{main\_Worker\_init}}

\paragraph{Our code}\hfill
\begin{lstlisting}[firstnumber=1574]
void main_Worker_init(int32_t var1, int32_t var2) {
	int32_t stack1, stack2, stack3, stack4, stack5, stack6, stack7, stack8, stack9, stack10, stack11, stack12, stack13;
	stack1 = var1;
	stack2 = newObject(javax_realtime_PriorityParametersID);
	stack3 = stack2;
	javax_safetycritical_PriorityScheduler_instance(&stack4);
	if (((java_lang_Object*)  ((uintptr_t)stack4))->classID == javax_safetycritical_PrioritySchedulerID) {
		javax_safetycritical_PriorityScheduler_getMaxPriority(stack4, &stack4);
	}
	javax_realtime_PriorityParameters_init(stack3, stack4);
	stack3 = newObject(javax_realtime_AperiodicParametersID);
	stack4 = stack3;
	javax_realtime_AperiodicParameters_init(stack4);
	stack4 = newObject(javax_realtime_memory_ScopeParametersID);
	stack5 = stack4;
	stack6 = 0;
	stack7 = 40000;
	stack8 = 0;
	stack9 = 20000;
	stack10 = 0;
	stack11 = 0;
	stack12 = 0;
	stack13 = 0;
	javax_realtime_memory_ScopeParameters_init(stack5, stack6, stack7, stack8, stack9, stack10, stack11, stack12, stack13);
	stack5 = newObject(javax_realtime_ConfigurationParametersID);
	stack6 = stack5;
	stack7 = -1;
	stack8 = -1;
	stack9 = newObject(java_lang_LongArray1ID);
	stack10 = stack9;
	stack11 = 0;
	stack12 = 6144;
	java_lang_LongArray1_init_J_V(stack10, stack11, stack12);
	javax_realtime_ConfigurationParameters_init(stack6, stack7, stack8, stack9);
	javax_safetycritical_AperiodicEventHandler_init(stack1, stack2, stack3, stack4, stack5);
	stack1 = var1;
	stack2 = var2;
	((main_Worker *) ((uintptr_t)stack1))->_signal = stack2;
	stack1 = var1;
	stack2 = 0;
	((main_Worker *) ((uintptr_t)stack1))->_iteration = stack2;

}
\end{lstlisting}

\paragraph{Corresponding icecap code}\hfill
\begin{lstlisting}[firstnumber=55301]
int16 main_Worker_init_(int32 *fp, int32 this, int32 event)
{
	int32* sp;
	int32 i_val12;
	int16 rval_m_5;
	int32 i_val11;
	int32 rval_5;
#if defined(JAVA_LANG_THROWABLE_INIT_)
	unsigned short pc;
#endif
	int16 excep;
	unsigned short handler_pc;
	int16 rval_m_9;
	int32 rval_9;
	int16 rval_m_13;
	int16 rval_m_21;
	int32 lsb_int32;
	int32 msb_int32;
	int32 i_val10;
	int32 i_val9;
	int32 i_val8;
	int32 i_val7;
	int32 i_val6;
	int32 i_val5;
	int32 i_val4;
	int16 rval_m_45;
	int16 s_val9;
	Object* narray;
	uint16 _count_;
	int8 b_val7;
	int8 index_int8;
	uint32* cobj_68;
	int16 rval_m_69;
	int32 hvm_arg_no_5_73;
	int32 hvm_arg_no_4_73;
	int32 hvm_arg_no_3_73;
	int32 hvm_arg_no_2_73;
	int32 hvm_arg_no_1_73;
	int16 rval_m_73;
	unsigned char* cobj;
	sp = &fp[4]; /* make room for local VM state on the stack */
	/*		super( */
	i_val12 = this;
	/*				new PriorityParameters(PriorityScheduler.instance().getMaxPriority()), */
	*sp = (int32)i_val12;
	sp++;
	if (handleNewClassIndex(sp, 63) == 0) {
		fp[0] = *sp;
		return getClassIndex((Object*) (pointer) *sp);
	}
	sp++;
	/*				new PriorityParameters(PriorityScheduler.instance().getMaxPriority()), */
	i_val12 = *(sp - 1);
	/*				new PriorityParameters(PriorityScheduler.instance().getMaxPriority()), */
	sp += 1;
	rval_m_5 = javax_safetycritical_PriorityScheduler_instance(sp);
	if (rval_m_5 == -1) {
		rval_5 = *(int32*)sp;
		i_val11 = rval_5;
	}
	else
	{
		fp[0] = *sp;
		return rval_m_5;
	}
	sp -= 1;
	/*				new PriorityParameters(PriorityScheduler.instance().getMaxPriority()), */
	if (i_val11 == 0) {
#if defined(JAVA_LANG_THROWABLE_INIT_)
		pc = 9;
#endif
		goto throwNullPointer;
	}
	sp += 1;
	rval_m_9 = javax_realtime_PriorityScheduler_getMaxPriority(sp, i_val11);
	if (rval_m_9 == -1) {
		rval_9 = *(int32*)sp;
		i_val11 = rval_9;
	}
	else
	{
		fp[0] = *sp;
		return rval_m_9;
	}
	sp -= 1;
	/*				new PriorityParameters(PriorityScheduler.instance().getMaxPriority()), */
	rval_m_13 = javax_realtime_PriorityParameters_init_(sp, i_val12, i_val11);
	if (rval_m_13 == -1) {
		;
	}
	else
	{
		fp[0] = *sp;
		return rval_m_13;
	}
	/*				new AperiodicParameters(), */
	if (handleNewClassIndex(sp, 104) == 0) {
		fp[0] = *sp;
		return getClassIndex((Object*) (pointer) *sp);
	}
	sp++;
	/*				new AperiodicParameters(), */
	i_val12 = *(sp - 1);
	/*				new AperiodicParameters(), */
	*sp = (int32)i_val12;
	sp++;
	sp -= 1;
	rval_m_21 = javax_realtime_AperiodicParameters_init_(sp);
	if (rval_m_21 == -1) {
		;
	}
	else
	{
		fp[0] = *sp;
		return rval_m_21;
	}
	/*				new StorageParameters( */
	if (handleNewClassIndex(sp, 64) == 0) {
		fp[0] = *sp;
		return getClassIndex((Object*) (pointer) *sp);
	}
	sp++;
	/*				new StorageParameters( */
	i_val12 = *(sp - 1);
	/*						Const.PRIVATE_BACKING_STORE, */
	i_val11 = ((struct _staticClassFields_c *)(pointer)HEAP_REF((pointer)classData, staticClassFields_c*)) -> PRIVATE_BACKING_STORE_f;
	/*						Const.PRIVATE_BACKING_STORE, */
	lsb_int32 = i_val11;
	if (lsb_int32 < 0) {
		msb_int32 = -1;
	} else {
		msb_int32 = 0;
	}
	i_val11 = msb_int32;
	i_val10 = lsb_int32;
	/*						Const.PRIVATE_MEM, */
	i_val9 = ((struct _staticClassFields_c *)(pointer)HEAP_REF((pointer)classData, staticClassFields_c*)) -> PRIVATE_MEM_f;
	/*						Const.PRIVATE_MEM, */
	lsb_int32 = i_val9;
	if (lsb_int32 < 0) {
		msb_int32 = -1;
	} else {
		msb_int32 = 0;
	}
	i_val9 = msb_int32;
	i_val8 = lsb_int32;
	/*						0, 0), */
	i_val7 = 0;
	i_val6 = 0;
	/*						0, 0), */
	i_val5 = 0;
	i_val4 = 0;
	/*				new StorageParameters( */
	rval_m_45 = javax_safetycritical_StorageParameters_init_(sp, i_val12, i_val11, i_val10, i_val9, i_val8, i_val7, i_val6, i_val5, i_val4);
	if (rval_m_45 == -1) {
		;
	}
	else
	{
		fp[0] = *sp;
		return rval_m_45;
	}
	/*				new ConfigurationParameters(-1, -1, new long[] {Const.HANDLER_STACK_SIZE})); */
	if (handleNewClassIndex(sp, 14) == 0) {
		fp[0] = *sp;
		return getClassIndex((Object*) (pointer) *sp);
	}
	sp++;
	/*				new ConfigurationParameters(-1, -1, new long[] {Const.HANDLER_STACK_SIZE})); */
	i_val12 = *(sp - 1);
	/*				new ConfigurationParameters(-1, -1, new long[] {Const.HANDLER_STACK_SIZE})); */
	i_val11 = -1;
	/*				new ConfigurationParameters(-1, -1, new long[] {Const.HANDLER_STACK_SIZE})); */
	i_val10 = -1;
	/*				new ConfigurationParameters(-1, -1, new long[] {Const.HANDLER_STACK_SIZE})); */
	s_val9 = 1;
	/*				new ConfigurationParameters(-1, -1, new long[] {Const.HANDLER_STACK_SIZE})); */
	_count_ = s_val9;
	narray = (Object*) createArray(48, (uint16) _count_ FLASHARG((0)));
	if (narray == 0) {
#if defined(JAVA_LANG_THROWABLE_INIT_)
		pc = 56;
#endif
		goto throwOutOfMemory;
	}
	i_val9 = (int32) (pointer) narray;
	/*				new ConfigurationParameters(-1, -1, new long[] {Const.HANDLER_STACK_SIZE})); */
	i_val8 = i_val9;
	/*				new ConfigurationParameters(-1, -1, new long[] {Const.HANDLER_STACK_SIZE})); */
	b_val7 = 0;
	/*				new ConfigurationParameters(-1, -1, new long[] {Const.HANDLER_STACK_SIZE})); */
	i_val6 = ((struct _staticClassFields_c *)(pointer)HEAP_REF((pointer)classData, staticClassFields_c*)) -> HANDLER_STACK_SIZE_f;
	/*				new ConfigurationParameters(-1, -1, new long[] {Const.HANDLER_STACK_SIZE})); */
	lsb_int32 = i_val6;
	if (lsb_int32 < 0) {
		msb_int32 = -1;
	} else {
		msb_int32 = 0;
	}
	i_val6 = msb_int32;
	i_val5 = lsb_int32;
	/*				new ConfigurationParameters(-1, -1, new long[] {Const.HANDLER_STACK_SIZE})); */
	lsb_int32 = i_val5;
	msb_int32 = i_val6;
	index_int8 = b_val7;
	cobj_68 = HEAP_REF((pointer)(i_val8 + sizeof(Object) + 2), uint32*);
	cobj_68[index_int8 << 1] = msb_int32;
	cobj_68[(index_int8 << 1) + 1] = lsb_int32;
	/*				new ConfigurationParameters(-1, -1, new long[] {Const.HANDLER_STACK_SIZE})); */
	rval_m_69 = javax_realtime_ConfigurationParameters_init_(sp, i_val12, i_val11, i_val10, i_val9);
	if (rval_m_69 == -1) {
		;
	}
	else
	{
		fp[0] = *sp;
		return rval_m_69;
	}
	/*				new ConfigurationParameters(-1, -1, new long[] {Const.HANDLER_STACK_SIZE})); */
	sp--;
	hvm_arg_no_5_73 = (int32)(*sp);
	sp--;
	hvm_arg_no_4_73 = (int32)(*sp);
	sp--;
	hvm_arg_no_3_73 = (int32)(*sp);
	sp--;
	hvm_arg_no_2_73 = (int32)(*sp);
	sp--;
	hvm_arg_no_1_73 = (int32)(*sp);
	rval_m_73 = javax_safetycritical_AperiodicEventHandler_init_(sp, hvm_arg_no_1_73, hvm_arg_no_2_73, hvm_arg_no_3_73, hvm_arg_no_4_73, hvm_arg_no_5_73);
	if (rval_m_73 == -1) {
		;
	}
	else
	{
		fp[0] = *sp;
		return rval_m_73;
	}
	/*		this._signal = event; */
	i_val12 = this;
	/*		this._signal = event; */
	i_val11 = event;
	/*		this._signal = event; */
	lsb_int32 = i_val11;
	cobj = (unsigned char *) (pointer)i_val12;
	((struct _main_Worker_c *)HEAP_REF(cobj, void*)) -> _signal_f = lsb_int32;
	/*		this._iteration = 0; */
	i_val12 = this;
	/*		this._iteration = 0; */
	i_val11 = 0;
	/*		this._iteration = 0; */
	lsb_int32 = i_val11;
	cobj = (unsigned char *) (pointer)i_val12;
	((struct _main_Worker_c *)HEAP_REF(cobj, void*)) -> _iteration_f = lsb_int32;
	/*	} */
	return -1;
	throwNullPointer:
	excep = initializeException(sp, JAVA_LANG_NULLPOINTEREXCEPTION, JAVA_LANG_NULLPOINTEREXCEPTION_INIT_);
	goto throwIt;
	throwOutOfMemory:
	excep = initializeException(sp, JAVA_LANG_OUTOFMEMORYERROR, JAVA_LANG_OUTOFMEMORYERROR_INIT_);
	goto throwIt;
	throwIt:
#if defined(JAVA_LANG_THROWABLE_INIT_)
	handler_pc = handleAthrow(&methods[534], excep, pc);
#else
	handler_pc = -1;
#endif
	sp++;
	switch(handler_pc) {
		case (unsigned short)-1: /* Not handled */
		default:
		fp[0] = *(sp - 1);
		return excep;
	}
}
\end{lstlisting}

\subsubsection{\texttt{main\_MySafelet\_getSequencer}}

\paragraph{Our code}\hfill
\begin{lstlisting}[firstnumber=1618]
void main_MySafelet_getSequencer(int32_t var1, int32_t * retVal) {
	int32_t stack1, stack2;
	stack1 = newObject(main_MainSequenceID);
	stack2 = stack1;
	main_MainSequence_init(stack2);
	*retVal = stack1;
}
\end{lstlisting}

\paragraph{Corresponding icecap code}\hfill
\begin{lstlisting}[firstnumber=54461]
int16 main_MySafelet_getSequencer(int32 *fp, int32 this)
{
	int32* sp;
	int32 i_val1;
	int16 rval_m_4;
	sp = &fp[3]; /* make room for local VM state on the stack */
	/*		return new MainSequence(); */
	if (handleNewClassIndex(sp, 110) == 0) {
		fp[0] = *sp;
		return getClassIndex((Object*) (pointer) *sp);
	}
	sp++;
	/*		return new MainSequence(); */
	i_val1 = *(sp - 1);
	/*		return new MainSequence(); */
	*sp = (int32)i_val1;
	sp++;
	sp -= 1;
	rval_m_4 = main_MainSequence_init_(sp);
	if (rval_m_4 == -1) {
		;
	}
	else
	{
		fp[0] = *sp;
		return rval_m_4;
	}
	/*		return new MainSequence(); */
	sp--;
	*((int32*)fp) = (int32)(*sp);
	return -1;
}
\end{lstlisting}

\subsubsection{\texttt{main\_PersistentSignal\_init}}

\paragraph{Our code}\hfill
\begin{lstlisting}[firstnumber=1657]
void main_PersistentSignal_init(int32_t var1) {
	int32_t stack1, stack2;
	stack1 = var1;
	java_lang_Object_init(stack1);
	stack1 = var1;
	javax_safetycritical_PriorityScheduler_instance(&stack2);
	if (((java_lang_Object*)  ((uintptr_t)stack2))->classID == javax_safetycritical_PrioritySchedulerID) {
		javax_safetycritical_PriorityScheduler_getMaxPriority(stack2, &stack2);
	}
	javax_safetycritical_Services_setCeiling(stack1, stack2);
	stack1 = var1;
	stack2 = 0;
	((main_PersistentSignal *) ((uintptr_t)stack1))->_set = stack2;

}
\end{lstlisting}

\paragraph{Corresponding icecap code}\hfill
\begin{lstlisting}[firstnumber=54512]
int16 main_PersistentSignal_init_(int32 *fp) {
	int32* sp;
	int32 i_val1;
	int16 rval_m_1;
	int16 rval_m_6;
	int32 i_val0;
	int32 rval_6;
#if defined(JAVA_LANG_THROWABLE_INIT_)
	unsigned short pc;
#endif
	int16 excep;
	unsigned short handler_pc;
	int16 rval_m_10;
	int32 rval_10;
	int16 rval_m_14;
	int8 b_val0;
	unsigned char* cobj;
	int8 lsb_int8;
	int32
	this;
	this = (int32)(*(fp + 0));
	sp = &fp[3]; /* make room for local VM state on the stack */
	/*		super(); */
	i_val1 = this;
	/*		super(); */
	*sp = (int32) i_val1;
	sp++;
	sp -= 1;
	rval_m_1 = java_lang_Object_init_(sp);
	if (rval_m_1 == -1) {
		;
	} else {
		fp[0] = *sp;
		return rval_m_1;
	}
	/*		Services.setCeiling(this, PriorityScheduler.instance().getMaxPriority()); */
	i_val1 = this;
	/*		Services.setCeiling(this, PriorityScheduler.instance().getMaxPriority()); */
	sp += 1;
	rval_m_6 = javax_safetycritical_PriorityScheduler_instance(sp);
	if (rval_m_6 == -1) {
		rval_6 = *(int32*) sp;
		i_val0 = rval_6;
	} else {
		fp[0] = *sp;
		return rval_m_6;
	}
	sp -= 1;
	/*		Services.setCeiling(this, PriorityScheduler.instance().getMaxPriority()); */
	if (i_val0 == 0) {
#if defined(JAVA_LANG_THROWABLE_INIT_)
		pc = 10;
#endif
		goto throwNullPointer;
	}
	sp += 1;
	rval_m_10 = javax_realtime_PriorityScheduler_getMaxPriority(sp, i_val0);
	if (rval_m_10 == -1) {
		rval_10 = *(int32*) sp;
		i_val0 = rval_10;
	} else {
		fp[0] = *sp;
		return rval_m_10;
	}
	sp -= 1;
	/*		Services.setCeiling(this, PriorityScheduler.instance().getMaxPriority()); */
	rval_m_14 = javax_safetycritical_Services_setCeiling(sp, i_val1, i_val0);
	if (rval_m_14 == -1) {
		;
	} else {
		fp[0] = *sp;
		return rval_m_14;
	}
	/*		this._set = false; */
	i_val1 = this;
	/*		this._set = false; */
	b_val0 = 0;
	/*		this._set = false; */
	lsb_int8 = b_val0;
	cobj = (unsigned char *) (pointer) i_val1;
	((struct _main_PersistentSignal_c *)HEAP_REF(cobj, void*)) -> _set_f = lsb_int8;
	/*	} */
	return -1;
	throwNullPointer: excep = initializeException(sp,
			JAVA_LANG_NULLPOINTEREXCEPTION,
			JAVA_LANG_NULLPOINTEREXCEPTION_INIT_);
	goto throwIt;
	throwIt:
#if defined(JAVA_LANG_THROWABLE_INIT_)
	handler_pc = handleAthrow(&methods[526], excep, pc);
#else
	handler_pc = -1;
#endif
	sp++;
	switch (handler_pc) {
	case (unsigned short) -1: /* Not handled */
	default:
		fp[0] = *(sp - 1);
		return excep;
	}
}
\end{lstlisting}

\subsubsection{\texttt{main\_MainMission\_init}}

\paragraph{Our code}\hfill
\begin{lstlisting}[firstnumber=1749]
void main_MainMission_init(int32_t var1) {
	int32_t stack1;
	stack1 = var1;
	javax_safetycritical_Mission_init(stack1);

}
\end{lstlisting}

\paragraph{Corresponding icecap code}\hfill
\begin{lstlisting}[firstnumber=53957]
int16 main_MainMission_init_(int32 *fp) {
	int32* sp;
	int32 i_val0;
	int16 rval_m_1;
	int32
	this;
	this = (int32)(*(fp + 0));
	sp = &fp[3]; /* make room for local VM state on the stack */
	/*public class MainMission extends Mission { */
	i_val0 = this;
	/*public class MainMission extends Mission { */
	*sp = (int32) i_val0;
	sp++;
	sp -= 1;
	rval_m_1 = javax_safetycritical_Mission_init_(sp);
	if (rval_m_1 == -1) {
		;
	} else {
		fp[0] = *sp;
		return rval_m_1;
	}
	/*public class MainMission extends Mission { */
	return -1;
}
\end{lstlisting}

\subsubsection{\texttt{main\_PersistentSignal\_set}}

\paragraph{Our code}\hfill
\begin{lstlisting}[firstnumber=1811]
void main_PersistentSignal_set(int32_t var1) {
	int32_t stack1, stack2;
	stack1 = var1;
	stack2 = 1;
	((main_PersistentSignal *) ((uintptr_t)stack1))->_set = stack2;
	releaseLock(var1);
}
\end{lstlisting}

\paragraph{Corresponding icecap code}\hfill
\begin{lstlisting}[firstnumber=54666]
int16 main_PersistentSignal_set(int32 *fp, int32 this)
{
	int32 i_val1;
	int8 b_val0;
	unsigned char* cobj;
	int8 lsb_int8;
	/*		this._set = true; */
	i_val1 = this;
	/*		this._set = true; */
	b_val0 = 1;
	/*		this._set = true; */
	lsb_int8 = b_val0;
	cobj = (unsigned char *) (pointer)i_val1;
	((struct _main_PersistentSignal_c *)HEAP_REF(cobj, void*)) -> _set_f = lsb_int8;
	/*	} */
	handleMonitorEnterExit((Object*)(pointer)this, 0, fp + 1, "");
	return -1;
}
\end{lstlisting}

\subsubsection{\texttt{main\_MainSequence\_getNextMission}}

\paragraph{Our code}\hfill
\begin{lstlisting}[firstnumber=1869]
void main_MainSequence_getNextMission(int32_t var1, int32_t * retVal) {
	int32_t stack1, stack2;
	stack1 = newObject(main_MainMissionID);
	stack2 = stack1;
	main_MainMission_init(stack2);
	*retVal = stack1;
}
\end{lstlisting}

\paragraph{Corresponding icecap code}\hfill
\begin{lstlisting}[firstnumber=54381]
int16 main_MainSequence_getNextMission(int32 *fp, int32 this)
{
	int32* sp;
	int32 i_val1;
	int16 rval_m_4;
	sp = &fp[3]; /* make room for local VM state on the stack */
	/*		return new MainMission();   */
	if (handleNewClassIndex(sp, 73) == 0) {
		fp[0] = *sp;
		return getClassIndex((Object*) (pointer) *sp);
	}
	sp++;
	/*		return new MainMission();   */
	i_val1 = *(sp - 1);
	/*		return new MainMission();   */
	*sp = (int32)i_val1;
	sp++;
	sp -= 1;
	rval_m_4 = main_MainMission_init_(sp);
	if (rval_m_4 == -1) {
		;
	}
	else
	{
		fp[0] = *sp;
		return rval_m_4;
	}
	/*		return new MainMission();   */
	sp--;
	*((int32*)fp) = (int32)(*sp);
	return -1;
}
\end{lstlisting}

\subsubsection{\texttt{main\_MySafelet\_init}}

\paragraph{Our code}\hfill
\begin{lstlisting}[firstnumber=1970]
void main_MySafelet_init(int32_t var1) {
	int32_t stack1;
	stack1 = var1;
	java_lang_Object_init(stack1);

}
\end{lstlisting}

\paragraph{Corresponding icecap code}\hfill
\begin{lstlisting}[firstnumber=54430]
int16 main_MySafelet_init_(int32 *fp) {
	int32* sp;
	int32 i_val0;
	int16 rval_m_1;
	int32
	this;
	this = (int32)(*(fp + 0));
	sp = &fp[3]; /* make room for local VM state on the stack */
	/*public class MySafelet implements Safelet { */
	i_val0 = this;
	/*public class MySafelet implements Safelet { */
	*sp = (int32) i_val0;
	sp++;
	sp -= 1;
	rval_m_1 = java_lang_Object_init_(sp);
	if (rval_m_1 == -1) {
		;
	} else {
		fp[0] = *sp;
		return rval_m_1;
	}
	/*public class MySafelet implements Safelet { */
	return -1;
}
\end{lstlisting}

\subsubsection{\texttt{main\_MySafelet\_initializeApplication}}

\paragraph{Our code}\hfill
\begin{lstlisting}[firstnumber=2184]
void main_MySafelet_initializeApplication(int32_t var1) {
	
}
\end{lstlisting}

\paragraph{Corresponding icecap code}\hfill
\begin{lstlisting}[firstnumber=54500]
int16 main_MySafelet_initializeApplication(int32 *fp, int32 this)
{
	/*	} */
	return -1;
}
\end{lstlisting}

\subsubsection{\texttt{main\_Worker\_handleAsyncEvent}}

\paragraph{Our code}\hfill
\begin{lstlisting}[firstnumber=2230]
void main_Worker_handleAsyncEvent(int32_t var1) {
	int32_t stack1, stack2, stack3;
	stack1 = var1;
	stack2 = stack1;
	stack2 = ((main_Worker *)  ((uintptr_t)stack2))->_iteration;
	stack3 = 1;
	stack2 = stack3 + stack2;
	((main_Worker *) ((uintptr_t)stack1))->_iteration = stack2;
	stack1 = -2;
	devices_Console_write(stack1);
	stack1 = var1;
	stack1 = ((main_Worker *)  ((uintptr_t)stack1))->_iteration;
	devices_Console_write(stack1);
	stack1 = var1;
	stack1 = ((main_Worker *)  ((uintptr_t)stack1))->_signal;
	if (((java_lang_Object*)  ((uintptr_t)stack1))->classID == main_PersistentSignalID) {
		takeLock(stack1);
		main_PersistentSignal_set(stack1);
	}

}
\end{lstlisting}

\paragraph{Corresponding icecap code}\hfill
\begin{lstlisting}[firstnumber=55584]
int16 main_Worker_handleAsyncEvent(int32 *fp, int32 this)
{
	int32* sp;
	int32 i_val2;
	int32 i_val1;
	unsigned char* cobj;
	int8 b_val0;
	int8 msb_int8;
	int32 lsb_int32;
	int16 rval_m_18;
	int16 rval_m_29;
#if defined(JAVA_LANG_THROWABLE_INIT_)
	unsigned short pc;
#endif
	int16 excep;
	unsigned short handler_pc;
	int16 rval_m_40;
	sp = &fp[3]; /* make room for local VM state on the stack */
	/*		this._iteration++; */
	i_val2 = this;
	/*		this._iteration++; */
	i_val1 = i_val2;
	/*		this._iteration++; */
	cobj = (unsigned char *) (pointer)i_val1;
	i_val1 = ((struct _main_Worker_c *)HEAP_REF(cobj, void*)) -> _iteration_f;
	/*		this._iteration++; */
	b_val0 = 1;
	/*		this._iteration++; */
	msb_int8 = b_val0;
	lsb_int32 = i_val1;
	lsb_int32 += msb_int8;
	i_val1 = lsb_int32;
	/*		this._iteration++; */
	lsb_int32 = i_val1;
	cobj = (unsigned char *) (pointer)i_val2;
	((struct _main_Worker_c *)HEAP_REF(cobj, void*)) -> _iteration_f = lsb_int32;
	/*		devices.Console.println(-2); */
	i_val2 = (signed char)-2;
	/*		devices.Console.println(-2); */
	rval_m_18 = devices_Console_println(sp, i_val2);
	if (rval_m_18 == -1) {
		;
	}
	else
	{
		fp[0] = *sp;
		return rval_m_18;
	}
	/*		devices.Console.println(this._iteration); */
	i_val2 = this;
	/*		devices.Console.println(this._iteration); */
	cobj = (unsigned char *) (pointer)i_val2;
	i_val2 = ((struct _main_Worker_c *)HEAP_REF(cobj, void*)) -> _iteration_f;
	/*		devices.Console.println(this._iteration); */
	rval_m_29 = devices_Console_println(sp, i_val2);
	if (rval_m_29 == -1) {
		;
	}
	else
	{
		fp[0] = *sp;
		return rval_m_29;
	}
	/*		this._signal.set();		 */
	i_val2 = this;
	/*		this._signal.set();		 */
	cobj = (unsigned char *) (pointer)i_val2;
	i_val2 = ((struct _main_Worker_c *)HEAP_REF(cobj, void*)) -> _signal_f;
	/*		this._signal.set();		 */
	if (i_val2 == 0) {
#if defined(JAVA_LANG_THROWABLE_INIT_)
		pc = 40;
#endif
		goto throwNullPointer;
	}
	handleMonitorEnterExit((Object*)(pointer)i_val2, 1, sp, "");
	rval_m_40 = main_PersistentSignal_set(sp, i_val2);
	if (rval_m_40 == -1) {
		;
	}
	else
	{
		fp[0] = *sp;
		return rval_m_40;
	}
	/*	} */
	return -1;
	throwNullPointer:
	excep = initializeException(sp, JAVA_LANG_NULLPOINTEREXCEPTION, JAVA_LANG_NULLPOINTEREXCEPTION_INIT_);
	goto throwIt;
	throwIt:
#if defined(JAVA_LANG_THROWABLE_INIT_)
	handler_pc = handleAthrow(&methods[535], excep, pc);
#else
	handler_pc = -1;
#endif
	sp++;
	switch(handler_pc) {
		case (unsigned short)-1: /* Not handled */
		default:
		fp[0] = *(sp - 1);
		return excep;
	}
}
\end{lstlisting}

\subsubsection{\texttt{main\_MySafelet\_immortalMemorySize}}

\paragraph{Our code}\hfill
\begin{lstlisting}[firstnumber=2301]
void main_MySafelet_immortalMemorySize(int32_t var1, int32_t * retVal_msb, int32_t * retVal_lsb) {
	int32_t stack1, stack2;
	stack1 = 0;
	stack2 = 10000;
	*retVal_lsb = stack2;
	*retVal_msb = stack1;
}
\end{lstlisting}

\paragraph{Corresponding icecap code}\hfill\\
There is no corresponding icecap code for this method.

\subsubsection{\texttt{main\_MainMission\_initialize}}

\paragraph{Our code}\hfill
\begin{lstlisting}[firstnumber=2329]
void main_MainMission_initialize(int32_t var1) {
	int32_t var2, var3;
	int32_t stack1, stack2, stack3, stack4, stack5, stack6, stack7, stack8;
	stack1 = newObject(main_PersistentSignalID);
	stack2 = stack1;
	main_PersistentSignal_init(stack2);
	var2 = stack1;
	stack1 = newObject(main_WorkerID);
	stack2 = stack1;
	stack3 = var2;
	main_Worker_init(stack2, stack3);
	var3 = stack1;
	stack1 = var3;
	if (((java_lang_Object*)  ((uintptr_t)stack1))->classID == main_WorkerID) {
		javax_safetycritical_ManagedEventHandler_register(stack1);
	}
	stack1 = newObject(main_ProducerID);
	stack2 = stack1;
	stack3 = var2;
	stack4 = var3;
	stack5 = 0;
	stack6 = 2000;
	stack7 = 0;
	stack8 = 0;
	main_Producer_init(stack2, stack3, stack4, stack5, stack6, stack7, stack8);
	if (((java_lang_Object*)  ((uintptr_t)stack1))->classID == main_ProducerID) {
		javax_safetycritical_ManagedEventHandler_register(stack1);
	}

}
\end{lstlisting}

\paragraph{Corresponding icecap code}\hfill
\begin{lstlisting}[firstnumber=53988]
int16 main_MainMission_initialize(int32 *fp, int32 this)
{
	int32* sp;
	int32 i_val7;
	int16 rval_m_4;
	int32 i_val6;
	int16 rval_m_14;
	int16 rval_m_20;
	int32 i_val5;
	int32 msi;
	int32 lsi;
	const unsigned char *data_;
	const ConstantInfo* constant_;
	int32 i_val4;
	int32 i_val3;
	int32 i_val2;
	int32 i_val1;
	int16 rval_m_34;
	int32 hvm_arg_no_1_38;
	int16 rval_m_38;
	int32 signal;
	int32 worker;
	sp = &fp[5]; /* make room for local VM state on the stack */
	/*		PersistentSignal signal = new PersistentSignal(); */
	if (handleNewClassIndex(sp, 7) == 0) {
		fp[0] = *sp;
		return getClassIndex((Object*) (pointer) *sp);
	}
	sp++;
	/*		PersistentSignal signal = new PersistentSignal(); */
	i_val7 = *(sp - 1);
	/*		PersistentSignal signal = new PersistentSignal(); */
	*sp = (int32)i_val7;
	sp++;
	sp -= 1;
	rval_m_4 = main_PersistentSignal_init_(sp);
	if (rval_m_4 == -1) {
		;
	}
	else
	{
		fp[0] = *sp;
		return rval_m_4;
	}
	/*		PersistentSignal signal = new PersistentSignal(); */
	sp--;
	signal = (int32)(*sp);
	/*		Worker worker = new Worker(signal); */
	if (handleNewClassIndex(sp, 69) == 0) {
		fp[0] = *sp;
		return getClassIndex((Object*) (pointer) *sp);
	}
	sp++;
	/*		Worker worker = new Worker(signal); */
	i_val7 = *(sp - 1);
	/*		Worker worker = new Worker(signal); */
	i_val6 = signal;
	/*		Worker worker = new Worker(signal); */
	rval_m_14 = main_Worker_init_(sp, i_val7, i_val6);
	if (rval_m_14 == -1) {
		;
	}
	else
	{
		fp[0] = *sp;
		return rval_m_14;
	}
	/*		Worker worker = new Worker(signal); */
	sp--;
	worker = (int32)(*sp);
	/*		worker.register(); */
	i_val7 = worker;
	/*		worker.register(); */
	rval_m_20 = javax_safetycritical_AperiodicEventHandler_register(sp, i_val7);
	if (rval_m_20 == -1) {
		;
	}
	else
	{
		fp[0] = *sp;
		return rval_m_20;
	}
	/*		(new Producer(signal, worker, 2000, 0)).register(); */
	if (handleNewClassIndex(sp, 92) == 0) {
		fp[0] = *sp;
		return getClassIndex((Object*) (pointer) *sp);
	}
	sp++;
	/*		(new Producer(signal, worker, 2000, 0)).register(); */
	i_val7 = *(sp - 1);
	/*		(new Producer(signal, worker, 2000, 0)).register(); */
	i_val6 = signal;
	/*		(new Producer(signal, worker, 2000, 0)).register(); */
	i_val5 = worker;
	/*		(new Producer(signal, worker, 2000, 0)).register(); */
	constant_ = &constants[96];
	data_ = (const unsigned char *) pgm_read_pointer(&constant_->data, const void **);
	msi = ((int32) pgm_read_byte(data_)) << 24;
	msi |= ((int32) pgm_read_byte(data_ +1)) << 16;
	msi |= pgm_read_byte(data_ + 2) << 8;
	msi |= pgm_read_byte(data_ + 3);
	lsi = ((int32) pgm_read_byte(data_ + 4)) << 24;
	lsi |= ((int32) pgm_read_byte(data_ + 5)) << 16;
	lsi |= pgm_read_byte(data_ + 6) << 8;
	lsi |= pgm_read_byte(data_ + 7);
	i_val4 = msi;
	i_val3 = lsi;
	/*		(new Producer(signal, worker, 2000, 0)).register(); */
	i_val2 = 0;
	i_val1 = 0;
	/*		(new Producer(signal, worker, 2000, 0)).register(); */
	rval_m_34 = main_Producer_init_(sp, i_val7, i_val6, i_val5, i_val4, i_val3, i_val2, i_val1);
	if (rval_m_34 == -1) {
		;
	}
	else
	{
		fp[0] = *sp;
		return rval_m_34;
	}
	/*		(new Producer(signal, worker, 2000, 0)).register(); */
	sp--;
	hvm_arg_no_1_38 = (int32)(*sp);
	rval_m_38 = javax_safetycritical_PeriodicEventHandler_register(sp, hvm_arg_no_1_38);
	if (rval_m_38 == -1) {
		;
	}
	else
	{
		fp[0] = *sp;
		return rval_m_38;
	}
	/*	} */
	return -1;
}
\end{lstlisting}

%%%%%%%%%%%%%%%%%%%%%%%%%%%%%%%%%%%%%%%%%%%%%%%%%%%%%%%%%%%%%%%%%%%%%%%%%%%%%%%%

\section{\texorpdfstring{\texttt{Buffer}}{Buffer}}
\label{Buffer-code-section}

\subsection{Java Code}
\label{Buffer-java-code-subsection}

\subsubsection{\texttt{BoundedBuffer.java}}
\lstinputlisting{../../workspace2/ModelBuffer/src/main/BoundedBuffer.java}

\subsubsection{\texttt{Buffer.java}}
\lstinputlisting{../../workspace2/ModelBuffer/src/main/Buffer.java}

\subsubsection{\texttt{Consumer.java}}
\lstinputlisting{../../workspace2/ModelBuffer/src/main/Consumer.java}

\subsubsection{\texttt{MainMission.java}}
\lstinputlisting{../../workspace2/ModelBuffer/src/main/MainMission.java}

\subsubsection{\texttt{MainSequence.java}}
\lstinputlisting{../../workspace2/ModelBuffer/src/main/MainSequence.java}

\subsubsection{\texttt{MySafelet.java}}
\lstinputlisting{../../workspace2/ModelBuffer/src/main/MySafelet.java}

\subsubsection{\texttt{Producer.java}}
\lstinputlisting{../../workspace2/ModelBuffer/src/main/Producer.java}

\subsection{Code generated by our prototype}
\label{Buffer-code-our-subsection}

\subsubsection{\texttt{main\_Producer\_init}}

\begin{lstlisting}[firstnumber=224]
void main_Producer_init(int32_t var1, int32_t var2, int32_t var3) {
	int32_t stack1, stack2, stack3, stack4, stack5, stack6, stack7, stack8, stack9, stack10, stack11, stack12, stack13;
	stack1 = var1;
	stack2 = newObject(javax_realtime_PriorityParametersID);
	stack3 = stack2;
	javax_safetycritical_PriorityScheduler_instance(&stack4);
	if (((java_lang_Object*)  ((uintptr_t)stack4))->classID == javax_safetycritical_PrioritySchedulerID) {
		javax_safetycritical_PriorityScheduler_getNormPriority(stack4, &stack4);
	}
	javax_realtime_PriorityParameters_init(stack3, stack4);
	stack3 = newObject(javax_realtime_PeriodicParametersID);
	stack4 = stack3;
	stack5 = newObject(javax_realtime_RelativeTimeID);
	stack6 = stack5;
	javax_realtime_RelativeTime_init(stack6);
	stack6 = newObject(javax_realtime_RelativeTimeID);
	stack7 = stack6;
	stack8 = 0;
	stack9 = 3000;
	stack10 = 0;
	javax_realtime_RelativeTime_init(stack7, stack8, stack9, stack10);
	javax_realtime_PeriodicParameters_init(stack4, stack5, stack6);
	stack4 = newObject(javax_realtime_memory_ScopeParametersID);
	stack5 = stack4;
	stack6 = 0;
	stack7 = 40000;
	stack8 = 0;
	stack9 = 20000;
	stack10 = 0;
	stack11 = 0;
	stack12 = 0;
	stack13 = 0;
	javax_realtime_memory_ScopeParameters_init(stack5, stack6, stack7, stack8, stack9, stack10, stack11, stack12, stack13);
	stack5 = newObject(javax_realtime_ConfigurationParametersID);
	stack6 = stack5;
	stack7 = -1;
	stack8 = -1;
	stack9 = newObject(java_lang_LongArray1ID);
	stack10 = stack9;
	stack11 = 0;
	stack12 = 6144;
	java_lang_LongArray1_init_J_V(stack10, stack11, stack12);
	javax_realtime_ConfigurationParameters_init(stack6, stack7, stack8, stack9);
	javax_safetycritical_PeriodicEventHandler_init(stack1, stack2, stack3, stack4, stack5);
	stack1 = var1;
	stack2 = 5;
	((main_Producer *) ((uintptr_t)stack1))->MAX_NUM_OF_OBJECTS = stack2;
	stack1 = var1;
	stack2 = 0;
	((main_Producer *) ((uintptr_t)stack1))->NUM_OF_OBJECTS = stack2;
	stack1 = var1;
	stack2 = newObject(main_Producer1ID);
	stack3 = stack2;
	stack4 = var1;
	main_Producer1_init(stack3, stack4);
	((main_Producer *) ((uintptr_t)stack1))->_switch = stack2;
	stack1 = var1;
	stack2 = var3;
	((main_Producer *) ((uintptr_t)stack1))->buffer = stack2;
	stack1 = var1;
	stack2 = var2;
	((main_Producer *) ((uintptr_t)stack1))->consume = stack2;

}
\end{lstlisting}

\subsubsection{\texttt{main\_MySafelet\_globalBackingStoreSize}}

\begin{lstlisting}[firstnumber=327]
void main_MySafelet_globalBackingStoreSize(int32_t var1, int32_t * retVal_msb, int32_t * retVal_lsb) {
	int32_t stack1, stack2;
	stack1 = 0;
	stack2 = 0;
	*retVal_lsb = stack2;
	*retVal_msb = stack1;
}
\end{lstlisting}

\subsubsection{\texttt{main\_Producer1\_run}}

\begin{lstlisting}[firstnumber=369]
void main_Producer1_run(int32_t var1) {
	int32_t stack1, stack2, stack3;
	stack1 = var1;
	stack1 = ((main_Producer1 *)  ((uintptr_t)stack1))->this0;
	stack2 = newObject(java_lang_ObjectID);
	stack3 = stack2;
	java_lang_Object_init(stack3);
	main_Producer_access0(stack1, stack2);

}
\end{lstlisting}

\subsubsection{\texttt{main\_Consumer\_handleAsyncEvent}}

\begin{lstlisting}[firstnumber=380]
void main_Consumer_handleAsyncEvent(int32_t var1) {
	int32_t var2;
	int32_t stack1;
	stack1 = var1;
	stack1 = ((main_Consumer *)  ((uintptr_t)stack1))->buffer;
	if (((java_lang_Object*)  ((uintptr_t)stack1))->classID == main_BoundedBufferID) {
		takeLock(stack1);
		main_BoundedBuffer_get(stack1, &stack1);
	}
	var2 = stack1;
	stack1 = var2;
	if (((java_lang_Object*)  ((uintptr_t)stack1))->classID == main_MainSequenceID) {
		java_lang_Object_hashCode(stack1, &stack1);
	} else if (((java_lang_Object*)  ((uintptr_t)stack1))->classID == main_ProducerID) {
		java_lang_Object_hashCode(stack1, &stack1);
	} else if (((java_lang_Object*)  ((uintptr_t)stack1))->classID == java_lang_BooleanArray5ID) {
		java_lang_Object_hashCode(stack1, &stack1);
	} else if (((java_lang_Object*)  ((uintptr_t)stack1))->classID == java_lang_BooleanArray4ID) {
		java_lang_Object_hashCode(stack1, &stack1);
	} else if (((java_lang_Object*)  ((uintptr_t)stack1))->classID == javax_safetycritical_io_ConsoleInputID) {
		java_lang_Object_hashCode(stack1, &stack1);
	} else if (((java_lang_Object*)  ((uintptr_t)stack1))->classID == java_lang_BooleanArray1ID) {
		java_lang_Object_hashCode(stack1, &stack1);
	} else if (((java_lang_Object*)  ((uintptr_t)stack1))->classID == java_lang_BooleanArray3ID) {
		java_lang_Object_hashCode(stack1, &stack1);
	} else if (((java_lang_Object*)  ((uintptr_t)stack1))->classID == javax_realtime_AperiodicParametersID) {
		java_lang_Object_hashCode(stack1, &stack1);
	} else if (((java_lang_Object*)  ((uintptr_t)stack1))->classID == java_lang_BooleanArray2ID) {
		java_lang_Object_hashCode(stack1, &stack1);
	} else if (((java_lang_Object*)  ((uintptr_t)stack1))->classID == main_MainMissionID) {
		java_lang_Object_hashCode(stack1, &stack1);
	} else if (((java_lang_Object*)  ((uintptr_t)stack1))->classID == main_Producer1ID) {
		java_lang_Object_hashCode(stack1, &stack1);
	} else if (((java_lang_Object*)  ((uintptr_t)stack1))->classID == java_io_DataOutputStreamID) {
		java_lang_Object_hashCode(stack1, &stack1);
	} else if (((java_lang_Object*)  ((uintptr_t)stack1))->classID == javax_realtime_PeriodicParametersID) {
		java_lang_Object_hashCode(stack1, &stack1);
	} else if (((java_lang_Object*)  ((uintptr_t)stack1))->classID == java_lang_Array3ID) {
		java_lang_Object_hashCode(stack1, &stack1);
	} else if (((java_lang_Object*)  ((uintptr_t)stack1))->classID == java_lang_Array2ID) {
		java_lang_Object_hashCode(stack1, &stack1);
	} else if (((java_lang_Object*)  ((uintptr_t)stack1))->classID == java_lang_Array1ID) {
		java_lang_Object_hashCode(stack1, &stack1);
	} else if (((java_lang_Object*)  ((uintptr_t)stack1))->classID == javax_realtime_ConfigurationParametersID) {
		java_lang_Object_hashCode(stack1, &stack1);
	} else if (((java_lang_Object*)  ((uintptr_t)stack1))->classID == javax_realtime_RelativeTimeID) {
		java_lang_Object_hashCode(stack1, &stack1);
	} else if (((java_lang_Object*)  ((uintptr_t)stack1))->classID == javax_safetycritical_io_ConsoleConnectionID) {
		java_lang_Object_hashCode(stack1, &stack1);
	} else if (((java_lang_Object*)  ((uintptr_t)stack1))->classID == java_io_DataInputStreamID) {
		java_lang_Object_hashCode(stack1, &stack1);
	} else if (((java_lang_Object*)  ((uintptr_t)stack1))->classID == main_ConsumerID) {
		java_lang_Object_hashCode(stack1, &stack1);
	} else if (((java_lang_Object*)  ((uintptr_t)stack1))->classID == java_lang_LongArray1ID) {
		java_lang_Object_hashCode(stack1, &stack1);
	} else if (((java_lang_Object*)  ((uintptr_t)stack1))->classID == javax_safetycritical_PrioritySchedulerID) {
		java_lang_Object_hashCode(stack1, &stack1);
	} else if (((java_lang_Object*)  ((uintptr_t)stack1))->classID == main_BoundedBufferID) {
		java_lang_Object_hashCode(stack1, &stack1);
	} else if (((java_lang_Object*)  ((uintptr_t)stack1))->classID == javax_realtime_memory_ScopeParametersID) {
		java_lang_Object_hashCode(stack1, &stack1);
	} else if (((java_lang_Object*)  ((uintptr_t)stack1))->classID == java_lang_Array5ID) {
		java_lang_Object_hashCode(stack1, &stack1);
	} else if (((java_lang_Object*)  ((uintptr_t)stack1))->classID == java_lang_Array4ID) {
		java_lang_Object_hashCode(stack1, &stack1);
	} else if (((java_lang_Object*)  ((uintptr_t)stack1))->classID == javax_safetycritical_io_ConsoleOutputID) {
		java_lang_Object_hashCode(stack1, &stack1);
	} else if (((java_lang_Object*)  ((uintptr_t)stack1))->classID == java_lang_ObjectID) {
		java_lang_Object_hashCode(stack1, &stack1);
	} else if (((java_lang_Object*)  ((uintptr_t)stack1))->classID == javax_realtime_PriorityParametersID) {
		java_lang_Object_hashCode(stack1, &stack1);
	}
	devices_Console_write(stack1);

}
\end{lstlisting}

\subsubsection{\texttt{main\_Producer\_access0}}

\begin{lstlisting}[firstnumber=603]
void main_Producer_access0(int32_t var1, int32_t var2) {
	int32_t stack1, stack2;
	stack1 = var1;
	stack2 = var2;
	((main_Producer *) ((uintptr_t)stack1))->data = stack2;

}
\end{lstlisting}

\subsubsection{\texttt{main\_MySafelet\_cleanUp}}

\begin{lstlisting}[firstnumber=749]
void main_MySafelet_cleanUp__V(int32_t var1) {
	
}
\end{lstlisting}

\subsubsection{\texttt{main\_BoundedBuffer\_put}}

\begin{lstlisting}[firstnumber=888]
void main_BoundedBuffer_put(int32_t var1, int32_t var2) {
	int32_t stack1, stack2, stack3;
	stack1 = var1;
	stack1 = ((main_BoundedBuffer *)  ((uintptr_t)stack1))->stored;
	stack2 = var1;
	stack2 = ((main_BoundedBuffer *)  ((uintptr_t)stack2))->max;
	if (stack1 != stack2) {
		stack1 = var1;
		stack2 = var1;
		stack2 = ((main_BoundedBuffer *)  ((uintptr_t)stack2))->last;
		stack3 = 1;
		stack2 = stack3 + stack2;
		stack3 = var1;
		stack3 = ((main_BoundedBuffer *)  ((uintptr_t)stack3))->max;
		stack2 = stack3 % stack2;
		((main_BoundedBuffer *) ((uintptr_t)stack1))->last = stack2;
		stack1 = var1;
		stack2 = stack1;
		stack2 = ((main_BoundedBuffer *)  ((uintptr_t)stack2))->stored;
		stack3 = 1;
		stack2 = stack3 + stack2;
		((main_BoundedBuffer *) ((uintptr_t)stack1))->stored = stack2;
		stack1 = var1;
		stack1 = ((main_BoundedBuffer *)  ((uintptr_t)stack1))->data;
		stack2 = var1;
		stack2 = ((main_BoundedBuffer *)  ((uintptr_t)stack2))->last;
		stack3 = var2;
		if (((java_lang_Object*)  ((uintptr_t)stack1))->classID == java_lang_Array3ID) {
			java_lang_Array3_store(stack1, stack2, stack3);
		} else if (((java_lang_Object*)  ((uintptr_t)stack1))->classID == java_lang_Array2ID) {
			java_lang_Array2_store(stack1, stack2, stack3);
		} else if (((java_lang_Object*)  ((uintptr_t)stack1))->classID == java_lang_Array1ID) {
			java_lang_Array1_store(stack1, stack2, stack3);
		} else if (((java_lang_Object*)  ((uintptr_t)stack1))->classID == java_lang_Array5ID) {
			java_lang_Array5_store(stack1, stack2, stack3);
		} else if (((java_lang_Object*)  ((uintptr_t)stack1))->classID == java_lang_Array4ID) {
			java_lang_Array4_store(stack1, stack2, stack3);
		}
		releaseLock(var1);
	} else {
		releaseLock(var1);
	}
}
\end{lstlisting}

\subsubsection{\texttt{main\_Producer1\_init}}

\begin{lstlisting}[firstnumber=1318]
void main_Producer1_init(int32_t var1, int32_t var2) {
	int32_t stack1, stack2;
	stack1 = var1;
	stack2 = var2;
	((main_Producer1 *) ((uintptr_t)stack1))->this = stack2;
	stack1 = var1;
	java_lang_Object_init(stack1);

}
\end{lstlisting}

\subsubsection{\texttt{main\_MySafelet\_handleStartupError}}

\begin{lstlisting}[firstnumber=1328]
void main_MySafelet_handleStartupError(int32_t var1, int32_t var2, int32_t var3, int32_t var4, int32_t * retVal) {
	int32_t stack1;
	stack1 = 0;
	*retVal = stack1;
}
\end{lstlisting}

\subsubsection{\texttt{main\_MainMission\_missionMemorySize}}

\begin{lstlisting}[firstnumber=1407]
void main_MainMission_missionMemorySize(int32_t var1, int32_t * retVal_msb, int32_t * retVal_lsb) {
	int32_t stack1, stack2;
	stack1 = 0;
	stack2 = 1000000;
	*retVal_lsb = stack2;
	*retVal_msb = stack1;
}
\end{lstlisting}

\subsubsection{\texttt{main\_MainSequence\_init}}

\begin{lstlisting}[firstnumber=1422]
void main_MainSequence_init(int32_t var1) {
	int32_t stack1, stack2, stack3, stack4, stack5, stack6, stack7, stack8, stack9, stack10, stack11, stack12;
	stack1 = var1;
	stack2 = newObject(javax_realtime_PriorityParametersID);
	stack3 = stack2;
	javax_safetycritical_PriorityScheduler_instance(&stack4);
	if (((java_lang_Object*)  ((uintptr_t)stack4))->classID == javax_safetycritical_PrioritySchedulerID) {
		javax_safetycritical_PriorityScheduler_getMaxPriority(stack4, &stack4);
	}
	javax_realtime_PriorityParameters_init(stack3, stack4);
	stack3 = newObject(javax_realtime_memory_ScopeParametersID);
	stack4 = stack3;
	stack5 = 0;
	stack6 = 702000;
	stack7 = 0;
	stack8 = 20000;
	stack9 = 0;
	stack10 = 100000;
	stack11 = 0;
	stack12 = 200000;
	javax_realtime_memory_ScopeParameters_init(stack4, stack5, stack6, stack7, stack8, stack9, stack10, stack11, stack12);
	stack4 = newObject(javax_realtime_ConfigurationParametersID);
	stack5 = stack4;
	stack6 = -1;
	stack7 = -1;
	stack8 = newObject(java_lang_LongArray1ID);
	stack9 = stack8;
	stack10 = 0;
	stack11 = 6144;
	java_lang_LongArray1_init(stack9, stack10, stack11);
	javax_realtime_ConfigurationParameters_init(stack5, stack6, stack7, stack8);
	javax_safetycritical_MissionSequencer_init(stack1, stack2, stack3, stack4);

}
\end{lstlisting}

\subsubsection{\texttt{main\_Producer\_handleAsyncEvent}}

\begin{lstlisting}[firstnumber=1676]
void main_Producer_handleAsyncEvent(int32_t var1) {
	int32_t stack1, stack2, stack3;
	stack1 = var1;
	stack1 = ((main_Producer *)  ((uintptr_t)stack1))->NUM_OF_OBJECTS;
	stack2 = 5;
	if (stack1 > stack2) {
		
	} else {
		stack1 = var1;
		stack1 = ((main_Producer *)  ((uintptr_t)stack1))->_switch;
		javax_safetycritical_ManagedMemory_executeInOuterArea(stack1);
		stack1 = var1;
		stack2 = stack1;
		stack2 = ((main_Producer *)  ((uintptr_t)stack2))->NUM_OF_OBJECTS;
		stack3 = 1;
		stack2 = stack3 + stack2;
		((main_Producer *) ((uintptr_t)stack1))->NUM_OF_OBJECTS = stack2;
		stack1 = var1;
		stack1 = ((main_Producer *)  ((uintptr_t)stack1))->buffer;
		stack2 = var1;
		stack2 = ((main_Producer *)  ((uintptr_t)stack2))->data;
		if (((java_lang_Object*)  ((uintptr_t)stack1))->classID == main_BoundedBufferID) {
			takeLock(stack1);
			main_BoundedBuffer_put(stack1, stack2);
		}
		stack1 = var1;
		stack1 = ((main_Producer *)  ((uintptr_t)stack1))->consume;
		if (((java_lang_Object*)  ((uintptr_t)stack1))->classID == main_ConsumerID) {
			javax_safetycritical_AperiodicEventHandler_release(stack1);
		}

	}
}
\end{lstlisting}

\subsubsection{\texttt{main\_MySafelet\_getSequencer}}

\begin{lstlisting}[firstnumber=1710]
void main_MySafelet_getSequencer(int32_t var1, int32_t * retVal) {
	int32_t stack1, stack2;
	stack1 = newObject(main_MainSequenceID);
	stack2 = stack1;
	main_MainSequence_init(stack2);
	*retVal = stack1;
}
\end{lstlisting}

\subsubsection{\texttt{main\_BoundedBuffer\_init}}

\begin{lstlisting}[firstnumber=1763]
void main_BoundedBuffer_init(int32_t var1) {
	int32_t stack1, stack2;
	stack1 = var1;
	java_lang_Object_init(stack1);
	stack1 = var1;
	stack2 = 5;
	((main_BoundedBuffer *) ((uintptr_t)stack1))->max = stack2;
	stack1 = var1;
	javax_safetycritical_PriorityScheduler_instance(&stack2);
	if (((java_lang_Object*)  ((uintptr_t)stack2))->classID == javax_safetycritical_PrioritySchedulerID) {
		javax_safetycritical_PriorityScheduler_getMaxPriority(stack2, &stack2);
	}
	javax_safetycritical_Services_setCeiling(stack1, stack2);
	stack1 = var1;
	stack2 = var1;
	stack2 = ((main_BoundedBuffer *)  ((uintptr_t)stack2))->max;
	java_lang_Array_newArray(stack2, &stack2);
	((main_BoundedBuffer *) ((uintptr_t)stack1))->data = stack2;
	stack1 = var1;
	stack2 = 0;
	((main_BoundedBuffer *) ((uintptr_t)stack1))->first = stack2;
	stack1 = var1;
	stack2 = 0;
	((main_BoundedBuffer *) ((uintptr_t)stack1))->last = stack2;
	stack1 = var1;
	stack2 = 0;
	((main_BoundedBuffer *) ((uintptr_t)stack1))->stored = stack2;

}
\end{lstlisting}

\subsubsection{\texttt{main\_BoundedBuffer\_isFull}}

\begin{lstlisting}[firstnumber=1820]
void main_BoundedBuffer_isFull(int32_t var1, int32_t * retVal) {
	int32_t stack1, stack2;
	stack1 = var1;
	stack1 = ((main_BoundedBuffer *)  ((uintptr_t)stack1))->stored;
	stack2 = var1;
	stack2 = ((main_BoundedBuffer *)  ((uintptr_t)stack2))->max;
	if (stack1 != stack2) {
		stack1 = 0;
		releaseLock(var1);
	} else {
		stack1 = 1;
		releaseLock(var1);
	}
	*retVal = stack1;
}
\end{lstlisting}

\subsubsection{\texttt{main\_MainMission\_init}}

\paragraph{Our code}\hfill
\begin{lstlisting}[firstnumber=1871]
void main_MainMission_init(int32_t var1) {
	int32_t stack1;
	stack1 = var1;
	javax_safetycritical_Mission_init(stack1);

}
\end{lstlisting}

\subsubsection{\texttt{main\_MainSequence\_getNextMission}}

\begin{lstlisting}[firstnumber=1983]
void main_MainSequence_getNextMission(int32_t var1, int32_t * retVal) {
	int32_t stack1, stack2;
	stack1 = newObject(main_MainMissionID);
	stack2 = stack1;
	main_MainMission_init(stack2);
	*retVal = stack1;
}
\end{lstlisting}

\subsubsection{\texttt{main\_MySafelet\_init}}

\begin{lstlisting}[firstnumber=2084]
void main_MySafelet_init(int32_t var1) {
	int32_t stack1;
	stack1 = var1;
	java_lang_Object_init(stack1);

}
\end{lstlisting}

\subsubsection{\texttt{main\_BoundedBuffer\_get}}

\begin{lstlisting}[firstnumber=2180]
void main_BoundedBuffer_get(int32_t var1, int32_t * retVal) {
	int32_t stack1, stack2, stack3;
	stack1 = var1;
	stack1 = ((main_BoundedBuffer *)  ((uintptr_t)stack1))->stored;
	if (stack1 != 0) {
		stack1 = var1;
		stack2 = var1;
		stack2 = ((main_BoundedBuffer *)  ((uintptr_t)stack2))->first;
		stack3 = 1;
		stack2 = stack3 + stack2;
		stack3 = var1;
		stack3 = ((main_BoundedBuffer *)  ((uintptr_t)stack3))->max;
		stack2 = stack3 % stack2;
		((main_BoundedBuffer *) ((uintptr_t)stack1))->first = stack2;
		stack1 = var1;
		stack2 = stack1;
		stack2 = ((main_BoundedBuffer *)  ((uintptr_t)stack2))->stored;
		stack3 = 1;
		stack2 = stack3 - stack2;
		((main_BoundedBuffer *) ((uintptr_t)stack1))->stored = stack2;
		stack1 = var1;
		stack1 = ((main_BoundedBuffer *)  ((uintptr_t)stack1))->data;
		stack2 = var1;
		stack2 = ((main_BoundedBuffer *)  ((uintptr_t)stack2))->first;
		if (((java_lang_Object*)  ((uintptr_t)stack1))->classID == java_lang_Array3ID) {
			java_lang_Array3_load(stack1, stack2, &stack1);
		} else if (((java_lang_Object*)  ((uintptr_t)stack1))->classID == java_lang_Array2ID) {
			java_lang_Array2_load(stack1, stack2, &stack1);
		} else if (((java_lang_Object*)  ((uintptr_t)stack1))->classID == java_lang_Array1ID) {
			java_lang_Array1_load(stack1, stack2, &stack1);
		} else if (((java_lang_Object*)  ((uintptr_t)stack1))->classID == java_lang_Array5ID) {
			java_lang_Array5_load(stack1, stack2, &stack1);
		} else if (((java_lang_Object*)  ((uintptr_t)stack1))->classID == java_lang_Array4ID) {
			java_lang_Array4_load(stack1, stack2, &stack1);
		}
		releaseLock(var1);
	} else {
		stack1 = 0;
		releaseLock(var1);
	}
	*retVal = stack1;
}
\end{lstlisting}

\subsubsection{\texttt{main\_MySafelet\_initializeApplication}}

\begin{lstlisting}[firstnumber=2341]
void main_MySafelet_initializeApplication(int32_t var1) {
	
}
\end{lstlisting}

\subsubsection{\texttt{main\_Consumer\_init}}

\begin{lstlisting}[firstnumber=2426]
void main_Consumer_init(int32_t var1, int32_t var2) {
	int32_t stack1, stack2, stack3, stack4, stack5, stack6, stack7, stack8, stack9, stack10, stack11, stack12, stack13;
	stack1 = var1;
	stack2 = newObject(javax_realtime_PriorityParametersID);
	stack3 = stack2;
	javax_safetycritical_PriorityScheduler_instance(&stack4);
	if (((java_lang_Object*)  ((uintptr_t)stack4))->classID == javax_safetycritical_PrioritySchedulerID) {
		javax_safetycritical_PriorityScheduler_getMaxPriority(stack4, &stack4);
	}
	javax_realtime_PriorityParameters_init(stack3, stack4);
	stack3 = newObject(javax_realtime_AperiodicParametersID);
	stack4 = stack3;
	javax_realtime_AperiodicParameters_init(stack4);
	stack4 = newObject(javax_realtime_memory_ScopeParametersID);
	stack5 = stack4;
	stack6 = 0;
	stack7 = 40000;
	stack8 = 0;
	stack9 = 20000;
	stack10 = 0;
	stack11 = 0;
	stack12 = 0;
	stack13 = 0;
	javax_realtime_memory_ScopeParameters_init(stack5, stack6, stack7, stack8, stack9, stack10, stack11, stack12, stack13);
	stack5 = newObject(javax_realtime_ConfigurationParametersID);
	stack6 = stack5;
	stack7 = -1;
	stack8 = -1;
	stack9 = newObject(java_lang_LongArray1ID);
	stack10 = stack9;
	stack11 = 0;
	stack12 = 6144;
	java_lang_LongArray1_init(stack10, stack11, stack12);
	javax_realtime_ConfigurationParameters_init(stack6, stack7, stack8, stack9);
	javax_safetycritical_AperiodicEventHandler_init(stack1, stack2, stack3, stack4, stack5);
	stack1 = var1;
	stack2 = var2;
	((main_Consumer *) ((uintptr_t)stack1))->buffer = stack2;

}
\end{lstlisting}

\subsubsection{\texttt{main\_MySafelet\_immortalMemorySize}}

\begin{lstlisting}[firstnumber=2477]
void main_MySafelet_immortalMemorySize(int32_t var1, int32_t * retVal_msb, int32_t * retVal_lsb) {
	int32_t stack1, stack2;
	stack1 = 0;
	stack2 = 0;
	*retVal_lsb = stack2;
	*retVal_msb = stack1;
}
\end{lstlisting}

\subsubsection{\texttt{main\_MainMission\_initialize}}

\begin{lstlisting}[firstnumber=2505]
void main_MainMission_initialize(int32_t var1) {
	int32_t var2, var3;
	int32_t stack1, stack2, stack3, stack4;
	stack1 = newObject(main_BoundedBufferID);
	stack2 = stack1;
	main_BoundedBuffer_init(stack2);
	var2 = stack1;
	stack1 = newObject(main_ConsumerID);
	stack2 = stack1;
	stack3 = var2;
	main_Consumer_init(stack2, stack3);
	var3 = stack1;
	stack1 = var3;
	if (((java_lang_Object*)  ((uintptr_t)stack1))->classID == main_ConsumerID) {
		javax_safetycritical_ManagedEventHandler_register(stack1);
	}
	stack1 = newObject(main_ProducerID);
	stack2 = stack1;
	stack3 = var3;
	stack4 = var2;
	main_Producer_init(stack2, stack3, stack4);
	if (((java_lang_Object*)  ((uintptr_t)stack1))->classID == main_ProducerID) {
		javax_safetycritical_ManagedEventHandler_register(stack1);
	}

}
\end{lstlisting}

\section{\texorpdfstring{\texttt{Barrier}}{Barrier}}
\label{Barrier-code-section}

\subsection{Java Code}
\label{Barrier-java-code-subsection}

\subsubsection{\texttt{Barrier.java}}
\lstinputlisting{../../workspace2/ModelBarrier/src/main/Barrier.java}

\subsubsection{\texttt{Button.java}}
\lstinputlisting{../../workspace2/ModelBarrier/src/main/Barrier.java}

\subsubsection{\texttt{FireHandler.java}}
\lstinputlisting{../../workspace2/ModelBarrier/src/main/FireHandler.java}

\subsubsection{\texttt{LaunchHandler.java}}
\lstinputlisting{../../workspace2/ModelBarrier/src/main/LaunchHandler.java}

\subsubsection{\texttt{MainMission.java}}
\lstinputlisting{../../workspace2/ModelBarrier/src/main/MainMission.java}

\subsubsection{\texttt{MainSequence.java}}
\lstinputlisting{../../workspace2/ModelBarrier/src/main/MainSequence.java}

\subsubsection{\texttt{MySafelet.java}}
\lstinputlisting{../../workspace2/ModelBarrier/src/main/MySafelet.java}

\subsection{Code generated by our prototype}
\label{Barrier-code-our-subsection}

\subsubsection{\texttt{main\_MySafelet\_globalBackingStoreSize}}

\begin{lstlisting}[firstnumber=262]
void main_MySafelet_globalBackingStoreSize(int32_t var1, int32_t * retVal_msb, int32_t * retVal_lsb) {
	int32_t stack1, stack2;
	stack1 = 0;
	stack2 = 0;
	*retVal_lsb = stack2;
	*retVal_msb = stack1;
}
\end{lstlisting}

\subsubsection{\texttt{main\_LaunchHandler\_handleAsyncEvent}}

\begin{lstlisting}[firstnumber=465]
void main_LaunchHandler_handleAsyncEvent(int32_t var1) {
	int32_t stack1;
	stack1 = -1;
	devices_Console_write(stack1);

}
\end{lstlisting}

\subsubsection{\texttt{main\_Barrier\_isOkToFire}}

\begin{lstlisting}[firstnumber=472]
void main_Barrier_isOkToFire(int32_t var1, int32_t * retVal) {
	int32_t var2, var3;
	int32_t stack1, stack2;
	stack1 = 1;
	var2 = stack1;
	stack1 = 0;
	var3 = stack1;
	stack1 = var3;
	stack2 = var1;
	stack2 = ((main_Barrier *)  ((uintptr_t)stack2))->flag;
	if (((java_lang_Object*)  ((uintptr_t)stack2))->classID == java_lang_BooleanArray5ID) {
		java_lang_BooleanArray5_length(stack2, &stack2);
	} else if (((java_lang_Object*)  ((uintptr_t)stack2))->classID == java_lang_BooleanArray4ID) {
		java_lang_BooleanArray4_length(stack2, &stack2);
	} else if (((java_lang_Object*)  ((uintptr_t)stack2))->classID == java_lang_BooleanArray1ID) {
		java_lang_BooleanArray1_length(stack2, &stack2);
	} else if (((java_lang_Object*)  ((uintptr_t)stack2))->classID == java_lang_BooleanArray3ID) {
		java_lang_BooleanArray3_length(stack2, &stack2);
	} else if (((java_lang_Object*)  ((uintptr_t)stack2))->classID == java_lang_BooleanArray2ID) {
		java_lang_BooleanArray2_length(stack2, &stack2);
	}
	while (stack1 < stack2) {
		stack1 = var1;
		stack1 = ((main_Barrier *)  ((uintptr_t)stack1))->flag;
		stack2 = var3;
		if (((java_lang_Object*)  ((uintptr_t)stack1))->classID == java_lang_BooleanArray5ID) {
			java_lang_BooleanArray5_load(stack1, stack2, &stack1);
		} else if (((java_lang_Object*)  ((uintptr_t)stack1))->classID == java_lang_BooleanArray4ID) {
			java_lang_BooleanArray4_load(stack1, stack2, &stack1);
		} else if (((java_lang_Object*)  ((uintptr_t)stack1))->classID == java_lang_BooleanArray1ID) {
			java_lang_BooleanArray1_load(stack1, stack2, &stack1);
		} else if (((java_lang_Object*)  ((uintptr_t)stack1))->classID == java_lang_BooleanArray3ID) {
			java_lang_BooleanArray3_load(stack1, stack2, &stack1);
		} else if (((java_lang_Object*)  ((uintptr_t)stack1))->classID == java_lang_BooleanArray2ID) {
			java_lang_BooleanArray2_load(stack1, stack2, &stack1);
		}
		if (!(stack1 != 0)) {
			stack1 = 0;
			var2 = stack1;
		}
		var3 = var3 + 1;
		stack1 = var3;
		stack2 = var1;
		stack2 = ((main_Barrier *)  ((uintptr_t)stack2))->flagLjava_lang_BooleanArray_;
		if (((java_lang_Object*)  ((uintptr_t)stack2))->classID == java_lang_BooleanArray5ID) {
			java_lang_BooleanArray5_length(stack2, &stack2);
		} else if (((java_lang_Object*)  ((uintptr_t)stack2))->classID == java_lang_BooleanArray4ID) {
			java_lang_BooleanArray4_length(stack2, &stack2);
		} else if (((java_lang_Object*)  ((uintptr_t)stack2))->classID == java_lang_BooleanArray1ID) {
			java_lang_BooleanArray1_length(stack2, &stack2);
		} else if (((java_lang_Object*)  ((uintptr_t)stack2))->classID == java_lang_BooleanArray3ID) {
			java_lang_BooleanArray3_length(stack2, &stack2);
		} else if (((java_lang_Object*)  ((uintptr_t)stack2))->classID == java_lang_BooleanArray2ID) {
			java_lang_BooleanArray2_length(stack2, &stack2);
		}

	}
	stack1 = var2;
	releaseLock(var1);
	*retVal = stack1;
}
\end{lstlisting}

\subsubsection{\texttt{main\_MySafelet\_cleanUp}}

\begin{lstlisting}[firstnumber=658]
void main_MySafelet_cleanUp(int32_t var1) {
	
}
\end{lstlisting}

\subsubsection{\texttt{main\_MySafelet\_handleStartupError}}

\begin{lstlisting}[firstnumber=1183]
void main_MySafelet_handleStartupError(int32_t var1, int32_t var2, int32_t var3, int32_t var4, int32_t * retVal) {
	int32_t stack1;
	stack1 = 0;
	*retVal = stack1;
}
\end{lstlisting}

\subsubsection{\texttt{main\_MainMission\_missionMemorySize}}

\begin{lstlisting}[firstnumber=1262]
void main_MainMission_missionMemorySize(int32_t var1, int32_t * retVal_msb, int32_t * retVal_lsb) {
	int32_t stack1, stack2;
	stack1 = 0;
	stack2 = 1000000;
	*retVal_lsb = stack2;
	*retVal_msb = stack1;
}  
\end{lstlisting}

\subsubsection{\texttt{main\_Barrier\_isAlreadyTriggered}}

\begin{lstlisting}[firstnumber=1270]
void main_Barrier_isAlreadyTriggered(int32_t var1, int32_t var2, int32_t * retVal) {
	int32_t stack1, stack2;
	stack1 = var1;
	stack1 = ((main_Barrier *)  ((uintptr_t)stack1))->flag;
	stack2 = var2;
	if (((java_lang_Object*)  ((uintptr_t)stack1))->classID == java_lang_BooleanArray5ID) {
		java_lang_BooleanArray5_load(stack1, stack2, &stack1);
	} else if (((java_lang_Object*)  ((uintptr_t)stack1))->classID == java_lang_BooleanArray4ID) {
		java_lang_BooleanArray4_load(stack1, stack2, &stack1);
	} else if (((java_lang_Object*)  ((uintptr_t)stack1))->classID == java_lang_BooleanArray1ID) {
		java_lang_BooleanArray1_load(stack1, stack2, &stack1);
	} else if (((java_lang_Object*)  ((uintptr_t)stack1))->classID == java_lang_BooleanArray3ID) {
		java_lang_BooleanArray3_load(stack1, stack2, &stack1);
	} else if (((java_lang_Object*)  ((uintptr_t)stack1))->classID == java_lang_BooleanArray2ID) {
		java_lang_BooleanArray2_load(stack1, stack2, &stack1);
	}
	releaseLock(var1);
	*retVal = stack1;
}
\end{lstlisting}

\subsubsection{\texttt{main\_MainSequence\_init}}

\begin{lstlisting}[firstnumber=1297]
void main_MainSequence_init(int32_t var1) {
	int32_t stack1, stack2, stack3, stack4, stack5, stack6, stack7, stack8, stack9, stack10, stack11, stack12;
	stack1 = var1;
	stack2 = newObject(javax_realtime_PriorityParametersID);
	stack3 = stack2;
	javax_safetycritical_PriorityScheduler_instance(&stack4);
	if (((java_lang_Object*)  ((uintptr_t)stack4))->classID == javax_safetycritical_PrioritySchedulerID) {
		javax_safetycritical_PriorityScheduler_getMaxPriority(stack4, &stack4);
	}
	javax_realtime_PriorityParameters_init(stack3, stack4);
	stack3 = newObject(javax_realtime_memory_ScopeParametersID);
	stack4 = stack3;
	stack5 = 0;
	stack6 = 702000;
	stack7 = 0;
	stack8 = 20000;
	stack9 = 0;
	stack10 = 100000;
	stack11 = 0;
	stack12 = 200000;
	javax_realtime_memory_ScopeParameters_init(stack4, stack5, stack6, stack7, stack8, stack9, stack10, stack11, stack12);
	stack4 = newObject(javax_realtime_ConfigurationParametersID);
	stack5 = stack4;
	stack6 = -1;
	stack7 = -1;
	stack8 = newObject(java_lang_LongArray1ID);
	stack9 = stack8;
	stack10 = 0;
	stack11 = 6144;
	java_lang_LongArray1_init(stack9, stack10, stack11);
	javax_realtime_ConfigurationParameters_init(stack5, stack6, stack7, stack8);
	javax_safetycritical_MissionSequencer_init(stack1, stack2, stack3, stack4);

}
\end{lstlisting}

\subsubsection{\texttt{main\_MySafelet\_getSequencer}}

\begin{lstlisting}[firstnumber=1551]
void main_MySafelet_getSequencer(int32_t var1, int32_t * retVal) {
	int32_t stack1, stack2;
	stack1 = newObject(main_MainSequenceID);
	stack2 = stack1;
	main_MainSequence_init(stack2);
	*retVal = stack1;
}
\end{lstlisting}

\subsubsection{\texttt{main\_MainMission\_init}}

\begin{lstlisting}[firstnumber=1666]
void main_MainMission_init(int32_t var1) {
	int32_t stack1;
	stack1 = var1;
	javax_safetycritical_Mission_init(stack1);

}
\end{lstlisting}

\subsubsection{\texttt{main\_Barrier\_init}}

\begin{lstlisting}[firstnumber=1728]
void main_Barrier_init(int32_t var1, int32_t var2, int32_t var3) {
	int32_t stack1, stack2;
	stack1 = var1;
	java_lang_Object_init(stack1);
	stack1 = var1;
	javax_safetycritical_PriorityScheduler_instance(&stack2);
	if (((java_lang_Object*)  ((uintptr_t)stack2))->classID == javax_safetycritical_PrioritySchedulerID) {
		javax_safetycritical_PriorityScheduler_getMaxPriority(stack2, &stack2);
	}
	javax_safetycritical_Services_setCeiling(stack1, stack2);
	stack1 = var1;
	stack2 = var2;
	java_lang_BooleanArray_newArray(stack2, &stack2);
	((main_Barrier *) ((uintptr_t)stack1))->flag = stack2;
	stack1 = var1;
	stack2 = var3;
	((main_Barrier *) ((uintptr_t)stack1))->e = stack2;

}
\end{lstlisting}

\subsubsection{\texttt{main\_Button\_handleAsyncEvent}}

\begin{lstlisting}[firstnumber=1765]
void main_Button_handleAsyncEvent(int32_t var1) {
	int32_t stack1;
	stack1 = var1;
	stack1 = ((main_Button *)  ((uintptr_t)stack1))->event;
	if (((java_lang_Object*)  ((uintptr_t)stack1))->classID == main_FireHandlerID) {
		javax_safetycritical_AperiodicEventHandler_release(stack1);
	} else if (((java_lang_Object*)  ((uintptr_t)stack1))->classID == main_LaunchHandlerID) {
		javax_safetycritical_AperiodicEventHandler_release(stack1);
	}

}
\end{lstlisting}

\subsubsection{\texttt{main\_Barrier\_trigger}}

\begin{lstlisting}[firstnumber=1803]
void main_Barrier_trigger(int32_t var1, int32_t var2) {
	int32_t stack1, stack2, stack3;
	stack1 = var1;
	stack1 = ((main_Barrier *)  ((uintptr_t)stack1))->flag;
	stack2 = var2;
	stack3 = 1;
	if (((java_lang_Object*)  ((uintptr_t)stack1))->classID == java_lang_BooleanArray5ID) {
		java_lang_BooleanArray5_store(stack1, stack2, stack3);
	} else if (((java_lang_Object*)  ((uintptr_t)stack1))->classID == java_lang_BooleanArray4ID) {
		java_lang_BooleanArray4_store(stack1, stack2, stack3);
	} else if (((java_lang_Object*)  ((uintptr_t)stack1))->classID == java_lang_BooleanArray1ID) {
		java_lang_BooleanArray1_store(stack1, stack2, stack3);
	} else if (((java_lang_Object*)  ((uintptr_t)stack1))->classID == java_lang_BooleanArray3ID) {
		java_lang_BooleanArray3_store(stack1, stack2, stack3);
	} else if (((java_lang_Object*)  ((uintptr_t)stack1))->classID == java_lang_BooleanArray2ID) {
		java_lang_BooleanArray2_store(stack1, stack2, stack3);
	}
	stack1 = var1;
	if (((java_lang_Object*)  ((uintptr_t)stack1))->classID == main_BarrierID) {
		takeLock(stack1);
		main_Barrier_isOkToFire(stack1, &stack1);
	}
	if (!(stack1 == 0)) {
		stack1 = var1;
		stack1 = ((main_Barrier *)  ((uintptr_t)stack1))->e;
		if (((java_lang_Object*)  ((uintptr_t)stack1))->classID == main_FireHandlerID) {
			javax_safetycritical_AperiodicEventHandler_release(stack1);
		} else if (((java_lang_Object*)  ((uintptr_t)stack1))->classID == main_LaunchHandlerID) {
			javax_safetycritical_AperiodicEventHandler_release(stack1);
		}
		stack1 = var1;
		takeLock(stack1);
		main_Barrier_reset(stack1);
	}
	releaseLock(var1);
}
\end{lstlisting}


\subsubsection{\texttt{main\_MainSequence\_getNextMission}}

\begin{lstlisting}[firstnumber=1847]
void main_MainSequence_getNextMission(int32_t var1, int32_t * retVal) {
	int32_t stack1, stack2;
	stack1 = newObject(main_MainMissionID);
	stack2 = stack1;
	main_MainMission_init(stack2);
	*retVal = stack1;
}
\end{lstlisting}

\subsubsection{\texttt{main\_MySafelet\_init}}

\begin{lstlisting}[firstnumber=1948]
void main_MySafelet_init(int32_t var1) {
	int32_t stack1;
	stack1 = var1;
	java_lang_Object_init(stack1);

}
\end{lstlisting}

\subsubsection{\texttt{main\_MySafelet\_initializeApplication}}

\begin{lstlisting}[firstnumber=2162]
void main_MySafelet_initializeApplication(int32_t var1) {
	
}
\end{lstlisting}

\subsubsection{\texttt{main\_Button\_init}}

\begin{lstlisting}[firstnumber=2208]
void main_Button_init(int32_t var1, int32_t var2, int32_t var3, int32_t var4, int32_t var5, int32_t var6) {
	int32_t stack1, stack2, stack3, stack4, stack5, stack6, stack7, stack8, stack9, stack10, stack11, stack12, stack13;
	stack1 = var1;
	stack2 = newObject(javax_realtime_PriorityParametersID);
	stack3 = stack2;
	javax_safetycritical_PriorityScheduler_instance(&stack4);
	if (((java_lang_Object*)  ((uintptr_t)stack4))->classID == javax_safetycritical_PrioritySchedulerID) {
		javax_safetycritical_PriorityScheduler_getNormPriority(stack4, &stack4);
	}
	javax_realtime_PriorityParameters_init(stack3, stack4);
	stack3 = newObject(javax_realtime_PeriodicParametersID);
	stack4 = stack3;
	stack5 = newObject(javax_realtime_RelativeTimeID);
	stack6 = stack5;
	stack7 = var5;
	stack8 = var6;
	stack9 = 0;
	javax_realtime_RelativeTime_init(stack6, stack7, stack8, stack9);
	stack6 = newObject(javax_realtime_RelativeTimeID);
	stack7 = stack6;
	stack8 = var3;
	stack9 = var4;
	stack10 = 0;
	javax_realtime_RelativeTime_init(stack7, stack8, stack9, stack10);
	javax_realtime_PeriodicParameters_init(stack4, stack5, stack6);
	stack4 = newObject(javax_realtime_memory_ScopeParametersID);
	stack5 = stack4;
	stack6 = 0;
	stack7 = 40000;
	stack8 = 0;
	stack9 = 20000;
	stack10 = 0;
	stack11 = 0;
	stack12 = 0;
	stack13 = 0;
	javax_realtime_memory_ScopeParameters_init(stack5, stack6, stack7, stack8, stack9, stack10, stack11, stack12, stack13);
	stack5 = newObject(javax_realtime_ConfigurationParametersID);
	stack6 = stack5;
	stack7 = -1;
	stack8 = -1;
	stack9 = newObject(java_lang_LongArray1ID);
	stack10 = stack9;
	stack11 = 0;
	stack12 = 6144;
	java_lang_LongArray1_init(stack10, stack11, stack12);
	javax_realtime_ConfigurationParameters_init(stack6, stack7, stack8, stack9);
	javax_safetycritical_PeriodicEventHandler_init(stack1, stack2, stack3, stack4, stack5);
	stack1 = var1;
	stack2 = var2;
	((main_Button *) ((uintptr_t)stack1))->event = stack2;

}
\end{lstlisting}

\subsubsection{\texttt{main\_FireHandler\_init}}

\begin{lstlisting}[firstnumber=2261]
void main_FireHandler_init(int32_t var1, int32_t var2, int32_t var3) {
	int32_t stack1, stack2, stack3, stack4, stack5, stack6, stack7, stack8, stack9, stack10, stack11, stack12, stack13;
	stack1 = var1;
	stack2 = newObject(javax_realtime_PriorityParametersID);
	stack3 = stack2;
	javax_safetycritical_PriorityScheduler_instance(&stack4);
	if (((java_lang_Object*)  ((uintptr_t)stack4))->classID == javax_safetycritical_PrioritySchedulerID) {
		javax_safetycritical_PriorityScheduler_getMaxPriority(stack4, &stack4);
	}
	javax_realtime_PriorityParameters_init(stack3, stack4);
	stack3 = newObject(javax_realtime_AperiodicParametersID);
	stack4 = stack3;
	javax_realtime_AperiodicParameters_init(stack4);
	stack4 = newObject(javax_realtime_memory_ScopeParametersID);
	stack5 = stack4;
	stack6 = 0;
	stack7 = 40000;
	stack8 = 0;
	stack9 = 20000;
	stack10 = 0;
	stack11 = 0;
	stack12 = 0;
	stack13 = 0;
	javax_realtime_memory_ScopeParameters_init(stack5, stack6, stack7, stack8, stack9, stack10, stack11, stack12, stack13);
	stack5 = newObject(javax_realtime_ConfigurationParametersID);
	stack6 = stack5;
	stack7 = -1;
	stack8 = -1;
	stack9 = newObject(java_lang_LongArray1ID);
	stack10 = stack9;
	stack11 = 0;
	stack12 = 6144;
	java_lang_LongArray1_init(stack10, stack11, stack12);
	javax_realtime_ConfigurationParameters_init(stack6, stack7, stack8, stack9);
	javax_safetycritical_AperiodicEventHandler_init(stack1, stack2, stack3, stack4, stack5);
	stack1 = var1;
	stack2 = var2;
	((main_FireHandler *) ((uintptr_t)stack1))->barrier = stack2;
	stack1 = var1;
	stack2 = var3;
	((main_FireHandler *) ((uintptr_t)stack1))->id = stack2;

}
\end{lstlisting}

\subsubsection{\texttt{main\_MySafelet\_immortalMemorySize}}

\begin{lstlisting}[firstnumber=2354]
void main_MySafelet_immortalMemorySize(int32_t var1, int32_t * retVal_msb, int32_t * retVal_lsb) {
	int32_t stack1, stack2;
	stack1 = 0;
	stack2 = 0;
	*retVal_lsb = stack2;
	*retVal_msb = stack1;
}
\end{lstlisting}

\subsubsection{\texttt{main\_Barrier\_reset}}

\begin{lstlisting}[firstnumber=2362]
void main_Barrier_reset(int32_t var1) {
	int32_t var2;
	int32_t stack1, stack2, stack3;
	stack1 = 0;
	var2 = stack1;
	stack1 = var2;
	stack2 = var1;
	stack2 = ((main_Barrier *)  ((uintptr_t)stack2))->flag;
	if (((java_lang_Object*)  ((uintptr_t)stack2))->classID == java_lang_BooleanArray5ID) {
		java_lang_BooleanArray5_length(stack2, &stack2);
	} else if (((java_lang_Object*)  ((uintptr_t)stack2))->classID == java_lang_BooleanArray4ID) {
		java_lang_BooleanArray4_length(stack2, &stack2);
	} else if (((java_lang_Object*)  ((uintptr_t)stack2))->classID == java_lang_BooleanArray1ID) {
		java_lang_BooleanArray1_length(stack2, &stack2);
	} else if (((java_lang_Object*)  ((uintptr_t)stack2))->classID == java_lang_BooleanArray3ID) {
		java_lang_BooleanArray3_length(stack2, &stack2);
	} else if (((java_lang_Object*)  ((uintptr_t)stack2))->classID == java_lang_BooleanArray2ID) {
		java_lang_BooleanArray2_length(stack2, &stack2);
	}
	while (stack1 < stack2) {
		stack1 = var1;
		stack1 = ((main_Barrier *)  ((uintptr_t)stack1))->flag;
		stack2 = var2;
		stack3 = 0;
		if (((java_lang_Object*)  ((uintptr_t)stack1))->classID == java_lang_BooleanArray5ID) {
			java_lang_BooleanArray5_store(stack1, stack2, stack3);
		} else if (((java_lang_Object*)  ((uintptr_t)stack1))->classID == java_lang_BooleanArray4ID) {
			java_lang_BooleanArray4_store(stack1, stack2, stack3);
		} else if (((java_lang_Object*)  ((uintptr_t)stack1))->classID == java_lang_BooleanArray1ID) {
			java_lang_BooleanArray1_store(stack1, stack2, stack3);
		} else if (((java_lang_Object*)  ((uintptr_t)stack1))->classID == java_lang_BooleanArray3ID) {
			java_lang_BooleanArray3_store(stack1, stack2, stack3);
		} else if (((java_lang_Object*)  ((uintptr_t)stack1))->classID == java_lang_BooleanArray2ID) {
			java_lang_BooleanArray2_store(stack1, stack2, stack3);
		}
		var2 = var2 + 1;
		stack1 = var2;
		stack2 = var1;
		stack2 = ((main_Barrier *)  ((uintptr_t)stack2))->flag;
		if (((java_lang_Object*)  ((uintptr_t)stack2))->classID == java_lang_BooleanArray5ID) {
			java_lang_BooleanArray5_length(stack2, &stack2);
		} else if (((java_lang_Object*)  ((uintptr_t)stack2))->classID == java_lang_BooleanArray4ID) {
			java_lang_BooleanArray4_length(stack2, &stack2);
		} else if (((java_lang_Object*)  ((uintptr_t)stack2))->classID == java_lang_BooleanArray1ID) {
			java_lang_BooleanArray1_length(stack2, &stack2);
		} else if (((java_lang_Object*)  ((uintptr_t)stack2))->classID == java_lang_BooleanArray3ID) {
			java_lang_BooleanArray3_length(stack2, &stack2);
		} else if (((java_lang_Object*)  ((uintptr_t)stack2))->classID == java_lang_BooleanArray2ID) {
			java_lang_BooleanArray2_length(stack2, &stack2);
		}

	}
	releaseLock(var1);
}
\end{lstlisting}

\subsubsection{\texttt{main\_MainMission\_initialize}}

\begin{lstlisting}[firstnumber=2437]
void main_MainMission_initialize(int32_t var1) {
	int32_t var2, var3, var4, var5;
	int32_t stack1, stack2, stack3, stack4, stack5, stack6, stack7;
	stack1 = newObject(main_LaunchHandlerID);
	stack2 = stack1;
	main_LaunchHandler_init(stack2);
	var2 = stack1;
	stack1 = var2;
	if (((java_lang_Object*)  ((uintptr_t)stack1))->classID == main_FireHandlerID) {
		javax_safetycritical_ManagedEventHandler_register(stack1);
	} else if (((java_lang_Object*)  ((uintptr_t)stack1))->classID == main_LaunchHandlerID) {
		javax_safetycritical_ManagedEventHandler_register(stack1);
	}
	stack1 = newObject(main_BarrierID);
	stack2 = stack1;
	stack3 = 2;
	stack4 = var2;
	main_Barrier_init(stack2, stack3, stack4);
	var3 = stack1;
	stack1 = newObject(main_FireHandlerID);
	stack2 = stack1;
	stack3 = var3;
	stack4 = 0;
	main_FireHandler_init(stack2, stack3, stack4);
	var4 = stack1;
	stack1 = newObject(main_FireHandlerID);
	stack2 = stack1;
	stack3 = var3;
	stack4 = 1;
	main_FireHandler_init(stack2, stack3, stack4);
	var5 = stack1;
	stack1 = var4;
	if (((java_lang_Object*)  ((uintptr_t)stack1))->classID == main_FireHandlerID) {
		javax_safetycritical_ManagedEventHandler_register(stack1);
	} else if (((java_lang_Object*)  ((uintptr_t)stack1))->classID == main_LaunchHandlerID) {
		javax_safetycritical_ManagedEventHandler_register(stack1);
	}
	stack1 = var5;
	if (((java_lang_Object*)  ((uintptr_t)stack1))->classID == main_FireHandlerID) {
		javax_safetycritical_ManagedEventHandler_register(stack1);
	} else if (((java_lang_Object*)  ((uintptr_t)stack1))->classID == main_LaunchHandlerID) {
		javax_safetycritical_ManagedEventHandler_register(stack1);
	}
	stack1 = newObject(main_ButtonID);
	stack2 = stack1;
	stack3 = var4;
	stack4 = 0;
	stack5 = 2000;
	stack6 = 0;
	stack7 = 0;
	main_Button_init(stack2, stack3, stack4, stack5, stack6, stack7);
	if (((java_lang_Object*)  ((uintptr_t)stack1))->classID == main_ButtonID) {
		javax_safetycritical_ManagedEventHandler_register(stack1);
	}
	stack1 = newObject(main_ButtonID);
	stack2 = stack1;
	stack3 = var5;
	stack4 = 0;
	stack5 = 9000;
	stack6 = 0;
	stack7 = 9000;
	main_Button_init(stack2, stack3, stack4, stack5, stack6, stack7);
	if (((java_lang_Object*)  ((uintptr_t)stack1))->classID == main_ButtonID) {
		javax_safetycritical_ManagedEventHandler_register(stack1);
	}

}
\end{lstlisting}

\subsubsection{\texttt{main\_LaunchHandler\_init}}

\begin{lstlisting}[firstnumber=2612]
void main_LaunchHandler_init(int32_t var1) {
	int32_t stack1, stack2, stack3, stack4, stack5, stack6, stack7, stack8, stack9, stack10, stack11, stack12, stack13;
	stack1 = var1;
	stack2 = newObject(javax_realtime_PriorityParametersID);
	stack3 = stack2;
	javax_safetycritical_PriorityScheduler_instance(&stack4);
	if (((java_lang_Object*)  ((uintptr_t)stack4))->classID == javax_safetycritical_PrioritySchedulerID) {
		javax_safetycritical_PriorityScheduler_getMaxPriority(stack4, &stack4);
	}
	javax_realtime_PriorityParameters_init(stack3, stack4);
	stack3 = newObject(javax_realtime_AperiodicParametersID);
	stack4 = stack3;
	javax_realtime_AperiodicParameters_init(stack4);
	stack4 = newObject(javax_realtime_memory_ScopeParametersID);
	stack5 = stack4;
	stack6 = 0;
	stack7 = 40000;
	stack8 = 0;
	stack9 = 20000;
	stack10 = 0;
	stack11 = 0;
	stack12 = 0;
	stack13 = 0;
	javax_realtime_memory_ScopeParameters_init(stack5, stack6, stack7, stack8, stack9, stack10, stack11, stack12, stack13);
	stack5 = newObject(javax_realtime_ConfigurationParametersID);
	stack6 = stack5;
	stack7 = -1;
	stack8 = -1;
	stack9 = newObject(java_lang_LongArray1ID);
	stack10 = stack9;
	stack11 = 0;
	stack12 = 6144;
	java_lang_LongArray1_init(stack10, stack11, stack12);
	javax_realtime_ConfigurationParameters_init(stack6, stack7, stack8, stack9);
	javax_safetycritical_AperiodicEventHandler_init(stack1, stack2, stack3, stack4, stack5);

}
\end{lstlisting}

\subsubsection{\texttt{main\_FireHandler\_handleAsyncEvent}}

\begin{lstlisting}[firstnumber=2759]
void main_FireHandler_handleAsyncEvent(int32_t var1) {
	int32_t stack1, stack2;
	stack1 = var1;
	stack1 = ((main_FireHandler *)  ((uintptr_t)stack1))->id;
	devices_Console_write(stack1);
	stack1 = var1;
	stack1 = ((main_FireHandler *)  ((uintptr_t)stack1))->barrier;
	stack2 = var1;
	stack2 = ((main_FireHandler *)  ((uintptr_t)stack2))->id;
	if (((java_lang_Object*)  ((uintptr_t)stack1))->classID == main_BarrierID) {
		takeLock(stack1);
		main_Barrier_isAlreadyTriggered(stack1, stack2, &stack1);
	}
	if (stack1 == 0) {
		stack1 = var1;
		stack1 = ((main_FireHandler *)  ((uintptr_t)stack1))->barrier;
		stack2 = var1;
		stack2 = ((main_FireHandler *)  ((uintptr_t)stack2))->id;
		if (((java_lang_Object*)  ((uintptr_t)stack1))->classID == main_BarrierID) {
			takeLock(stack1);
			main_Barrier_trigger(stack1, stack2);
		}

	} else {
		
	}
}
\end{lstlisting}

\end{multicols}
\end{landscape}