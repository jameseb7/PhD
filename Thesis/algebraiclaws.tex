This appendix contains algebraic laws that are not specific to our
compilation strategy but are used in our compilation strategy and the
proofs of our compilation rules.

% proved in Isabelle
\begin{law}[alt-intro1]
  \label{alt-intro-law}
  If $x : T$ is a state variable,
  \begin{circus}
    A
    =
    \circif {} \circelse_{i \in T} x = i \circthen A \circfi
  \end{circus}
\end{law}

% proved in Isabelle
\begin{law}[alt-intro2]
  \label{alt-intro2-law}
  \begin{circus}
    A
    =
    \circblockbegin
    \circif g \circthen A \\
    {} \circelse \lnot g \circthen A \\
    \circfi
    \circblockend
  \end{circus}
\end{law}

% proved in Isabelle
\begin{law}[alt-seq-dist]
  \label{alt-seq-dist-law}
  \begin{circus}
    \circif {} \circelse_{i} g_i \circthen A_i \circfi \circseq B
    =
    \circif {} \circelse_{i} g_i \circthen A_i \circseq B \circfi
  \end{circus}
\end{law}

% proved in Isabelle
\begin{law}[alt-assump-intro]
  \label{alt-assump-intro-law}
  \begin{circus}
    \circif {} \circelse_{i} g_i \circthen A_i \circfi
    =
    \circif {} \circelse_{i} g_i \circthen \{g_i\} \circseq A_i \circfi
  \end{circus}
\end{law}

% proved in Isabelle
\begin{law}[assump-assign-dist]
  \label{assump-assign-dist-law}
  If $g_1 \implies g_2[e/x]$ then,
  \begin{circus}
    \{ g_1 \} \circseq x := e
    =
    \{ g_1 \} \circseq x := e \circseq \{ g_2 \}
  \end{circus}
\end{law}

% proved in Isabelle
\begin{law}[assump-seq-subst]
  \label{assump-seq-subst-law}
  \begin{circus}
    \{x = e\} \circseq A
    =
    \{x = e\} \circseq A[e/x]
  \end{circus}
\end{law}

% proved in Isabelle
\begin{law}[assump-assign-subst]
  \label{assump-assign-subst-law}
  \begin{circus}
    \{x = e\} \circseq y := f
    \circrefines_A
    \{x = e\} \circseq y := f[e/x]
  \end{circus}
\end{law}

% Proved in Isabelle (similar to Law[assign-seq-dist] in Alvaro's thesis)
\begin{law}[assign-seq-subst]
  \label{assign-seq-subst-law}
  If $x$ is free in $e$ then,
  \begin{circus}
    x := e \circseq A
    =
    x := e \circseq A[e/x]
  \end{circus}
\end{law}

% Proved in Isabelle
\begin{law}[assump-alt-guard-replace]
  \label{assump-alt-guard-replace-law}
  If $\forall i @ h \land g1_i \iff h \land g2_i$ then,
  \begin{circus}
    \{h\} \circseq \circif {} \circelse_{i} g1_i \circthen A_i \circfi
    =
    \{h\} \circseq \circif {} \circelse_{i} g2_i \circthen A_i \circfi
  \end{circus}
\end{law}

% Proved in Isabelle (similar to Law[assump-alt-dist] in Alvaro's thesis)
\begin{law}[assump-alt-dist]
  \label{assump-alt-dist-law}
  \begin{circus}
    \{h\} \circseq \circif {} \circelse_{i} g_i \circthen A_i \circfi
    =
    \{h\} \circseq \circif {} \circelse_{i} g_i \circthen \{h\} \circseq A_i \circfi
  \end{circus}
\end{law}

% Proved in Isabelle
\begin{law}[assign-assump-intro]
  \label{assign-assump-intro-law}
  If $x$ is free in $e$ then,
  \begin{circus}
    x := e = x := e \circseq \{x = e\}
  \end{circus}
\end{law}

% Proved in Isabelle
\begin{law}[assump-assign-intro]
  \label{assump-assign-intro-law}
  \begin{circus}
    \{ x = e \} = \{ x = e \} \circseq x := e
  \end{circus}
\end{law}

% Proved in Isabelle
\begin{law}[assump-assign-cond-collapse-true]
  \label{assump-assign-cond-collapse-true-law}
  \begin{circus}
    \{g\} \circseq x := \IF g \THEN e \ELSE f
    \circrefines_A
    \{g\} \circseq x := e
  \end{circus}
\end{law}

% Proved in Isabelle
\begin{law}[assump-assign-cond-collapse-false]
  \label{assump-assign-cond-collapse-false-law}
  \begin{circus}
    \{ \lnot g \} \circseq x := \IF g \THEN e \ELSE f
    \circrefines_A
    \{ \lnot g \} \circseq x := f
  \end{circus}
\end{law}

% Proved in Isabelle
\begin{law}[assump-alt-collapse1]
  \label{assump-alt-collapse1-law}
  \begin{circus}
    \{g\} \circseq
    \circblockbegin
    \circif g \circthen A \\
    {} \circelse \lnot g \circthen B \\
    \circfi
    \circblockend
    =
    \{g\} \circseq A
  \end{circus}
\end{law}

% Proved in Isabelle
\begin{law}[assump-alt-collapse2]
  \label{assump-alt-collapse2-law}
  If $e \in I$ then,
  \begin{circus}
    \{ x = e \} \circseq
    \circblockbegin
    \circif {} \circelse_{i \in I} x = i \circthen A_i \\
    \circfi
    \circblockend
    =
    \{ x = e \} \circseq A_e
  \end{circus}
\end{law}

% Proved in Isabelle
\begin{law}[assump-alt-intro]
  \label{assump-alt-intro-law}
  \begin{circus}
    \{ \bigvee i \in I @ g_i \} \circseq A = \circif {} \circelse_{i \in I} g_i \circthen \{ g_i \} \circseq A \circfi
  \end{circus}
\end{law}

% Proved in Isabelle
\begin{law}[schema-pre-assump-intro]
  \label{schema-pre-assump-intro-law}
  \begin{circus}
    \lschexpract S \rschexpract = \{ \pre S \} \circseq \lschexpract S \rschexpract
  \end{circus}
\end{law}

% Proved in Isabelle (similar to Law C.53 in Marcel's thesis)
\begin{law}[assump-schema-dist]
  \label{assump-schema-dist-law}
  If $g \land p \implies h'$
  \begin{circus}
    \{ g \} \circseq \lschexpract [ decl | p ] \rschexpract
    \circrefines_A
    \lschexpract [ decl | p ] \rschexpract \circseq \{ h \}
  \end{circus}
\end{law}

% TODO: Needs Isabelle proof (similar to Law 18.4 in Programming from Specifications)
\begin{law}[alt-branch-elim]
  \label{alt-branch-elim-law}
  \begin{circus}
    \begin{array}{l}
      \circif \cdots \\
      {} \circelse_i g_i \circthen A_i \\
      {} \circelse h \circthen \Chaos \\
      {} \cdots {} \\
      \circfi
    \end{array}
    =
    \begin{array}{l}
      \circif \cdots \\
      {} \circelse_i g_i \circthen A_i \\
      {} \cdots {} \\
      \circfi
    \end{array}
  \end{circus}
  provided the guards of the alternation are disjoint.
\end{law}


% Definition B.28 in Marcel's thesis
\begin{law}[copy-rule]
  \label{copy-rule-law}
  If $N$ is the name of an action in the current process then,
  \begin{circus}
    N(e) = B(N)(e)
  \end{circus}
  where $B$ is a function that returns the body of an action given its
  name.
\end{law}

% Definition B.37 in Marcel's thesis
\begin{law}[val-def]
  \label{val-def-law}
  If $x$ is free in $e$ then,
  \begin{circus}
    (\circval x : T \circspot A)(e)
    =
    (\circvar x : T \circspot x := e \circspot A)
  \end{circus}
\end{law}

% Law C.26 in Marcel's thesis
\begin{law}[assump-conj]
  \label{assump-conj-law}
  \begin{circus}
    \{ g_1 \} \circseq \{ g_2 \} = \{ g_1 \land g_2 \}
  \end{circus}
\end{law}

% Law C.27 in Marcel's thesis
\begin{law}[assump-assump-intro]
  \label{assump-assump-intro-law}
  If $g \implies g_1$ then,
  \begin{circus}
    \{g\} = \{g\} \circseq \{g_1\}
  \end{circus}
\end{law}

% Law c.28 in Marcel's thesis
\begin{law}[schema-assump-intro]
  \label{schema-assump-intro-law}
  \begin{circus}
    [\Delta State; i? : T_i; o! : T_o | p \land assump'] \\
    {} = {} \\
    [\Delta State; i? : T_i; o! : T_o | p \land assump'] \circseq
    \{ assump \} \\
  \end{circus}
\end{law}

% Law C.32 in Marcel's thesis
\begin{law}[assump-guard-elim1]
  \label{assump-guard-elim1-law}
  If $g_1 \implies g_2$ then,
  \begin{circus}
    \{g_1\} \circseq (\lcircguard g_2 \rcircguard \circguard A)
    =
    \{g_1\} \circseq A
  \end{circus}
\end{law}

% Law C.33 in Marcel's thesis
\begin{law}[assump-guard-elim2]
  \label{assump-guard-elim2-law}
  If $g_1 \implies \lnot g_2$ then,
  \begin{circus}
    \{g_1\} \circseq (\lcircguard g_2 \rcircguard \circguard A)
    =
    \{g_1\} \circseq \Stop
  \end{circus}
\end{law}

% Law C.35 in Marcel's thesis
\begin{law}[assump-elim]
  \label{assump-elim-law}
  \begin{circus}
    \{g\} \circrefines_A \Skip
  \end{circus}
\end{law}

% Law C.36 in Marcel's thesis
\begin{law}[assump-subst]
  \label{assump-subst-law}
  If $g_1 \implies g_2$ then,
  \begin{circus}
    \{g_1\} \circrefines_A \{g_2\}
  \end{circus}
\end{law}
  
% Law C.37 in Marcel's thesis
\begin{law}[assump-extchoice-distr]
  \label{assump-extchoice-distr-law}
  \begin{circus}
    \{g\} \circseq (A \extchoice B)
    =
    (\{g\} \circseq A) \extchoice (\{g\} \circseq B)
  \end{circus}
\end{law}

% Law C.46 in Marcel's thesis
\begin{law}[assump-output-prefix-dist]
  \label{assump-output-prefix-dist-law}
  \begin{circus}
    \{ g \} \circseq c!x \then A = \{ g \} \circseq c!x \then \{ g \} \circseq A
  \end{circus}
\end{law}
  
% Law C.48 in Marcel's thesis
\begin{law}[assump-input-prefix-dist]
  \label{assump-input-prefix-dist-law}
  If $x$ is not free in $g$ then,
  \begin{circus}
    \{ g \} \circseq c?x \then A = \{ g \} \circseq c?x \then \{ g \} \circseq A
  \end{circus}
\end{law}

% Law C.67 in Marcel's thesis
\begin{law}[true-guard]
  \label{true-guard-law}
  \begin{circus}
    \lcircguard \true \rcircguard \circguard A = A
  \end{circus}
\end{law}

% Law C.68 in Marcel's thesis
\begin{law}[false-guard]
  \label{false-guard-law}
  \begin{circus}
    \lcircguard \false \rcircguard \circguard A = \Stop
  \end{circus}
\end{law}

% Law C.71 in Marcel's thesis
\begin{law}[schema-seq-intro]
  \label{schema-seq-intro-law}
  \begin{circus}
    [\Delta S_1; \Delta S_2; i? : T | preS_1 \land preS_2 \land CS_1 \land CS_2] \\
    \circrefines_A \\
    [\Delta S_1; \Xi S_2; i? : T | preS_1 \land CS_1] \circseq [\Xi S_1; \Delta S_2; i? : T | preS_2 \land CS_2]
  \end{circus}
  provided
  \begin{itemize}
  \item $\alpha(S_1) \cap \alpha(S_2) = \emptyset$
  \item $FV(preS_1) \subseteq \alpha(S_1) \cup \{i?\}$
  \item $FV(preS_2) \subseteq \alpha(S_2) \cup \{i?\}$
  \item $DFV(CS_1) \subseteq \alpha(S_1')$
  \item $DFV(CS_2) \subseteq \alpha(S_2')$
  \item $UDFV(CS_2) \cap DFV(CS_1) = \emptyset$
  \end{itemize}
  where $\alpha(S)$ is the alphabet of a schema $S$, $FV(P)$ is the
  free variable set of the predicate $P$, and $DFV(P)$ and $UDFV(P)$
  are the dashed and undashed free variables of a predicated $P$.
\end{law}

% Law C.75 in Marcel's thesis
\begin{law}[schema-refinement]
  \label{schema-refinement-law}
  \begin{circus}
    \lschexpract [ decls | P_1 ] \rschexpract
    \circrefines_A
    \lschexpract [ decls | P_2 ] \rschexpract
  \end{circus}
  provided
  \begin{itemize}
  \item $\pre [ decls | P_1 ] \implies \pre [ decls | P_2 ]$
  \item $(\pre [ decls | P_1 ] \land P_2) \implies P_1$
  \end{itemize}
\end{law}

% From Law C.101 in Marcel's thesis
\begin{law}[input-prefix-seq-assoc]
  \label{input-prefix-seq-assoc-law}
  If $v$ is not free in $B$ then,
  \begin{circus}
    c?v \then (A \circseq B) = (c?v \then A) \circseq B
  \end{circus}
\end{law}

% Law C.110 in Marcel's thesis
\begin{law}[extchoice-comm]
  \label{extchoice-comm-law}
  \begin{circus}
    A \extchoice B = B \extchoice A
  \end{circus}
\end{law}

% From Law C.112 in Marcel's thesis
\begin{law}[extchoice-seq-distr]
  \label{extchoice-seq-distr-law}
  \begin{circus}
    (\extchoice i \circspot c_i \then A_i) \circseq B = (\extchoice i \circspot c_i \then A_i \circseq B) 
  \end{circus}
\end{law}

% Law C.114 in Marcel's thesis
\begin{law}[extchoice-unit]
  \label{extchoice-unit-law}
  \begin{circus}
    \Stop \extchoice A = A
  \end{circus}
\end{law}

% Law C.128 in Marcel's thesis
\begin{law}[rec-unfold]
  \label{rec-unfold-law}
  \begin{circus}
    \circmu X \circspot F(X) = F(\circmu X \circspot F(X))
  \end{circus}
\end{law}

% Law C.129 in Marcel's thesis
\begin{law}[rec-least-fixed-point]
  \label{rec-least-fixed-point-law}
  If $F(Y) \circrefines_A Y$ then,
  \begin{circus}
    \circmu X \circspot F(X) \circrefines_A Y
  \end{circus}
\end{law}

% Law C.132A in Marcel's thesis
\begin{law}[seq-unitl]
  \label{seq-unitl-law}
  \begin{circus}
    \Skip \circseq A = A
  \end{circus}
\end{law}

% Law C.132B in Marcel's thesis
\begin{law}[seq-unitr]
  \label{seq-unitr-law}
  \begin{circus}
     A \circseq \Skip = A
  \end{circus}
\end{law}

% Law C.134 in Marcel's thesis
\begin{law}[seq-zerol]
  \label{seq-zerol-law}
  \begin{circus}
    \Chaos \circseq A = \Chaos
  \end{circus}
\end{law}

% From Corollary 3.1 in Marcel's thesis and the deifinition of hiding
\begin{law}[schema-hide-var-conv]
  \label{schema-hide-var-conv-law}
  If $x$ and $x'$ have type $T$ in $S$ then,
  \begin{circus}
    \lschexpract S \hide (x,x') \rschexpract
    =
    (\circvar x : T \circspot \lschexpract S \rschexpract)
  \end{circus}
\end{law}
  
% Law[var-seq-ext-left] in Alvaro's thesis (from C.137 in Marcel's thesis)
\begin{law}[var-seq-extl]
  \label{var-seq-extl-law}
  If $x$ is not free in $A_1$ then,
  \begin{circus}
    A \circseq (\circvar x : T \circspot B)
    =
    (\circvar x : T \circspot A \circseq B)
  \end{circus}
\end{law}

% Law[var-seq-ext-right] in Alvaro's thesis (from C.137 in Marcel's thesis)
\begin{law}[var-seq-extr]
  \label{var-seq-extr-law}
  If $x$ is not free in $A_1$ then,
  \begin{circus}
    (\circvar x : T \circspot A) \circseq B
    =
    (\circvar x : T \circspot A \circseq B)
  \end{circus}
\end{law}

% Law[var-assign-intro] in Alvaro's thesis
\begin{law}[var-assign-intro]
  \label{var-assign-intro-law}
  If $x$ is not free in $A$ then,
  \begin{circus}
    A
    =
    \circvar x : T \circspot x := e \circseq A
  \end{circus}
\end{law}

% Law[var-prefix-ext] in Alvaro's thesis
\begin{law}[input-prefix-var-ext]
  \label{input-prefix-var-ext-law}
  If $x$ and $y$ are distinct then,
  \begin{circus}
    \circvar x : T \circspot c?y \then A
    =
    c?y \then (\circvar x : T \circspot A)
  \end{circus}
\end{law}

% Law[assign-schema-conv] in Alvaro's thesis
\begin{law}[assign-schema-conv]
  \label{assign-schema-conv-law}
  \begin{circus}
    \{ inv \} \circseq v := e
    =
    [\Delta S | vs' = vs \land v' = e]
  \end{circus}
  where
  \begin{itemize}
  \item $S == [dv; dvs | inv]$
  \item $dv$ is the declaration of $v$
  \item $dvs$ are the declarations of the variables $vs$
  \item $inv \implies
    inv \land (\exists dv; dvs @
    inv' \land vs' = vs land v' = e)$
  \end{itemize}
\end{law}

% Law[assign-seq-com] in Alvaro's thesis
\begin{law}[assign-seq-dist]
  \label{assign-seq-dist-law}
  If $x$ is not written to by $A$ then,
  \begin{circus}
    x := e \circseq A
    =
    A[e/x] \circseq x := e
  \end{circus}
\end{law}

% Law[var-rename] in Alvaro's thesis
\begin{law}[var-rename]
  \label{var-rename-law}
  If $y$ is not free in $A$ then,
  \begin{circus}
    (\circvar x : T \circspot A) = (\circvar y : T \circspot A[y/x])
  \end{circus}
\end{law}

% Law scompC in ZRC paper
\begin{law}[schema-comp-seq-conv]
  \label{schema-comp-seq-conv-law}
  \begin{circus}
    \lschexpract S_1 \comp S_2 \rschexpract \circrefines_A \lschexpract S_1 \rschexpract \circseq \lschexpract S_2 \rschexpract
  \end{circus}
  provided $S_1$ and $S_2$ act over the same state, have no common
  output variables, and $pre_C \land S_1 \implies pre_2'$, where
  \begin{itemize}
  \item $pre_C = \pre(S_1 \comp S_2) \land inv \land t \land t_1 \land t_2$;
  \item $\pre~S_2 = pre_2 \land inv \land t \land t_2$;
  \item $t$, $t_1$ and $t_2$ are the restriticions imposed by the
    state variables, the inputs of $S_1$, and the inputs of $S_2$,
    respectively. 
  \end{itemize}
\end{law}

% From Definition B.40 in Marcel's thesis and Law vrbI in ZRC paper
\begin{law}[schema-var-intro]
  \label{schema-var-intro-law}
  If $x$ is not declared in $decls$ then,
  \begin{circus}
    \lschexpract [decls | pred] \rschexpract
    =
    \circvar x : T \circspot \lschexpract [decls; x, x' : T | pred] \rschexpract
  \end{circus}
\end{law}

% From the definition of process refinement (Definition 3.2 in Marcel's thesis)
\begin{law}[action-intro]
  \label{action-intro-law}
  If $N$ is not referenced in the body of $P$ then,
  \begin{circus}
    \begin{array}{l}
      \circprocess P \circdef \circbegin \\
      \t1 {} \cdots {} \\
      \t1 \circstate S \\
      \t1 {} \cdots {} \\
      \t1 \circspot A \\
      \t1 \circend
    \end{array}
    =
    \begin{array}{l}
      \circprocess P \circdef \circbegin \\
      \t1 {} \cdots {} \\
      \t1 \circstate S \\
      \t1 {} \cdots {} \\
      \t1 N \circdef B \\
      \t1 {} \cdots {} \\
      \t1 \circspot A \\
      \t1 \circend
    \end{array}
  \end{circus}
\end{law}

% From definition B.40 in Marcel's thesis
\begin{law}[schema-action-fixed-var-intro]
  \label{schema-action-fixed-var-intro-law}
  If $x$ and $x'$ are not in the declared variables of $S$ but $x$ is
  in scope then,
  \begin{circus}
    \lschexpract S \rschexpract = \lschexpract S \land x' = x \rschexpract
  \end{circus}
\end{law}

% From Exercise 2.6.4(2) in the UTP book with
% F(X) = \{g\} \circseq f(X)
% G(X) = X \circseq \{g\}
% H(X) = f(X)
\begin{law}[rec-assump-distr]
  \label{rec-assump-distr-law}
  If $\{g\} \circseq F(X \circseq \{g\}) \circrefines_A F(X) \circseq \{g\}$ then,
  \begin{circus}
    (\circmu X \circspot \{g\} \circseq F(X))
    \circrefines_A
    (\circmu X \circspot F(X)) \circseq \{g\}
  \end{circus}
\end{law}

% From the dual of Exercise 2.6.4 in the UTP book and Theorem 7.15
% from Simon's reactive contracts paper with
% F(X) = \{g\} \circseq f(X)
% G(X) = \{g\} \circseq X 
% H(X) = f(X)
\begin{law}[assump-rec-distl]
  \label{assump-rec-distl-law}
  If, for all actions $X$ and $n \in \nat$,
  \[\begin{array}{l}
      (F(X) \land tr \leq tr' \land \# (tr'-tr) < n + 1) \\
      {} = {} \\
      (F(X \land tr \leq tr' \land \# (tr'-tr) < n) \land tr \leq tr' \land \# (tr'-tr) < n + 1)
    \end{array}\]
  and $\{g\} \circseq F(X) \circrefines_A \{g\} \circseq F(\{g\} \circseq X)$ then,
  \begin{circus}
    \{g\} \circseq (\circmu X \circspot F(X))
    \circrefines_A
    (\circmu X \circspot \{g\} \circseq F(X))
  \end{circus}
\end{law}

% From Definition B.36 in Marcel's thesis and Laws 2.9L1 and 2.9L2 in the UTP book
\begin{law}[var-var-comm]
  \label{var-var-comm-law}
  \begin{circus}
    \circvar x : X \circspot \circvar y : Y \circspot A = \circvar y : Y \circspot \circvar x : X \circspot A
  \end{circus}
\end{law}

% From Exercise 2.6.4(2) in the UTP book with
% F(X) = A \circseq X
% G(X) = X \circseq B
% H(X) = A \circseq X
\begin{law}[rec-action-intro]
  \label{rec-action-intro-law}
  For any action $B$,
  \begin{circus}
    (\circmu X \circspot A \circseq X) \circrefines_A (\circmu X \circspot A \circseq X) \circseq B
  \end{circus}
\end{law}

% TODO: needs proof (should follow from Definition B.40 in Marcel's
% thesis, and Law vrbI in ZRC or a generalisation of
% Law[var-assign-intro] in Alvaro's thesis)
\begin{law}[var-schema-intro]
  \label{var-schema-intro-law}
  If $x$ is not free in $A$ and $invardecls$ represents declarations
  of undashed variables then,
  \begin{circus}
    A = (\circvar x : T \circspot \lschexpract [invardecls; x! : T | pred] \rschexpract \circseq A)
  \end{circus}
\end{law}

% TODO: needs proof (similar to Law[alt-var-dist-both] in Alvaro's
% thesis)
\begin{law}[alt-var-dist]
  \label{alt-var-dist-law}
  If, for all $i$, $x$ is free in $g_i$ then, 
  \begin{circus}
    \circif \circelse_i g_i \circthen \circvar x : T \circspot A_i \circfi
    =
    \circvar x : T \circspot \circif \circelse_i g_i \circthen A_i \circfi 
  \end{circus}
\end{law}

% From Definition 4.3 and Theorem 4.1 in Marcel's thesis
\begin{minipage}{\textwidth}
  \begin{law}[forwards-data-refinement]
    \label{forwards-data-refinement-law}
  If, for processes $P1$ and $P2$, and action local state $L$, there
  exists a relation $CI$, between $P1.State$, $P2.State$, and $L$, such that,
  \begin{circus}
    \forall P2.State; L @ (\exists P1.State @ CI),
  \end{circus}
  and
  \begin{circus}
    \forall P1.State; P2.State; P2.State'; L @ CI \land P2.Act \implies (\exists P1.State'; L' @ P1.Act \land CI' ),
  \end{circus}
  then,
  \begin{circus}
    \begin{array}{l}
      \circprocess P1 \circdef \circbegin \\
      \t1 {} \cdots {} \\
      \t1 \circstate P1.State \\
      \t1 {} \cdots {} \\
      \t1 \circspot P1.Act \\
      \t1 \circend
    \end{array}
    \circrefines_P
    \begin{array}{l}
      \circprocess P2 \circdef \circbegin \\
      \t1 {} \cdots {} \\
      \t1 \circstate P2.State \\
      \t1 {} \cdots {} \\
      \t1 \circspot P2.Act \\
      \t1 \circend
    \end{array}
  \end{circus}
\end{law}
\end{minipage}

% From Definition B.51 in Marcel's Thesis
\begin{law}[process-param-elim]
  \label{process-param-elim-law}
  If $x$ is not referenced in the body of $P$, then
  \begin{circus}
    \begin{array}{l}
      \circprocess P \circdef x : T \circspot \circbegin \\
      \t1 {} \cdots {} \\
      \t1 \circstate S \\
      \t1 {} \cdots {} \\
      \t1 \circspot A \\
      \t1 \circend
    \end{array}
    =
    \begin{array}{l}
      \circprocess P \circdef \circbegin \\
      \t1 {} \cdots {} \\
      \t1 \circstate S \\
      \t1 {} \cdots {} \\
      \t1 \circspot A \\
      \t1 \circend
    \end{array}
  \end{circus}
\end{law}