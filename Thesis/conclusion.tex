\chapter{Conclusions}
\label{conclusions-chapter}
% three sections: summary of contributions, related work, and future
% work
In this chapter we conclude by summarising the contributions of this
dissertation in Section~\ref{summary-section}.
We then discuss directions of future work in
Section~\ref{future-work-section}.

\section{Summary of Contributions}
\label{summary-section}

\addchange{Added references to previous parts of the thesis in
  Section\protect~\protect\ref{summary-section}} We have considered
the safety-critical variant of Java\added{, in
  Section~\ref{scj-section},} and the virtual machines designed to run
programs written in it\added{, in
  Section~\ref{virtual-machines-section}}.
None of the virtual machines is formally verified and\deleted{ that}
many of them precompile programs to native code.
Given the need for a formally verified virtual machine, we have
developed a framework within which an SCJVM can be verified.

Having noted that SCJ virtual machines employ compilation, we have
surveyed some of the work on compiler correctness, particularly those
related to Java compilation\added{, in
  Section~\ref{compiler-correctness-section}}.
We have established that two approaches to compiler correctness have
been used:~the commuting-diagram approach and the algebraic approach.
We have adopted the algebraic approach and chosen \Circus{}\added{, described
in Section~\ref{circus-section},} as a specification language.

To specify an SCJVM we have identified the requirements of the virtual
machine services to support SCJ programs.
We have also constructed a formal model of those requirements in the
\Circus{} specification language.
\added{These virtual machine services requirements and their formal
  model are discussed in Chapter~\ref{scjvm-services-chapter}.}

Contact with one of the authors of the SCJ specification has allowed
us to obtain clarifications where the specification was unclear.
The development of the formal model has helped in the identification
of the areas that require clarification.
It may be noted that the interface we have defined is not the only one
that can support SCJ, but its overall functionality must be present in
all SCJ virtual machines in some way.
The SCJ specification has been changed to reflect\deleted{ of} many of
these clarifications.
In particular, the current thread during an interrupt, the backing
store space required during mission setup, and the initialisation of
the \texttt{Safelet} have all been clarified as a result of our
contact with the authors of the SCJ specification.

We have also created a formal model of the core execution environment
that executes SCJ programs in an SCJVM.
This model has been created by identifying a minimal subset of Java
bytecode and defining its semantics, and then constructing a \Circus{}
model of an interpreter for that subset with the necessary
infrastructure around it.
\added{We have discussed the core execution environment model in
  Chapter~\ref{cee-chapter}.}

Finally, we have specified a strategy for compiling SCJ bytecode to
C\added{, presented in Chapter~\ref{strategy-chapter}}.
This is a strategy to apply individual compilation rules, which are
stated as algebraic laws, to transform our model of an SCJVM
interpreter, loaded with an SCJ bytecode program, to a \Circus{}
respresentation of the equivalent C code.
Since the compilation rules are stated formally as \Circus{}
refinement laws, we can, and have, written proofs of them.
\added{We have discussed these proofs in
  Section~\ref{proofs-of-laws-section}.}
This allows us to be sure of their correctness, so that our
compilation strategy preserves the semantics of the original SCJ
bytecode.
In this way, we have created a strategy for correct compilation from
SCJ bytecode to C.

Our work is done in the context of a wider effort to facilitate fully
verified SCJ programs.
There has already been work on generating correct SCJ programs from
\Circus{} specifications~\cite{cavalcanti2011, cavalcanti2013} and
formalisation of the SCJ memory model~\cite{cavalcanti2011a}.
These works allow for verification of SCJ programs, with our work
covering the next stage in ensuring those programs can be run
correctly.

Since our work addresses the execution of Java bytecode, it must still
be ensured that SCJ programs can be compiled to bytecode correctly.
Since SCJ does not make any syntactic changes to Java and the
semantic changes can be dealt with at the level of Java bytecode, a
standard Java compiler suffices for SCJ.
As discussed earlier, there has been plenty of work on correct
compilation of Java programs~\cite{klein2006, strecker2002,
  lochbihler2010, duran2005, stark2001} so it can be seen that there
is already sufficient work to permit correct compilation to Java
bytecode.
This then leaves us with correct SCJ programs in Java bytecode and the
focus of our work is on the next stage of running those programs.

Finally, as we are adopting the approach of compilation to C, it must
also be ensured that the C code can be compiled correctly.
We note that there has been much work on verified C
compilation~\cite{leroy2009a, leroy2009b, leroy2012, leinenbach2005,
  blazy2006} and, in particular, that the CompCert project provides a
functioning formally verified C compiler that can be used.

\addchange{Changed concluding sentence of
  Section\protect~\protect\ref{summary-section} to be more balanced}
\added{
So, our work provides the basis for the implementation of a verified
compiler, the development of which would provide the final piece
required for complete verification of SCJ programs down to executable
code.
}
\deleted{
So our proposed work is the final piece needed for complete
verification of SCJ programs down to executable code.
}

\section{Future Work}
\label{future-work-section}

There are various possibilities for future work arising from our work.
Firstly, our work may be further validated by consideration of a wider
range of examples.
This may involve further extension of the model and compilation
strategy to consider instructions and features not covered by our
work.
These extensions would not involve significant changes to the
strategy, since most of the instructions not included in our subset
are similar to those in our subset.

A further direction for future work to validate the strategy would be
to mechanise the compilation rules and their proofs using an automated
theorem prover, such as Isabelle/UTP.
This would confirm the correctness of the rules and allow for easier
reasoning about the strategy as whole.
Code generation from such a mechanisation could also be used to
produce an implementation of the strategy.

\addchange{Added paragraph on how our work may be applied to future
  work on the algebraic approach}
\added{
Our strategy also shows how the algebraic approach developed
in~\cite{sampaio1993} may be adapted to compile from low-level
languages to higher-level languages.
Future work could build upon this to develop compilation strategies
for other low-level languages in a similar way, contributing to wider
work on the algebraic approach to compilation.
}

Other possible directions for future work include the full
verification of an SCJ virtual machine using our framework or even the
creation of a correct-by-construction virtual machine from our
specification.
The option of deriving a correct virtual machine from our
specification may be more desirable than verifying an existing one.
This is because virtual machines can often be complex and therefore
difficult to verify in a structured way.
Moreover, while the effort of proving a virtual machine correct may
uncover bugs, it may be a challenge to fix them.
Also, the design of an existing virtual machine may not exactly meet
the structure of our specification, requiring restructuring to allow
the proof effort to begin.
The work verifying the icecap scheduler in~\cite{freitas2016} shows
this, since the tight coupling between components in icecap made
modelling and verification challenging.

On the other hand, the fact that \Circus{} allows for refinement means
that a correct virtual machine can be constructed from our model in a
stepwise and modular fashion, being shown to be correct at each stage
of the process.
Facilitating such work is the ultimate aim of our work, in order to
provide for the correct running of SCJ programs.

