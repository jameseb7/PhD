\chapter{Compilation Strategy}
\label{strategy-chapter}

Our compilation strategy refines the $CEE(bc,cs,sid)$ process defined
in Section~\ref{model-section} to obtain a process that includes a
representation of C code as described in
Section~\ref{embedding-section}. 
The overall theorem for the strategy is as follows.
\begin{thm}[Compilation Strategy]\label{main-theorem}
  Given $bc$, $cs$ and $sid$, there are processes $StructMan_{cs}$ and
  $CProg_{bc,cs}$ such that,
  \begin{circus}
    CEE(bc,cs,sid) \circrefines StructMan_{cs} \parallel
    CProg_{bc,cs} \parallel Launcher(sid).
  \end{circus}
\end{thm}
$StructMan_{cs}$ manages objects represented by C structs that
incorporate the class information from $cs$, refining the process
$ObjMan$, which handles abstract objects.
$StructMan_{cs}$ has Z schemas representing struct types for objects
of each class.
%
% \begin{zed}
%   InputHandlerObj == [classid : ClassID; input, buffer : ObjectID]
% \end{zed}
%
These schemas contain the identifier $classid$ of the object's class, so
that polymorphic method calls can be made by choice over the object's
class. 
There are also components for each of the fields of the
object.

The schema types for each type of object are combined into a single
free type $ObjectStruct$.
% \begin{zed}
%   ObjectStruct ::= \dots | InputHandlerCon \ldata InputHandlerObj \rdata \dots
% \end{zed}
% We also define functions for casting between objects of different
% classes and for obtaining the class identifier of any object. 
% This matches the casting of object pointers in the C code that the
% icecap HVM generates.
$StructMan_{cs}$ contains a map from memory addresses managed by the
SCJVM to the $ObjectStruct$ type, representing the C structs in
memory, and provides access to the individual values in that map.

$CProg_{bc,cs}$ refines the $Interpreter$, with the $Thr$ processes
refined into the $CThr_{bc,cs}$ processes described in the previous
section.
This means that the threads from SCJ are mapped onto threads in C,
since we do not dictate a particular thread switch mechanism in either
the source or target models.

The compilation strategy is split into three stages, described in the
following section.
Each stage has a theorem describing it, for which the strategy acts as
a proof.
The proof of Theorem~\ref{main-theorem} is obtained by an application
of the theorems for each stage.

\section{Overview}

Each stage of the compilation strategy handles a different part of the
$Interpreter$ state:~the $pc$, the $frameStack$, and objects.
They operate over each of the $Thr$ processes, managed by the SCJVM
services.

The first stage introduces the control constructs of the C code.
This removes the use of $pc$ to determine the control flow of the
program.
The choice over $pc$ values is then replaced with a choice over method
identifiers pointing to sequences of operations representing method
bodies.

In the second stage, the information contained on the $frameStack$,
which is the local variable array and operand stack for each method,
is introduced in the C code.
This is done by introducing variables and parameters to represent each
method's local variables and operand stack slots.
A data refinement is then used to transform each operation over the
$frameStack$ to operate on the new variables.
The $frameStack$ is then eliminated from the state.

In the final stage, the class information from $cs$ is used to create
a representation of C structs.
This means that $ObjMan$, which has a very abstract representation of
objects, is transformed into $StructMan$.
The process for each thread is then made to access the structs for the
objects in a more concrete way that represents the way struct fields
are accessed in C code.

% This yields final method actions of a form similar to that of the
% example shown below, which is taken from the \texttt{InputHandler}
% presented in Section~\ref{model-section}.
% \begin{circusaction}
%   InputHandler\_HandleAsyncEvent \circdef \\
%   \t1 \circval var0 \circspot \circvar var1, stack0, stack1 : Word \circspot \\
%   \t1 stack0 := var0 \circseq Poll \circseq getObject!stack0 \then getObjectRet?struct \\
%   \t1 {} \then stack0 := (castInputHandler~struct).input \circseq \dots
% \end{circusaction}
% The \texttt{handleAsyncEvent()} method of \texttt{InputHandler} is
% compiled to the action $InputHandler\_HandleAsyncEvent$, with the
% implicit \texttt{this} parameter represented as a value parameter
% $var0$.
% The local variable ($var1$) and stack slots ($stack0$ and $stack1$)
% are represented as \Circus{} variables.
% The operations of the C code are composed in sequence, with an action
% named $Poll$ that polls for thread switches present at the points
% where thread switches may occur. 
% Stack operations are represented as assignments. 
% For instance, $stack0 := var0$ arises from the compilation to load a
% local variable into a stack slot.
% Access to objects is performed by communicating with $StructMan_{cs}$
% to obtain the struct for the object, then casting it to the correct
% type, and accessing the required value.
% Above, we obtain the value of the $input$ field from an $InputHandler$
% object.
% The communication with $StructMan_{cs}$ is performed via the
% $getObject$ channel and the function $castInputHandler$ is used to map
% the $ObjectStruct$ returned from the communication to a type
% representing an object of \texttt{InputHandler}.

We describe each of these stages in a separate section.
The first stage, which we call \emph{Elimination of Program Counter},
is described in Section~\ref{elimination-of-program-counter-section}.
The second stage, called \emph{Elimination of Frame Stack}, is
described in Section~\ref{elimination-of-frame-stack-section}.
Finally, third of the strategy, which is called \emph{Data Refinement
  of Objects}, is described in
Section~\ref{data-refinement-of-objects-section}.

\section{Elimination of Program Counter}
\label{elimination-of-program-counter-section}

This stage eliminates $pc$ from the state of each
thread's process, $Thr(bc,cs,t)$, introducing the control flow constructs of C in the
process. 
It may be summarised by the following theorem.
%
\begin{thm}[Elimination of Program Counter]\label{thread-splitting-thm}
  \begin{circus}
    Thr(bc,cs,t) \circrefines ThrCF_{bc,cs}(cs,t)
  \end{circus}%
\end{thm}
%
In this stage we act mainly upon the $Running$ action of $Thr$; its
loop is unrolled to introduce the control flow that follows each
bytecode instruction.
The aim is to get each method's bytecode instructions into a form in
which the control flow, but not the data operations, are described
using C constructs and, moreover, each path of execution (including
every branch of the conditionals) ends in a return instruction or a
loop.
We refer to a method in this form as a \emph{complete} method.

It is important to observe that it is possible to transform the
bytecode instructions of every method so that they become complete.
If we consider the control flow of a method beginning from that
method's entry point, each bytecode instruction reached must either be
a return instruction, or followed by another bytecode.
If another bytecode follows the bytecode's execution, then it must be
either a bytecode already considered, resulting in a loop, or one not
already considered.
Since there are finitely many bytecode instructions in a method, a
loop or return must eventually be reached.
Failure to do so would lead to an instruction beyond the end of the
method, which is forbidden by the structural restrictions on Java
bytecode~\cite{lindholm2014}. 
We assume bytecode input to our strategy will have undergone bytecode
verification so this cannot happen.

When a method is complete, it can be defined by a separate \Circus{}
action.
When the code for all the methods has been split in this way, the
choice of bytecode instruction using the program counter value can be
removed and replaced with a choice over method identifiers.
Thus dependency on the program counter can be completely removed,
allowing it to be eliminated from the state of $Thr$.

The overall strategy for transforming $Thr$ in this stage and
achieving this elimination is described by
Algorithm~\ref{epc-algorithm}.
It begins at line~\ref{algorithm-expand-bytecode} by expanding the
\Circus{} definitions of the bytecode instructions from the $bc$ map
into the $Running$ action, pulling out the program counter updates so
that they can be more easily manipulated by the strategy.
In line~\ref{algorithm-introduce-forward-sequence}, instructions that
are forward \texttt{goto}s or are simply followed by execution of the
bytecode at the next $pc$ value are sequenced with the instructions
following them.
After that, for each method, its loops and conditionals are introduced
in line~\ref{algorithm-introduce-loops-and-conditionals}. 
Afterwards, any complete methods are separated out, in
line~\ref{algorithm-separate-complete-methods}, and any method calls
involving completed methods are resolved by sequencing the method call
with the \Circus{} action representing the method, in
line~\ref{algorithm-resolve-method-calls}.

This is repeated until all methods have been separated out, as
indicated by the while loop in line~\ref{algorithm-method-loop}.
The $MainThread$ and $NotStarted$ actions are then refined in
line~\ref{algorithm-refine-main-actions} to provide a choice over
method identifiers, rather than $pc$ values, thus removing all uses of
$pc$ from the interpreter.
The $pc$ component is then removed from the state in
line~\ref{algorithm-remove-pc-from-state} of the algorithm.

\begin{algorithm}[t]
  \begin{algorithmic}[1]
    \State \Call{ExpandBytecode}{} \label{algorithm-expand-bytecode}
    \State \Call{IntroduceSequentialComposition}{} \label{algorithm-introduce-forward-sequence}
    \While{$\lnot$\Call{AllMethodsSeparated}{}} \label{algorithm-method-loop}
    \State \Call{IntroduceLoopsAndConditionals}{} \label{algorithm-introduce-loops-and-conditionals}
    \State \Call{SeparateCompleteMethods}{} \label{algorithm-separate-complete-methods}
    \State \Call{ResolveMethodCalls}{} \label{algorithm-resolve-method-calls}
    \EndWhile
    \State \Call{RefineMainActions}{} \label{algorithm-refine-main-actions}
    \State \Call{RemovePCFromState}{} \label{algorithm-remove-pc-from-state}
  \end{algorithmic}
  \caption{Elimination of Program Counter}
  \label{epc-algorithm}
\end{algorithm}

Each of the procedures used in Algorithm~\ref{epc-algorithm} is
defined in a separate section in the sequel.
Beforehand, we give a more detailed overview of the strategy.

\subsection{Overview}
\label{overview-subsection}

We explain the strategy with an example, the Java code for which is
shown in Figure~\ref{example-code-figure}.
\begin{figure}[t!]
  \begin{lstlisting}[language=Java,basicstyle=\ttfamily\footnotesize]
    import java.io.InputStream;
    import java.io.OutputStream;
    import javax.realtime.AperiodicParameters;
    import javax.realtime.ConfigurationParameters;
    import javax.realtime.PriorityParameters;
    import javax.safetycritical.AperiodicEventHandler;
    import javax.safetycritical.StorageParameters;
    import javax.safetycritical.io.ConsoleConnection;
    
    public class TPK extends AperiodicEventHandler {
      
      public TPK(PriorityParameters priority,
                 AperiodicParameters release,
                 StorageParameters storage,
                 ConfigurationParameters config) {
        super(priority, release, storage, config);
      }
      
      public void handleAsyncEvent() {
        ConsoleConnection console = new ConsoleConnection(null);
        
        InputStream input = console.openInputStream();
        OutputStream output = console.openOutputStream();
        
        for(int i = 0; i <= 10; i = i + 1) {
          int y = f(input.read());
          
          if (y > 400) {
            output.write(0);
          } else {
            output.write(y);
          }
        }
      }
      
      public static int f(int x){
        return x + x + x + 5;
      }
      
    }
  \end{lstlisting}
  \caption{Our example program}
  \label{example-code-figure}
\end{figure}
\begin{figure}[p]
  \scriptsize
  \setlength{\zedindent}{0cm}
  \setlength{\zedtab}{0.3cm}
  \setlength{\zedleftsep}{0.1cm}
  \setlength{\linewidth}{10cm}
  \begin{vwcol}[widths={0.7,0.3},rule=none]
    \begin{axdef}
      TPK : Class
    \where
      TPK = \lblot \\
      \t1 constantPool == \{ \\
      \t2 1 \mapsto ClassRef~TPKClassID, \\
      \t2 3 \mapsto ClassRef~AperiodicEventHandlerClassID, \\
      \t2 8 \mapsto MethodRef~AperiodicEventHandlerClassID~APEHinit, \\
      \t2 27 \mapsto ClassRef~ConsoleConnectionClassID, \\
      \t2 29 \mapsto  MethodRef~ConsoleConnectionClassID~CCinit, \\
      \t2 32 \mapsto MethodRef~ConsoleConnectionClassID~openInputStream, \\
      \t2 36 \mapsto MethodRef~ConsoleConnectionClassID~openOutputStream, \\
      \t2 40 \mapsto MethodRef~InputStreamClassID~read, \\
      \t2 41 \mapsto ClassRef~InputStreamClassID, \\
      \t2 46 \mapsto MethodRef~TPKClassID~f, \\
      \t2 50 \mapsto MethodRef~OutputStreamClassID~write, \\
      \t2 51 \mapsto ClassRef~OutputStreamClassID \\
      \t1 \}, \\
      \t1 this == 1, \\
      \t1 super == 3, \\
      \t1 interfaces == \{\}, \\
      \t1 methodEntry == \{ \\
      \t2 f \mapsto 43, \\
      \t2 handleAsyncEvent \mapsto 7, \\
      \t2 APEHinit \mapsto 0, \\
      \t1 \}, \\
      \t1 methodEnd == \{ \\
      \t2 f \mapsto 50, \\
      \t2 handleAsyncEvent \mapsto 42, \\
      \t2 APEHinit \mapsto 6 \\
      \t1 \}, \\
      \t1 methodLocals == \{ \\
      \t2 f \mapsto 1, \\
      \t2 handleAsyncEvent \mapsto 6, \\
      \t2 APEHinit \mapsto 5, \\
      \t1 \}, \\
      \t1 methodStackSize == \{ \\
      \t2 f \mapsto 2, \\
      \t2 handleAsyncEvent \mapsto 3, \\
      \t2 APEHinit \mapsto 5, \\
      \t1 \}, \\
      \t1 fields == \{\}, \\
      \t1 staticFields == \{\} \\
      \rblot
    \end{axdef}
    \begin{axdef}
      cs : ClassID \pfun Class
      \where
      cs = \{ \\
      \t1 TPKClassID \mapsto TPK \\
      \t1 \cdots \\
      \}
    \end{axdef}
    %\columnbreak
    \begin{axdef}
      bc : ProgramAddress \pfun Bytecode
      \where
      bc = \{ \\
      	\t1 0 \mapsto aload~0, \\
        \t1 1 \mapsto aload~1, \\
        \t1 2 \mapsto aload~2, \\
        \t1 3 \mapsto aload~3, \\
        \t1 4 \mapsto aload~4, \\
        \t1 5 \mapsto invokespecial~8, \\
        \t1 6 \mapsto return, \\
        \t1 7 \mapsto new~27, \\
        \t1 8 \mapsto dup, \\
        \t1 9 \mapsto aconst\_null, \\
        \t1 10 \mapsto invokespecial~29, \\
        \t1 11 \mapsto astore~1, \\
        \t1 12 \mapsto aload~1, \\
        \t1 13 \mapsto invokevirtual~32, \\
        \t1 14 \mapsto astore~2, \\
        \t1 15 \mapsto aload~1, \\
        \t1 16 \mapsto invokevirtual~36, \\
        \t1 17 \mapsto astore~3, \\
        \t1 18 \mapsto iconst~0, \\
        \t1 19 \mapsto astore~4, \\
        \t1 20 \mapsto goto~19, \\
        \t1 21 \mapsto aload~2, \\
        \t1 22 \mapsto invokevirtual~40, \\
        \t1 23 \mapsto invokestatic~46, \\
        \t1 24 \mapsto astore~5, \\
        \t1 25 \mapsto aload~5, \\
        \t1 26 \mapsto iconst~400, \\
        \t1 27 \mapsto if\_icmple~5, \\
        \t1 28 \mapsto aload~3, \\
        \t1 29 \mapsto iconst~0, \\
        \t1 30 \mapsto invokevirtual~50, \\
        \t1 31 \mapsto goto~4, \\
        \t1 32 \mapsto aload~3, \\
        \t1 33 \mapsto aload~5, \\
        \t1 34 \mapsto invokevirtual~50, \\
        \t1 35 \mapsto aload~4, \\
        \t1 36 \mapsto iconst~1, \\
        \t1 37 \mapsto iadd, \\
        \t1 38 \mapsto astore~4, \\
        \t1 39 \mapsto aload~4, \\
        \t1 40 \mapsto iconst~10, \\
        \t1 41 \mapsto if\_icmple~(\negate 20), \\
        \t1 42 \mapsto return, \\
        \t1 43 \mapsto aload~0, \\
        \t1 44 \mapsto aload~0, \\
        \t1 45 \mapsto iadd, \\
        \t1 46 \mapsto aload~0, \\
        \t1 47 \mapsto iadd, \\
        \t1 48 \mapsto iconst~5, \\
        \t1 49 \mapsto iadd, \\
        \t1 50 \mapsto areturn, \\
        \t1 {} \cdots {} \\
        \}
      \end{axdef}
      \vspace{0.1cm}
  \end{vwcol}
  \caption{The \Circus{} code corresponding to our example program}
  \label{example-model-figure}
\end{figure}%
Our example is based on the Trabb Pardo-Knuth
algorithm~\cite{knuth1980}, used for comparison of programming
languages, since it includes a variety of programming language
constructs that provide a good test of the strategy.
We have simplified the algorithm by removing the reading into an
array, since our bytecode subset does not include array operations.
Attempting to add arrays makes the example much longer, while not
giving any interesting insight into our compilation strategy.

We have also written the example as an SCJ program, with the algorithm
as the body of an aperiodic event handler, \texttt{TPK}, one or more
instances of which can be registered as part of a mission and released
during mission execution.
As already mentioned, each release of the handler causes its
\texttt{handleAsyncEvent()} method to be executed.
This method creates an instance of a \texttt{ConsoleConnection}, which
is the only standard input/output connection required by SCJ.
Instances of \texttt{InputStream} and \texttt{OutputStream} are then
obtained from the \texttt{ConsoleConnection}.

After the input and output streams have been obtained, we enter a for
loop in which an integer is read from the \texttt{InputStream}, a
static method \texttt{f()} is applied to it, and the result is output
if it is less than 400, otherwise 0 is output.
The method \texttt{f()} takes an integer as input, multiplies it by 3
and adds 5 to it.

The \texttt{TPK} class is part of a larger program that includes many
other classes, including a \texttt{Safelet}, a
\texttt{MissionSequencer}, a \texttt{Mission}, and the classes that
make up the SCJ API.
Considering these classes in our example would make the example much
larger and more complex, while not introducing any more interesting
aspects for the strategy to consider.
We, therefore, omit a presentation of these classes, though it should be
noted that they are part of the complete example.

Throughout the strategy we assume the extra classes have gone through
similar processing to that we illustrate for the $TPK$ class.
This adds little complexity to the strategy since the bytecode
instructions the strategy acts upon are placed in a contiguous array
that is acted upon consistently for all classes, and the current class
of a given bytecode instruction can always be determined from its
address in the array.

The Java code must be run through a Java compiler to generate the
corresponding bytecode, which then defines the $bc$ and $cs$ constants
of our model.
The $bc$ and $cs$ values for our example are shown in
Figure~\ref{example-model-figure}.


Applying the bytecode expansion on
line~\ref{algorithm-expand-bytecode} of Algorithm~\ref{epc-algorithm}
yields the $Running$ action shown in
Figure~\ref{bytecode-expansion-example-figure}.
\begin{figure}[t]
  \setlength{\zedindent}{0cm}
  \setlength{\zedtab}{0.3cm}
  \setlength{\zedleftsep}{0.1cm}
  \begin{circus}
    Running \circdef \\
    \t1 \circif frameStack = \emptyset \circthen \Skip \\
    \t1 {} \circelse frameStack \neq \emptyset \circthen {} \\
    \t2 \circif pc = 0 \circthen HandleAloadEPC(0) \circseq pc := 1 \\
    \t2 {} \circelse pc = 1 \circthen HandleAloadEPC(1) \circseq pc := 2 \\
    \t2 {} \circelse pc = 2 \circthen HandleAloadEPC(2) \circseq pc := 3 \\
    \t2 {} \circelse pc = 3 \circthen HandleAloadEPC(3) \circseq pc := 4 \\
    \t2 {} \circelse pc = 4 \circthen HandleAloadEPC(4) \circseq pc := 5 \\
    \t2 {} \circelse pc = 5 \circthen HandleInvokespecialEPC(8) \\
    \t2 {} \circelse pc = 6 \circthen HandleReturnEPC \\
    \t2 {} \circelse pc = 7 \circthen HandleNewEPC(27) \circseq pc := 8 \\
    \t2 {} \circelse pc = 8 \circthen HandleDupEPC \circseq pc := 9 \\
    \t2 {} \circelse pc = 9 \circthen HandleAconst\_nullEPC \circseq pc := 10 \\
    \t2 {} \circelse pc = 10 \circthen HandleInvokespecialEPC(29) \\
    \t2 {} \circelse pc = 11 \circthen HandleAstoreEPC(1) \circseq pc := 12 \\
    % \t2 {} \circelse pc = 12 \circthen HandleAloadEPC(1) \circseq pc := 13 \\
    % \t2 {} \circelse pc = 13 \circthen HandleInvokevirtualEPC(32) \\
    % \t2 {} \circelse pc = 14 \circthen HandleAstoreEPC(2) \circseq pc := 15 \\
    % \t2 {} \circelse pc = 15 \circthen HandleAloadEPC(1) \circseq pc := 16 \\
    % \t2 {} \circelse pc = 16 \circthen HandleInvokevirtualEPC(36) \\
    % \t2 {} \circelse pc = 17 \circthen HandleAstoreEPC(3) \circseq pc := 18 \\
    % \t2 {} \circelse pc = 18 \circthen HandleIconstEPC(0) \circseq pc := 19 \\
    % \t2 {} \circelse pc = 19 \circthen HandleAstoreEPC(4) \circseq pc := 20 \\
    % \t2 {} \circelse pc = 20 \circthen pc := 39 \\
    \t2 {} \cdots {} \\
    % \t2 {} \circelse pc = 21 \circthen HandleAloadEPC(2) \circseq pc := 22 \\
    % \t2 {} \circelse pc = 22 \circthen HandleInvokevirtualEPC(40) \\
    % \t2 {} \circelse pc = 23 \circthen HandleInvokestaticEPC(46) \\
    % \t2 {} \circelse pc = 24 \circthen HandleAstoreEPC(5) \circseq pc := 25 \\
    % \t2 {} \circelse pc = 25 \circthen HandleAloadEPC(5) \circseq pc := 26 \\
    % \t2 {} \circelse pc = 26 \circthen HandleIconstEPC(400) \circseq pc := 27 \\
    % \t2 {} \circelse pc = 27 \circthen \circvar value1, value2 : Word \circspot InterpreterPop2 \circseq \\
    % \t3 pc := \IF value1 \leq value2 \THEN 32 \ELSE 28 \\
    % \t2 {} \circelse pc = 28 \circthen HandleAloadEPC(3) \circseq pc := 29 \\
    % \t2 {} \circelse pc = 29 \circthen HandleIconstEPC(0) \circseq pc := 30 \\
    % \t2 {} \circelse pc = 30 \circthen HandleInvokevirtualEPC(50) \\
    % \t2 {} \circelse pc = 31 \circthen pc := 35 \\
    % \t2 {} \circelse pc = 32 \circthen HandleAloadEPC(3) \circseq pc := 33 \\
    % \t2 {} \circelse pc = 33 \circthen HandleAloadEPC(5) \circseq pc := 34 \\
    % \t2 {} \circelse pc = 34 \circthen HandleInvokevirtualEPC(50) \\
    % \t2 {} \circelse pc = 35 \circthen HandleAloadEPC(4) \circseq pc := 36 \\
    % \t2 {} \circelse pc = 36 \circthen HandleIconstEPC(1) \circseq pc := 37 \\
    % \t2 {} \circelse pc = 37 \circthen HandleIaddEPC \circseq pc := 38 \\
    % \t2 {} \circelse pc = 38 \circthen HandleAstoreEPC(4) \circseq pc := 39 \\
    % \t2 {} \circelse pc = 39 \circthen HandleAloadEPC(4) \circseq pc := 40 \\
    % \t2 {} \circelse pc = 40 \circthen HandleIconstEPC(10) \circseq pc := 41 \\
    % \t2 {} \circelse pc = 41 \circthen \circvar value1, value2 : Word \circspot InterpreterPop2 \circseq \\
    % \t3 pc := \IF value1 \leq value2 \THEN 21 \ELSE 42 \\
    % \t2 {} \circelse pc = 42 \circthen HandleReturnEPC \\
    % \t2 {} \circelse pc = 43 \circthen HandleAloadEPC(0) \circseq pc := 44 \\
    % \t2 {} \circelse pc = 44 \circthen HandleAloadEPC(0) \circseq pc := 45 \\
    % \t2 {} \circelse pc = 45 \circthen HandleIaddEPC \circseq pc := 46 \\
    % \t2 {} \circelse pc = 46 \circthen HandleAloadEPC(0) \circseq pc := 47 \\
    % \t2 {} \circelse pc = 47 \circthen HandleIaddEPC \circseq pc := 48 \\
    % \t2 {} \circelse pc = 48 \circthen HandleIconstEPC(5) \circseq pc := 49 \\
    % \t2 {} \circelse pc = 49 \circthen HandleIaddEPC \circseq pc := 50 \\
    % \t2 {} \circelse pc = 50 \circthen HandleAreturnEPC \\
    \t2 \circfi \circseq Poll \circseq Running \\
    \t1 \circfi
  \end{circus}
  \caption{The $Running$ action after bytecode expansion}
  \label{bytecode-expansion-example-figure}
\end{figure}
This step expands the bytecode instruction definitions, by copying
$HandleInstruction$ into $Running$, and converting it to a choice of
actions based on the value of the program counter, mirroring the
contents of the $bc$ map for each value.

The actions that make up $HandleInstruction$ are also replaced with
actions that incorporate instruction parameters from the $bc$ map and
have $pc$ updates separated from stack updates so they can be more
easily operated on in this stage of the strategy.
This can be seen in Figure~\ref{bytecode-expansion-example-figure},
where, in the $pc = 0$ case, $aload~0$ has been converted to
$HandleAloadEPC(0) \circseq pc := 1$, with the parameter, $0$, to the
bytecode instruction becoming a parameter of the new instruction
handling action $HandleAloadEPC$, and the update to $pc$ placed after
the data operation.

The reason for making parameters of the bytecode instructions into
parameters of the handling actions is to remove the need to reference
the bytecode instructions in the $bc$ map, as that involves use of the
$pc$ value, which we seek to remove in this stage.
This also has the benefit of fully incorporating $bc$ into the $Thr$
process, ensuring all the information required to introduce C code
constructs is available directly in \Circus{}, which makes stating
compilation laws simpler.
This is described in more detail in
Section~\ref{expand-bytecode-subsection}, where we define the
\Call{ExpandBytecode}{} procedure.

\begin{figure}[t!]
  \setlength{\zedindent}{0cm}
  \setlength{\zedtab}{0.3cm}
  \setlength{\zedleftsep}{0.1cm}
  \begin{circus}
    Running \circdef \\
    \t1 \circif frameStack = \emptyset \circthen \Skip \\
    \t1 {} \circelse frameStack \neq \emptyset \circthen {} \\
    \t2 \circif pc = 0 \circthen HandleAloadEPC(0) \circseq pc := 1 \circseq Poll \circseq HandleAloadEPC(1) \circseq \\
    \t3 pc := 2 \circseq Poll \circseq HandleAloadEPC(2) \circseq pc := 3 \circseq Poll \circseq HandleAloadEPC(4) \circseq \\
    \t3 pc := 5 \circseq Poll \circseq HandleInvokespecialEPC(8) \\
    \t2 {} \cdots {} \\
    % \t2 {} \circelse pc = 1 \circthen HandleAloadEPC(1) \circseq pc := 2 \\
    % \t2 {} \circelse pc = 2 \circthen HandleAloadEPC(2) \circseq pc := 3 \\
    % \t2 {} \circelse pc = 3 \circthen HandleAloadEPC(3) \circseq pc := 4 \\
    % \t2 {} \circelse pc = 4 \circthen HandleAloadEPC(4) \circseq pc := 5 \\
    % \t2 {} \circelse pc = 5 \circthen HandleInvokespecialEPC(8) \\
    \t2 {} \circelse pc = 6 \circthen HandleReturnEPC \\
    \t2 {} \circelse pc = 7 \circthen HandleNewEPC(27) \circseq pc := 8 \circseq Poll \circseq HandleDupEPC \circseq pc := 9 \circseq \\
    \t3 Poll \circseq HandleAconst\_nullEPC \circseq pc := 10 \circseq Poll \circseq HandleInvokespecialEPC(29) \\
    \t2 {} \cdots {} \\
    % \t2 {} \circelse pc = 8 \circthen HandleDupEPC \circseq pc := 9 \\
    % \t2 {} \circelse pc = 9 \circthen HandleAconst\_nullEPC \circseq pc := 10 \\
    % \t2 {} \circelse pc = 10 \circthen HandleInvokespecialEPC(29) \\
    \t2 {} \circelse pc = 11 \circthen HandleAstoreEPC(1) \circseq pc := 12 \circseq Poll \circseq HandleAloadEPC(1) \circseq \\
    \t3 pc := 13 \circseq Poll \circseq HandleInvokevirtualEPC(32) \\
    \t2 {} \cdots {} \\
    % \t2 {} \circelse pc = 12 \circthen HandleAloadEPC(1) \circseq pc := 13 \\
    % \t2 {} \circelse pc = 13 \circthen HandleInvokevirtualEPC(32) \\
    \t2 {} \circelse pc = 14 \circthen HandleAstoreEPC(2) \circseq pc := 15 \circseq Poll \circseq HandleAloadEPC(1) \circseq \\
    \t3 pc := 16 \circseq Poll \circseq HandleInvokevirtualEPC(36) \\
    \t2 {} \cdots {} \\
    % \t2 {} \circelse pc = 15 \circthen HandleAloadEPC(1) \circseq pc := 16 \\
    % \t2 {} \circelse pc = 16 \circthen HandleInvokevirtualEPC(36) \\
    \t2 {} \circelse pc = 17 \circthen HandleAstoreEPC(3) \circseq pc := 18 \circseq Poll \circseq HandleIconstEPC(0) \circseq \\
    \t3 pc := 19 \circseq Poll \circseq HandleAstoreEPC(4) \circseq pc := 20 \circseq Poll \circseq pc := 39 \\
    \t2 {} \cdots {} \\
    % \t2 {} \circelse pc = 18 \circthen HandleIconstEPC(0) \circseq pc := 19 \\
    % \t2 {} \circelse pc = 19 \circthen HandleAstoreEPC(4) \circseq pc := 20 \\
    % \t2 {} \circelse pc = 20 \circthen pc := 39 \\
    % \t2 {} \circelse pc = 21 \circthen HandleAloadEPC(2) \circseq pc := 22 \circseq Poll \circseq HandleInvokevirtualEPC(40) \\
    % \t2 {} \circelse pc = 22 \circthen HandleInvokevirtualEPC(40) \\
    % \t2 {} \circelse pc = 23 \circthen HandleInvokestaticEPC(46) \\
    % \t2 {} \circelse pc = 24 \circthen HandleAstoreEPC(5) \circseq pc := 25 \circseq Poll \circseq HandleAloadEPC(5) \circseq \\
    % \t3 pc := 26 \circseq Poll \circseq HandleIconstEPC(400) \circseq pc := 27 \circseq Poll \circseq \\
    % \t3 \circvar value1, value2 : Word \circspot InterpreterPop2 \circseq \\
    % \t3 pc := \IF value1 \leq value2 \THEN 32 \ELSE 28 \\
    % \t2 {} \circelse pc = 25 \circthen HandleAloadEPC(5) \circseq pc := 26 \\
    % \t2 {} \circelse pc = 26 \circthen HandleIconstEPC(400) \circseq pc := 27 \\
    % \t2 {} \circelse pc = 27 \circthen \circvar value1, value2 : Word \circspot InterpreterPop2 \circseq \\
    % \t3 pc := \IF value1 \leq value2 \THEN 32 \ELSE 28 \\
    % \t2 {} \circelse pc = 28 \circthen HandleAloadEPC(3) \circseq pc := 29 \circseq Poll \circseq HandleIconstEPC(0) \circseq \\
    % \t3 pc := 30 \circseq Poll \circseq HandleInvokevirtualEPC(50) \\
    % \t2 {} \circelse pc = 29 \circthen HandleIconstEPC(0) \circseq pc := 30 \\
    % \t2 {} \circelse pc = 30 \circthen HandleInvokevirtualEPC(50) \\
    % \t2 {} \circelse pc = 31 \circthen pc := 35 \\
    % \t2 {} \circelse pc = 32 \circthen HandleAloadEPC(3) \circseq pc := 33 \circseq Poll \circseq HandleAloadEPC(5) \circseq \\
    % \t3 pc := 34 \circseq Poll \circseq HandleInvokevirtualEPC(50) \\
    % \t2 {} \circelse pc = 33 \circthen HandleAloadEPC(5) \circseq pc := 34 \\
    % \t2 {} \circelse pc = 34 \circthen HandleInvokevirtualEPC(50) \\
    % \t2 {} \circelse pc = 35 \circthen HandleAloadEPC(4) \circseq pc := 36 \circseq Poll \circseq HandleIconstEPC(1) \circseq \\
    % \t3 pc := 37 \circseq Poll \circseq HandleIaddEPC \circseq pc := 38 \circseq Poll \circseq HandleAstoreEPC(4) \circseq \\
    % \t3 pc := 39 \\
    % \t2 {} \circelse pc = 36 \circthen HandleIconstEPC(1) \circseq pc := 37 \\
    % \t2 {} \circelse pc = 37 \circthen HandleIaddEPC \circseq pc := 38 \\
    % \t2 {} \circelse pc = 38 \circthen HandleAstoreEPC(4) \circseq pc := 39 \\
    \t2 {} \circelse pc = 39 \circthen HandleAloadEPC(4) \circseq pc := 40 \circseq Poll \circseq HandleIconstEPC(10) \circseq \\
    \t3 pc := 41 \circseq Poll \circseq \circvar value1, value2 : Word \circspot InterpreterPop2 \circseq \\
    \t3 pc := \IF value1 \leq value2 \THEN 21 \ELSE 42 \\
    \t2 {} \cdots {} \\
    % \t2 {} \circelse pc = 40 \circthen HandleIconstEPC(10) \circseq pc := 41 \\
    % \t2 {} \circelse pc = 41 \circthen \circvar value1, value2 : Word \circspot InterpreterPop2 \circseq \\
    % \t3 pc := \IF value1 \leq value2 \THEN 21 \ELSE 42 \\
    \t2 {} \circelse pc = 42 \circthen HandleReturnEPC \\
    \t2 {} \circelse pc = 43 \circthen HandleAloadEPC(0) \circseq pc := 44 \circseq Poll \circseq HandleAloadEPC(0) \circseq \\
    \t3 pc := 45 \circseq Poll \circseq HandleIaddEPC \circseq pc := 46 \circseq Poll \circseq HandleAloadEPC(0) \circseq \\
    \t3 pc := 47 \circseq Poll \circseq HandleIaddEPC \circseq pc := 48 \circseq Poll \circseq HandleIaddEPC \circseq \\
    \t3 pc := 50 \circseq Poll \circseq HandleAreturnEPC \\
    \t2 {} \cdots {} \\
    % \t2 {} \circelse pc = 44 \circthen HandleAloadEPC(0) \circseq pc := 45 \\
    % \t2 {} \circelse pc = 45 \circthen HandleIaddEPC \circseq pc := 46 \\
    % \t2 {} \circelse pc = 46 \circthen HandleAloadEPC(0) \circseq pc := 47 \\
    % \t2 {} \circelse pc = 47 \circthen HandleIaddEPC \circseq pc := 48 \\
    % \t2 {} \circelse pc = 48 \circthen HandleIconstEPC(5) \circseq pc := 49 \\
    % \t2 {} \circelse pc = 49 \circthen HandleIaddEPC \circseq pc := 50 \\
    % \t2 {} \circelse pc = 50 \circthen HandleAreturnEPC \\
    \t2 \circfi \circseq Poll \circseq Running \\
    \t1 \circfi
  \end{circus}
  \caption{The $Running$ action after forward sequence introduction}
  \label{forward-sequence-introduction-example-figure}
\end{figure}

On line~\ref{algorithm-introduce-forward-sequence} of the algorithm,
sequential composition is introduced for instructions that do not
affect the sequential flow of the program.
Such instructions are identified by considering the control flow graph
of the program and locating nodes with a single outgoing edge going to
target node exactly one incoming edge.
The introduction of sequential composition is performed by unrolling
the loop in $Running$ to introduce the control flow following each of
these instructions.
This causes the instruction to be sequentially composed with the next
instruction, with $Poll$ in between to allow for thread switches
between instructions.
This is performed exhaustively to get the code in the form shown in
Figure~\ref{forward-sequence-introduction-example-figure}, where the
choice over $pc$ has sequences of instructions collected together at
the point where they start, up to the point at which a more complex
control flow (such as a method call, conditional or a loop) occurs.
The introduction of sequential composition is described in more detail in
Section~\ref{introduce-forward-sequence-subsection}, where we define
the \Call{IntroduceSequentialComposition}{} procedure.

Handling the remaining constructs requires consideration of dependency
between methods to ensure method calls can be resolved correctly.
We say a method call is \emph{resolved} when the method invocation
bytecode has been placed in sequential composition with a call to a
\Circus{} action containing the body of the method being invoked,
which is then followed by the sequence of instructions that occur
after the invocation bytecode instruction in the calling method.
After a method call has been resolved, it no longer breaks up the
sequence of instructions it occurs in.

Since we have the bytecode instructions of all the methods needed, we
can always resolve the call of a complete method, provided that method
has already been split into its own \Circus{} action.
To ensure the method that a method call depends on is complete, we
first perform loop and conditional introduction upon it.
Since introducing loops and conditionals requires unbroken sequences
of instructions that form the bodies of loops and branches of
conditionals, introduction of loops and conditionals can only be
performed on methods that have no unresolved method calls.
For this reason, we perform method call resolution and loop and
conditional introduction repeatedly until all method calls are
resolved and the resulting complete methods have all been separated
out.
This is expressed in Algorithm~\ref{epc-algorithm} by the while loop
on line~\ref{algorithm-method-loop}.

Introduction of loops and conditionals to the body of a method with no
unresolved method calls occurs on
line~\ref{algorithm-introduce-loops-and-conditionals} of the
algorithm.
To introduce loops and conditionals we consider the control flow graph
of the method again, though it is now much simpler than the control
flow graph used for sequence introduction since straight sequences of
instructions have already been combined together.
Patterns representing conditionals and loops are then identified using
the control flow graph and the corresponding constructs are
introduced.
As loops and conditionals are introduced, nodes in the control flow
graph are merged until the graph consists of a single node, which is
the starting point of the method, containing the complete method body.

In our example, \texttt{handleAsyncEvent()} is the only method that
needs loops and conditionals introducing but, since it also contains
method calls that break up the body of a loop, we must wait until its
method calls have been resolved before introducing loops and
conditionals.
The result of introducing loops and conditionals after method calls
have been resolved in \texttt{handleAsyncEvent()} is shown in
Figure~\ref{loop-and-conditional-introduction-example-figure}.
The process of introducing loops and conditionals is described in more
detail in Section~\ref{introduce-loops-and-conditionals-subsection},
where we define the \Call{IntroduceLoopsAndConditionals}{} procedure.
\begin{figure}[t!]
  \setlength{\zedindent}{0cm}
  \setlength{\zedtab}{0.3cm}
  \setlength{\zedleftsep}{0.1cm}
  \begin{circus}
    Running \circdef \\
    \t1 \circif frameStack = \emptyset \circthen \Skip \\
    \t1 {} \circelse frameStack \neq \emptyset \circthen {} \\
    \t2 {} \circif pc = 0 \circthen HandleAloadEPC(0) \circseq pc := 1 \circseq Poll \circseq HandleAloadEPC(1) \circseq \\
    \t3 pc := 2 \circseq Poll \circseq HandleAloadEPC(2) \circseq pc := 3 \circseq Poll \circseq HandleAloadEPC(3) \circseq \\
    \t3 pc := 4 \circseq Poll \circseq HandleAloadEPC(4) \circseq pc := 5 \circseq Poll \circseq \\
    \t3 HandleInvokespecialEPC(8) \circseq Poll \circseq AperiodicEventHandler\_APEHInit \circseq \\
    \t3 Poll \circseq HandleReturnEPC \\
    \t2 {} \cdots {} \\
    \t2 {} \circelse pc = 7 \circthen HandleNewEPC(27) \circseq pc := 8 \circseq Poll \circseq HandleDupEPC \circseq pc := 9 \circseq \\
    \t3 {} \cdots {} \\
    % \t3 Poll \circseq HandleAconst\_nullEPC \circseq pc := 10 \circseq Poll \circseq HandleInvokespecialEPC(29) \circseq \\
    % \t3 Poll \circseq ConsoleConnection\_CCInit \circseq Poll \circseq HandleAstoreEPC(1) \circseq pc := 12 \circseq Poll \circseq \\
    % \t3 HandleAloadEPC(1) \circseq pc := 13 \circseq Poll \circseq HandleInvokevirtualEPC(32) \circseq Poll \circseq \\
    % \t3 OpenInputStream \circseq Poll \circseq HandleAstoreEPC(2) \circseq pc := 15 \circseq Poll \circseq \\
    % \t3 HandleAloadEPC(1) \circseq pc := 16 \circseq Poll \circseq HandleInvokevirtualEPC(36) \circseq Poll \circseq \\
    % \t3 OpenOutputStream \circseq Poll \circseq HandleAstoreEPC(3) \circseq pc := 18 \circseq Poll \circseq \\
    % \t3 HandleIconstEPC(0) \circseq pc := 19 \circseq Poll \circseq HandleAstoreEPC(4) \circseq pc := 20 \circseq \\
    \t3 Poll \circseq pc := 39 \circseq Poll \circseq \circmu Y \circspot \\
    \t4 HandleAloadEPC(4) \circseq pc := 40 \circseq Poll \circseq HandleIconstEPC(10) \circseq pc := 41 \circseq \\
    \t4 Poll \circseq \circvar value1, value2 : Word \circspot InterpreterPop2 \circseq \\
    \t4 pc := \IF value1 \leq value2 \THEN 21 \ELSE 42 \circseq Poll \circseq \\
    \t4 \circif value1 \leq value2 \circthen HandleAloadEPC(2) \circseq pc := 22 \circseq Poll \circseq \\
    \t5 {} \cdots {} \\
    \t5 HandleIconstEPC(400) \circseq pc := 27 \circseq Poll \circseq \circvar value1, value2 : Word \circspot \\
    \t5 InterpreterPop2 \circseq pc := \IF value1 \leq value2 \THEN 32 \ELSE 28 \circseq Poll \circseq \\
    \t5 \circif value1 \leq value2 \circthen HandleAloadEPC(3) \circseq pc := 33 \circseq Poll \circseq \\
    \t6 HandleAloadEPC(5) \circseq pc := 34 \circseq Poll \circseq HandleInvokevirtualEPC(50) \circseq \\
    \t6 Poll \circseq ConsoleOutput\_Write \\
    \t5 {} \circelse value1 > value2 \circthen HandleAloadEPC(3) \circseq pc := 29 \circseq Poll \circseq \\
    \t6  HandleIconstEPC(0) \circseq pc := 30 \circseq  Poll \circseq HandleInvokevirtualEPC(50) \circseq \\
    \t6 Poll \circseq ConsoleOutput\_Write \\
    \t5 \circfi \circseq pc := 35 \circseq Poll \circseq  HandleAloadEPC(4) \circseq pc := 36 \circseq Poll \circseq \\
    \t5 HandleIconstEPC(1) \circseq pc := 37 \circseq Poll \circseq HandleIaddEPC \circseq pc := 38 \circseq \\
    \t5 Poll \circseq HandleAstoreEPC(4) \circseq pc := 39 \circseq Poll \circseq Y \\
    \t5 {} \circelse value1 > value2 \circthen  HandleReturnEPC \\
    \t4 \circfi \\
    % \t2 {} \circelse pc = 8 \circthen HandleDupEPC \circseq pc := 9 \circseq Poll \circseq HandleAconst\_nullEPC \circseq pc := 10 \circseq \\
    % \t3 Poll \circseq HandleInvokespecialEPC(29) \circseq Poll \circseq CCInit \circseq Poll \circseq HandleAstoreEPC(1) \circseq \\
    % \t3 pc := 12 \circseq Poll \circseq HandleAloadEPC(1) \circseq pc := 13 \circseq Poll \circseq \\
    % \t3 HandleInvokevirtualEPC(32) \circseq Poll \circseq OpenInputStream \circseq Poll \circseq HandleAstoreEPC(2) \circseq \\
    % \t3 pc := 15 \circseq Poll \circseq HandleAloadEPC(1) \circseq pc := 16 \circseq Poll \circseq HandleInvokevirtualEPC(36) \circseq Poll \circseq OpenOutputStream \circseq Poll \circseq HandleAstoreEPC(3) \circseq pc := 18 \circseq Poll \circseq HandleIconstEPC(0) \circseq \\
    % \t3 pc := 19 \circseq Poll \circseq HandleAstoreEPC(4) \circseq pc := 20 \circseq Poll \circseq pc := 39 \circseq Poll \circseq \\
    % \t3 HandleAloadEPC(4) \circseq pc := 40 \circseq Poll \circseq HandleIconstEPC(10) \circseq pc := 41 \circseq \\
    % \t3 Poll \circseq \circvar value1, value2 : Word \circspot InterpreterPop2 \circseq \\
    % \t3 pc := \IF value1 \leq value2 \THEN 21 \ELSE 42 \\
    % \t2 {} \circelse pc = 9 \circthen HandleAconst\_nullEPC \circseq pc := 10 \circseq Poll \circseq HandleInvokespecialEPC(29) \circseq Poll \circseq CCInit \circseq Poll \circseq HandleAstoreEPC(1) \circseq pc := 12 \circseq Poll \circseq HandleAloadEPC(1) \circseq pc := 13 \circseq Poll \circseq HandleInvokevirtualEPC(32) \circseq Poll \circseq OpenInputStream \circseq Poll \circseq HandleAstoreEPC(2) \circseq pc := 15 \circseq Poll \circseq HandleAloadEPC(1) \circseq pc := 16 \circseq Poll \circseq HandleInvokevirtualEPC(36) \circseq Poll \circseq OpenOutputStream \circseq Poll \circseq HandleAstoreEPC(3) \circseq pc := 18 \circseq Poll \circseq HandleIconstEPC(0) \circseq \\
    % \t3 pc := 19 \circseq Poll \circseq HandleAstoreEPC(4) \circseq pc := 20 \circseq Poll \circseq pc := 39 \circseq Poll \circseq \\
    % \t3 HandleAloadEPC(4) \circseq pc := 40 \circseq Poll \circseq HandleIconstEPC(10) \circseq pc := 41 \circseq \\
    % \t3 Poll \circseq \circvar value1, value2 : Word \circspot InterpreterPop2 \circseq \\
    % \t3 pc := \IF value1 \leq value2 \THEN 21 \ELSE 42 \\
    % \t2 {} \circelse pc = 10 \circthen HandleInvokespecialEPC(29) \circseq Poll \circseq CCInit \circseq Poll \circseq HandleAstoreEPC(1) \circseq pc := 12 \circseq Poll \circseq HandleAloadEPC(1) \circseq pc := 13 \circseq Poll \circseq HandleInvokevirtualEPC(32) \circseq Poll \circseq OpenInputStream \circseq Poll \circseq HandleAstoreEPC(2) \circseq pc := 15 \circseq Poll \circseq HandleAloadEPC(1) \circseq pc := 16 \circseq Poll \circseq HandleInvokevirtualEPC(36) \circseq Poll \circseq OpenOutputStream \circseq Poll \circseq HandleAstoreEPC(3) \circseq pc := 18 \circseq Poll \circseq HandleIconstEPC(0) \circseq \\
    % \t3 pc := 19 \circseq Poll \circseq HandleAstoreEPC(4) \circseq pc := 20 \circseq Poll \circseq pc := 39 \circseq Poll \circseq \\
    % \t3 HandleAloadEPC(4) \circseq pc := 40 \circseq Poll \circseq HandleIconstEPC(10) \circseq pc := 41 \circseq \\
    % \t3 Poll \circseq \circvar value1, value2 : Word \circspot InterpreterPop2 \circseq \\
    % \t3 pc := \IF value1 \leq value2 \THEN 21 \ELSE 42 \\
    % \t2 {} \circelse pc = 11 \circthen HandleAstoreEPC(1) \circseq pc := 12 \circseq Poll \circseq HandleAloadEPC(1) \circseq \\
    % \t3 pc := 13 \circseq Poll \circseq HandleInvokevirtualEPC(32) \circseq Poll \circseq OpenInputStream \circseq Poll \circseq HandleAstoreEPC(2) \circseq pc := 15 \circseq Poll \circseq HandleAloadEPC(1) \circseq pc := 16 \circseq Poll \circseq HandleInvokevirtualEPC(36) \circseq Poll \circseq OpenOutputStream \circseq Poll \circseq HandleAstoreEPC(3) \circseq pc := 18 \circseq Poll \circseq HandleIconstEPC(0) \circseq \\
    % \t3 pc := 19 \circseq Poll \circseq HandleAstoreEPC(4) \circseq pc := 20 \circseq Poll \circseq pc := 39 \circseq Poll \circseq \\
    % \t3 HandleAloadEPC(4) \circseq pc := 40 \circseq Poll \circseq HandleIconstEPC(10) \circseq pc := 41 \circseq \\
    % \t3 Poll \circseq \circvar value1, value2 : Word \circspot InterpreterPop2 \circseq \\
    % \t3 pc := \IF value1 \leq value2 \THEN 21 \ELSE 42 \\
    % \t2 {} \circelse pc = 12 \circthen HandleAloadEPC(1) \circseq pc := 13 \circseq Poll \circseq HandleInvokevirtualEPC(32) \circseq Poll \circseq OpenInputStream \circseq Poll \circseq HandleAstoreEPC(2) \circseq pc := 15 \circseq Poll \circseq HandleAloadEPC(1) \circseq pc := 16 \circseq Poll \circseq HandleInvokevirtualEPC(36) \circseq Poll \circseq OpenOutputStream \circseq Poll \circseq HandleAstoreEPC(3) \circseq pc := 18 \circseq Poll \circseq HandleIconstEPC(0) \circseq \\
    % \t3 pc := 19 \circseq Poll \circseq HandleAstoreEPC(4) \circseq pc := 20 \circseq Poll \circseq pc := 39 \circseq Poll \circseq \\
    % \t3 HandleAloadEPC(4) \circseq pc := 40 \circseq Poll \circseq HandleIconstEPC(10) \circseq pc := 41 \circseq \\
    % \t3 Poll \circseq \circvar value1, value2 : Word \circspot InterpreterPop2 \circseq \\
    % \t3 pc := \IF value1 \leq value2 \THEN 21 \ELSE 42 \\
    % \t2 {} \circelse pc = 13 \circthen HandleInvokevirtualEPC(32) \circseq Poll \circseq OpenInputStream \circseq Poll \circseq HandleAstoreEPC(2) \circseq pc := 15 \circseq Poll \circseq HandleAloadEPC(1) \circseq pc := 16 \circseq Poll \circseq HandleInvokevirtualEPC(36) \circseq Poll \circseq OpenOutputStream \circseq Poll \circseq HandleAstoreEPC(3) \circseq pc := 18 \circseq Poll \circseq HandleIconstEPC(0) \circseq \\
    % \t3 pc := 19 \circseq Poll \circseq HandleAstoreEPC(4) \circseq pc := 20 \circseq Poll \circseq pc := 39 \circseq Poll \circseq \\
    % \t3 HandleAloadEPC(4) \circseq pc := 40 \circseq Poll \circseq HandleIconstEPC(10) \circseq pc := 41 \circseq \\
    % \t3 Poll \circseq \circvar value1, value2 : Word \circspot InterpreterPop2 \circseq \\
    % \t3 pc := \IF value1 \leq value2 \THEN 21 \ELSE 42 \\
    % \t2 {} \circelse pc = 14 \circthen HandleAstoreEPC(2) \circseq pc := 15 \circseq Poll \circseq HandleAloadEPC(1) \circseq \\
    % \t3 pc := 16 \circseq Poll \circseq HandleInvokevirtualEPC(36) \circseq Poll \circseq OpenOutputStream \circseq Poll \circseq HandleAstoreEPC(3) \circseq pc := 18 \circseq Poll \circseq HandleIconstEPC(0) \circseq \\
    % \t3 pc := 19 \circseq Poll \circseq HandleAstoreEPC(4) \circseq pc := 20 \circseq Poll \circseq pc := 39 \circseq Poll \circseq \\
    % \t3 HandleAloadEPC(4) \circseq pc := 40 \circseq Poll \circseq HandleIconstEPC(10) \circseq pc := 41 \circseq \\
    % \t3 Poll \circseq \circvar value1, value2 : Word \circspot InterpreterPop2 \circseq \\
    % \t3 pc := \IF value1 \leq value2 \THEN 21 \ELSE 42 \\
    % \t2 {} \circelse pc = 15 \circthen HandleAloadEPC(1) \circseq pc := 16 \circseq Poll \circseq HandleInvokevirtualEPC(36) \circseq Poll \circseq OpenOutputStream \circseq Poll \circseq HandleAstoreEPC(3) \circseq pc := 18 \circseq Poll \circseq HandleIconstEPC(0) \circseq \\
    % \t3 pc := 19 \circseq Poll \circseq HandleAstoreEPC(4) \circseq pc := 20 \circseq Poll \circseq pc := 39 \circseq Poll \circseq \\
    % \t3 HandleAloadEPC(4) \circseq pc := 40 \circseq Poll \circseq HandleIconstEPC(10) \circseq pc := 41 \circseq \\
    % \t3 Poll \circseq \circvar value1, value2 : Word \circspot InterpreterPop2 \circseq \\
    % \t3 pc := \IF value1 \leq value2 \THEN 21 \ELSE 42 \\
    % \t2 {} \circelse pc = 16 \circthen HandleInvokevirtualEPC(36) \circseq Poll \circseq OpenOutputStream \circseq Poll \circseq HandleAstoreEPC(3) \circseq pc := 18 \circseq Poll \circseq HandleIconstEPC(0) \circseq \\
    % \t3 pc := 19 \circseq Poll \circseq HandleAstoreEPC(4) \circseq pc := 20 \circseq Poll \circseq pc := 39 \circseq Poll \circseq \\
    % \t3 HandleAloadEPC(4) \circseq pc := 40 \circseq Poll \circseq HandleIconstEPC(10) \circseq pc := 41 \circseq \\
    % \t3 Poll \circseq \circvar value1, value2 : Word \circspot InterpreterPop2 \circseq \\
    % \t3 pc := \IF value1 \leq value2 \THEN 21 \ELSE 42 \\
    % \t2 {} \circelse pc = 17 \circthen HandleAstoreEPC(3) \circseq pc := 18 \circseq Poll \circseq HandleIconstEPC(0) \circseq \\
    % \t3 pc := 19 \circseq Poll \circseq HandleAstoreEPC(4) \circseq pc := 20 \circseq Poll \circseq pc := 39 \circseq Poll \circseq \\
    % \t3 HandleAloadEPC(4) \circseq pc := 40 \circseq Poll \circseq HandleIconstEPC(10) \circseq pc := 41 \circseq \\
    % \t3 Poll \circseq \circvar value1, value2 : Word \circspot InterpreterPop2 \circseq \\
    % \t3 pc := \IF value1 \leq value2 \THEN 21 \ELSE 42 \\
    % \t2 {} \circelse pc = 18 \circthen HandleIconstEPC(0) \circseq pc := 19 \circseq Poll \circseq HandleAstoreEPC(4) \circseq pc := 20 \circseq Poll \circseq pc := 39 \circseq Poll \circseq HandleAloadEPC(4) \circseq pc := 40 \circseq Poll \circseq HandleIconstEPC(10) \circseq pc := 41 \circseq Poll \circseq \circvar value1, value2 : Word \circspot InterpreterPop2 \circseq pc := \IF value1 \leq value2 \THEN 21 \ELSE 42 \\
    % \t2 {} \circelse pc = 19 \circthen HandleAstoreEPC(4) \circseq pc := 20 \circseq Poll \circseq pc := 39 \circseq Poll \circseq HandleAloadEPC(4) \circseq pc := 40 \circseq Poll \circseq HandleIconstEPC(10) \circseq pc := 41 \circseq Poll \circseq \circvar value1, value2 : Word \circspot InterpreterPop2 \circseq pc := \IF value1 \leq value2 \THEN 21 \ELSE 42 \\
    % \t2 {} \circelse pc = 20 \circthen pc := 39 \circseq Poll \circseq HandleAloadEPC(4) \circseq pc := 40 \circseq Poll \circseq HandleIconstEPC(10) \circseq pc := 41 \circseq Poll \circseq \circvar value1, value2 : Word \circspot InterpreterPop2 \circseq pc := \IF value1 \leq value2 \THEN 21 \ELSE 42 \\
    % \t2 {} \circelse pc = 21 \circthen HandleAloadEPC(2) \circseq pc := 22 \circseq Poll \circseq HandleInvokevirtualEPC(40) \circseq Poll \circseq Read \circseq Poll \circseq HandleInvokestaticEPC(46) \circseq Poll \circseq F \circseq Poll \circseq HandleAstoreEPC(5) \circseq pc := 25 \circseq Poll \circseq HandleAloadEPC(5) \circseq pc := 26 \circseq HandleIconstEPC(400) \circseq pc := 27 \circseq Poll \circseq \circvar value1, value2 : Word \circspot InterpreterPop2 \circseq pc := \IF value1 \leq value2 \THEN 32 \ELSE 28 \\
    % \t2 {} \circelse pc = 22 \circthen HandleInvokevirtualEPC(40) \circseq Poll \circseq Read \circseq Poll \circseq HandleInvokestaticEPC(46) \circseq Poll \circseq F \circseq Poll \circseq HandleAstoreEPC(5) \circseq pc := 25 \circseq Poll \circseq HandleAloadEPC(5) \circseq pc := 26 \circseq HandleIconstEPC(400) \circseq pc := 27 \circseq Poll \circseq \circvar value1, value2 : Word \circspot InterpreterPop2 \circseq pc := \IF value1 \leq value2 \THEN 32 \ELSE 28 \\
    % \t2 {} \circelse pc = 23 \circthen HandleInvokestaticEPC(46) \circseq Poll \circseq F \circseq Poll \circseq HandleAstoreEPC(5) \circseq pc := 25 \circseq Poll \circseq HandleAloadEPC(5) \circseq pc := 26 \circseq HandleIconstEPC(400) \circseq pc := 27 \circseq Poll \circseq \circvar value1, value2 : Word \circspot InterpreterPop2 \circseq pc := \IF value1 \leq value2 \THEN 32 \ELSE 28 \\
    % \t2 {} \circelse pc = 24 \circthen HandleAstoreEPC(5) \circseq pc := 25 \circseq Poll \circseq HandleAloadEPC(5) \circseq pc := 26 \circseq HandleIconstEPC(400) \circseq pc := 27 \circseq Poll \circseq \circvar value1, value2 : Word \circspot InterpreterPop2 \circseq pc := \IF value1 \leq value2 \THEN 32 \ELSE 28 \\
    % \t2 {} \circelse pc = 25 \circthen HandleAloadEPC(5) \circseq pc := 26 \circseq Poll \circseq HandleIconstEPC(400) \circseq pc := 27 \circseq Poll \circseq \circvar value1, value2 : Word \circspot InterpreterPop2 \circseq pc := \IF value1 \leq value2 \THEN 32 \ELSE 28 \\
    % \t2 {} \circelse pc = 26 \circthen HandleIconstEPC(400) \circseq pc := 27 \circseq Poll \circseq \circvar value1, value2 : Word \circspot InterpreterPop2 \circseq pc := \IF value1 \leq value2 \THEN 32 \ELSE 28 \\
    % \t2 {} \circelse pc = 27 \circthen \circvar value1, value2 : Word \circspot InterpreterPop2 \circseq \\
    % \t3 pc := \IF value1 \leq value2 \THEN 32 \ELSE 28 \\
    % \t2 {} \circelse pc = 28 \circthen HandleAloadEPC(3) \circseq pc := 29 \circseq Poll \circseq HandleIconstEPC(0) \circseq pc := 30 \circseq Poll \circseq HandleInvokevirtualEPC(50) \circseq Poll \circseq Write \circseq Poll \circseq pc := 38 \circseq Poll \circseq HandleAstoreEPC(4) \circseq pc := 39 \circseq Poll \circseq HandleAloadEPC(4) \circseq pc := 40 \circseq Poll \circseq HandleIconstEPC(10) \circseq pc := 41 \circseq Poll \circseq \circvar value1, value2 : Word \circspot InterpreterPop2 \circseq pc := \IF value1 \leq value2 \THEN 21 \ELSE 42 \\
    % \t2 {} \circelse pc = 29 \circthen HandleIconstEPC(0) \circseq pc := 30 \circseq Poll \circseq HandleInvokevirtualEPC(50) \circseq Poll \circseq Write \circseq Poll \circseq pc := 38 \circseq Poll \circseq HandleAstoreEPC(4) \circseq pc := 39 \circseq Poll \circseq HandleAloadEPC(4) \circseq pc := 40 \circseq Poll \circseq HandleIconstEPC(10) \circseq pc := 41 \circseq Poll \circseq \circvar value1, value2 : Word \circspot InterpreterPop2 \circseq pc := \IF value1 \leq value2 \THEN 21 \ELSE 42 \\
    % \t2 {} \circelse pc = 30 \circthen HandleInvokevirtualEPC(50) \circseq Poll \circseq Write \circseq Poll \circseq pc := 38 \circseq Poll \circseq HandleAstoreEPC(4) \circseq pc := 39 \circseq Poll \circseq HandleAloadEPC(4) \circseq pc := 40 \circseq Poll \circseq HandleIconstEPC(10) \circseq pc := 41 \circseq Poll \circseq \circvar value1, value2 : Word \circspot InterpreterPop2 \circseq pc := \IF value1 \leq value2 \THEN 21 \ELSE 42 \\
    % \t2 {} \circelse pc = 31 \circthen pc := 38 \circseq Poll \circseq HandleAstoreEPC(4) \circseq pc := 39 \circseq Poll \circseq HandleAloadEPC(4) \circseq pc := 40 \circseq Poll \circseq HandleIconstEPC(10) \circseq pc := 41 \circseq Poll \circseq \circvar value1, value2 : Word \circspot InterpreterPop2 \circseq pc := \IF value1 \leq value2 \THEN 21 \ELSE 42 \\
    % \t2 {} \circelse pc = 32 \circthen HandleAloadEPC(3) \circseq pc := 33 \circseq Poll \circseq HandleAloadEPC(5) \circseq pc := 34 \circseq Poll \circseq HandleInvokevirtualEPC(50) \circseq Poll \circseq Write \circseq Poll \circseq HandleAloadEPC(4) \circseq pc := 36 \circseq Poll \circseq HandleIconstEPC(1) \circseq pc := 37 \circseq Poll \circseq HandleIaddEPC \circseq pc := 38 \circseq Poll \circseq HandleAstoreEPC(4) \circseq pc := 39 \circseq Poll \circseq HandleAloadEPC(4) \circseq pc := 40 \circseq Poll \circseq HandleIconstEPC(10) \circseq pc := 41 \circseq Poll \circseq \circvar value1, value2 : Word \circspot InterpreterPop2 \circseq pc := \IF value1 \leq value2 \THEN 21 \ELSE 42 \\
    % \t2 {} \circelse pc = 33 \circthen HandleAloadEPC(5) \circseq pc := 34 \circseq Poll \circseq HandleInvokevirtualEPC(50) \circseq Poll \circseq Write \circseq Poll \circseq HandleAloadEPC(4) \circseq pc := 36 \circseq Poll \circseq HandleIconstEPC(1) \circseq pc := 37 \circseq Poll \circseq HandleIaddEPC \circseq pc := 38 \circseq Poll \circseq HandleAstoreEPC(4) \circseq pc := 39 \circseq Poll \circseq HandleAloadEPC(4) \circseq pc := 40 \circseq Poll \circseq HandleIconstEPC(10) \circseq pc := 41 \circseq Poll \circseq \circvar value1, value2 : Word \circspot InterpreterPop2 \circseq pc := \IF value1 \leq value2 \THEN 21 \ELSE 42 \\
    % \t2 {} \circelse pc = 34 \circthen HandleInvokevirtualEPC(50) \circseq Poll \circseq Write \circseq Poll \circseq HandleAloadEPC(4) \circseq pc := 36 \circseq Poll \circseq HandleIconstEPC(1) \circseq pc := 37 \circseq Poll \circseq HandleIaddEPC \circseq pc := 38 \circseq Poll \circseq HandleAstoreEPC(4) \circseq pc := 39 \circseq Poll \circseq HandleAloadEPC(4) \circseq pc := 40 \circseq Poll \circseq HandleIconstEPC(10) \circseq pc := 41 \circseq Poll \circseq \circvar value1, value2 : Word \circspot InterpreterPop2 \circseq pc := \IF value1 \leq value2 \THEN 21 \ELSE 42 \\
    % \t2 {} \circelse pc = 35 \circthen HandleAloadEPC(4) \circseq pc := 36 \circseq Poll \circseq HandleIconstEPC(1) \circseq pc := 37 \circseq Poll \circseq HandleIaddEPC \circseq pc := 38 \circseq Poll \circseq HandleAstoreEPC(4) \circseq pc := 39 \circseq Poll \circseq HandleAloadEPC(4) \circseq pc := 40 \circseq Poll \circseq HandleIconstEPC(10) \circseq pc := 41 \circseq Poll \circseq \circvar value1, value2 : Word \circspot InterpreterPop2 \circseq pc := \IF value1 \leq value2 \THEN 21 \ELSE 42 \\
    % \t2 {} \circelse pc = 36 \circthen HandleIconstEPC(1) \circseq pc := 37 \circseq Poll \circseq HandleIaddEPC \circseq pc := 38 \circseq Poll \circseq HandleAstoreEPC(4) \circseq pc := 39 \circseq Poll \circseq HandleAloadEPC(4) \circseq pc := 40 \circseq Poll \circseq HandleIconstEPC(10) \circseq pc := 41 \circseq Poll \circseq \circvar value1, value2 : Word \circspot InterpreterPop2 \circseq pc := \IF value1 \leq value2 \THEN 21 \ELSE 42 \\
    % \t2 {} \circelse pc = 37 \circthen HandleIaddEPC \circseq pc := 38 \circseq Poll \circseq HandleAstoreEPC(4) \circseq pc := 39 \circseq Poll \circseq HandleAloadEPC(4) \circseq pc := 40 \circseq Poll \circseq HandleIconstEPC(10) \circseq pc := 41 \circseq Poll \circseq \circvar value1, value2 : Word \circspot InterpreterPop2 \circseq pc := \IF value1 \leq value2 \THEN 21 \ELSE 42 \\
    % \t2 {} \circelse pc = 38 \circthen HandleAstoreEPC(4) \circseq pc := 39 \circseq Poll \circseq HandleAloadEPC(4) \circseq pc := 40 \circseq Poll \circseq HandleIconstEPC(10) \circseq pc := 41 \circseq Poll \circseq \circvar value1, value2 : Word \circspot InterpreterPop2 \circseq pc := \IF value1 \leq value2 \THEN 21 \ELSE 42 \\
    % \t2 {} \circelse pc = 39 \circthen HandleAloadEPC(4) \circseq pc := 40 \circseq Poll \circseq HandleIconstEPC(10) \circseq pc := 41 \circseq Poll \circseq \circvar value1, value2 : Word \circspot InterpreterPop2 \circseq pc := \IF value1 \leq value2 \THEN 21 \ELSE 42 \\
    % \t2 {} \circelse pc = 40 \circthen HandleIconstEPC(10) \circseq pc := 41 \circseq Poll \circseq \circvar value1, value2 : Word \circspot InterpreterPop2 \circseq pc := \IF value1 \leq value2 \THEN 21 \ELSE 42 \\
    % \t2 {} \circelse pc = 41 \circthen \circvar value1, value2 : Word \circspot InterpreterPop2 \circseq \\
    % \t3 pc := \IF value1 \leq value2 \THEN 21 \ELSE 42 \\
    % \t2 {} \circelse pc = 42 \circthen HandleReturnEPC \\
    \t2 {} \cdots {} \\
    \t2 {} \circelse pc = 43 \circthen TPK\_F \\
    \t2 {} \cdots {} \\
    \t2 \circfi \circseq Poll \circseq Running \\
    \t1 \circfi
  \end{circus}
  \caption{The $Running$ action after loop and conditional introduction}
  \label{loop-and-conditional-introduction-example-figure}
\end{figure}

After loops and conditionals have been introduced, methods that are
then complete can be copied into separate actions.
This occurs in line~\ref{algorithm-separate-complete-methods} of this
algorithm.
This is done with a simple application of the copy rule, replacing the
actions at the entry points of the split methods with references to
newly created method actions.
This can be seen in
Figure~\ref{method-call-resolution-example-figure}, where the $TPK\_F$
action has been created by splitting the sequence of actions for the
\texttt{f()} method of \texttt{TPK} from the $pc = 43$ case.
As this step is relatively simple, we do not explain it in a separate
section.
\begin{figure}[t]
  \setlength{\zedindent}{0cm}
  \setlength{\zedtab}{0.3cm}
  \setlength{\zedleftsep}{0cm}
  \begin{circus}
    Running \circdef \\
    \t1 \circif frameStack = \emptyset \circthen \Skip \\
    \t1 {} \circelse frameStack \neq \emptyset \circthen {} \\
    \t2 {} \circif pc = 0 \circthen HandleAloadEPC(0) \circseq pc := 1 \circseq Poll \circseq HandleAloadEPC(1) \circseq \\
    \t3 pc := 2 \circseq Poll \circseq HandleAloadEPC(2) \circseq pc := 3 \circseq Poll \circseq HandleAloadEPC(3) \circseq \\
    \t3 pc := 4 \circseq Poll \circseq HandleAloadEPC(4) \circseq pc := 5 \circseq Poll \circseq \\
    \t3 HandleInvokespecialEPC(8) \circseq Poll \circseq AperiodicEventHandler\_APEHInit \circseq \\
    \t3 Poll \circseq HandleReturnEPC \\
    \t2 {} \cdots {} \\
    % \t2 {} \circelse pc = 1 \circthen HandleAloadEPC(1) \circseq pc := 2 \circseq Poll \circseq HandleAloadEPC(2) \circseq pc := 3 \circseq \\
    % \t3 Poll \circseq HandleAloadEPC(3) \circseq pc := 4 \circseq Poll \circseq HandleAloadEPC(4) \circseq pc := 5 \circseq \\
    % \t3 Poll \circseq HandleInvokespecialEPC(8) \circseq Poll \circseq APEHInit \circseq Poll \circseq HandleReturnEPC \\
    % \t2 {} \circelse pc = 2 \circthen HandleAloadEPC(2) \circseq pc := 3 \circseq Poll \circseq HandleAloadEPC(3) \circseq pc := 4 \circseq \\
    % \t3 Poll \circseq HandleAloadEPC(4) \circseq pc := 5 \circseq Poll \circseq HandleInvokespecialEPC(8) \circseq \\
    % \t3 Poll \circseq APEHInit \circseq Poll \circseq HandleReturnEPC \\
    % \t2 {} \circelse pc = 3 \circthen HandleAloadEPC(3) \circseq pc := 4 \circseq Poll \circseq HandleAloadEPC(4) \circseq pc := 5 \circseq \\
    % \t3 Poll \circseq HandleInvokespecialEPC(8) \circseq Poll \circseq APEHInit \circseq Poll \circseq HandleReturnEPC \\
    % \t2 {} \circelse pc = 4 \circthen HandleAloadEPC(4) \circseq pc := 5 \circseq Poll \circseq HandleInvokespecialEPC(8) \circseq \\
    % \t3 Poll \circseq APEHInit \circseq Poll \circseq HandleReturnEPC \\
    % \t2 {} \circelse pc = 5 \circthen HandleInvokespecialEPC(8) \circseq Poll \circseq APEHInit \circseq Poll \circseq HandleReturnEPC \\
    % \t2 {} \circelse pc = 6 \circthen HandleReturnEPC \\
    % \t2 {} \circelse pc = 7 \circthen HandleNewEPC(27) \circseq pc := 8 \circseq Poll \circseq HandleDupEPC \circseq pc := 9 \circseq \\
    % \t3 Poll \circseq HandleAconst\_nullEPC \circseq pc := 10 \circseq Poll \circseq HandleInvokespecialEPC(29) \circseq \\
    % \t3 Poll \circseq ConsoleConnection\_CCInit \circseq Poll \circseq HandleAstoreEPC(1) \circseq pc := 12 \circseq Poll \circseq \\
    % \t3 HandleAloadEPC(1) \circseq pc := 13 \circseq Poll \circseq HandleInvokevirtualEPC(32) \circseq Poll \circseq \\
    % \t3 OpenInputStream \circseq Poll \circseq HandleAstoreEPC(2) \circseq pc := 15 \circseq Poll \circseq \\
    % \t3 HandleAloadEPC(1) \circseq pc := 16 \circseq Poll \circseq HandleInvokevirtualEPC(36) \circseq Poll \circseq \\
    % \t3 OpenOutputStream \circseq Poll \circseq HandleAstoreEPC(3) \circseq pc := 18 \circseq Poll \circseq \\
    % \t3 HandleIconstEPC(0) \circseq pc := 19 \circseq Poll \circseq HandleAstoreEPC(4) \circseq pc := 20 \circseq \\
    % \t3 Poll \circseq pc := 39 \\
    % \t2 {} \cdots {} \\
    % \t2 {} \circelse pc = 8 \circthen HandleDupEPC \circseq pc := 9 \circseq Poll \circseq HandleAconst\_nullEPC \circseq pc := 10 \circseq \\
    % \t3 Poll \circseq HandleInvokespecialEPC(29) \circseq Poll \circseq CCInit \circseq Poll \circseq HandleAstoreEPC(1) \circseq \\
    % \t3 pc := 12 \circseq Poll \circseq HandleAloadEPC(1) \circseq pc := 13 \circseq Poll \circseq \\
    % \t3 HandleInvokevirtualEPC(32) \circseq Poll \circseq OpenInputStream \circseq Poll \circseq HandleAstoreEPC(2) \circseq \\
    % \t3 pc := 15 \circseq Poll \circseq HandleAloadEPC(1) \circseq pc := 16 \circseq Poll \circseq HandleInvokevirtualEPC(36) \circseq Poll \circseq OpenOutputStream \circseq Poll \circseq HandleAstoreEPC(3) \circseq pc := 18 \circseq Poll \circseq HandleIconstEPC(0) \circseq \\
    % \t3 pc := 19 \circseq Poll \circseq HandleAstoreEPC(4) \circseq pc := 20 \circseq Poll \circseq pc := 39 \circseq Poll \circseq \\
    % \t3 HandleAloadEPC(4) \circseq pc := 40 \circseq Poll \circseq HandleIconstEPC(10) \circseq pc := 41 \circseq \\
    % \t3 Poll \circseq \circvar value1, value2 : Word \circspot InterpreterPop2 \circseq \\
    % \t3 pc := \IF value1 \leq value2 \THEN 21 \ELSE 42 \\
    % \t2 {} \circelse pc = 9 \circthen HandleAconst\_nullEPC \circseq pc := 10 \circseq Poll \circseq HandleInvokespecialEPC(29) \circseq Poll \circseq CCInit \circseq Poll \circseq HandleAstoreEPC(1) \circseq pc := 12 \circseq Poll \circseq HandleAloadEPC(1) \circseq pc := 13 \circseq Poll \circseq HandleInvokevirtualEPC(32) \circseq Poll \circseq OpenInputStream \circseq Poll \circseq HandleAstoreEPC(2) \circseq pc := 15 \circseq Poll \circseq HandleAloadEPC(1) \circseq pc := 16 \circseq Poll \circseq HandleInvokevirtualEPC(36) \circseq Poll \circseq OpenOutputStream \circseq Poll \circseq HandleAstoreEPC(3) \circseq pc := 18 \circseq Poll \circseq HandleIconstEPC(0) \circseq \\
    % \t3 pc := 19 \circseq Poll \circseq HandleAstoreEPC(4) \circseq pc := 20 \circseq Poll \circseq pc := 39 \circseq Poll \circseq \\
    % \t3 HandleAloadEPC(4) \circseq pc := 40 \circseq Poll \circseq HandleIconstEPC(10) \circseq pc := 41 \circseq \\
    % \t3 Poll \circseq \circvar value1, value2 : Word \circspot InterpreterPop2 \circseq \\
    % \t3 pc := \IF value1 \leq value2 \THEN 21 \ELSE 42 \\
    % \t2 {} \circelse pc = 10 \circthen HandleInvokespecialEPC(29) \circseq Poll \circseq CCInit \circseq Poll \circseq HandleAstoreEPC(1) \circseq pc := 12 \circseq Poll \circseq HandleAloadEPC(1) \circseq pc := 13 \circseq Poll \circseq HandleInvokevirtualEPC(32) \circseq Poll \circseq OpenInputStream \circseq Poll \circseq HandleAstoreEPC(2) \circseq pc := 15 \circseq Poll \circseq HandleAloadEPC(1) \circseq pc := 16 \circseq Poll \circseq HandleInvokevirtualEPC(36) \circseq Poll \circseq OpenOutputStream \circseq Poll \circseq HandleAstoreEPC(3) \circseq pc := 18 \circseq Poll \circseq HandleIconstEPC(0) \circseq \\
    % \t3 pc := 19 \circseq Poll \circseq HandleAstoreEPC(4) \circseq pc := 20 \circseq Poll \circseq pc := 39 \circseq Poll \circseq \\
    % \t3 HandleAloadEPC(4) \circseq pc := 40 \circseq Poll \circseq HandleIconstEPC(10) \circseq pc := 41 \circseq \\
    % \t3 Poll \circseq \circvar value1, value2 : Word \circspot InterpreterPop2 \circseq \\
    % \t3 pc := \IF value1 \leq value2 \THEN 21 \ELSE 42 \\
    % \t2 {} \circelse pc = 11 \circthen HandleAstoreEPC(1) \circseq pc := 12 \circseq Poll \circseq HandleAloadEPC(1) \circseq \\
    % \t3 pc := 13 \circseq Poll \circseq HandleInvokevirtualEPC(32) \circseq Poll \circseq OpenInputStream \circseq Poll \circseq HandleAstoreEPC(2) \circseq pc := 15 \circseq Poll \circseq HandleAloadEPC(1) \circseq pc := 16 \circseq Poll \circseq HandleInvokevirtualEPC(36) \circseq Poll \circseq OpenOutputStream \circseq Poll \circseq HandleAstoreEPC(3) \circseq pc := 18 \circseq Poll \circseq HandleIconstEPC(0) \circseq \\
    % \t3 pc := 19 \circseq Poll \circseq HandleAstoreEPC(4) \circseq pc := 20 \circseq Poll \circseq pc := 39 \circseq Poll \circseq \\
    % \t3 HandleAloadEPC(4) \circseq pc := 40 \circseq Poll \circseq HandleIconstEPC(10) \circseq pc := 41 \circseq \\
    % \t3 Poll \circseq \circvar value1, value2 : Word \circspot InterpreterPop2 \circseq \\
    % \t3 pc := \IF value1 \leq value2 \THEN 21 \ELSE 42 \\
    % \t2 {} \circelse pc = 12 \circthen HandleAloadEPC(1) \circseq pc := 13 \circseq Poll \circseq HandleInvokevirtualEPC(32) \circseq Poll \circseq OpenInputStream \circseq Poll \circseq HandleAstoreEPC(2) \circseq pc := 15 \circseq Poll \circseq HandleAloadEPC(1) \circseq pc := 16 \circseq Poll \circseq HandleInvokevirtualEPC(36) \circseq Poll \circseq OpenOutputStream \circseq Poll \circseq HandleAstoreEPC(3) \circseq pc := 18 \circseq Poll \circseq HandleIconstEPC(0) \circseq \\
    % \t3 pc := 19 \circseq Poll \circseq HandleAstoreEPC(4) \circseq pc := 20 \circseq Poll \circseq pc := 39 \circseq Poll \circseq \\
    % \t3 HandleAloadEPC(4) \circseq pc := 40 \circseq Poll \circseq HandleIconstEPC(10) \circseq pc := 41 \circseq \\
    % \t3 Poll \circseq \circvar value1, value2 : Word \circspot InterpreterPop2 \circseq \\
    % \t3 pc := \IF value1 \leq value2 \THEN 21 \ELSE 42 \\
    % \t2 {} \circelse pc = 13 \circthen HandleInvokevirtualEPC(32) \circseq Poll \circseq OpenInputStream \circseq Poll \circseq HandleAstoreEPC(2) \circseq pc := 15 \circseq Poll \circseq HandleAloadEPC(1) \circseq pc := 16 \circseq Poll \circseq HandleInvokevirtualEPC(36) \circseq Poll \circseq OpenOutputStream \circseq Poll \circseq HandleAstoreEPC(3) \circseq pc := 18 \circseq Poll \circseq HandleIconstEPC(0) \circseq \\
    % \t3 pc := 19 \circseq Poll \circseq HandleAstoreEPC(4) \circseq pc := 20 \circseq Poll \circseq pc := 39 \circseq Poll \circseq \\
    % \t3 HandleAloadEPC(4) \circseq pc := 40 \circseq Poll \circseq HandleIconstEPC(10) \circseq pc := 41 \circseq \\
    % \t3 Poll \circseq \circvar value1, value2 : Word \circspot InterpreterPop2 \circseq \\
    % \t3 pc := \IF value1 \leq value2 \THEN 21 \ELSE 42 \\
    % \t2 {} \circelse pc = 14 \circthen HandleAstoreEPC(2) \circseq pc := 15 \circseq Poll \circseq HandleAloadEPC(1) \circseq \\
    % \t3 pc := 16 \circseq Poll \circseq HandleInvokevirtualEPC(36) \circseq Poll \circseq OpenOutputStream \circseq Poll \circseq HandleAstoreEPC(3) \circseq pc := 18 \circseq Poll \circseq HandleIconstEPC(0) \circseq \\
    % \t3 pc := 19 \circseq Poll \circseq HandleAstoreEPC(4) \circseq pc := 20 \circseq Poll \circseq pc := 39 \circseq Poll \circseq \\
    % \t3 HandleAloadEPC(4) \circseq pc := 40 \circseq Poll \circseq HandleIconstEPC(10) \circseq pc := 41 \circseq \\
    % \t3 Poll \circseq \circvar value1, value2 : Word \circspot InterpreterPop2 \circseq \\
    % \t3 pc := \IF value1 \leq value2 \THEN 21 \ELSE 42 \\
    % \t2 {} \circelse pc = 15 \circthen HandleAloadEPC(1) \circseq pc := 16 \circseq Poll \circseq HandleInvokevirtualEPC(36) \circseq Poll \circseq OpenOutputStream \circseq Poll \circseq HandleAstoreEPC(3) \circseq pc := 18 \circseq Poll \circseq HandleIconstEPC(0) \circseq \\
    % \t3 pc := 19 \circseq Poll \circseq HandleAstoreEPC(4) \circseq pc := 20 \circseq Poll \circseq pc := 39 \circseq Poll \circseq \\
    % \t3 HandleAloadEPC(4) \circseq pc := 40 \circseq Poll \circseq HandleIconstEPC(10) \circseq pc := 41 \circseq \\
    % \t3 Poll \circseq \circvar value1, value2 : Word \circspot InterpreterPop2 \circseq \\
    % \t3 pc := \IF value1 \leq value2 \THEN 21 \ELSE 42 \\
    % \t2 {} \circelse pc = 16 \circthen HandleInvokevirtualEPC(36) \circseq Poll \circseq OpenOutputStream \circseq Poll \circseq HandleAstoreEPC(3) \circseq pc := 18 \circseq Poll \circseq HandleIconstEPC(0) \circseq \\
    % \t3 pc := 19 \circseq Poll \circseq HandleAstoreEPC(4) \circseq pc := 20 \circseq Poll \circseq pc := 39 \circseq Poll \circseq \\
    % \t3 HandleAloadEPC(4) \circseq pc := 40 \circseq Poll \circseq HandleIconstEPC(10) \circseq pc := 41 \circseq \\
    % \t3 Poll \circseq \circvar value1, value2 : Word \circspot InterpreterPop2 \circseq \\
    % \t3 pc := \IF value1 \leq value2 \THEN 21 \ELSE 42 \\
    % \t2 {} \circelse pc = 17 \circthen HandleAstoreEPC(3) \circseq pc := 18 \circseq Poll \circseq HandleIconstEPC(0) \circseq \\
    % \t3 pc := 19 \circseq Poll \circseq HandleAstoreEPC(4) \circseq pc := 20 \circseq Poll \circseq pc := 39 \circseq Poll \circseq \\
    % \t3 HandleAloadEPC(4) \circseq pc := 40 \circseq Poll \circseq HandleIconstEPC(10) \circseq pc := 41 \circseq \\
    % \t3 Poll \circseq \circvar value1, value2 : Word \circspot InterpreterPop2 \circseq \\
    % \t3 pc := \IF value1 \leq value2 \THEN 21 \ELSE 42 \\
    % \t2 {} \circelse pc = 18 \circthen HandleIconstEPC(0) \circseq pc := 19 \circseq Poll \circseq HandleAstoreEPC(4) \circseq pc := 20 \circseq Poll \circseq pc := 39 \circseq Poll \circseq HandleAloadEPC(4) \circseq pc := 40 \circseq Poll \circseq HandleIconstEPC(10) \circseq pc := 41 \circseq Poll \circseq \circvar value1, value2 : Word \circspot InterpreterPop2 \circseq pc := \IF value1 \leq value2 \THEN 21 \ELSE 42 \\
    % \t2 {} \circelse pc = 19 \circthen HandleAstoreEPC(4) \circseq pc := 20 \circseq Poll \circseq pc := 39 \circseq Poll \circseq HandleAloadEPC(4) \circseq pc := 40 \circseq Poll \circseq HandleIconstEPC(10) \circseq pc := 41 \circseq Poll \circseq \circvar value1, value2 : Word \circspot InterpreterPop2 \circseq pc := \IF value1 \leq value2 \THEN 21 \ELSE 42 \\
    % \t2 {} \circelse pc = 20 \circthen pc := 39 \circseq Poll \circseq HandleAloadEPC(4) \circseq pc := 40 \circseq Poll \circseq HandleIconstEPC(10) \circseq pc := 41 \circseq Poll \circseq \circvar value1, value2 : Word \circspot InterpreterPop2 \circseq pc := \IF value1 \leq value2 \THEN 21 \ELSE 42 \\
    \t2 {} \circelse pc = 21 \circthen HandleAloadEPC(2) \circseq pc := 22 \circseq Poll \circseq \\
    \t3 HandleInvokevirtualEPC(40) \circseq Poll \circseq ConsoleInput\_Read \circseq Poll \circseq \\
    \t3 HandleInvokestaticEPC(46) \circseq Poll \circseq TPK\_F \circseq Poll \circseq HandleAstoreEPC(5) \circseq \\
    \t3 pc := 25 \circseq Poll \circseq HandleAloadEPC(5) \circseq pc := 26 \circseq HandleIconstEPC(400) \circseq \\
    \t3 pc := 27 \circseq Poll \circseq \circvar value1, value2 : Word \circspot InterpreterPop2 \circseq \\
    \t3 pc := \IF value1 \leq value2 \THEN 32 \ELSE 28 \\
    % \t2 {} \circelse pc = 22 \circthen HandleInvokevirtualEPC(40) \circseq Poll \circseq Read \circseq Poll \circseq HandleInvokestaticEPC(46) \circseq Poll \circseq F \circseq Poll \circseq HandleAstoreEPC(5) \circseq pc := 25 \circseq Poll \circseq HandleAloadEPC(5) \circseq pc := 26 \circseq HandleIconstEPC(400) \circseq pc := 27 \circseq Poll \circseq \circvar value1, value2 : Word \circspot InterpreterPop2 \circseq pc := \IF value1 \leq value2 \THEN 32 \ELSE 28 \\
    % \t2 {} \circelse pc = 23 \circthen HandleInvokestaticEPC(46) \circseq Poll \circseq F \circseq Poll \circseq HandleAstoreEPC(5) \circseq pc := 25 \circseq Poll \circseq HandleAloadEPC(5) \circseq pc := 26 \circseq HandleIconstEPC(400) \circseq pc := 27 \circseq Poll \circseq \circvar value1, value2 : Word \circspot InterpreterPop2 \circseq pc := \IF value1 \leq value2 \THEN 32 \ELSE 28 \\
    % \t2 {} \circelse pc = 24 \circthen HandleAstoreEPC(5) \circseq pc := 25 \circseq Poll \circseq HandleAloadEPC(5) \circseq pc := 26 \circseq HandleIconstEPC(400) \circseq pc := 27 \circseq Poll \circseq \circvar value1, value2 : Word \circspot InterpreterPop2 \circseq pc := \IF value1 \leq value2 \THEN 32 \ELSE 28 \\
    % \t2 {} \circelse pc = 25 \circthen HandleAloadEPC(5) \circseq pc := 26 \circseq Poll \circseq HandleIconstEPC(400) \circseq pc := 27 \circseq Poll \circseq \circvar value1, value2 : Word \circspot InterpreterPop2 \circseq pc := \IF value1 \leq value2 \THEN 32 \ELSE 28 \\
    % \t2 {} \circelse pc = 26 \circthen HandleIconstEPC(400) \circseq pc := 27 \circseq Poll \circseq \circvar value1, value2 : Word \circspot InterpreterPop2 \circseq pc := \IF value1 \leq value2 \THEN 32 \ELSE 28 \\
    % \t2 {} \circelse pc = 27 \circthen \circvar value1, value2 : Word \circspot InterpreterPop2 \circseq \\
    % \t3 pc := \IF value1 \leq value2 \THEN 32 \ELSE 28 \\
    % \t2 {} \circelse pc = 28 \circthen HandleAloadEPC(3) \circseq pc := 29 \circseq Poll \circseq HandleIconstEPC(0) \circseq pc := 30 \circseq Poll \circseq HandleInvokevirtualEPC(50) \circseq Poll \circseq Write \circseq Poll \circseq pc := 38 \circseq Poll \circseq HandleAstoreEPC(4) \circseq pc := 39 \circseq Poll \circseq HandleAloadEPC(4) \circseq pc := 40 \circseq Poll \circseq HandleIconstEPC(10) \circseq pc := 41 \circseq Poll \circseq \circvar value1, value2 : Word \circspot InterpreterPop2 \circseq pc := \IF value1 \leq value2 \THEN 21 \ELSE 42 \\
    % \t2 {} \circelse pc = 29 \circthen HandleIconstEPC(0) \circseq pc := 30 \circseq Poll \circseq HandleInvokevirtualEPC(50) \circseq Poll \circseq Write \circseq Poll \circseq pc := 38 \circseq Poll \circseq HandleAstoreEPC(4) \circseq pc := 39 \circseq Poll \circseq HandleAloadEPC(4) \circseq pc := 40 \circseq Poll \circseq HandleIconstEPC(10) \circseq pc := 41 \circseq Poll \circseq \circvar value1, value2 : Word \circspot InterpreterPop2 \circseq pc := \IF value1 \leq value2 \THEN 21 \ELSE 42 \\
    % \t2 {} \circelse pc = 30 \circthen HandleInvokevirtualEPC(50) \circseq Poll \circseq Write \circseq Poll \circseq pc := 38 \circseq Poll \circseq HandleAstoreEPC(4) \circseq pc := 39 \circseq Poll \circseq HandleAloadEPC(4) \circseq pc := 40 \circseq Poll \circseq HandleIconstEPC(10) \circseq pc := 41 \circseq Poll \circseq \circvar value1, value2 : Word \circspot InterpreterPop2 \circseq pc := \IF value1 \leq value2 \THEN 21 \ELSE 42 \\
    % \t2 {} \circelse pc = 31 \circthen pc := 38 \circseq Poll \circseq HandleAstoreEPC(4) \circseq pc := 39 \circseq Poll \circseq HandleAloadEPC(4) \circseq pc := 40 \circseq Poll \circseq HandleIconstEPC(10) \circseq pc := 41 \circseq Poll \circseq \circvar value1, value2 : Word \circspot InterpreterPop2 \circseq pc := \IF value1 \leq value2 \THEN 21 \ELSE 42 \\
    % \t2 {} \circelse pc = 32 \circthen HandleAloadEPC(3) \circseq pc := 33 \circseq Poll \circseq HandleAloadEPC(5) \circseq pc := 34 \circseq Poll \circseq HandleInvokevirtualEPC(50) \circseq Poll \circseq Write \circseq Poll \circseq HandleAloadEPC(4) \circseq pc := 36 \circseq Poll \circseq HandleIconstEPC(1) \circseq pc := 37 \circseq Poll \circseq HandleIaddEPC \circseq pc := 38 \circseq Poll \circseq HandleAstoreEPC(4) \circseq pc := 39  \\
    % \t2 {} \circelse pc = 33 \circthen HandleAloadEPC(5) \circseq pc := 34 \\
    % \t2 {} \circelse pc = 34 \circthen HandleInvokevirtualEPC(50) \\
    % \t2 {} \circelse pc = 35 \circthen HandleAloadEPC(4) \circseq pc := 36 \circseq Poll \circseq HandleIconstEPC(1) \circseq \\
    % \t3 pc := 37 \circseq Poll \circseq HandleIaddEPC \circseq pc := 38 \circseq Poll \circseq HandleAstoreEPC(4) \circseq \\
    % \t3 pc := 39 \\
    % \t2 {} \circelse pc = 36 \circthen HandleIconstEPC(1) \circseq pc := 37 \\
    % \t2 {} \circelse pc = 37 \circthen HandleIaddEPC \circseq pc := 38 \\
    % \t2 {} \circelse pc = 38 \circthen HandleAstoreEPC(4) \circseq pc := 39 \\
    \t2 {} \circelse pc = 39 \circthen HandleAloadEPC(4) \circseq pc := 40 \circseq Poll \circseq HandleIconstEPC(10) \circseq \\
    \t3 pc := 41 \circseq Poll \circseq \circvar value1, value2 : Word \circspot InterpreterPop2 \circseq \\
    \t3 pc := \IF value1 \leq value2 \THEN 21 \ELSE 42 \\
    % \t2 \t2 {} \circelse pc = 40 \circthen HandleIconstEPC(10) \circseq pc := 41 \\
    % \t2 {} \circelse pc = 41 \circthen \circvar value1, value2 : Word \circspot InterpreterPop2 \circseq \\
    % \t3 pc := \IF value1 \leq value2 \THEN 21 \ELSE 42 \\
    % \t2 {} \circelse pc = 42 \circthen HandleReturnEPC \\
    \t2 {} \cdots {} \\
    \t2 {} \circelse pc = 43 \circthen TPK\_F \\
    \t2 {} \cdots {} \\
    % \t2 {} \circelse pc = 44 \circthen HandleAloadEPC(0) \circseq pc := 45 \\
    % \t2 {} \circelse pc = 45 \circthen HandleIaddEPC \circseq pc := 46 \\
    % \t2 {} \circelse pc = 46 \circthen HandleAloadEPC(0) \circseq pc := 47 \\
    % \t2 {} \circelse pc = 47 \circthen HandleIaddEPC \circseq pc := 48 \\
    % \t2 {} \circelse pc = 48 \circthen HandleIconstEPC(5) \circseq pc := 49 \\
    % \t2 {} \circelse pc = 49 \circthen HandleIaddEPC \circseq pc := 50 \\
    % \t2 {} \circelse pc = 50 \circthen HandleAreturnEPC \\
    \t2 \circfi \circseq Poll \circseq Running \\
    \t1 \circfi
  \end{circus}
  \begin{circus}
    TPK\_F \circdef HandleAloadEPC(0) \circseq pc := 44 \circseq Poll \circseq HandleAloadEPC(0) \circseq pc := 45 \circseq \\
    \t1 Poll \circseq HandleIaddEPC \circseq pc := 46 \circseq Poll \circseq HandleAloadEPC(0) \circseq pc := 47 \circseq \\
    \t1 Poll \circseq HandleIaddEPC \circseq pc := 48 \circseq Poll \circseq HandleIconstEPC(5) \circseq pc := 49 \circseq \\
    \t1 Poll \circseq HandleIaddEPC \circseq pc := 50 \circseq Poll \circseq HandleAreturnEPC
  \end{circus}
  \caption{The $Running$ action after method call resolution}
  \label{method-call-resolution-example-figure}
\end{figure}

Calls to those methods can then be resolved, sequencing the method
invocation instruction with a call to the \Circus{} action representing its body and the
instructions following the method call. 
This occurs on line~\ref{algorithm-resolve-method-calls} of the
algorithm, and can be seen in
Figure~\ref{method-call-resolution-example-figure}, which shows our
example after method call resolution has been applied.

The target of each method call can be determined from the parameter to
the method invocation instruction.
This parameter is an index into the constant pool of the current class
that points to a method reference for the method being called.
The correct current class for each bytecode instruction is always
known, since the information on the method entries and ends is
contained in the class information, and there is a one-to-one mapping
between classes and blocks of bytecode instructions that form methods.
After the target of the method call has been determined, the
invocation instruction can be sequenced with a call to the
corresponding \Circus{} action.

An example of this is the occurrence of $HandleInvokestaticEPC(46)$ in
the sequence of actions at $pc = 21$.
As can be seen from Figure~\ref{example-model-figure}, the constant
pool index $46$ corresponds to the method identifier for the method
\texttt{f()} of the \texttt{TPK}.
The sequence of instructions corresponding to this method is in an
action $TPK\_F$, created in the previous step, on
line~\ref{algorithm-separate-complete-methods}.
This action is sequenced with the invocation instruction, with the
$Poll$ action inbetween (to allow thread switches before the first
instruction of the called method).
The instructions following the method call are then sequenced after
it, with another $Poll$ action (to allow thread switches following the
return instruction).
Method call resolution is described in more detail in
Section~\ref{resolve-method-calls-subsection}, where we define the
\Call{SeparateCompleteMethods}{} and \Call{ResolveMethodCalls}{}
procedures.

\begin{figure}[p!]
  \setlength{\zedindent}{0cm}
  \setlength{\zedtab}{0.3cm}
  \setlength{\zedleftsep}{0.1cm}
  \begin{circus}
    Running \circdef \\
    \t1 \circif frameStack = \emptyset \circthen \Skip \\
    \t1 {} \circelse frameStack \neq \emptyset \circthen {} \\
    \t2 {} \circif pc = 0 \circthen TPK\_APEHInit \\
    \t2 {} \cdots {} \\
    \t2 {} \circelse pc = 7 \circthen TPK\_HandleAsyncEvent \\
    % \t2 {} \circelse pc = 8 \circthen HandleDupEPC \circseq pc := 9 \circseq Poll \circseq HandleAconst\_nullEPC \circseq pc := 10 \circseq \\
    % \t3 Poll \circseq HandleInvokespecialEPC(29) \circseq Poll \circseq CCInit \circseq Poll \circseq HandleAstoreEPC(1) \circseq \\
    % \t3 pc := 12 \circseq Poll \circseq HandleAloadEPC(1) \circseq pc := 13 \circseq Poll \circseq \\
    % \t3 HandleInvokevirtualEPC(32) \circseq Poll \circseq OpenInputStream \circseq Poll \circseq HandleAstoreEPC(2) \circseq \\
    % \t3 pc := 15 \circseq Poll \circseq HandleAloadEPC(1) \circseq pc := 16 \circseq Poll \circseq HandleInvokevirtualEPC(36) \circseq Poll \circseq OpenOutputStream \circseq Poll \circseq HandleAstoreEPC(3) \circseq pc := 18 \circseq Poll \circseq HandleIconstEPC(0) \circseq \\
    % \t3 pc := 19 \circseq Poll \circseq HandleAstoreEPC(4) \circseq pc := 20 \circseq Poll \circseq pc := 39 \circseq Poll \circseq \\
    % \t3 HandleAloadEPC(4) \circseq pc := 40 \circseq Poll \circseq HandleIconstEPC(10) \circseq pc := 41 \circseq \\
    % \t3 Poll \circseq \circvar value1, value2 : Word \circspot InterpreterPop2 \circseq \\
    % \t3 pc := \IF value1 \leq value2 \THEN 21 \ELSE 42 \\
    % \t2 {} \circelse pc = 9 \circthen HandleAconst\_nullEPC \circseq pc := 10 \circseq Poll \circseq HandleInvokespecialEPC(29) \circseq Poll \circseq CCInit \circseq Poll \circseq HandleAstoreEPC(1) \circseq pc := 12 \circseq Poll \circseq HandleAloadEPC(1) \circseq pc := 13 \circseq Poll \circseq HandleInvokevirtualEPC(32) \circseq Poll \circseq OpenInputStream \circseq Poll \circseq HandleAstoreEPC(2) \circseq pc := 15 \circseq Poll \circseq HandleAloadEPC(1) \circseq pc := 16 \circseq Poll \circseq HandleInvokevirtualEPC(36) \circseq Poll \circseq OpenOutputStream \circseq Poll \circseq HandleAstoreEPC(3) \circseq pc := 18 \circseq Poll \circseq HandleIconstEPC(0) \circseq \\
    % \t3 pc := 19 \circseq Poll \circseq HandleAstoreEPC(4) \circseq pc := 20 \circseq Poll \circseq pc := 39 \circseq Poll \circseq \\
    % \t3 HandleAloadEPC(4) \circseq pc := 40 \circseq Poll \circseq HandleIconstEPC(10) \circseq pc := 41 \circseq \\
    % \t3 Poll \circseq \circvar value1, value2 : Word \circspot InterpreterPop2 \circseq \\
    % \t3 pc := \IF value1 \leq value2 \THEN 21 \ELSE 42 \\
    % \t2 {} \circelse pc = 10 \circthen HandleInvokespecialEPC(29) \circseq Poll \circseq CCInit \circseq Poll \circseq HandleAstoreEPC(1) \circseq pc := 12 \circseq Poll \circseq HandleAloadEPC(1) \circseq pc := 13 \circseq Poll \circseq HandleInvokevirtualEPC(32) \circseq Poll \circseq OpenInputStream \circseq Poll \circseq HandleAstoreEPC(2) \circseq pc := 15 \circseq Poll \circseq HandleAloadEPC(1) \circseq pc := 16 \circseq Poll \circseq HandleInvokevirtualEPC(36) \circseq Poll \circseq OpenOutputStream \circseq Poll \circseq HandleAstoreEPC(3) \circseq pc := 18 \circseq Poll \circseq HandleIconstEPC(0) \circseq \\
    % \t3 pc := 19 \circseq Poll \circseq HandleAstoreEPC(4) \circseq pc := 20 \circseq Poll \circseq pc := 39 \circseq Poll \circseq \\
    % \t3 HandleAloadEPC(4) \circseq pc := 40 \circseq Poll \circseq HandleIconstEPC(10) \circseq pc := 41 \circseq \\
    % \t3 Poll \circseq \circvar value1, value2 : Word \circspot InterpreterPop2 \circseq \\
    % \t3 pc := \IF value1 \leq value2 \THEN 21 \ELSE 42 \\
    % \t2 {} \circelse pc = 11 \circthen HandleAstoreEPC(1) \circseq pc := 12 \circseq Poll \circseq HandleAloadEPC(1) \circseq \\
    % \t3 pc := 13 \circseq Poll \circseq HandleInvokevirtualEPC(32) \circseq Poll \circseq OpenInputStream \circseq Poll \circseq HandleAstoreEPC(2) \circseq pc := 15 \circseq Poll \circseq HandleAloadEPC(1) \circseq pc := 16 \circseq Poll \circseq HandleInvokevirtualEPC(36) \circseq Poll \circseq OpenOutputStream \circseq Poll \circseq HandleAstoreEPC(3) \circseq pc := 18 \circseq Poll \circseq HandleIconstEPC(0) \circseq \\
    % \t3 pc := 19 \circseq Poll \circseq HandleAstoreEPC(4) \circseq pc := 20 \circseq Poll \circseq pc := 39 \circseq Poll \circseq \\
    % \t3 HandleAloadEPC(4) \circseq pc := 40 \circseq Poll \circseq HandleIconstEPC(10) \circseq pc := 41 \circseq \\
    % \t3 Poll \circseq \circvar value1, value2 : Word \circspot InterpreterPop2 \circseq \\
    % \t3 pc := \IF value1 \leq value2 \THEN 21 \ELSE 42 \\
    % \t2 {} \circelse pc = 12 \circthen HandleAloadEPC(1) \circseq pc := 13 \circseq Poll \circseq HandleInvokevirtualEPC(32) \circseq Poll \circseq OpenInputStream \circseq Poll \circseq HandleAstoreEPC(2) \circseq pc := 15 \circseq Poll \circseq HandleAloadEPC(1) \circseq pc := 16 \circseq Poll \circseq HandleInvokevirtualEPC(36) \circseq Poll \circseq OpenOutputStream \circseq Poll \circseq HandleAstoreEPC(3) \circseq pc := 18 \circseq Poll \circseq HandleIconstEPC(0) \circseq \\
    % \t3 pc := 19 \circseq Poll \circseq HandleAstoreEPC(4) \circseq pc := 20 \circseq Poll \circseq pc := 39 \circseq Poll \circseq \\
    % \t3 HandleAloadEPC(4) \circseq pc := 40 \circseq Poll \circseq HandleIconstEPC(10) \circseq pc := 41 \circseq \\
    % \t3 Poll \circseq \circvar value1, value2 : Word \circspot InterpreterPop2 \circseq \\
    % \t3 pc := \IF value1 \leq value2 \THEN 21 \ELSE 42 \\
    % \t2 {} \circelse pc = 13 \circthen HandleInvokevirtualEPC(32) \circseq Poll \circseq OpenInputStream \circseq Poll \circseq HandleAstoreEPC(2) \circseq pc := 15 \circseq Poll \circseq HandleAloadEPC(1) \circseq pc := 16 \circseq Poll \circseq HandleInvokevirtualEPC(36) \circseq Poll \circseq OpenOutputStream \circseq Poll \circseq HandleAstoreEPC(3) \circseq pc := 18 \circseq Poll \circseq HandleIconstEPC(0) \circseq \\
    % \t3 pc := 19 \circseq Poll \circseq HandleAstoreEPC(4) \circseq pc := 20 \circseq Poll \circseq pc := 39 \circseq Poll \circseq \\
    % \t3 HandleAloadEPC(4) \circseq pc := 40 \circseq Poll \circseq HandleIconstEPC(10) \circseq pc := 41 \circseq \\
    % \t3 Poll \circseq \circvar value1, value2 : Word \circspot InterpreterPop2 \circseq \\
    % \t3 pc := \IF value1 \leq value2 \THEN 21 \ELSE 42 \\
    % \t2 {} \circelse pc = 14 \circthen HandleAstoreEPC(2) \circseq pc := 15 \circseq Poll \circseq HandleAloadEPC(1) \circseq \\
    % \t3 pc := 16 \circseq Poll \circseq HandleInvokevirtualEPC(36) \circseq Poll \circseq OpenOutputStream \circseq Poll \circseq HandleAstoreEPC(3) \circseq pc := 18 \circseq Poll \circseq HandleIconstEPC(0) \circseq \\
    % \t3 pc := 19 \circseq Poll \circseq HandleAstoreEPC(4) \circseq pc := 20 \circseq Poll \circseq pc := 39 \circseq Poll \circseq \\
    % \t3 HandleAloadEPC(4) \circseq pc := 40 \circseq Poll \circseq HandleIconstEPC(10) \circseq pc := 41 \circseq \\
    % \t3 Poll \circseq \circvar value1, value2 : Word \circspot InterpreterPop2 \circseq \\
    % \t3 pc := \IF value1 \leq value2 \THEN 21 \ELSE 42 \\
    % \t2 {} \circelse pc = 15 \circthen HandleAloadEPC(1) \circseq pc := 16 \circseq Poll \circseq HandleInvokevirtualEPC(36) \circseq Poll \circseq OpenOutputStream \circseq Poll \circseq HandleAstoreEPC(3) \circseq pc := 18 \circseq Poll \circseq HandleIconstEPC(0) \circseq \\
    % \t3 pc := 19 \circseq Poll \circseq HandleAstoreEPC(4) \circseq pc := 20 \circseq Poll \circseq pc := 39 \circseq Poll \circseq \\
    % \t3 HandleAloadEPC(4) \circseq pc := 40 \circseq Poll \circseq HandleIconstEPC(10) \circseq pc := 41 \circseq \\
    % \t3 Poll \circseq \circvar value1, value2 : Word \circspot InterpreterPop2 \circseq \\
    % \t3 pc := \IF value1 \leq value2 \THEN 21 \ELSE 42 \\
    % \t2 {} \circelse pc = 16 \circthen HandleInvokevirtualEPC(36) \circseq Poll \circseq OpenOutputStream \circseq Poll \circseq HandleAstoreEPC(3) \circseq pc := 18 \circseq Poll \circseq HandleIconstEPC(0) \circseq \\
    % \t3 pc := 19 \circseq Poll \circseq HandleAstoreEPC(4) \circseq pc := 20 \circseq Poll \circseq pc := 39 \circseq Poll \circseq \\
    % \t3 HandleAloadEPC(4) \circseq pc := 40 \circseq Poll \circseq HandleIconstEPC(10) \circseq pc := 41 \circseq \\
    % \t3 Poll \circseq \circvar value1, value2 : Word \circspot InterpreterPop2 \circseq \\
    % \t3 pc := \IF value1 \leq value2 \THEN 21 \ELSE 42 \\
    % \t2 {} \circelse pc = 17 \circthen HandleAstoreEPC(3) \circseq pc := 18 \circseq Poll \circseq HandleIconstEPC(0) \circseq \\
    % \t3 pc := 19 \circseq Poll \circseq HandleAstoreEPC(4) \circseq pc := 20 \circseq Poll \circseq pc := 39 \circseq Poll \circseq \\
    % \t3 HandleAloadEPC(4) \circseq pc := 40 \circseq Poll \circseq HandleIconstEPC(10) \circseq pc := 41 \circseq \\
    % \t3 Poll \circseq \circvar value1, value2 : Word \circspot InterpreterPop2 \circseq \\
    % \t3 pc := \IF value1 \leq value2 \THEN 21 \ELSE 42 \\
    % \t2 {} \circelse pc = 18 \circthen HandleIconstEPC(0) \circseq pc := 19 \circseq Poll \circseq HandleAstoreEPC(4) \circseq pc := 20 \circseq Poll \circseq pc := 39 \circseq Poll \circseq HandleAloadEPC(4) \circseq pc := 40 \circseq Poll \circseq HandleIconstEPC(10) \circseq pc := 41 \circseq Poll \circseq \circvar value1, value2 : Word \circspot InterpreterPop2 \circseq pc := \IF value1 \leq value2 \THEN 21 \ELSE 42 \\
    % \t2 {} \circelse pc = 19 \circthen HandleAstoreEPC(4) \circseq pc := 20 \circseq Poll \circseq pc := 39 \circseq Poll \circseq HandleAloadEPC(4) \circseq pc := 40 \circseq Poll \circseq HandleIconstEPC(10) \circseq pc := 41 \circseq Poll \circseq \circvar value1, value2 : Word \circspot InterpreterPop2 \circseq pc := \IF value1 \leq value2 \THEN 21 \ELSE 42 \\
    % \t2 {} \circelse pc = 20 \circthen pc := 39 \circseq Poll \circseq HandleAloadEPC(4) \circseq pc := 40 \circseq Poll \circseq HandleIconstEPC(10) \circseq pc := 41 \circseq Poll \circseq \circvar value1, value2 : Word \circspot InterpreterPop2 \circseq pc := \IF value1 \leq value2 \THEN 21 \ELSE 42 \\
    % \t2 {} \circelse pc = 21 \circthen HandleAloadEPC(2) \circseq pc := 22 \circseq Poll \circseq HandleInvokevirtualEPC(40) \circseq Poll \circseq Read \circseq Poll \circseq HandleInvokestaticEPC(46) \circseq Poll \circseq F \circseq Poll \circseq HandleAstoreEPC(5) \circseq pc := 25 \circseq Poll \circseq HandleAloadEPC(5) \circseq pc := 26 \circseq HandleIconstEPC(400) \circseq pc := 27 \circseq Poll \circseq \circvar value1, value2 : Word \circspot InterpreterPop2 \circseq pc := \IF value1 \leq value2 \THEN 32 \ELSE 28 \\
    % \t2 {} \circelse pc = 22 \circthen HandleInvokevirtualEPC(40) \circseq Poll \circseq Read \circseq Poll \circseq HandleInvokestaticEPC(46) \circseq Poll \circseq F \circseq Poll \circseq HandleAstoreEPC(5) \circseq pc := 25 \circseq Poll \circseq HandleAloadEPC(5) \circseq pc := 26 \circseq HandleIconstEPC(400) \circseq pc := 27 \circseq Poll \circseq \circvar value1, value2 : Word \circspot InterpreterPop2 \circseq pc := \IF value1 \leq value2 \THEN 32 \ELSE 28 \\
    % \t2 {} \circelse pc = 23 \circthen HandleInvokestaticEPC(46) \circseq Poll \circseq F \circseq Poll \circseq HandleAstoreEPC(5) \circseq pc := 25 \circseq Poll \circseq HandleAloadEPC(5) \circseq pc := 26 \circseq HandleIconstEPC(400) \circseq pc := 27 \circseq Poll \circseq \circvar value1, value2 : Word \circspot InterpreterPop2 \circseq pc := \IF value1 \leq value2 \THEN 32 \ELSE 28 \\
    % \t2 {} \circelse pc = 24 \circthen HandleAstoreEPC(5) \circseq pc := 25 \circseq Poll \circseq HandleAloadEPC(5) \circseq pc := 26 \circseq HandleIconstEPC(400) \circseq pc := 27 \circseq Poll \circseq \circvar value1, value2 : Word \circspot InterpreterPop2 \circseq pc := \IF value1 \leq value2 \THEN 32 \ELSE 28 \\
    % \t2 {} \circelse pc = 25 \circthen HandleAloadEPC(5) \circseq pc := 26 \circseq Poll \circseq HandleIconstEPC(400) \circseq pc := 27 \circseq Poll \circseq \circvar value1, value2 : Word \circspot InterpreterPop2 \circseq pc := \IF value1 \leq value2 \THEN 32 \ELSE 28 \\
    % \t2 {} \circelse pc = 26 \circthen HandleIconstEPC(400) \circseq pc := 27 \circseq Poll \circseq \circvar value1, value2 : Word \circspot InterpreterPop2 \circseq pc := \IF value1 \leq value2 \THEN 32 \ELSE 28 \\
    % \t2 {} \circelse pc = 27 \circthen \circvar value1, value2 : Word \circspot InterpreterPop2 \circseq \\
    % \t3 pc := \IF value1 \leq value2 \THEN 32 \ELSE 28 \\
    % \t2 {} \circelse pc = 28 \circthen HandleAloadEPC(3) \circseq pc := 29 \circseq Poll \circseq HandleIconstEPC(0) \circseq pc := 30 \circseq Poll \circseq HandleInvokevirtualEPC(50) \circseq Poll \circseq Write \circseq Poll \circseq pc := 38 \circseq Poll \circseq HandleAstoreEPC(4) \circseq pc := 39 \circseq Poll \circseq HandleAloadEPC(4) \circseq pc := 40 \circseq Poll \circseq HandleIconstEPC(10) \circseq pc := 41 \circseq Poll \circseq \circvar value1, value2 : Word \circspot InterpreterPop2 \circseq pc := \IF value1 \leq value2 \THEN 21 \ELSE 42 \\
    % \t2 {} \circelse pc = 29 \circthen HandleIconstEPC(0) \circseq pc := 30 \circseq Poll \circseq HandleInvokevirtualEPC(50) \circseq Poll \circseq Write \circseq Poll \circseq pc := 38 \circseq Poll \circseq HandleAstoreEPC(4) \circseq pc := 39 \circseq Poll \circseq HandleAloadEPC(4) \circseq pc := 40 \circseq Poll \circseq HandleIconstEPC(10) \circseq pc := 41 \circseq Poll \circseq \circvar value1, value2 : Word \circspot InterpreterPop2 \circseq pc := \IF value1 \leq value2 \THEN 21 \ELSE 42 \\
    % \t2 {} \circelse pc = 30 \circthen HandleInvokevirtualEPC(50) \circseq Poll \circseq Write \circseq Poll \circseq pc := 38 \circseq Poll \circseq HandleAstoreEPC(4) \circseq pc := 39 \circseq Poll \circseq HandleAloadEPC(4) \circseq pc := 40 \circseq Poll \circseq HandleIconstEPC(10) \circseq pc := 41 \circseq Poll \circseq \circvar value1, value2 : Word \circspot InterpreterPop2 \circseq pc := \IF value1 \leq value2 \THEN 21 \ELSE 42 \\
    % \t2 {} \circelse pc = 31 \circthen pc := 38 \circseq Poll \circseq HandleAstoreEPC(4) \circseq pc := 39 \circseq Poll \circseq HandleAloadEPC(4) \circseq pc := 40 \circseq Poll \circseq HandleIconstEPC(10) \circseq pc := 41 \circseq Poll \circseq \circvar value1, value2 : Word \circspot InterpreterPop2 \circseq pc := \IF value1 \leq value2 \THEN 21 \ELSE 42 \\
    % \t2 {} \circelse pc = 32 \circthen HandleAloadEPC(3) \circseq pc := 33 \circseq Poll \circseq HandleAloadEPC(5) \circseq pc := 34 \circseq Poll \circseq HandleInvokevirtualEPC(50) \circseq Poll \circseq Write \circseq Poll \circseq HandleAloadEPC(4) \circseq pc := 36 \circseq Poll \circseq HandleIconstEPC(1) \circseq pc := 37 \circseq Poll \circseq HandleIaddEPC \circseq pc := 38 \circseq Poll \circseq HandleAstoreEPC(4) \circseq pc := 39 \circseq Poll \circseq HandleAloadEPC(4) \circseq pc := 40 \circseq Poll \circseq HandleIconstEPC(10) \circseq pc := 41 \circseq Poll \circseq \circvar value1, value2 : Word \circspot InterpreterPop2 \circseq pc := \IF value1 \leq value2 \THEN 21 \ELSE 42 \\
    % \t2 {} \circelse pc = 33 \circthen HandleAloadEPC(5) \circseq pc := 34 \circseq Poll \circseq HandleInvokevirtualEPC(50) \circseq Poll \circseq Write \circseq Poll \circseq HandleAloadEPC(4) \circseq pc := 36 \circseq Poll \circseq HandleIconstEPC(1) \circseq pc := 37 \circseq Poll \circseq HandleIaddEPC \circseq pc := 38 \circseq Poll \circseq HandleAstoreEPC(4) \circseq pc := 39 \circseq Poll \circseq HandleAloadEPC(4) \circseq pc := 40 \circseq Poll \circseq HandleIconstEPC(10) \circseq pc := 41 \circseq Poll \circseq \circvar value1, value2 : Word \circspot InterpreterPop2 \circseq pc := \IF value1 \leq value2 \THEN 21 \ELSE 42 \\
    % \t2 {} \circelse pc = 34 \circthen HandleInvokevirtualEPC(50) \circseq Poll \circseq Write \circseq Poll \circseq HandleAloadEPC(4) \circseq pc := 36 \circseq Poll \circseq HandleIconstEPC(1) \circseq pc := 37 \circseq Poll \circseq HandleIaddEPC \circseq pc := 38 \circseq Poll \circseq HandleAstoreEPC(4) \circseq pc := 39 \circseq Poll \circseq HandleAloadEPC(4) \circseq pc := 40 \circseq Poll \circseq HandleIconstEPC(10) \circseq pc := 41 \circseq Poll \circseq \circvar value1, value2 : Word \circspot InterpreterPop2 \circseq pc := \IF value1 \leq value2 \THEN 21 \ELSE 42 \\
    % \t2 {} \circelse pc = 35 \circthen HandleAloadEPC(4) \circseq pc := 36 \circseq Poll \circseq HandleIconstEPC(1) \circseq pc := 37 \circseq Poll \circseq HandleIaddEPC \circseq pc := 38 \circseq Poll \circseq HandleAstoreEPC(4) \circseq pc := 39 \circseq Poll \circseq HandleAloadEPC(4) \circseq pc := 40 \circseq Poll \circseq HandleIconstEPC(10) \circseq pc := 41 \circseq Poll \circseq \circvar value1, value2 : Word \circspot InterpreterPop2 \circseq pc := \IF value1 \leq value2 \THEN 21 \ELSE 42 \\
    % \t2 {} \circelse pc = 36 \circthen HandleIconstEPC(1) \circseq pc := 37 \circseq Poll \circseq HandleIaddEPC \circseq pc := 38 \circseq Poll \circseq HandleAstoreEPC(4) \circseq pc := 39 \circseq Poll \circseq HandleAloadEPC(4) \circseq pc := 40 \circseq Poll \circseq HandleIconstEPC(10) \circseq pc := 41 \circseq Poll \circseq \circvar value1, value2 : Word \circspot InterpreterPop2 \circseq pc := \IF value1 \leq value2 \THEN 21 \ELSE 42 \\
    % \t2 {} \circelse pc = 37 \circthen HandleIaddEPC \circseq pc := 38 \circseq Poll \circseq HandleAstoreEPC(4) \circseq pc := 39 \circseq Poll \circseq HandleAloadEPC(4) \circseq pc := 40 \circseq Poll \circseq HandleIconstEPC(10) \circseq pc := 41 \circseq Poll \circseq \circvar value1, value2 : Word \circspot InterpreterPop2 \circseq pc := \IF value1 \leq value2 \THEN 21 \ELSE 42 \\
    % \t2 {} \circelse pc = 38 \circthen HandleAstoreEPC(4) \circseq pc := 39 \circseq Poll \circseq HandleAloadEPC(4) \circseq pc := 40 \circseq Poll \circseq HandleIconstEPC(10) \circseq pc := 41 \circseq Poll \circseq \circvar value1, value2 : Word \circspot InterpreterPop2 \circseq pc := \IF value1 \leq value2 \THEN 21 \ELSE 42 \\
    % \t2 {} \circelse pc = 39 \circthen HandleAloadEPC(4) \circseq pc := 40 \circseq Poll \circseq HandleIconstEPC(10) \circseq pc := 41 \circseq Poll \circseq \circvar value1, value2 : Word \circspot InterpreterPop2 \circseq pc := \IF value1 \leq value2 \THEN 21 \ELSE 42 \\
    % \t2 {} \circelse pc = 40 \circthen HandleIconstEPC(10) \circseq pc := 41 \circseq Poll \circseq \circvar value1, value2 : Word \circspot InterpreterPop2 \circseq pc := \IF value1 \leq value2 \THEN 21 \ELSE 42 \\
    % \t2 {} \circelse pc = 41 \circthen \circvar value1, value2 : Word \circspot InterpreterPop2 \circseq \\
    % \t3 pc := \IF value1 \leq value2 \THEN 21 \ELSE 42 \\
    % \t2 {} \circelse pc = 42 \circthen HandleReturnEPC \\
    \t2 {} \cdots {} \\
    \t2 {} \circelse pc = 43 \circthen F \\
    \t2 {} \cdots {} \\
    \t2 \circfi \circseq Poll \circseq Running \\
    \t1 \circfi
  \end{circus}
  \vspace{-1cm}
  \begin{circus}
    TPK\_APEHInit \circdef HandleAloadEPC(0) \circseq pc := 1 \circseq Poll \circseq HandleAloadEPC(1) \circseq pc := 2 \circseq \\
    \t3 Poll \circseq HandleAloadEPC(2) \circseq pc := 3 \circseq Poll \circseq HandleAloadEPC(3) \circseq pc := 4 \circseq \\
    \t3 Poll \circseq HandleAloadEPC(4) \circseq pc := 5 \circseq Poll \circseq HandleInvokespecialEPC(8) \circseq \\
    \t3 Poll \circseq AperiodicEventHandler\_APEHInit \circseq Poll \circseq HandleReturnEPC \\
  \end{circus}
  \vspace{-1.5cm}
  \begin{circus}
    TPK\_HandleAsyncEvent \circdef HandleNewEPC(27) \circseq pc := 8 \circseq Poll \circseq HandleDupEPC \circseq \\
    \t1 {} \cdots {} \\
    % \t3 Poll \circseq HandleAconst\_nullEPC \circseq pc := 10 \circseq Poll \circseq HandleInvokespecialEPC(29) \circseq \\
    % \t3 Poll \circseq ConsoleConnection\_CCInit \circseq Poll \circseq HandleAstoreEPC(1) \circseq pc := 12 \circseq Poll \circseq \\
    % \t3 HandleAloadEPC(1) \circseq pc := 13 \circseq Poll \circseq HandleInvokevirtualEPC(32) \circseq Poll \circseq \\
    % \t3 OpenInputStream \circseq Poll \circseq HandleAstoreEPC(2) \circseq pc := 15 \circseq Poll \circseq \\
    % \t3 HandleAloadEPC(1) \circseq pc := 16 \circseq Poll \circseq HandleInvokevirtualEPC(36) \circseq Poll \circseq \\
    % \t3 OpenOutputStream \circseq Poll \circseq HandleAstoreEPC(3) \circseq pc := 18 \circseq Poll \circseq \\
    % \t3 HandleIconstEPC(0) \circseq pc := 19 \circseq Poll \circseq HandleAstoreEPC(4) \circseq pc := 20 \circseq \\
    \t1 Poll \circseq pc := 39 \circseq Poll \circseq \circmu Y \circspot \\
    \t2 HandleAloadEPC(4) \circseq pc := 40 \circseq Poll \circseq HandleIconstEPC(10) \circseq pc := 41 \circseq \\
    \t2 Poll \circseq \circvar value1, value2 : Word \circspot InterpreterPop2 \circseq \\
    \t2 pc := \IF value1 \leq value2 \THEN 21 \ELSE 42 \circseq Poll \circseq \\
    \t2 \circif value1 \leq value2 \circthen HandleAloadEPC(2) \circseq pc := 22 \circseq Poll \circseq \\
    \t3 {} \cdots {} \\
    \t3 HandleIconstEPC(400) \circseq pc := 27 \circseq Poll \circseq \circvar value1, value2 : Word \circspot \\
    \t3 InterpreterPop2 \circseq pc := \IF value1 \leq value2 \THEN 32 \ELSE 28 \circseq Poll \circseq \\
    \t3 \circif value1 \leq value2 \circthen HandleAloadEPC(3) \circseq pc := 33 \circseq Poll \circseq \\
    \t4 HandleAloadEPC(5) \circseq pc := 34 \circseq Poll \circseq HandleInvokevirtualEPC(50) \circseq \\
    \t4 Poll \circseq ConsoleOutput\_Write \\
    \t3 {} \circelse value1 > value2 \circthen HandleAloadEPC(3) \circseq pc := 29 \circseq Poll \circseq \\
    \t4  HandleIconstEPC(0) \circseq pc := 30 \circseq  Poll \circseq HandleInvokevirtualEPC(50) \circseq \\
    \t4 Poll \circseq ConsoleOutput\_Write \\
    \t3 \circfi \circseq pc := 35 \circseq Poll \circseq  HandleAloadEPC(4) \circseq pc := 36 \circseq Poll \circseq \\
    \t3 HandleIconstEPC(1) \circseq pc := 37 \circseq Poll \circseq HandleIaddEPC \circseq pc := 38 \circseq \\
    \t3 Poll \circseq HandleAstoreEPC(4) \circseq pc := 39 \circseq Poll \circseq Y \\
    \t3 {} \circelse value1 > value2 \circthen  HandleReturnEPC \\
    \t2 \circfi \\
  \end{circus}
  \caption{The $Running$ action after all the methods are separated}
  \label{final-method-separation-example-figure}
\end{figure}

As mentioned previously, these steps are then repeated, in the loop
beginning at line~\ref{algorithm-method-loop} to introduce the loops
and conditionals in methods that would otherwise have unresolved
method calls in the middle of loops and conditionals.
Afterwards, those methods can be separated out and this loop, conditional and
method resolution repeated until every method has been separated out
in this way.

We do not allow recursion to ensure this terminates, since this
requires all the methods called by a given method to be resolved
before the method itself can be resolved.
This is a sensible restriction since recursion is not normally allowed
in safety-critical applications because of the potential for
unpredictable failure due to stack overflow.

The $Running$ action of our example at the end of the loop, when all
loops and conditionals have been introduced, all the methods have been
separated out, and all method calls have been resolved, is shown in
Figure~\ref{final-method-separation-example-figure}.
At this point, the choice over the $pc$ value maps entry points of
methods onto the actions representing those methods, with the other
$pc$ values now redundant.

\begin{figure}[t!]
  \setlength{\zedindent}{0cm}
  \setlength{\zedtab}{0.3cm}
  \setlength{\zedleftsep}{0.1cm}
  \begin{circusaction}
    ExecuteMethod \circdef \circval cid : ClassID; mid : MethodID \circspot \\
    \t1 \circif (cid, mid) = (TPKClassID, APEHInit) \circthen TPK\_APEHInit \\
    \t1 {} \circelse (cid, mid) = (TPKClassID, handleAsyncEvent) \circthen TPK\_HandleAsyncEvent \\
    \t1 {} \circelse (cid, mid) = (TPKClassID, f) \circthen TPK\_F \\
    \t1 {} \cdots {} \\
    \t1 \circfi
  \end{circusaction}
  \begin{circusaction}
    MainThread \circdef \\
    \t1 \circblockbegin
    \circvar cid : ClassID; method : MethodID; methodArgs : \seq Word \circspot \\
    interpreter?c?m?a \then cid, method, methodArgs := c, m, a \circseq  \\
    \lschexpract \exists class? == cs~cid; baseFrame? == \true @ InterpreterNewStackFrame \rschexpract \circseq \\
    ExecuteMethod(cid, method) \circseq MainThread \\
    \circblockend \\
    \t1 {} \extchoice {} \\
    \t1 CEEswitchThread?from?to \prefixcolon (from = main) \then Blocked \circseq MainThread
  \end{circusaction}
  \begin{circusaction}
    NotStarted \circdef \circvar cid : ClassID; method : MethodID; methodArgs : \seq Word \circspot \\
    \t1 CEEstartThread?toStart?bsid?c?m?a \prefixcolon (toStart = thread) \then {} \\
    \t2 cid, method, methodArgs := c, m, a \circseq \\
    \t1 \lschexpract \exists class? == cs~cid; baseFrame? == \true @ InterpreterNewStackFrame \rschexpract \circseq \\
    \t1 Blocked \circseq ExecuteMethod(cid, method) \circseq CEEremoveThread!thread \then NotStarted
  \end{circusaction}
  \caption{The $ExecuteMethod$, $NotStarted$, and $MainThread$ actions after main action refinement}
  \label{refine-main-actions-example-figure}
\end{figure}

\begin{figure}[p!]
  \begin{circus}
    HandleAsyncEvent \circdef HandleNewEPC(27) \circseq Poll \circseq HandleDupEPC \circseq \\
    \t1 Poll \circseq HandleAconst\_nullEPC \circseq Poll \circseq HandleInvokespecialEPC(29) \circseq \\
    \t1 Poll \circseq CCInit \circseq Poll \circseq HandleAstoreEPC(1) \circseq Poll \circseq \\
    \t1 HandleAloadEPC(1) \circseq Poll \circseq HandleInvokevirtualEPC(32) \circseq Poll \circseq \\
    \t1 OpenInputStream \circseq Poll \circseq HandleAstoreEPC(2) \circseq Poll \circseq \\
    \t1 HandleAloadEPC(1) \circseq Poll \circseq HandleInvokevirtualEPC(36) \circseq Poll \circseq \\
    \t1 OpenOutputStream \circseq Poll \circseq HandleAstoreEPC(3) \circseq Poll \circseq \\
    \t1 HandleIconstEPC(0) \circseq Poll \circseq HandleAstoreEPC(4) \circseq \\
    \t1 Poll \circseq Poll \circseq HandleAloadEPC(4) \circseq Poll \circseq \\
    \t1 HandleIconstEPC(10) \circseq Poll \circseq \circvar value1, value2 : Word \circspot \\
    \t1 InterpreterPop2 \circseq Poll \circseq \circmu Y \circspot \\
    \t1 \circif value1 \leq value2 \circthen HandleAloadEPC(2) \circseq Poll \circseq \\
    \t2 HandleInvokevirtualEPC(40) \circseq Poll \circseq Read \circseq Poll \circseq \\
    \t2 HandleInvokestaticEPC(46) \circseq Poll \circseq F \circseq Poll \circseq \\
    \t2 HandleAstoreEPC(5) \circseq Poll \circseq HandleAloadEPC(5) \circseq \\
    \t2 HandleIconstEPC(400) \circseq Poll \circseq \circvar value1, value2 : Word \circspot \\
    \t2 InterpreterPop2 \circseq Poll \circseq \\
    \t2 \circif value1 \leq value2 \circthen HandleAloadEPC(3) \circseq Poll \circseq \\
    \t3 HandleAloadEPC(5) \circseq Poll \circseq HandleInvokevirtualEPC(50) \circseq \\
    \t3 Poll \circseq Write \circseq Poll \circseq HandleAloadEPC(4) \circseq Poll \circseq \\
    \t3 HandleIconstEPC(1) \circseq Poll \circseq HandleIaddEPC \circseq Poll \circseq \\
    \t3 HandleAstoreEPC(4) \circseq Poll \circseq HandleAloadEPC(4) \circseq Poll \circseq \\
    \t3 HandleIconstEPC(10) \circseq Poll \circseq \circvar value1, value2 : Word \circspot \\
    \t3 InterpreterPop2 \circseq Poll \circseq \\
    \t3 \circif value1 \leq value2 \circthen Y \\
    \t3 {} \circelse value1 > value2 \circthen HandleReturnEPC \\
    \t3 \circfi \\
    \t2 {} \circelse value1 > value2 \circthen HandleAloadEPC(3) \circseq Poll \circseq \\
    \t3 HandleIconstEPC(0) \circseq Poll \circseq HandleInvokevirtualEPC(50) \circseq \\
    \t3 Poll \circseq Write \circseq Poll \circseq Poll \circseq HandleAstoreEPC(4) \circseq Poll \circseq \\
    \t3 HandleAloadEPC(4) \circseq Poll \circseq HandleIconstEPC(10) \circseq Poll \circseq \\
    \t3 \circvar value1, value2 : Word \circspot InterpreterPop2 \circseq Poll \circseq \\
    \t3 \circif value1 \leq value2 \circthen Y \\
    \t3 {} \circelse value1 > value2 \circthen  HandleReturnEPC \\
    \t3 \circfi \\
    \t2 \circfi \\
    \t1 {} \circelse value1 > value2 \circthen  HandleReturnEPC \\
    \t1 \circfi \\
  \end{circus}
  \caption{The $HandleAsyncEvent$ action after $pc$ has been eliminated from the state}
  \label{pc-elimination-HandleAsyncEvent-example-figure}
\end{figure}

The next step is then to eliminate these redundant paths and remove
the dependency on $pc$ to select the method action.
This occurs at line~\ref{algorithm-refine-main-actions} of the
algorithm, in which the $NotStarted$ and $MainThread$ actions are
refined to replace the $Running$ action with an $ExecuteMethod$ action
that contains a choice of method action based on the method and class
identifier of the method.
This can be seen in Figure~\ref{refine-main-actions-example-figure},
which shows the $ExecuteMethod$ action corresponding to our example,
and the refined $NotStarted$ and $MainThread$ actions that reference
it.
We describe this refinement in more detail in
Section~\ref{refine-main-actions-subsection}, where we define the
\Call{RefineMainActions}{} procedure.

When all of the previous steps are completed, reliance on $pc$ to
determine control flow has been completely removed.
The $pc$ state component can then be removed in a simple data
refinement that also removes all the assignments to $pc$, resulting in
the $HandleAsyncEvent$ action shown in
Figure~\ref{pc-elimination-HandleAsyncEvent-example-figure}.

The remaining instruction handling actions then only affect the stack,
the removal of which is the concern of the next stage of the
compilation strategy.
The data refinement to remove $pc$ is applied at the end of the
algorithm, on line~\ref{algorithm-remove-pc-from-state}, and is
described in more detail in
Section~\ref{remove-pc-from-state-subsection}, where we define the
\Call{RemovePCFromState}{} procedure.

We now proceed to describe each of the steps of program counter
elimination in more detail.

\FloatBarrier

\subsection{Expand Bytecode}
\label{expand-bytecode-subsection}

Before the control flow can be introduced, the bytecode instructions
provided in the $bc$ parameter to $Thr$ must be expanded to allow
consideration of their semantics.
This is achieved using the procedure shown in
Algorithm~\ref{expand-bytecode-algorithm}.
\begin{algorithm}
  \begin{algorithmic}[1]
    \Procedure{ExpandBytecode}{}
    \State \Call{IntroduceChoiceOverPC}{} \label{algorithm-introduce-choice-over-pc}
    \For{$pc \gets \dom bc$}
    \State \Call{CollapseHandleInstruction}{} \label{algorithm-collapse-handle-instruction}
    \State \Call{ExpandHandleAction}{} \label{algorithm-expand-handle-action}
    \EndFor
    \EndProcedure
  \end{algorithmic}
  \caption{Expand Bytecode}
  \label{expand-bytecode-algorithm}
\end{algorithm}
This begins on line~\ref{algorithm-introduce-choice-over-pc} by
introducing a choice over all the possible values of $pc$ associated
with the $HandleInstruction$ action in $Running$, replacing
$HandleInstruction$ with a choice of the form shown below.
\begin{circus}
  \circif pc = 0 \circthen HandleInstruction \\
  {} \circelse pc = 1 \circthen HandleInstruction \\
  {} \circelse pc = 2 \circthen HandleInstruction \\
  {} \cdots {} \\
  \circfi
\end{circus}
After that, we operate on the occurrence of $HandleInstruction$ at
each $pc$ value, replacing it with its definition, which is a choice
between actions to handle each type of instruction (e.g.\ $HandleDup$,
$HandleAload$ etc.).
For brevity, we refer to these handling actions that define
$HandleInstruction$ as $Handle^*$ actions.
Since the value of $bc$ at a given $pc$ value is known, we can
determine which of the $Handle^*$ actions is chosen for each
occurrence of $HandleInstruction$.
The occurrences of $HandleInstruction$ are therefore collapsed to the
appropriate $Handle^*$ actions on line~\ref{algorithm-collapse-handle-instruction}.
This produces the following choice for our example.
\begin{circus}
  \circif pc = 0 \circthen HandleAload \\
  {} \circelse pc = 1 \circthen HandleAload \\
  {} \circelse pc = 2 \circthen HandleAload \\
  {} \cdots {} \\
  {} \circelse pc = 7 \circthen HandleNew \\
  {} \circelse pc = 8 \circthen HandleDup \\
  {} \circelse pc = 9 \circthen HandleAconst\_null \\
  {} \cdots {} \\
  \circfi
\end{circus}
The $Handle^*$ actions are then replaced with new actions.
Those new actions are not guarded on the value of $bc$ at the
current $pc$ value, since the choice those guards mediate has already
been collapsed.
The parameters of the bytecode instructions are also transferred to
become parameters of the new actions.
Finally, the updates to $pc$ contained in the $Handle^*$ actions are
extracted in the form of assignments to $pc$.

This final transformation is not carried out in the case of the method
invocation and return instructions, where the $pc$ updates are closely
connected to the operations on the stack and require special handling.
The new actions' names are formed by appending $EPC$ to the names of
the $Handle^*$ actions.
The overall mapping from bytecode instructions to the new actions is
shown in Table~\ref{handle-action-table}.
\begin{table}
  \centering
  \begin{tabular}{lp{8.5cm}}
    \hline
    Bytecode ($bc~i$) & Action ($handleAction(bc~i)$) \\
    \hline
    $aconst\_null$ & $HandleAconst\_nullEPC \circseq pc := i+1$ \\
    $dup$ & $HandleDupEPC \circseq pc := i+1$ \\
    $aload~lvi$ & $HandleAloadEPC(lvi) \circseq pc := i+1$ \\
    $astore~lvi$ & $HandleAstoreEPC(lvi) \circseq pc := i+1$ \\
    $iadd$ & $HandleIaddEPC \circseq pc := i+1$ \\
    $iconst~n$ & $HandleIconstEPC(n) \circseq pc := i+1$ \\
    $ineg$ & $HandleInegEPC \circseq pc := i+1$ \\
    $goto~ofst$ & $pc := i+ofst$ \\
    $if\_icmple(ofst)$ & $\circvar value1, value2 : Word \circspot$ \endgraf
                         \t1 $InterpreterPop2 \circseq$ \endgraf
                         \t1 $\IF value1 \leq value2 \THEN i+ofst \ELSE i+1$ \\
    $areturn$ & $HandleAreturnEPC$ \\
    $return$ & $HandleReturnEPC$ \\
    $getfield~cpi$ & $HandleGetfieldEPC(cpi) \circseq pc := i+1$ \\
    $putfield~cpi$ & $HandlePutfieldEPC(cpi) \circseq pc := i+1$ \\
    $getstatic~cpi$ & $HandleGetstaticEPC(cpi) \circseq pc := i+1$ \\
    $putstatic~cpi$ & $HandlePutstaticEPC(cpi) \circseq pc := i+1$ \\
    $invokevirtual~cpi$ & $pc := i \circseq HandleInvokevirtualEPC(cpi)$ \\
    $invokespecial~cpi$ & $pc := i \circseq HandleInvokespecialEPC(cpi)$ \\
    $invokestatic~cpi$ & $pc := i \circseq HandleInvokestaticEPC(cpi)$ \\
    \hline
  \end{tabular}
  \caption{The syntactic function $handleAction$}
  \label{handle-action-table}
\end{table}
Note that the $Handle^*$ actions are eliminated completely in the case
of the \texttt{goto} and \texttt{if\_icmple} instructions, since the
$pc$ update is the main effect of these actions.

This replacement of the $Handle*$ actions with these new actions
occurs in line~\ref{algorithm-expand-handle-action}. 
After this transformation has been applied, the $Running$ action for
our example is as shown in
Figure~\ref{bytecode-expansion-example-figure}.

The overall transformation of the $HandleInstruction$ action in this
step is summarised by \Autoref{bytecode-expansion-rule}, which
makes use of a syntactic function $handleAction$ that maps bytecode
instructions onto \Circus{} actions as shown in
Table~\ref{handle-action-table}.
\begin{restatable}[Bytecode Expansion]{crule}{BytecodeExpansionRule}
  \label{bytecode-expansion-rule}
  For a given $bc$
  \begin{circus}
    HandleInstruction_{bc} \circrefines_A
    \begin{array}{l}
      \circif {} \circelse_i pc = i \then handleAction(bc~i) \circfi
    \end{array}
  \end{circus}
  where $handleAction$ is a syntactic function defined by
  Table~\ref{handle-action-table}.
\end{restatable}
After the bytecode semantics is expanded in the $Running$ action, the
control flow that corresponds to each $pc$ update can be introduced.

\subsection{Introduce Sequential Composition}
\label{introduce-forward-sequence-subsection}

\begin{algorithm}
  \begin{algorithmic}[1]
    \Procedure{IntroduceSequentialComposition}{}
    \State $cfg \gets$ \Call{MakeControlFlowGraph}{}
    \label{algorithm-make-control-flow-graph}
    \For{$node \gets cfg$}
    \label{algorithm-sequence-cfg-loop}
    \While{\Call{HasSimpleSequence}{$node$}}
    \label{algorithm-forward-sequence-condition}
    \State \Call{Apply\Autoref{sequence-introduction-rule}}{$node$}
    \label{algorithm-forward-sequence-application}
    \EndWhile
    \EndFor
    \EndProcedure
  \end{algorithmic}
  \caption{Introduce Sequential Composition}
  \label{introduce-forward-sequence-algorithm}
\end{algorithm}
The simplest control flow to introduce is that of instructions where
execution continues at the next program counter value.
These control flows are introduced as shown in
Algorithm~\ref{introduce-forward-sequence-algorithm}.
The algorithm consists of constructing a control flow graph for each
method in the program, as specified on
line~\ref{algorithm-make-control-flow-graph}.
Since the introduction of sequential composition does not depend on
the relationships between methods, the control flow graph is
constructed as a disconnected control flow graph containing the
control flow of all the methods in the program.
Although method calls have not had their $pc$ updates formally made
explicit, we can assume a method call will be followed by the
instruction at the next program counter value.
The control flow graph for our example is shown in
Figure~\ref{example-control-flow-graph-figure}.

\begin{figure}
  \begin{center}
    \footnotesize
    \begin{tikzpicture}[every new ->/.style={-latex}]
      % \node (start) at (0,0) {};
      \foreach \x in {0,...,6}  { \node (\x) at (\x, 1cm) {\x}; }
      \foreach \x in {7,...,20}  { \node (\x) at (0.85*\x - 0.85*7,0) {\x}; }
      \foreach \x in {21,...,27} { \node (\x) at (0.85*\x - 0.85*21, -2cm) {\x}; }
      \foreach \x in {28,...,31} { \node (\x) at (0.85*\x - 0.85*21, -1cm) {\x}; }
      \foreach \x in {32,...,34} { \node (\x) at (0.85*\x - 0.85*25, -3cm) {\x}; }
      \foreach \x in {35,...,42} { \node (\x) at (0.85*\x - 0.85*24.3, -2cm) {\x}; }
      \foreach \x in {43,...,50} { \node (\x) at (-43+\x, -4.5cm) {\x}; }

      \graph{ (7) -> (8) -> (9) -> (10) -> (11) -> (12) ->
        (13) -> (14) -> (15) -> (16) -> (17) -> (18) -> (19) -> (20)
        -> (39) -> (40) -> (41) -> (42); (21) -> (22) -> (23) -> (24)
        -> (25) -> (26) -> (27) -> {
          (28) -> (29) -> (30) -> (31);
          (32) -> (33) -> (34);
        } -> (35) -> (36) -> (37) -> (38) -> (39);
      };

      \graph{(0) -> (1) -> (2) -> (3) -> (4) -> (5) -> (6)};
      \graph{(43) -> (44) -> (45) -> (46) -> (47) -> (48) -> (49) -> (50)};

      \draw[-latex] (41) edge[out=270,in=270,looseness=0.35] (21);
    \end{tikzpicture}
  \end{center}
  \caption{Control flow graph for our example program}
  \label{example-control-flow-graph-figure}
\end{figure}

After the control flow graph is constructed, we consider each node in
turn, as specified by the for loop on
line~\ref{algorithm-sequence-cfg-loop}.
As mentioned earlier, we require a node to have only a single outgoing
edge and its target to have only a single incoming edge in order to be
considered for the introduction of sequential composition.
The reason for this is that nodes with two outgoing edges are points
at which conditionals should be introduced, rather than sequential
compositions.
Such nodes in our example are the nodes for $pc$ values $27$ and $41$,
which represent the start of conditionals.
Likewise, nodes with multiple incoming edges represent points at which
a more complex control flows occur.
For our example, such nodes include $39$, which is the start of a
loop, and $35$, which is the end of a conditional.
These prevent introduction of sequential composition for the $pc$
values $20$, $31$, $34$, and $38$, since those the targets of those
nodes are nodes $35$ and $39$.

For a node that meets the above requirement and isn't a method call,
we can introduce sequential composition at that node by applying
\Autoref{sequence-introduction-rule}, on
line~\ref{algorithm-forward-sequence-application} of the algorithm.
\begin{restatable}[Sequence introduction]{crule}{SequenceIntroductionRule}
  \label{sequence-introduction-rule}
  \def\zedindent{0.25cm}
  If $i \neq j$ and
  \begin{circus}
    \{frameStack \neq \emptyset\} \circseq A \\
    {} = {} \\
    \{frameStack \neq \emptyset\} \circseq A \circseq \{frameStack \neq \emptyset\}
  \end{circus}
  then,
  \begin{circus}
    \begin{array}{l}
      \circmu X \circspot \\
      \t1 \circif frameStack = \emptyset \circthen \Skip \\
      \t1 {} \circelse frameStack \neq \emptyset \circthen {} \\
      \t2 \circif {} \cdots {} \\
      \t2 {} \circelse pc = i \circthen A \circseq pc := j \\
      \t2 {} \cdots {} \\
      \t2 {} \circelse pc = j \circthen B \\
      \t2 {} \cdots {} \\
      \t2 \circfi \circseq Poll \circseq X \\
      \t1 \circfi
    \end{array}
    \circrefines_A
    \begin{array}{l}
      \circmu X \circspot \\
      \t1 \circif frameStack = \emptyset \circthen \Skip \\
      \t1 {} \circelse frameStack \neq \emptyset \circthen {} \\
      \t2 \circif {} \cdots {} \\
      \t2 {} \circelse pc = i \circthen {} \\
      \t3 A \circseq pc := j \circseq Poll \circseq B \\
      \t2 {} \cdots {} \\
      \t2 {} \circelse pc = j \circthen B \\
      \t2 {} \cdots {} \\
      \t2 \circfi \circseq Poll \circseq X \\
      \t1 \circfi
    \end{array}
  \end{circus}
\end{restatable}
This rule works by unrolling the loop in $Running$ to sequence an
instruction with the instruction that is executed after it, inserting
$Poll$ inbetween.
It is required that the $pc$ value of the node's target, $j$, not be
the same as the $pc$ value of the node, $i$, since that would
introduce a loop, rather than a sequential composition.
Also, the sequence of instructions at the node, $A$, must not affect
the non-emptiness of the $frameStack$ to ensure that the choice at the
start of the main loop in $Running$ can be resolved.

Since \Autoref{sequence-introduction-rule} pulls two nodes
together, we can continue to introduce sequential composition at a
node after the first application of
\Autoref{sequence-introduction-rule}, until that node no longer
satisfies the conditions for introducing sequential composition.
This is specified by the while loop at
line~\ref{algorithm-forward-sequence-condition} of the algorithm.
This means the control flow graph is updated as
\Autoref{sequence-introduction-rule} is applied, to take into
account the merging of nodes.
The resulting control flow graph after introduction of sequential
composition has been performed at every point is shown in
Figure~\ref{example-control-flow-graph-after-sequence-introduction-figure}.
\begin{figure}
  \begin{center}
    \begin{tikzpicture}[every new ->/.style={-latex}]
      \path (0,0) node (0) {0} -- ++(1,0) node (6) {6};
      \path (0,-1cm) node (7) {7} -- ++(1,0) node (11) {11} -- ++(1,0) node (14) {14} -- ++(1,0) node (17) {17};
      \path (0,-3cm) node (21) {21} -- ++(1,0) node (23) {23} -- ++(1,0) node (24) {24};
      \path (3,-2.3cm) node (28) {28} -- ++(1,0) node (31) {31};
      \node at (3,-3.7cm) (32) {32};
      \path (5,-3cm) node (35) {35} -- ++(1,0) node (39) {39} -- ++(1,0) node (42) {42};
      
      \graph{
        (7) -> (11) -> (14) -> (17) -> (39) -> (42);
        (21) -> (23) -> (24) -> {
          (28) -> (31);
          (32);
        } -> (35) -> (39);
      };

      \graph[grow down]{(0) -> (6)};
      \node at (0,-5cm) {43};

      \draw[-latex] (39) edge[out=270,in=270,looseness=0.6] (21);
    \end{tikzpicture}
  \end{center}
  \caption{Control flow graph for our example after sequential composition introduction}
  \label{example-control-flow-graph-after-sequence-introduction-figure}
\end{figure}
The only remaining nodes in this graph are those where the sequence of
instructions ends with a method call or which represent a more complex
control flow.
In particular, the instructions for the \texttt{f()} method of
\texttt{TPK}, which begin at $pc = 43$, have been completely sequenced
together into a single node.
The code which corresponds to this control flow graph is that shown
earlier in Figure~\ref{forward-sequence-introduction-example-figure}

\subsection{Introduce Loops and Conditionals}
\label{introduce-loops-and-conditionals-subsection}

After sequential composition has been introduced for all methods, we
must begin considering each method separately to ensure method calls
are handled properly.
This means the strategy must loop, introducing loops and conditionals
to those methods that have no unresolved method calls and resolving
calls of methods that are then complete, until every method is
complete and has been separated into its own action.
Introducing loops and conditionals is performed as described by
Algorithm~\ref{introduce-loops-and-conditionals-algorithm}.
This considers each method individually, as specified by the for loop
on line~\ref{algorithm-introduce-loops-and-conditionals-method-loop}
of the algorithm. 
The condition on line~\ref{algorithm-no-unresolved-calls-condition}
ensures that only those methods where all method calls have already
been resolved undergo loop and conditional introduction.

\begin{algorithm}
  \begin{algorithmic}[1]
    \Procedure{IntroduceLoopsAndConditionals}{}
    \For{$m \gets methods$}
    \label{algorithm-introduce-loops-and-conditionals-method-loop}
    \If{\Call{HasNoUresolvedCalls}{$m$}}
    \label{algorithm-no-unresolved-calls-condition}
    \State $cfg \gets$ \Call{MakeControlFlowGraph}{$m$}
    \label{algorithm-make-control-flow-graph2}
    \For{$node \gets$ \Call{ReverseNodes}{$cfg$}}
    \label{algorithm-node-checking-loop}
    \State \Call{Apply\Autoref{if-introduction-rule}}{$node$}
    \label{algorithm-introduce-if}
    \State \Call{Apply\Autoref{if-else-introduction-rule}}{$node$}
    \label{algorithm-introduce-if-else}
    \If{\Call{IsSimpleConditional}{$node$}}
    \label{algorithm-conditional-check}
    \State \Call{Apply\Autoref{conditional-introduction-rule}}{$node$}
    \label{algorithm-introduce-conditional}
    \EndIf
    \State \Call{Apply\Autoref{while-introduction-rule1}}{$node$}
    \label{algorithm-introduce-while1}
    \State \Call{Apply\Autoref{while-introduction-rule2}}{$node$}
    \label{algorithm-introduce-while2}
    \State \Call{Apply\Autoref{do-while-introduction-rule}}{$node$}
    \label{algorithm-introduce-do-while}
    \State \Call{Apply\Autoref{infinite-loop-introduction-rule}}{$node$}
    \label{algorithm-introduce-infinite-loop}
    \If{\Call{HasSimpleSequence}{$node$}}
    \label{algorithm-lci-sequence-check}
    \State \Call{Apply\Autoref{sequence-introduction-rule}}{$node$}
    \label{algorithm-lci-sequence-introduction}
    \EndIf

    \EndFor
    \EndIf
    \EndFor
    \EndProcedure
  \end{algorithmic}
  \caption{Introduce Loops and Conditionals}
  \label{introduce-loops-and-conditionals-algorithm}
\end{algorithm}

For each method that undergoes loop and conditional introduction, we
must again consider the control flow graph of the method to ensure the
loops and conditionals are introduced in the correct order to properly
form the bodies of loops and conditionals.
This involves constructing a control flow graph for the method, at
line~\ref{algorithm-make-control-flow-graph2}, beginning at the entry
point of the method and following each \texttt{goto} and
\texttt{if\_icmple} instruction until a loop is detected or a
\texttt{return} or \texttt{areturn} instruction is reached.
The graph for the our example, beginning at $pc=7$ (the entry point of
the \texttt{handleAsyncEvent()} method), is shown in
Figure~\ref{example-simplified-control-flow-graph-figure}, alongside
the \Circus{} code obtained at the beginning of this stage for the
method.
The edge which forms a loop from $pc=35$ to $pc=39$ is shown as a
dashed line since looping edges are ignored at certain points in this
part of the strategy.
\begin{figure}
  \begin{center}
    \begin{multicols}{4}
      \begin{tikzpicture}
        \node (7)  at (0,0)  {7};
        \node (39)  at (0,-1) {39};
        \node (42) at (-1,-2) {42};
        \node (21) at (1,-2) {21};
        \node (28) at (0.5,-3) {28};
        \node (32) at (1.5,-3) {32};
        \node (35) at (1,-4) {35};
        \draw[-latex] (7) to (39);
        \draw[-latex] (39) to (42);
        \draw[-latex] (39) to (21);
        \draw[-latex] (21) to (28);
        \draw[-latex] (21) to (32);
        \draw[-latex] (28) to (35);
        \draw[-latex] (32) to (35);
        % \draw[-latex,red!70!black,dashed,out=0,in=0,looseness=1.1] (35) to (39);
        \draw[-latex,dashed,out=0,in=0,looseness=1.1] (35) to (39);
      \end{tikzpicture}
      \columnbreak
      \scriptsize
      \setlength{\zedindent}{0cm}
      \begin{circus}
        Running \circdef \\
        \t1 \circif frameStack = \emptyset \circthen \Skip \\
        \t1 {} \circelse frameStack \neq \emptyset \circthen {} \\
        \t2 \circif pc = 0 \circthen {} \cdots {} \\
        \t2 {} \circelse pc = 7 \circthen HandleNewEPC(27) \circseq pc := 8 \circseq Poll \circseq \cdots \circseq pc := 39 \\
        \t2 {} \cdots {} \\
        \t2 {} \circelse pc = 21 \circthen HandleAloadEPC(2) \circseq pc := 22 \circseq Poll \circseq \cdots \circseq \\
        \t3 pc := \IF value1 \leq value2 \THEN 32 \ELSE 28 \\
        \t2 {} \cdots {} \\
        \t2 {} \circelse pc = 28 \circthen HandleAloadEPC(3) \circseq pc := 29 \circseq Poll \circseq \cdots \circseq pc := 35 \\
        \t2 {} \cdots {} \\
        \t2 {} \circelse pc = 32 \circthen HandleAloadEPC(3) \circseq pc := 33 \circseq Poll \circseq \cdots \circseq pc := 35 \\
        \t2 {} \cdots {} \\
        \t2 {} \circelse pc = 35 \circthen HandleAloadEPC(4) \circseq pc := 36 \circseq Poll \circseq \cdots \circseq pc := 39 \\
        \t2 {} \cdots {} \\
        \t2 {} \circelse pc = 39 \circthen HandleAloadEPC(4) \circseq pc := 36 \circseq Poll \circseq \cdots \circseq \\
        \t3 pc := \IF value1 \leq value2 \THEN 21 \ELSE 42 \\
        \t2 {} \cdots {} \\
        \t2 {} \circelse pc = 42 \circthen HandleReturnEPC \\
        \t2 \circfi \circseq Poll \circseq Running \\
        \t1 \circfi
      \end{circus}
    \end{multicols}
  \end{center}
  \caption{Simplified control flow graph and corresponding code for our example
    program}
  \label{example-simplified-control-flow-graph-figure}
\end{figure}

A method's control flow graph must be well-structured in order to
properly introduce the control flow structures in this section.
We define a rooted directed graph below.
The definition is standard, but we include it here to introduce the
terminology for the subsequent definition of what we mean by a
structured control flow graph.
\begin{defn}[Rooted Directed Graph]
  A \emph{rooted directed graph}, $G$, is a triple $(V,E,r)$, where
  \begin{itemize}
  \item $V$ is a set of \emph{nodes},
  \item $E$ is a set of ordered pairs of nodes in $V$, called
    \emph{edges}, and
  \item $r$ is a node in $V$, called the \emph{root} of the graph.
  \end{itemize}
  The first component of an edge is its \emph{source} and the second
  component is its \emph{target}.
  We say that an edge goes from its source to its target.
  The source of an edge going to a given node is said to be a
  \emph{predecessor} of that node; similarly, the target of an edge
  from a given node is a \emph{successor} of that node.
  For every node $n \in V$, the pair $(r,n)$ must be in the reflexive
  transitive closure of $E$, that is, there must be a path of edges
  from the root to any node in the graph.
  
  In diagrams we represent the nodes as points or as the names of the
  nodes, the edges as arrows, and the root node as a node with an
  arrow pointing to it that does not come from another node.
  For a graph $G$, we refer to the set
  $T(G) = \{ n \in V | \forall m \in V.\; (n,m) \notin E\}$ of nodes
  with no edges coming from them as the set of \emph{end nodes} of the
  graph.
\end{defn}
Having defined a rooted directed graph, we now define what we mean by
a structured control flow graph as a specific type of rooted directed
graph.
This definition we use for a structured program is based on Dijkstra's
notion of program structure found in~\cite{dijkstra1972}.
In that definition, there are a set of known program structures that
are permitted, and these structures may contain further occurrences of
the same structures (e.g.\ a conditional in which each branch is also
a conditional).
In order to formalise this, we first define what it means to replace a
node with a graph.
\begin{defn}[Node Replacement]
  Given two rooted directed graphs $G$ and $H$, we say $G'$ is the
  graph formed by \emph{replacing} a node $n$ of $G$ with $H$ if one
  of the following cases holds:
  \begin{itemize}
  \item $n$ has no predecessors in $G$, $H$ has only one end node, and
    \begin{itemize}
    \item $G'$ contains all the nodes of $H$ and $G$, except $n$,
    \item $G'$ contains the edges of $G$ except those going to or from
      $n$,
    \item $G'$ contains edges from the end node of $H$ to the
      successors of $n$ in $G$, and
    \item the root node of $H$ is the root node of $G'$;
    \end{itemize}
  \item $n$ has no successors in $G$, and
    \begin{itemize}
    \item $G'$ contains all the nodes of $H$ and $G$, except $n$,
    \item $G'$ contains the edges of $G$ except those going to or from
      $n$,
    \item $G'$ contains edges from the predecessors $n$ in $G$ to the
      root node of $H$, and
    \item the root node of $G$ is the root node of $G'$;
    \end{itemize}
  \item $H$ has a single end node and
    \begin{itemize}
    \item $G'$ contains all the nodes of $H$ and $G$, except $n$,
    \item $G'$ contains the edges of $G$ and the edges of $H$ except
      those going to or from $n$,
    \item $G'$ contains edges from the predecessors of $n$ in $G$ to
      the root node of $H$,
    \item $G'$ contains edges from the end node of $H$ to the
      successors of $n$ in $G$, and
    \item the root node of $G$ is the root node of $G'$;
    \end{itemize}
  \item $n$ has a single successor in $G$, $H$ has a single end
    node, and
    \begin{itemize}
    \item $G'$ contains all the nodes of $H$ and $G$, except $n$ and
      the end node of $H$,
    \item $G'$ contains the edges of $G$ except those going to or from
      $n$,
    \item $G'$ contains edges from the predecessors of the end node of
      $H$ to the successor of $n$ in $G$
    \item $G'$ contains edges from the predecessors of $n$ in $G$ to
      the root node of $H$, and
    \item the root node of $G$ is the root node of $G'$.
    \end{itemize}
  \end{itemize}
\end{defn}
Each of the different cases of node replacement represents a different
way of placing a graph inside another graph.
We show an example of each of these cases in
Figure~\ref{node-replacement-example-figures}.
The example used is that of an \texttt{if}-\texttt{else} conditional,
introduced later in Figure~\ref{if-else-figure}, with one of its nodes
replaced with another \texttt{if}-\texttt{else} conditional whose
nodes are shown in white.

The first case (Figure~\ref{root-replacement-figure}) is that of
placing a graph at the start of another graph, i.e.\ replacing the
root node of a graph that does not have a loop to its root node.
The second case (Figure~\ref{end-replacement-figure}) is that of
replacing one of the end nodes of a graph.
The third case (Figure~\ref{internal-replacement-figure}) is that of
replacing an internal node of the graph. 
There must be a single end node in this case in order to have a source
for the outgoing edges of the replaced node.
At the end of one of the branches of a conditional, the end node of
the replacing graph may be unified with the node at the end of the
conditional.
This represents such cases as loops and conditionals occurring at the
end of a branch of a conditional, with no instructions following them
inside the conditional, and is handled by the fourth case of node
replacement, shown in Figure~\ref{branch-end-replacement-figure}.
\begin{figure}
  \begin{subfigure}{0.24\textwidth}
    \begin{center}
      \begin{tikzpicture}
        \node at (0,2) (start) {};
        \node at ( 0, 1)  (A) {$\circ$};
        \node at (-1, 0)  (B) {$\circ$};
        \node at ( 1, 0)  (C) {$\circ$};
        \node at ( 0,-1)  (D) {$\circ$};
        \node at (-1,-2)  (E) {$\bullet$};
        \node at ( 1,-2)  (F) {$\bullet$};
        \node at ( 0,-3)  (G) {$\bullet$};
        \draw[-latex] (start) -- (A);
        \draw[-latex] (A) -- (B);
        \draw[-latex] (A) -- (C);
        \draw[-latex] (B) -- (D);
        \draw[-latex] (C) -- (D);
        \draw[-latex] (D) -- (E);
        \draw[-latex] (D) -- (F);
        \draw[-latex] (E) -- (G);
        \draw[-latex] (F) -- (G);
      \end{tikzpicture}
    \caption{\centering root node\newline replacement}
    \label{root-replacement-figure}
  \end{center}
  \end{subfigure}
  \begin{subfigure}{0.24\textwidth}
    \begin{center}
      \begin{tikzpicture}
        \node at (0,2) (start) {};
        \node at ( 0, 1)  (A) {$\bullet$};
        \node at (-1, 0)  (B) {$\bullet$};
        \node at ( 1, 0)  (C) {$\bullet$};
        \node at ( 0,-1)  (D) {$\circ$};
        \node at (-1,-2)  (E) {$\circ$};
        \node at ( 1,-2)  (F) {$\circ$};
        \node at ( 0,-3)  (G) {$\circ$};
        \draw[-latex] (start) -- (A);
        \draw[-latex] (A) -- (B);
        \draw[-latex] (A) -- (C);
        \draw[-latex] (B) -- (D);
        \draw[-latex] (C) -- (D);
        \draw[-latex] (D) -- (E);
        \draw[-latex] (D) -- (F);
        \draw[-latex] (E) -- (G);
        \draw[-latex] (F) -- (G);
      \end{tikzpicture}
    \caption{\centering end node\newline replacement}
    \label{end-replacement-figure}
    \end{center}
  \end{subfigure}
  \begin{subfigure}{0.24\textwidth}
    \begin{center}
      \begin{tikzpicture}
        \node at (0,3) (start) {};
        \node at ( 0, 2)  (A) {$\bullet$};
        \node at (-1, 1)  (B) {$\circ$};
        \node at ( 1, 0)  (C) {$\bullet$};
        \node at (-2, 0)  (D) {$\circ$};
        \node at ( 0, 0)  (E) {$\circ$};
        \node at (-1,-1)  (F) {$\circ$};
        \node at ( 0,-2)  (G) {$\bullet$};
        \draw[-latex] (start) -- (A);
        \draw[-latex] (A) -- (B);
        \draw[-latex] (A) -- (C);
        \draw[-latex] (B) -- (D);
        \draw[-latex] (B) -- (E);
        \draw[-latex] (D) -- (F);
        \draw[-latex] (E) -- (F);
        \draw[-latex] (F) -- (G);
        \draw[-latex] (C) -- (G);
      \end{tikzpicture}
      \caption{\centering internal node\newline replacement}
      \label{internal-replacement-figure}
    \end{center}
  \end{subfigure}
  \begin{subfigure}{0.24\textwidth}
    \begin{center}
      \begin{tikzpicture}
        \node at (0,3) (start) {};
        \node at ( 0, 2)  (A) {$\bullet$};
        \node at (-1, 1)  (B) {$\circ$};
        \node at ( 1, 0)  (C) {$\bullet$};
        \node at (-2, 0)  (D) {$\circ$};
        \node at ( 0, 0)  (E) {$\circ$};
        \node at ( 0,-2)  (F) {$\bullet$};
        \draw[-latex] (start) -- (A);
        \draw[-latex] (A) -- (B);
        \draw[-latex] (A) -- (C);
        \draw[-latex] (B) -- (D);
        \draw[-latex] (B) -- (E);
        \draw[-latex] (D) -- (F);
        \draw[-latex] (E) -- (F);
        \draw[-latex] (C) -- (F);
      \end{tikzpicture}
    \caption{\centering branch end\newline replacement}
    \label{branch-end-replacement-figure}
  \end{center}
  \end{subfigure}
  \caption{Examples of the different cases of node replacement}
  \label{node-replacement-example-figures}
\end{figure}

With node replacement defined, we can now define what we mean by a
structure control flow graph in terms of node replacement and the structured graphs shown in Figure~\ref{structured-cfg-figures}

\begin{defn}[Structured Control Flow Graph]
  If $G$ is a rooted directed graph, we say $G$ is a \emph{structured
    control flow graph} if $G$ is the trivial graph (the graph with a
  single node, which is also the root, and no edges) or if $G$ can be
  created by starting with the trivial graph and performing a finite
  number of node replacements to replace nodes with graphs of the
  forms shown in Figure~\ref{structured-cfg-figures}.
\end{defn}

% In particular, we target the requirements imposed by
% MISRA-C~\cite{misra2012}.
% This means that we do not allow \texttt{goto}s in the final C code and
% that loops must have a simple structure, with no use of
% \texttt{continue} and a single exit point.
% A single return per method is also a requirement of MISRA-C, but in a
% control flow graph there is no way to distinguish between a
% conditional return in the middle of a method and a conditional at the
% end of a method of which both branches are returns.
% Thus we treat all occurrences of returns in the middle of functions in
% the second way, which can then be treated as a single return at the
% end of the function in the translation to C code by extracting the
% returns from each branch.

\begin{figure}
  \begin{subfigure}{0.26\textwidth}
    \begin{center}
      \begin{tikzpicture}
        \useasboundingbox (-0.5,-1) rectangle (0.5,2);
        \node at (0,1.7) (start) {};
        \node at (0,1)  (A) {$\bullet$};
        \node at (0,-1) (B) {$\bullet$};
        \draw[-latex] (start) -- (A);
        \draw[-latex] (A) -- (B);
      \end{tikzpicture}
    \end{center}
    \caption{sequential composition}
    \label{sequence-figure}
  \end{subfigure}
  \begin{subfigure}{0.23\textwidth}
    \begin{center}
      \begin{tikzpicture}
        \useasboundingbox (-1,-1) rectangle (0,2);
        \node at (0,1.7) (start) {};
        \node at (0,1)  (A) {$\bullet$};
        \node at (-1,0) (B) {$\bullet$};
        \node at (0,-1) (C) {$\bullet$};
        \draw[-latex] (start) -- (A);
        \draw[-latex] (A) -- (B);
        \draw[-latex] (A) -- (C);
        \draw[-latex] (B) -- (C);
      \end{tikzpicture}
    \end{center}
    \caption{\texttt{if} conditional}
    \label{if-figure}
  \end{subfigure}
  \begin{subfigure}{0.23\textwidth}
    \begin{center}
      \begin{tikzpicture}
        \useasboundingbox (-1,-1) rectangle (1,2);
        \node at (0,1.7) (start) {};
        \node at (0,1)  (A) {$\bullet$};
        \node at (1,0)  (B) {$\bullet$};
        \node at (-1,0) (C) {$\bullet$};
        \node at (0,-1) (D) {$\bullet$};
        \draw[-latex] (start) -- (A);
        \draw[-latex] (A) -- (B);
        \draw[-latex] (A) -- (C);
        \draw[-latex] (B) -- (D);
        \draw[-latex] (C) -- (D);
      \end{tikzpicture}
    \end{center}
    \caption{\texttt{if}-\texttt{else} conditional}
    \label{if-else-figure}
  \end{subfigure}
  \begin{subfigure}{0.23\textwidth}
    \begin{center}
      \begin{tikzpicture}
        \useasboundingbox (-1,0) rectangle (1,3);
        \node at (0,1.7) (start) {};
        \node at (0,1)  (A) {$\bullet$};
        \node at (1,0)  (B) {$\bullet$};
        \node at (-1,0) (C) {$\bullet$};
        \draw[-latex] (start) -- (A);
        \draw[-latex] (A) -- (B);
        \draw[-latex] (A) -- (C);
      \end{tikzpicture}
    \end{center}
    \caption{divergent conditional}
    \label{divergent-figure}
  \end{subfigure} 
  \\
  \begin{subfigure}{0.33\textwidth}
    \begin{center}
      \begin{tikzpicture}
        \useasboundingbox (-1,-1) rectangle (1,2);
        \node at (0,1.7) (start) {};
        \node at (0,1) (A) {$\bullet$};
        \node at (0,-1) (B) {$\bullet$};
        \node at (1,-1) (C) {$\bullet$};
        \draw[-latex] (start) -- (A);
        \draw[-latex] (A) to (B);
        \draw[-latex] (A) to (C);
        \draw[-latex] (B) to[in=200,out=150] (A);
      \end{tikzpicture}
    \end{center}
    \caption{\texttt{while} loop}
    \label{while-figure}
  \end{subfigure}
  \begin{subfigure}{0.33\textwidth}
    \begin{center}
      \begin{tikzpicture}
        \useasboundingbox (-1,0) rectangle (1,3);
        \node at (0,1.7) (start) {};
        \node at (0,1)  (A) {$\bullet$};
        \node at (1,0)  (B) {$\bullet$};
        \draw[-latex] (start) -- (A);
        \draw[-latex] (A) to (B);
        \draw[-latex] (A) to[out=235,in=180,looseness=10] (A);
      \end{tikzpicture}
    \end{center}
    \caption{\texttt{do}-\texttt{while} loop}
    \label{do-while-figure}
  \end{subfigure}
  \begin{subfigure}{0.33\textwidth}
    \begin{center}
      \begin{tikzpicture}
        \useasboundingbox (-1,0) rectangle (1,3);
        \node at (0,1.7) (start) {};
        \node at (0,1)  (A) {$\bullet$};
        \draw[-latex] (start) -- (A);
        \draw[-latex] (A) to[out=270,in=180,looseness=10] (A);
      \end{tikzpicture}
    \end{center}
    \caption{infinite loop}
    \label{infinite-loop-figure}
  \end{subfigure}
  \caption{Control flow graphs of program structures}
  \label{structured-cfg-figures}
\end{figure}

The first structure (Figure~\ref{sequence-figure}) is that of simple
sequential composition, with an edge going from the root node to a
single end node.
The next three structures
(Figure~\ref{if-figure}--\subref{divergent-figure}) are conditional
structures:~Figure~\ref{if-figure} shows an \texttt{if} statement with
no \texttt{else} clause, Figure~\ref{if-else-figure} shows an
\texttt{if} statement with an \texttt{else} clause, and
Figure~\ref{divergent-figure} shows a conditional in which both
branches end with a (infinite) loop or a return so that there is
nothing following the conditional, we refer to such conditionals as
divergent conditionals since the branches do not come back together.
The remaining three structures
(Figure~\ref{while-figure}--\subref{infinite-loop-figure}) are all
loop structures:~Figure~\ref{while-figure} shows a loop in which the
loop condition is checked at the beginning (a \texttt{while} loop),
Figure~\ref{do-while-figure} show a loop in which the loop condition
is checked at the end (a \texttt{do}-\texttt{while} loop), and
Figure~\ref{infinite-loop-figure} shows a loop which loops
unconditionally, forming an infinite loop.

From this definition it is clear that the control flow graph of our
example, shown in
Figure~\ref{example-simplified-control-flow-graph-figure}, is a
structured control flow graph.
It may be obtained from the trivial graph by replacing the node with a
sequential composition, replacing the end node of the sequential
composition with a \texttt{while} loop, and then replacing the node
inside the \texttt{while} loop with an \texttt{if}-\texttt{else}
conditional.

In the strategy we check that the control flow graph is structured
when we construct it in this section, and abort the strategy if it
does not have the required structure.
The introduction of sequential composition does not cause the control
flow graph of a structured program to cease being structured, so we
may also check that the control flow graph is structured when we
construct it in
Section~\ref{introduce-forward-sequence-subsection}.
%TODO: mention the Dijkstra Graphs paper

%TODO: explain how our definition of structure differs from that of
% MISRA - leave for final considerations?

Since we have defined the desired program structure in terms of a
small number of standard structures, we can identify each of these
structures in the control flow graph and introduce them into the
program, collapsing the control flow graph in the process.
In order to easily identify the structures in isolation from other
structures, we begin at the end nodes of the method (ignoring looping
edges for the purposes of determining end nodes) and work backwards,
considering each node in turn.
This is specified by the loop beginning on
line~\ref{algorithm-node-checking-loop} of
Algorithm~\ref{introduce-loops-and-conditionals-algorithm}.
In our example this means we consider the $pc=42$ and $pc=35$ nodes
first, then $pc=28$ and $pc=32$, then $pc=21$, $pc=39$, and finally
$pc=7$.

For each node, we check each type of structure to see if the control
flow graph starting at that point matches the structure, and introduce
the structure if it does.
The first type of structure we check for are conditionals, beginning
with those conditionals that are followed by another node, that is,
those shown in Figure~\ref{if-figure} and~\subref{if-else-figure}.
These may be nested within one of the branches of another conditional
in one of the two ways shown below:
\begin{center}
  \begin{tikzpicture}
    \node (1) at (0, 0.0) {$\bullet$};
    \node (2) at (2, 0.5) {$\bullet$};
    \node (3) at (2,-0.5) {$\bullet$};
    \node (4) at (4, 1.0) {$\bullet$};
    \node (5) at (4,-1.0) {$\bullet$};
    \node (6) at (4, 0.2) {$\bullet$};
    \node (7) at (4,-0.2) {$\bullet$};
    \node (8) at (6, 0.0) {$\bullet$};
    
    \draw[-latex] (-1,0) to (1);
    
    \draw[-latex] (1) to (2);
    \draw[-latex] (1) to (3);
    \draw[-latex] (2) to (4);
    \draw[-latex] (3) to (5);
    \draw[-latex] (2) to (6);
    \draw[-latex] (3) to (7);
    \draw[-latex] (4) to (8);
    \draw[-latex] (5) to (8);
    \draw[-latex] (6) to (8);
    \draw[-latex] (7) to (8);
  \end{tikzpicture}
  \hfill
  \begin{tikzpicture}
    \node (1) at (0.0, 0.0) {$\bullet$};
    \node (2) at (1.5, 0.5) {$\bullet$};
    \node (3) at (1.5,-0.5) {$\bullet$};
    \node (4) at (3.0, 1.0) {$\bullet$};
    \node (5) at (3.0,-1.0) {$\bullet$};
    \node (6) at (3.0, 0.2) {$\bullet$};
    \node (7) at (3.0,-0.2) {$\bullet$};
    \node (8) at (4.0, 0.5) {$\bullet$};
    \node (9) at (4.0,-0.5) {$\bullet$};
    \node (0) at (5.0, 0.0) {$\bullet$};
    
    \draw[-latex] (-1,0) to (1);
    
    \draw[-latex] (1) to (2);
    \draw[-latex] (1) to (3);
    \draw[-latex] (2) to (4);
    \draw[-latex] (3) to (5);
    \draw[-latex] (2) to (6);
    \draw[-latex] (3) to (7);
    \draw[-latex] (4) to (8);
    \draw[-latex] (5) to (9);
    \draw[-latex] (6) to (8);
    \draw[-latex] (7) to (9);
    \draw[-latex] (8) to (0);
    \draw[-latex] (9) to (0);
  \end{tikzpicture}
\end{center}
In the first case the sequential composition with the node at the end
cannot be introduced until the outermost conditional is introduced,
because both of the inner conditionals end at the same point.
However, in the second case the inner conditionals can be completely
introduced, including the sequential composition with the node after
the end of each inner conditional, before the outer conditional is
introduced.
To ensure both cases are covered, we separate the introduction of the
conditional itself and the sequential composition with the node after
the conditional. 
In the first case the introduction of the sequential composition is
deferred until after the outermost conditional is resolved, whereas in
the second case it may be performed immediately after the introduction
of the conditional.

We provide separate compilation rules for introducing \texttt{if}
conditionals and \texttt{if}-\texttt{else} conditionals.
An \texttt{if} conditional with no else branch may be recognised from
the control flow graph as having the form shown in
Figure~\ref{if-figure}.
However, it can also be recognised from the form of the \Circus{} code
in the $Running$ action, which will be that of a node whose sequence
of instructions ends with an assignment of the form
$pc := \IF b \THEN x \ELSE y$, and for which the $pc = y$ node ends in
an assignment $pc := x$.
Note that the branches will not be the other way round (i.e. 
the $pc = x$ branch will not be the body of the conditional) since the
conditional branches come from Java's branching instructions which
branch to the specified address if the condition is true and go to the
next instruction if it is false.
We provide \Autoref{if-introduction-rule} for introducing such
conditionals.
\begin{restatable}[\texttt{if} conditional introduction]{crule}{IfConditionalIntroductionRule}
  \label{if-introduction-rule}
  \setlength{\zedindent}{0.25cm}
  % \setlength{\abovedisplayskip}{0.1cm}
  % \setlength{\belowdisplayskip}{0.1cm}
  If $i \neq j$, $i \neq k$, and 
  \begin{circus}
    \{frameStack \neq \emptyset\} \circseq A \\
    {} = {} \\
    \{frameStack \neq \emptyset\} \circseq A \circseq \{frameStack \neq \emptyset\}
  \end{circus}
  then
  \begin{circus}
    \begin{array}{l}
      \circmu X \circspot \\
      \t1 \circif frameStack = \emptyset \circthen \Skip \\
      \t1 {} \circelse frameStack \neq \emptyset \circthen {} \\
      \t2 \circif \cdots \\
      \t2 {} \circelse pc = i \circthen A \circseq \\
      \t3 pc := \IF b \THEN j \ELSE k \\
      \t2 {} \cdots {} \\
      \t2 {} \circelse pc = k \circthen B \circseq pc := j \\
      \t2 {} \cdots {} \\
      \t2 \circfi \circseq Poll \circseq X \\
      \t1 \circfi
    \end{array}
    \circrefines_A
    \begin{array}{l}
      \circmu X \circspot \\
      \t1 \circif frameStack = \emptyset \circthen \Skip \\
      \t1 {} \circelse frameStack \neq \emptyset \circthen {} \\
      \t2 \circif \cdots \\
      \t2 {} \circelse pc = i \circthen A \circseq Poll \circseq \\
      \t3 pc := \IF b \THEN j \ELSE k \circseq \\
      \t3 \circif b \circthen \Skip \\
      \t3 {} \circelse \lnot b \circthen B \\
      \t3 \circfi \circseq pc := j \\
      \t2 {} \cdots {} \\
      \t2 {} \circelse pc = k \circthen B \circseq pc := j \\
      \t2 {} \cdots {} \\
      \t2 \circfi \circseq Poll \circseq X \\
      \t1 \circfi 
    \end{array}
  \end{circus}
\end{restatable}
\Autoref{if-introduction-rule} introduces a conditional for nodes
that match the form described above, which in the rule is the
$pc = i$ node.
The conditional is introduced with the true branch being empty
(represented here by $\Skip$) and the false branch containing the
instructions in the body of the conditional.
The assignment $pc := j$ is moved outside the conditional from both
the empty true branch and the end of the false branch, so that a
sequential composition with the node after the conditional can be
introduced later on.
As in \Autoref{sequence-introduction-rule}, the sequence of actions
for the node must not affect the nonemptiness of the $frameStack$.
A similar condition is required for all the rules in this section.
We also require that the targets of the conditional are different from
the node at which the conditional is introduced, since that would
introduce a loop, which is not the purpose of this rule.
\Autoref{if-introduction-rule} is applied on
line~\ref{algorithm-introduce-if} of
Algorithm~\ref{introduce-loops-and-conditionals-algorithm}.
Note that, since the structure can be identified from the form of the
\Circus{} code alone, it is node necessary to guard the application of
the rule with a condition on the control flow graph.

After attempting to introduce an \texttt{if} conditional, we attempt
to introduce an \texttt{if}-\texttt{else} conditional, the form of
which is shown in Figure~\ref{if-else-figure}.
As with an \texttt{if} conditional, a node with an
\texttt{if}-\texttt{else} conditional will end with an assignment of
the form $pc := \IF b \THEN x \ELSE y$, but the $pc = x$ and $pc = y$
nodes are required to end with a common assignment $pc := z$.
Conditionals matching this form may be introduced using
\Autoref{if-else-introduction-rule}.
\begin{restatable}[\texttt{if}-\texttt{else} conditional introduction]{crule}{IfElseConditionalIntroductionRule}
  \label{if-else-introduction-rule}
  \setlength{\zedindent}{0.25cm}
  % \setlength{\abovedisplayskip}{0.1cm}
  % \setlength{\belowdisplayskip}{0.1cm}
  If $i \neq j$, $i \neq k$, and 
  \begin{circus}
    \{frameStack \neq \emptyset\} \circseq A \\
    {} = {} \\
    \{frameStack \neq \emptyset\} \circseq A \circseq \{frameStack \neq \emptyset\}
  \end{circus}
  then
  \begin{circus}
    \begin{array}{l}
      \circmu X \circspot \\
      \t1 \circif frameStack = \emptyset \circthen \Skip \\
      \t1 {} \circelse frameStack \neq \emptyset \circthen {} \\
      \t2 \circif \cdots \\
      \t2 {} \circelse pc = i \circthen A \circseq \\
      \t3 pc := \IF b \THEN j \ELSE k \\
      \t2 {} \cdots {} \\
      \t2 {} \circelse pc = j \circthen B \circseq pc := x \\
      \t2 {} \cdots {} \\
      \t2 {} \circelse pc = k \circthen C \circseq pc := x \\
      \t2 {} \cdots {} \\
      \t2 \circfi \circseq Poll \circseq X \\
      \t1 \circfi
    \end{array}
    \circrefines_A
    \begin{array}{l}
      \circmu X \circspot \\
      \t1 \circif frameStack = \emptyset \circthen \Skip \\
      \t1 {} \circelse frameStack \neq \emptyset \circthen {} \\
      \t2 \circif \cdots \\
      \t2 {} \circelse pc = i \circthen A \circseq Poll \circseq \\
      \t3 pc := \IF b \THEN j \ELSE k \circseq \\
      \t3 \circif b \circthen B \\
      \t3 {} \circelse \lnot b \circthen C \\
      \t3 \circfi \circseq pc := x \\
      \t2 {} \cdots {} \\
      \t2 {} \circelse pc = j \circthen B \circseq pc := x \\
      \t2 {} \cdots {} \\
      \t2 {} \circelse pc = k \circthen C \circseq pc := x \\
      \t2 {} \cdots {} \\
      \t2 \circfi \circseq Poll \circseq X \\
      \t1 \circfi 
    \end{array}
  \end{circus}
\end{restatable}
\Autoref{if-else-introduction-rule} operates similarly to
\Autoref{if-introduction-rule} in how it introduces the conditional
and moves the common $pc$ assignment outside the conditional.
However, \Autoref{if-else-introduction-rule} includes sequences of
instructions for both branches of the introduced conditional, each of
which end with a $pc$ assignment to jump to the node after the
conditional.
The preconditions of \Autoref{if-else-introduction-rule} are the
same as those of \Autoref{if-introduction-rule}.
\Autoref{if-else-introduction-rule} is applied on
line~\ref{algorithm-introduce-if-else} of
Algorithm~\ref{introduce-loops-and-conditionals-algorithm}.

Having attempted to introduce conditionals with a node following them,
we then consider conditionals that are not followed by a node.
These conditionals are those where both branches end in a return or an
infinite loop.
This includes conditionals where both branches are a return, which can
arise from multiple returns in the Java source code. 
Though multiple returns are not allowed in the final C code, the
returns will all end up in branches of conditionals at the end of the
method, so the actual return statement can be placed after the
conditionals to create a single return statement at the end of the C
function.

The form of this type of conditionals is that shown in
Figure~\ref{divergent-figure}.
We require that the nodes in both branches of the conditional have
only a single incoming edge each, and do not have any outgoing edges
(at the point in the strategy where this type of conditional is
introduced, that is, there may be more complex structures that have
already been introduced in the branches).
We check that the control flow graph beginning at the node being
considered has this form on line~\ref{algorithm-conditional-check} of
the algorithm.
This is to ensure unstructured conditionals such as the one shown
below are ruled out by the strategy.
\begin{center}
  \begin{tikzpicture}
    \node (0) at (-2, 0.0) {$\cdots$};
    \node (1) at (0, 0.0) {$\bullet$};
    \node (2) at (2, 0.5) {$\bullet$};
    \node (3) at (2,-0.5) {$\bullet$};
    \node (4) at (4, 1.0) {$\bullet$};
    \node (5) at (4,-1.0) {$\bullet$};
    \node (6) at (4, 0.0) {$\bullet$};

    \draw[-latex] (0) to (1);
    \draw[-latex] (1) to (2);
    \draw[-latex] (1) to (3);
    \draw[-latex] (2) to (4);
    \draw[-latex] (3) to (5);
    \draw[-latex] (2) to (6);
    \draw[-latex] (3) to (6);
  \end{tikzpicture}
\end{center}

If the correct structure is present, then we introduce the conditional
by applying \Autoref{conditional-introduction-rule} on
line~\ref{algorithm-introduce-conditional}.
This rule introduces the conditional in much the same way as
\Autoref{if-introduction-rule} and
\Autoref{if-else-introduction-rule}, but it does not place any
requirement on the structure of the conditional or move a $pc$
assignment outside of it.
\begin{restatable}[Conditional introduction]{crule}{ConditionalIntroductionRule}
  \label{conditional-introduction-rule}
  \setlength{\zedindent}{0.25cm}
  % \setlength{\abovedisplayskip}{0.1cm}
  % \setlength{\belowdisplayskip}{0.1cm}
  If $i \neq j$, $i \neq k$, and 
  \begin{circus}
    \{frameStack \neq \emptyset\} \circseq A \\
    {} = {} \\
    \{frameStack \neq \emptyset\} \circseq A \circseq \{frameStack \neq \emptyset\}
  \end{circus}
  then
  \begin{circus}
    \begin{array}{l}
      \circmu X \circspot \\
      \t1 \circif frameStack = \emptyset \circthen \Skip \\
      \t1 {} \circelse frameStack \neq \emptyset \circthen {} \\
      \t2 \circif \cdots \\
      \t2 {} \circelse pc = i \circthen A \circseq \\
      \t3 pc := \IF b \THEN j \ELSE k \\
      \t2 {} \cdots {} \\
      \t2 {} \circelse pc = j \circthen B \\
      \t2 {} \cdots {} \\
      \t2 {} \circelse pc = k \circthen C \\
      \t2 {} \cdots {} \\
      \t2 \circfi \circseq Poll \circseq X \\
      \t1 \circfi
    \end{array}
    \circrefines_A
    \begin{array}{l}
      \circmu X \circspot \\
      \t1 \circif frameStack = \emptyset \circthen \Skip \\
      \t1 {} \circelse frameStack \neq \emptyset \circthen {} \\
      \t2 \circif \cdots \\
      \t2 {} \circelse pc = i \circthen A \circseq \\
      \t3 pc := \IF b \THEN j \ELSE k \circseq Poll \circseq \\
      \t3 \circif b \circthen B \\
      \t3 {} \circelse \lnot b \circthen C \\
      \t3 \circfi \\
      \t2 {} \cdots {} \\
      \t2 {} \circelse pc = j \circthen B \\
      \t2 {} \cdots {} \\
      \t2 {} \circelse pc = k \circthen C \\
      \t2 {} \cdots {} \\
      \t2 \circfi \circseq Poll \circseq X \\
      \t1 \circfi 
    \end{array}
  \end{circus}
\end{restatable}

After attempting to introduce conditionals, we may attempt to
introduce loops.
There are three types of loop to consider, as shown earlier:
\texttt{while} loops (Figure~\ref{while-figure}),
\texttt{do}-\texttt{while} loops (Figure~\ref{do-while-figure}), and
infinite loops (Figure~\ref{infinite-loop-figure}).
A \texttt{while} loop has a form similar to that of a conditional,
except that one of the branches ends with a jump back to the beginning
of the node with the conditional.
This structure may be introduced using
\Autoref{while-introduction-rule1}.
This rule introduces a conditional at a node $pc=i$ with its false
branch ending in an assignment of $i$ to $pc$, and introduces a
recursion to the beginning of the $pc=i$ node in that branch of the
conditional, representing a loop.
Since this loop may be within a conditional, we simply move the $pc$
assignment for the true branch outside the conditional.
A sequential composition can then be introduced later, as with
\texttt{if} and \texttt{if}-\texttt{else} conditionals.
\begin{restatable}[\texttt{while} loop introduction 1]{crule}{WhileLoopIntroductionRuleA}
  \label{while-introduction-rule1}
  \def\zedindent{0.25cm}
  If $i \neq j$,
  \begin{circus}
    \{frameStack \neq \emptyset\} \circseq A \\
    {} = {} \\
    \{frameStack \neq \emptyset\} \circseq A \circseq \{frameStack \neq \emptyset\}
  \end{circus}
  then
  \begin{circus}
    \begin{array}{l}
      \circmu X \circspot \\
      \t1 \circif frameStack = \emptyset \circthen \Skip \\
      \t1 {} \circelse frameStack \neq \emptyset \circthen {} \\
      \t2 \circif \cdots \\
      \t2 {} \circelse pc = i \circthen A \circseq \\
      \t3 pc := \IF b \THEN j \ELSE k \\
      \t2 \cdots \\
      \t2 {} \circelse pc = j \circthen B \\
      \t2 \cdots \\
      \t2 {} \circelse pc = k \circthen C \circseq pc := i \\
      \t2 \cdots \\
      \t2 \circfi \circseq Poll \circseq X \\
      \t1 \circfi 
    \end{array}
    \circrefines_A
    \begin{array}{l}
      \circmu X \circspot \\
      \t1 \circif frameStack = \emptyset \circthen \Skip \\
      \t1 {} \circelse frameStack \neq \emptyset \circthen {} \\
      \t2 \circif \cdots \\
      \t2 {} \circelse pc = i \circthen \circmu Y \circspot A \circseq \\
      \t3 pc := \IF b \THEN j \ELSE k \circseq Poll \circseq \\
      \t3 \circif b \circthen \Skip \\
      \t3 {} \circelse \lnot b \circthen C \circseq pc := i \circseq Poll \circseq Y \\
      \t3 \circfi \circseq pc := j \\
      \t2 \cdots \\
      \t2 {} \circelse pc = j \circthen B \\
      \t2 \cdots \\
      \t2 {} \circelse pc = k \circthen C \circseq pc := i \\
      \t2 \cdots \\
      \t2 \circfi \circseq Poll \circseq X \\
      \t1 \circfi 
    \end{array}
  \end{circus}
\end{restatable}%
As a \texttt{while} loop may occur with the loop at the end of either
condition branch (since the loop may be created by a \texttt{goto}
instruction in the Java bytecode), we also provide a similar rule,
\Autoref{while-introduction-rule2}, that introduces the loop in the
true branch of the conditional.
These two rules are applied on lines~\ref{algorithm-introduce-while1}
and~\ref{algorithm-introduce-while2} of the algorithm.

The next type of loop we consider is the \texttt{do}-\texttt{while}
loop, which has the form shown in Figure~\ref{do-while-figure}.
These loops are distinguished from \texttt{while} loops by the fact
that the conditional $pc$ assignment which causes the loop is at the
end of the loop, rather than at the beginning or in the middle.
We introduce these loops using \Autoref{do-while-introduction-rule}.
\begin{restatable}[\texttt{do}-\texttt{while} loop introduction]{crule}{DoWhileLoopIntroductionRule}
  \label{do-while-introduction-rule}
  \def\zedindent{0.25cm}
  If $i \neq j$,
  \begin{circus}
    \{frameStack \neq \emptyset\} \circseq A \\
    {} = {} \\
    \{frameStack \neq \emptyset\} \circseq A \circseq \{frameStack \neq \emptyset\}
  \end{circus}
  then
  \begin{circus}
    \begin{array}{l}
      \circmu X \circspot \\
      \t1 \circif frameStack = \emptyset \circthen \Skip \\
      \t1 {} \circelse frameStack \neq \emptyset \circthen {} \\
      \t2 \circif \cdots \\
      \t2 {} \circelse pc = i \circthen A \circseq \\
      \t3 pc := \IF b \THEN i \ELSE j \\
      \t2 \cdots \\
      \t2 {} \circelse pc = j \circthen B \\
      \t2 \cdots \\
      \t2 \circfi \circseq Poll \circseq X \\
      \t1 \circfi 
    \end{array}
    \circrefines_A
    \begin{array}{l}
      \circmu X \circspot \\
      \t1 \circif frameStack = \emptyset \circthen \Skip \\
      \t1 {} \circelse frameStack \neq \emptyset \circthen {} \\
      \t2 \circif \cdots \\
      \t2 {} \circelse pc = i \circthen \circmu Y \circspot A \\
      \t3 pc := \IF b \THEN i \ELSE j \circseq Poll \circseq \\
      \t3 \circif b \circthen Y \\
      \t3 {} \circelse \lnot b \circthen \Skip \\
      \t3 \circfi \circseq pc := j \\
      \t2 \cdots \\
      \t2 \circfi \circseq Poll \circseq X \\
      \t1 \circfi 
    \end{array}
  \end{circus}
\end{restatable}%
This rule introduces a conditional, as with
\Autoref{while-introduction-rule1}, but the true branch contains just
the recursive call, since the conditional occurs at the end of the
loop.
The $pc$ assignment for the false branch is moved outside the
conditional to allow a sequential composition to be introduced later,
as in previous rules.
Note that the false branch can never cause the loop in this case,
since it will just go to the next instruction.
Attempting to redirect it and create the loop with a \texttt{goto}
instruction would add an instruction within the loop after the
conditional, so it would be dealt with as a \texttt{while} loop.
Therefore, it is not necessary to provide two compilation rules for
\texttt{do}-\texttt{while} loops, unlike \texttt{while} loops where
both cases must be accounted for.
\Autoref{do-while-introduction-rule} is applied on
line~\ref{algorithm-introduce-do-while} of the algorithm.

The final loop structure that we attempt to introduce is that of an
infinite loop.
Infinite loops are rare in most programs, but an infinite loop is
nonetheless a well-structured program construct that has use in a few
cases so we handle it here.
An infinite loop may be identified as a block of instructions that
ends with a $pc$ assignment that causes a jump back to the beginning
of the block of instructions.
Such a block will have a control flow graph of the form shown in
Figure~\ref{infinite-loop-figure}.
We introduce these loops using
\Autoref{infinite-loop-introduction-rule}, which introduces a
recursive call after the $pc$ assignment that causes the loop.
This rule is applied on line~\ref{algorithm-introduce-infinite-loop}
of the algorithm.
\begin{restatable}[Infinite loop introduction]{crule}{InfiniteLoopIntroductionRule}
  \label{infinite-loop-introduction-rule}
  If
  \begin{circus}
    \{frameStack \neq \emptyset\} \circseq A \\
    {} = {} \\
    \{frameStack \neq \emptyset\} \circseq A \circseq \{frameStack \neq \emptyset\}
  \end{circus}
  then
  \def\zedindent{0.25cm}
  \begin{circus}
    \begin{array}{l}
      \circmu X \circspot \\
      \t1 \circif frameStack = \emptyset \circthen \Skip \\
      \t1 {} \circelse frameStack \neq \emptyset \circthen {} \\
      \t2 \circif \cdots \\
      \t2 {} \circelse pc = i \circthen {} \\
      \t3 A \circseq pc := i \\
      \t2 {} \cdots {} \\
      \t2 \circfi \circseq Poll \circseq X \\
      \t1 \circfi
    \end{array}
    \circrefines_A
    \begin{array}{l}
      \circmu X \circspot \\
      \t1 \circif frameStack = \emptyset \circthen \Skip \\
      \t1 {} \circelse frameStack \neq \emptyset \circthen {} \\
      \t2 \circif \cdots \\
      \t2 {} \circelse pc = i \circthen {} \\
      \t3 \circmu Y \circspot A \circseq pc := i \circseq Poll \circseq Y \\
      \t2 {} \cdots {} \\
      \t2 \circfi \circseq Poll \circseq X \\
      \t1 \circfi
    \end{array}
  \end{circus}
\end{restatable}%

After we have attempted to introduce each of the structures for a
particular node, we attempt to introduce a sequential composition.
This ensures that \texttt{if}, \texttt{if}-\texttt{else},
\texttt{while} and \texttt{do}-\texttt{while} structures that occur
within conditionals are sequentially composed with the node following
them if possible.
It also handles cases where sequential compositions occur before
loops, preventing them from being introduced in
Section~\ref{introduce-forward-sequence-subsection} without
interfering with the introduction of the loop.
Such a case occurs at the $pc=7$ node in our example.
The requirement for sequential composition to be introduced is the
same as in Section~\ref{introduce-forward-sequence-subsection}:~it
must be a simple sequential composition from a node with a single
outgoing edge to a node with a single incoming edge.
Thus we check for a simple sequence on
line~\ref{algorithm-lci-sequence-check} of
Algorithm~\ref{introduce-loops-and-conditionals-algorithm}.
The sequential composition is then introduced on
line~\ref{algorithm-lci-sequence-introduction} if it is a simple
sequential composition.

As mentioned earlier, these steps are repeated for each node, working
backwards through the control flow graph of each method.
Given a structured control flow graph at the beginning, this means all
the structures in the method are introduced, reducing the control flow
graph to a single node.

In our example, we begin at the $pc=35$ node, where there are no
structures to introduce. 
The same holds true of the $pc=28$ and $pc=32$ nodes (note that the
edges coming from them are not simple sequential compositions).
An \texttt{if}-\texttt{else} conditional is introduced at $pc=21$,
absorbing the $pc=28$ and $pc=32$ nodes.
The sequential composition from the $pc=21$ node to the $pc=35$ node
can then be introduced immediately as it is now a simple sequential
composition (because it is not at the end of an outer conditional).
We then introduce a \texttt{while} loop at the $pc=39$ loop (using
\Autoref{while-introduction-rule2}), and the sequential composition
with the $pc=42$ node is introduced afterwards.
Finally, a sequential composition from the $pc=7$ to the $pc=39$ node
is introduced, collapsing the control flow graph to a single node.
The code at $pc=7$ is then that shown earlier in
Figure~\ref{loop-and-conditional-introduction-example-figure}.

\subsection{Resolve Method Calls}
\label{resolve-method-calls-subsection}

When a method is complete, calls to that method can then be resolved.
This is performed after introduction of loops and conditionals,
ensuring methods with loops and conditionals are complete so that this
step can be applied.
% As mentioned previously, since this requires all the method calls in a
% given method to be resolved first, we do not allow recursion.

This step begins with the copying of the method into a separate
action, so that it can be referenced elsewhere.
This is performed by as described by
Algorithm~\ref{separate-complete-methods-algorithm}.
\begin{algorithm}
  \begin{algorithmic}[1]
    \Procedure{SeparateCompleteMethods}{}
    \For{$m \gets methods$} \label{algorithm-method-separation-loop}
    % \For{$mc \gets$ \Call{MethodsCalls}{$m$}}
    % \EndFor
    \If{\Call{MethodIsComplete}{$m$}} \label{algorithm-check-method-completeness}
    \State \Call{ApplyCopyRule}{$m$}
    \EndIf
    \EndFor
    \EndProcedure
  \end{algorithmic}
  \caption{Separate Complete Methods}
  \label{separate-complete-methods-algorithm}
\end{algorithm}

Algorithm~\ref{separate-complete-methods-algorithm} looks at each
method separately, as specified by the loop on
line~\ref{algorithm-method-separation-loop}, and determines if it is
complete, on line~\ref{algorithm-check-method-completeness}.
This involves a simple syntactic check that each conditional branch
ends in a return instruction or a recursion.
Those methods that are complete are moved into a separate action by an
application of the copy rule.

In our example, the method \texttt{f()} of the \texttt{TPK} class,
which starts at $pc = 43$, is complete on the first iteration of the
loop on line~\ref{algorithm-method-loop} of
Algorithm~\ref{epc-algorithm}, with the $Running$ action as shown
below.
The method is complete in this case because it consists of a straight
sequence of instructions ending with $HandleAreturnEPC$, which
represents the \texttt{areturn} instruction.
\begin{circus}
  Running \circdef \\
  \t1 \circif frameStack = \emptyset \circthen \Skip \\
  \t1 {} \circelse frameStack \neq \emptyset \circthen {} \\
  \t2 \circif pc = 0 \circthen {} \cdots {} \\
  \t2 {} \cdots {} \\
  \t2 {} \circelse pc = 43 \circthen HandleAloadEPC(0) \circseq Poll \circseq HandleAloadEPC(0) \circseq \\
  \t3 Poll \circseq HandleIaddEPC \circseq Poll \circseq HandleAloadEPC(0) \circseq Poll \circseq \\
  \t3 HandleIaddEPC \circseq Poll \circseq HandleIconstEPC(5) \circseq Poll \circseq \\
  \t3 HandleIaddEPC \circseq Poll \circseq HandleAreturnEPC \\
  \t2 {} \cdots {} \\
  \t2 \circfi \circseq Poll \circseq Running \\
  \t1 \circfi
\end{circus}
The sequence of instructions at $pc = 43$ can then be copied into a
separate action, shown below.
The name of this action contains the name of the class and method
identifier of the method it represents.
\begin{circus}
  TPK\_f \circdef HandleAloadEPC(0) \circseq Poll \circseq HandleAloadEPC(0) \circseq Poll \circseq \\
  \t1 HandleIaddEPC \circseq Poll \circseq HandleAloadEPC(0) \circseq Poll \circseq HandleIaddEPC \circseq \\
  \t1 Poll \circseq HandleIconstEPC(5) \circseq Poll \circseq  HandleIaddEPC \circseq Poll \circseq \\
  \t1 HandleAreturnEPC 
\end{circus}

After all the complete methods have been copied into separate actions,
calls to those methods are resolved.
This is performed as described by
Algorithm~\ref{resolve-method-calls-algorithm}.
\begin{algorithm}
  \begin{algorithmic}[1]
    \Procedure{ResolveMethodCalls}{}
    \For{$m \gets methods$}
    \For{$mc \gets$ \Call{UnresolvedMethodsCalls}{$m$}}
    \State $targets \gets$ \Call{DetermineMethodCallTargets}{$mc$}
    \If{$\# targets = 1$}
    \State \Call{Apply\Autoref{method-call-resolution-rule}}{}
    \Else
    \State \Call{Apply\Autoref{dynamic-method-call-resolution-rule}}{}
    \EndIf
    \EndFor
    \EndFor
    \EndProcedure
  \end{algorithmic}
  \caption{Resolve Method Calls}
  \label{resolve-method-calls-algorithm}
\end{algorithm}


%%%%

To resolve a method call, the type of method call must be considered.
Some methods are handled directly by the SCJVM as they relate to the
SCJVM services.
Such methods are treated specially by the interpreter, communicating
with the launcher to perform the behaviour of the method.
In cases where the method invocation is simply handled by
communication with the launcher and then followed by execution of the
next instruction, the control flow can be introduced using
\Autoref{sequence-introduction-rule} as for other instructions with
simple control flow.

% TODO: discuss special methods with nested calls here

For method calls that do not require special handling, the control
flow is that of the corresponding method's action, followed by
execution of the next instruction.
If the method is called with static dispatch (as is the case with the
\texttt{invokespecial} and \texttt{invokestatic} instructions), the
correct method, and hence the corresponding action can be easily
determined.
The method call is then resolved using
\Autoref{method-call-resolution-rule}.
We require as a precondition of the rule that the method action
returns to the return address stored on the stack, to ensure that the
control flow is resolved correctly.
This will be the case for a method where every path of execution ends
in a return or loop, since returns establish the condition and loops
will either lead to a return eventually or form an infinite loop that
may be followed by any action (including the assumption we require).
%
\begin{restatable}[Method call resolution]{crule}{MethodCallResolutionRule}
  \label{method-call-resolution-rule}
  If an action $M$ is such that
  \[\{ (head~frameStack).storedPC = i \land frameStack = fs \} \circseq M \\
    {} = {} \\
    \{ (head~frameStack).storedPC = i \land frameStack = fs \}
    \circseq M \circseq \\
    \t1 \{ pc = i \land frameStack = tail~fs \}\]
  and $i \neq j$,
  \setlength{\zedindent}{0.25cm}
  \begin{circus}
    \begin{array}{l}
      \circmu X \circspot \\
      \t1 \circif frameStack = \emptyset \circthen \Skip \\
      \t1 {} \circelse frameStack \neq \emptyset \circthen {} \\
      \t2 \circif \cdots \\
      \t2 {} \circelse pc = i \circthen A \circseq \\
      \t3 \{ pc = k \} \circseq \\
      \t3 \lschexpract \exists retAddr? == pc+1 @ \\
      \t4 SetReturnAddr \rschexpract \circseq \\
      \t3 pc := j \\
      \t2 {} \circelse pc = k+1 \circthen B \\
      \t2 {} \circelse pc = j \circthen M \\
      \t2 \cdots \\
      \t2 \circfi \circseq Poll \circseq X \\
      \t1 \circfi 
    \end{array}
    \circrefines_A
    \begin{array}{l}
      \circmu X \circspot \\
      \t1 \circif frameStack = \emptyset \circthen \Skip \\
      \t1 {} \circelse frameStack \neq \emptyset \circthen {} \\
      \t2 \circif \cdots \\
      \t2 {} \circelse pc = i \circthen A \circseq \\
      \t3 \{ pc = k \} \circseq \\
      \t3 \lschexpract \exists retAddr? == pc+1 @ \\
      \t4 SetReturnAddr \rschexpract \circseq \\
      \t3 pc := j \circseq Poll \circseq M \circseq Poll \circseq B \\
      \t2 {} \circelse pc = k+1 \circthen B \\
      \t2 {} \circelse pc = j \circthen M \\
      \t2 \cdots \\
      \t2 \circfi \circseq Poll \circseq X \\
      \t1 \circfi 
    \end{array}
  \end{circus}
  %TODO: elimination of pc assignment
\end{restatable}
%
For method calls that have dynamic dispatch (as in the case of
\texttt{invokevirtual} instructions), we must determine what classes
the method may be called on.
The control flow is then a choice over the class of the object the
method is called on, with the method action corresponding to that
class chosen.
As with the static case, we require the method action to return to the
return address stored on the stack.
In the dynamic case, this requirement must be met by all the method
actions so that the next instruction can be executed after the choice
of methods.
\Autoref{dynamic-method-call-resolution-rule} is used to introduce
this choice.
%
\begin{restatable}[Dynamic method call resolution]{crule}{DynamicMethodCallResolutionRule}
  \label{dynamic-method-call-resolution-rule}
  If actions $M_1, \dots, M_n$ are such that
  \[\{ returnAddress = i \land frameStack = fs \} \circseq M_k \\
    {} = {} \\
    \{ returnAddress = i \land frameStack = fs \}
    \circseq M_k \circseq \{ pc = i \land frameStack = tail~fs \}\]
   for $k in \{ 1, \dots, n \}$ and $i \neq j$,
  \setlength{\zedindent}{0.25cm}
  \begin{circus}
    \begin{array}{l}
      \circmu X \circspot \\
      \t1 \circif frameStack = \emptyset \circthen \Skip \\
      \t1 {} \circelse frameStack \neq \emptyset \circthen {} \\
      \t2 \circif \cdots \\
      \t2 {} \circelse pc = i \circthen A(class, method) \circseq \\
      \t3 \{ class \in \{ c_1, \dots, c_n \}\} \circseq \\
      \t3 \lschexpract \exists retAddr? == pc+1 @ \\
      \t4 SetReturnAddr \rschexpract \circseq \\
      \t3 pc := entry(class, method) \\
      \t2 {} \circelse pc = k+1 \circthen B \\
      \t2 {} \circelse pc = j_1 \circthen M_1 \\
      \t2 \cdots \\
      \t2 {} \circelse pc = j_n \circthen M_n \\
      \t2 \circfi \circseq Poll \circseq X \\
      \t1 \circfi 
    \end{array}
    \circrefines_A
    \begin{array}{l}
      \circmu X \circspot \\
      \t1 \circif frameStack = \emptyset \circthen \Skip \\
      \t1 {} \circelse frameStack \neq \emptyset \circthen {} \\
      \t2 \circif \cdots \\
      \t2 {} \circelse pc = i \circthen A(class, method) \circseq \\
      \t3 \{ class \in \{ c_1, \dots, c_n \} \circseq \\
      \t3 \lschexpract \exists retAddr? == pc+1 @ \\
      \t4 SetReturnAddr \rschexpract \circseq \\
      \t3 pc := entry(class, method) \circseq Poll \circseq \\
      \t3 \circif class = c_1 \circthen M_1 \\
      \t3 \cdots \\
      \t3 {} \circelse class = c_n \circthen M_n \\
      \t3 \circfi \circseq Poll \circseq B \\
      \t2 {} \circelse pc = k+1 \circthen B \\
      \t2 {} \circelse pc = j_1 \circthen M_1 \\
      \t2 \cdots \\
      \t2 {} \circelse pc = j_n \circthen M_n \\
      \t2 \circfi \circseq Poll \circseq X \\
      \t1 \circfi
    \end{array}
  \end{circus}
  %TODO: elimination of pc assignment
\end{restatable}

The resolution of loops, conditionals and methods is performed in a
loop until all the methods have been separated into their own action.
The remaining use of the program counter in the main actions of $Thr$
can then be eliminated as described in the next section.

\subsection{Refine Main Actions}
\label{refine-main-actions-subsection}

At this stage of the strategy, the only place that the program counter
is used is when the first method is started, when it is used to select
the method action to execute, which will then proceed without any need
for the program counter value.
This can be eliminated by replacing it with a choice over the method
rather than the program counter.
This must be performed in the two places that the $Running$ action
occurs:~the $MainThread$ and $NotStarted$ actions.
Since the main action of $Thr$ is a guarded choice of these actions
depending on whether its thread parameter is the $main$ thread, these
actions may be thought of as two alternative main actions for $Thr$.

The context of the $Running$ action in both of these main actions is
the same:~the frame stack has only one frame and the program counter
is set to the entry point of a method.
This means that the same rule can be used for both main actions by
introducing an assumption that states that context.
Additionally, we can use the fact that each method action will, when
started with a frame stack containing a single frame, cause the frame
stack to become empty.
This allows us to eliminate the loop in $Running$, reducing it
entirely to a choice of method action from the class and method
identifier.
The overall transformation of $Running$ in its context is described by
\Autoref{main-action-refinement-rule}.
\begin{restatable}[Main Action Refinement]{crule}{MainActionRefinementRule}
  \label{main-action-refinement-rule}
  If $entry(c_i,m_i) = j_i$ for $i \in \{1, \dots, n\}$ and
  \begin{circus}
    \{ \# frameStack = 1 \} \circseq M_i \\
    {} = {} \\
    \{ \# frameStack = 1 \} \circseq M_i \circseq \{ frameStack = \emptyset \}
  \end{circus}
  \setlength{\zedindent}{0.25cm}
  \begin{circus}
    \begin{array}{l}
      \{ \# frameStack = 1 \\
      \t1 {} \land pc = entry(cid, mid) \} \circseq \\
      \circmu X \circspot \\
      \t1 \circif frameStack = \emptyset \circthen \Skip \\
      \t1 {} \circelse framestack \neq \emptyset \circthen {}  \\
      \t2 \circif pc = j_1 \circthen M_1 \\
      \t2 \cdots \\
      \t2 {} \circelse pc = j_n \circthen M_n \\
      \t2 \circfi \circseq Poll \circseq X \\
      \t1 \circfi
    \end{array}
    \circrefines_A
    \begin{array}{l}
      \{ \# frameStack = 1 \} \circseq \\
      \circif (cid, mid) = (c_1, m_1) \circthen M_1 \\
      \cdots \\
      {} \circelse (cid, mid) = (c_n, m_n) \circthen M_n \\
      \circfi \\
    \end{array}
  \end{circus}
\end{restatable}
Once this has been performed, the program counter value is no longer
used to determine the control flow of the program, so a trivial data
refinement can be performed to eliminate $pc$ from the state of the
$Thr$ process.

\subsection{Remove $pc$ From State}
\label{remove-pc-from-state-subsection}

\section{Elimination of Frame Stack}
\label{elimination-of-frame-stack-section}

\section{Data Refinement of Objects}
\label{data-refinement-of-objects-section}

