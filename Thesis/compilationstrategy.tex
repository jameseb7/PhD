\chapter{Compilation Strategy}
\label{strategy-chapter}

In this chapter we describe our compilation strategy for refining SCJ
bytecode to C code.
We begin in Section~\ref{compilation-overview-section} with an
overview of our compilation strategy.
Then, in Section~\ref{compilation-assumptions-section} we describe the
requirements on the source program for the compilation strategy to be
applied.
Afterwards, we describe each stage of the strategy in a separate
section.
The first stage, which we call \emph{Elimination of Program Counter},
is described in Section~\ref{elimination-of-program-counter-section}.
The second stage, called \emph{Elimination of Frame Stack}, is
described in Section~\ref{elimination-of-frame-stack-section}.
Finally, the third stage of the strategy, which is called \emph{Data
  Refinement of Objects}, is described in
Section~\ref{data-refinement-of-objects-section}.
We then show how the stages fit together to show the compilation as a
whole to be correct\deleted{,} in
Section~\ref{main-theorem-proof-section}, and conclude with some final
considerations in
Section~\ref{compilation-final-considerations-section}.

\section{Overview}
\label{compilation-overview-section}

Our compilation strategy refines the $CEE(bc,cs,instCS,sid,initOrder)$
process defined in Section~\ref{cee-interpreter-section} to obtain the
$CCEE_{bc,cs}(sid, initOrder)$ process in
Section~\ref{cee-c-code-section}.
The overall theorem for the strategy, and, therefore, the main result
presented in this chapter, is as follows.
\begin{thm}[Compilation Strategy]\label{main-theorem}
  Given $bc$, $cs$ and $sid$, there are processes $StructMan_{cs}$ and
  $CProg_{bc,cs}$ such that,
  \setlength{\zedindent}{0.5cm}
  \begin{circus}
    CEE(bc,cs,instCS,sid,initOrder) \circrefines StructMan_{cs} \parallel
    CProg_{bc,cs} \parallel Launcher(sid, initOrder).
  \end{circus}
\end{thm}
$StructMan_{cs}$ manages objects represented by C structs that
incorporate the class information from $cs$, refining the process
$ObjMan$, which handles abstract objects.
$CProg_{bc,cs}$ refines the $Interpreter$, with the $Thr$ processes
refined into the $CThr_{bc,cs}$ processes described in
Section~\ref{cee-c-program-subsection}.
This means that the threads from SCJ are mapped onto threads in C,
since we do not dictate a particular thread switch mechanism in either
the source or target models.

The compilation strategy is split into three stages.
Each stage has a theorem describing it, for which the strategy acts as
a proof.
The proof of Theorem~\ref{main-theorem}, presented in
Section~\ref{main-theorem-proof-section}, is obtained by an
application of the theorems for each stage.
Each stage of the compilation strategy handles a different part of the
$Interpreter$ state:~the $pc$, the $frameStack$, and objects.
They operate over each of the $Thr$ processes, managed by the SCJVM
services.

The first stage, \emph{Elimination of Program Counter}, introduces the
control constructs of the C code.
This removes the use of $pc$ to determine the control flow of the
program.
The choice over $pc$ values is replaced with a choice over method
identifiers pointing to sequences of operations representing method
bodies.

In the second stage, \emph{Elimination of Frame Stack}, the
information contained on the $frameStack$, which is the local variable
array and operand stack for each method, is introduced in the C code.
This is done by introducing variables and parameters to represent each
method's local variables and operand stack slots.
A data refinement is then used to transform each operation over the
$frameStack$ to operate on the new variables.
The $frameStack$ is then eliminated from the state.

In the final stage, \emph{Data Refinement of Objects}, the class
information from $cs$ is used to create a representation of C structs.
This means that $ObjMan$, which has a very abstract representation of
objects, is transformed into $StructMan$.
The operations on objects are then changed to access the structs for
the objects in a more concrete way that represents the way struct
fields are accessed in C code.

% This yields final method actions of a form similar to that of the
% example shown below, which is taken from the \texttt{InputHandler}
% presented in Section~\ref{model-section}.
% \begin{circusaction}
%   InputHandler\_HandleAsyncEvent \circdef \\
%   \t1 \circval var0 \circspot \circvar var1, stack0, stack1 : Word \circspot \\
%   \t1 stack0 := var0 \circseq Poll \circseq getObject!stack0 \then getObjectRet?struct \\
%   \t1 {} \then stack0 := (castInputHandler~struct).input \circseq \dots
% \end{circusaction}
% The \texttt{handleAsyncEvent()} method of \texttt{InputHandler} is
% compiled to the action $InputHandler\_HandleAsyncEvent$, with the
% implicit \texttt{this} parameter represented as a value parameter
% $var0$.
% The local variable ($var1$) and stack slots ($stack0$ and $stack1$)
% are represented as \Circus{} variables.
% The operations of the C code are composed in sequence, with an action
% named $Poll$ that polls for thread switches present at the points
% where thread switches may occur. 
% Stack operations are represented as assignments. 
% For instance, $stack0 := var0$ arises from the compilation to load a
% local variable into a stack slot.
% Access to objects is performed by communicating with $StructMan_{cs}$
% to obtain the struct for the object, then casting it to the correct
% type, and accessing the required value.
% Above, we obtain the value of the $input$ field from an $InputHandler$
% object.
% The communication with $StructMan_{cs}$ is performed via the
% $getObject$ channel and the function $castInputHandler$ is used to map
% the $ObjectStruct$ returned from the communication to a type
% representing an object of \texttt{InputHandler}.

\section{Assumptions about source bytecode}
\label{compilation-assumptions-section}

For our strategy to be successfully applied to bytecodes corresponding
to an SCJ program, it must meet some basic requirements that ensure it
is well-formed.
Firstly, the program must pass JVM bytecode verification.
This means it must be type-correct and that execution remains inside
the array of bytecode instructions for each method.
This can be checked before execution of the program and there has
already been much work on formal verification of bytecode
verifiers~\cite{coglio2000,klein2003,xavier2003}.

Secondly, since SCJ does not allow dynamic class loading, all required
classes and methods must be present before execution of the program.
This means that the $cs$ map provided as input to the CEE must contain
all the classes referenced by any other class in $cs$.
All the bytecode instructions required for these classes must also be
present in the $bc$ map.
Our CEE model diverges if any of these requirements is not met, so
these requirements hold for any SCJ program that executes correctly in
our SCJVM interpreter.

Thirdly, due to the nature of the applications that SCJ is aimed at,
it is important that they have a structure that is readable and
facilitates verification.
MISRA-C includes such a restriction on structure and, since we are
generating C code for a safety-critical application, we aim to produce
code that is compatible with MISRA-C.
This means that the SCJ bytecode program used as input to the strategy
must also have a control structure compatible with the requirements of
MISRA-C.

\begin{figure}
  \begin{subfigure}{0.26\textwidth}
    \begin{center}
      \begin{tikzpicture}
        \useasboundingbox (-0.5,-1) rectangle (0.5,2);
        \node at (0,1.7) (start) {};
        \node at (0,1)  (A) {$\bullet$};
        \node at (0,-1) (B) {$\bullet$};
        \draw[-latex] (start) -- (A);
        \draw[-latex] (A) -- (B);
      \end{tikzpicture}
    \end{center}
    \caption{sequential composition}
    \label{sequence-figure}
  \end{subfigure}
  \begin{subfigure}{0.22\textwidth}
    \begin{center}
      \begin{tikzpicture}
        \useasboundingbox (-1,-1) rectangle (0,2);
        \node at (0,1.7) (start) {};
        \node at (0,1)  (A) {$\bullet$};
        \node at (-1,0) (B) {$\bullet$};
        \node at (0,-1) (C) {$\bullet$};
        \draw[-latex] (start) -- (A);
        \draw[-latex] (A) -- (B);
        \draw[-latex] (A) -- (C);
        \draw[-latex] (B) -- (C);
      \end{tikzpicture}
    \end{center}
    \caption{\texttt{if} conditional}
    \label{if-figure}
  \end{subfigure}
  \begin{subfigure}{0.23\textwidth}
    \begin{center}
      \begin{tikzpicture}
        \useasboundingbox (-1,-1) rectangle (1,2);
        \node at (0,1.7) (start) {};
        \node at (0,1)  (A) {$\bullet$};
        \node at (1,0)  (B) {$\bullet$};
        \node at (-1,0) (C) {$\bullet$};
        \node at (0,-1) (D) {$\bullet$};
        \draw[-latex] (start) -- (A);
        \draw[-latex] (A) -- (B);
        \draw[-latex] (A) -- (C);
        \draw[-latex] (B) -- (D);
        \draw[-latex] (C) -- (D);
      \end{tikzpicture}
    \end{center}
    \caption{\texttt{if}-\texttt{else} conditional}
    \label{if-else-figure}
  \end{subfigure}
  \begin{subfigure}{0.25\textwidth}
    \begin{center}
      \begin{tikzpicture}
        \useasboundingbox (-1,0) rectangle (1,3);
        \node at (0,1.7) (start) {};
        \node at (0,1)  (A) {$\bullet$};
        \node at (1,0)  (B) {$\bullet$};
        \node at (-1,0) (C) {$\bullet$};
        \draw[-latex] (start) -- (A);
        \draw[-latex] (A) -- (B);
        \draw[-latex] (A) -- (C);
      \end{tikzpicture}
    \end{center}
    \caption{divergent conditional}
    \label{divergent-figure}
  \end{subfigure} 
  \\
  \begin{subfigure}{0.32\textwidth}
    \begin{center}
      \begin{tikzpicture}
        \useasboundingbox (-1,-1) rectangle (1,2);
        \node at (0,1.7) (start) {};
        \node at (0,1) (A) {$\bullet$};
        \node at (0,-1) (B) {$\bullet$};
        \node at (1,-1) (C) {$\bullet$};
        \draw[-latex] (start) -- (A);
        \draw[-latex] (A) to (B);
        \draw[-latex] (A) to (C);
        \draw[-latex] (B) to[in=200,out=150] (A);
      \end{tikzpicture}
    \end{center}
    \caption{\texttt{while} loop}
    \label{while-figure}
  \end{subfigure}
  \begin{subfigure}{0.32\textwidth}
    \begin{center}
      \begin{tikzpicture}
        \useasboundingbox (-1,0) rectangle (1,3);
        \node at (0,1.7) (start) {};
        \node at (0,1)  (A) {$\bullet$};
        \node at (1,0)  (B) {$\bullet$};
        \draw[-latex] (start) -- (A);
        \draw[-latex] (A) to (B);
        \draw[-latex] (A) to[out=235,in=180,looseness=10] (A);
      \end{tikzpicture}
    \end{center}
    \caption{\texttt{do}-\texttt{while} loop}
    \label{do-while-figure}
  \end{subfigure}
  \begin{subfigure}{0.32\textwidth}
    \begin{center}
      \begin{tikzpicture}
        \useasboundingbox (-1,0) rectangle (1,3);
        \node at (0,1.7) (start) {};
        \node at (0,1)  (A) {$\bullet$};
        \draw[-latex] (start) -- (A);
        \draw[-latex] (A) to[out=270,in=180,looseness=10] (A);
      \end{tikzpicture}
    \end{center}
    \caption{infinite loop}
    \label{infinite-loop-figure}
  \end{subfigure}
  \caption{Control flow graphs of program structures}
  \label{structured-cfg-figures}
\end{figure}

Precisely, we require the control flow graph of each method in the
input program to have a structure based on Dijkstra's notion of
program structure found in~\cite{dijkstra1972}.
In our definition of a structured program, the control flow graph must
be composed of the structures shown in
Figure~\ref{structured-cfg-figures}. 
The first structure (Figure~\ref{sequence-figure}) is that of simple
sequential composition, with an edge going from the root node to a
single end node.
The next three structures
(Figure~\ref{if-figure}--\subref{divergent-figure}) are conditional
structures. 
Figure~\ref{if-figure} shows an \texttt{if} statement with no
\texttt{else} clause. 
Figure~\ref{if-else-figure} shows an \texttt{if} statement with an
\texttt{else} clause. 
Figure~\ref{divergent-figure} shows a conditional in which both
branches end with a (infinite) loop or a return so that there is
nothing following the conditional; we refer to such conditionals as
divergent conditionals since the branches do not come back together.
The remaining three structures
(Figure~\ref{while-figure}--\subref{infinite-loop-figure}) are all
loops.
Figure~\ref{while-figure} shows a loop in which the loop condition is
checked at the beginning (a \texttt{while} loop).
Figure~\ref{do-while-figure} shows a loop in which the loop condition
is checked at the end (a \texttt{do}-\texttt{while} loop).
Figure~\ref{infinite-loop-figure} shows an infinite loop.

We provide below a formal definition of what it means for a control
flow graph to be structured. 
This definition is based on that in~\cite{bento2017}, which provides
an algorithm for recognising structured graphs.
We first define a rooted directed graph below. 
The definition is standard, but we include it here to introduce the
terminology for the subsequent definition.
\begin{defn}[Rooted Directed Graph] A \emph{rooted directed graph},
$G$, is a triple $(V,E,r)$, where
  \begin{itemize}
  \item $V$ is a set of \emph{nodes},
  \item $E$ is a set of ordered pairs of nodes in $V$, called
\emph{edges}, and
  \item $r$ is a node in $V$, called the \emph{root} of the graph.
  \end{itemize}
  The first component of an edge is its \emph{source}
  and the second component is its \emph{target}. 
  We say that an edge goes from its source to its target. 
  For every node $n \in V$, the pair $(r,n)$ must be in the reflexive
  transitive closure of $E$, that is, there must be a path of edges
  from the root to any node in the graph.
  For a graph $G$, we refer to the set
  $T(G) = \{ n \in V | \forall m \in V.\; (n,m) \notin E\}$ of nodes
  with no edges coming from them as the set of \emph{end nodes} of the
  graph.
\end{defn}
In diagrams we represent the nodes as points or as the names of the
nodes, the edges as arrows, and the root node as a node with an arrow
pointing to it that does not come from another node.
Additionally, we refer to the source of an edge going to a given node
as a \emph{predecessor} of that node; similarly, the target of an edge
from a given node is a \emph{successor} of that node.

We now define what it means to replace a node in a graph by another
graph.
We use this concept to construct more complex structured graphs from
those shown in Figure~\ref{structured-cfg-figures}.
Node replacement may occur in four different ways, depending on which
node is being replaced in a graph.
We illustrate the different cases of node replacement using the
example graphs $G$ and $H$ shown in Figure~\ref{G-H-examples-figure}.
The $G$ graph has the form of a conditional with two branches, and the
$H$ graph has the form of a \texttt{while} loop.
We label the nodes of the graphs separately for ease of reference.

\begin{figure}
  \begin{center}
    \begin{tikzpicture}
      \useasboundingbox (-3,-2) rectangle (2,2);
      \node at (-1,2) {$G$};
      \node at (0,1.7) (start) {};
      \node at (0,1)  (A) {$1$};
      \node at (1,0)  (B) {$2$};
      \node at (-1,0) (C) {$3$};
      \node at (0,-1) (D) {$4$};
      \draw[-latex] (start) -- (A);
      \draw[-latex] (A) -- (B);
      \draw[-latex] (A) -- (C);
      \draw[-latex] (B) -- (D);
      \draw[-latex] (C) -- (D);
    \end{tikzpicture}
    \begin{tikzpicture}
      \useasboundingbox (-3,-2) rectangle (2,2);
      \node at (-1,2) {$H$};
      \node at (0,1.7) (start) {};
      \node at (0,1) (A) {$a$};
      \node at (0,-1) (B) {$b$};
      \node at (1,-1) (C) {$c$};
      \draw[-latex] (start) -- (A);
      \draw[-latex] (A) to (B);
      \draw[-latex] (A) to (C);
      \draw[-latex] (B) to[in=200,out=150] (A);
    \end{tikzpicture}
    \caption{Example control flow graphs to illustrate node
      replacement}
    \label{G-H-examples-figure}
  \end{center}
\end{figure}

The first case is that of placing a graph at the start of another
graph, i.e.\ replacing the root node of a graph that does not have a
loop to its root node.
An example of this can be seen in
Figure~\ref{root-replacement-figure}, where the root node (node $1$)
of graph $G$ is replaced with graph $H$.
The unique end node of graph $H$, node $c$, takes the place of node
$1$.
The other nodes of $H$ are connected to it by the same edges as in
$H$.

The second case is that of replacing one of the end nodes of a graph.
This is shown in Figure~\ref{end-replacement-figure}, where node $4$
of graph $G$ is replaced with graph $H$.
Node $a$, the root node of graph $H$, takes the place of node $4$.
As in the previous case, the remaining nodes of $H$ are included,
connected to $a$ by the same edges as in $H$.

The third case (Figure~\ref{internal-replacement-figure}) is that of
replacing an internal node of the graph.
In our example, node $2$ of graph $G$ is replaced with graph $H$.
There is an edge from the predecessor of node $2$, which is node $1$
in this case, to the root node of $H$ (node $a$).
There is another edge from the end node of $H$ (node $c$), which is
required to be unique, to the successor of node $3$, which is node $4$
in this case.

The final case, an example of which is shown in
Figure~\ref{branch-end-replacement-figure}, is where control flow
constructs occur at the end of one branch of a conditional.
In our example, node $2$ of graph $G$ is replaced with graph $H$, as
in the previous case, but the end node of $H$ (node $c$) is identified
with the successor of node $2$ (node $4$), and so it is not included
in the graph.
Thus, this represents the case in which no instructions occur inside
the conditional branch after the while loop.
Such instructions are represented by node $c$ in
Figure~\ref{internal-replacement-figure}, which is excluded in
Figure~\ref{branch-end-replacement-figure}.

\begin{figure}
  \begin{subfigure}{0.24\textwidth}
    \begin{center}
      \begin{tikzpicture}
        \useasboundingbox (-1.5,-2.5) rectangle (1.5,2);
        \node at (0,2) (start) {};
        \node at ( 0, 1)  (A) {$a$};
        \node at (-1, 0)  (B) {$b$};
        \node at ( 0,0)   (D) {$c$};
        \node at (-1,-1)  (E) {$2$};
        \node at ( 1,-1)  (F) {$3$};
        \node at ( 0,-2)  (G) {$4$};
        \draw[-latex] (start) -- (A);
        \draw[-latex] (A) to (B);
        \draw[-latex] (A) to (D);
        \draw[-latex] (B) to[in=180,out=90] (A);
        \draw[-latex] (D) -- (E);
        \draw[-latex] (D) -- (F);
        \draw[-latex] (E) -- (G);
        \draw[-latex] (F) -- (G);
      \end{tikzpicture}
    \caption{\centering root node\newline replacement}
    \label{root-replacement-figure}
  \end{center}
  \end{subfigure}
  \begin{subfigure}{0.24\textwidth}
    \begin{center}
      \begin{tikzpicture}
        \useasboundingbox (-1.5,-2.5) rectangle (1.5,2);
        \node at (0,2) (start) {};
        \node at ( 0, 1)  (A) {$1$};
        \node at (-1, 0)  (B) {$2$};
        \node at ( 1, 0)  (C) {$3$};
        \node at ( 0,-1)  (D) {$a$};
        \node at ( 0,-2)  (E) {$b$};
        \node at ( 1,-2)  (F) {$c$};
        \draw[-latex] (start) -- (A);
        \draw[-latex] (A) -- (B);
        \draw[-latex] (A) -- (C);
        \draw[-latex] (B) -- (D);
        \draw[-latex] (C) -- (D);
        \draw[-latex] (D) to (E);
        \draw[-latex] (D) to (F);
        \draw[-latex] (E) to[in=200,out=150] (D);
      \end{tikzpicture}
    \caption{\centering end node\newline replacement}
    \label{end-replacement-figure}
    \end{center}
  \end{subfigure}
  \begin{subfigure}{0.24\textwidth}
    \begin{center}
      \begin{tikzpicture}
        \useasboundingbox (-1.5,-2.5) rectangle (1.5,2);
        \node at (0,2) (start) {};
        \node at ( 0, 1)    (A) {$1$};
        \node at (-1, 0)    (B) {$a$};
        \node at ( 1,-0.5)  (C) {$3$};
        \node at (-2,-1)    (D) {$b$};
        \node at (-1,-1)    (F) {$c$};
        \node at ( 0,-2)    (G) {$4$};
        \draw[-latex] (start) -- (A);
        \draw[-latex] (A) -- (B);
        \draw[-latex] (A) -- (C);
        \draw[-latex] (B) to (D);
        \draw[-latex] (B) to (F);
        \draw[-latex] (D) to[in=180,out=90] (B);
        \draw[-latex] (F) -- (G);
        \draw[-latex] (C) -- (G);
      \end{tikzpicture}
      \caption{\centering internal node\newline replacement}
      \label{internal-replacement-figure}
    \end{center}
  \end{subfigure}
  \begin{subfigure}{0.24\textwidth}
    \begin{center}
      \begin{tikzpicture}
        \useasboundingbox (-1.5,-2.5) rectangle (1.5,2);
        \node at (0,2) (start) {};
        \node at ( 0, 1)    (A) {$1$};
        \node at (-1,-0.5)  (B) {$a$};
        \node at ( 1,-0.5)  (C) {$3$};
        \node at (-2,-1.5)  (D) {$b$};
        \node at ( 0,-2)    (G) {$4$};
        \draw[-latex] (start) -- (A);
        \draw[-latex] (A) -- (B);
        \draw[-latex] (A) -- (C);
        \draw[-latex] (B) to (D);
        \draw[-latex] (B) to (G);
        \draw[-latex] (D) to[in=180,out=90] (B);
        \draw[-latex] (C) -- (G);
      \end{tikzpicture}
    \caption{\centering branch end\newline replacement}
    \label{branch-end-replacement-figure}
  \end{center}
  \end{subfigure}
  \caption{Examples of the different cases of node replacement}
  \label{node-replacement-example-figures}
\end{figure}

In general, we define node replacement using the formal definition
below.
This covers each of the four cases shown above.
\begin{defn}[Node Replacement]
  Given two rooted directed graphs $G$ and $H$, we say $G'$ is the
  graph formed by \emph{replacing} a node $n$ of $G$ with $H$ if one
  of the following cases holds:
  \begin{itemize}
  \item $n$ has no predecessors in $G$, $H$ has only one end node, and
    \begin{itemize}
    \item $G'$ contains all the nodes of $H$ and $G$, except $n$,
    \item $G'$ contains the edges of $G$ except those going to or from
      $n$,
    \item $G'$ contains edges from the end node of $H$ to the
      successors of $n$ in $G$, and
    \item the root node of $G'$ is the root node of $H$;
    \end{itemize}
  \item $n$ has no successors in $G$, and
    \begin{itemize}
    \item $G'$ contains all the nodes of $H$ and $G$, except $n$,
    \item $G'$ contains the edges of $G$ except those going to or from
      $n$,
    \item $G'$ contains edges from the predecessors $n$ in $G$ to the
      root node of $H$, and
    \item the root node of $G'$ is the root node of $G$, or, if $n$ is
      the root node of $G$, the root node of $H$;
    \end{itemize}
  \item $H$ has a single end node and
    \begin{itemize}
    \item $G'$ contains all the nodes of $H$ and $G$, except $n$,
    \item $G'$ contains the edges of $G$ and the edges of $H$ except
      those going to or from $n$,
    \item $G'$ contains edges from the predecessors of $n$ in $G$ to
      the root node of $H$,
    \item $G'$ contains edges from the end node of $H$ to the
      successors of $n$ in $G$, and
    \item the root node of $G'$ is the root node of $G$, or, if $n$ is
      the root node of $G$, the root node of $H$;
    \end{itemize}
  \item $n$ has a single successor in $G$, $H$ has a single end
    node, and
    \begin{itemize}
    \item $G'$ contains all the nodes of $H$ and $G$, except $n$ and
      the end node of $H$,
    \item $G'$ contains the edges of $G$ except those going to or from
      $n$,
    \item $G'$ contains edges from the predecessors of the end node of
      $H$ to the successor of $n$ in $G$
    \item $G'$ contains edges from the predecessors of $n$ in $G$ to
      the root node of $H$, and
    \item the root node of $G'$ is the root node of $G$, or, if $n$ is
      the root node of $G$, the root node of $H$.
    \end{itemize}
  \end{itemize}
\end{defn}

With node replacement defined, we can now finally define what we mean
by a structure control flow graph in terms of node replacement and the
structured graphs shown in Figure~\ref{structured-cfg-figures}

\begin{defn}[Structured Control Flow Graph]
  If $G$ is a rooted directed graph, we say $G$ is a \emph{structured
    control flow graph} if $G$ is the trivial graph (the graph with a
  single node, which is also the root, and no edges) or if $G$ can be
  created by starting with the trivial graph and performing a finite
  number of node replacements to replace nodes with graphs of the
  forms shown in Figure~\ref{structured-cfg-figures}.
\end{defn}

Before applying the strategy, it must be ensured that the control flow
graph for each method is well-structured according to this definition.

We also require that integer operations are used only where integer
overflow cannot occur.
This is due to the fact that we follow icecap's approach and compile
addition in Java bytecode to addition in C, making the assumption that
the addition does not overflow, since overflow is undefined behaviour
in C.
The Java code should be written in a style consistent with MISRA-C,
where behaviours that are undefined in C must not be relied on.

We do not handle synchronisation of static methods, since icecap does
not apply synchronisation to static methods.
It is important that we follow the approach of a practical tool on
this, to ensure that our strategy can be implemented.
No expressive power is lost by forbidding static synchronized methods,
since a singleton instance of a class with synchronized methods can be
supplied, rather than relying on static methods to access data owned
by the instance.

Each method call must have at least one target (as
determined by the rules given in
Section~\ref{resolve-method-calls-subsection}), to allow method calls
to be resolved.
Each \texttt{invokestatic} and \texttt{invokespecial} instruction has
exactly one target, so this poperty is always fulfilled for such
method calls.
For \texttt{invokevirtual} instructions, a method call only has no
targets if the method in which the instruction occurs is unused or if
the method is invoked on a null pointer (which is erroneous).
Methods not used in the program should not be included in the
parameters passed to $CEE$, matching icecap's behaviour of excluding
such methods from the generated code.

Finally, we require that no method in the program recurses, either
directly or indirectly.
This is because recursion is not recommended in safety-critical
applications because of the potential for unpredictable failure due to
stack overflow, and it is not allowed in MISRA-C for that reason.
Imposing this requirement allows us to handle methods individually
when introducing their control flow, without considering circular
dependencies between them.

We now proceed to describe each of the stages of the strategy in
detail, beginning with the \emph{Elimination of Program Counter} stage
in the next section.

\section{Elimination of Program Counter}
\label{elimination-of-program-counter-section}

The first stage eliminates $pc$ from the state of each thread's
process, $Thr(bc,cs,instCS,t)$, introducing the control flow constructs of C
as a result. 
It is summarised by the following theorem.
%
\begin{thm}[Elimination of Program Counter]\label{epc-thm}
  \begin{circus}
    Thr(bc,cs,instCS,t) \circrefines ThrCF_{bc,cs}(cs,t)
  \end{circus}%
\end{thm}
%
We act mainly upon the $Running$ action of $Thr$; its loop is unrolled
to introduce the control flow that follows each bytecode instruction.
The aim is to get each method's bytecode instructions into a form in
which the control flow, but not the data operations, are described
using C constructs and, moreover, each path of execution (including
every branch of the conditionals) ends in a return instruction or a
loop.
We refer to a method in this form as a \emph{complete} method.

It is important to observe that it is possible to transform the
bytecode instructions of every method so that they become complete.
If we consider the control flow of a method beginning from that
method's entry point, each bytecode instruction reached must either be
a return instruction, or followed by another bytecode.
If another bytecode follows the bytecode's execution, then it must be
either a bytecode already considered, resulting in a loop, or one not
already considered.
Since there are finitely many bytecode instructions in a method, a
loop or return must eventually be reached.
Failure to do so would lead to an instruction beyond the end of the
method, which is forbidden by the structural restrictions on Java
bytecode that are checked during bytecode verification. 
% We assume bytecode input to our strategy will have undergone bytecode
% verification so this cannot happen.

When a method is complete, it can be defined by a separate \Circus{}
action.
When the code for all the methods has been separated out in this way,
the choice of bytecode instruction using the program counter value can
be removed and replaced with a choice over method identifiers.
Thus dependency on the program counter can be completely removed,
allowing it to be eliminated from the state of $Thr$.

The detailed description of the strategy for transforming $Thr$ in
this stage and achieving this elimination is provided by
Algorithm~\ref{epc-algorithm}.
It begins at line~\ref{algorithm-expand-bytecode} by expanding the
\Circus{} definitions of the bytecode instructions from the $bc$ map
into the $Running$ action, pulling out the program counter updates so
that they can be more easily manipulated.
In line~\ref{algorithm-introduce-forward-sequence}, simple sequential
compositions, that is, those that do not involve handling loops or
conditionals, are introduced.
% instructions that
% are followed by the execution of a bytecode instruction that is not
% part of the start or end of a conditional or loop are sequenced with
% the instructions following them.
After that, for each method, its loops and conditionals are introduced
in line~\ref{algorithm-introduce-loops-and-conditionals}. 
Afterwards, any complete methods are separated out, in
line~\ref{algorithm-separate-complete-methods}, and any method calls
involving completed methods are resolved by sequencing the method call
with the \Circus{} action representing the method, in
line~\ref{algorithm-resolve-method-calls}.

This is repeated until all methods have been separated out, as
indicated by the while loop in
line\added{s}~\ref{algorithm-method-loop}\added{
  to~\ref{algorithm-method-loop-end}}.
The $MainThread$ and $NotStarted$ actions are then refined in
line~\ref{algorithm-refine-main-actions} to provide a choice over
method identifiers, rather than $pc$ values, thus removing all uses of
$pc$ from the interpreter.
The $pc$ component is then removed from the state in
line~\ref{algorithm-remove-pc-from-state} of the algorithm.

\begin{algorithm}[tp!]
  \begin{algorithmic}[1]
    \State \Call{ExpandBytecode}{} \label{algorithm-expand-bytecode}
    \State \Call{IntroduceSequentialComposition}{} \label{algorithm-introduce-forward-sequence}
    \While{$\lnot$\Call{AllMethodsSeparated}{}} \label{algorithm-method-loop}
    \State \Call{IntroduceLoopsAndConditionals}{} \label{algorithm-introduce-loops-and-conditionals}
    \State \Call{SeparateCompleteMethods}{} \label{algorithm-separate-complete-methods}
    \State \Call{ResolveMethodCalls}{} \label{algorithm-resolve-method-calls}
    \EndWhile \label{algorithm-method-loop-end}
    \State \Call{RefineMainActions}{} \label{algorithm-refine-main-actions}
    \State \Call{RemovePCFromState}{} \label{algorithm-remove-pc-from-state}
  \end{algorithmic}
  \caption{Elimination of Program Counter}
  \label{epc-algorithm}
\end{algorithm}

Each of the procedures used in Algorithm~\ref{epc-algorithm} is
defined in a separate section in the sequel.
Beforehand, we give a more detailed overview of the strategy using an
example.

\subsection{Running Example}
\label{overview-subsection}

We explain the strategy in detail with an example, the Java code for
which is shown in Figure~\ref{example-code-figure}.
\begin{figure}[tp]
  \begin{center}
  \begin{minipage}{14cm}
  \begin{lstlisting}[basicstyle=\ttfamily\footnotesize,keywordstyle=\bf\footnotesize,language=Java,numbers=left,numberstyle=\tiny,stepnumber=1, numbersep=5pt,escapeinside={(*@}{@*)}]
public class TPK extends AperiodicEventHandler {

  public TPK(PriorityParameters priority,
             AperiodicParameters release,
             StorageParameters storage,
             ConfigurationParameters config) {
    super(priority, release, storage, config);
  }
      
  public void handleAsyncEvent() {
    ConsoleConnection console = new ConsoleConnection(null); (*@\label{example-ConsoleConnection-line}@*)
        
    InputStream input = console.openInputStream(); (*@\label{example-InputStream-line}@*)
    OutputStream output = console.openOutputStream(); (*@\label{example-OutputStream-line}@*)
        
    for(int i = 0; i <= 10; i = i + 1) { (*@\label{example-for-loop-line}@*)
      int y = f(input.read());
          
      if (y > 400) {
        output.write(0);
      } else {
        output.write(y);
      }
    }
  }
      
  public static int f(int x){
    return x + x + x + 5;
  }
      
}
\end{lstlisting}
\end{minipage}
\end{center}
  \caption{Our example program}
  \label{example-code-figure}
\end{figure}
Our example is based on the Trabb Pardo-Knuth
algorithm~\cite{knuth1980}, used for comparison of programming
languages, since it includes a variety of programming constructs that
provide a good test of the strategy.
We have simplified the algorithm by removing the reading into an
array, since our bytecode subset does not include array operations.
Adding arrays makes the example much longer, while not giving any
interesting insight into our compilation strategy.
As previously explained, extending the bytecode set considered to deal
with arrays is not difficult.

We have also written the example as an SCJ program, with the algorithm
as the body of an aperiodic event handler, \texttt{TPK}, one or more
instances of which can be registered as part of a mission and released
during mission execution.
As already mentioned, each release of the handler causes its
\texttt{handleAsyncEvent()} method to be executed.
This method creates an instance of a \texttt{ConsoleConnection}
(line~\ref{example-ConsoleConnection-line}), which is the only
standard input/output connection required by SCJ.
Instances of \texttt{InputStream} and \texttt{OutputStream} are then
obtained from the \texttt{ConsoleConnection}
(lines~\ref{example-InputStream-line}
and~\ref{example-OutputStream-line}).

After the input and output streams have been obtained, we enter a for
loop (line~\ref{example-for-loop-line}) in which an integer is read
from the \texttt{InputStream}, a static method \texttt{f()} is applied
to it, and the result is output if it is less than 400, otherwise, 0
is output.
The method \texttt{f()} takes an integer as input, multiplies it by 3
and adds 5 to it.

The \texttt{TPK} class is part of a larger program that includes other
classes, including a \texttt{Safelet}, a \texttt{MissionSequencer}, a
\texttt{Mission}, and the classes that make up the SCJ API.
% Considering these classes in our example would make the example much
% larger and more complex, while not introducing any more interesting
% aspects for the strategy to consider.
We omit a presentation of these classes, though it should be noted
that they are part of the complete example.
For compilation, they need to go through a similar refinement to that
we illustrate for the $TPK$ class.
This adds little complexity to the strategy since the bytecode array
is acted upon consistently for all classes, and the current class of a
given bytecode instruction can always be determined from its address
in the array.

The Java code must be run through a Java compiler to generate the
corresponding bytecode, which then defines the $bc$ and $cs$ constants
of our model.
Their values for our example are shown in
Figure~\ref{example-model-figure}, along with $TPK$ class information.
While most of the compilation of the methods of \texttt{TPK} depends
only on the data in the $TPK$ class information, the object data for
instances of \texttt{TPK} includes fields from its superclasses.
In particular, the fields for \texttt{TPK} are contributed by the
\texttt{AperiodicEventHandler} and \texttt{ManagedEventHandler}
classes (the superclasses of \texttt{ManagedEventHandler} do not
contribute any fields), whose $Class$ data structures are presented in
Figure~\ref{example-superclasses-model-figure}.
The generation of object structures from this field information is
discussed in more detail in
Section~\ref{data-refinement-of-objects-section}.

\begin{figure}[p]
  \begin{center}
  \setlength{\linewidth}{12cm}
  \begin{tabular}{p{9.5cm}p{4.5cm}}
    \scriptsize
    \setlength{\zedindent}{0cm}
    \setlength{\zedtab}{0.3cm}
    \setlength{\zedleftsep}{0cm}
    \begin{axdef}
      TPK : Class
    \where
      TPK = \lblot \\
      \t1 constantPool == \{ \\
      \t2 1 \mapsto ClassRef~TPKClassID, \\
      \t2 3 \mapsto ClassRef~AperiodicEventHandlerClassID, \\
      \t2 8 \mapsto MethodRef~AperiodicEventHandlerClassID~APEHinit, \\
      \t2 27 \mapsto ClassRef~ConsoleConnectionClassID, \\
      \t2 29 \mapsto  MethodRef~ConsoleConnectionClassID~CCinit, \\
      \t2 32 \mapsto MethodRef~ConsoleConnectionClassID~openInputStream, \\
      \t2 36 \mapsto MethodRef~ConsoleConnectionClassID~openOutputStream, \\
      \t2 40 \mapsto MethodRef~InputStreamClassID~read, \\
      \t2 41 \mapsto ClassRef~InputStreamClassID, \\
      \t2 46 \mapsto MethodRef~TPKClassID~f, \\
      \t2 50 \mapsto MethodRef~OutputStreamClassID~write, \\
      \t2 51 \mapsto ClassRef~OutputStreamClassID \\
      \t1 \}, \\
      \t1 this == 1, \\
      \t1 super == 3, \\
      \t1 interfaces == \{\}, \\
      \t1 methodEntry == \{ \\
      \t2 f \mapsto 43, \\
      \t2 handleAsyncEvent \mapsto 7, \\
      \t2 APEHinit \mapsto 0, \\
      \t1 \}, \\
      \t1 methodEnd == \{ \\
      \t2 f \mapsto 50, \\
      \t2 handleAsyncEvent \mapsto 42, \\
      \t2 APEHinit \mapsto 6 \\
      \t1 \}, \\
      \t1 methodLocals == \{ \\
      \t2 f \mapsto 1, \\
      \t2 handleAsyncEvent \mapsto 6, \\
      \t2 APEHinit \mapsto 5, \\
      \t1 \}, \\
      \t1 methodStackSize == \{ \\
      \t2 f \mapsto 2, \\
      \t2 handleAsyncEvent \mapsto 3, \\
      \t2 APEHinit \mapsto 5, \\
      \t1 \}, \\
      \t1 staticMethods == \{ f \} \\
      \t1 fields == \{\}, \\
      \t1 staticFields == \{\} \\
      \rblot
    \end{axdef}
    \begin{axdef}
      cs : ClassID \pfun Class
      \where
      cs = \{ \\
      \t1 TPKClassID \mapsto TPK, \\
      \t1 AperiodicEventHandlerClassID \mapsto AperiodicEventHandler, \\
      \t1 ManagedEventHandlerClassID \mapsto ManagedEventHandler, \\
      \t1 \cdots \\
      \}
    \end{axdef}
    &
    \scriptsize
    \setlength{\zedindent}{0cm}
    \setlength{\zedtab}{0.3cm}
    \setlength{\zedleftsep}{0cm}
    \begin{axdef}
      bc : ProgramAddress \pfun Bytecode
      \where
      bc = \{ \\
      	\t1 0 \mapsto aload~0, \\
        \t1 1 \mapsto aload~1, \\
        \t1 2 \mapsto aload~2, \\
        \t1 3 \mapsto aload~3, \\
        \t1 4 \mapsto aload~4, \\
        \t1 5 \mapsto invokespecial~8, \\
        \t1 6 \mapsto return, \\
        \t1 7 \mapsto new~27, \\
        \t1 8 \mapsto dup, \\
        \t1 9 \mapsto aconst\_null, \\
        \t1 10 \mapsto invokespecial~29, \\
        \t1 11 \mapsto astore~1, \\
        \t1 12 \mapsto aload~1, \\
        \t1 13 \mapsto invokevirtual~32, \\
        \t1 14 \mapsto astore~2, \\
        \t1 15 \mapsto aload~1, \\
        \t1 16 \mapsto invokevirtual~36, \\
        \t1 17 \mapsto astore~3, \\
        \t1 18 \mapsto iconst~0, \\
        \t1 19 \mapsto astore~4, \\
        \t1 20 \mapsto goto~19, \\
        \t1 21 \mapsto aload~2, \\
        \t1 22 \mapsto invokevirtual~40, \\
        \t1 23 \mapsto invokestatic~46, \\
        \t1 24 \mapsto astore~5, \\
        \t1 25 \mapsto aload~5, \\
        \t1 26 \mapsto iconst~400, \\
        \t1 27 \mapsto if\_icmple~5, \\
        \t1 28 \mapsto aload~3, \\
        \t1 29 \mapsto iconst~0, \\
        \t1 30 \mapsto invokevirtual~50, \\
        \t1 31 \mapsto goto~4, \\
        \t1 32 \mapsto aload~3, \\
        \t1 33 \mapsto aload~5, \\
        \t1 34 \mapsto invokevirtual~50, \\
        \t1 35 \mapsto aload~4, \\
        \t1 36 \mapsto iconst~1, \\
        \t1 37 \mapsto iadd, \\
        \t1 38 \mapsto astore~4, \\
        \t1 39 \mapsto aload~4, \\
        \t1 40 \mapsto iconst~10, \\
        \t1 41 \mapsto if\_icmple~(\negate 20), \\
        \t1 42 \mapsto return, \\
        \t1 43 \mapsto aload~0, \\
        \t1 44 \mapsto aload~0, \\
        \t1 45 \mapsto iadd, \\
        \t1 46 \mapsto aload~0, \\
        \t1 47 \mapsto iadd, \\
        \t1 48 \mapsto iconst~5, \\
        \t1 49 \mapsto iadd, \\
        \t1 50 \mapsto areturn, \\
        \t1 {} \cdots {} \\
        \}
      \end{axdef}
  \end{tabular}
  \end{center}
  \caption{The \Circus{} code corresponding to our example program}
  \label{example-model-figure}
\end{figure}%

\begin{figure}[tp!]
  \begin{center}
    % \setlength{\linewidth}{12cm}
    \begin{minipage}{7.5cm}
      \scriptsize
      \setlength{\abovedisplayskip}{0cm}
      \setlength{\belowdisplayskip}{0cm}
      \setlength{\zedindent}{0cm}
      \setlength{\zedtab}{0.3cm}
      \setlength{\zedleftsep}{0cm}
    \begin{axdef}
      AperiodicEventHandler : Class
    \where
      AperiodicEventHandler = \lblot \\
      \t1 constantPool == \{ \\
      \t2 1 \mapsto ClassRef~AperiodicEventHandlerClassID, \\
      \t2 3 \mapsto ClassRef~ManagedEventHandlerClassID, \\
      % \t2 8 \mapsto MethodRef~ManagedEventHandlerClassID~MEHinit, \\
      \t2 {} \cdots {} \\
      % \t2 11 \mapsto FieldRef~AperiodicEventHandler~priority, \\
      % \t2 15 \mapsto FieldRef~AperiodicEventHandler~backingStoreSpace, \\
      % \t2 18 \mapsto FieldRef~AperiodicEventHandler~allocAreaSpace, \\
      % \t2 21 \mapsto FieldRef~AperiodicEventHandler~stackSize, \\
      % \t2 24 \mapsto MethodRef~AperiodicEventHandler~initAPEH, \\
      % \t2 40 \mapsto MethodRef~AperiodicEventHandler~releaseAperiodic \\
      \t1 \}, \\
      \t1 this == 1, \\
      \t1 super == 3, \\
      \t1 interfaces == \{\}, \\
      \t1 methodEntry == \{ {} \cdots {}
      % \t2 MEHinit \mapsto 0, \\
      % \t2 release \mapsto 16 \\
      \}, \\
      \t1 methodEnd == \{ {} \cdots {}
      % \t2 MEHinit \mapsto 15, \\
      % \t2 release \mapsto 18 \\
      \}, \\
      \t1 methodLocals == \{ {} \cdots {}
      % \t2 MEHinit \mapsto 5, \\
      % \t2 release \mapsto 1 \\
      \}, \\
      \t1 methodStackSize == \{ {} \cdots {} 
      % \t2 MEHinit \mapsto 5, \\
      % \t2 release \mapsto 1 \\
      \}, \\
      \t1 fields == \{\}, \\
      \t1 staticFields == \{\} \\
      \rblot
    \end{axdef}%
  \end{minipage}
  \hfill
  \begin{minipage}{8cm}
    \scriptsize
    \setlength{\zedindent}{0cm}
    \setlength{\zedtab}{0.3cm}
    \setlength{\zedleftsep}{0cm}
    \begin{axdef}
      ManagedEventHandler : Class
    \where
      ManagedEventHandler = \lblot \\
      \t1 constantPool == \{ \\
      \t2 1 \mapsto ClassRef~ManagedEventHandlerClassID, \\
      \t2 3 \mapsto ClassRef~BoundAsyncEventHandlerClassID, \\
      \t2 5 \mapsto ClassRef~ManagedSchedulableClassID, \\
      % \t2 15 \mapsto MethodRef~BoundAsyncEventHandlerClassID~BAEHinit, \\
      \t2 {} \cdots {} \\
      % \t2 18 \mapsto MethodRef~PriorityParameters~getPriority, \\
      % \t2 19 \mapsto ClassRef~PriorityParameters, \\
      % \t2 24 \mapsto FieldRef~ManagedEventHandler~priority, \\
      % \t2 26 \mapsto MethodRef~ScopeParameters~getMaxInitialBackingStore, \\
      % \t2 27 \mapsto ClassRef~ScopeParameters, \\
      % \t2 32 \mapsto FieldRef~ManagedEventHandler~backingStoreSpace, \\
      % \t2 34 \mapsto MethodRef~ScopeParameters~getMaxInitialArea, \\
      % \t2 37 \mapsto FieldRef~ManagedEventHandler~allocAreaSpace, \\
      % \t2 39 \mapsto MethodRef~ConfigurationParameters~getSizes, \\
      % \t2 40 \mapsto ClassRef~ConfigurationParameters, \\
      % \t2 45 \mapsto MethodRef~LongArray~load, \\
      % \t2 46 \mapsto ClassRef~LongArray, \\
      % \t2 51 \mapsto FieldRef~ManagedEventHandler~stackSize, \\
      % \t2 66 \mapsto MethodRef~ManagedEventHandler~register \\
      \t1 \}, \\
      \t1 this == 1, \\
      \t1 super == 3, \\
      \t1 interfaces == \{5\}, \\
      \t1 methodEntry == \{ {} \cdots {}
      % \t2 MEHinit \mapsto 19, \\
      % \t2 getName \mapsto 41, \\
      % \t2 signalTermination \mapsto 46, \\
      % \t2 cleanUp \mapsto 40, \\
      % \t2 register \mapsto 43 \\
      \}, \\
      \t1 methodEnd == \{ {} \cdots {}
      % \t2 MEHinit \mapsto 39, \\
      % \t2 getName \mapsto 42, \\
      % \t2 signalTermination \mapsto 46, \\
      % \t2 cleanUp \mapsto 40, \\
      % \t2 register \mapsto 45 \\
      \}, \\
      \t1 methodLocals == \{ {} \cdots {}
      % \t2 MEHinit \mapsto 4, \\
      % \t2 getName \mapsto 1, \\
      % \t2 signalTermination \mapsto 1, \\
      % \t2 cleanUp \mapsto 1, \\
      % \t2 register \mapsto 1 \\
      \}, \\
      \t1 methodStackSize == \{ {} \cdots {}
      % \t2 MEHinit \mapsto 3, \\
      % \t2 getName \mapsto 1, \\
      % \t2 signalTermination \mapsto 0, \\
      % \t2 cleanUp \mapsto 0, \\
      % \t2 register \mapsto 1 \\
      \}, \\
      \t1 fields == \{ \\
      \t2 threadID, \\
      \t2 backingStoreSpace, \\
      \t2 allocAreaSpace, \\
      \t2 stackSize \\
      \t1 \}, \\
      \t1 staticFields == \{\} \\
      \rblot
    \end{axdef}
  \end{minipage}
  \end{center}
  \caption{The $Class$ structures for \texttt{AperiodicEventHandler}
    and \texttt{ManagedEventHandler}}
  \label{example-superclasses-model-figure}
\end{figure}%

Applying the bytecode expansion on
line~\ref{algorithm-expand-bytecode} of Algorithm~\ref{epc-algorithm}
yields the $Running$ action shown in
Figure~\ref{bytecode-expansion-example-figure}.
\begin{figure}[tp]
  \setlength{\zedindent}{0cm}
  \setlength{\zedtab}{0.3cm}
  \setlength{\zedleftsep}{0.1cm}
  \begin{circus}
    Running \circdef \\
    \t1 \circif frameStack = \emptyset \circthen \Skip \\
    \t1 {} \circelse frameStack \neq \emptyset \circthen {} \\
    \t2 \circif pc = 0 \circthen HandleAloadEPC(0) \circseq pc := 1 \\
    \t2 {} \circelse pc = 1 \circthen HandleAloadEPC(1) \circseq pc := 2 \\
    \t2 {} \circelse pc = 2 \circthen HandleAloadEPC(2) \circseq pc := 3 \\
    \t2 {} \circelse pc = 3 \circthen HandleAloadEPC(3) \circseq pc := 4 \\
    \t2 {} \circelse pc = 4 \circthen HandleAloadEPC(4) \circseq pc := 5 \\
    \t2 {} \circelse pc = 5 \circthen \{ pc = 5 \} \circseq HandleInvokespecialEPC(8) \\
    \t2 {} \circelse pc = 6 \circthen HandleReturnEPC \\
    \t2 {} \circelse pc = 7 \circthen HandleNewEPC(27) \circseq pc := 8 \\
    \t2 {} \circelse pc = 8 \circthen HandleDupEPC \circseq pc := 9 \\
    \t2 {} \circelse pc = 9 \circthen HandleAconst\_nullEPC \circseq pc := 10 \\
    \t2 {} \circelse pc = 10 \circthen \{ pc = 10 \} \circseq HandleInvokespecialEPC(29) \\
    \t2 {} \circelse pc = 11 \circthen HandleAstoreEPC(1) \circseq pc := 12 \\
    % \t2 {} \circelse pc = 12 \circthen HandleAloadEPC(1) \circseq pc := 13 \\
    % \t2 {} \circelse pc = 13 \circthen HandleInvokevirtualEPC(32) \\
    % \t2 {} \circelse pc = 14 \circthen HandleAstoreEPC(2) \circseq pc := 15 \\
    % \t2 {} \circelse pc = 15 \circthen HandleAloadEPC(1) \circseq pc := 16 \\
    % \t2 {} \circelse pc = 16 \circthen HandleInvokevirtualEPC(36) \\
    % \t2 {} \circelse pc = 17 \circthen HandleAstoreEPC(3) \circseq pc := 18 \\
    % \t2 {} \circelse pc = 18 \circthen HandleIconstEPC(0) \circseq pc := 19 \\
    % \t2 {} \circelse pc = 19 \circthen HandleAstoreEPC(4) \circseq pc := 20 \\
    % \t2 {} \circelse pc = 20 \circthen pc := 39 \\
    \t2 {} \cdots {} \\
    % \t2 {} \circelse pc = 21 \circthen HandleAloadEPC(2) \circseq pc := 22 \\
    % \t2 {} \circelse pc = 22 \circthen HandleInvokevirtualEPC(40) \\
    % \t2 {} \circelse pc = 23 \circthen HandleInvokestaticEPC(46) \\
    % \t2 {} \circelse pc = 24 \circthen HandleAstoreEPC(5) \circseq pc := 25 \\
    % \t2 {} \circelse pc = 25 \circthen HandleAloadEPC(5) \circseq pc := 26 \\
    % \t2 {} \circelse pc = 26 \circthen HandleIconstEPC(400) \circseq pc := 27 \\
    % \t2 {} \circelse pc = 27 \circthen \circvar value1, value2 : Word \circspot InterpreterPop2 \circseq \\
    % \t3 pc := \IF value1 \leq value2 \THEN 32 \ELSE 28 \\
    % \t2 {} \circelse pc = 28 \circthen HandleAloadEPC(3) \circseq pc := 29 \\
    % \t2 {} \circelse pc = 29 \circthen HandleIconstEPC(0) \circseq pc := 30 \\
    % \t2 {} \circelse pc = 30 \circthen HandleInvokevirtualEPC(50) \\
    % \t2 {} \circelse pc = 31 \circthen pc := 35 \\
    % \t2 {} \circelse pc = 32 \circthen HandleAloadEPC(3) \circseq pc := 33 \\
    % \t2 {} \circelse pc = 33 \circthen HandleAloadEPC(5) \circseq pc := 34 \\
    % \t2 {} \circelse pc = 34 \circthen HandleInvokevirtualEPC(50) \\
    % \t2 {} \circelse pc = 35 \circthen HandleAloadEPC(4) \circseq pc := 36 \\
    % \t2 {} \circelse pc = 36 \circthen HandleIconstEPC(1) \circseq pc := 37 \\
    % \t2 {} \circelse pc = 37 \circthen HandleIaddEPC \circseq pc := 38 \\
    % \t2 {} \circelse pc = 38 \circthen HandleAstoreEPC(4) \circseq pc := 39 \\
    % \t2 {} \circelse pc = 39 \circthen HandleAloadEPC(4) \circseq pc := 40 \\
    % \t2 {} \circelse pc = 40 \circthen HandleIconstEPC(10) \circseq pc := 41 \\
    % \t2 {} \circelse pc = 41 \circthen \circvar value1, value2 : Word \circspot InterpreterPop2 \circseq \\
    % \t3 pc := \IF value1 \leq value2 \THEN 21 \ELSE 42 \\
    % \t2 {} \circelse pc = 42 \circthen HandleReturnEPC \\
    % \t2 {} \circelse pc = 43 \circthen HandleAloadEPC(0) \circseq pc := 44 \\
    % \t2 {} \circelse pc = 44 \circthen HandleAloadEPC(0) \circseq pc := 45 \\
    % \t2 {} \circelse pc = 45 \circthen HandleIaddEPC \circseq pc := 46 \\
    % \t2 {} \circelse pc = 46 \circthen HandleAloadEPC(0) \circseq pc := 47 \\
    % \t2 {} \circelse pc = 47 \circthen HandleIaddEPC \circseq pc := 48 \\
    % \t2 {} \circelse pc = 48 \circthen HandleIconstEPC(5) \circseq pc := 49 \\
    % \t2 {} \circelse pc = 49 \circthen HandleIaddEPC \circseq pc := 50 \\
    % \t2 {} \circelse pc = 50 \circthen HandleAreturnEPC \\
    \t2 \circfi \circseq Poll \circseq Running \\
    \t1 \circfi
  \end{circus}
  \caption{The $Running$ action after bytecode expansion}
  \label{bytecode-expansion-example-figure}
\end{figure}
This step copies $HandleInstruction$ into $Running$, and converts it
to a choice of actions based on the value of the program counter,
$pc$, mirroring the contents of the $bc$ map for each value.

The actions that make up $HandleInstruction$ are also replaced with
actions that incorporate instruction parameters from the $bc$ map, and
have $pc$ updates separated from stack updates.
This can be seen in Figure~\ref{bytecode-expansion-example-figure},
where, for instance, in the $pc = 0$ case, $aload~0$ has been
converted to $HandleAloadEPC(0) \circseq pc := 1$, with the parameter,
$0$, to the bytecode instruction becoming a parameter of the new
instruction handling action $HandleAloadEPC$, and the update to $pc$
placed after the data operation.

The reason for making parameters of the bytecode instructions into
parameters of the handling actions is to remove the need to reference
the bytecode instructions in the $bc$ map, as that involves use of the
$pc$ value, which we seek to remove in this stage.
This also has the benefit of fully incorporating $bc$ into the $Thr$
process, ensuring all the information required to introduce C code
constructs is available directly in \Circus{}.
This makes stating compilation laws simpler, and is described in more
detail in Section~\ref{expand-bytecode-subsection}, where we define
the \Call{ExpandBytecode}{} procedure.

\begin{figure}[t!]
  \setlength{\zedindent}{0cm}
  \setlength{\zedtab}{0.3cm}
  \setlength{\zedleftsep}{0.1cm}
  \begin{circus}
    Running \circdef \\
    \t1 \circif frameStack = \emptyset \circthen \Skip \\
    \t1 {} \circelse frameStack \neq \emptyset \circthen {} \\
    \t2 \circif pc = 0 \circthen HandleAloadEPC(0) \circseq pc := 1 \circseq Poll \circseq HandleAloadEPC(1) \circseq \\
    \t3 pc := 2 \circseq Poll \circseq HandleAloadEPC(2) \circseq pc := 3 \circseq Poll \circseq HandleAloadEPC(4) \circseq \\
    \t3 pc := 5 \circseq Poll \circseq \{ pc = 5 \} \circseq HandleInvokespecialEPC(8) \\
    \t2 {} \cdots {} \\
    % \t2 {} \circelse pc = 1 \circthen HandleAloadEPC(1) \circseq pc := 2 \\
    % \t2 {} \circelse pc = 2 \circthen HandleAloadEPC(2) \circseq pc := 3 \\
    % \t2 {} \circelse pc = 3 \circthen HandleAloadEPC(3) \circseq pc := 4 \\
    % \t2 {} \circelse pc = 4 \circthen HandleAloadEPC(4) \circseq pc := 5 \\
    % \t2 {} \circelse pc = 5 \circthen HandleInvokespecialEPC(8) \\
    \t2 {} \circelse pc = 6 \circthen HandleReturnEPC \\
    \t2 {} \circelse pc = 7 \circthen HandleNewEPC(27) \circseq pc := 8 \circseq Poll \circseq HandleDupEPC \circseq pc := 9 \circseq Poll \circseq \\
    \t3 HandleAconst\_nullEPC \circseq pc := 10 \circseq Poll \circseq \{ pc = 10 \} \circseq HandleInvokespecialEPC(29) \\
    \t2 {} \cdots {} \\
    % \t2 {} \circelse pc = 8 \circthen HandleDupEPC \circseq pc := 9 \\
    % \t2 {} \circelse pc = 9 \circthen HandleAconst\_nullEPC \circseq pc := 10 \\
    % \t2 {} \circelse pc = 10 \circthen HandleInvokespecialEPC(29) \\
    % \t2 {} \circelse pc = 11 \circthen HandleAstoreEPC(1) \circseq pc := 12 \circseq Poll \circseq HandleAloadEPC(1) \circseq \\
    % \t3 pc := 13 \circseq Poll \circseq HandleInvokevirtualEPC(32) \\
    % \t2 {} \cdots {} \\
    % \t2 {} \circelse pc = 12 \circthen HandleAloadEPC(1) \circseq pc := 13 \\
    % \t2 {} \circelse pc = 13 \circthen HandleInvokevirtualEPC(32) \\
    \t2 {} \circelse pc = 14 \circthen HandleAstoreEPC(2) \circseq pc := 15 \circseq Poll \circseq HandleAloadEPC(1) \circseq \\
    \t3 pc := 16 \circseq Poll \circseq \{ pc = 16 \} \circseq HandleInvokevirtualEPC(36) \\
    \t2 {} \cdots {} \\
    % \t2 {} \circelse pc = 15 \circthen HandleAloadEPC(1) \circseq pc := 16 \\
    % \t2 {} \circelse pc = 16 \circthen HandleInvokevirtualEPC(36) \\
    \t2 {} \circelse pc = 17 \circthen HandleAstoreEPC(3) \circseq pc := 18 \circseq Poll \circseq HandleIconstEPC(0) \circseq \\
    \t3 pc := 19 \circseq Poll \circseq HandleAstoreEPC(4) \circseq pc := 20 \circseq Poll \circseq pc := 39 \\
    \t2 {} \cdots {} \\
    % \t2 {} \circelse pc = 18 \circthen HandleIconstEPC(0) \circseq pc := 19 \\
    % \t2 {} \circelse pc = 19 \circthen HandleAstoreEPC(4) \circseq pc := 20 \\
    % \t2 {} \circelse pc = 20 \circthen pc := 39 \\
    % \t2 {} \circelse pc = 21 \circthen HandleAloadEPC(2) \circseq pc := 22 \circseq Poll \circseq HandleInvokevirtualEPC(40) \\
    % \t2 {} \circelse pc = 22 \circthen HandleInvokevirtualEPC(40) \\
    % \t2 {} \circelse pc = 23 \circthen HandleInvokestaticEPC(46) \\
    % \t2 {} \circelse pc = 24 \circthen HandleAstoreEPC(5) \circseq pc := 25 \circseq Poll \circseq HandleAloadEPC(5) \circseq \\
    % \t3 pc := 26 \circseq Poll \circseq HandleIconstEPC(400) \circseq pc := 27 \circseq Poll \circseq \\
    % \t3 \circvar value1, value2 : Word \circspot InterpreterPop2 \circseq \\
    % \t3 pc := \IF value1 \leq value2 \THEN 32 \ELSE 28 \\
    % \t2 {} \circelse pc = 25 \circthen HandleAloadEPC(5) \circseq pc := 26 \\
    % \t2 {} \circelse pc = 26 \circthen HandleIconstEPC(400) \circseq pc := 27 \\
    % \t2 {} \circelse pc = 27 \circthen \circvar value1, value2 : Word \circspot InterpreterPop2 \circseq \\
    % \t3 pc := \IF value1 \leq value2 \THEN 32 \ELSE 28 \\
    % \t2 {} \circelse pc = 28 \circthen HandleAloadEPC(3) \circseq pc := 29 \circseq Poll \circseq HandleIconstEPC(0) \circseq \\
    % \t3 pc := 30 \circseq Poll \circseq HandleInvokevirtualEPC(50) \\
    % \t2 {} \circelse pc = 29 \circthen HandleIconstEPC(0) \circseq pc := 30 \\
    % \t2 {} \circelse pc = 30 \circthen HandleInvokevirtualEPC(50) \\
    % \t2 {} \circelse pc = 31 \circthen pc := 35 \\
    % \t2 {} \circelse pc = 32 \circthen HandleAloadEPC(3) \circseq pc := 33 \circseq Poll \circseq HandleAloadEPC(5) \circseq \\
    % \t3 pc := 34 \circseq Poll \circseq HandleInvokevirtualEPC(50) \\
    % \t2 {} \circelse pc = 33 \circthen HandleAloadEPC(5) \circseq pc := 34 \\
    % \t2 {} \circelse pc = 34 \circthen HandleInvokevirtualEPC(50) \\
    % \t2 {} \circelse pc = 35 \circthen HandleAloadEPC(4) \circseq pc := 36 \circseq Poll \circseq HandleIconstEPC(1) \circseq \\
    % \t3 pc := 37 \circseq Poll \circseq HandleIaddEPC \circseq pc := 38 \circseq Poll \circseq HandleAstoreEPC(4) \circseq \\
    % \t3 pc := 39 \\
    % \t2 {} \circelse pc = 36 \circthen HandleIconstEPC(1) \circseq pc := 37 \\
    % \t2 {} \circelse pc = 37 \circthen HandleIaddEPC \circseq pc := 38 \\
    % \t2 {} \circelse pc = 38 \circthen HandleAstoreEPC(4) \circseq pc := 39 \\
    \t2 {} \circelse pc = 39 \circthen HandleAloadEPC(4) \circseq pc := 40 \circseq Poll \circseq HandleIconstEPC(10) \circseq \\
    \t3 pc := 41 \circseq Poll \circseq \circvar value1, value2 : Word \circspot InterpreterPop2 \circseq \\
    \t3 pc := \IF value1 \leq value2 \THEN 21 \ELSE 42 \\
    \t2 {} \cdots {} \\
    % \t2 {} \circelse pc = 40 \circthen HandleIconstEPC(10) \circseq pc := 41 \\
    % \t2 {} \circelse pc = 41 \circthen \circvar value1, value2 : Word \circspot InterpreterPop2 \circseq \\
    % \t3 pc := \IF value1 \leq value2 \THEN 21 \ELSE 42 \\
    \t2 {} \circelse pc = 42 \circthen HandleReturnEPC \\
    \t2 {} \circelse pc = 43 \circthen HandleAloadEPC(0) \circseq pc := 44 \circseq Poll \circseq HandleAloadEPC(0) \circseq \\
    \t3 pc := 45 \circseq Poll \circseq HandleIaddEPC \circseq pc := 46 \circseq Poll \circseq HandleAloadEPC(0) \circseq \\
    \t3 pc := 47 \circseq Poll \circseq HandleIaddEPC \circseq pc := 48 \circseq Poll \circseq HandleIaddEPC \circseq \\
    \t3 pc := 50 \circseq Poll \circseq HandleAreturnEPC \\
    \t2 {} \cdots {} \\
    % \t2 {} \circelse pc = 44 \circthen HandleAloadEPC(0) \circseq pc := 45 \\
    % \t2 {} \circelse pc = 45 \circthen HandleIaddEPC \circseq pc := 46 \\
    % \t2 {} \circelse pc = 46 \circthen HandleAloadEPC(0) \circseq pc := 47 \\
    % \t2 {} \circelse pc = 47 \circthen HandleIaddEPC \circseq pc := 48 \\
    % \t2 {} \circelse pc = 48 \circthen HandleIconstEPC(5) \circseq pc := 49 \\
    % \t2 {} \circelse pc = 49 \circthen HandleIaddEPC \circseq pc := 50 \\
    % \t2 {} \circelse pc = 50 \circthen HandleAreturnEPC \\
    \t2 \circfi \circseq Poll \circseq Running \\
    \t1 \circfi
  \end{circus}
  \caption{The $Running$ action after forward sequence introduction}
  \label{forward-sequence-introduction-example-figure}
\end{figure}

On line~\ref{algorithm-introduce-forward-sequence} of the
Algorithm~\ref{epc-algorithm}, sequential composition is introduced
for instructions that do not affect the sequential flow of the
program.
Such instructions are identified by considering the control flow graph
of the program and locating nodes with a single outgoing edge going to
a target node with exactly one incoming edge.
The introduction of sequential composition is performed by unrolling
the loop in $Running$ to introduce the control flow following each of
these instructions.
This causes the instruction to be sequentially composed with the next
instruction, with $Poll$ inbetween to allow for thread switches
between instructions.
This is performed exhaustively to get the code in the form shown in
Figure~\ref{forward-sequence-introduction-example-figure}, where the
choice over $pc$ has sequences of instructions collected together at
the point where they start, up to the point at which a more complex
control flow (such as a method call, conditional or a loop) occurs.
The introduction of sequential composition is described in more detail
in Section~\ref{introduce-forward-sequence-subsection}, where we
define the \Call{IntroduceSequentialComposition}{} procedure.

Handling the remaining constructs requires consideration of
dependencies between methods to ensure method calls can be resolved
correctly.
We say a method call is \emph{resolved} when the method invocation
bytecode has been placed in sequential composition with a call to a
\Circus{} action containing the body of the method being invoked,
which is then followed by the sequence of instructions that occurs
after the invocation bytecode in the calling method.
After a method call has been resolved, it no longer breaks up the
sequence of instructions it occurs in.
%TODO: clarify this

Since we have the bytecode instructions of all the methods needed, we
can always resolve the call of a complete method, provided that method
has already been split into its own \Circus{} action.
To obtain a complete method, we first perform loop and conditional
introduction upon the method.
Since introducing loops and conditionals requires unbroken sequences
of instructions that form the bodies of the loops and the branches of
conditionals, introduction of loops and conditionals can only be
performed on methods that have no unresolved method calls.

In our example, \texttt{handleAsyncEvent()} is the only method that
needs loops and conditionals introducing but, since it also contains
method calls that break up the body of a loop, we must wait until its
method calls have been resolved before introducing loops and
conditionals.
For this reason, we perform method call resolution, and loop and
conditional introduction repeatedly until all method calls are
resolved and the resulting complete methods have all been separated
out.
This is expressed in Algorithm~\ref{epc-algorithm} by the while loop
on line~\ref{algorithm-method-loop}.

Introduction of loops and conditionals to the body of a method with no
unresolved method calls occurs on
line~\ref{algorithm-introduce-loops-and-conditionals} of the
algorithm.
To introduce loops and conditionals we consider the control flow graph
of the method again, though it is now much simpler than the control
flow graph used for sequence introduction, since straight sequences of
instructions have already been combined together.
Patterns representing conditionals and loops are then identified using
the control flow graph and the corresponding constructs are
introduced.
As loops and conditionals are introduced, nodes in the control flow
graph are merged until the graph consists of a single node, which is
the starting point of the method, containing the complete method body.

The result of introducing loops and conditionals in
\texttt{handleAsyncEvent()} after method call resolution is shown in
Figure~\ref{loop-and-conditional-introduction-example-figure}.
The process of introducing loops and conditionals is described in more
detail in Section~\ref{introduce-loops-and-conditionals-subsection},
where we define the \Call{IntroduceLoopsAndConditionals}{} procedure.
\begin{figure}[t!]
  \setlength{\zedindent}{0cm}
  \setlength{\zedtab}{0.5cm}
  \setlength{\zedleftsep}{0.1cm}
  \begin{circus}
    Running \circdef \\
    \t1 \circif frameStack = \emptyset \circthen \Skip \\
    \t1 {} \circelse frameStack \neq \emptyset \circthen {} \\
    \t2 \circif pc = 0 \circthen HandleAloadEPC(0) \circseq pc := 1 \circseq Poll \circseq HandleAloadEPC(1) \circseq \\
    \t3 {} \cdots {} \\
    % \t3 pc := 2 \circseq Poll \circseq HandleAloadEPC(2) \circseq pc := 3 \circseq Poll \circseq HandleAloadEPC(4) \circseq \\
    % \t3 pc := 5 \circseq Poll \circseq HandleInvokespecialEPC(8) \circseq Poll \circseq AperiodicEventHandler\_APEHinit \circseq \\
    \t3 Poll \circseq HandleReturnEPC \\
    \t2 {} \circelse pc = 7 \circthen HandleNewEPC(27) \circseq pc := 8 \circseq Poll \circseq HandleDupEPC \circseq pc := 9 \circseq \\
    % \t3 Poll \circseq HandleAconst\_nullEPC \circseq pc := 10 \circseq Poll \circseq (\circvar poppedArgs : \seq Word \circspot \\
    % \t4 \lschexpract \exists argsToPop? == m + 1 @ InterpreterStackFrameInvoke \rschexpract \circseq \\
    % \t4 \lschexpract InterpreterNewStackFrame[\\
    % \t5 ConsoleConnection/class?, CCinit/methodID?, poppedArgs/methodArgs?] \rschexpract) \circseq \\
    % \t3 Poll \circseq ConsoleConnection\_CCinit \circseq pc := 11 \circseq Poll \circseq HandleAstoreEPC(1) \circseq \\
    % pc := 12 \circseq Poll \circseq \\
    % \t3 HandleAloadEPC(1) \circseq pc := 13 \circseq Poll \circseq HandleInvokevirtualEPC(32) \circseq Poll \circseq \\
    % \t3 ConsoleConnection\_openInputStream \circseq Poll \circseq  HandleAstoreEPC(2) \circseq pc := 15 \circseq Poll \circseq \\
    % \t3 HandleAloadEPC(1) \circseq pc := 16 \circseq Poll \circseq HandleInvokevirtualEPC(36) \circseq Poll \circseq \\
    % \t3 ConsoleConnection\_openOutputStream \circseq Poll \circseq HandleAstoreEPC(3) \circseq pc := 18 \circseq Poll \circseq \\
    % \t3 HandleIconstEPC(0) \circseq pc := 19 \circseq Poll \circseq HandleAstoreEPC(4) \circseq pc := 20 \circseq Poll \circseq \\
    \t3 {} \cdots {} \\
    \t3 pc := 39 \circseq Poll \circseq \circmu Y \circspot \\
    % HandleAloadEPC(4) \circseq pc := 40 \circseq Poll \circseq HandleIconstEPC(10) \circseq \\
    \t4 {} \cdots {} \\
    \t4 pc := 41 \circseq Poll \circseq (\circvar value1, value2 : Word \circspot InterpreterPop2 \circseq \\
    \t4 \circif value1 \leq value2 \circthen pc := 21 \circseq Poll \circseq HandleAloadEPC(2) \circseq pc := 22 \circseq Poll \circseq \\
    % \t4 HandleInvokevirtualEPC(40) \circseq Poll \circseq InputStream\_Read \circseq Poll \circseq \\
    % \t4 HandleInvokestaticEPC(46) \circseq Poll \circseq TPK\_f  \circseq Poll \circseq \\
    % \t4 HandleAstoreEPC(5) \circseq pc := 25 \circseq Poll \circseq HandleAloadEPC(5) \circseq pc := 26 \circseq Poll \circseq \\
    % \t4 HandleIconstEPC(400) \circseq
    \t5 {} \cdots {} \\
    \t5 pc := 27 \circseq Poll \circseq (\circvar value1, value2 : Word \circspot InterpreterPop2 \circseq \\
    \t5 \circif value1 \leq value2 \circthen pc := 32 \circseq HandleAloadEPC(3) \circseq pc := 33 \circseq Poll \circseq \\
    % \t5 HandleAloadEPC(5) \circseq pc := 34 \circseq Poll \circseq HandleInvokevirtualEPC(50) \circseq Poll \circseq \\
    % \t5 OutputStream\_write \\
    \t6 {} \cdots {} \\
    \t5 {} \circelse value1 > value2 \circthen pc := 28 \circseq HandleAloadEPC(3) \circseq pc := 29 \circseq Poll \circseq \\
    % \t5 HandleIconstEPC(0) \circseq pc := 30 \circseq Poll \circseq HandleInvokevirtualEPC(50) \circseq \\
    % \t5 OutputStream\_write \\
    \t6 {} \cdots {} \\
    \t5 \circfi) \circseq pc := 35 \circseq Poll \circseq HandleAloadEPC(4) \circseq pc := 36 \circseq Poll \circseq \\
    \t5 HandleIconstEPC(1) \circseq pc := 37 \circseq Poll \circseq HandleIaddEPC \circseq pc := 38 \circseq Poll \circseq \\
    \t5 HandleAstoreEPC(4) \circseq pc := 39 \circseq Poll \circseq Y \\
    \t4 {} \circelse value1 > value2 \circthen \Skip \\
    \t4 \circfi \circseq pc := 42 \circseq Poll \circseq HandleReturnEPC \\
    \t2 {} \cdots {} \\
    \t2 {} \circelse pc = 43 \circthen TPK\_f \\
    % \t2 {} \circelse pc = 43 \circthen HandleAloadEPC(0) \circseq pc := 44 \circseq Poll \circseq HandleAloadEPC(0) \circseq \\
    % \t3 pc := 45 \circseq Poll \circseq HandleIaddEPC \circseq pc := 46 \circseq Poll \circseq HandleAloadEPC(0) \circseq \\
    % \t3 pc := 47 \circseq Poll \circseq HandleIaddEPC \circseq pc := 48 \circseq Poll \circseq HandleIaddEPC \circseq \\
    % \t3 pc := 50 \circseq Poll \circseq HandleAreturnEPC \\
    \t2 {} \cdots {} \\
    \t2 \circfi) \circseq Poll \circseq Running \\
    \t1 \circfi
  \end{circus}
  \caption{The $Running$ action after loop and conditional introduction}
  \label{loop-and-conditional-introduction-example-figure}
\end{figure}

After loops and conditionals have been introduced, methods that are
then complete can be copied into separate actions.
This occurs in line~\ref{algorithm-separate-complete-methods} of this
algorithm.
It is done with a simple application of the copy rule, replacing the
actions at the entry points of the split methods with references to
newly created method actions.
This can be seen in
Figure~\ref{method-call-resolution-example-figure}, where the $TPK\_f$
action has been created by copying the sequence of actions for the
\texttt{f()} method of \texttt{TPK} from the $pc = 43$ case.
As this step is relatively simple, we do not explain it in a separate
section.
\begin{figure}[t!]
  \setlength{\zedindent}{0cm}
  \setlength{\zedtab}{0.3cm}
  \setlength{\zedleftsep}{0cm}
  \begin{circus}
    Running \circdef \\
    \t1 \circif frameStack = \emptyset \circthen \Skip \\
    \t1 {} \circelse frameStack \neq \emptyset \circthen {} \\
    \t2 {} \circif pc = 0 \circthen HandleAloadEPC(0) \circseq pc := 1 \circseq Poll \circseq HandleAloadEPC(1) \circseq \\
    \t3 pc := 2 \circseq Poll \circseq HandleAloadEPC(2) \circseq pc := 3 \circseq Poll \circseq HandleAloadEPC(3) \circseq \\
    \t3 pc := 4 \circseq Poll \circseq HandleAloadEPC(4) \circseq pc := 5 \circseq Poll \circseq (\circvar poppedArgs : \seq Word \circspot \\
    \t4 \lschexpract \exists argsToPop?  == 6 @ InterpreterStackFrameInvoke \rschexpract \circseq \\
    \t4 \lschexpract InterpreterNewStackFrame[\\
    \t5 AperiodicEventHandler/class?, APEHinit/methodID?, poppedArgs/methodArgs?] \rschexpract \circseq \\
    \t3 Poll \circseq AperiodicEventHandler\_APEHinit) \circseq pc := 6 \circseq Poll \circseq HandleReturnEPC \\
    \t2 {} \cdots {} \\
    % \t2 {} \circelse pc = 1 \circthen HandleAloadEPC(1) \circseq pc := 2 \circseq Poll \circseq HandleAloadEPC(2) \circseq pc := 3 \circseq \\
    % \t3 Poll \circseq HandleAloadEPC(3) \circseq pc := 4 \circseq Poll \circseq HandleAloadEPC(4) \circseq pc := 5 \circseq \\
    % \t3 Poll \circseq HandleInvokespecialEPC(8) \circseq Poll \circseq APEHInit \circseq Poll \circseq HandleReturnEPC \\
    % \t2 {} \circelse pc = 2 \circthen HandleAloadEPC(2) \circseq pc := 3 \circseq Poll \circseq HandleAloadEPC(3) \circseq pc := 4 \circseq \\
    % \t3 Poll \circseq HandleAloadEPC(4) \circseq pc := 5 \circseq Poll \circseq HandleInvokespecialEPC(8) \circseq \\
    % \t3 Poll \circseq APEHInit \circseq Poll \circseq HandleReturnEPC \\
    % \t2 {} \circelse pc = 3 \circthen HandleAloadEPC(3) \circseq pc := 4 \circseq Poll \circseq HandleAloadEPC(4) \circseq pc := 5 \circseq \\
    % \t3 Poll \circseq HandleInvokespecialEPC(8) \circseq Poll \circseq APEHInit \circseq Poll \circseq HandleReturnEPC \\
    % \t2 {} \circelse pc = 4 \circthen HandleAloadEPC(4) \circseq pc := 5 \circseq Poll \circseq HandleInvokespecialEPC(8) \circseq \\
    % \t3 Poll \circseq APEHInit \circseq Poll \circseq HandleReturnEPC \\
    % \t2 {} \circelse pc = 5 \circthen HandleInvokespecialEPC(8) \circseq Poll \circseq APEHInit \circseq Poll \circseq HandleReturnEPC \\
    % \t2 {} \circelse pc = 6 \circthen HandleReturnEPC \\
    % \t2 {} \circelse pc = 7 \circthen HandleNewEPC(27) \circseq pc := 8 \circseq Poll \circseq HandleDupEPC \circseq pc := 9 \circseq \\
    % \t3 Poll \circseq HandleAconst\_nullEPC \circseq pc := 10 \circseq Poll \circseq HandleInvokespecialEPC(29) \circseq \\
    % \t3 Poll \circseq ConsoleConnection\_CCInit \circseq Poll \circseq HandleAstoreEPC(1) \circseq pc := 12 \circseq Poll \circseq \\
    % \t3 HandleAloadEPC(1) \circseq pc := 13 \circseq Poll \circseq HandleInvokevirtualEPC(32) \circseq Poll \circseq \\
    % \t3 OpenInputStream \circseq Poll \circseq HandleAstoreEPC(2) \circseq pc := 15 \circseq Poll \circseq \\
    % \t3 HandleAloadEPC(1) \circseq pc := 16 \circseq Poll \circseq HandleInvokevirtualEPC(36) \circseq Poll \circseq \\
    % \t3 OpenOutputStream \circseq Poll \circseq HandleAstoreEPC(3) \circseq pc := 18 \circseq Poll \circseq \\
    % \t3 HandleIconstEPC(0) \circseq pc := 19 \circseq Poll \circseq HandleAstoreEPC(4) \circseq pc := 20 \circseq \\
    % \t3 Poll \circseq pc := 39 \\
    % \t2 {} \cdots {} \\
    % \t2 {} \circelse pc = 8 \circthen HandleDupEPC \circseq pc := 9 \circseq Poll \circseq HandleAconst\_nullEPC \circseq pc := 10 \circseq \\
    % \t3 Poll \circseq HandleInvokespecialEPC(29) \circseq Poll \circseq CCInit \circseq Poll \circseq HandleAstoreEPC(1) \circseq \\
    % \t3 pc := 12 \circseq Poll \circseq HandleAloadEPC(1) \circseq pc := 13 \circseq Poll \circseq \\
    % \t3 HandleInvokevirtualEPC(32) \circseq Poll \circseq OpenInputStream \circseq Poll \circseq HandleAstoreEPC(2) \circseq \\
    % \t3 pc := 15 \circseq Poll \circseq HandleAloadEPC(1) \circseq pc := 16 \circseq Poll \circseq HandleInvokevirtualEPC(36) \circseq Poll \circseq OpenOutputStream \circseq Poll \circseq HandleAstoreEPC(3) \circseq pc := 18 \circseq Poll \circseq HandleIconstEPC(0) \circseq \\
    % \t3 pc := 19 \circseq Poll \circseq HandleAstoreEPC(4) \circseq pc := 20 \circseq Poll \circseq pc := 39 \circseq Poll \circseq \\
    % \t3 HandleAloadEPC(4) \circseq pc := 40 \circseq Poll \circseq HandleIconstEPC(10) \circseq pc := 41 \circseq \\
    % \t3 Poll \circseq \circvar value1, value2 : Word \circspot InterpreterPop2 \circseq \\
    % \t3 pc := \IF value1 \leq value2 \THEN 21 \ELSE 42 \\
    % \t2 {} \circelse pc = 9 \circthen HandleAconst\_nullEPC \circseq pc := 10 \circseq Poll \circseq HandleInvokespecialEPC(29) \circseq Poll \circseq CCInit \circseq Poll \circseq HandleAstoreEPC(1) \circseq pc := 12 \circseq Poll \circseq HandleAloadEPC(1) \circseq pc := 13 \circseq Poll \circseq HandleInvokevirtualEPC(32) \circseq Poll \circseq OpenInputStream \circseq Poll \circseq HandleAstoreEPC(2) \circseq pc := 15 \circseq Poll \circseq HandleAloadEPC(1) \circseq pc := 16 \circseq Poll \circseq HandleInvokevirtualEPC(36) \circseq Poll \circseq OpenOutputStream \circseq Poll \circseq HandleAstoreEPC(3) \circseq pc := 18 \circseq Poll \circseq HandleIconstEPC(0) \circseq \\
    % \t3 pc := 19 \circseq Poll \circseq HandleAstoreEPC(4) \circseq pc := 20 \circseq Poll \circseq pc := 39 \circseq Poll \circseq \\
    % \t3 HandleAloadEPC(4) \circseq pc := 40 \circseq Poll \circseq HandleIconstEPC(10) \circseq pc := 41 \circseq \\
    % \t3 Poll \circseq \circvar value1, value2 : Word \circspot InterpreterPop2 \circseq \\
    % \t3 pc := \IF value1 \leq value2 \THEN 21 \ELSE 42 \\
    % \t2 {} \circelse pc = 10 \circthen HandleInvokespecialEPC(29) \circseq Poll \circseq CCInit \circseq Poll \circseq HandleAstoreEPC(1) \circseq pc := 12 \circseq Poll \circseq HandleAloadEPC(1) \circseq pc := 13 \circseq Poll \circseq HandleInvokevirtualEPC(32) \circseq Poll \circseq OpenInputStream \circseq Poll \circseq HandleAstoreEPC(2) \circseq pc := 15 \circseq Poll \circseq HandleAloadEPC(1) \circseq pc := 16 \circseq Poll \circseq HandleInvokevirtualEPC(36) \circseq Poll \circseq OpenOutputStream \circseq Poll \circseq HandleAstoreEPC(3) \circseq pc := 18 \circseq Poll \circseq HandleIconstEPC(0) \circseq \\
    % \t3 pc := 19 \circseq Poll \circseq HandleAstoreEPC(4) \circseq pc := 20 \circseq Poll \circseq pc := 39 \circseq Poll \circseq \\
    % \t3 HandleAloadEPC(4) \circseq pc := 40 \circseq Poll \circseq HandleIconstEPC(10) \circseq pc := 41 \circseq \\
    % \t3 Poll \circseq \circvar value1, value2 : Word \circspot InterpreterPop2 \circseq \\
    % \t3 pc := \IF value1 \leq value2 \THEN 21 \ELSE 42 \\
    % \t2 {} \circelse pc = 11 \circthen HandleAstoreEPC(1) \circseq pc := 12 \circseq Poll \circseq HandleAloadEPC(1) \circseq \\
    % \t3 pc := 13 \circseq Poll \circseq HandleInvokevirtualEPC(32) \circseq Poll \circseq OpenInputStream \circseq Poll \circseq HandleAstoreEPC(2) \circseq pc := 15 \circseq Poll \circseq HandleAloadEPC(1) \circseq pc := 16 \circseq Poll \circseq HandleInvokevirtualEPC(36) \circseq Poll \circseq OpenOutputStream \circseq Poll \circseq HandleAstoreEPC(3) \circseq pc := 18 \circseq Poll \circseq HandleIconstEPC(0) \circseq \\
    % \t3 pc := 19 \circseq Poll \circseq HandleAstoreEPC(4) \circseq pc := 20 \circseq Poll \circseq pc := 39 \circseq Poll \circseq \\
    % \t3 HandleAloadEPC(4) \circseq pc := 40 \circseq Poll \circseq HandleIconstEPC(10) \circseq pc := 41 \circseq \\
    % \t3 Poll \circseq \circvar value1, value2 : Word \circspot InterpreterPop2 \circseq \\
    % \t3 pc := \IF value1 \leq value2 \THEN 21 \ELSE 42 \\
    % \t2 {} \circelse pc = 12 \circthen HandleAloadEPC(1) \circseq pc := 13 \circseq Poll \circseq HandleInvokevirtualEPC(32) \circseq Poll \circseq OpenInputStream \circseq Poll \circseq HandleAstoreEPC(2) \circseq pc := 15 \circseq Poll \circseq HandleAloadEPC(1) \circseq pc := 16 \circseq Poll \circseq HandleInvokevirtualEPC(36) \circseq Poll \circseq OpenOutputStream \circseq Poll \circseq HandleAstoreEPC(3) \circseq pc := 18 \circseq Poll \circseq HandleIconstEPC(0) \circseq \\
    % \t3 pc := 19 \circseq Poll \circseq HandleAstoreEPC(4) \circseq pc := 20 \circseq Poll \circseq pc := 39 \circseq Poll \circseq \\
    % \t3 HandleAloadEPC(4) \circseq pc := 40 \circseq Poll \circseq HandleIconstEPC(10) \circseq pc := 41 \circseq \\
    % \t3 Poll \circseq \circvar value1, value2 : Word \circspot InterpreterPop2 \circseq \\
    % \t3 pc := \IF value1 \leq value2 \THEN 21 \ELSE 42 \\
    % \t2 {} \circelse pc = 13 \circthen HandleInvokevirtualEPC(32) \circseq Poll \circseq OpenInputStream \circseq Poll \circseq HandleAstoreEPC(2) \circseq pc := 15 \circseq Poll \circseq HandleAloadEPC(1) \circseq pc := 16 \circseq Poll \circseq HandleInvokevirtualEPC(36) \circseq Poll \circseq OpenOutputStream \circseq Poll \circseq HandleAstoreEPC(3) \circseq pc := 18 \circseq Poll \circseq HandleIconstEPC(0) \circseq \\
    % \t3 pc := 19 \circseq Poll \circseq HandleAstoreEPC(4) \circseq pc := 20 \circseq Poll \circseq pc := 39 \circseq Poll \circseq \\
    % \t3 HandleAloadEPC(4) \circseq pc := 40 \circseq Poll \circseq HandleIconstEPC(10) \circseq pc := 41 \circseq \\
    % \t3 Poll \circseq \circvar value1, value2 : Word \circspot InterpreterPop2 \circseq \\
    % \t3 pc := \IF value1 \leq value2 \THEN 21 \ELSE 42 \\
    % \t2 {} \circelse pc = 14 \circthen HandleAstoreEPC(2) \circseq pc := 15 \circseq Poll \circseq HandleAloadEPC(1) \circseq \\
    % \t3 pc := 16 \circseq Poll \circseq HandleInvokevirtualEPC(36) \circseq Poll \circseq OpenOutputStream \circseq Poll \circseq HandleAstoreEPC(3) \circseq pc := 18 \circseq Poll \circseq HandleIconstEPC(0) \circseq \\
    % \t3 pc := 19 \circseq Poll \circseq HandleAstoreEPC(4) \circseq pc := 20 \circseq Poll \circseq pc := 39 \circseq Poll \circseq \\
    % \t3 HandleAloadEPC(4) \circseq pc := 40 \circseq Poll \circseq HandleIconstEPC(10) \circseq pc := 41 \circseq \\
    % \t3 Poll \circseq \circvar value1, value2 : Word \circspot InterpreterPop2 \circseq \\
    % \t3 pc := \IF value1 \leq value2 \THEN 21 \ELSE 42 \\
    % \t2 {} \circelse pc = 15 \circthen HandleAloadEPC(1) \circseq pc := 16 \circseq Poll \circseq HandleInvokevirtualEPC(36) \circseq Poll \circseq OpenOutputStream \circseq Poll \circseq HandleAstoreEPC(3) \circseq pc := 18 \circseq Poll \circseq HandleIconstEPC(0) \circseq \\
    % \t3 pc := 19 \circseq Poll \circseq HandleAstoreEPC(4) \circseq pc := 20 \circseq Poll \circseq pc := 39 \circseq Poll \circseq \\
    % \t3 HandleAloadEPC(4) \circseq pc := 40 \circseq Poll \circseq HandleIconstEPC(10) \circseq pc := 41 \circseq \\
    % \t3 Poll \circseq \circvar value1, value2 : Word \circspot InterpreterPop2 \circseq \\
    % \t3 pc := \IF value1 \leq value2 \THEN 21 \ELSE 42 \\
    % \t2 {} \circelse pc = 16 \circthen HandleInvokevirtualEPC(36) \circseq Poll \circseq OpenOutputStream \circseq Poll \circseq HandleAstoreEPC(3) \circseq pc := 18 \circseq Poll \circseq HandleIconstEPC(0) \circseq \\
    % \t3 pc := 19 \circseq Poll \circseq HandleAstoreEPC(4) \circseq pc := 20 \circseq Poll \circseq pc := 39 \circseq Poll \circseq \\
    % \t3 HandleAloadEPC(4) \circseq pc := 40 \circseq Poll \circseq HandleIconstEPC(10) \circseq pc := 41 \circseq \\
    % \t3 Poll \circseq \circvar value1, value2 : Word \circspot InterpreterPop2 \circseq \\
    % \t3 pc := \IF value1 \leq value2 \THEN 21 \ELSE 42 \\
    % \t2 {} \circelse pc = 17 \circthen HandleAstoreEPC(3) \circseq pc := 18 \circseq Poll \circseq HandleIconstEPC(0) \circseq \\
    % \t3 pc := 19 \circseq Poll \circseq HandleAstoreEPC(4) \circseq pc := 20 \circseq Poll \circseq pc := 39 \circseq Poll \circseq \\
    % \t3 HandleAloadEPC(4) \circseq pc := 40 \circseq Poll \circseq HandleIconstEPC(10) \circseq pc := 41 \circseq \\
    % \t3 Poll \circseq \circvar value1, value2 : Word \circspot InterpreterPop2 \circseq \\
    % \t3 pc := \IF value1 \leq value2 \THEN 21 \ELSE 42 \\
    % \t2 {} \circelse pc = 18 \circthen HandleIconstEPC(0) \circseq pc := 19 \circseq Poll \circseq HandleAstoreEPC(4) \circseq pc := 20 \circseq Poll \circseq pc := 39 \circseq Poll \circseq HandleAloadEPC(4) \circseq pc := 40 \circseq Poll \circseq HandleIconstEPC(10) \circseq pc := 41 \circseq Poll \circseq \circvar value1, value2 : Word \circspot InterpreterPop2 \circseq pc := \IF value1 \leq value2 \THEN 21 \ELSE 42 \\
    % \t2 {} \circelse pc = 19 \circthen HandleAstoreEPC(4) \circseq pc := 20 \circseq Poll \circseq pc := 39 \circseq Poll \circseq HandleAloadEPC(4) \circseq pc := 40 \circseq Poll \circseq HandleIconstEPC(10) \circseq pc := 41 \circseq Poll \circseq \circvar value1, value2 : Word \circspot InterpreterPop2 \circseq pc := \IF value1 \leq value2 \THEN 21 \ELSE 42 \\
    % \t2 {} \circelse pc = 20 \circthen pc := 39 \circseq Poll \circseq HandleAloadEPC(4) \circseq pc := 40 \circseq Poll \circseq HandleIconstEPC(10) \circseq pc := 41 \circseq Poll \circseq \circvar value1, value2 : Word \circspot InterpreterPop2 \circseq pc := \IF value1 \leq value2 \THEN 21 \ELSE 42 \\
    \t2 {} \circelse pc = 21 \circthen HandleAloadEPC(2) \circseq pc := 22 \circseq Poll \circseq (\circvar poppedArgs : \seq Word \circspot \\
    \t4 \lschexpract \exists argsToPop?  == 1 @ InterpreterStackFrameInvoke \rschexpract \circseq \\
    \t4 getClassIDOf!(head~poppedArgs)?cid \then {} \\
    \t5 \circif cid = ConsoleInputClassID \circthen {} \\
    \t6 \lschexpract InterpreterNewStackFrame[\\
    \t7 ConsoleInput/class?, read/methodID?, poppedArgs/methodArgs?] \rschexpract \circseq \\
    \t6 Poll \circseq ConsoleInput\_read \\
    \t5 \circfi) \circseq pc := 23 \circseq Poll \circseq (\circvar poppedArgs : \seq Word \circspot \\
    \t4 \lschexpract \exists argsToPop? == 1 @ InterpreterStackFrameInvoke \rschexpract \circseq \\
    \t4 \lschexpract InterpreterNewStackFrame[\\
    \t5 TPK/class?, f/methodID?, poppedArgs/methodArgs?] \rschexpract \circseq \\
    \t3 Poll \circseq TPK\_f) \circseq pc := 24 \circseq Poll \circseq HandleAstoreEPC(5) \circseq pc := 25 \circseq Poll \circseq \\
    % \t3 HandleAloadEPC(5) \circseq pc := 26 \circseq HandleIconstEPC(400) \circseq pc := 27 \circseq Poll \circseq \\
    % \t3 \circvar value1, value2 : Word \circspot InterpreterPop2EPC \circseq \\
    % \t3 pc := \IF value1 \leq value2 \THEN 32 \ELSE 28 \\
    % \t2 {} \cdots {} \\
    % \t2 {} \circelse pc = 22 \circthen HandleInvokevirtualEPC(40) \circseq Poll \circseq Read \circseq Poll \circseq HandleInvokestaticEPC(46) \circseq Poll \circseq F \circseq Poll \circseq HandleAstoreEPC(5) \circseq pc := 25 \circseq Poll \circseq HandleAloadEPC(5) \circseq pc := 26 \circseq HandleIconstEPC(400) \circseq pc := 27 \circseq Poll \circseq \circvar value1, value2 : Word \circspot InterpreterPop2 \circseq pc := \IF value1 \leq value2 \THEN 32 \ELSE 28 \\
    % \t2 {} \circelse pc = 23 \circthen HandleInvokestaticEPC(46) \circseq Poll \circseq F \circseq Poll \circseq HandleAstoreEPC(5) \circseq pc := 25 \circseq Poll \circseq HandleAloadEPC(5) \circseq pc := 26 \circseq HandleIconstEPC(400) \circseq pc := 27 \circseq Poll \circseq \circvar value1, value2 : Word \circspot InterpreterPop2 \circseq pc := \IF value1 \leq value2 \THEN 32 \ELSE 28 \\
    % \t2 {} \circelse pc = 24 \circthen HandleAstoreEPC(5) \circseq pc := 25 \circseq Poll \circseq HandleAloadEPC(5) \circseq pc := 26 \circseq HandleIconstEPC(400) \circseq pc := 27 \circseq Poll \circseq \circvar value1, value2 : Word \circspot InterpreterPop2 \circseq pc := \IF value1 \leq value2 \THEN 32 \ELSE 28 \\
    % \t2 {} \circelse pc = 25 \circthen HandleAloadEPC(5) \circseq pc := 26 \circseq Poll \circseq HandleIconstEPC(400) \circseq pc := 27 \circseq Poll \circseq \circvar value1, value2 : Word \circspot InterpreterPop2 \circseq pc := \IF value1 \leq value2 \THEN 32 \ELSE 28 \\
    % \t2 {} \circelse pc = 26 \circthen HandleIconstEPC(400) \circseq pc := 27 \circseq Poll \circseq \circvar value1, value2 : Word \circspot InterpreterPop2 \circseq pc := \IF value1 \leq value2 \THEN 32 \ELSE 28 \\
    % \t2 {} \circelse pc = 27 \circthen \circvar value1, value2 : Word \circspot InterpreterPop2 \circseq \\
    % \t3 pc := \IF value1 \leq value2 \THEN 32 \ELSE 28 \\
    % \t2 {} \circelse pc = 28 \circthen HandleAloadEPC(3) \circseq pc := 29 \circseq Poll \circseq HandleIconstEPC(0) \circseq pc := 30 \circseq Poll \circseq HandleInvokevirtualEPC(50) \circseq Poll \circseq Write \circseq Poll \circseq pc := 38 \circseq Poll \circseq HandleAstoreEPC(4) \circseq pc := 39 \circseq Poll \circseq HandleAloadEPC(4) \circseq pc := 40 \circseq Poll \circseq HandleIconstEPC(10) \circseq pc := 41 \circseq Poll \circseq \circvar value1, value2 : Word \circspot InterpreterPop2 \circseq pc := \IF value1 \leq value2 \THEN 21 \ELSE 42 \\
    % \t2 {} \circelse pc = 29 \circthen HandleIconstEPC(0) \circseq pc := 30 \circseq Poll \circseq HandleInvokevirtualEPC(50) \circseq Poll \circseq Write \circseq Poll \circseq pc := 38 \circseq Poll \circseq HandleAstoreEPC(4) \circseq pc := 39 \circseq Poll \circseq HandleAloadEPC(4) \circseq pc := 40 \circseq Poll \circseq HandleIconstEPC(10) \circseq pc := 41 \circseq Poll \circseq \circvar value1, value2 : Word \circspot InterpreterPop2 \circseq pc := \IF value1 \leq value2 \THEN 21 \ELSE 42 \\
    % \t2 {} \circelse pc = 30 \circthen HandleInvokevirtualEPC(50) \circseq Poll \circseq Write \circseq Poll \circseq pc := 38 \circseq Poll \circseq HandleAstoreEPC(4) \circseq pc := 39 \circseq Poll \circseq HandleAloadEPC(4) \circseq pc := 40 \circseq Poll \circseq HandleIconstEPC(10) \circseq pc := 41 \circseq Poll \circseq \circvar value1, value2 : Word \circspot InterpreterPop2 \circseq pc := \IF value1 \leq value2 \THEN 21 \ELSE 42 \\
    % \t2 {} \circelse pc = 31 \circthen pc := 38 \circseq Poll \circseq HandleAstoreEPC(4) \circseq pc := 39 \circseq Poll \circseq HandleAloadEPC(4) \circseq pc := 40 \circseq Poll \circseq HandleIconstEPC(10) \circseq pc := 41 \circseq Poll \circseq \circvar value1, value2 : Word \circspot InterpreterPop2 \circseq pc := \IF value1 \leq value2 \THEN 21 \ELSE 42 \\
    % \t2 {} \circelse pc = 32 \circthen HandleAloadEPC(3) \circseq pc := 33 \circseq Poll \circseq HandleAloadEPC(5) \circseq pc := 34 \circseq Poll \circseq HandleInvokevirtualEPC(50) \circseq Poll \circseq Write \circseq Poll \circseq HandleAloadEPC(4) \circseq pc := 36 \circseq Poll \circseq HandleIconstEPC(1) \circseq pc := 37 \circseq Poll \circseq HandleIaddEPC \circseq pc := 38 \circseq Poll \circseq HandleAstoreEPC(4) \circseq pc := 39  \\
    % \t2 {} \circelse pc = 33 \circthen HandleAloadEPC(5) \circseq pc := 34 \\
    % \t2 {} \circelse pc = 34 \circthen HandleInvokevirtualEPC(50) \\
    % \t2 {} \circelse pc = 35 \circthen HandleAloadEPC(4) \circseq pc := 36 \circseq Poll \circseq HandleIconstEPC(1) \circseq \\
    % \t3 pc := 37 \circseq Poll \circseq HandleIaddEPC \circseq pc := 38 \circseq Poll \circseq HandleAstoreEPC(4) \circseq \\
    % \t3 pc := 39 \\
    % \t2 {} \circelse pc = 36 \circthen HandleIconstEPC(1) \circseq pc := 37 \\
    % \t2 {} \circelse pc = 37 \circthen HandleIaddEPC \circseq pc := 38 \\
    % \t2 {} \circelse pc = 38 \circthen HandleAstoreEPC(4) \circseq pc := 39 \\
    % \t2 {} \circelse pc = 39 \circthen HandleAloadEPC(4) \circseq pc := 40 \circseq Poll \circseq HandleIconstEPC(10) \circseq \\
    % \t3 pc := 41 \circseq Poll \circseq \circvar value1, value2 : Word \circspot InterpreterPop2EPC \circseq \\
    % \t3 pc := \IF value1 \leq value2 \THEN 21 \ELSE 42 \\
    % \t2 \t2 {} \circelse pc = 40 \circthen HandleIconstEPC(10) \circseq pc := 41 \\
    % \t2 {} \circelse pc = 41 \circthen \circvar value1, value2 : Word \circspot InterpreterPop2 \circseq \\
    % \t3 pc := \IF value1 \leq value2 \THEN 21 \ELSE 42 \\
    % \t2 {} \circelse pc = 42 \circthen HandleReturnEPC \\
    \t2 {} \cdots {} \\
    \t2 {} \circelse pc = 43 \circthen TPK\_f \\
    \t2 {} \cdots {} \\
    % \t2 {} \circelse pc = 44 \circthen HandleAloadEPC(0) \circseq pc := 45 \\
    % \t2 {} \circelse pc = 45 \circthen HandleIaddEPC \circseq pc := 46 \\
    % \t2 {} \circelse pc = 46 \circthen HandleAloadEPC(0) \circseq pc := 47 \\
    % \t2 {} \circelse pc = 47 \circthen HandleIaddEPC \circseq pc := 48 \\
    % \t2 {} \circelse pc = 48 \circthen HandleIconstEPC(5) \circseq pc := 49 \\
    % \t2 {} \circelse pc = 49 \circthen HandleIaddEPC \circseq pc := 50 \\
    % \t2 {} \circelse pc = 50 \circthen HandleAreturnEPC \\
    \t2 \circfi \circseq Poll \circseq Running \\
    \t1 \circfi
  \end{circus}
  \vspace{-1cm}
  \begin{circus}
    TPK\_f \circdef HandleAloadEPC(0) \circseq pc := 44 \circseq Poll \circseq \\
    \t1 HandleAloadEPC(0) \circseq pc := 45 \circseq Poll \circseq HandleIaddEPC \circseq pc := 46 \circseq Poll \circseq \\
    \t1 HandleAloadEPC(0) \circseq pc := 47 \circseq Poll \circseq HandleIaddEPC \circseq pc := 48 \circseq Poll \circseq \\
    \t1 HandleIconstEPC(5) \circseq pc := 49 \circseq Poll \circseq HandleIaddEPC \circseq pc := 50 \circseq Poll \circseq \\
    \t1 HandleAreturnEPC
  \end{circus}
  \caption{The $Running$ action after method call resolution}
  \label{method-call-resolution-example-figure}
\end{figure}

Calls to methods with separate actions can then be resolved,
sequencing the method invocation instruction with a call to the
\Circus{} action representing its body and the instructions following
the method call. 
This occurs on line~\ref{algorithm-resolve-method-calls} of the
algorithm, and can be seen in
Figure~\ref{method-call-resolution-example-figure}, which shows our
example after method call resolution has been applied.

The target of each method call can be determined from the parameter to
the method invocation instruction.
This parameter is an index into the constant pool of the current class
that points to a reference to the method being called.
The correct current class for each bytecode instruction is always
known, since the information on the method entries and ends is
contained in the class information, and there is a one-to-one mapping
between classes and blocks of bytecode instructions that form methods.
After the target of the method call has been determined, the
invocation instruction can be sequenced with a call to the
corresponding \Circus{} action.

An example of a resolved method call is the call to $TPK\_f$ at
$pc = 23$, in the sequence of actions beginning at $pc = 21$ in
Figure~\ref{method-call-resolution-example-figure}. 
This comes from resolving the method invocation instruction
$invokestatic~46$.
As can be seen from Figure~\ref{example-model-figure}, the constant
pool index $46$ corresponds to the method identifier for the method
\texttt{f()} of \texttt{TPK}.
The sequence of instructions corresponding to this method is in an
action $TPK\_f$, created in the previous step, on
line~\ref{algorithm-separate-complete-methods} of
Algorithm~\ref{epc-algorithm}.

The semantics for the invocation instruction is expanded to
instantiate the data operations it contains.
These are then sequenced with the method action $TPK\_f$, with the
$Poll$ action inbetween (to allow thread switches before the first
instruction of the called method).
The instructions following the method call are sequenced after it,
with another $Poll$ action (to allow thread switches following the
return from the method).
Method call resolution is described in more detail in
Section~\ref{resolve-method-calls-subsection}, where we define the
\Call{SeparateCompleteMethods}{} and \Call{ResolveMethodCalls}{}
procedures.

\begin{figure}[t!]
  \setlength{\zedindent}{0cm}
  \setlength{\zedtab}{0.4cm}
  \setlength{\zedleftsep}{0.1cm}
  \begin{circus}
    Running \circdef \\
    \t1 \circif frameStack = \emptyset \circthen \Skip \\
    \t1 {} \circelse frameStack \neq \emptyset \circthen {} \\
    \t2 {} \circif pc = 0 \circthen TPK\_APEHinit \\
    \t2 {} \cdots {} \\
    \t2 {} \circelse pc = 7 \circthen TPK\_handleAsyncEvent \\
    \t2 {} \cdots {} \\
    \t2 {} \circelse pc = 43 \circthen TPK\_f \\
    \t2 {} \cdots {} \\
    \t2 \circfi \circseq Poll \circseq Running \\
    \t1 \circfi
  \end{circus}
  \vspace{-1cm}
  \begin{circus}
    TPK\_APEHinit \circdef \\
    \t1 HandleAloadEPC(0) \circseq pc := 1 \circseq Poll \circseq HandleAloadEPC(1) \circseq pc := 2 \circseq Poll \circseq \\
    \t1 HandleAloadEPC(2) \circseq pc := 3 \circseq Poll \circseq HandleAloadEPC(3) \circseq pc := 4 \circseq Poll \circseq \\
    \t1 HandleAloadEPC(4) \circseq pc := 5 \circseq Poll \circseq (\circvar poppedArgs : \seq Word \circspot \\
    \t2 \lschexpract \exists argsToPop? == 6 @ InterpreterStackFrameInvoke \rschexpract \circseq \\
    \t2 \lschexpract InterpreterNewStackFrame[\\
    \t3 AperiodicEventHandler/class?, APEHinit/methodID?, poppedArgs/methodArgs?] \rschexpract \circseq \\
    \t1 Poll \circseq AperiodicEventHandler\_APEHinit) \circseq pc := 6 \circseq Poll \circseq HandleReturnEPC
  \end{circus}
  \vspace{-1cm}
  \begin{circus}
    TPK\_handleAsyncEvent \circdef \\
    \t1 HandleNewEPC(27) \circseq pc := 8 \circseq Poll \circseq HandleDupEPC \circseq pc := 9 \circseq \\
    % \t1 Poll \circseq HandleAconst\_nullEPC \circseq pc := 10 \circseq Poll \circseq (\circvar poppedArgs : \seq Word \circspot \\
    % \t2 \lschexpract \exists argsToPop? == m + 1 @ InterpreterStackFrameInvoke \rschexpract \circseq \\
    % \t3 \lschexpract InterpreterNewStackFrame[\\
    % \t4 ConsoleConnection/class?, CCinit/methodID?, poppedArgs/methodArgs?] \rschexpract) \circseq \\
    % \t1 Poll \circseq ConsoleConnection\_CCinit \circseq pc := 11 \circseq Poll \circseq HandleAstoreEPC(1) \circseq pc := 12 \circseq Poll \circseq \\
    \t1 {} \cdots {} \\
%     \t1 HandleAloadEPC(1) \circseq pc := 13 \circseq Poll \circseq (\circvar poppedArgs : \seq Word \circspot \\
%     \t2 \lschexpract \exists argsToPop? == m + 1 @ InterpreterStackFrameInvoke \rschexpract \circseq \\
%     \t2 getClassIDOf!(head~poppedArgs)?cid \then (\circvar class : Class \circspot \\
%     \t3 \lschexpract ResolveMethod[ConsoleConnectionClassID/classID?, openInputStream/methodID?] \rschexpract \circseq \\
%     \t3 \lschexpract InterpreterNewStackFrame[openInputStream/methodID?, poppedArgs/methodArgs?] \rschexpract)) \circseq \\
%     \t1 Poll \circseq ConsoleConnection\_openInputStream \circseq pc := 14 \circseq Poll \circseq  HandleAstoreEPC(2) \circseq pc := 15 \circseq Poll \circseq \\
%     \t1 HandleAloadEPC(1) \circseq pc := 16 \circseq Poll \circseq (\circvar poppedArgs : \seq Word \circspot \\
%     \t2 \lschexpract \exists argsToPop? == m + 1 @ InterpreterStackFrameInvoke \rschexpract \circseq \\
%     \t2 getClassIDOf!(head~poppedArgs)?cid \then (\circvar class : Class \circspot \\
%     \t3 \lschexpract ResolveMethod[ConsoleConnectionClassID/classID?, openOutputStream/methodID?] \rschexpract \circseq \\
%     \t3 \lschexpract InterpreterNewStackFrame[openOutputStream/methodID?, poppedArgs/methodArgs?] \rschexpract)) \circseq \\    
%     \t1 Poll \circseq ConsoleConnection\_openOutputStream \circseq pc := 17 \circseq Poll \circseq HandleAstoreEPC(3) \circseq pc := 18 \circseq Poll \circseq HandleIconstEPC(0) \circseq
    \t1 pc := 19 \circseq Poll \circseq HandleAstoreEPC(4) \circseq pc := 20 \circseq Poll \circseq pc := 39 \circseq  Poll \circseq (\circmu Y \circspot \\
    \t2 HandleAloadEPC(4) \circseq pc := 40 \circseq Poll \circseq HandleIconstEPC(10) \circseq  pc := 41 \circseq Poll \circseq \\
    \t2 (\circvar value1, value2 : Word \circspot InterpreterPop2 \circseq \\
    \t2 \circif value1 \leq value2 \circthen pc := 21 \circseq Poll \circseq HandleAloadEPC(2) \circseq pc := 22 \circseq Poll \circseq \\
    \t3 {} \cdots {} \\
%     \t2 (\circvar poppedArgs : \seq Word \circspot \lschexpract \exists argsToPop? == m + 1 @ InterpreterStackFrameInvoke \rschexpract \circseq \\
%     \t2 getClassIDOf!(head~poppedArgs)?cid \then (\circvar class : Class \circspot \\
%     \t3 \lschexpract ResolveMethod[ConsoleInputClassID/classID?, read/methodID?] \rschexpract \circseq \\
%     \t3 \lschexpract InterpreterNewStackFrame[read/methodID?, poppedArgs/methodArgs?] \rschexpract)) \circseq \\
%     \t2 Poll \circseq ConsoleInput\_read \circseq pc := 23 \circseq Poll \circseq (\circvar poppedArgs : \seq Word \circspot \\
%     \t2 \lschexpract \exists argsToPop? == m @ InterpreterStackFrameInvoke \rschexpract \circseq \\
%     \t2 (\circvar class : Class \circspot \lschexpract ResolveMethod[TPKClassID/classID?, f/methodID?] \rschexpract \circseq \\
%     \t3 \lschexpract InterpreterNewStackFrame[f/methodID?, poppedArgs/methodArgs?] \rschexpract)) \circseq \\
%     \t2 Poll \circseq TPK\_f  \circseq pc := 24 \circseq Poll \circseq HandleAstoreEPC(5) \circseq pc := 25 \circseq Poll \circseq \\
%     \t2 HandleAloadEPC(5) \circseq pc := 26 \circseq Poll \circseq HandleIconstEPC(400) \circseq pc := 27 \circseq Poll \circseq \circvar value1, value2 : Word \circspot InterpreterPop2 \circseq \\
%     \t2 \circif value1 \leq value2 \circthen pc := 32 \circseq HandleAloadEPC(3) \circseq pc := 33 \circseq Poll \circseq \\
%     \t3 HandleAloadEPC(5) \circseq pc := 34 \circseq Poll \circseq (\circvar poppedArgs : \seq Word \circspot \\
%     \t2 \lschexpract \exists argsToPop? == m + 1 @ InterpreterStackFrameInvoke \rschexpract \circseq \\
%     \t2 getClassIDOf!(head~poppedArgs)?cid \then (\circvar class : Class \circspot \\
%     \t3 \lschexpract ResolveMethod[ConsoleOutputClassID/classID?, write/methodID?] \rschexpract \circseq \\
%     \t3 \lschexpract InterpreterNewStackFrame[write/methodID?, poppedArgs/methodArgs?] \rschexpract)) \circseq \\
%     \t3 Poll \circseq ConsoleOutput\_write \\
%     \t2 {} \circelse value1 > value2 \circthen pc := 28 \circseq HandleAloadEPC(3) \circseq pc := 29 \circseq Poll \circseq \\
%     \t3 HandleIconstEPC(0) \circseq pc := 30 \circseq Poll \circseq (\circvar poppedArgs : \seq Word \circspot \\
%     \t2 \lschexpract \exists argsToPop? == m + 1 @ InterpreterStackFrameInvoke \rschexpract \circseq \\
%     \t2 getClassIDOf!(head~poppedArgs)?cid \then (\circvar class : Class \circspot \\
%     \t3 \lschexpract ResolveMethod[ConsoleOutputClassID/classID?, write/methodID?] \rschexpract \circseq \\
%     \t3 \lschexpract InterpreterNewStackFrame[write/methodID?, poppedArgs/methodArgs?] \rschexpract)) \circseq \\
%     \t3 Poll \circseq ConsoleOutput\_write \\
%     \t2 \circfi \circseq pc := 35 \circseq Poll \circseq HandleAloadEPC(4) \circseq pc := 36 \circseq Poll \circseq HandleIconstEPC(1) \circseq pc := 37 \circseq Poll \circseq HandleIaddEPC \circseq
    \t3 pc := 38 \circseq Poll \circseq HandleAstoreEPC(4) \circseq pc := 39 \circseq Poll \circseq Y \\
    \t2 {} \circelse value1 > value2 \circthen \Skip \\
    \t2 \circfi)) \circseq pc := 42 \circseq Poll \circseq HandleReturnEPC
  \end{circus}
  \caption{The $Running$ action after all the methods are separated}
  \label{final-method-separation-example-figure}
\end{figure}
  
As mentioned previously, these steps are then repeated, in the loop
beginning at line~\ref{algorithm-method-loop} of
Algorithm~\ref{epc-algorithm} to introduce the loops and conditionals
in methods that have unresolved method calls in the middle of loops
and conditionals.
Afterwards, those methods can be separated out and this loop,
conditional and method resolution repeated until every method has been
separated out in this way.
This always terminates, since we do not allow recursion, and so there
are no loops in the dependencies between methods.

The $Running$ action of our example at the end of the loop in
Algorithm~\ref{epc-algorithm}, when all loops and conditionals have
been introduced, all the methods have been separated out, and all
method calls have been resolved, is shown in
Figure~\ref{final-method-separation-example-figure}.
At this point, the choice over the $pc$ value maps entry points of
methods onto the actions representing those methods, with the other
$pc$ values now redundant.

\begin{figure}[t!]
  \setlength{\zedindent}{0cm}
  \begin{circusaction}
    ExecuteMethod \circdef \\
    \t1 \circval classID : ClassID; \circval methodID : MethodID; \circval methodArgs : \seq Word \circspot \\
    \t1 \circif (classID, methodID) = (TPKClassID, APEHinit) \circthen {} \\
    \t2 InterpreterNewStackFrame[TPK/class?] \circseq TPK\_APEHinit \\
    \t1 {} \circelse (classID, methodID) = (TPKClassID, handleAsyncEvent) \circthen {} \\
    \t2 InterpreterNewStackFrame[TPK/class?] \circseq TPK\_handleAsyncEvent \\
    \t1 {} \circelse (classID, methodID) = (TPKClassID, f) \circthen {} \\
    \t2 InterpreterNewStackFrame[TPK/class?] \circseq TPK\_f \\
    \t1 {} \cdots {} \\
    \t1 \circfi
  \end{circusaction}
  \vspace{-0.5cm}
  \begin{circusaction}
    MainThread \circdef \\
    \t1 setStack?t \prefixcolon (t = thread) ?stack \then frameStackID := Initialised~stack \circseq \circmu X \circspot \\
    \t1 \circblockbegin
    executeMethod? t \prefixcolon (t = thread)?c?m?a \then ExecuteMethod(c,m,a) \circseq Poll \circseq X \\
    {} \extchoice {} \\
    CEEswitchThread?from?to \prefixcolon (from = thread) \then Blocked \circseq X
    \circblockend
  \end{circusaction}
  \vspace{-0.5cm}
  \begin{circusaction}
    Started \circdef \\
    \t1 \circblockbegin
    executeMethod? t \prefixcolon (t = thread)?c?m?a \then ExecuteMethod(c,m,a) \circseq Poll \circseq \\
    \t1 \circblockbegin
    continue?t \prefixcolon (t = thread) \then Started \\
    {} \extchoice {} \\
    endThread?t \prefixcolon (t = thread) \then \Skip
    \circblockend \\
    {} \extchoice {} \\
    CEEswitchThread?from?to \prefixcolon (from = thread) \then Blocked \circseq Started \\
    {} \extchoice {} \\
    endThread?t \prefixcolon (t = thread) \then \Skip
    \circblockend \circseq \\
    \t1 removeThreadMemory!thread \then CEEremoveThread!thread \\
    \t1 {} \then CEEswitchThread?from?to \prefixcolon (from = thread) \then NotStarted
  \end{circusaction}
  \caption{The $ExecuteMethod$, $MainThread$, and $Started$ actions
    after main action refinement}
  \label{refine-main-actions-example-figure}
\end{figure}

\begin{figure}
  \setlength{\zedtab}{0.4cm}
  \setlength{\zedindent}{0pt}
  \setlength{\zedleftsep}{0pt}
  \setlength{\abovedisplayskip}{0pt}
  \setlength{\belowdisplayskip}{0pt}
  \setlength{\abovedisplayshortskip}{0pt}
  \setlength{\belowdisplayshortskip}{0pt}
  \begin{circus}
    TPK\_handleAsyncEvent \circdef \\
    \t1 HandleNewEPC(27) \circseq Poll \circseq HandleDupEPC \circseq Poll \circseq  HandleAconst\_nullEPC \circseq Poll \circseq \\
    \t1 (\circvar poppedArgs : \seq Word \circspot \\
    \t2 \lschexpract \exists argsToPop? == 2 @ InterpreterStackFrameInvoke \rschexpract \circseq \\
    \t2 \lschexpract InterpreterNewStackFrame[\\
    \t3 ConsoleConnection/class?, CCinit/methodID?, poppedArgs/methodArgs?] \rschexpract \circseq Poll \circseq \\
    \t1 ConsoleConnection\_CCinit) \circseq Poll \circseq HandleAstoreEPC(1) \circseq Poll \circseq HandleAloadEPC(1) \circseq \\
    \t1 {} \cdots {} \\
    % \t1 Poll \circseq (\circvar poppedArgs : \seq Word \circspot \lschexpract \exists argsToPop? == m + 1 @ InterpreterStackFrameInvoke \rschexpract \circseq \\
    % \t1 getClassIDOf!(head~poppedArgs)?cid \then \lschexpract InterpreterNewStackFrame[ \\
    % \t2 ConsoleConnection/class?, openInputStream/methodID?, poppedArgs/methodArgs?] \rschexpract) \circseq \\
    % \t1 Poll \circseq ConsoleConnection\_openInputStream \circseq Poll \circseq  HandleAstoreEPC(2) \circseq Poll \circseq \\
    % \t1 HandleAloadEPC(1) \circseq Poll \circseq (\circvar poppedArgs : \seq Word \circspot \\
    % \t1 \lschexpract \exists argsToPop? == m + 1 @ InterpreterStackFrameInvoke \rschexpract \circseq \\
    % \t1 getClassIDOf!(head~poppedArgs)?cid \then \lschexpract InterpreterNewStackFrame[\\
    % \t2 ConsoleConnection/class?, openOutputStream/methodID?, poppedArgs/methodArgs?] \rschexpract) \circseq \\
    % \t1 Poll \circseq ConsoleConnection\_openOutputStream \circseq Poll \circseq HandleAstoreEPC(3) \circseq \\
    \t1 Poll \circseq HandleIconstEPC(0) \circseq Poll \circseq HandleAstoreEPC(4) \circseq Poll \circseq Poll \circseq (\circmu Y \circspot \\
    \t2 HandleAloadEPC(4) \circseq Poll \circseq HandleIconstEPC(10) \circseq Poll \circseq \\
    \t2 (\circvar value1, value2 : Word \circspot InterpreterPop2 \circseq \\
    \t2 \circif value1 \leq value2 \circthen Poll \circseq HandleAloadEPC(2) \circseq Poll \circseq \\
    \t3 (\circvar poppedArgs : \seq Word \circspot \\
    \t4 \lschexpract \exists argsToPop? == 1 @ InterpreterStackFrameInvoke \rschexpract \circseq \\
    \t4 getClassIDOf!(head~poppedArgs)?cid \then {} \\
    \t5 \circif cid = ConsoleInputClassID \circthen {} \\
    \t6 \lschexpract InterpreterNewStackFrame[\\
    \t7 ConsoleInput/class?, read/methodID?, poppedArgs/methodArgs?] \rschexpract \circseq \\
    \t6 Poll \circseq ConsoleInput\_read \\
    \t5 \circfi) \circseq \\
    \t3 (\circvar poppedArgs : \seq Word \circspot \\
    \t4 \lschexpract \exists argsToPop? == 1 @ InterpreterStackFrameInvoke \rschexpract \circseq \\
    \t4 \lschexpract InterpreterNewStackFrame[\\
    \t5 TPK/class?, f/methodID?, poppedArgs/methodArgs?] \rschexpract \circseq \\
    \t3 Poll \circseq TPK\_f) \circseq Poll \circseq HandleAstoreEPC(5) \circseq Poll \circseq HandleAloadEPC(5) \circseq \\
    \t3 {} \cdots {} \\
    % \t3 Poll \circseq HandleIconstEPC(400) \circseq Poll \circseq \circvar value1, value2 : Word \circspot InterpreterPop2 \circseq \\
    % \t3 \circif value1 \leq value2 \circthen HandleAloadEPC(3) \circseq Poll \circseq HandleAloadEPC(5) \circseq Poll \circseq \\
    % \t4 (\circvar poppedArgs : \seq Word \circspot \lschexpract \exists argsToPop? == m + 1 @ InterpreterStackFrameInvoke \rschexpract \circseq \\
    % \t4 getClassIDOf!(head~poppedArgs)?cid \then \lschexpract InterpreterNewStackFrame[ \\
    % \t5 ConsoleOutput/class?, write/methodID?, poppedArgs/methodArgs?] \rschexpract)) \circseq \\
    % \t4 Poll \circseq ConsoleOutput\_write \\
    % \t3 {} \circelse value1 > value2 \circthen HandleAloadEPC(3) \circseq Poll \circseq HandleIconstEPC(0) \circseq Poll \circseq \\
    % \t4 (\circvar poppedArgs : \seq Word \circspot \lschexpract \exists argsToPop? == m + 1 @ InterpreterStackFrameInvoke \rschexpract \circseq \\
    % \t4 getClassIDOf!(head~poppedArgs)?cid \then \lschexpract InterpreterNewStackFrame[ \\
    % \t5 ConsoleOutput/class?, write/methodID?, poppedArgs/methodArgs?] \rschexpract)) \circseq \\
    % \t4 Poll \circseq ConsoleOutput\_write \\
    % \t3 \circfi \circseq Poll \circseq HandleAloadEPC(4) \circseq Poll \circseq HandleIconstEPC(1) \circseq Poll \circseq HandleIaddEPC \circseq \\
    \t3 Poll \circseq HandleAstoreEPC(4) \circseq Poll \circseq Y \\
    \t2 {} \circelse value1 > value2 \circthen \Skip \\
    \t2 \circfi)) \circseq Poll \circseq HandleReturnEPC
  \end{circus}
  \caption{The $TPK\_handleAsyncEvent$ action after $pc$ has been eliminated from the state}
  \label{pc-elimination-HandleAsyncEvent-example-figure}
\end{figure}

The next step is then to eliminate these redundant paths and remove
the dependency on $pc$ to select the method action.
%TODO: make sure "at line" is used consistently
This occurs at line~\ref{algorithm-refine-main-actions} of
Algorithm~\ref{epc-algorithm}, in which the $Started$ and $MainThread$
actions are refined to replace the $Running$ action with an
$ExecuteMethod$ action that contains a choice of method action based
on the method and class identifiers of the method.
This can be seen in Figure~\ref{refine-main-actions-example-figure},
which shows the $ExecuteMethod$ action corresponding to our example,
and the refined $MainThread$ and $Started$ actions that reference it.
We describe this refinement in more detail in
Section~\ref{refine-main-actions-subsection}, where we define the
\Call{RefineMainActions}{} procedure.

When all of the previous steps are completed, reliance on $pc$ to
determine control flow has been completely removed.
The $pc$ state component can then be removed in a simple data
refinement that also removes all the assignments to $pc$, resulting in
the $TPK\_handleAsyncEvent$ action shown in
Figure~\ref{pc-elimination-HandleAsyncEvent-example-figure}.
The data refinement to remove $pc$ is applied at the end of the
algorithm, on line~\ref{algorithm-remove-pc-from-state}, and is
described in more detail in
Section~\ref{remove-pc-from-state-subsection}, where we define the
\Call{RemovePCFromState}{} procedure.

The remaining instruction handling actions then only affect the stack,
the removal of which is the concern of the next stage of the
compilation strategy.

We now proceed to describe each of the steps of
Algorithm~\ref{epc-algorithm} in more detail.


%\FloatBarrier

\subsection{Expand Bytecode}
\label{expand-bytecode-subsection}

Before the control flow can be introduced, the bytecode instructions
provided in the $bc$ parameter to $Thr$ must be expanded to allow
consideration of their semantics.
This is achieved using the procedure shown in
Algorithm~\ref{expand-bytecode-algorithm}.
\begin{algorithm}[thp]
  \begin{algorithmic}[1]
    \State \ApplyFor{Rule~[\nameref{pc-expansion-rule}]}{$bc$}
    \label{algorithm-introduce-choice-over-pc}
    \For{$(handleActionName, handleActionBody) \gets$ \Call{Handle*EPCActions}{}}
    \label{algorithm-introduce-handleEPC-actions-loop}
    \State \ApplyFor{Law~[\nameref{action-intro-law}]}{$handleActionName$, $handleActionBody$}
    \EndFor
    \For{$i \gets \dom bc$}
    \label{algorithm-expand-bytecode-loop}
    \State \ApplyFor{Rule~[\nameref{HandleInstruction-refinement-rule}]}{$bc$, $i$}
    \label{algorithm-HandleInstruction-refinement}
    \Try
    \label{algorithm-expand-bytecode-try-block-begin}
    \State \ApplyFor{Rule~[\nameref{CheckSynchronizedReturn-synchronized-refinement-rule}]}{$i$}
    \label{algorithm-CheckSynchronisedReturn-synchronized-refinement}
    \State \ApplyFor{Rule~[\nameref{CheckSynchronizedReturn-nonsynchronized-refinement-rule}]}{$i$}
    \label{algorithm-CheckSynchronisedReturn-nonsynchronized-refinement}
    \EndTry
    \EndFor
  \end{algorithmic}
  \caption{\Call{ExpandBytecode}{}}
  \label{expand-bytecode-algorithm}
\end{algorithm}
It begins on line~\ref{algorithm-introduce-choice-over-pc} by applying
Rule~[\nameref{pc-expansion-rule}], shown in
Figure~\ref{pc-expansion-rule-figure}.
It introduces a choice over all the possible values of $pc$ in the
domain of $bc$ at the $HandleInstruction$ action in $Running$.
This does not affect the behaviour of $HandleInstruction$, because it
behaves as $\Chaos$ when $pc$ is outside the domain of $bc$.
We write $HandleInstruction$ with a $bc$ subscript to indicate that it
makes use of $bc$, which is a parameter of the $Thr$ process in which
$HandleInstruction$ occurs.
\begin{figure}[thp]
\begin{restatable}[$pc$-expansion]{crule}{PCExpansionRule}
  \label{pc-expansion-rule}
  Given $bc : ProgramAddress \pfun Bytecode$,
  \begin{circus}
    HandleInstruction_{bc} = \circif {} \circelse_{i\in\dom bc} pc = i \then HandleInstruction_{bc} \circfi
  \end{circus}
\end{restatable}
\caption{Rule~[\nameref{pc-expansion-rule}]}
\label{pc-expansion-rule-figure}
\end{figure}
The proof of this rule and others can be found in
Appendix~\ref{compilation-rules-proofs-appendix}.
After applying Rule~[\nameref{pc-expansion-rule}], we operate on the
occurrence of $HandleInstruction$ at each branch of the conditional at
line~\ref{algorithm-expand-bytecode-loop}.
We apply Rule~[\nameref{HandleInstruction-refinement-rule}], shown in
Figure~\ref{HandleInstruction-refinement-rule-figure}, on
line~\ref{algorithm-HandleInstruction-refinement} to refine each
occurrence to a more specific form that is easier to operate on during
the rest of the strategy.
These new actions are determined from the bytecode instruction in $bc$
at each $pc$ value by applying a syntactic function $handleAction$,
which is defined by Table~\ref{handle-action-table}.
\begin{figure}[thp]
\begin{restatable}[$HandleInstruction$-refinement]{crule}{HandleInstructionRefinementRule}
  \label{HandleInstruction-refinement-rule}
  Given $i : ProgramAddress$, if $i \in \dom bc$ then,
  \begin{circus}
    \begin{array}{l}
      \circif {} \cdots {} \\
      {} \circelse pc = i \then HandleInstruction_{bc} \\
      {} \cdots {} \\
      \circfi
    \end{array}
    \circrefines_A
    \begin{array}{l}
      \circif {} \cdots {} \\
      {} \circelse pc = i \then handleAction(bc~i) \\
      {} \cdots {} \\
      \circfi
    \end{array}
  \end{circus}
  where $handleAction$ is a syntactic function defined by
  Table~\ref{handle-action-table}.
\end{restatable}
\caption{Rule~[\nameref{HandleInstruction-refinement-rule}]}
\label{HandleInstruction-refinement-rule-figure}
\end{figure}
\begin{table}
  \centering
  \begin{tabular}{lp{8.5cm}}
    \hline
    Bytecode ($bc~i$) & Action ($handleAction(bc~i)$) \\
    \hline
    $aconst\_null$ & $HandleAconst\_nullEPC \circseq pc := i+1$ \\
    $dup$ & $HandleDupEPC \circseq pc := i+1$ \\
    $aload~lvi$ & $HandleAloadEPC(lvi) \circseq pc := i+1$ \\
    $astore~lvi$ & $HandleAstoreEPC(lvi) \circseq pc := i+1$ \\
    $iadd$ & $HandleIaddEPC \circseq pc := i+1$ \\
    $iconst~n$ & $HandleIconstEPC(n) \circseq pc := i+1$ \\
    $ineg$ & $HandleInegEPC \circseq pc := i+1$ \\
    $goto~ofst$ & $pc := i+ofst$ \\
    $if\_icmple~ofst$ & $\circvar value1, value2 : Word \circspot$ \endgraf
                        \t1 $\lschexpract InterpreterPopEPC \rschexpract \circseq$ \endgraf
                        \t1 $\lschexpract InterpreterPopEPC \rschexpract \circseq$ \endgraf
                         \t1 $pc := \IF value1 \leq value2 \THEN i+ofst \ELSE i+1$ \\
    $areturn$ & $CheckSynchronizedReturn \circseq HandleAreturnEPC$ \\
    $return$ & $CheckSynchronizedReturn \circseq HandleReturnEPC$ \\
    $new~cpi$ & $HandleNewEPC(cpi) \circseq pc := i+1$ \\
    $getfield~cpi$ & $HandleGetfieldEPC(cpi) \circseq pc := i+1$ \\
    $putfield~cpi$ & $HandlePutfieldEPC(cpi) \circseq pc := i+1$ \\
    $getstatic~cpi$ & $HandleGetstaticEPC(cpi) \circseq pc := i+1$ \\
    $putstatic~cpi$ & $HandlePutstaticEPC(cpi) \circseq pc := i+1$ \\
    $invokevirtual~cpi$ & $\{pc = i\} \circseq HandleInvokevirtualEPC(cpi)$ \\
    $invokespecial~cpi$ & $\{pc = i\} \circseq HandleInvokespecialEPC(cpi)$ \\
    $invokestatic~cpi$ & $\{pc = i\} \circseq HandleInvokestaticEPC(cpi)$ \\
    \hline
  \end{tabular}
  \caption{The syntactic function $handleAction$}
  \label{handle-action-table}
\end{table}
The actions generated by $handleAction$ use new actions for handling
the individual bytecode instructions.
These are similar to the actions used to define $HandleInstruction$
(e.g.\ $HandleDup$, $HandleAload$ etc.), which we refer to as
$Handle*$ actions.
We name the new actions used by $handleAction$ by appending $EPC$ to
the names of the $Handle*$ actions they are based on, and we refer to
them as $Handle{*}EPC$ actions.
The $Handle{*}EPC$ actions are introduced in the for loop on
line~\ref{algorithm-introduce-handleEPC-actions-loop} of
Algorithm~\ref{expand-bytecode-algorithm}, before the application of
Rule~[\nameref{HandleInstruction-refinement-rule}], by application of
Law~[\nameref{action-intro-law}], which introduces unused actions to
processes.
These are actions of a fixed form, described below, so we can
introduce them directly.

In addition to the $Handle{*}EPC$ actions, the actions output from
$handleAction$ also include $pc$ updates extracted from the $Handle*$
actions.
The output from $handleAction$ for the $goto$ and $if\_icmple$
instructions consists solely of a $pc$ update with no $Handle{*}EPC$
actions, since updating the value of pc is the main effect of those
instructions.

The differences between the $Handle{*}EPC$ actions and the $Handle*$
actions on which they are based are explained using the $HandleAstore$
action as an example.
We recall that it is defined as shown below.
\begin{circusaction}
  HandleAstore \circdef \lcircguard pc \in \dom bc \land bc~pc \in \ran astore \rcircguard \circguard \\
  \t1 \circvar variableIndex : \nat \circspot variableIndex := (astore\inv)~(bc~pc) \circseq \lschexpract InterpreterAstore \rschexpract
\end{circusaction}
Its corresponding $Handle{*}EPC$ action, $HandleAstoreEPC$, is shown
below.
\begin{circusaction}
  HandleAstoreEPC \circdef \circval variableIndex : \nat \circspot \lschexpract InterpreterAstoreEPC \rschexpract
\end{circusaction}
The first difference of $HandleAstoreEPC$ from $HandleAstore$ is that
it is not guarded by the condition on the value of $bc$ at the current
$pc$ value.
The choice that such guards mediate is collapsed by
Rule~[\nameref{HandleInstruction-refinement-rule}], since the value of
$bc$ at a given $pc$ value is determined by the supplied $bc$
parameter of $Thr$.

The second difference is that the parameters of the bytecode
instructions are transferred to become parameters of the
$Handle{*}EPC$ actions, so $HandleAstoreEPC$ has a $variableIndex$
parameter.
This corresponds to the $variableIndex$ variable in $HandleAstore$,
which is used to store the value extracted from the $astore$
instruction.
This transformation is, of course, not performed for instructions that
do not take parameters.
This transformation is standard in the context of a call to a
parametrised action.

Finally, the schema $InterpreterAstore$ is replaced with a schema
$InterpreterAstoreEPC$, which does not affect $pc$, since
Rule~[\nameref{HandleInstruction-refinement-rule}] extracts the
updates to $pc$ from the $Handle{*}$ actions.
The $pc$ updates are not removed in the case of the actions for
handling method invocation and return, where the $pc$ updates are
closely connected to the operations on the stack and require special
handling.
Instead, an assumption on the value of $pc$ is introduced for the
method invocation handling actions, since the $pc$ information is used
in setting the return address.
We discuss how we operate on the method invocation and return handling
actions in Section~\ref{resolve-method-calls-subsection}.

We also note that the $CheckSynchronizedReturn$ action is moved
outside the $Handle{*}EPC$ actions handling return instructions.
This is so that this action can be removed, since we have sufficient
information to determine whether the method is synchronized or not.
This is handled on
lines~\ref{algorithm-CheckSynchronisedReturn-synchronized-refinement}
and~\ref{algorithm-CheckSynchronisedReturn-synchronized-refinement} of
Algorithm~\ref{expand-bytecode-algorithm}, by the application of
Rule~[\nameref{CheckSynchronizedReturn-synchronized-refinement-rule}]
and
Rule~[\nameref{CheckSynchronizedReturn-nonsynchronized-refinement-rule}].
These rules are applied in a try block, beginning on
line~\ref{algorithm-expand-bytecode-try-block-begin}, which tries to
apply each rule in turn, stopping when one succeeds.

Rule~[\nameref{CheckSynchronizedReturn-synchronized-refinement-rule}]
matches a branch of the choice in $Running$ corresponding to a given
$pc$ value, $i$, which begins with a $CheckSynchronizedReturn$ action.
This rule is applied whenever the unique class, $c$, and method, $m$,
in which $i$ occurs are such that $m$ is synchronized and not static
in $c$.
% Recall that we do not apply synchronisation to static methods,
% following the approach of icecap.
The rule refines $CheckSynchronizedReturn$ to a communication with the
$Launcher$ on the $releaseLock$ channel, instructing it to release the
lock on the \texttt{this} object.
Rule~[\nameref{CheckSynchronizedReturn-nonsynchronized-refinement-rule}]
is similar, but applies when $m$ is static or not synchronized in $c$,
and eliminates $CheckSynchronizedReturn$.

\begin{figure}[th]
  \begin{restatable}[$CheckSynchronizedReturn$-sync-refinement]{crule}{CheckSynchronizedReturnSynchronizedRefinementRule}
    \label{CheckSynchronizedReturn-synchronized-refinement-rule}
    Given $i : ProgramAddress$,
    \setlength{\zedindent}{0.5cm}
    \setlength{\zedtab}{0.5cm}
    \begin{circus}
      \begin{array}{l}
        \circif {} \cdots {} \\
        {} \circelse pc = i \circthen {} \\
        \t1 CheckSynchronizedReturn \circseq A \\
        {} \cdots {} \\
        \circfi
      \end{array}
      \circrefines_A
      \begin{array}{l}
        \circif {} \cdots {} \\
        {} \circelse pc = i \circthen {} \\
        \t1 releaseLock!((last~frameStack).localVariables~1) \\
        \t1 \then releaseLockRet \then \Skip \circseq A \\
        {} \cdots {} \\
        \circfi
      \end{array}
    \end{circus}
    provided
    \begin{displaymath}
      \exists c : Class; m : MethodID | \\
      \t1 c \in \ran cs \land m \in \dom c.methodEntry @ \\
      \t1 i \in c.methodEntry~m \upto c.methodEnd~m \land \\
      \t1 m \in c.synchronisedMethods \land m \notin c.staticMethods
  \end{displaymath}
  \end{restatable}
  \caption{Rule~[\nameref{CheckSynchronizedReturn-synchronized-refinement-rule}]}
  \label{CheckSynchronizedReturn-synchronized-refinement-rule-figure}
\end{figure}

At the end of Algorithm~\ref{expand-bytecode-algorithm}, our example
has the form shown earlier in
Figure~\ref{bytecode-expansion-example-figure}.
After the bytecode semantics is expanded in the $Running$ action by
this step, the control flow that corresponds to each $pc$ update can
be introduced.
This is dicussed in the next section.

\subsection{Introduce Sequential Composition}
\label{introduce-forward-sequence-subsection}

\begin{algorithm}
  \begin{algorithmic}[1]
    \State $cfg \gets$ \Call{MakeControlFlowGraph}{}
    \label{algorithm-make-control-flow-graph}
    \For{$node \gets cfg$}
    \label{algorithm-sequence-cfg-loop}
    \While{\Call{HasSimpleSequence}{$node$}}
    \label{algorithm-forward-sequence-condition}
    \State \ApplyFor{Rule~[\nameref{sequence-introduction-rule}]}{$node$}
    \label{algorithm-forward-sequence-application}
    \EndWhile
    \EndFor
  \end{algorithmic}
  \caption{IntroduceSequentialComposition}
  \label{introduce-forward-sequence-algorithm}
\end{algorithm}
The simplest control flows to introduce are those of instructions
where execution continues at the next program counter value.
These control flows are introduced as shown in
Algorithm~\ref{introduce-forward-sequence-algorithm}, which defines
the \Call{IntroduceSequentialComposition}{} procedure.
The algorithm constructs a control flow graph for each method in the
program, as specified on line~\ref{algorithm-make-control-flow-graph}.
Since the introduction of sequential composition does not depend on
the relationships between methods, the control flow graph is
constructed as a disconnected graph containing the control flow of all
the methods in the program.
The nodes in this graph correspond to the branches in the choice over
$pc$ values introduced in the previous section.

We construct the control flow graph by starting at the entry point for
each method and following the $pc$ update at the end of each node,
introducing an edge in the process.
For method call instructions, we introduce an edge to the node for the
next $pc$ value, as if the instruction were replaced with
$pc := pc + 1$.
This is consistent with how method calls are handled later in the
strategy, since execution resumes at the next instruction after the
called method returns.
We do not add an edge from a return instruction, since no further
instructions are executed in the method after a return instruction.
We finish construction of the control graph for a method when there
are no further edges to add.
This terminates since there are only finitely many instructions in a
method so edges for all the reachable nodes will eventually be added.


The control flow graph for our example is shown in
Figure~\ref{example-control-flow-graph-figure}.
We label the nodes of the graph with the $pc$ values of the
instructions at the nodes.
Due to our assumptions about the source bytecode, the subgraph
corresponding to each method's control flow is a structured graph as
defined in Section~\ref{compilation-assumptions-section}.

\begin{figure}
  \begin{center}
    \footnotesize
    \begin{tikzpicture}[every new ->/.style={-latex}]
      % \node (start) at (0,0) {};
      \foreach \x in {0,...,6}  { \node (\x) at (\x, 1cm) {\x}; }
      \foreach \x in {7,...,20}  { \node (\x) at (0.85*\x - 0.85*7,0) {\x}; }
      \foreach \x in {21,...,27} { \node (\x) at (0.85*\x - 0.85*21, -2cm) {\x}; }
      \foreach \x in {28,...,31} { \node (\x) at (0.85*\x - 0.85*21, -1cm) {\x}; }
      \foreach \x in {32,...,34} { \node (\x) at (0.85*\x - 0.85*25, -3cm) {\x}; }
      \foreach \x in {35,...,42} { \node (\x) at (0.85*\x - 0.85*24.3, -2cm) {\x}; }
      \foreach \x in {43,...,50} { \node (\x) at (-43+\x, -4.5cm) {\x}; }

      \graph{ (7) -> (8) -> (9) -> (10) -> (11) -> (12) ->
        (13) -> (14) -> (15) -> (16) -> (17) -> (18) -> (19) -> (20)
        -> (39) -> (40) -> (41) -> (42); (21) -> (22) -> (23) -> (24)
        -> (25) -> (26) -> (27) -> {
          (28) -> (29) -> (30) -> (31);
          (32) -> (33) -> (34);
        } -> (35) -> (36) -> (37) -> (38) -> (39);
      };

      \graph{(0) -> (1) -> (2) -> (3) -> (4) -> (5) -> (6)};
      \graph{(43) -> (44) -> (45) -> (46) -> (47) -> (48) -> (49) -> (50)};

      \draw[-latex] (41) edge[out=270,in=270,looseness=0.35] (21);
    \end{tikzpicture}
  \end{center}
  \caption{Control flow graph for our example program}
  \label{example-control-flow-graph-figure}
\end{figure}

After the control flow graph is constructed, we consider each node in
turn, as specified by the for loop on
line~\ref{algorithm-sequence-cfg-loop}.
As mentioned earlier, we require a node to have only a single outgoing
edge and its target to have only a single incoming edge in order for
it to be considered for the introduction of sequential composition.
The reason for this is that nodes with two outgoing edges are points
at which conditionals should be introduced.
Such nodes in our example are the nodes for $pc$ values $27$ and $41$,
which represent the start of conditionals.
Likewise, nodes with multiple incoming edges represent points at which
a more complex control flows occur.
For our example, such nodes include $39$, which is the start of a
loop, and $35$, which is the end of a conditional.
These prevent introduction of sequential composition for the $pc$
values $20$, $31$, $34$, and $38$, since the targets of those nodes
are nodes $35$ and $39$.

For a node that meets the above requirement and is not a method call,
we can introduce sequential composition at that node by applying
Rule~[\nameref{sequence-introduction-rule}], on
line~\ref{algorithm-forward-sequence-application} of the algorithm.
This rule works by unrolling the loop in $Running$ to sequence an
instruction at $pc$ value $i$ with the instruction that is executed
after it, inserting $Poll$ inbetween.
It is required that the $pc$ value of the node's target, $j$, not be
the same as $i$, since that would introduce a loop, rather than a
sequential composition.
Also, the sequence of instructions at the node, $A$, must not affect
the non-emptiness of the $frameStack$ to ensure that the choice at the
start of the main loop in $Running$ can be resolved.
\begin{restatable}[sequence-intro]{crule}{SequenceIntroductionRule}
  \label{sequence-introduction-rule}
  Given $i : ProgramAddress$, if $i \neq j$ and \def\zedindent{0.25cm}
  \begin{circus}
    \{frameStack \neq \emptyset\} \circseq A \\
    {} = {} \\
    \{frameStack \neq \emptyset\} \circseq A \circseq \{frameStack \neq \emptyset\}
  \end{circus}
  then,
  \begin{circus}
    \begin{array}{l}
      \circmu X \circspot \\
      \t1 \circif frameStack = \emptyset \circthen \Skip \\
      \t1 {} \circelse frameStack \neq \emptyset \circthen {} \\
      \t2 \circif {} \cdots {} \\
      \t2 {} \circelse pc = i \circthen A \circseq pc := j \\
      \t2 {} \cdots {} \\
      \t2 {} \circelse pc = j \circthen B \\
      \t2 {} \cdots {} \\
      \t2 \circfi \circseq Poll \circseq X \\
      \t1 \circfi
    \end{array}
    \circrefines_A
    \begin{array}{l}
      \circmu X \circspot \\
      \t1 \circif frameStack = \emptyset \circthen \Skip \\
      \t1 {} \circelse frameStack \neq \emptyset \circthen {} \\
      \t2 \circif {} \cdots {} \\
      \t2 {} \circelse pc = i \circthen {} \\
      \t3 A \circseq pc := j \circseq Poll \circseq B \\
      \t2 {} \cdots {} \\
      \t2 {} \circelse pc = j \circthen B \\
      \t2 {} \cdots {} \\
      \t2 \circfi \circseq Poll \circseq X \\
      \t1 \circfi
    \end{array}
  \end{circus}
\end{restatable}

Since Rule~[\nameref{sequence-introduction-rule}] pulls two nodes
together, we can continue to introduce sequential composition at a
node after the first application of
Rule~[\nameref{sequence-introduction-rule}], until that node no longer
satisfies the conditions for introducing sequential composition.
This is specified by the while loop at
line~\ref{algorithm-forward-sequence-condition} of the algorithm.
This means the control flow graph is updated as
Rule~[\nameref{sequence-introduction-rule}] is applied, to take into
account the merging of nodes.
Since there are finitely many nodes, the merging of nodes eventually
results in a graph in which no further sequential compositions can be
introduced and so the loop terminates.

The resulting control flow graph after introduction of sequential
composition has been performed at every point is shown in
Figure~\ref{example-control-flow-graph-after-sequence-introduction-figure}.
We note that this graph is still a union of structured graphs since
merging sequentially composed nodes does not affect whether a graph is
structured.
This is due to the fact that sequential composition is one of the
constructs used to define structured control flow graphs
(Figure~\ref{sequence-figure}), and merging the nodes may be seen as
performing the reverse of node replacement for it.
\begin{figure}
  \begin{center}
    \begin{tikzpicture}[every new ->/.style={-latex}]
      \path (0,0) node (0) {0} -- ++(1,0) node (6) {6};
      \path (0,-1cm) node (7) {7} -- ++(1,0) node (11) {11} -- ++(1,0) node (14) {14} -- ++(1,0) node (17) {17};
      \path (0,-3cm) node (21) {21} -- ++(1,0) node (23) {23} -- ++(1,0) node (24) {24};
      \path (3,-2.3cm) node (28) {28} -- ++(1,0) node (31) {31};
      \node at (3,-3.7cm) (32) {32};
      \path (5,-3cm) node (35) {35} -- ++(1,0) node (39) {39} -- ++(1,0) node (42) {42};
      
      \graph{
        (7) -> (11) -> (14) -> (17) -> (39) -> (42);
        (21) -> (23) -> (24) -> {
          (28) -> (31);
          (32);
        } -> (35) -> (39);
      };

      \graph[grow down]{(0) -> (6)};
      \node at (0,-5cm) {43};

      \draw[-latex] (39) edge[out=270,in=270,looseness=0.6] (21);
    \end{tikzpicture}
  \end{center}
  \caption{Control flow graph for our example after sequential composition introduction}
  \label{example-control-flow-graph-after-sequence-introduction-figure}
\end{figure}

The only remaining nodes in this graph are those where the sequence of
instructions ends with a method call or return, or which represent a
more complex control flow.
In particular, the instructions for the \texttt{f()} method of
\texttt{TPK}, which begin at $pc = 43$, have been completely sequenced
together into a single node.
The code that corresponds to this control flow graph is that shown
earlier in Figure~\ref{forward-sequence-introduction-example-figure}.

\subsection{Introduce Loops and Conditionals}
\label{introduce-loops-and-conditionals-subsection}

After sequential composition has been introduced for all methods, we
must consider each method separately to handle method calls.
This means the strategy must loop, introducing loops and conditionals
to those methods that have no unresolved method calls and resolving
calls of methods that are then complete, until every method is
complete and has been separated into its own action.

Introducing loops and conditionals is performed as described by
Algorithm~\ref{introduce-loops-and-conditionals-algorithm}.
This considers each method individually, as specified by the for loop
on line~\ref{algorithm-introduce-loops-and-conditionals-method-loop}
of the algorithm. 
The condition on line~\ref{algorithm-no-unresolved-calls-condition}
ensures that only those methods where all method calls have already
been resolved undergo loop and conditional introduction.
Since we do not allow recursion, there is always at least one method
that does not depend on another method in the program.
It may be the case that a method depends only on special methods, in
which case this stage has no effect on that method until the special
method calls have been resolved.
Special method calls can always be resolved as they do not depend on
other methods in the program.

The \Call{HasNoUresolvedCalls}{$m$} procedure, used in the condition
on line~\ref{algorithm-no-unresolved-calls-condition}, checks that no
node in the control flow graph of $m$ ends in a method call, as a way
of determining whether $m$ has unresolved calls.
Since method resolution sequences a method call with the instructions
following it, a method call with nothing following it is a call that
has not yet been resolved.

\begin{algorithm}
  \begin{algorithmic}[1]
    \For{$m \gets methods$}
    \label{algorithm-introduce-loops-and-conditionals-method-loop}
    \If{\Call{HasNoUresolvedCalls}{$m$}}
    \label{algorithm-no-unresolved-calls-condition}
    \State $cfg \gets$ \Call{MakeMethodControlFlowGraph}{$m$}
    \label{algorithm-make-control-flow-graph2}
    \For{$node \gets$ \Call{ReverseNodes}{$cfg$}}
    \label{algorithm-node-checking-loop}
    \State \ApplyFor{Rule~[\nameref{if-introduction-rule}]}{$node$}
    \label{algorithm-introduce-if}
    \State \ApplyFor{Rule~[\nameref{if-else-introduction-rule}]}{$node$}
    \label{algorithm-introduce-if-else}
    \If{\Call{IsSimpleConditional}{$node$}}
    \label{algorithm-conditional-check}
    \State \ApplyFor{Rule~[\nameref{conditional-introduction-rule}]}{$node$}
    \label{algorithm-introduce-conditional}
    \EndIf
    \State \ApplyFor{Rule~[\nameref{while-introduction-rule1}]}{$node$}
    \label{algorithm-introduce-while1}
    \State \ApplyFor{Rule~[\nameref{while-introduction-rule2}]}{$node$}
    \label{algorithm-introduce-while2}
    \State \ApplyFor{Rule~[\nameref{do-while-introduction-rule}]}{$node$}
    \label{algorithm-introduce-do-while}
    \State \ApplyFor{Rule~[\nameref{infinite-loop-introduction-rule}]}{$node$}
    \label{algorithm-introduce-infinite-loop}
    \If{\Call{HasSimpleSequence}{$node$}}
    \label{algorithm-lci-sequence-check}
    \State \ApplyFor{Rule~[\nameref{sequence-introduction-rule}]}{$node$}
    \label{algorithm-lci-sequence-introduction}
    \EndIf
    \EndFor
    \EndIf
    \EndFor
  \end{algorithmic}
  \caption{IntroduceLoopsAndConditionals}
  \label{introduce-loops-and-conditionals-algorithm}
\end{algorithm}

For each method that undergoes loop and conditional introduction, we
consider again its control-flow graph to ensure the loops and
conditionals are introduced in the correct order to properly form
their bodies.
This involves constructing a control-flow graph for the method, at
line~\ref{algorithm-make-control-flow-graph2}.
The control-flow graph for a method $m$ is created by the procedure
\Call{MakeMethodControlFlowGraph}{$m$}.
This is similar to the \Call{MakeControlFlowGraph}{} procedure used in
the previous section, but it just constructs the graph
for a single method, starting at the entry point of that method.

The graph for the \texttt{handleAsyncEvent()} method in our example
(beginning at $pc=7$, its entry point) is shown in
Figure~\ref{example-simplified-control-flow-graph-figure}, alongside
the \Circus{} code obtained at the beginning of this stage for the
method.
The edge that forms a loop from $pc=35$ to $pc=39$ is shown as a
dashed line since looping edges are ignored at certain points in this
part of the strategy.
\begin{figure}
  \begin{center}
    % \begin{multicols}{4}
    \begin{minipage}{0.3\linewidth}
      \begin{tikzpicture}
        \node (7)  at (0,0)  {7};
        \node (39)  at (0,-1) {39};
        \node (42) at (-1,-2) {42};
        \node (21) at (1,-2) {21};
        \node (28) at (0.5,-3) {28};
        \node (32) at (1.5,-3) {32};
        \node (35) at (1,-4) {35};
        \draw[-latex] (7) to (39);
        \draw[-latex] (39) to (42);
        \draw[-latex] (39) to (21);
        \draw[-latex] (21) to (28);
        \draw[-latex] (21) to (32);
        \draw[-latex] (28) to (35);
        \draw[-latex] (32) to (35);
        % \draw[-latex,red!70!black,dashed,out=0,in=0,looseness=1.1] (35) to (39);
        \draw[-latex,dashed,out=0,in=0,looseness=1.1] (35) to (39);
      \end{tikzpicture}
    \end{minipage}
    % \columnbreak
    \begin{minipage}{0.6\linewidth}
      \scriptsize
      \setlength{\zedindent}{0cm}
      \begin{circus}
        Running \circdef \\
        \t1 \circif frameStack = \emptyset \circthen \Skip \\
        \t1 {} \circelse frameStack \neq \emptyset \circthen {} \\
        \t2 \circif pc = 0 \circthen {} \cdots {} \\
        \t2 {} \circelse pc = 7 \circthen HandleNewEPC(27) \circseq pc := 8 \circseq Poll \circseq \cdots \circseq \\
        \t3 pc := 39 \\
        \t2 {} \cdots {} \\
        \t2 {} \circelse pc = 21 \circthen HandleAloadEPC(2) \circseq pc := 22 \circseq Poll \circseq \cdots \circseq \\
        \t3 pc := \IF value1 \leq value2 \THEN 32 \ELSE 28 \\
        \t2 {} \cdots {} \\
        \t2 {} \circelse pc = 28 \circthen HandleAloadEPC(3) \circseq pc := 29 \circseq Poll \circseq \cdots \circseq \\
        \t3 pc := 35 \\
        \t2 {} \cdots {} \\
        \t2 {} \circelse pc = 32 \circthen HandleAloadEPC(3) \circseq pc := 33 \circseq Poll \circseq \cdots \circseq \\
        \t3 pc := 35 \\
        \t2 {} \cdots {} \\
        \t2 {} \circelse pc = 35 \circthen HandleAloadEPC(4) \circseq pc := 36 \circseq Poll \circseq \cdots \circseq \\
        \t3 pc := 39 \\
        \t2 {} \cdots {} \\
        \t2 {} \circelse pc = 39 \circthen HandleAloadEPC(4) \circseq pc := 36 \circseq Poll \circseq \cdots \circseq \\
        \t3 pc := \IF value1 \leq value2 \THEN 21 \ELSE 42 \\
        \t2 {} \cdots {} \\
        \t2 {} \circelse pc = 42 \circthen HandleReturnEPC \\
        \t2 \circfi \circseq Poll \circseq Running \\
        \t1 \circfi
      \end{circus}
    \end{minipage}
    %\end{multicols}
  \end{center}
  \caption{Simplified control flow graph and corresponding code for our example
    program}
  \label{example-simplified-control-flow-graph-figure}
\end{figure}

%TODO: explain how our definition of structure differs from that of
% MISRA - leave for final considerations?

The control-flow graph of each method is structured since the
transformations of the graph up to this point consist solely of
collapsing sequential compositions, which, as mentioned in the
previous section, does not cause a structured graph to become
unstructured.
Since we have defined the desired program structure in terms of a
small number of standard structures (shown in
Figure~\ref{structured-cfg-figures}), we can identify each of these
structures in the graph and introduce them into the program,
collapsing the graph in the process.

In order to easily identify the structures in isolation from other
structures, we begin at the end nodes of the method (ignoring looping
edges for the purposes of determining end nodes) and work backwards,
considering each node in turn.
This is specified by the loop beginning on
line~\ref{algorithm-node-checking-loop} of
Algorithm~\ref{introduce-loops-and-conditionals-algorithm}.
% TODO: ReverseNodes should return a sequence and should be more
% carefully defined
The procedure \Call{ReverseNodes}{$cfg$} iterates over the nodes of
$cfg$ beginning at the end nodes and moving in the reverse direction
of the graph edges, ignoring looping edges.
In our example this means we consider the $pc=42$ and $pc=35$ nodes
first, then $pc=28$ and $pc=32$, then $pc=21$, $pc=39$, and finally
$pc=7$.

For each node, we check each type of structure to see if the
control-flow graph starting at that point matches the structure, and
introduce the structure if it does.
Some of the structures (Figure~\ref{if-figure},
\subref{if-else-figure}, \subref{while-figure} and
\subref{do-while-figure}) are followed by further instructions.
A sequential composition must be introduced with the instructions
following the structure.
However, in programs with graphs such as the one shown below, the
sequential composition can only be introduced after the outer
conditional has been introduced.
Thus, the introduction of the sequential composition cannot be made
part of the rule for introducing the conditional.
\begin{center}
  \begin{tikzpicture}
    \node (1) at (0.0, 0.0) {$\bullet$};
    \node (2) at (1.5, 0.5) {$\bullet$};
    \node (3) at (1.5,-0.5) {$\bullet$};
    \node (4) at (3.0, 1.0) {$\bullet$};
    \node (5) at (3.0,-1.0) {$\bullet$};
    \node (6) at (3.0, 0.2) {$\bullet$};
    \node (7) at (3.0,-0.2) {$\bullet$};
    \node (8) at (4.0, 0.5) {$\bullet$};
    \node (9) at (4.0,-0.5) {$\bullet$};
    \node (0) at (5.0, 0.0) {$\bullet$};
    
    \draw[-latex] (-1,0) to (1);
    
    \draw[-latex] (1) to (2);
    \draw[-latex] (1) to (3);
    \draw[-latex] (2) to (4);
    \draw[-latex] (3) to (5);
    \draw[-latex] (2) to (6);
    \draw[-latex] (3) to (7);
    \draw[-latex] (4) to (8);
    \draw[-latex] (5) to (9);
    \draw[-latex] (6) to (8);
    \draw[-latex] (7) to (9);
    \draw[-latex] (8) to (0);
    \draw[-latex] (9) to (0);
  \end{tikzpicture}
\end{center}

The first type of structure we check for are conditionals.
There are three conditional structures:~\texttt{if} conditionals
(Figure~\ref{if-figure}), \texttt{if}-\texttt{else} conditionals
(Figure~\ref{if-else-figure}), and divergent conditionals
(Figure~\ref{divergent-figure}).
We introduce each with a separate rule, specialised to the form of the
conditional.

An \texttt{if} conditional with no else branch is introduced using
Rule~[\nameref{if-introduction-rule}], shown below.
Such a structure can be recognised from the form of the \Circus{} code
in the $Running$ action, which is that of a node whose sequence of
instructions ends with an assignment of the form
$pc := \IF b \THEN x \ELSE y$, and for which the $pc = y$ node ends in
an assignment $pc := x$.
Note that the branches cannot be the other way round (i.e.\ the
$pc = x$ branch will not be the body of the conditional) since the
conditional branches come from Java's branching instructions, which
branch to the specified address if the condition is true and go to the
next instruction if it is false.
\begin{restatable}[\texttt{if}-conditional-intro]{crule}{IfConditionalIntroductionRule}
  \label{if-introduction-rule}
  \setlength{\zedindent}{0.25cm}
  % \setlength{\abovedisplayskip}{0.1cm}
  % \setlength{\belowdisplayskip}{0.1cm}
  Given $i : ProgramAddress$, if $i \neq j$, $i \neq k$, and 
  \begin{circus}
    \{frameStack \neq \emptyset\} \circseq A \\
    {} = {} \\
    \{frameStack \neq \emptyset\} \circseq A \circseq \{frameStack \neq \emptyset\}
  \end{circus}
  then
  \begin{circus}
    \begin{array}{l}
      \circmu X \circspot \\
      \t1 \circif frameStack = \emptyset \circthen \Skip \\
      \t1 {} \circelse frameStack \neq \emptyset \circthen {} \\
      \t2 \circif \cdots \\
      \t2 {} \circelse pc = i \circthen A \circseq \\
      \t3 pc := \IF b \THEN j \ELSE k \\
      \t2 {} \cdots {} \\
      \t2 {} \circelse pc = k \circthen B \circseq pc := j \\
      \t2 {} \cdots {} \\
      \t2 \circfi \circseq Poll \circseq X \\
      \t1 \circfi
    \end{array}
    \circrefines_A
    \begin{array}{l}
      \circmu X \circspot \\
      \t1 \circif frameStack = \emptyset \circthen \Skip \\
      \t1 {} \circelse frameStack \neq \emptyset \circthen {} \\
      \t2 \circif \cdots \\
      \t2 {} \circelse pc = i \circthen A \circseq \\
      \t3 \circif b \circthen \Skip \\
      \t3 {} \circelse \lnot b \circthen pc := k \circseq Poll \circseq B \\
      \t3 \circfi \circseq pc := j \\
      \t2 {} \cdots {} \\
      \t2 {} \circelse pc = k \circthen B \circseq pc := j \\
      \t2 {} \cdots {} \\
      \t2 \circfi \circseq Poll \circseq X \\
      \t1 \circfi 
    \end{array}
  \end{circus}
\end{restatable}
Rule~[\nameref{if-introduction-rule}] introduces a conditional for
nodes that match the form described above, which in the rule is the
$pc = i$ node.
The conditional is introduced with the true branch empty (represented
by $\Skip$) and the false branch containing the instructions in the
body of the conditional.
The assignment $pc := j$ is moved outside the conditional from both
the true amd false branches.

As in Rule~[\nameref{sequence-introduction-rule}], the sequence of
actions for the node must not affect the nonemptiness of the
$frameStack$.
A similar condition is required for all the rules in this section.
We also require that the targets of the conditional are different from
the node at which the conditional is introduced, since that would
introduce a loop, which is not the purpose of this rule.
Rule~[\nameref{if-introduction-rule}] is applied on
line~\ref{algorithm-introduce-if} of
Algorithm~\ref{introduce-loops-and-conditionals-algorithm}.
Note that, since the structure can be identified from the form of the
\Circus{} code alone, it is not necessary to guard the application of
the rule with a condition on the control-flow graph.

We introduce \texttt{if}-\texttt{else} conditionals using
Rule~[\nameref{if-else-introduction-rule}] and divergent conditionals
using Rule~[\nameref{conditional-introduction-rule}].
Since these are similar to Rule~[\nameref{if-introduction-rule}], we
omit them here.
They can be found in Appendix~\ref{compilation-rules-appendix}.
We apply these rules on lines~\ref{algorithm-introduce-if-else}
and~\ref{algorithm-introduce-conditional}.

Rule~[\nameref{conditional-introduction-rule}] introduces a
conditional with no restrictions on its form.
To ensure it is only applied to nodes that match the form of
Figure~\ref{divergent-figure}, we guard its application by the
condition \Call{IsSimpleConditional}{$node$} on
line~\ref{algorithm-conditional-check}.
The procedure \Call{IsSimpleConditional}{$node$} checks if the targets
of $node$ have no outgoing nodes.
This is a condition on the control-flow graph that cannot be expressed
in the statement of the rule.

After attempting to introduce conditionals, we attempt to introduce
loops.
There are three types of loop to consider, as shown earlier:
\texttt{while} loops (Figure~\ref{while-figure}),
\texttt{do}-\texttt{while} loops (Figure~\ref{do-while-figure}), and
infinite loops (Figure~\ref{infinite-loop-figure}).
A \texttt{while} loop has a form similar to that of a conditional,
except that one of the branches ends with a jump back to the beginning
of the node with the conditional.
This structure may be introduced using
Rule~[\nameref{while-introduction-rule1}] below.
This rule introduces a conditional at a node $pc=i$ with its false
branch ending in an assignment of $i$ to $pc$, and introduces a
recursion to the beginning of the $pc=i$ node in that branch of the
conditional, representing a loop.
Since this loop may be within a conditional, we simply move the $pc$
assignment for the true branch outside the conditional, so that a
sequential composition can be introduced later, as with \texttt{if}
and \texttt{if}-\texttt{else} conditionals.
\begin{restatable}[\texttt{while}-loop-intro1]{crule}{WhileLoopIntroductionRuleA}
  \label{while-introduction-rule1}
  \setlength{\zedindent}{0.2cm}
  \setlength{\zedtab}{0.9\zedtab}
  Given $i : ProgramAddress$, if $i \neq j$,
  \begin{circus}
    \{frameStack \neq \emptyset\} \circseq A \\
    {} = {} \\
    \{frameStack \neq \emptyset\} \circseq A \circseq \{frameStack \neq \emptyset\}
  \end{circus}
  then
  \begin{circus}
    \begin{array}{l}
      \circmu X \circspot \\
      \t1 \circif frameStack = \emptyset \circthen \Skip \\
      \t1 {} \circelse frameStack \neq \emptyset \circthen {} \\
      \t2 \circif \cdots \\
      \t2 {} \circelse pc = i \circthen A \circseq \\
      \t3 pc := \IF b \THEN j \ELSE k \\
      \t2 \cdots \\
      \t2 {} \circelse pc = j \circthen B \\
      \t2 \cdots \\
      \t2 {} \circelse pc = k \circthen C \circseq pc := i \\
      \t2 \cdots \\
      \t2 \circfi \circseq Poll \circseq X \\
      \t1 \circfi 
    \end{array}
    \circrefines_A
    \begin{array}{l}
      \circmu X \circspot \\
      \t1 \circif frameStack = \emptyset \circthen \Skip \\
      \t1 {} \circelse frameStack \neq \emptyset \circthen {} \\
      \t2 \circif \cdots \\
      \t2 {} \circelse pc = i \circthen (\circmu Y \circspot A \circseq \\
      \t3 \circif b \circthen \Skip \\
      \t3 {} \circelse \lnot b \circthen {} \\
      \t4 pc := k \circseq Poll \circseq C \circseq pc := i \circseq Poll \circseq Y \\
      \t3 \circfi) \circseq pc := j \\
      \t2 \cdots \\
      \t2 {} \circelse pc = j \circthen B \\
      \t2 \cdots \\
      \t2 {} \circelse pc = k \circthen C \circseq pc := i \\
      \t2 \cdots \\
      \t2 \circfi \circseq Poll \circseq X \\
      \t1 \circfi 
    \end{array}
  \end{circus}
\end{restatable}%
As a \texttt{while} loop may occur with the loop at the end of either
condition branch (since the loop may be created by a \texttt{goto}
instruction in the Java bytecode), we also provide a similar rule,
Rule~[\nameref{while-introduction-rule2}], which introduces the loop
in the true branch of the conditional.
These two rules are applied on lines~\ref{algorithm-introduce-while1}
and~\ref{algorithm-introduce-while2} of the algorithm.
Rule~[\nameref{while-introduction-rule2}] is presented in
Appendix~\ref{compilation-rules-appendix}.

The second type of loop we introduce is the \texttt{do}-\texttt{while}
loop.
A \texttt{do}-\texttt{while} loop is similar to a \texttt{while} loop,
but is distinguished by the fact that the conditional $pc$ assignment
that causes the loop is at the end of the loop, rather than at the
beginning or in the middle.
We introduce these loops using
Rule~[\nameref{do-while-introduction-rule}], which we omit due to its
similarity with Rule~[\nameref{while-introduction-rule1}]; it is
presented in Appendix~\ref{compilation-rules-appendix}.
This rule is applied on line~\ref{algorithm-introduce-do-while} of the
algorithm.
Note that the false branch can never cause the loop in this case,
since it will just go to the next instruction.
Attempting to redirect it and create the loop with a \texttt{goto}
instruction would add an instruction within the loop after the
conditional, so it would be dealt with as a \texttt{while} loop.
Therefore, it is not necessary to provide two compilation rules for
\texttt{do}-\texttt{while} loops, unlike \texttt{while} loops where

The final loop structure that we attempt to introduce is that of an
infinite loop.
An infinite loop may be identified as a block of instructions that
ends with a $pc$ assignment that causes a jump back to the beginning
of the block of instructions.
We introduce these loops using
Rule~[\nameref{infinite-loop-introduction-rule}], presented in
Appendix~\ref{compilation-rules-appendix}.
This rule is applied on line~\ref{algorithm-introduce-infinite-loop}
of the algorithm.

After we have attempted to introduce each of the structures for a
particular node, we attempt to introduce a sequential composition.
This ensures that \texttt{if}, \texttt{if}-\texttt{else},
\texttt{while} and \texttt{do}-\texttt{while} structures that occur
within conditionals are sequentially composed with the node following
them if possible.
It also handles cases where sequential compositions occur before
loops, preventing them from being introduced in
Section~\ref{introduce-forward-sequence-subsection} without
interfering with the introduction of the loop.
Such a case occurs at the $pc=7$ node in our example.

The requirement for sequential composition to be introduced is the
same as in Section~\ref{introduce-forward-sequence-subsection}:~it
must be a simple sequential composition from a node with a single
outgoing edge to a node with a single incoming edge.
Thus we check for a simple sequence on
line~\ref{algorithm-lci-sequence-check} of
Algorithm~\ref{introduce-loops-and-conditionals-algorithm}.
The sequential composition is then introduced on
line~\ref{algorithm-lci-sequence-introduction} if it is a simple
sequential composition.

As mentioned earlier, these steps are repeated for each node, working
backwards through the control-flow graph of each method.
Each of the rules for introducing control flow structures reduces the
graph to either a sequential composition graph
(Figure~\ref{sequence-figure}) or a single node.
Divergent conditionals and infinite loops are the structures whose
control-flow graphs are reduced to a single node by their introduction
rules.

The remaining structures are reduced to sequential composition graphs.
The reduction of the seqential composition graph depends on which form
of node replacement is used to embed the structure in the control-flow
graph of the method.
There are four cases to consider:~root-node replacement
(Figure~\ref{root-replacement-figure}), end-node replacement
(Figure~\ref{end-replacement-figure}), internal-node replacement
(Figure~\ref{internal-replacement-figure}) and branch-end replacement
(Figure~\ref{branch-end-replacement-figure}).
Replacing the root node of a
graph $G$ with a graph $H$ can be viewed as replacing the end node of
$H$ with $G$.
Since we are considering the nodes moving backwards through the
control flow graph, we will always treat this as an end node
replacement.

In the cases of end-node replacement and internal-node replacement we
can introduce the sequential composition immediately, reducing the
graph to a single node.
In the case of branch-end replacement, if some graph $H$ is embedded
in a graph $G$, then reducing $H$ to a sequential composition results
in the overall graph having the form of $G$.
This can be seen from the example shown in
Figure~\ref{branch-end-replacement-figure}, where reducing the
\texttt{while} loop structure formed by nodes $a$, $b$ and $4$ to a
sequential composition yields the graph shown in
Figure~\ref{branch-end-reduction-figure}.
This has the form of a \texttt{if}-\texttt{else} conditional
(Figure~\ref{if-else-figure}).
Such structures are introduced on further iterations of the loop over
the nodes.
Thus, given a structured control flow graph at the beginning of this
stage, the control flow graph is reduced to a single node, with all
the control flow structures in the method introduced.

\begin{figure}
  \begin{center}
    \begin{tikzpicture}
      \useasboundingbox (-1.5,-2.5) rectangle (1.5,2);
      \node at (0,2) (start) {};
      \node at ( 0, 1)    (A) {$1$};
      \node at (-1,-0.5)  (B) {$a$};
      \node at ( 1,-0.5)  (C) {$3$};
      %\node at (-2,-1.5)  (D) {$b$};
      \node at ( 0,-2)    (G) {$4$};
      \draw[-latex] (start) -- (A);
      \draw[-latex] (A) -- (B);
      \draw[-latex] (A) -- (C);
      %\draw[-latex] (B) to (D);
      \draw[-latex] (B) to (G);
      %\draw[-latex] (D) to[in=180,out=90] (B);
      \draw[-latex] (C) -- (G);
    \end{tikzpicture}
  \end{center}
  \caption{The graph of Figure~\ref{branch-end-replacement-figure} after loop introduction}
  \label{branch-end-reduction-figure}
\end{figure}

In our example, we begin at the $pc=35$ node, where there are no
structures to introduce. 
The same holds true of the $pc=28$ and $pc=32$ nodes (note that the
edges coming from them are not simple sequential compositions).
An \texttt{if}-\texttt{else} conditional is introduced at $pc=21$,
absorbing the $pc=28$ and $pc=32$ nodes.
The sequential composition from the $pc=21$ node to the $pc=35$ node
can then be introduced immediately as it is now a simple sequential
composition (because it is not at the end of an outer conditional).
We then introduce a \texttt{while} loop at the $pc=39$ loop (using
Rule~[\nameref{while-introduction-rule2}]), and the sequential
composition with the $pc=42$ node is introduced afterwards.
Finally, a sequential composition from the $pc=7$ to the $pc=39$ node
is introduced, collapsing the control flow graph to a single node.
The code at $pc=7$ is then that shown earlier in
Figure~\ref{loop-and-conditional-introduction-example-figure}.

\subsection{Resolve Method Calls}
\label{resolve-method-calls-subsection}

When a method is complete, calls to that method can then be resolved.
This step begins with the copying of the method into a separate
action, so that it can be referenced elsewhere.
This is performed as described by
Algorithm~\ref{separate-complete-methods-algorithm}.
\begin{algorithm}
  \begin{algorithmic}[1]
    \For{$m \gets methods$} \label{algorithm-method-separation-loop}
    \If{\Call{MethodIsComplete}{$m$}} \label{algorithm-check-method-completeness}
    \State \ApplyFor{Rule~[\nameref{introduce-class-information-rule}]}{$m$}
    \label{algorithm-introduce-class-information}
    \State \ApplyFor{Law~[\nameref{action-intro-law}]}{$m$} \label{algorithm-introduce-method-action}
    \State \ApplyFor{Law~[\nameref{copy-rule-law}]}{$m$} \label{algorithm-copy-method-action}
    \EndIf
    \EndFor
  \end{algorithmic}
  \caption{SeparateCompleteMethods}
  \label{separate-complete-methods-algorithm}
\end{algorithm}

Algorithm~\ref{separate-complete-methods-algorithm} looks at each
method separately, as specified by the loop on
line~\ref{algorithm-method-separation-loop}, and determines if it is
complete, on line~\ref{algorithm-check-method-completeness}.
This involves a simple syntactic check that each conditional branch
ends in a return instruction or a recursion.

If a method is complete, we first introduce an assumption at the start
of the sequence of actions for the method that establishes what the
value of $currentClass$ is during the method's execution
(line~\ref{algorithm-introduce-class-information}).
This is performed by an application of
Rule~[\nameref{introduce-class-information-rule}], shown in
Figure~\ref{introduce-class-information-rule-figure}.
The $currentClass$ is determined by finding the class information $c$
in the range of $cs$ that contains a method $m$ with the same entry
point, $i$, as the complete method being operated on.
By recording this information in an assumption, we can use it in later
stages of the compilation, when the value of the $pc$ is no longer
available.
\begin{figure}
\begin{restatable}[introduce-class-information]{crule}{IntroduceClassInformationRule}
  \label{introduce-class-information-rule}
  Given $i : ProgramAddress$,
  \begin{circus}
    \begin{array}{l}
      \circmu X \circspot \\
      \t1 \circif frameStack = \emptyset \circthen \Skip \\
      \t1 {} \circelse frameStack \neq \emptyset \circthen {} \\
      \t2 \circif \cdots \\
      \t2 {} \circelse pc = i \circthen A \\
      \t2 {} \cdots {} \\
      \t2 \circfi \circseq Poll \circseq X \\
      \t1 \circfi
    \end{array}
    \circrefines_A
    \begin{array}{l}
      \circmu X \circspot \\
      \t1 \circif frameStack = \emptyset \circthen \Skip \\
      \t1 {} \circelse frameStack \neq \emptyset \circthen {} \\
      \t2 \circif \cdots \\
      \t2 {} \circelse pc = i \circthen \{currentClass = c\} \circseq A \\
      \t2 {} \cdots {} \\
      \t2 \circfi \circseq Poll \circseq X \\
      \t1 \circfi
    \end{array}
  \end{circus}
  where $c : Class$ is such that
  \begin{displaymath}
    c \in \ran cs \land \exists  m : MethodID | m \in \dom c.methodEntry @ c.methodEntry~m = i.
  \end{displaymath}
\end{restatable}
\caption{Rule~[\nameref{introduce-class-information-rule}]}
\label{introduce-class-information-rule-figure}
\end{figure}

For the methods that are complete, the sequence of actions for the
method are placed in a separate action, which is introduced using
Law~[\nameref{action-intro-law}] on
line~\ref{algorithm-introduce-method-action}.
We form the name of this action from the name of the class to which
the method belongs and the name of the method, concatenated together
with an underscore.
This is similar to how icecap forms method names, although the name
used for the action does not have an effect upon the correctness of
the strategy, provided it is unique.
Once the method action has been introduced, the sequence of actions at
the method's entry point in the $Running$ action is replaced with a
reference to the newly introduced action by applying
Law~[\nameref{copy-rule-law}] on
line~\ref{algorithm-copy-method-action}.

In our example, the method \texttt{f()} of the \texttt{TPK} class,
which starts at $pc = 43$, is complete on the first iteration of the
loop on line~\ref{algorithm-method-loop} of
Algorithm~\ref{epc-algorithm}.
This can be seen in
Figure~\ref{forward-sequence-introduction-example-figure}.
The method is complete in this case because it consists of a straight
sequence of instructions ending with $HandleAreturnEPC$, which
represents the \texttt{areturn} instruction.
The sequence of instructions at $pc = 43$ is copied into the action $TPK\_f$,
which can be seen in
Figure~\ref{method-call-resolution-example-figure}.
The $pc = 43$ branch is replaced with a call to $TPK\_f$.

After all the complete methods have been copied into separate actions,
calls to those methods are resolved.
This is performed as described by
Algorithm~\ref{resolve-method-calls-algorithm}.
In this algorithm, while we indicate the parameters supplied to a rule
with the word \textbf{for}, as in previous algorithms, we use the word
\textbf{to} to indicate which part of the $Running$ action the rule is
applied to.
In all previous algorithms, the laws are applied to the whole action,
and so we omit the \textbf{to} clause.
\begin{algorithm}[th]
  \begin{algorithmic}[1]
    \For{$m \gets methods$}
    \label{algorithm-mci-method-loop} 
    \For{$mc \gets$ \Call{UnresolvedMethodsCalls}{$m$}}
    \label{algorithm-mci-method-call-loop}
    \If{\Call{IsResolvable}{$mc$}}
    \label{algorithm-is-resolvable-check}
    \If{\Call{HasSingleTarget}{$mc$}}
    \label{algorithm-multiple-targets-check}
    \Try
    \label{algorithm-try-block-begin}
    \State \ApplyTo{Rule~[\nameref{refine-invokespecial-rule}]}{$mc$}
    \State \ApplyTo{Rule~[\nameref{refine-invokestatic-rule}]}{$mc$}
    \State \ApplyTo{Rule~[\nameref{refine-invokevirtual-single-rule}]}{$mc$}
    \EndTry
    \Try
    \label{algorithm-try-block2-start}
    \State \ApplyTo{Rule~[\nameref{resolve-special-method-rule}]}{$mc$}
    \State \ApplyTo{Rule~[\nameref{resolve-special-method-virtual-rule}]}{$mc$}
    \State \ApplyFor{Rule~[\nameref{resolve-normal-method-rule}]}{$mc$}
    \State \ApplyFor{Rule~[\nameref{resolve-normal-method-virtual-rule}]}{$mc$}
    \EndTry
    \Else
    \State \ApplyTo{Rule~[\nameref{refine-invokevirtual-multi-rule}]}{$mc$}
    \label{algorithm-refine-invokevirtual-multi}
    \For{$target \gets $\Call{Targets}{$mc$}}
    \label{algorithm-targets-loop}
    \Try
    \State \ApplyToFor{Rule~[\nameref{resolve-special-method-branch-rule}]}{$mc$}{$target$}
    \State \ApplyFor{Rule~[\nameref{resolve-normal-method-branch-rule}]}{$mc$, $target$}
    \EndTry
    \EndFor
    \State \ApplyFor{Law~[\nameref{alt-seq-dist-law}]}{$mc$}
    \label{algorithm-move-pc-assignments}
    \EndIf
    \If{\Call{HasSimpleSequence}{$node$}}
    \label{algorithm-mci-sequence-introduction-start}
    \State \ApplyFor{Rule~[\nameref{sequence-introduction-rule}]}{$node$}
    \EndIf
    \label{algorithm-mci-sequence-introduction-end}
    \EndIf
    \EndFor
    \EndFor
  \end{algorithmic}
  \caption{ResolveMethodCalls}
  \label{resolve-method-calls-algorithm}
\end{algorithm}

The algorithm begins by checking each unresolved method call in each
method, as specified by the loops on
lines~\ref{algorithm-mci-method-loop}
and~\ref{algorithm-mci-method-call-loop}.
The list of unresolved method calls for a given method, $m$, is
computed by \Call{UnresolvedMethodsCalls}{$m$}, which finds the $pc$
values for which the sequence of instructions ends with a method
invocation instruction with no $pc$ assignment following it.
An example of an unresolved method call can be seen in the $pc = 14$
branch of Figure~\ref{forward-sequence-introduction-example-figure},
reproduced in part below.
\begin{circusaction}
  \t2 {} \cdots {} \\
  \t2 {} \circelse pc = 14 \circthen \cdots \circseq pc := 16 \circseq Poll \circseq HandleInvokevirtualEPC(36) \\
  \t2 {} \cdots {}
\end{circusaction}
This ends with the action $HandleInvokeVirtualEPC(36)$, which handles
the \texttt{invokevirtual} instruction for the constant pool index
$36$.
It is unresolved because it does not have any $pc$ assignment or other
actions following it.

For each method call that needs resolving, we check if it can be
resolved at this point in the compilation strategy.
This is performed on line~\ref{algorithm-is-resolvable-check}, where
the boolean value \Call{IsResolvable}{$mc$} is checked.
\Call{IsResolvable}{$mc$} is true if all the targets of the method
call $mc$ are either special methods or non-special methods that are
already complete and have been separated into their own actions (as
described in Algorithm~\ref{separate-complete-methods-algorithm}).

If the method call is resolvable, then we check whether there is a
single target for the method call or multiple targets.
Calls to a single target can be transformed to a simple reference to
the corresponding method action.
Calls with multiple targets are transformed into a choice over the
behaviours for each of the possible targets.
The check of the number of targets is performed on
line~\ref{algorithm-multiple-targets-check} of
Algorithm~\ref{resolve-method-calls-algorithm}.
This uses the condition \Call{HasSingleTarget}{$mc$}, which is true if
the instruction to be handled is \texttt{invokespecial} or
\texttt{invokestatic}, or an \texttt{invokevirtual} instruction
referencing a class with no subclasses (other than itself).
It is false otherwise, that is, if the instruction is
\texttt{invokevirtual} and there are multiple subclasses of the class
referenced by the instruction.

For method calls with a single target, we replace the action that
handles the method invocation instruction with an action that pops the
arguments for the method from the stack and handles invocation of the
specific method referenced by the instruction.
This is handled slightly differently for each of the method invocation
instructions in our bytecode subset, so we have three rules for
performing this transformation, one for each
instruction:~Rule~[\nameref{refine-invokestatic-rule}],
Rule~[\nameref{refine-invokespecial-rule}] and
Rule~[\nameref{refine-invokevirtual-single-rule}].
% TODO: move this to the loops and conditionals section if used
% previously
These rules are applied in a try block, beginning on
line~\ref{algorithm-try-block-begin}, which tries to apply each rule in
turn, stopping when one succeeds.

In Figure~\ref{refine-invokestatic-rule-figure} we show
Rule~[\nameref{refine-invokestatic-rule}], which handles
\texttt{invokestatic} instructions.
This rule, as with other rules in this section, is applied to an
action beginning with an assumption on the value of $pc$.
This allows it to be applied at the start of a branch of the choice in
$Running$, after introducing the assumption using
Law~[\nameref{alt-assump-intro-law}], or after a sequential
composition, after introducing the assumption using
Law~[\nameref{assign-assump-intro-law}] and distributing it over
$Poll$ using Lemma~\ref{Poll-assumption-distribution-lemma}.
After the application of the rule, the assumption can be eliminated
using Law~[\nameref{assump-elim-law}] and
Law~[\nameref{seq-unitl-law}].
For brevity, we omit the application of these laws from the algorithm,
and we understand that assumptions are introduced at the points
required for the rules to be applied.
\begin{figure}[thp]
\begin{restatable}[refine-invokestatic]{crule}{RefineInvokestaticRule}
  \label{refine-invokestatic-rule}
  \setlength{\zedindent}{0.25cm}
  \begin{circus}
    \begin{array}{l}
      \{ pc = i \} \circseq \\
      \t1 HandleInvokestaticEPC(cpi)
    \end{array}
    \circrefines_A
    \begin{array}{l}
      \{ pc = i \} \circseq \circvar poppedArgs : \seq Word \circspot \\
      \t1 \lschexpract \exists argsToPop? == methodArguments~m @ \\
      \t2 InterpreterStackFrameInvoke \rschexpract \circseq \\
      \t1 Invoke(c, m, poppedArgs, \true)
    \end{array}
  \end{circus}
  where $m : MethodID$ and $c : ClassID$ are such that
  \begin{displaymath}
    \exists c_0 : Class | c_0 \in \ran cs @ \\
    \t1 (\exists m_0 : MethodID | m_0 \in \dom c_0.methodEntry @ \\
    \t2 i \in c_0.methodEntry~m_0 \upto c_0.methodEnd~m_0) \\
    \t1 cpi \in methodRefIndices~c_0 \land c_0.constantPool~cpi = MethodRef~(c,m).
  \end{displaymath}
\end{restatable}
\caption{Rule~[\nameref{refine-invokestatic-rule}]}
\label{refine-invokestatic-rule-figure}
\end{figure}
Rule~[\nameref{refine-invokestatic-rule}] refines the
$HandleInvokeStaticEPC(cpi)$ to an action that pops the method's
arguments from the stack using the $InterpreterStackFrameInvoke$
operation and then behaves as the $Invoke$ action described in
Section~\ref{cee-interpreter-section}.
The method handled by $Invoke$ is identified by a class identifier,
$c$, and a method identifier, $m$.
These identifiers are determined from the $cpi$ parameter passed to
$HandleInvokeStaticEPC$, which is an index into the $constantPool$ of
the current class information.
To determine the identifiers, we first determine the current class
information, $c_0$, which is the class in $cs$ that contains a method,
$m_0$, whose bytecode spans over the current $pc$ value, $i$.
Within $c_0$, the $constantPool$ entry at $cpi$ must be a $MethodRef$.
The $c$ and $m$ values of the method to be invoked are those contained
in the $MethodRef$.
These are passed to $Invoke$, along with the arguments popped from the
stack, $poppedArgs$, and a boolean value indicating whether the method
call is static, which is $\true$ in the case of \texttt{invokestatic}.

Rule~[\nameref{refine-invokespecial-rule}] and
Rule~[\nameref{refine-invokevirtual-single-rule}] are similar to
Rule~[\nameref{refine-invokestatic-rule}], but they produce slightly
different sequences of actions due to the differences in the semantics
of the method invocation instructions, described in
Section~\ref{cee-interpreter-subsection}.
They provide for popping an additional \texttt{this} argument from the
stack and pass a $\false$ boolean value to $Invoke$.
Rule~[\nameref{refine-invokevirtual-single-rule}] also leaves a
communication on the $getClassIDOf$ channel in place.
The communication is eliminated later in the strategy, during the
\emph{Data Refinement of Objects} stage, described in
Section~\ref{data-refinement-of-objects-section}.

After one of the above rules is applied, the method invocation is
resolved by transforming the $Invoke$ action to the behaviour of the
method being invoked.
A $pc$ assignment is also introduced after the method's behaviour so
that it can be sequentially composed with the instructions after the
method call.
There are four rules for this, applied in another try block on
line~\ref{algorithm-try-block2-start}:~Rule~[\nameref{resolve-special-method-rule}]
and Rule~[\nameref{resolve-special-method-virtual-rule}], which
handles resolution of special methods, and
Rule~[\nameref{resolve-normal-method-rule}] and
Rule~[\nameref{resolve-normal-method-virtual-rule}], which handle
resolution of non-special methods.

Rule~[\nameref{resolve-special-method-rule}], shown in
Figure~\ref{resolve-special-method-rule-figure}, operates by simply
replacing the call to the $Invoke$ action with actions that specify
the behaviour for the special method, and introducing a $pc$
assignment after those actions.
This collapses the choice in the definition of the $Invoke$ action.
The actions that define behaviour of the special method are identified
by the syntactic function $specialMethodAction$, which is defined by
Table~\ref{special-method-action-table}.
It determines which behaviour should be used based on the class and
method identifiers passed to $Invoke$ and the boolean value indicating
whether the method is static.
\begin{figure}[thp]
\begin{restatable}[resolve-special-method]{crule}{ResolveSpecialMethodRule}
  \label{resolve-special-method-rule}
  If $c$, $m$ and $static$ match one of the rows of
  Table~\ref{special-method-action-table}, then
  \setlength{\zedindent}{0.25cm} \setlength{\zedtab}{0.5cm}
  \begin{circus}
    \begin{array}{l}
      \{ pc = i \} \circseq \circvar poppedArgs : \seq Word \circspot \\
      \lschexpract \exists argsToPop? == e @ \\
      \t1 InterpreterStackFrameInvoke \rschexpract \circseq \\
      Invoke(c, m, poppedArgs, s)
    \end{array}
    \circrefines_A
    \begin{array}{l}
      \{ pc = i \} \circseq \circvar poppedArgs : \seq Word \circspot \\
      \lschexpract \exists argsToPop? == e @ \\
      \t1 InterpreterStackFrameInvoke \rschexpract \circseq \\
      specialMethodAction(c, m, s) \circseq \\
      pc := i + 1
    \end{array}
  \end{circus}
  where $specialMethodAction$ is the syntactic function defined by
  Table~\ref{special-method-action-table}.
\end{restatable}
\caption{Rule~[\nameref{resolve-special-method-rule}]}
\label{resolve-special-method-rule-figure}
\end{figure}
Rule~[\nameref{resolve-special-method-virtual-rule}] is similar to
Rule~[\nameref{resolve-special-method-rule}], but handles the extra
communication on $getClassIDOf$ in the case of an
\texttt{invokevirtual} instruction.

\begin{table}
  \centering
  \small
  \setlength{\abovedisplayskip}{-5pt}
  \setlength{\belowdisplayskip}{-10pt}
  \setlength{\abovedisplayshortskip}{0pt}
  \setlength{\belowdisplayshortskip}{0pt}
  \setlength{\zedindent}{-0.1cm}
  \setlength{\zedleftsep}{0cm}
  \renewcommand{\arraystretch}{1}
  \rowcolors{1}{white}{lightgray}
  \begin{tabular}{p{6.5cm}p{7.7cm}}
    \hline
    Conditions on $c$, $m$ and $s$ & $specialMethodAction(c, m, s)$ \\
    \hline
    \begin{circus}
      (c,resumeThreadClass) \in subclassRel~cs \\
      \land m = resumeThreadID \land s = \true
    \end{circus} &
                   \begin{circus}
                     resumeThread!(WordToThreadID~(methodArgs~1)) \\
                     \t1 {} \then resumeThreadRet \then \Skip
                   \end{circus}\\
    \begin{circus}
      (c,suspendClass) \in subclassRel~cs \\
      \land m = suspendID \land s = \true
    \end{circus} &
                   \begin{circus}
                     suspend \then suspendRet \then \Skip
                   \end{circus}\\
    \begin{circus}
      (c,writeClass) \in subclassRel~cs \\
      \land m = writeID \land s = \true
    \end{circus} &
                   \begin{circus}
                     output!(methodArgs~1) \then \Skip
                   \end{circus}\\
    \begin{circus}
      (c,readClass) \in subclassRel~cs \\
      \land m = readID \land s = \true
    \end{circus} &
                   \begin{circus}
                     input?value \then \lschexpract InterpreterPush \hide (pc,pc') \rschexpract
                   \end{circus}\\
    \begin{circus}
      (c,managedSchedulableClass) \\
      \t1 {} \in subclassRel~cs \\
      \land m = registerID \land s = \false
    \end{circus} &
                   \begin{circus}
                     register!thread!(head~methodArgs) \\
                     \t1 {} \then registerRet \then \Skip
                   \end{circus}\\
    \begin{circus}
      (c,managedMemoryClass) \in subclassRel~cs \\
      \land m = enterPrivateMemoryHelperID \\
      \land s = \true
    \end{circus} &
                   \begin{circus}
                     enterPrivateMemory!thread!(methodArgs~1) \\
                     \t1 {} \then enterPrivateMemoryRet \then \Skip 
                   \end{circus}\\
    \begin{circus}
      (c,managedMemoryClass) \in subclassRel~cs \\
      \land m = executeInAreaOfHelperID \\
      \land s = \true
    \end{circus} &
                   \begin{circus}
                     executeInAreaOf!thread!(methodArgs~1) \\
                     \t1 {} \then executeInAreaOfRet \then \Skip
                   \end{circus}\\
    \begin{circus}
      (c,managedMemoryClass) \in subclassRel~cs \\
      \land m = executeInOuterAreaHelperID \\
      \land s = \true
    \end{circus} &
                   \begin{circus}
                     executeInOuterArea!thread \\
                     \t1 {} \then executeInOuterAreaRet \then \Skip
                   \end{circus}\\
    \begin{circus}
      (c,managedMemoryClass) \in subclassRel~cs \\
      \land m = exitMemoryID \land s = \true
    \end{circus} &
                   \begin{circus}
                     exitMemory!thread \\
                     \t1 {} \then exitMemoryRet \then \Skip
                   \end{circus}\\
    \begin{circus}
      (c,aperiodicEventHandlerClass) \\
      \t1 {} \in subclassRel~cs \\
      \land m = initAPEHID \land s = \false
    \end{circus} &
                   \begin{circus}
                     initAPEH!thread!(seqTo5Tuple~methodArgs) \\
                     \t1 {} \then initAPEHRet \then \Skip
                   \end{circus}\\
    \begin{circus}
      (c,periodicEventHandlerClass) \\
      \t1 {} \in subclassRel~cs \\
      \land m = initPEHID \land s = \false
    \end{circus} &
                   \begin{circus}
                     initPEH!thread!(seqTo7Tuple~methodArgs) \\
                     \t1 {} \then initPEHRet \then \Skip
                   \end{circus}\\
    \begin{circus}
      (c,oneShotEventHandlerClass) \\
      \t1 {} \in subclassRel~cs \\
      \land m = initOSEHAbsID \land s = \false
    \end{circus} &
                   \begin{circus}
                     initOSEHAbs!thread!(seqTo6Tuple~methodArgs) \\
                     \t1 {} \then initOSEHAbsRet \then \Skip
                   \end{circus}\\
    \begin{circus}
      (c,oneShotEventHandlerClass) \\
      \t1 {} \in subclassRel~cs \\
      \land m = initOSEHRelID \land s = \false
    \end{circus} &
                   \begin{circus}
                     initOSEHRel!thread!(seqTo6Tuple~methodArgs) \\
                     \t1 {} \then initOSEHRelRet \then \Skip
                   \end{circus}\\
  \end{tabular}
  \caption{The syntactic function $specialMethodAction(c, m, static)$}
  \label{special-method-action-table}
\end{table}

Rule~[\nameref{resolve-normal-method-rule}], shown in
Figure~\ref{resolve-normal-method-rule-figure}, resolves non-special
methods by unrolling the loop in $Running$ to sequence the method call
with the action defining the method's behaviour.
The entry point of the method is obtained from the class information
for the method, which is determined as described by the data operation
$ResolveMethod$.
The first proviso of the rule requires that the nonemptiness of the
$frameStack$ is not affected by the instructions before the method
invocation, as with previous compilation rules in this stage. 
The second proviso of the rule ensures that the entry point, $k$, is
that given in the class information provided by $ResolveMethod$ for
the class identifier $c$ and method identifier $m$.
The action containing the behaviour of the method is the action $M$ at
the $pc = k$ branch of the choice in $Running$.
The third proviso requires that the execution of $M$ must result in
the top stack frame being popped and the $pc$ being set to the value
stored in the next stack frame.
This is needed to ensure that the method can be sequenced with the
behaviour after it.
It is true for all complete methods, since the return instructions
establish the required property and any property may be assumed to
hold after an infinite loop.
% TODO: put this, and other rules, in a float to prevent breaking
% across pages
\begin{figure}[thp]
\begin{restatable}[resolve-normal-method]{crule}{ResolveNormalMethodRule}
  \label{resolve-normal-method-rule}
  Given $i : ProgramAddress$, if
  \setlength{\zedindent}{0.5cm}
  \begin{circus}
    \{frameStack \neq \emptyset\} \circseq A \\
    {} = {} \\
    \{frameStack \neq \emptyset\} \circseq A \circseq \{frameStack \neq \emptyset\},
  \end{circus}
  and there exists $classInfo : Class$ in $\ran cs$ such that,
  \begin{circus}
    \{ methodID = m \land classID = c \} \circseq \lschexpract ResolveMethod \rschexpract \\
    {} = {} \\
    \{ methodID = m \land classID = c \} \circseq \lschexpract ResolveMethod \rschexpract \circseq \\
    \t1 \{ class = classInfo \land classInfo.methodEntry~m = k \},
  \end{circus}
  and, for any $x : ProgramAddress$,
  \begin{circus}
    \{ (last~(front~frameStack)).storedPC = x \} \circseq M \\
    {} = {} \\
    \{ (last~(front~frameStack)).storedPC = x \} \circseq M \circseq \{ pc = x \},
  \end{circus}
  and $m$ and $c$ do not match any of the conditions in
  Table~\ref{special-method-action-table} then,
  \setlength{\zedindent}{0.2cm}
  \setlength{\zedtab}{0.45cm}
  \begin{circus}
    \begin{array}{l}
      \circmu X \circspot \\
      \t1 \circif frameStack = \emptyset \circthen \Skip \\
      \t1 {} \circelse frameStack \neq \emptyset \circthen {} \\
      \t2 \circif \cdots \\
      \t2 {} \circelse pc = i \circthen A \circseq \{ pc = j \} \circseq \\
      \t3 \circvar poppedArgs : \seq Word \circspot \\
      \t3 \lschexpract \exists argsToPop? == e @ \\
      \t4 InterpreterStackFrameInvoke \rschexpract \circseq \\
      \t3 Invoke(c, m, poppedArgs, s) \\
      \t2 {} \circelse pc = k \circthen M \\
      \t2 \cdots \\
      \t2 \circfi \circseq Poll \circseq X \\
      \t1 \circfi 
    \end{array}
    \circrefines_A
    \begin{array}{l}
      \circmu X \circspot \\
      \t1 \circif frameStack = \emptyset \circthen \Skip \\
      \t1 {} \circelse frameStack \neq \emptyset \circthen {} \\
      \t2 \circif \cdots \\
      \t2 {} \circelse pc = i \circthen A \circseq \{ pc = j \} \circseq \\
      \t3 (\circvar poppedArgs : \seq Word \circspot \\
      \t3 \lschexpract \exists argsToPop? == e @ \\
      \t4 InterpreterStackFrameInvoke \rschexpract \circseq \\
      \t3 \lschexpract InterpreterNewStackFrame[ \\
      \t4 classInfo/class? \\
      \t4 m/methodID?, \\
      \t4 poppedArgs/methodArgs?] \rschexpract) \circseq \\
      \t3 Poll \circseq M \circseq pc := j + 1 \\
      \t2 {} \circelse pc = k \circthen M \\
      \t2 \cdots \\
      \t2 \circfi \circseq Poll \circseq X \\
      \t1 \circfi 
    \end{array}
  \end{circus}
\end{restatable}
\caption{Rule~[\nameref{resolve-normal-method-rule}]}
\label{resolve-normal-method-rule-figure}
\end{figure}
Rule~[\nameref{resolve-normal-method-rule}] applies only to those
class and method identifiers that are not handled by
Rule~[\nameref{resolve-special-method-rule}].
Because of this, Rule~[\nameref{resolve-normal-method-rule}] collapses
the choice in the $Invoke$ action, replacing it with the data
operation $InterpreterNewStackFrame$, sequenced with the action $Poll$
and the method action, $M$, defining the method's behaviour.
An assignment is placed after $M$ to set $pc$ to the address of the
next instruction.
% , allowing a sequential composition with the
% instructions after the method call to be introduced.
Rule~[\nameref{resolve-normal-method-virtual-rule}] is similar but, as
with Rule~[\nameref{resolve-special-method-virtual-rule}], handles the
$getClassIDOf$ communication.

For method calls with multiple targets, we introduce the choice over
those targets by using
Rule~[\nameref{refine-invokevirtual-multi-rule}]
(Figure~\ref{refine-invokevirtual-multi-rule-figure}), which is
applied on line~\ref{algorithm-refine-invokevirtual-multi}.
This replaces the action $HandleInvokeVirtualEPC$ with an action that
pops the arguments of the function from the stack using the data
operation $InterpreterStackFrameInvoke$, and then makes a choice of
which method to invoke using the class of the \texttt{this} argument
for the method.
The method invocations are left as references to the $Invoke$ actions,
to be resolved later in
Algorithm~\ref{resolve-method-calls-algorithm}.
\begin{figure}[thp]
\begin{restatable}[refine-invokevirtual-multi]{crule}{RefineInvokeVirtualMultiRule}
  \label{refine-invokevirtual-multi-rule}
  Given $i : ProgramAddress$,
  \setlength{\zedindent}{0.15cm}
  \setlength{\zedtab}{0.5cm}
  \begin{circus}
    \begin{array}{l}
      \{ pc = j \} \circseq \\
      \t1 HandleInvokevirtualEPC(cpi)
    \end{array}
    \circrefines_A
    \begin{array}{l}
      \{ pc = j \} \circseq \circvar poppedArgs : \seq Word \circspot \\
      \t1 \lschexpract \exists argsToPop? == methodArguments~m @ \\
      \t2 InterpreterStackFrameInvoke \rschexpract \circseq \\
      \t1 getClassIDOf!(head~poppedArgs)?cid \then {} \\
      \t1 \circif cid = c_1 \circthen Invoke(c_1, m, poppedArgs, \false) \\
      \t1 {} \cdots {} \\
      \t1 {} \circelse cid = c_n \circthen Invoke(c_n, m, poppedArgs, \false) \\
      \t1 \circfi
    \end{array}
  \end{circus}
  where $m : MethodID$ and $c_1, \ldots, c_n : ClassID$ are such that
  \begin{displaymath}
    \exists c_0 : Class; m_0 : MethodID | c_0 \in \ran cs \land m_0 \in \dom c_0.methodEntry @ \\
    \t1 cpi \in methodRefIndices~c_0 \land \\
    \t1 j \in c_0.methodEntry~m_0 \upto c_0.methodEnd~m_0 \land \\
    \t1 \exists c : ClassID @ c_0.constantPool~cpi = MethodRef~(c,m) \land \\
    \t1 \{ x : ClassID | (x,c) \in subclassRel~cs \} = \{c_1, \ldots , c_n\}
  \end{displaymath}
  and provided $n > 1$.
\end{restatable}
\caption{Rule~[\nameref{refine-invokevirtual-multi-rule}]}
\label{refine-invokevirtual-multi-rule-figure}
\end{figure}
The class identifiers used in the choice are determined by looking up
the constant pool index, $cpi$, as for
Rule~[\nameref{refine-invokestatic-rule}], to obtain a class
identifier, $c$, and method identifier, $m$.
The identifier $m$ determines which method should be invoked, but the
class of the method to be invoked is determined from the class of the
\texttt{this} object popped from the stack.
% TODO: should we reference the fact that this is mentioned above
Since Java bytecode verification ensures that the class is assignable
to $c$, we need only consider the identifiers of subclasses of
$c$:~$c_1, \cdots , c_n$.
We require that $n$ be greater than $1$, since this rule handles cases
with multiple targets.

After the action handling the method invocation instruction has been
refined to introduce a choice over the different methods, the
individual methods can be resolved.
This is performed in a way similar to the single method case, but we
must operate of each branch of the choice separately.
Thus, we iterate over each branch of the choice in the loop beginning
on line~\ref{algorithm-targets-loop}, using the function
\Call{Targets}{$mc$} to obtain a list of the possible targets of the
method call $mc$.
For each target, we apply
Rule~[\nameref{resolve-special-method-branch-rule}] or
Rule~[\nameref{resolve-normal-method-branch-rule}].
These are similar to Rule~[\nameref{resolve-special-method-rule}] and
Rule~[\nameref{resolve-normal-method-rule}], but operate over only a
single branch of the choice of targets.
We omit these rules due to their similarity with the rules previously
presented.
They can be found in Appendix~\ref{compilation-rules-appendix}.
After each of the targets has been resolved, the $pc$ assignment is
moved outside the choice by an application of
Law~[\nameref{alt-seq-dist-law}] on
line~\ref{algorithm-move-pc-assignments}.

In both the case of a single target and the case of multiple targets,
we attempt to introduce a sequential composition with the instructions
after the method call.
This is done on lines~\ref{algorithm-mci-sequence-introduction-start}
to~\ref{algorithm-mci-sequence-introduction-end} of
Algorithm~\ref{resolve-method-calls-algorithm} in the same way as in
Algorithm~\ref{introduce-loops-and-conditionals-algorithm}.
It may not be possible to introduce the sequential composition at this
point if, for example, a method call occurs at the end of a
conditional branch, since we must wait until the conditional has been
introduced before the sequential composition can be introduced.

As an example of method call resolution, we consider the
\texttt{invokestatic} instruction at $pc = 23$. 
Before method call resolution this appear in the choice in $Running$
as shown below.
\begin{circusaction}
  \t2 {} \cdots {} \\
  \t2 {} \circelse pc = 23 \circthen HandleInvokestatic(46) \\
  \t2 {} \cdots {} \\
\end{circusaction}
The $pc$ value $23$ is between the $methodEntry$ and $methodEnd$
values for $handleAsyncEvent$ in the $Class$ information $TPK$, shown
in Figure~\ref{example-model-figure}.
The constant pool index $46$ is thus looked up in $TPK$'s
$constantPool$, yielding a $MethodRef$ containing the class identifier
$TPKClassID$ and method identifier $f$.
Since the instruction being handled is an \texttt{invokestatic}
instruction, there is only a single target, which is the method
referred to by these identifiers.
That method is the \texttt{f()} method of \texttt{TPK}, whose entry
point is at $pc = 43$.
There is a straight sequence of instructions at this entry point,
ending with an \texttt{areturn} instruction. 
Thus, it has already been sequenced together at the point when method
resolution occurs for the first time, and separated into a method
action $TPK\_f$, which can be seen in
Figure~\ref{method-call-resolution-example-figure}.
This method call can thus be resolved.

The $HandleInvokestatic(46)$ action is refined using
Rule~[\nameref{refine-invokestatic-rule}]. 
We introduce an assumption $\{ pc = 23 \}$ at the start of the branch
to make the rule applicable.
After applying the rule, the sequence of actions starting at $pc = 23$
has the following form (with the $pc$ assumption left in).
\begin{circusaction}
  \t2 {} \cdots {} \\
  \t2 {} \circelse pc = 23 \circthen \{ pc = 23 \} \circseq \circvar poppedArgs : \seq Word \circspot \\
  \t3 \lschexpract \exists argsToPop? == methodArguments~m @ \\
  \t4 InterpreterStackFrameInvoke \rschexpract \circseq \\
  \t3 Invoke(TPKClassID, f, poppedArgs, \true) \\
  \t2 {} \cdots {}
\end{circusaction}

After refining the action with
Rule~[\nameref{refine-invokestatic-rule}], we resolve the method call
using Rule~[\nameref{resolve-normal-method-rule}], since \texttt{f()}
is not a special method.
The first proviso of this rule ensures that it is applied with
$k = 43$, since $TPK$ matches the class identifier $TPKClassID$ and
contains information for the method identifier $f$.
The second proviso is met, since $TPK\_f$ ends with
$HandleAreturnEPC$, which pops the last frame from the $frameStack$
and sets $pc$ to the stored value.
After the application of Rule~[\nameref{resolve-normal-method-rule}],
the sequence of actions has the form below, with the method invocation
sequenced with the $TPK\_f$ action and an assignment $pc := 24$.
\begin{circusaction}
  \t2 {} \cdots {} \\
  \t2 {} \circelse pc = 23 \circthen \{ pc = 23 \} \circseq (\circvar poppedArgs : \seq Word \circspot \\
  \t3 \lschexpract \exists argsToPop? == methodArguments~m @ InterpreterStackFrameInvoke \rschexpract \circseq \\
  \t3 \lschexpract InterpreterNewStackFrame[ \\
  \t4 TPK/class?, f/methodID?, poppedArgs/methodArgs?] \rschexpract) \circseq \\
  \t3 Poll \circseq TPK\_f \circseq pc := 24 \\
  \t2 {} \cdots {}
\end{circusaction}
The assumption on the value of $pc$ can then be removed and a
sequential composition can be introduced with the instructions at
$pc = 24$, to yield the code in
Figure~\ref{method-call-resolution-example-figure}.

As mentioned previously, the resolution of methods calls and
introduction of loops and conditionals is performed in a loop until all
the methods have been separated into their own action.
After that, the remaining use of the program counter in the main
actions of $Thr$ is eliminated as described in the next section.

\subsection{Refine Main Actions}
\label{refine-main-actions-subsection}

After all the control flow of each method has been introduced and each
method has been separated into its own method action, the only
remaining use of $pc$ is to select a method action when a method is
executed in response to a request from the $Launcher$.
This occurs in the $MainThread$ and $Started$ actions, where a call to
$Running$ follows a call to $StartInterpreter$.
To remove this final use of $pc$, we replace $Running$ with a call to
a new action, $ExecuteMethod$, which chooses a method action based on
a class and method identifier.
This performed as specified in
Algorithm~\ref{refine-main-actions-algorithm}.

\begin{algorithm}
  \begin{algorithmic}[1]
    \State \ApplyToFor{Law~[\nameref{copy-rule-law}]}{$MainThread$}{$Running$}
    \label{algorithm-MainThread-expand-Running}
    \State \ApplyToFor{Law~[\nameref{copy-rule-law}]}{$Started$}{$Running$}
    \label{algorithm-Started-expand-Running}
    \State \ApplyTo{Rule~[\nameref{Running-refinement-rule}]}{$MainThread$}
    \label{algorithm-MainThread-Running-refinement}
    \State \ApplyTo{Rule~[\nameref{Running-refinement-rule}]}{$Started$}
    \label{algorithm-Started-Running-refinement}
    \State \ApplyFor{Law~[\nameref{action-intro-law}]}{$ExecuteMethod$}
    \label{algorithm-ExecuteMethod-introduction}
    \State \ApplyToFor{Law~[\nameref{copy-rule-law}]}{$MainThread$}{$ExecuteMethod$}
    \label{algorithm-MainThread-copy-ExecuteMethod}
    \State \ApplyToFor{Law~[\nameref{copy-rule-law}]}{$Started$}{$ExecuteMethod$}
    \label{algorithm-Started-copy-ExecuteMethod}
  \end{algorithmic}
  \caption{RefineMainActions}
  \label{refine-main-actions-algorithm}
\end{algorithm}

Algorithm~\ref{refine-main-actions-algorithm} differs from previous
algorithms in that it does not operate purely upon the $Running$
action.
We instead refine the composition of $StartInterpreter$ and $Running$
in $MainThread$ and $Started$.
First, Law~[\nameref{copy-rule-law}] is applied on
lines~\ref{algorithm-MainThread-expand-Running}
and~\ref{algorithm-Started-expand-Running} to replace the call to
$Running$ with its body in $MainThread$ and $Started$.
Then, we apply Rule~[\nameref{Running-refinement-rule}], shown in
Figure~\ref{Running-refinement-rule-figure}, on
lines~\ref{algorithm-MainThread-Running-refinement}
and~\ref{algorithm-Started-Running-refinement}..
This refines the composition of $Started$ with the body of $Running$
in $MainThread$ and $Started$.

\begin{figure}[thp]
\begin{restatable}[$Running$-refinement]{crule}{RunningRefinementRule}
  \label{Running-refinement-rule}
  If $(c_1,m_1), \ldots , (c_n,m_n)$ are all the
  $ClassID \cross MethodID$ values such that
  $classID = c_i \land methodID = m_i \implies \pre ResolveMethod$,
  and for each $i \in \{1 \upto n\}$, there exists
  $classInfo_i : Class$ and $entry_i : ProgramAddress$ such that,
  \begin{circus}
    \{ classID = c_i \land methodID = m_i \} \circseq ResolveMethod \\
    {} = {} \\
    \{ classID = c_i \land methodID = m_i \} \circseq ResolveMethod \circseq \\
    \t1 \{ class = classInfo_i \land classInfo_i.methodEntry~m_i = entry_i \},
  \end{circus}
  and, for each $i \in \{1 \upto n\}$,
  \begin{circus}
    \{ \# frameStack = 1\} \circseq M_i \\
    {} = {} \\
    \{ \# frameStack = 1\} \circseq M_i \circseq \{ framestack = \emptyset\},
  \end{circus}
  then,
  \begin{circus}
    \begin{array}{l}
      StartInterpreter \circseq \circmu X \circspot \\
      \t1 \circif frameStack = \emptyset \circthen \Skip \\
      \t1 {} \circelse framestack \neq \emptyset \circthen {}  \\
      \t2 \circif pc = entry_1 \circthen M_1 \\
      \t2 {} \cdots {} \\
      \t2 {} \circelse pc = entry_n \circthen M_n \\
      \t2 \circfi \circseq Poll \circseq X \\
      \t1 \circfi
    \end{array}
    \circrefines_A
    \begin{array}{l}
      executeMethod?t \prefixcolon (t = thread) ?c?m?a \then {} \\
      (\circval classID : ClassID; \\
      \circval methodID : MethodID; \\
      \circval methodArgs : \seq Word \circspot \\
      \circif {(classID, methodID) = (c_1,m_1)} \circthen {} \\
      \t1 InterpreterNewStackFrame[\\
      \t2 classInfo_1/class?] \circseq M_1 \\
      {} \cdots {} \\
      {} \circelse (classID, methodID) = (c_n,m_n) \circthen {} \\
      \t1 InterpreterNewStackFrame[\\
      \t2 classInfo_n/class?] \circseq M_n \\ 
      \circfi)(c, m, a)
    \end{array}
  \end{circus}
\end{restatable}
\caption{Rule~[\nameref{Running-refinement-rule}]}
\label{Running-refinement-rule-figure}
\end{figure}

Rule~[\nameref{Running-refinement-rule}] collects all the class and
method identifiers that are resolved by the data operation
$ResolveMethod$.
It expands the definiton of $StartInterpreter$, and introduces a
choice over these class and method identifiers, comparing them to the
identifiers communicated on the $executeMethod$ channel.
Within each branch of the choice for a class identifier $c_i$ and
method identifier $m_i$, a new stack frame is first created by the
$InterpreterNewStackFrame$ operation, using the class information,
$classInfo_i$, provided by $ResolveMethod$ for $c_i$ and $m_i$.
The branch then behaves as the method action in $Running$ that
corresponds to the method entry point, $entry_i$, associated with
$m_i$ in $classInfo_i$.
Note that the mapping from class and method identifiers to class
information and method entries is not necessarily injective, since
inherited methods will share the same class information and bytecode
instructions.
The choice over class and method identifiers is wrapped in a value
parameter block, since it forms the body of the $ExecuteMethod$
action.

After Rule~[\nameref{Running-refinement-rule}] has been applied, the
$ExecuteMethod$ action is introduced to the $Thr$ process using
Law~[\nameref{action-intro-law}], on
line~\ref{algorithm-ExecuteMethod-introduction} of
Algorithm~\ref{refine-main-actions-algorithm}.
The body of $ExecuteMethod$ introduced to $MainThread$ and
$NotStarted$ by Rule~[\nameref{Running-refinement-rule}] is then
replaced with a call to $ExecuteMethod$ by application of
Law~[\nameref{copy-rule-law}], on
lines~\ref{algorithm-MainThread-copy-ExecuteMethod}
and~\ref{algorithm-Started-copy-ExecuteMethod}.
This results in $MainThread$ and $Started$ having the form shown
previously in Figure~\ref{refine-main-actions-example-figure}.

\subsection{Remove \texorpdfstring{$pc$}{pc} From State}
\label{remove-pc-from-state-subsection}

After $MainThread$ and $Started$ have been refined, $pc$ is no longer
used by $Thr$, and so we can remove it from the state of $Thr$, as
specified in Algorithm~\ref{pc-elimination-algorithm}.
This algorithm operates over the $Thr$ process as a whole, since $pc$
must be removed from every action in $Thr$ simultaneously.

\begin{algorithm}
  \begin{algorithmic}[1]
    \State \ApplyFor{Law~[\nameref{forwards-data-refinement-law}]}{$InterpreterStateEPC$, $CI$}
    \label{algorithm-pc-data-refinement}
    \State \Apply{Law~[\nameref{seq-unitl-law}]}
    \label{algorithm-eliminate-Skips}
    \State \ApplyFor{Law~[\nameref{process-param-elim-law}]}{$bc$}
    \label{algorithm-eliminate-bc-parameter}
  \end{algorithmic}
  \caption{RemovePCFromState}
  \label{pc-elimination-algorithm}
\end{algorithm}

Algorithm~\ref{pc-elimination-algorithm} begins with the application
of Law~[\nameref{forwards-data-refinement-law}] at
line~\ref{algorithm-pc-data-refinement}.
This law describes a standard \Circus{} data refinement between
processes, in which a coupling invariant is defined to describe the
relationship between the old state of the process and the new state of
the process.
We characterise the refinement by providing the new process state and
the coupling invariant.
In this case, the relation defined by the coupling invariant is a
function, so the actions of the new process can be calculated from the
actions of the old process.
Thus, the new state and coupling invariant are sufficient to uniquely
characterise the data refinement.

For our refinement, the new state is $InterpreterStateEPC$, shown
below.
It is similar to $InterpreterState$, but the $pc$ component is
removed.
The $frameStack$ also has a different type, being a sequence of
$StackFrameEPC$ structures, which are similar to $StackFrame$, but
without the $storedPC$ component, since that is only used for storing
a value from $pc$.
The invariant of $InterpreterStateEPC$ is the same as for
$InterpreterState$, but without the requirement that the
$currentClass$ and stack frame $frameClass$ values be consistent with
the $pc$ value.
\begin{schema}{InterpreterStateEPC}
  frameStack : \seq StackFrameEPC \\
  % pc : ProgramAddress \\
  currentClass : Class
\where
  % definition of currentClass (only important if the frame stack is nonempty)
  frameStack \neq \emptyset \implies currentClass = (last~frameStack).frameClass
  % need to ensure pc is consistent with currentClass
  % currentClass = (\mu c : \ran cs | \\
  % \t1 (\exists_1 m : MethodID | m \in \dom c.methodEntry @  \\
  % \t2 pc \in c.methodEntry~m \upto c.methodEnd~m)) \\
  % \forall f : \ran frameStack @ f.frameClass = (\mu c :\ran cs | \\
  % \t1 (\exists_1 m : MethodID | m \in \dom c.methodEntry @  \\
  % \t2 f.storedPC \in c.methodEntry~m \upto c.methodEnd~m))
\end{schema}

% \begin{schema}{StackFrameEPC}
%   localVariables : \seq Word \\
%   operandStack : \seq Word \\
%   % storedPC : ProgramAddress \\
%   frameClass : Class \\
%   stackSize : \nat
% \where
%   \# operandStack \leq stackSize
% \end{schema}

The coupling invariant, $CI$, for the refinement from
$InterpreterState$ to $InterpreterStateEPC$ is shown below, with
$InterpreterStateEPC$ decorated with ${}_1$ to distinguish its
components.
$CI$ equates the $currentClass$ components of the two schemas, since
they are unaffected.
The $frameStack$ components are declared to have the same domain, and
each $StackFrame$ in the $frameStack$ is mapped onto a $StackFrameEPC$
with the same $localVariables$, $operandStack$, $frameClass$, and
$stackSize$ values.
The $pc$ and $storedPC$ values of $InterpreterState$ are discarded,
since they are not present in $InterpreterStateEPC$.
\begin{schema}{CI}
  InterpreterState \\
  InterpreterStateEPC_1
\where
  currentClass = currentClass_1 \\
  \dom frameStack = \dom frameStack_1 \\
  \forall i : \dom frameStack @ \\
  \t1 (frameStack~i).localVariables = (frameStack_1~i).localVariables \land \\
  \t1 (frameStack~i).operandStack = (frameStack_1~i).operandStack \land \\
  \t1 (frameStack~i).frameClass = (frameStack_1~i).frameClass \land \\
  \t1 (frameStack~i).stackSize = (frameStack_1~i).stackSize
\end{schema}

This data refinement has the effect of removing $pc$ from each of the
data operations in $Thr$.
This effect is minimal for most data operations, since their $pc$
updates have already been extracted. 
However, $InterpreterStackFrameInvoke$ no longer stores the current
$pc$ value in the $storedPC$ component of the topmost stack frame, and
$InterpreterNewStackFrame$ does not set the $pc$ value.
Additionally, the method return operations $InterpreterAreturn$ and
$IntepreterReturn$ do not set set the value of $pc$ using the
$storedPC$ value of the previous stack frame.

The $pc$ assignments introduced between bytecode instructions during
the strategy are also affected by the data refinement.
The data refinement removes $pc$ from the assignments, leaving only
their effect on the other components of the state.
Since the assignments leave all other state components unchanged, the
data-refined $pc$ assignments have no effect, making them equivalent
to $\Skip$.
These $\Skip$ actions are then eliminated by applying
Law~[\nameref{seq-unitl-law}] wherever possible, at
line~\ref{algorithm-eliminate-Skips} of
Algorithm~\ref{pc-elimination-algorithm}.

Finally, we eliminate the $bc$ parameter to the process, since it is
also no longer needed, using an application of
Law~[\nameref{process-param-elim-law}], at
line~\ref{algorithm-eliminate-bc-parameter}.
This completes the refinement of $Thr(bc,cs,t)$ into
$ThrCF_{bc,cs}(cs,t)$, referenced in
Theorem~\ref{epc-thm}, which has its control flow
introduced and does not include $pc$ in its state.
The next stage of the strategy operates on $ThrCF_{bc,cs}(cs,t)$ to
eliminate the $frameStack$.


\section{Elimination of Frame Stack}
\label{elimination-of-frame-stack-section}

The second stage of the compilation strategy eliminates the
$frameStack$ from the state of each thread's process,
$ThrCF_{bc,cs}(cs,t)$. 
The information stored in the stack frames on $frameStack$ are
transferred into variables representing the local variables and
operand stack slots for each method.
The operations representing the bytecode instructions are refined to
operations over these variables.
This refines $ThrCF_{bc,cs}(cs,t)$ to the $CThr_{bc,cs}(t)$ process
described in Section~\ref{cee-c-program-subsection}, so this stage may
be summarised by the following theorem.
%
\begin{thm}[Elimination of Frame Stack]\label{efs-thm}
  \begin{circus}
    ThrCF_{bc,cs}(cs,t) \circrefines CThr_{bc,cs}(t)
  \end{circus}
\end{thm}
%

In this stage, we operate mainly on the method actions introduced in
the previous stage.
Algorithm~\ref{efs-algorithm} describes the strategy for transforming
the method actions to introduce variables and eliminate the
$frameStack$.
\begin{algorithm}[tp!]
  \begin{algorithmic}[1]
    \State \Call{RemoveLauncherReturns}{}
    \label{algorithm-remove-launcher-returns}
    % \State \Call{RemoveCurrentClassAndFrameStackIDFromState}{}
    % \label{algorithm-remove-currentClass}
    \State \Call{LocaliseStackFrames}{}
    \label{algorithm-localise-stack-frames}
    \State \Call{IntroduceVariables}{}
    \label{algorithm-introduce-variables}
    \State \Call{RemoveFrameStackFromState}{}
    \label{algorithm-remove-frameStack}
  \end{algorithmic}
  \caption{Elimination of Frame Stack}
  \label{efs-algorithm}
\end{algorithm}
It begins on line~\ref{algorithm-remove-launcher-returns}, by refining
the return instructions that occur at the end of each method to remove
the $CheckLauncherReturn$ actions that occur in those instructions,
resolving the check of whether $frameStack$ is empty.
This removes the only remaining use of $frameStack$ as a whole,
enabling us to consider the stack frames for each method individually.
We introduce a variable in each method that contains its stack
frame on line~\ref{algorithm-localise-stack-frames} of the algorithm,
and convert the operations of the method to operate over the new
variable rather than the global $frameStack$.
Then, on line~\ref{algorithm-introduce-variables}, we perform local
data refinements to convert the stack frame for each method into
variables representing the local variables and operand stack slots of
the method.
Finally, we eliminate the, now unused, $frameStack$ from the state of
the process, on line~\ref{algorithm-remove-frameStack}.

We discuss each of these steps in more detail in separate sections,
explaining them with reference to the running example introduced in
Section~\ref{overview-subsection}.
The removal of launcher returns is discussed first, in
Section~\ref{remove-launcher-returns-subsection}.
Then, the localisation of stack frames is discussed in
Section~\ref{localise-stack-frames-subsection}, followed by variable
introduction in Section~\ref{introduce-variables-subsection}.
Finally we discuss the remove of $frameStack$ from the state of the
process, in Section~\ref{remove-frameStack-from-state-subsection}.


\subsection{Remove Launcher Returns}
\label{remove-launcher-returns-subsection}

After the previous stage, each conditional branch in a method ends
with a return instruction or an infinite loop.
This can be seen in
Figure~\ref{pc-elimination-HandleAsyncEvent-example-figure}, presented
earlier, where the method $TPK\_handleAsyncEvent$ ends with a
$HandleReturnEPC$ action.
In the first step of this stage, at
line~\ref{algorithm-remove-launcher-returns} of
Algorithm~\ref{efs-algorithm}, such actions are moved outside the
method and their definitions are expanded so that their communication
with the $Launcher$ can be handled.
This is performed as described in
Algorithm~\ref{remove-launcher-returns-algorithm}, which defines the
procedure \Call{RemoveLauncherReturns}{}.
\begin{algorithm}[ht]
  \begin{algorithmic}[1]
    \For{$methodName \gets $ \Call{MethodActionNames}{$cs$}}
    \label{algorithm-method-single-return-loop}
    \State $methodBody \gets$ \Call{ActionBody}{$methodName$}
    \State $returnAction \gets$ \Call{ReturnAction}{$methodBody$}
    \label{algorithm-determine-return-action}   
    \State \ExhaustivelyApplyToFor{Law~[\nameref{rec-action-intro-law}]}{$methodBody$}{$returnAction$}
    \label{algorithm-introduce-infinite-loop-returns}
    \State \ExhaustivelyApplyTo{Law~[\nameref{alt-seq-dist-law}]}{$methodBody$}
    \label{algorithm-distribute-returns}
    \State \Call{RedefineMethodExcludingReturn}{$methodName$,$returnAction$}
    \label{algorithm-redefine-method-action-excluding-return-action}
    \EndFor
    \State \Call{IntroduceFrameStackAssumptions}{}
    \label{algorithm-introduce-frameStack-assumptions}
    \State \ExhaustivelyApply{Rule~[\nameref{refine-HandleReturnEPC-empty-frameStack-rule}]}
    \label{algorithm-eliminate-returns-try-begin}
    \State \ExhaustivelyApply{Rule~[\nameref{refine-HandleReturnEPC-nonempty-frameStack-rule}]}
    \State \ExhaustivelyApply{Rule~[\nameref{refine-HandleAreturnEPC-empty-frameStack-rule}]}
    \State \ExhaustivelyApply{Rule~[\nameref{refine-HandleAreturnEPC-nonempty-frameStack-rule}]}
    \label{algorithm-eliminate-returns-try-end}
    \State \ExhaustivelyApply{Law~[\nameref{assump-elim-law}]}
    \label{algorithm-remove-launcher-returns-assump-elim}
    \State \ExhaustivelyApply{Law~[\nameref{seq-unitl-law}]}
    \label{algorithm-remove-launcher-returns-seq-unitl}
    \State \ApplyToFor{Law~[\nameref{copy-rule-law}]}{\Call{ActionBody}{$MainThread$}}{$ExecuteMethod$}
    \label{algorithm-copy-ExecuteMethod-in-MainThread}
    \State \ApplyToFor{Law~[\nameref{copy-rule-law}]}{\Call{ActionBody}{$Started$}}{$ExecuteMethod$}
    \label{algorithm-copy-ExecuteMethod-in-Started}
    \State \ApplyTo{Rule~[\nameref{ExecuteMethod-refinement-rule}]}{\Call{ActionBody}{$MainThread$}}
    \label{algorithm-ExecuteMethod-refinement-MainThread}
    \State \ApplyTo{Rule~[\nameref{ExecuteMethod-refinement-rule}]}{\Call{ActionBody}{$Started$}}
    \label{algorithm-ExecuteMethod-refinement-Started}
    \State \ApplyReverseFor{Law~[\nameref{action-intro-law}]}{$ExecuteMethod$, \Call{ActionBody}{$ExecuteMethod$}}
    \label{algorithm-remove-ExecuteMethod}
    \MatchIn{$\begin{array}[t]{l}(\circvar retVal : Word \circspot (A)(c, m, a, retVal) \circseq \\\t1 executeMethodRet!thread!retVal \then \Skip)\end{array}$\\$\t2$}{\Call{ActionBody}{$Started$}}
  \State \ApplyFor{Law~[\nameref{action-intro-law}]}{$ExecuteMethod$, $A$}
    \label{algorithm-reintroduce-ExecuteMethod}
    \State \ApplyReverseToFor{Law~[\nameref{copy-rule-law}]}{$MainThread$}{$ExecuteMethod$}
    \label{algorithm-copy-ExecuteMethod-out-MainThread}
    \State \ApplyReverseToFor{Law~[\nameref{copy-rule-law}]}{$Started$}{$ExecuteMethod$}
    \label{algorithm-copy-ExecuteMethod-out-Started}
  \end{algorithmic}
  \caption{\Call{RemoveLauncherReturns}{}}
  \label{remove-launcher-returns-algorithm}
\end{algorithm}
  
Algorithm~\ref{remove-launcher-returns-algorithm} begins by iterating
over each of the method actions, in the for loop beginning on
line~\ref{algorithm-method-single-return-loop}.
This determines the name, $methodName$, for each method's action from
the class information, $cs$, via a function
\Call{MethodActionNames}{}.
We take the method's body, $methodBody$, as the body of the action
corresponding to $methodName$.

The return actions that may occur at the end of method branches are
either $HandleAreturnEPC$ or $HandleReturnEPC$.
$HandleAreturnEPC$ occurs only in methods that return a value and,
conversely, $HandleReturnEPC$ occurs only in methods that do not
return a value.
We can thus determine which return action a method uses by examining
$methodBody$ to see which action occurs at the end of the branches.
A method in which all branches end in infinite loops is treated as
using the return action $HandleReturnEPC$, since it does not produce a
value.
We determine the return action type, $returnAction$, for $methodBody$
on line~\ref{algorithm-determine-return-action}, using a syntactic
function \Call{ReturnAction}{}.

With $returnAction$ identified, we convert the method to a form in
which it has one occurrence of that action at the end of its body.
This is achieved by introducing occurrences of $returnAction$ after
infinite loops in $methodBody$ using
Law~[\nameref{rec-action-intro-law}] and distributing occurrences of
$returnAction$ outside conditionals using
Law~[\nameref{alt-seq-dist-law}].
These laws are applied on
lines~\ref{algorithm-introduce-infinite-loop-returns}
and~\ref{algorithm-distribute-returns} of
Algorithm~\ref{remove-launcher-returns-algorithm}.

When the method has a single return instruction at the end, it is
redefined to exclude the return action.
This is performed using Law~[\nameref{copy-rule-law}] and
Law~[\nameref{action-intro-law}], but since the use of these laws to
redefine an action in this way is standard, it is specified in a
separate procedure
\Call{RedefineMethodExcludingReturn}{$methodName$,$returnAction$},
called on
line~\ref{algorithm-redefine-method-action-excluding-return-action} of
Algorithm~\ref{remove-launcher-returns-algorithm}.
This procedure is defined in
Algorithm~\ref{redefine-method-action-excluding-return-action-algorithm},
which is included in
Appendix~\ref{remove-launcher-returns-appendix-subsection}.
% This is performed by introducing an intermediate action named
% $methodName'$ on
% line~\ref{algorithm-remove-returns-introduce-method-action}, using
% Law~[\nameref{action-intro-law}], containing the actions of
% $methodBody$ before $returnAction$.
% Those actions are then replaced with a reference to $methodName'$ on
% line~\ref{algorithm-eliminate-returns-copy-method-out} via an
% application of Law~[\nameref{copy-rule-law}].
% The definition of $methodName$ is then expanded everywhere on
% line~\ref{algorithm-eliminate-returns-copy-method-in}, replacing
% references to it with a reference to $methodName'$ and $returnAction$.
% The $methodName$ action is then removed on
% line~\ref{algorithm-remove-returns-eliminate-method-action} and the
% $methodName'$ action is renamed to $methodName$ on
% line~\ref{algorithm-rename-method-action}, using
% Law~[\nameref{action-rename-law}], which allows renaming an action to
% a fresh name.

We then introduce assumptions that state the depth of the
$frameStack$, so that we can determine whether the $frameStack$ is
empty or not at each return instruction.
This introduction of assumptions is performed by the call to the
\Call{IntroduceFrameStackAssumptions}{} procedure on
line~\ref{algorithm-introduce-frameStack-assumptions}.
It is defined by
Algorithm~\ref{introduce-frameStack-assumptions-algorithm}, which is
included in Appendix~\ref{remove-launcher-returns-appendix-subsection}
along with the rules used by it.

This procedure introduces an assumption $\{\# frameStack = 0 \}$ from
the $InterpreterInitEPC$ schema (which is the result of applying the
data refinement in Section~\ref{remove-pc-from-state-subsection} to
$InterpreterInit$) at the start of the main action of $ThrCF_{bc,cs}$.
The assumption is then distributed throughout the process by
exhaustive application of restricted versions of standard algebraic
assumption-distribution laws, and rules stating how the size of
$frameStack$ is affected by the operations that appear in the code
resulting from the elimination of program counter.

The restrictions added to these laws, in the form of extra provisos,
guarantee that the assumption is not distributed if an identical
assumption is already in place, thus preventing unbounded distribution
of the assumptions and ensuring the procedure terminates.
% As the procedure is a straightforward application of these rules and
% laws, we omit its definition here.
The result is that the return instructions following the method
actions in $ExecuteMethod$ have an assumption $\# frameStack = 1$
before them, and the return instructions occurring in the middle of
other methods have an assumption $\# frameStack = k$ for some $k > 1$.

% \begin{figure}[t!]
%   \centering
%   \setlength{\zedtab}{0.4cm}
%   \setlength{\zedindent}{0pt}
%   \setlength{\zedleftsep}{0pt}
%   \setlength{\abovedisplayskip}{0pt}
%   \setlength{\belowdisplayskip}{0pt}
%   \setlength{\abovedisplayshortskip}{0pt}
%   \setlength{\belowdisplayshortskip}{0pt}
%   \begin{circusaction}
%     ExecuteMethod \circdef \\
%     \t1 \circval classID : ClassID; \circval methodID : MethodID; \circval methodArgs : \seq Word \circspot \\
%     \t1 \circif (classID, methodID) = (TPKClassID, APEHinit) \circthen {} \\
%     \t2 InterpreterNewStackFrame[TPK/class?] \circseq \\
%     \t2 TPK\_APEHinit \circseq HandleReturnEPC \circseq \{ framestack = \emptyset \} \\
%     \t1 {} \circelse (classID, methodID) = (TPKClassID, handleAsyncEvent) \circthen {} \\
%     \t2 InterpreterNewStackFrame[TPK/class?] \circseq \\
%     \t2 TPK\_handleAsyncEvent \circseq HandleReturnEPC \circseq \{ framestack = \emptyset \} \\
%     \t1 {} \circelse (classID, methodID) = (TPKClassID, f) \circthen {} \\
%     \t2 InterpreterNewStackFrame[TPK/class?] \circseq \\
%     \t2 TPK\_f \circseq HandleAreturnEPC \circseq \{ framestack = \emptyset \} \\
%     \t1 {} \cdots {} \\
%     \t1 \circfi
%   \end{circusaction}

%   \begin{circusaction}
%     TPK\_handleAsyncEvent \circdef \\
%     \t1 HandleNewEPC(27) \circseq Poll \circseq HandleDupEPC \circseq Poll \circseq  HandleAconst\_nullEPC \circseq Poll \circseq \\
%     \t1 (\circvar poppedArgs : \seq Word \circspot \\
%     \t2 \lschexpract \exists argsToPop? == m + 1 @ InterpreterStackFrameInvoke \rschexpract \circseq \\
%     \t2 \lschexpract InterpreterNewStackFrame[\\
%     \t3 ConsoleConnection/class?, CCinit/methodID?, poppedArgs/methodArgs?] \rschexpract) \circseq Poll \circseq \\
%     \t1 ConsoleConnection\_CCinit \circseq HandleReturnEPC \circseq \{ frameStack \neq \emptyset \} \circseq Poll \circseq \\
%     %\t1 HandleAstoreEPC(1) \circseq Poll \circseq HandleAloadEPC(1) \circseq \\
%     \t1 {} \cdots {} \\
%     % \t1 Poll \circseq (\circvar poppedArgs : \seq Word \circspot \lschexpract \exists argsToPop? == m + 1 @ InterpreterStackFrameInvoke \rschexpract \circseq \\
%     % \t1 getClassIDOf!(head~poppedArgs)?cid \then \lschexpract InterpreterNewStackFrame[ \\
%     % \t2 ConsoleConnection/class?, openInputStream/methodID?, poppedArgs/methodArgs?] \rschexpract) \circseq \\
%     % \t1 Poll \circseq ConsoleConnection\_openInputStream \circseq Poll \circseq  HandleAstoreEPC(2) \circseq Poll \circseq \\
%     % \t1 HandleAloadEPC(1) \circseq Poll \circseq (\circvar poppedArgs : \seq Word \circspot \\
%     % \t1 \lschexpract \exists argsToPop? == m + 1 @ InterpreterStackFrameInvoke \rschexpract \circseq \\
%     % \t1 getClassIDOf!(head~poppedArgs)?cid \then \lschexpract InterpreterNewStackFrame[\\
%     % \t2 ConsoleConnection/class?, openOutputStream/methodID?, poppedArgs/methodArgs?] \rschexpract) \circseq \\
%     % \t1 Poll \circseq ConsoleConnection\_openOutputStream \circseq Poll \circseq HandleAstoreEPC(3) \circseq \\
%     %\t1 {} \cdots {} \\
%     % \t1 Poll \circseq HandleIconstEPC(0) \circseq Poll \circseq HandleAstoreEPC(4) \circseq Poll \circseq Poll \circseq \circmu Y \circspot \\
%     % \t2 HandleAloadEPC(4) \circseq Poll \circseq HandleIconstEPC(10) \circseq Poll \circseq \\
%     % \t2 \circvar value1, value2 : Word \circspot InterpreterPop2 \circseq \\
%     % \t2 \circif value1 \leq value2 \circthen \\
%     % \t3 {} \cdots {}  \\
%     % Poll \circseq HandleAloadEPC(2) \circseq Poll \circseq \\
%     % \t3 (\circvar poppedArgs : \seq Word \circspot \\
%     % \t4 \lschexpract \exists argsToPop? == m + 1 @ InterpreterStackFrameInvoke \rschexpract \circseq \\
%     % \t4 getClassIDOf!(head~poppedArgs)?cid \then \lschexpract InterpreterNewStackFrame[ \\
%     % \t5 ConsoleInput/class?, read/methodID?, poppedArgs/methodArgs?] \rschexpract) \circseq \\
%     % \t3 Poll \circseq ConsoleInput\_read \circseq Poll \circseq \\
%     \t3 (\circvar poppedArgs : \seq Word \circspot \\
%     \t4 \lschexpract \exists argsToPop? == m @ InterpreterStackFrameInvoke \rschexpract \circseq \\
%     \t4 \lschexpract InterpreterNewStackFrame[\\
%     \t5 TPK/class?, f/methodID?, poppedArgs/methodArgs?] \rschexpract) \circseq \\
%     \t3 Poll \circseq TPK\_f \circseq HandleAreturnEPC \circseq \{ frameStack \neq \emptyset \} \circseq Poll \circseq \\
%     % \t3 HandleAstoreEPC(5) \circseq Poll \circseq HandleAloadEPC(5) \circseq \\
%     \t3 {} \cdots {} \\
%     % \t3 Poll \circseq HandleIconstEPC(400) \circseq Poll \circseq \circvar value1, value2 : Word \circspot InterpreterPop2 \circseq \\
%     % \t3 \circif value1 \leq value2 \circthen HandleAloadEPC(3) \circseq Poll \circseq HandleAloadEPC(5) \circseq Poll \circseq \\
%     % \t4 (\circvar poppedArgs : \seq Word \circspot \lschexpract \exists argsToPop? == m + 1 @ InterpreterStackFrameInvoke \rschexpract \circseq \\
%     % \t4 getClassIDOf!(head~poppedArgs)?cid \then \lschexpract InterpreterNewStackFrame[ \\
%     % \t5 ConsoleOutput/class?, write/methodID?, poppedArgs/methodArgs?] \rschexpract)) \circseq \\
%     % \t4 Poll \circseq ConsoleOutput\_write \\
%     % \t3 {} \circelse value1 > value2 \circthen HandleAloadEPC(3) \circseq Poll \circseq HandleIconstEPC(0) \circseq Poll \circseq \\
%     % \t4 (\circvar poppedArgs : \seq Word \circspot \lschexpract \exists argsToPop? == m + 1 @ InterpreterStackFrameInvoke \rschexpract \circseq \\
%     % \t4 getClassIDOf!(head~poppedArgs)?cid \then \lschexpract InterpreterNewStackFrame[ \\
%     % \t5 ConsoleOutput/class?, write/methodID?, poppedArgs/methodArgs?] \rschexpract)) \circseq \\
%     % \t4 Poll \circseq ConsoleOutput\_write \\
%     % \t3 \circfi \circseq Poll \circseq HandleAloadEPC(4) \circseq Poll \circseq HandleIconstEPC(1) \circseq Poll \circseq HandleIaddEPC \circseq \\
%     \t3 Poll \circseq HandleAstoreEPC(4) \circseq Poll \circseq Y \\
%     \t2 {} \circelse value1 > value2 \circthen \Skip \\
%     \t2 \circfi \circseq Poll
%   \end{circusaction}
%   \caption{The $ExecuteMethod$ and $TPK\_handleAsyncEvent$ actions
%     after introduction of $frameStack$ assumptions}
%   \label{efs-return-assumption-distribution-figure}
% \end{figure}

After assumptions on the state of the $frameStack$ have been
introduced, we can handle the return actions at each point where they
occur, applying the rules on
lines~\ref{algorithm-eliminate-returns-try-begin}
to~\ref{algorithm-eliminate-returns-try-end} of
Algorithm~\ref{remove-launcher-returns-algorithm} wherever possible.
An example is
Rule~[\nameref{refine-HandleReturnEPC-empty-frameStack-rule}], shown
in Figure~\ref{refine-HandleReturnEPC-empty-frameStack-rule-figure}.
This rule replaces an occurrence of $HandleReturnEPC$ where the
$frameStack$ has a size of $1$, with a call to the data operation
$InterpreterReturnEPC$ followed by a communication with the $Launcher$
on the $executeMethodRet$ channel.
The value communicated on $executeMethodRet$ is $returnValue$,
introduced in a variable block.

Rule~[\nameref{refine-HandleReturnEPC-empty-frameStack-rule}]
essentially expands the definition of the $HandleReturnEPC$ action,
shown below, and resolves the choice in $CheckLauncherReturn$
(presented earlier in Section~\ref{cee-interpreter-subsection}) over
whether $frameStack$ is empty.
This involves distributing the assumption over the data operation
$InterpreterReturnEPC$, which removes the last stack frame from the
$frameStack$, causing it to be empty when $\# frameStack = 1$. 
\begin{circus}
  HandleReturnEPC \circdef \circvar returnValue : Word \circspot \\
  \t1 \lschexpract InterpreterReturnEPC \rschexpract \circseq CheckLauncherReturn(returnValue)
\end{circus}

\begin{figure}[tp!]
  \begin{restatable}[refine-$HandleReturnEPC$-empty-$frameStack$]{crule}{RefineHandleReturnEPCEmptyFrameStackRule}
  \label{refine-HandleReturnEPC-empty-frameStack-rule}
  \begin{circus}
    \begin{array}{l}
      \{\# frameStack = 1\} \circseq \\
      HandleReturnEPC
    \end{array}
    \circrefines_A
    \begin{array}{l}
      \circvar returnValue : Word \circspot \\
      \lschexpract InterpreterReturnEPC \rschexpract \circseq \\
      executeMethodRet!thread!returnValue \then \Skip
    \end{array}
  \end{circus}
\end{restatable}
\caption{Rule~[\nameref{refine-HandleReturnEPC-empty-frameStack-rule}]}
\label{refine-HandleReturnEPC-empty-frameStack-rule-figure}
\end{figure}

The other rules used on lines~\ref{algorithm-eliminate-returns-try-begin}
to~\ref{algorithm-eliminate-returns-try-end} are similar, handling the
cases for the $frameStack$ being left nonempty and the
$HandleAreturnEPC$ action.
They can be found in Appendix~\ref{compilation-rules-appendix}.

In the cases when the $frameStack$ is not empty after execution of the
return instruction, which occurs when a method is called from within
another method, the resolution of the choice in $CheckLauncherReturn$
collapses it to the branch where there is no communication on
$executeMethodRet$.
The variable block for $returnValue$ is thus removed as part of the
rules handling those cases, since it is not used.

After application of these rules, any remaining assumptions are
eliminated by application of Law~[\nameref{assump-elim-law}] and
Law~[\nameref{seq-unitl-law}] on
lines~\ref{algorithm-remove-launcher-returns-assump-elim}
and~\ref{algorithm-remove-launcher-returns-seq-unitl}, since they are
no longer necessary.

\begin{figure}[tp!]
  \centering
  % \setlength{\zedtab}{0.4cm}
  % \setlength{\zedindent}{0pt}
  % \setlength{\zedleftsep}{0pt}
  % \setlength{\abovedisplayskip}{0pt}
  % \setlength{\belowdisplayskip}{0pt}
  % \setlength{\abovedisplayshortskip}{0pt}
  % \setlength{\belowdisplayshortskip}{0pt}
  \begin{circusaction}
    ExecuteMethod \circdef \\
    \t1 \circval classID : ClassID; \circval methodID : MethodID; \circval methodArgs : \seq Word \circspot \\
    \t1 \circif (classID, methodID) = (TPKClassID, APEHinit) \circthen {} \\
    \t2 InterpreterNewStackFrame[TPK/class?, APEHinit/methodID?] \circseq Poll \circseq \\
    \t2 TPK\_APEHinit \circseq \\
    \t2 (\circvar returnValue : Word \circspot \lschexpract InterpreterReturnEPC \rschexpract \circseq \\
    \t3 executeMethodRet!thread!returnValue \then \Skip) \\
    \t1 {} \circelse (classID, methodID) = (TPKClassID, handleAsyncEvent) \circthen {} \\
    \t2 InterpreterNewStackFrame[TPK/class?, handleAsyncEvent/methodID?] \circseq Poll \circseq \\
    \t2 TPK\_handleAsyncEvent \circseq \\
    \t2 (\circvar returnValue : Word \circspot \lschexpract InterpreterReturnEPC \rschexpract \circseq \\
    \t3 executeMethodRet!thread!returnValue \then \Skip) \\
    \t1 {} \circelse (classID, methodID) = (TPKClassID, f) \circthen {} \\
    \t2 InterpreterNewStackFrame[TPK/class?, f/methodID?] \circseq Poll \circseq \\
    \t2 TPK\_f \circseq \\
    \t2 (\circvar returnValue : Word \circspot \lschexpract InterpreterAreturn2EPC \rschexpract \circseq \\
    \t3 executeMethodRet!thread!returnValue \then \Skip) \\
    \t1 {} \cdots {} \\
    \t1 \circfi
  \end{circusaction}
  \caption{$ExecuteMethod$ after refining return instructions}
  \label{efs-ExecutMethod-refined-return-instructions-figure}
\end{figure}

Since the return action at the end of each method in $ExecuteMethod$
causes $frameStack$ to be empty, the application of these
transformations to our running example results in the $ExecuteMethod$
action shown in
Figure~\ref{efs-ExecutMethod-refined-return-instructions-figure}.
Each method action is followed by a $returnValue$ variable block with
a data operation followed by an $executeMethodRet$ communication,
which result from the refinement of the return actions.
While the data operations differ depending on whether a value is
returned from the method or not, the variable block and
$executeMethodRet$ communication are the same for each method.
We thus distribute them outside $ExecuteMethod$, to avoid their
unecessary duplication in each of the branches of $ExecuteMethod$.
This is performed by first replacing $ExecuteMethod$ with its
definition, via an application of Law~[\nameref{copy-rule-law}] on
lines~\ref{algorithm-copy-ExecuteMethod-in-MainThread}
and~\ref{algorithm-copy-ExecuteMethod-in-Started}, then applying
Rule~[\nameref{ExecuteMethod-refinement-rule}], shown in
Figure~\ref{ExecuteMethod-refinement-rule-figure}, on
lines~\ref{algorithm-ExecuteMethod-refinement-MainThread}
and~\ref{algorithm-ExecuteMethod-refinement-Started}, to distribute
the variable block and communication.

\begin{figure}[tbhp!]
  \begin{restatable}[$ExecuteMethod$-refinement]{crule}{ExecuteMethodRefinementRule}
    \label{ExecuteMethod-refinement-rule}
    If, for all $i$, $returnValue$ is not free in $A_i$, then
    \setlength{\zedindent}{0.5cm}
    \setlength{\abovedisplayskip}{0pt}
    \setlength{\belowdisplayskip}{0pt}
    \setlength{\abovedisplayshortskip}{0pt}
    \setlength{\belowdisplayshortskip}{0pt}
  \begin{circus}
    \begin{array}{l}
      (\circval classID : ClassID; \\
      \circval methodID : MethodID; \\
      \circval methodArgs : \seq Word \circspot \\
      \circif {(classID, methodID) = (c_1,m_1)} \circthen {} \\
      \t1 A_1 \circseq \circvar returnValue : Word \circspot  B_1 \circseq \\
      \t1 executeMethodRet!thread!returnValue \\
      \t1 {} \then \Skip \\
      {} \cdots {} \\
      {} \circelse (classID, methodID) = (c_n,m_n) \circthen {} \\
      \t1 A_n \circseq \circvar returnValue : Word \circspot B_n \circseq \\
      \t1 executeMethodRet!thread!returnValue \\
      \t1 {} \then \Skip \\
      \circfi)(c, m, a)
    \end{array}
    \circrefines_A
    \begin{array}{l}
      \circvar retVal : Word \circspot \\
       (\circval classID : ClassID; \\
      \circval methodID : MethodID; \\
      \circval methodArgs : \seq Word; \\
      \circres returnValue : Word \circspot \\
      \circif {(classID, methodID) = (c_1,m_1)} \circthen {} \\
      \t1 A_1 \circseq B_1 \\
      {} \cdots {} \\
      {} \circelse (classID, methodID) = (c_n,m_n) \circthen {} \\
      \t1 A_n \circseq B_n \\
      \circfi)(c, m, a, retVal) \circseq \\
      executeMethodRet!thread!retVal \\
      {} \then \Skip 
    \end{array}
  \end{circus}
\end{restatable}
\caption{Rule~[\nameref{ExecuteMethod-refinement-rule}]}
\label{ExecuteMethod-refinement-rule-figure}
\end{figure}

After Rule~[\nameref{ExecuteMethod-refinement-rule}] has been applied,
$ExecuteMethod$ is redefined as the actions inside the parametrised
block on the left-hand side of that rule.
This is performed using Law~[\nameref{action-intro-law}] on
lines~\ref{algorithm-remove-ExecuteMethod}
to~\ref{algorithm-reintroduce-ExecuteMethod}, eliminating the existing
definition of $ExecuteMethod$ and introducing a new definition.
The actions are then copied back out using the new definition by
application of Law~[\nameref{copy-rule-law}] on
lines~\ref{algorithm-copy-ExecuteMethod-out-MainThread}
and~\ref{algorithm-copy-ExecuteMethod-out-Started}.
This results in the $MainThread$ and $Started$ actions shown in
Figure~\ref{efs-eliminate-returns-MainThread-Started-figure}, where we
have highlighted the parts of those actions that have changed as a
result of applying the rules in this section.
Note that, since these actions have a very specific format, we can be
sure that the application of Law~[\nameref{copy-rule-law}] replaces
only the intended components of those actions.
Within the body of a method, the return actions have been refined to
simple data operations with no $executeMethodRet$ communication, as
can be seen in
Figure~\ref{efs-eliminate-returns-handleAsyncEvent-example-figure},
which shows the form of $TPK\_handleAsyncEvent$, with those data
operations highlighted.

\begin{figure}[tp!]
  \centering
  \setlength{\zedtab}{0.4cm}
  \setlength{\zedindent}{0pt}
  \setlength{\zedleftsep}{0pt}
  \setlength{\abovedisplayskip}{0pt}
  \setlength{\belowdisplayskip}{0pt}
  \setlength{\abovedisplayshortskip}{0pt}
  \setlength{\belowdisplayshortskip}{0pt}
  \begin{circusaction}
    MainThread \circdef \\
    \t1 setStack?t \prefixcolon (t = thread) ?stack \then frameStackID := Initialised~stack \circseq \circmu X \circspot \\
    \t1 \circblockbegin
    \colorbox{lightgray}{$\circvar retVal : Word \circspot$}
    executeMethod? t \prefixcolon (t = thread)?c?m?a \then {} \\
    \t1 \colorbox{lightgray}{$ExecuteMethod(c, m, a, retVal) \circseq
    executeMethodRet!thread!retVal \then Poll$} \circseq  X \\
    {} \extchoice {} \\
    CEEswitchThread?from?to \prefixcolon (from = thread) \then Blocked \circseq X
    \circblockend
  \end{circusaction}
  
  \begin{circusaction}
    Started \circdef \\
    \t1 \circblockbegin
    \colorbox{lightgray}{$\circvar retVal : Word \circspot$} executeMethod? t \prefixcolon (t = thread)?c?m?a \then {} \circseq \\
    \t1 \colorbox{lightgray}{$ExecuteMethod(c, m, a, retVal) \circseq executeMethodRet!thread!retVal \then Poll$} \circseq \\
    \t1 \circblockbegin
    continue?t \prefixcolon (t = thread) \then Started \\
    {} \extchoice {} \\
    endThread?t \prefixcolon (t = thread) \then \Skip
    \circblockend \\
    {} \extchoice {} \\
    CEEswitchThread?from?to \prefixcolon (from = thread) \then Blocked \circseq Started \\
    {} \extchoice {} \\
    endThread?t \prefixcolon (t = thread) \then \Skip
    \circblockend \circseq \\
    \t1 removeThreadMemory!thread \then CEEremoveThread!thread \\
    \t1 {} \then CEEswitchThread?from?to \prefixcolon (from = thread) \then NotStarted
  \end{circusaction}
  \caption{$MainThread$ and $Started$ after $Launcher$ return
    elimination}
  \label{efs-eliminate-returns-MainThread-Started-figure}
\end{figure}

\begin{figure}[tp!]
  \centering
  \setlength{\zedtab}{0.4cm}
  \setlength{\zedindent}{0pt}
  \setlength{\zedleftsep}{0pt}
  \setlength{\abovedisplayskip}{0pt}
  \setlength{\belowdisplayskip}{0pt}
  \setlength{\abovedisplayshortskip}{0pt}
  \setlength{\belowdisplayshortskip}{0pt}
  \begin{circusaction}
    TPK\_handleAsyncEvent \circdef \\
    \t1 HandleNewEPC(27) \circseq Poll \circseq HandleDupEPC \circseq Poll \circseq  HandleAconst\_nullEPC \circseq Poll \circseq \\
    \t1 (\circvar poppedArgs : \seq Word \circspot \\
    \t2 \lschexpract \exists argsToPop? == 2 @ InterpreterStackFrameInvoke \rschexpract \circseq \\
    \t2 \lschexpract InterpreterNewStackFrame[\\
    \t3 ConsoleConnection/class?, CCinit/methodID?, poppedArgs/methodArgs?] \rschexpract) \circseq Poll \circseq \\
    \t1 ConsoleConnection\_CCinit \circseq \colorbox{lightgray}{$\lschexpract InterpreterReturnEPC \rschexpract$} \circseq Poll \circseq \\
    %\t1 HandleAstoreEPC(1) \circseq Poll \circseq HandleAloadEPC(1) \circseq \\
    \t1 {} \cdots {} \\
    % \t1 Poll \circseq (\circvar poppedArgs : \seq Word \circspot \lschexpract \exists argsToPop? == m + 1 @ InterpreterStackFrameInvoke \rschexpract \circseq \\
    % \t1 getClassIDOf!(head~poppedArgs)?cid \then \lschexpract InterpreterNewStackFrame[ \\
    % \t2 ConsoleConnection/class?, openInputStream/methodID?, poppedArgs/methodArgs?] \rschexpract) \circseq \\
    % \t1 Poll \circseq ConsoleConnection\_openInputStream \circseq Poll \circseq  HandleAstoreEPC(2) \circseq Poll \circseq \\
    % \t1 HandleAloadEPC(1) \circseq Poll \circseq (\circvar poppedArgs : \seq Word \circspot \\
    % \t1 \lschexpract \exists argsToPop? == m + 1 @ InterpreterStackFrameInvoke \rschexpract \circseq \\
    % \t1 getClassIDOf!(head~poppedArgs)?cid \then \lschexpract InterpreterNewStackFrame[\\
    % \t2 ConsoleConnection/class?, openOutputStream/methodID?, poppedArgs/methodArgs?] \rschexpract) \circseq \\
    % \t1 Poll \circseq ConsoleConnection\_openOutputStream \circseq Poll \circseq HandleAstoreEPC(3) \circseq \\
    %\t1 {} \cdots {} \\
    % \t1 Poll \circseq HandleIconstEPC(0) \circseq Poll \circseq HandleAstoreEPC(4) \circseq Poll \circseq Poll \circseq \circmu Y \circspot \\
    % \t2 HandleAloadEPC(4) \circseq Poll \circseq HandleIconstEPC(10) \circseq Poll \circseq \\
    % \t2 \circvar value1, value2 : Word \circspot InterpreterPop2 \circseq \\
    % \t2 \circif value1 \leq value2 \circthen \\
    % \t3 {} \cdots {}  \\
    % Poll \circseq HandleAloadEPC(2) \circseq Poll \circseq \\
    % \t3 (\circvar poppedArgs : \seq Word \circspot \\
    % \t4 \lschexpract \exists argsToPop? == m + 1 @ InterpreterStackFrameInvoke \rschexpract \circseq \\
    % \t4 getClassIDOf!(head~poppedArgs)?cid \then \lschexpract InterpreterNewStackFrame[ \\
    % \t5 ConsoleInput/class?, read/methodID?, poppedArgs/methodArgs?] \rschexpract) \circseq \\
    % \t3 Poll \circseq ConsoleInput\_read \circseq Poll \circseq \\
    \t3 (\circvar poppedArgs : \seq Word \circspot \\
    \t4 \lschexpract \exists argsToPop? == 1 @ InterpreterStackFrameInvoke \rschexpract \circseq \\
    \t4 \lschexpract InterpreterNewStackFrame[\\
    \t5 TPK/class?, f/methodID?, poppedArgs/methodArgs?] \rschexpract) \circseq \\
    \t3 Poll \circseq TPK\_f \circseq \colorbox{lightgray}{$\lschexpract InterpreterAreturn1EPC \rschexpract$} \circseq Poll \circseq \\
    % \t3 HandleAstoreEPC(5) \circseq Poll \circseq HandleAloadEPC(5) \circseq \\
    \t3 {} \cdots {} \\
    % \t3 Poll \circseq HandleIconstEPC(400) \circseq Poll \circseq \circvar value1, value2 : Word \circspot InterpreterPop2 \circseq \\
    % \t3 \circif value1 \leq value2 \circthen HandleAloadEPC(3) \circseq Poll \circseq HandleAloadEPC(5) \circseq Poll \circseq \\
    % \t4 (\circvar poppedArgs : \seq Word \circspot \lschexpract \exists argsToPop? == m + 1 @ InterpreterStackFrameInvoke \rschexpract \circseq \\
    % \t4 getClassIDOf!(head~poppedArgs)?cid \then \lschexpract InterpreterNewStackFrame[ \\
    % \t5 ConsoleOutput/class?, write/methodID?, poppedArgs/methodArgs?] \rschexpract)) \circseq \\
    % \t4 Poll \circseq ConsoleOutput\_write \\
    % \t3 {} \circelse value1 > value2 \circthen HandleAloadEPC(3) \circseq Poll \circseq HandleIconstEPC(0) \circseq Poll \circseq \\
    % \t4 (\circvar poppedArgs : \seq Word \circspot \lschexpract \exists argsToPop? == m + 1 @ InterpreterStackFrameInvoke \rschexpract \circseq \\
    % \t4 getClassIDOf!(head~poppedArgs)?cid \then \lschexpract InterpreterNewStackFrame[ \\
    % \t5 ConsoleOutput/class?, write/methodID?, poppedArgs/methodArgs?] \rschexpract)) \circseq \\
    % \t4 Poll \circseq ConsoleOutput\_write \\
    % \t3 \circfi \circseq Poll \circseq HandleAloadEPC(4) \circseq Poll \circseq HandleIconstEPC(1) \circseq Poll \circseq HandleIaddEPC \circseq \\
    \t3 Poll \circseq HandleAstoreEPC(4) \circseq Poll \circseq Y \\
    \t2 {} \circelse value1 > value2 \circthen \Skip \\
    \t2 \circfi \circseq Poll 
  \end{circusaction}
  \caption{$TPK\_handleAsyncEvent$ after $Launcher$ return
    elimination}
  \label{efs-eliminate-returns-handleAsyncEvent-example-figure}
\end{figure}

\subsection{Localise Stack Frames}
\label{localise-stack-frames-subsection}

After the $CheckLauncherReturn$ actions have been handled, the process
no longer has any actions that use the whole $frameStack$.
We can therefore refine each method to only operate on a local stack
frame variable.
This is performed as described in
Algorithm~\ref{localise-stack-frames-algorithm}, which defines the
procedure \Call{LocaliseStackFrames}{}.

\begin{algorithm}
  \begin{algorithmic}[1]
    \arraycolsep=0cm
    \State \ApplyFor{Law~[\nameref{forwards-data-refinement-law}]}{$InterpreterStateFS$, $FrameStackCI$}
    \label{algorithm-remove-currentClass-data-refinement}
    \State $iterationOrder \gets$ \Call{MethodDependencyOrder}{}
    \label{algorithm-method-dependency-order-call}
    \For{$methodName \gets iterationOrder$}
    \label{algorithm-localise-stack-frames-loop}
    \State $numArgs \gets$ \Call{MethodArguments}{$methodName$}
    \State $methodBody \gets$ \Call{ActionBody}{$methodName$}
    \State \ExhaustivelyApplyFor{Rule~[\nameref{arguments-introduction-rule}]}{$methodName$, $numArgs$}
    \label{algorithm-arguments-introduction}
    \MatchThen{%
      $\begin{array}[t]{l}
         (\circval arg1, \ldots, arg{<}n{>} : Word \circspot \\
         \t1 \lschexpract \exists methodArgs? == \langle arg1, \ldots, arg{<}n{>} \rangle @ \\
         \t2 InterpreterNewStackFrame[c/class?, m/methodID?] \rschexpract \circseq \\
         \t1 methodName \circseq \lschexpract InterpreterReturn \rschexpract)(args~1, \ldots, args~n)
       \end{array}$}
     \State \ApplyFor{Law~[\nameref{action-intro-law}]}{$methodName'$, %
       $\begin{array}[t]{l}
          (\circval arg1, \ldots, arg{<}n{>} : Word \circspot \\
          \t1 \lschexpract \exists methodArgs? \\
          \t3 {} == \langle arg1, \ldots, arg{<}n{>} \rangle @ \\
          \t2 InterpreterNewStackFrame[ \\
          \t3 c/class?, m/methodID?] \rschexpract \circseq \\
          \t1 methodName \circseq \lschexpract InterpreterReturn \rschexpract)
        \end{array}$}
        \label{algorithm-localise-stack-frames-new-method-action}
      \State \ExhaustivelyApplyReverseFor{Law~[\nameref{copy-rule-law}]}{$methodName'$}
      \label{algorithm-localise-stack-frames-copy-out}
      \State \ApplyToFor{Law~[\nameref{copy-rule-law}]}{\Call{ActionBody}{$methodName'$}}{$methodName$}
      \label{algorithm-localise-stack-frames-copy-in}
      \State \ApplyReverseFor{Law~[\nameref{action-intro-law}]}{$methodName$, $methodBody$}
      \label{algorithm-localise-stack-frames-method-elim}
      \State \ApplyFor{Law~[\nameref{action-rename-law}]}{$methodName'$, $methodName$}
      \label{algorithm-localise-stack-frames-method-rename}
      \State \ApplyToFor{Rule~[\nameref{HandleReturnEPC-stackFrame-introduction-rule}]}{\Call{ActionBody}{$methodName$}}{$numArgs$}
      \label{algorithm-HandleReturnEPC-stackFrame-introduction-rule}
    \EndFor
  \end{algorithmic}
  \caption{LocaliseStackFrames}
  \label{localise-stack-frames-algorithm}
\end{algorithm}
 
Since we must be able to operate directly on the $frameStack$ when
introducing stack frame variables, we first apply a data refinement to
remove $currentClass$ from the state.
We defined $currentClass$ in the model as a convenience when accessing
the $frameClass$ of the topmost stack frame, which is no longer
necessary when we have separate variables for each stack frame.
We also remove $frameStackID$ at this stage, since its only purpose is
to guard whether frameStack is permitted to be non-empty, which holds
from the form of the model.
The data refinement is applied on
line~\ref{algorithm-remove-currentClass-data-refinement} of
Algorithm~\ref{localise-stack-frames-algorithm}, and transforms the
state to $InterpreterStateFS$, shown below, which only contains
$frameStack$.
\begin{schema}{InterpreterStateFS}
  frameStack : StackFrameEPC
\end{schema}

The relationship between $InterpreterStateEPC$ and
$InterpreterStateFS$ is described by the coupling invariant
$FrameStackCI$, shown below.
It ensures $frameStack$ is unaffected by the refinement and replaces
occurrences of $currentClass$ with $(last~frameStack).frameClass$.
The $frameStackID$ is hidden from the new state.
\begin{schema}{FrameStackCI}
  InterpreterStateEPC \\
  InterpreterStateFS_1
\where
  frameStack = frameStack_1 \\
  currentClass = (last~frameStack_1).frameClass
\end{schema}

$FrameStackCI$ describes a functional data refinement, so the new
actions can be calculated in each case.
None of the actions in the process read from $frameStackID$, so it can
be safely hidden and the operations to set the $frameStackID$ are
reduced to $\Skip$.
As mentioned above, $(last~frameStack).frameClass$ is inserted
wherever $currentClass$ occurs in the old actions.
We can then proceed with introducing stack frame variables.

When introducing the variables to represent stack frames, we must
begin with those stack frames at the greatest depth on the stack.
This ensures uses of the $frameStack$ within a nested method do not
interfere with replacing uses of the $frameStack$ in an outer method.
The order in which we introduce stack frame variables to the methods
is specified by a procedure \Call{MethodDependencyOrder}{}, which
constructs a sequence of method action names indicating the order in
which the method actions should be handled.
This sequence is constructed by first adding to the sequence any
methods that contain no method calls, then adding any methods that
only call methods already in the sequence, and repeating until all
methods are in the sequence.
Since we do not allow recursion, this will always terminate.
We construct this sequence, $iterationOrder$, on
line~\ref{algorithm-method-dependency-order-call} of
Algorithm~\ref{localise-stack-frames-algorithm}.

We then loop, introducing a stack frame variable for each method in
the order specified by $iterationOrder$, in the for loop on
line~\ref{algorithm-localise-stack-frames-loop}.
Within the for loop, we first introduce value parameters, representing
the arguments to the method, around the call to the method action.
This ensures that the body of the method is completely independent of
the context in which it is called, enabling us to separate the whole
method body (including stack frame creation and return actions) into
its own action.
Introduction of method arguments is performed using
Rule~[\nameref{arguments-introduction-rule}], shown in
Figure~\ref{arguments-introduction-rule-figure}.
This rule is applied to two parameters:~$methodName$, the name of the
method being considered, and $numArgs$, the number of arguments to the
method (which is encoded as part of the method identifier, and so can
be determined during the strategy).
We apply this rule everywhere it applies on
line~\ref{algorithm-HandleReturnEPC-stackFrame-introduction-rule}.

\begin{figure}[thp]
\begin{restatable}[$Return$-arguments-intro]{crule}{ArgumentsIntroductionRule}
  \label{arguments-introduction-rule}
  %\setlength{\zedtab}{0.4cm}
  %\setlength{\zedindent}{0pt}
  %\setlength{\zedleftsep}{0pt}
  Given an action name $M$ and $n : \nat$,
  \begin{circus}
    \begin{array}{l}
      \lschexpract InterpreterNewStackFrame[ \\
      \t1 c/class?, \\
      \t1 m/methodID?, \\
      \t1 args/methodArgs?] \rschexpract \circseq \\
      M \circseq \lschexpract InterpreterReturn \rschexpract
    \end{array}
    \circrefines_A
    \begin{array}{l}
      (\circval arg1, \ldots, arg{<}n{>} : Word \circspot \\
      \t1 \lschexpract \exists methodArgs? == \langle arg1, \ldots, arg{<}n{>} \rangle @ \\
      \t2 InterpreterNewStackFrame[ \\
      \t3 c/class?, \\
      \t3 m/methodID?] \rschexpract \circseq \\
      \t1 M \circseq \lschexpract InterpreterReturn \rschexpract \\
      )(args~1, \ldots, args~n)
    \end{array}
  \end{circus}
  where $\ell = c.methodLocals~m$ and $s = c.methodStackSize~m$.
\end{restatable}
\caption{Rule~[\nameref{arguments-introduction-rule}]}
\label{arguments-introduction-rule-figure}
\end{figure}

Rule~[\nameref{arguments-introduction-rule}] introduces value
parameters representing method arguments around a method ending with
an $InterpreterReturn$ operation.
The arguments array passed into the $methodArgs$ input of
$InterpreterNewStackFrame$ is split into its individual elements,
which are passed into the parameters and recombined to be passed into
$InterpreterNewStackFrame$.
This splitting of the array ensures that the individual arguments can
be more easily handled in the next step, where we introduce local
variables.

After the method arguments have been introduced, we redefine the
method action to include the value parameters representing those
arguments.
That is performed by first introducing a temporary method action,
$methodName'$, matching the parametrised block around the call to
$methodName$, via an application of Law~[\nameref{action-intro-law}]
on line~\ref{algorithm-localise-stack-frames-new-method-action}.
The occurrences of the body of $methodName'$ are replaced with a
reference to the new action using Law~[\nameref{copy-rule-law}] on
line~\ref{algorithm-localise-stack-frames-copy-out}.
The occurrence of $methodName$ in the body of $methodName'$ is then
expanded using Law~[\nameref{copy-rule-law}] and $methodName$ is
eliminated using Law~[\nameref{action-intro-law}], since it no longer
occurs in the process.
Finally, the temporary action $methodName'$ is renamed to $methodName$
using Law~[\nameref{action-rename-law}] on
line~\ref{algorithm-localise-stack-frames-method-rename}.

Having completely separated the method into its own, independent,
action, we then introduce the stack frame variable for the method
using Rule~[\nameref{HandleReturnEPC-stackFrame-introduction-rule}],
shown in
Figure~\ref{HandleReturnEPC-stackFrame-introduction-rule-figure}.
This is applied to the body of $methodName$ on
line~\ref{algorithm-HandleReturnEPC-stackFrame-introduction-rule},
with the number of arguments, $numArgs$, passed to it.

\begin{figure}[thp]
\begin{restatable}[$Return$-$stackFrame$-intro]{crule}{HandleReturnEPCStackFrameIntroductionRule}
  \label{HandleReturnEPC-stackFrame-introduction-rule}
  \setlength{\zedtab}{0.4cm}
  \setlength{\zedindent}{0.5cm}
  %\setlength{\zedleftsep}{0pt}
  Given $n : \nat$, if $A$ operates solely on $last~frameStack$ and do
  not change the length of $frameStack$, then
  \begin{circus}
    \begin{array}{l}
      \lschexpract \exists methodArgs? == \langle arg1, \ldots, arg{<}n{>} \rangle @ \\
      \t2 InterpreterNewStackFrame[ \\
      \t3 c/class?, \\
      \t3 m/methodID? \rschexpract \circseq \\
      A \circseq \lschexpract InterpreterReturn \rschexpract
    \end{array}
    \circrefines_A
    \begin{array}{l}
      \circvar stackFrame : StackFrameEPC \circspot \\
      \t1 \lschexpract [arg1?, \ldots, arg{<}n{>}? : Word; \\
      \t1 stackFrame' : StackFrameEPC  | \\
      \t2 \langle arg1?, \ldots, arg{<}n{>}? \rangle \\
      \t3 {} \subseteq stackFrame'.localVariables \land \\
      \t2 \# stackFrame'.localVariables = \ell \land \\
      \t2 stackFrame'.operandStack = \langle\rangle \land \\
      \t2 stackFrame'.frameClass = c \land \\
      \t2 stackFrame'.stackSize = s] \rschexpract \circseq \\
      \t1 A[stackFrame/last~frameStack, \\
      \t2 stackFrame'/last~frameStack']
    \end{array}
  \end{circus}
  where $\ell = c.methodLocals~m$ and $s = c.methodStackSize~m$.
\end{restatable}
\caption{Rule~[\nameref{HandleReturnEPC-stackFrame-introduction-rule}]}
\label{HandleReturnEPC-stackFrame-introduction-rule-figure}
\end{figure}

Rule~[\nameref{HandleReturnEPC-stackFrame-introduction-rule}]
introduces a variable $stackFrame$, of type $StackFrameEPC$, over the
body of a method that ends with an $InterpreterReturn$ operation.
The $stackFrame$ variable is initialised in the same way as for the
stack frame created by $InterpreterNewStackFrame$, and each reference
to $last~frameStack$ in the body of the method is replaced with a
reference to $stackFrame$.
Replacing the references to $last~frameStack$ requires that the size
of $frameStack$ does not change during the method.
However, this requirement is met since method calls are the only
operation that changes the size of $frameStack$ and we replace
references to the $frameStack$ in nested methods first, by the
definition of $iterationOrder$.

Note that the operations performed on
lines~\ref{algorithm-arguments-introduction}
and~\ref{algorithm-HandleReturnEPC-stackFrame-introduction-rule}
specifically handle methods that do not return a value.
We omit the handling of methods that do return a value. 
Handling such methods would require rules similar to
Rule~[\nameref{arguments-introduction-rule}] for method bodies
followed by $InterpreterAreturn1$ and $InterpreterAreturn2$, which
would introduce a result parameter for the method in addition to the
value parameters representing the method's arguments.
The new method action would then have to match the different method
parameters.
We would also require a rule similar
Rule~[\nameref{HandleReturnEPC-stackFrame-introduction-rule}] to
handle the slightly different ending of the method action that the
return handling would create.
These rules would be applied in a way similar to the existing rules.

In our example, a $stackFrame$ variable will be introduced for
$TPK\_f$ first, since it does not call any other methods.
$TPK\_handleAsyncEvent$ only has its $stackFrame$ variable introduced
after $stackFrame$ variables have been introduced for all methods
called in $TPK\_handleAsyncEvent$.
When the $stackFrame$ variable for $TPK\_handleAsyncEvent$ has been
introduced, it has the form shown in
Figure~\ref{efs-localise-stack-frames-example-figure}.

For brevity, we define new actions, which we refer to as $Handle*SF$
actions.
These are not formally introduced as actions in the compilation
strategy as they are an abbreviation used for presenting examples and
stating compilation rules.
They are refined to a different form later in the elimination of frame
stack stage.
The $Handle*SF$ actions are similar to the $Handle*EPC$ actions,
except they have every reference to $last~frameStack$ (or
$last~frameStack'$) replaced with a reference to $stackFrame$ (or
$stackFrame'$), and have undergone the data refinement described
above.
We name them by replacing $EPC$ in the names of the $Handle*EPC$
actions with $SF$.
Similarly, we define an $InvokeSF$ operation that performs the
operation of $InterpreterStackFrameInvoke$ over $stackFrame$ instead
of $last~frameStack$.

% TODO: fix this for algorithm changes
\begin{figure}[tp!]
  \centering
  \setlength{\zedtab}{0.4cm}
  \setlength{\zedindent}{0pt}
  \setlength{\zedleftsep}{0pt}
  \setlength{\abovedisplayskip}{0pt}
  \setlength{\belowdisplayskip}{0pt}
  \setlength{\abovedisplayshortskip}{0pt}
  \setlength{\belowdisplayshortskip}{0pt}
  \begin{circusaction}
    TPK\_f \circdef \\
    \t1 \circval arg1 : Word \circspot \\
    \t1 \circvar stackFrame : StackFrameEPC \circspot \\
    \t1 \lschexpract [arg1? : Word; stackFrame' : StackFrameEPC | \\
    \t2 \langle arg1? \rangle \subseteq stackFrame'.localVariables \land \\
    \t2 \# stackFrame'.localVariables = 6 \land \\
    \t2 stackFrame'.operandStack = \langle\rangle \land \\
    \t2 stackFrame'.frameClass = TPK \land \\
    \t2 stackFrame'.stackSize = 3] \rschexpract \circseq \\
    \t1 Poll \circseq HandleNewSF(27) \circseq Poll \circseq HandleDupSF \circseq Poll \circseq  HandleAconst\_nullSF \circseq Poll \circseq \\
    \t1 (\circvar poppedArgs : \seq Word \circspot \\
    \t2 \lschexpract \exists argsToPop? == 2 @ InvokeSF \rschexpract \circseq \\
    \t2 ConsoleConnection\_CCinit(poppedArgs~1, poppedArgs~2)) \circseq Poll \circseq \\
    \t1 HandleAstoreSF(1) \circseq Poll \circseq HandleAloadSF(1) \circseq \\
    \t1 {} \cdots {} \\
    % \t1 Poll \circseq (\circvar poppedArgs : \seq Word \circspot \lschexpract \exists argsToPop? == m + 1 @ InterpreterStackFrameInvoke \rschexpract \circseq \\
    % \t1 getClassIDOf!(head~poppedArgs)?cid \then \lschexpract InterpreterNewStackFrame[ \\
    % \t2 ConsoleConnection/class?, openInputStream/methodID?, poppedArgs/methodArgs?] \rschexpract) \circseq \\
    % \t1 Poll \circseq ConsoleConnection\_openInputStream \circseq Poll \circseq  HandleAstoreEPC(2) \circseq Poll \circseq \\
    % \t1 HandleAloadEPC(1) \circseq Poll \circseq (\circvar poppedArgs : \seq Word \circspot \\
    % \t1 \lschexpract \exists argsToPop? == m + 1 @ InterpreterStackFrameInvoke \rschexpract \circseq \\
    % \t1 getClassIDOf!(head~poppedArgs)?cid \then \lschexpract InterpreterNewStackFrame[\\
    % \t2 ConsoleConnection/class?, openOutputStream/methodID?, poppedArgs/methodArgs?] \rschexpract) \circseq \\
    % \t1 Poll \circseq ConsoleConnection\_openOutputStream \circseq Poll \circseq HandleAstoreEPC(3) \circseq \\
    %\t1 {} \cdots {} \\
    \t1 Poll \circseq HandleIconstSF(0) \circseq Poll \circseq HandleAstoreSF(4) \circseq Poll \circseq Poll \circseq \circmu Y \circspot \\
    \t2 HandleAloadSF(4) \circseq Poll \circseq HandleIconstSF(10) \circseq Poll \circseq \\
    \t2 \circvar value1, value2 : Word \circspot \\
    \t3 \lschexpract InterpreterPop2[stackFrame/last~frameStack,stackFrame'/last~frameStack'] \rschexpract \circseq \\
    \t2 \circif value1 \leq value2 \circthen {} \\
    \t3 {} \cdots {}  \\
    % Poll \circseq HandleAloadEPC(2) \circseq Poll \circseq \\
    % \t3 (\circvar poppedArgs : \seq Word \circspot \\
    % \t4 \lschexpract \exists argsToPop? == m + 1 @ InterpreterStackFrameInvoke \rschexpract \circseq \\
    % \t4 getClassIDOf!(head~poppedArgs)?cid \then \lschexpract InterpreterNewStackFrame[ \\
    % \t5 ConsoleInput/class?, read/methodID?, poppedArgs/methodArgs?] \rschexpract) \circseq \\
    % \t3 Poll \circseq ConsoleInput\_read \circseq Poll \circseq \\
    % \t3 (\circvar poppedArgs : \seq Word \circspot \\
    % \t4 \lschexpract \exists argsToPop? == 1 @ InvokeSF \rschexpract \circseq \\
    % \t4 TPK\_f()) \circseq \lschexpract InterpreterAreturn1 \rschexpract \circseq Poll \circseq \\
    % \t3 HandleAstoreEPC(5) \circseq Poll \circseq HandleAloadEPC(5) \circseq \\
    % \t3 {} \cdots {} \\
    % \t3 Poll \circseq HandleIconstEPC(400) \circseq Poll \circseq \circvar value1, value2 : Word \circspot InterpreterPop2 \circseq \\
    % \t3 \circif value1 \leq value2 \circthen HandleAloadEPC(3) \circseq Poll \circseq HandleAloadEPC(5) \circseq Poll \circseq \\
    % \t4 (\circvar poppedArgs : \seq Word \circspot \lschexpract \exists argsToPop? == m + 1 @ InterpreterStackFrameInvoke \rschexpract \circseq \\
    % \t4 getClassIDOf!(head~poppedArgs)?cid \then \lschexpract InterpreterNewStackFrame[ \\
    % \t5 ConsoleOutput/class?, write/methodID?, poppedArgs/methodArgs?] \rschexpract)) \circseq \\
    % \t4 Poll \circseq ConsoleOutput\_write \\
    % \t3 {} \circelse value1 > value2 \circthen HandleAloadEPC(3) \circseq Poll \circseq HandleIconstEPC(0) \circseq Poll \circseq \\
    % \t4 (\circvar poppedArgs : \seq Word \circspot \lschexpract \exists argsToPop? == m + 1 @ InterpreterStackFrameInvoke \rschexpract \circseq \\
    % \t4 getClassIDOf!(head~poppedArgs)?cid \then \lschexpract InterpreterNewStackFrame[ \\
    % \t5 ConsoleOutput/class?, write/methodID?, poppedArgs/methodArgs?] \rschexpract)) \circseq \\
    % \t4 Poll \circseq ConsoleOutput\_write \\
    % \t3 \circfi \circseq Poll \circseq HandleAloadEPC(4) \circseq Poll \circseq HandleIconstEPC(1) \circseq Poll \circseq HandleIaddEPC \circseq \\
    \t3 Poll \circseq HandleAstoreSF(4) \circseq Poll \circseq Y \\
    \t2 {} \circelse value1 > value2 \circthen \Skip \\
    \t2 \circfi \circseq Poll
  \end{circusaction}
  \caption{$TPK\_handleAsyncEvent$ after its $stackFrame$ variable is
    introduced}
  \label{efs-localise-stack-frames-example-figure}
\end{figure}

\subsection{Introduce Variables}
\label{introduce-variables-subsection}

Following the introduction of local $stackFrame$ variables, we perform
local data refinements to introduce the local variables and stack
slots for each $stackFrame$, on
line~\ref{algorithm-introduce-variables} of
Algorithm~\ref{efs-algorithm}.
This is performed as described in
Algorithm~\ref{introduce-variables-algorithm}, which defines the
\Call{IntroduceVariables}{} procedure.

\begin{algorithm}
  \begin{algorithmic}[1]
    \For{$methodName \gets$ \Call{MethodNames}{$cs$}}
    \label{algortihm-introduce-variables-loop}
    \State \Call{IntroduceFrameClassAssumptions}{{\Call{ActionBody}{$methodName$}}}
    \label{algorithm-introduce-frameClass-assumptions}
    \State \ExhaustivelyApplyTo{Rule~[\nameref{refine-PutfieldSF-rule}]}{\Call{ActionBody}{$methodName$}}
    \label{algorithm-apply-refine-PutfieldSF}
    \State \ExhaustivelyApplyTo{Rule~[\nameref{refine-GetfieldSF-rule}]}{\Call{ActionBody}{$methodName$}}
    \State \ExhaustivelyApplyTo{Rule~[\nameref{refine-PutstaticSF-rule}]}{\Call{ActionBody}{$methodName$}}
    \State \ExhaustivelyApplyTo{Rule~[\nameref{refine-GetstaticSF-rule}]}{\Call{ActionBody}{$methodName$}}
    \State \ExhaustivelyApplyTo{Rule~[\nameref{refine-NewSF-rule}]}{\Call{ActionBody}{$methodName$}}
    \label{algorithm-apply-refine-NewSF}
    \State \ExhaustivelyApplyTo{Law~[\nameref{assump-elim-law}]}{\Call{ActionBody}{$methodName$}}
    \label{algorithm-frameClass-assump-elim}
    \State \ExhaustivelyApplyTo{Law~[\nameref{seq-unitl-law}]}{\Call{ActionBody}{$methodName$}}
    \label{algorithm-frameClass-seq-unitl}
    % introduce operand stack assumptions
    \State \Call{IntroduceOperandStackAssumptions}{\Call{ActionBody}{$methodName$}}
    \label{algorithm-introduce-operandStack-assumptions}
    \State $\ell \gets$ \Call{MethodLocals}{$methodName$}
    \label{algorithm-get-method-locals}
    \State $s \gets$ \Call{MethodStackSize}{$methodName$}
    \label{algorithm-get-method-stack-size}
    \State {\bf apply}
    \label{algorithm-local-data-refinement}
    {Law~[\nameref{forwards-data-refinement-law}]}
    ({\arraycolsep=0cm
      $\begin{array}[t]{l}
         [var1, \ldots, var{<}\ell{>} : Word;\\
          stack1, \ldots, stack{<}s{>} : Word]
       \end{array}$,
       $V{<}\ell{>}S{<}s{>}CI$})
     \Statex $\t1${\bf to} {\Call{ActionBody}{$methodName$}}
    \State \ExhaustivelyApplyTo{Law~[\nameref{assump-elim-law}]}{\Call{ActionBody}{$methodName$}}
    \label{algorithm-operandStack-assump-elim}
    \State \ExhaustivelyApplyTo{Law~[\nameref{seq-unitl-law}]}{\Call{ActionBody}{$methodName$}}
    \label{algorithm-operandStack-seq-unitl}
    \State \ExhaustivelyApplyTo{Rule~[\nameref{eliminate-value1-value2-conditional-rule}]}{\Call{ActionBody}{$methodName$}}
    \label{algorithm-apply-eliminate-value1-value2-conditional-rule}
    \State \ExhaustivelyApplyTo{Rule~[\nameref{eliminate-oid-getField-rule}]}{\Call{ActionBody}{$methodName$}}
    \State \ExhaustivelyApplyTo{Rule~[\nameref{eliminate-oid-value-putField-rule}]}{\Call{ActionBody}{$methodName$}}
    \State \ExhaustivelyApplyTo{Rule~[\nameref{eliminate-value-putStatic-rule}]}{\Call{ActionBody}{$methodName$}}
    \label{algorithm-apply-eliminate-value-putStatic-rule}
    \State \ExhaustivelyApply{Rule~[\nameref{method-parameter-introduction-rule}]}
    \label{algorithm-poppedArgs-elimination}
    \State \ExhaustivelyApply{Rule~[\nameref{poppedArgs-sync-elim-rule}]}
    %\State \ExhaustivelyApply{Rule~[\nameref{getClassIDOf-method-parameter-introduction-rule}]}
    \State \ExhaustivelyApply{Rule~[\nameref{getClassIDOf-multi-method-parameter-introduction-rule}]}
    \label{algorithm-multi-poppedArgs-elimination}
    \MatchIn{
      $\circmu X \circspot A \circseq
      \begin{array}[t]{l}
        \circif b \circthen B \circseq X \\
        {} \circelse \lnot b \circthen \Skip \\
        \circfi
      \end{array}$
     }{\Call{ActionBody}{$methodName$}}
     \State \ExhaustivelyApplyFor{Law~[\nameref{rec-rolling-rule-law}]}{%
       \arraycolsep=0cm
       $(\lambda X \circspot A \circseq X)$,
       $(\lambda X \circspot
       \begin{array}[t]{l}
         \circif b \circthen B \circseq X \\
         {} \circelse \lnot b \circthen \Skip \\
         \circfi
       \end{array})$%
     }
     \label{algorithm-move-loop-start}
     % convert local vars to parameters
     \State \ApplyTo{Rule~[\nameref{var-parameter-conversion-rule}]}{\Call{ActionBody}{$methodName$}}
     \label{algorithm-var-parameter-conversion}
     \State \Call{RedefineMethodActionToExcludeParameters}{$methodName$}
     \label{algorithm-redefine-method-action-to-exclude-parameters}
     % remove unnecessary parameters
     \State \ApplyFor{Rule~[\nameref{argument-variable-elimination-rule}]}{$methodName$}
     \label{algorithm-argument-variable-elimination}     
    \EndFor
  \end{algorithmic}
  \caption{\Call{IntroduceVariables}{}}
  \label{introduce-variables-algorithm}
\end{algorithm}

Algorithm~\ref{introduce-variables-algorithm} operates upon each of
the method actions in turn on
line~\ref{algortihm-introduce-variables-loop}, determining the names
of the actions from $cs$ via the function \Call{MethodNames}{}.
Within this loop we refer to the name of the method action under
consideration as $methodName$.
Unlike the introduction of the $stackFrame$ variables, the order in
which the methods are iterated over does not matter, since each has
its own $stackFrame$ variable that undergoes local refinement.

We first refine field access operations to remove their reliance on
the $frameClass$ component of the $stackFrame$, which is removed in
the data refinement later in this section.
This is done by first introducing and distributing assumptions stating
the value of $frameClass$, using the procedure
\Call{IntroduceFrameClassAssumptions}{} on
line~\ref{algorithm-introduce-frameClass-assumptions}, which is
applied to the body of $methodName$.
This procedure is similar to \Call{IntroduceFrameStackAssumptions}{}
in that it introduces an assumption and distributes it with restricted
forms of standard algebraic laws.
However, \Call{IntroduceFrameClassAssumptions}{} acts on the body of a
single method and introduces an assumption about the value of the
$frameClass$ component of $stackFrame$ from the schema action
initialising $stackFrame$.
The \Call{IntroduceFrameClassAssumptions}{} procedure is defined by
Algorithm~\ref{introduce-frameClass-assumptions-algorithm}, which we
omit here, but is included in
Appendix~\ref{introduce-variables-appendix-subsection}.

We can then apply Rules~[\nameref{refine-PutfieldSF-rule}],
[\nameref{refine-GetfieldSF-rule}],
[\nameref{refine-PutstaticSF-rule}],
[\nameref{refine-GetstaticSF-rule}] and~[\nameref{refine-NewSF-rule}]
wherever possible to refine the field accesses and object creation
actions, on lines~\ref{algorithm-apply-refine-PutfieldSF}
to~\ref{algorithm-apply-refine-NewSF}.
As an example of one of these rules, we show
Rule~[\nameref{refine-PutfieldSF-rule}] in
Figure~\ref{refine-PutfieldSF-rule-figure}.
It refines a $PutfieldSF(cpi)$ instruction preceded by an assumption
stating the value of the $frameClass$ component of $stackFrame$.
With the application of the rule, the definition of $PutfieldSF$ is
expanded and the class identifier, $cid$, and field identifier, $fid$,
at the constant pool index $cpi$ are substituted in place of the
accesses to the constant pool.
This removes the reference to the $constantPool$ of the $frameClass$,
and hence the reference to the $frameClass$.
Rule~[\nameref{refine-GetfieldSF-rule}],
Rule~[\nameref{refine-PutstaticSF-rule}],
Rule~[\nameref{refine-GetstaticSF-rule}] and
Rule~[\nameref{refine-NewSF-rule}] are similar so we omit them here.
They can be found, along with the other compilation rules, in
Appendix~\ref{introduce-variables-appendix-subsection}.
After these laws have been applied, we eliminate any remaining
$frameClass$ assumptions by applying Law~[\nameref{assump-elim-law}]
and Law~[\nameref{seq-unitl-law}] on
lines~\ref{algorithm-frameClass-assump-elim}
and~\ref{algorithm-frameClass-seq-unitl}.

\begin{figure}
  \centering
  % \setlength{\zedtab}{0.4cm}
  % \setlength{\zedindent}{0pt}
  % \setlength{\zedleftsep}{0pt}
  % \setlength{\abovedisplayskip}{0pt}
  % \setlength{\belowdisplayskip}{0pt}
  % \setlength{\abovedisplayshortskip}{0pt}
  % \setlength{\belowdisplayshortskip}{0pt}
  \begin{restatable}[refine-$PutfieldSF$]{crule}{RefinePutfieldSFRule}
    \label{refine-PutfieldSF-rule}
    \begin{circus}
      \begin{array}{l}
        \{stackFrame.frameClass = c\} \circseq \\
        PutfieldSF(cpi)
      \end{array}
      \circrefines_A
      \begin{array}{l}
        (\circvar oid : ObjectID; value : Word \circspot \\
        \t1 \lschexpract InterpreterPop[ \\
        \t2 stackFrame/last~frameStack, \\
        \t2 stackFrame'/last~frameStack'] \rschexpract \circseq \\
        \t1 \lschexpract InterpreterPop[ \\
        \t2 oid!/value!, \\
        \t2 stackFrame/last~frameStack, \\
        \t2 stackFrame'/last~frameStack'] \rschexpract \circseq \\
        \t1 putField!oid!cid!fid!value \then \Skip)
      \end{array}
    \end{circus}
    where
    \begin{circus}
      cpi \in fieldRefIndices~c \land \\
      c.constantPool~cpi = FieldRef~(cid, fid)
    \end{circus}
  \end{restatable}
  \caption{Rule~[\nameref{refine-PutfieldSF-rule}]}
  \label{refine-PutfieldSF-rule-figure}
\end{figure}

After references to the $frameClass$ have been removed, we can perform
a local data refinement on the body of the method to convert the
$stackFrame$ variable to separate variables for the local variables
and operand stack slots.
Since we are converting the $operandStack$ component of $stackFrame$
from a sequence to a fixed set of variables representing an array of
stack slots, we must know the length of $operandStack$ before each
operation in order to determine which variable corresponds to the top
of the stack, and hence should be affected by the operation.
We ensure that this information is available, by introducing and
distributing assumptions on the size of $operandStack$.
This is performed by the \Call{IntroduceOperandStackAssumptions}{}
procedure, called on
line~\ref{algorithm-introduce-operandStack-assumptions}.
It is similar to \Call{IntroduceFrameClassAssumptions}{} and is
defined by
Algorithm~\ref{introduce-operandStack-assumptions-algorithm}, which is
included in Appendix~\ref{introduce-variables-appendix-subsection}.

After an $operandStack$ assumption has been introduced before each
data operation, the local data refinement is performed on
line~\ref{algorithm-local-data-refinement}.
The new state for the data refinement contains the local variables,
which are all of type $Word$, and are named $var$ followed by an
integer beginning at $1$ and going up to the total number of local
variables for the method, $\ell$.
It also contains operand stack slots, named $stack$ followed by an
integer from $1$ up to the maximum stack size for the method, $s$.
These values $\ell$ and $s$ are obtained from the $methodLocals$ and
$methodStackSize$ information for the method in $cs$, on
lines~\ref{algorithm-get-method-locals}
and~\ref{algorithm-get-method-stack-size} of
Algorithm~\ref{introduce-variables-algorithm}.
For example, the values associated with $handleAsyncEvent$ in the
$TPK$ class in Figure~\ref{example-model-figure} are $6$ for $\ell$
and $3$ for $s$. 
The coupling invariant for the data refinement of a method $m$ is then
given by the template $V{<}\ell{>}S{<}s{>}CI$, shown below.

% I don't think this is functional...does that matter?
\begin{schema}{V{<}\ell{>}S{<}s{>}CI}
  stackFrame : StackFrameEPC \\
  var1, \ldots, var{<}\ell{>} : Word \\
  stack1, \ldots, stack{<}s{>} : Word
\where
  \# stackFrame.localVariables = {<}\ell{>} \\
  stackFrame.localVariables~1 = var1 \\
  \t1 \vdots \\
  stackFrame.localVariables~{<}\ell{>} = var{<}\ell{>} \\
  stackFrame.stackSize = {<}s{>} \\
  \# stackFrame.operandStack \geq 1 \implies \\
  \t1 stackFrame.operandStack~1 = stack1 \\
  \t1 \vdots \\
  \# stackFrame.operandStack \geq {<}s{>} \implies \\
  \t1 stackFrame.operandStack~{<}s{>} = stack{<}s{>}
\end{schema}

$V{<}\ell{>}S{<}s{>}CI$ requires the number of local variables to be
equal to $\ell$, relates each of the values in the $localVariables$
sequence in $stackFrame$ to the corresponding local variables, $var1$
to $var{<}\ell{>}$.
It also requires the maximum operand stack size, $stackSize$, to be
$s$ and relates each value in $operandStack$ to the corresponding
stack slots, $stack1$ to $stack{<}s{>}$, but only if $operandStack$ is
long enough to contain such a value.
The values of the stack slots outside the length of the $operandStack$
at each point in the program are not specified, and so are chosen
nondeterministically, since they are not used until they have been
initialised with the correct value.
This nondeterminism allows us to avoid introducing unnecessary
assignments to initialise the stack slots and return them to a default
value when they are no longer used, which would be required if we
specified a value for unused stack slots in the coupling invariant.

However, the nondeterminism in $V{<}\ell{>}S{<}s{>}CI$ means that it
does not define a function, since there are multiple possible states
for the operand stack slots that correspond to a non-full
$operandStack$.
This means that we cannot directly compute the actions resulting from
the refinement (since there are multiple possibilities). 
So we must specify how each of the data operations is refined.
We, therefore, state compilation rules in terms of \Circus{}
simulations between actions.

Most of the bytecode instructions at this stage in the strategy have
their semantics stated in terms of a data operation over $stackFrame$,
in the form of a $Handle*SF$ action.
We state the simulations for such instructions as simulations of a
$Handle*SF$ action preceded by an $operandStack$ size assumption,
which can be viewed as adding an extra precondition to the action.
An example of such a simulation is
Rule~[\nameref{HandleAloadSF-simulation-rule}], shown in
Figure~\ref{HandleAloadSF-simulation-rule-figure}.
It states that $HandleAloadSF(lvi)$, with an assumption that the size
of $operandStack$ is $k$, is simulated by an assignment
$stack{<}k+1{>} := var{<}lvi{>}$.
This rule applies, for example, to the $HandleAloadSF(1)$ action
deriving from the $aload~1$ instruction at $pc = 12$ in
$TPK\_handleAsyncEvent$, which is refined to $stack1 := var1$, since
the stack is empty at that point.

\begin{figure}[thp]
  \begin{restatable}[$HandleAloadSF$-simulation]{crule}{HandleAloadSFSimulationRule}
    \label{HandleAloadSF-simulation-rule}
    \begin{circus}
      \begin{array}{l}
        \{\# stackFrame.operandStack = k\} \circseq \\
        HandleAloadSF(lvi)
      \end{array}
      \circsimulates
      \begin{array}{l}
        stack{<}k+1{>} := var{<}lvi{>}
      \end{array}
    \end{circus}
  \end{restatable}
  \caption{Rule~[\nameref{HandleAloadSF-simulation-rule}]}
  \label{HandleAloadSF-simulation-rule-figure}
\end{figure}

The instructions that manipulate objects by communicating with the
object manager have already had their definitions expanded earlier in
this stage.
Their communication with the object manager need not be changed by the
data refinement.
However, the data operations used by these operations to pop or push
the values communicated from or to the operand stack must be refined.
An example of the simulation for such an operation is
Rule~[\nameref{InterpreterPopEPC-simulation-rule}], shown in
Figure~\ref{InterpreterPopEPC-simulation-rule-figure}.
This establishes a simulation between $InterpreterPopEPC$, modified to
act over $stackFrame$, and an assignment of a stack slot value to the
variable $value$, which is in scope in the contexts where
$InterpreterPopEPC$ is used.

\begin{figure}[thp]
  \begin{restatable}[$InterpreterPopEPC$-simulation]{crule}{InterpreterPopEPCSimulationRule}
    \label{InterpreterPopEPC-simulation-rule}
    \begin{circus}
      \begin{array}{l}
        \{\# stackFrame.operandStack = k\} \circseq \\
        \lschexpract InterpreterPopEPC[ \\
        \t1 stackFrame/last~frameStack, \\
        \t1 stackFrame'/last~frameStack']\rschexpract
      \end{array}
      \circsimulates
      \begin{array}{l}
        value := stack{<}k{>}
      \end{array}
    \end{circus}
  \end{restatable}
  \caption{Rule~[\nameref{InterpreterPopEPC-simulation-rule}]}
  \label{InterpreterPopEPC-simulation-rule-figure}
\end{figure}

Method invocations also use data operations to pop the method's
arguments from the stack and pass them to the method, which must be
refined in the data refinement.
The $InvokeSF$ operation, which pops the method's arguments from the
stack, is simulated by an assignment of a sequence of operand stack
values to the $poppedArgs$ variables, as stated in
Rule~[\nameref{InvokeSF-simulation-rule}], shown in
Figure~\ref{InvokeSF-simulation-rule-figure}.

\begin{figure}[thp]
  \begin{restatable}[$InvokeSF$-simulation]{crule}{InvokeSFSimulationRule}
    \label{InvokeSF-simulation-rule}
    \begin{circus}
      \begin{array}{l}
        \{\# stackFrame.operandStack = k\} \circseq \\
        \lschexpract \exists argsToPop? == m @ InvokeSF \rschexpract
      \end{array}
      \circsimulates
      \begin{array}{l}
        poppedArgs := \\
        \t1 \langle stack{<}k-m+1{>}, \ldots , stack{<}k{>} \rangle
      \end{array}
    \end{circus}
  \end{restatable}
  \caption{Rule~[\nameref{InvokeSF-simulation-rule}]}
  \label{InvokeSF-simulation-rule-figure}
\end{figure}

The passing of the arguments to the invoked method has already been
refined in the introduction of the $stackFrame$ variable, but the
schema initialising $stackFrame$ must be further refined to initialise
the local variables.
The simulation for the $stackFrame$ initialisation schema is stated by
Rule~[\nameref{stackFrame-init-simulation-rule}], shown in
Figure~\ref{stackFrame-init-simulation-rule-figure}.
It is simulated by a sequence of assignments setting the local
variables to the values of the arguments.
The initialisation sets $operandStack$ to be empty, so there is no
need to assign values to the stack slot variables; they can be left
arbitrary.

\begin{figure}[thp]
  \begin{restatable}[$stackFrame$-init-simulation]{crule}{StackFrameInitSimulationRule}
    \label{stackFrame-init-simulation-rule}
    \begin{circus}
      \begin{array}{l}
        \lschexpract [arg1?, \ldots, arg{<}n{>}? : Word; \\
        \t1 stackFrame' : StackFrameEPC | \\
        \t1 \langle arg_1, \ldots, arg_n \rangle \subseteq stackFrame'.localVariables \land \\
        \t1 \# stackFrame'.localVariables = \ell \land \\
        \t1 stackFrame'.operandStack = \langle\rangle \land \\
        \t1 stackFrame'.frameClass = c \land \\
        \t1 stackFrame'.stackSize = s] \rschexpract
      \end{array}
      \circsimulates
      \begin{array}{l}
        var{<}1{>} := arg_1 \circseq \\
        \t1 \vdots \\
        var{<}n{>} := arg_n
      \end{array}
    \end{circus}
  \end{restatable}
  \caption{Rule~[\nameref{stackFrame-init-simulation-rule}]}
  \label{stackFrame-init-simulation-rule-figure}
\end{figure}

The simulation rules we have omitted here can be found with the rest
of the compilation rules in
Appendix~\ref{introduce-variables-appendix-subsection}.
These, together with the standard laws for distributing simulations
through \Circus{} constructs, are sufficient to unambiguously define
the local data refinement to be applied to the method.
After the data refinement, we eliminate any remaining assumptions by
applying Law~[\nameref{assump-elim-law}] and
Law~[\nameref{seq-unitl-law}] on
lines~\ref{algorithm-operandStack-assump-elim}
and~\ref{algorithm-operandStack-seq-unitl}.

We then eliminate the additional variables used in the data operations
that pop values from the stack.
Those that push values to the stack are pushing values received from a
channel, which require a separate assignment operation and so cannot
be eliminated.
The additional variables are eliminated using the rules applied on
lines~\ref{algorithm-apply-eliminate-value1-value2-conditional-rule}
to~\ref{algorithm-apply-eliminate-value-putStatic-rule} of
Algorithm~\ref{introduce-variables-algorithm}.
In particular,
Rule~[\nameref{eliminate-value1-value2-conditional-rule}], shown in
Figure~\ref{eliminate-value1-value2-conditional-rule-figure}, applies
to the $TPK\_handleAsyncEvent$ action in our example.
It removes the need for additional $value1$ and $value2$ variables,
replacing the references to them in the conditional with the stack
slot variables whose values they store.

\begin{figure}
  \begin{restatable}[cond-$value1$-$value2$-elim]{crule}{EliminateValueOneValueTwoConditional}
    \label{eliminate-value1-value2-conditional-rule}
    \begin{circus}
      \begin{array}{l}
        (\circvar value1, value2 : Word \circspot \\
        \t1 value1 := stack{<}k{>} \circseq \\
        \t1 value2 := stack{<}k+1{>} \circseq \\
        \t1 \circif value1 \leq value2 \circthen {} \\
        \t2 {} \cdots {} \\
        \t1 {} \circelse value1 > value2 \circthen {} \\
        \t2 {} \cdots {} \\
        \t1 \circfi)
      \end{array}
      \circrefines_A
      \begin{array}{l}
        \circif stack{<}k{>} \leq stack{<}k+1{>} \circthen {} \\
        \t1 {} \cdots {} \\
        {} \circelse stack{<}k{>} > stack{<}k+1{>} \circthen {} \\
        \t1 {} \cdots {} \\
        \circfi)
      \end{array}
    \end{circus}
  \end{restatable}
  \caption{Rule~[\nameref{eliminate-value1-value2-conditional-rule}]}
  \label{eliminate-value1-value2-conditional-rule-figure}
\end{figure}

We also eliminate the intermediate $poppedArgs$ variable used when
passing variables to method calls in the body of a method.
This is performed by the rules applied on
lines~\ref{algorithm-poppedArgs-elimination}
to~\ref{algorithm-multi-poppedArgs-elimination}, which are applied to
every method call in the body of $methodName$.
These rules eliminate $poppedArgs$ and copy the values stored in it
into the values passed to the value parameters of the called method.
This can be seen in
Rule~[\nameref{method-parameter-introduction-rule}], shown in
Figure~\ref{method-parameter-introduction-rule-figure}, which
eliminates $poppedArgs$ when associated with method calls arising from
a \texttt{invokespecial} or \texttt{invokestatic} instruction.
Rule~[\nameref{poppedArgs-sync-elim-rule}] and
Rule~[\nameref{getClassIDOf-multi-method-parameter-introduction-rule}]
are similar, but account for the extra communication before
synchronized methods, and the extra $getClassIDOf$ communication and
multiple targets arising from \texttt{invokevirtual} instructions,
respectively.

\begin{figure}[thp]
  \begin{restatable}[$poppedArgs$-elim]{crule}{MethodParameterIntroductionRule}
    \label{method-parameter-introduction-rule}
    \begin{circus}
      \begin{array}{l}
        (\circvar poppedArgs : \seq Word \circspot \\
        poppedArgs := \langle arg_1, \ldots, arg_n \rangle \circseq \\
        M(poppedArgs~1, \ldots, poppedArgs~n))
      \end{array}
      \circrefines_A
      \begin{array}{l}
        M(arg_1, \ldots, arg_n)
      \end{array}
    \end{circus}
  \end{restatable}
  \caption{Rule~[\nameref{method-parameter-introduction-rule}]}
  \label{method-parameter-introduction-rule-figure}
\end{figure}

We also move any actions that are at the start of a loop before the
loop condition is checked.
Such actions cannot be represented in C without the use of the comma
operator, which is not allowed in MISRA-C.
The actions are moved, using Law~[\nameref{rec-rolling-rule-law}], so
that they are before the start of the loop and at the end of the loop
body.
This is performed on line~\ref{algorithm-move-loop-start}.

After these rules have been applied to the body of
$TPK\_handleAsyncEvent$, it has the form shown in
Figure~\ref{efs-introduce-variables-mid-example-figure}.
The effect of these rules can be seen in the fact that values stored
in stack slots such as $stack1$ are passed directly to the arguments
of called functions.
The conditionals also compare stack slots directly and the assignments
to those stack slots ($stack1 := var4$ and $stack2 := 10$) have been
moved to before the start of the loop and just before the end of the
loop, rather than just after the beginning of the loop.
The argument to the function is passed in via the $arg1$ variable and
assigned to the local variable $var1$. 
This indirection is unnecessary, and we wish instead to have the
argument passed directly into $var1$.
We thus perform some final transformations to turn the local variables
corresponding to the methods arguments into parameters and eliminate
the $arg1, \ldots, arg{<}n{>}$ parameters for the method.

\begin{figure}[tp!]
  \centering
  \setlength{\zedtab}{0.5cm}
  \setlength{\zedindent}{0pt}
  \setlength{\zedleftsep}{0pt}
  \setlength{\abovedisplayskip}{0pt}
  \setlength{\belowdisplayskip}{0pt}
  \setlength{\abovedisplayshortskip}{0pt}
  \setlength{\belowdisplayshortskip}{0pt}
  \begin{circusaction}
    TPK\_handleAsyncEvent \circdef \circval arg1 : Word \circspot \\
    \t1 \circvar var1 var2, var3, var4, var5, var6 : Word \circspot \circvar stack1, stack2, stack3 : Word \circspot Poll \circseq \\
    \t1 var1 := arg1
    \t1 newObject!thread!ConsoleConnectionClassID \\
    \t2 {} \then  newObjectRet!oid \then stack1 := oid \circseq Poll \circseq \\
    \t1 stack2 := stack1 \circseq Poll \circseq \\
    \t1 stack3 := null \circseq Poll \circseq \\
    \t1 {} \cdots {} \\
    % \t1 ConsoleConnection\_CCinit(stack2, stack3) \circseq Poll \circseq \\
    % \t1 var1 := stack1 \circseq Poll \circseq \\
    % \t1 stack1 := var1 \circseq Poll \circseq \\
    % \t1 getClassIDOf!stack1?cid \\
    % \t2 {} \then ConsoleConnection\_openInputStream(stack1, stack1) \circseq Poll \circseq \\
    % \t1 var2 := stack1 \circseq Poll \circseq \\
    % \t1 stack1 := var1 \circseq Poll \circseq \\
    % \t1 getClassIDOf!stack1?cid \\
    % \t2 {} \then ConsoleConnection\_openOutputStream(stack1, stack1) \circseq Poll \circseq \\
    % \t1 var3 := stack1 \circseq Poll \circseq \\
    % \t1 stack1 := 0 \circseq Poll \circseq \\
    \t1 var4 := stack1 \circseq Poll \circseq Poll \circseq \\
    \t1 stack1 := var4 \circseq Poll \circseq \\
    \t1 stack2 := 10 \circseq Poll \circseq \\
    \t1 \circmu Y \circspot \\
    \t2 \circif stack1 \leq stack2 \circthen Poll \circseq \\
    \t3 stack1 := var2 \circseq Poll \circseq \\
    \t3 getClassIDOf!stack1?cid \then ConsoleInput\_read(stack1, stack1) \circseq Poll \circseq \\
    \t3 TPK\_f(stack1, stack1) \circseq Poll \circseq \\
    \t3 var5 := stack1 \circseq Poll \circseq \\
    \t3 {} \cdots {} \\
    % \t3 stack1 := var5 \circseq Poll \circseq \\
    % \t3 stack2 := 400 \circseq Poll \circseq \\
    % \t3 \circif stack1 \leq stack2 \circthen Poll \circseq \\
    % \t4 stack1 := var3 \circseq Poll \circseq \\
    % \t4 stack2 := var5 \circseq Poll \circseq \\
    % \t4 getClassIDOf!stack1?cid \then ConsoleOutput\_write(stack1, stack2) \\
    % \t3 {} \circelse stack1 > stack2 \circthen Poll \circseq \\
    % \t4 stack1 := var3 \circseq Poll \circseq \\
    % \t4 stack2 := 0 \circseq Poll \circseq \\
    % \t4 getClassIDOf!stack1?cid \then ConsoleOutput\_write(stack1, stack2) \\
    % \t3 \circfi \circseq Poll \circseq \\
    % \t3 stack1 := var4 \circseq Poll \circseq \\
    % \t3 stack2 := 1 \circseq Poll \circseq \\
    % \t3 stack1 := stack1 + stack2 \circseq Poll \circseq \\
    % \t3 var4 := stack1 \circseq Poll \\
    \t3 stack1 := var4 \circseq Poll \circseq \\
    \t3 stack2 := 10 \circseq Poll \circseq Y \\
    \t2 {} \circelse stack1 > stack2 \circthen \Skip \\
    \t2 \circfi \circseq Poll
  \end{circusaction}
  \caption{$TPK\_handleAsyncEvent$ after its variables have been introduced}
  \label{efs-introduce-variables-mid-example-figure}
\end{figure}

First, we make the first $n$ local variables into parameters using
Rule~[\nameref{var-parameter-conversion-rule}], shown in
Figure~\ref{var-parameter-conversion-rule-figure}.
This matches the \Circus{} variable blocks representing local
variables and stack slots, along with the assignments initialising the
first $n$ local variables.
The rule moves these assignments and, using the definition of value
parameter, converts them into instantiations of value parameters.
This is applied to the method's action on
line~\ref{algorithm-var-parameter-conversion}.

\begin{figure}[thp]
  \begin{restatable}[$var$-parameter-conversion]{crule}{VarParameterConversionRule}
    \label{var-parameter-conversion-rule}
    \begin{circus}
      \begin{array}{l}
        (\circvar var1, \ldots, var{<}\ell{>} : Word \circspot \\
        \circvar stack1, \ldots, stack{<}s{>} : Word \circspot \\
        \t1 var1 := arg1 \circseq \\
        \t1 \cdots \\
        \t1 var{<}n{>} := arg{<}n{>} \circseq \\
        \t1 A)
      \end{array}
      \circrefines_A
      \begin{array}{l}
        (\circval var1, \ldots var{<}n{>} : Word \circspot \\
        \circvar var{<}n+1{>}, \ldots, var{<}\ell{>} : Word \circspot \\
        \circvar stack1, \ldots, stack{<}s{>} : Word \circspot \\
        \t1 A)(arg1, \ldots, arg{<}n{>})
      \end{array}
    \end{circus}
  \end{restatable}  
  \caption{Rule~[\nameref{var-parameter-conversion-rule}]}
  \label{var-parameter-conversion-rule-figure}
\end{figure}

After local variables have been converted into arguments, we redefine
the method action to exclude the parametrised block for the
$arg1, \ldots, arg{<}n{>}$ parameters, so that the, now redundant,
parameters can be eliminated.
This is performed, as with previous redefinitions of method actions,
in a separate procedure,
\Call{RedefineMethodActionToExcludeParameters}{}, defined by
Algorithm~\ref{redefine-method-action-to-exclude-parameters-algorithm}
in Appendix~\ref{introduce-variables-appendix-subsection}.
This procedure is called on
line~\ref{algorithm-redefine-method-action-to-exclude-parameters} of
Algorithm~\ref{introduce-variables-algorithm}.

After this, the argument parameters $arg1, \ldots, arg{<}n{>}$ are
outside the method action and can be eliminated by application of
Rule~[\nameref{argument-variable-elimination-rule}], shown in
Figure~\ref{argument-variable-elimination-rule-figure}.
This rule takes the name of the method action as a parameter and
eliminates the argument parameters around the call to the method
action, passing their values directly to the method action.
It is applied on line~\ref{algorithm-argument-variable-elimination}

\begin{figure}[thp]
  \begin{restatable}[argument-variable-elimination]{crule}{ArgumentVariableEliminationRule}
    \label{argument-variable-elimination-rule}
    Given an action name $M$,
    \begin{circus}
      \begin{array}{l}
        (\circval arg1, \ldots, arg{<}n{>} : Word \circspot \\
        \t1 M(arg1, \ldots, arg{<}n{>}))(arg_1, \ldots, arg_n)
      \end{array}
      \circrefines_A
      \begin{array}{l}
        M(arg_1, \ldots, arg_n)
      \end{array}
    \end{circus}
  \end{restatable}
  \caption{Rule~[\nameref{argument-variable-elimination-rule}]}
  \label{argument-variable-elimination-rule-figure}
\end{figure}

As in the previous section, we only handle methods that do not return
a value.
Handling methods that do return a value requires additional
compilation rules, similar to
Rule~[\nameref{var-parameter-conversion-rule}] and
Rule~[\nameref{argument-variable-elimination-rule}], to account for
the additional result parameter present in such methods.
The result parameter is not replaced with a local variable, since it
is only used for returning the value and, as such, does not map onto a
specific local variable.
Instead, it is simply moved to be grouped with the local variable
parameters.
The rules on lines~\ref{algorithm-poppedArgs-elimination}
to~\ref{algorithm-multi-poppedArgs-elimination} also require separate
versions to handle the storing of the returned value in the calling
method.

\begin{figure}[tp!]
  \centering
  \setlength{\zedtab}{0.5cm}
  \setlength{\zedindent}{0pt}
  \setlength{\zedleftsep}{0pt}
  \setlength{\abovedisplayskip}{0pt}
  \setlength{\belowdisplayskip}{0pt}
  \setlength{\abovedisplayshortskip}{0pt}
  \setlength{\belowdisplayshortskip}{0pt}
  \begin{circusaction}
    TPK\_handleAsyncEvent \circdef \circval var1 : Word \circspot \\
    \t1 \circvar var2, var3, var4, var5, var6 : Word \circspot \circvar stack1, stack2, stack3 : Word \circspot Poll \circseq \\
    \t1 newObject!thread!ConsoleConnectionClassID \\
    \t2 {} \then  newObjectRet!oid \then stack1 := oid \circseq Poll \circseq \\
    \t1 stack2 := stack1 \circseq Poll \circseq \\
    \t1 stack3 := null \circseq Poll \circseq \\
    \t1 ConsoleConnection\_CCinit(stack2, stack3) \circseq Poll \circseq \\
    \t1 var1 := stack1 \circseq Poll \circseq \\
    \t1 {} \cdots {} \\
    % \t1 stack1 := var1 \circseq Poll \circseq \\
    % \t1 getClassIDOf!stack1?cid \\
    % \t2 {} \then ConsoleConnection\_openInputStream(stack1, stack1) \circseq Poll \circseq \\
    % \t1 var2 := stack1 \circseq Poll \circseq \\
    % \t1 stack1 := var1 \circseq Poll \circseq \\
    % \t1 getClassIDOf!stack1?cid \\
    % \t2 {} \then ConsoleConnection\_openOutputStream(stack1, stack1) \circseq Poll \circseq \\
    % \t1 var3 := stack1 \circseq Poll \circseq \\
    \t1 stack1 := 0 \circseq Poll \circseq \\
    \t1 var4 := stack1 \circseq Poll \circseq Poll \circseq \\
    \t1 stack1 := var4 \circseq Poll \circseq \\
    \t1 stack2 := 10 \circseq Poll \circseq \\
    \t1 \circmu Y \circspot \\
    \t2 \circif stack1 \leq stack2 \circthen Poll \circseq \\
    \t3 stack1 := var2 \circseq Poll \circseq \\
    \t3 getClassIDOf!stack1?cid \then {} \\
    \t4 \circif cid = ConsoleInputClassID \circthen ConsoleInput\_read(stack1, stack1) \circseq \\
    \t4 \circfi \circseq Poll \circseq \\
    \t3 TPK\_f(stack1, stack1) \circseq Poll \circseq \\
    \t3 var5 := stack1 \circseq Poll \circseq \\
    \t3 stack1 := var5 \circseq Poll \circseq \\
    \t3 stack2 := 400 \circseq Poll \circseq \\
    \t3 \circif stack1 \leq stack2 \circthen Poll \circseq \\
    \t4 stack1 := var3 \circseq Poll \circseq \\
    \t4 stack2 := var5 \circseq Poll \circseq \\
    \t4 getClassIDOf!stack1?cid \then {} \\
    \t5 \circif cid = ConsoleOutputClassID \circthen ConsoleOutput\_write(stack1, stack2) \\
    \t5 \circfi
    \t3 {} \circelse stack1 > stack2 \circthen Poll \circseq \\
    \t4 {} \cdots {} \\
    % \t4 stack1 := var3 \circseq Poll \circseq \\
    % \t4 stack2 := 0 \circseq Poll \circseq \\
    % \t4 getClassIDOf!stack1?cid \then {} \\
    % \t5 \circif cid = ConsoleOutputClassID \circthen ConsoleOutput\_write(stack1, stack2) \\
    % \t5 \circfi
    \t3 \circfi \circseq Poll \circseq \\
    \t3 stack1 := var4 \circseq Poll \circseq \\
    \t3 stack2 := 1 \circseq Poll \circseq \\
    \t3 stack1 := stack1 + stack2 \circseq Poll \circseq \\
    \t3 var4 := stack1 \circseq Poll \\
    \t3 stack1 := var4 \circseq Poll \circseq \\
    \t3 stack2 := 10 \circseq Poll \circseq Y \\
    \t2 {} \circelse stack1 > stack2 \circthen \Skip \\
    \t2 \circfi \circseq Poll
  \end{circusaction}
  \caption{$TPK\_handleAsyncEvent$ at the end of the Introduce
    Variables step}
  \label{efs-introduce-variables-example-figure}
\end{figure}

\begin{figure}[tp!]
  \centering
  \setlength{\zedtab}{0.5cm}
  \setlength{\zedindent}{0pt}
  \setlength{\zedleftsep}{0pt}
  \setlength{\abovedisplayskip}{0pt}
  \setlength{\belowdisplayskip}{0pt}
  \setlength{\abovedisplayshortskip}{0pt}
  \setlength{\belowdisplayshortskip}{0pt}
  \begin{lstlisting}[basicstyle=\ttfamily,keywordstyle=\bf,language=C,
    numbers=left,numberstyle=\scriptsize,stepnumber=1, numbersep=5pt,
    escapeinside={(*@}{@*)}]
void TPK_handleAsyncEvent(int32_t var1) {
    int32_t var2, var3, var4, var5, var6;
    int32_t stack1, stack2, stack3;

    stack1 = newObject(ConsoleConnectionClassID);
    stack2 = stack1;
    stack3 = 0;
    ConsoleConnection_CCinit(stack2, stack3);
    var1 = stack1;
    
    (*@
    % stack1 = var1;
    % stack1 = ConsoleConnection_openInputStream(stack1);
    % var2 = stack1;
    
    % stack1 = var1;
    % stack1 = ConsoleConnection_openOutputStream(stack1);
    % var3 = stack1;
    @*)...

    stack1 = 0; (*@ \label{C-code-unnecessary-assignment} @*)
    var4 = stack1;
    stack1 = var4; (*@ \label{C-code-unnecessary-assignment-end} @*)
    stack2 = 10;
    while (stack1 <= stack2) {
        stack1 = var2;
        if (stack1->classID = ConsoleInputClassID) {
            stack1 = ConsoleInput_read(stack1);
        }
        stack1 = TPK_f(stack1);
        var5 = stack1;
        
        stack1 = var5;
        stack2 = 400;
        if (stack1 <= stack2) {  
            stack1 = var3;
            stack2 = var5;
            if (stack1->classID = ConsoleOutputClassID) {
                ConsoleOutput_write(stack1, stack2);
            }
        } else {
            (*@
            % stack1 = var3;
            % stack2 = 0;
            % if (stack1->classID = ConsoleOutputClassID) {
            %     ConsoleOutput_write(stack1, stack2);
            % }
            @*)...
        }
          
        stack1 = var4;
        stack2 = 1;
        stack1 = stack1 + stack2;
        var4 = stack1;

        stack1 = var4;
        stack2 = 10;
    }
}
\end{lstlisting}
  \caption{The C code corresponding to $TPK\_handleAsyncEvent$}
  \label{efs-introduce-variables-c-code-figure}
\end{figure}

When the variables have been introduced, the model that we obtain has
a form that corresponds directly to the C code for each method.
This can be seen from
Figures~\ref{efs-introduce-variables-example-figure}
and~\ref{efs-introduce-variables-c-code-figure}, which show the
\Circus{} code for $TPK\_handleAsyncEvent$ and its corresponding C
code.
Note that the $getClassIDOf$ communications with the object manager
correspond to accesses to C structs representing objects.
These structs are introduced in the final stage the strategy, in
Section~\ref{data-refinement-of-objects-section}.


\subsection{Remove \texorpdfstring{$frameStack$}{frameStack} From
  State}
\label{remove-frameStack-from-state-subsection}

After all methods have been refined to use individual variables, the
$frameStack$ is no longer used.
We can thus eliminate the $frameStack$ from the state. 
This is performed as described in
Algorithm~\ref{remove-frameStack-from-state-algorithm}, which defines
the \Call{RemoveFrameStackFromState}{} procedure.

\begin{algorithm}
  \begin{algorithmic}
    \State \ApplyFor{Law~[\nameref{forwards-data-refinement-law}]}{$[]$, $FrameStackEliminationCI$}
    \label{algorithm-remove-frameStack-data-refinement}
    \State \ApplyFor{Law~[\nameref{process-param-elim-law}]}{$cs$}
    \label{algorithm-cs-elimination}
  \end{algorithmic}
  \caption{RemoveFrameStackFromState}
  \label{remove-frameStack-from-state-algorithm}
\end{algorithm}

First, on line~\ref{algorithm-remove-frameStack-data-refinement}, we
perform a data refinement to remove the $frameStack$ from the state.
The new state after the data refinement is the empty schema, and the
coupling invariant, $FrameStackEliminationCI$, maps all $frameStack$
values onto the empty state.
Since $frameStack$ is no longer used in the process, the only action
affected is the state initialisation, which becomes $Skip$.
After this, on line~\ref{algorithm-cs-elimination}, we eliminate the
$cs$ parameter from the process, since it is no longer used.
The result is the process $CThr_{bc,cs}(t)$, as shown in
Theorem~\ref{efs-thm}.
The only thing remaining to be done is to refine the representation of
objects, which is performed in the next stage of the strategy.


\section{Data Refinement of Objects}
\label{data-refinement-of-objects-section}

The final stage of the compilation strategy introduces the
representation of objects in C.
Unlike the previous stages of the strategy, this stage operates on the
object manager, $ObjMan$, refining it to the $StructMan_{cs}$ process
described in Section~\ref{cee-struct-manager-subsection}.
Thus, this stage may be summarised by the following theorem.
\begin{thm}[Data Refinement of Objects]\label{dro-thm}
  \begin{circus}
    ObjMan(cs) \circrefines StructMan_{cs}
  \end{circus}
\end{thm}

\begin{algorithm}[b]
  \begin{algorithmic}[1]
    \State \ApplyFor{Law~[\nameref{forwards-data-refinement-law}]}{$StructManState$, $ObjectCI$}
    \label{algorithm-dro-data-refinement}
    \State \ApplyTo{Rule~[\nameref{refine-NewObject-rule}]}{$NewObject$}
    \label{algorithm-dro-refine-NewObject}
    \State \ApplyTo{Rule~[\nameref{refine-GetField-rule}]}{$GetField$}
    \State \ApplyTo{Rule~[\nameref{refine-PutField-rule}]}{$PutField$}
    \State \ApplyTo{Rule~[\nameref{refine-GetStatic-rule}]}{$GetStatic$}
    \State \ApplyTo{Rule~[\nameref{refine-PutStatic-rule}]}{$PutStatic$}
    \label{algorithm-dro-refine-PutStatic}
    \State \ApplyFor{Law~[\nameref{process-param-elim-law}]}{$cs$}
    \label{algorithm-dro-cs-elimination}
  \end{algorithmic}
  \caption{Data Refinement of Objects}
  \label{dro-algorithm}
\end{algorithm}

The process of refining $ObjMan$ is described by
Algorithm~\ref{dro-algorithm}. 
It begins on line~\ref{algorithm-dro-data-refinement} with a data
refinement to change the representation of objects used in the
interpreter to the C structs used in the final code.
% This data refinement uses new types and functions to represent struct
% types for each class in $cs$, having the form described in
% Section~\ref{cee-struct-manager-subsection}.
The fields in all superclasses of a class are collected together to
form the fields used to define its struct.
Note that an interface cannot have non-static fields and so, since we
require that the inputs to the compilation strategy pass the standard
checks performed on Java class files, a class's objects only inherit
fields from its true superclasses, even though our superclass relation
includes implemented interfaces.

As an example, we present the struct for objects of the \texttt{TPK}
class, $TPKObj$, below.
Since \texttt{TPK} declares no fields of its own (as indicated by the
empty $fields$ set for $TPK$ in Figure~\ref{example-model-figure} from
Section~\ref{overview-subsection}), it only includes fields from the
superclasses of \texttt{TPK} (shown in
Figure~\ref{example-superclasses-model-figure}).
Its fields are thus those of \texttt{ManagedEventHandler}, since
\texttt{AperiodicEventHandler} does not add any information to the
base information for an event handler.
These fields are in addition to the $classID$ field, which is
contained in every object's struct and identifies the class of the
object.
The other fields are the $threadID$, $backingStoreSize$,
$allocAreaSize$ and $stackSize$ fields from the $ManagedEventHandler$
class information structure.
These provide space in which the information about the event handler
may be stored by the $Launcher$ during mission startup.
\begin{schema}{TPKObj}
  classID : ClassID \\
  threadID : Word \\
  backingStoreSize : Word \\
  allocAreaSize : Word \\
  stackSize : Word
\end{schema}

Similar types are created for $ManagedEventHandler$ and
$AperiodicEventHandler$, with the same fields, and named
$ManagedEventHandlerObj$ and $AperiodicEventHandlerObj$ repectively.
This means that $TPKObj$ values can be converted to those types, since
they contain fields of the same name and type.
Other aperiodic event handlers may have additional fields to store
data specific to them.
Their object types can be converted to $AperiodicEventHandlerObj$ by
simply discarding the additional fields.
The struct types for each class are collected together into an
$ObjectStruct$ type, shown below.
\begin{zed}
  ObjectStruct ::= \\
  \t1 TPKCon \ldata TPKObj \rdata | \\
  \t1 AperiodicEventHandlerCon \ldata AperiodicEventHandlerObj \rdata | \\
  \t1 ManagedEventHandlerCon \ldata ManagedEventHandlerObj \rdata | \cdots {} 
\end{zed}

The values of $ObjectStruct$ that can be converted to
$ManagedEventHandlerObj$ can be cast to it by the function
$castManagedEventHandler$.
We also define functions for performing a combined cast and field
update and collect the static fields from all classes into a
$StaticFields$ structure, as described in
Section~\ref{cee-struct-manager-subsection}.

% TODO: does this need to be said?
These types and functions are all used in the struct manager, so we
introduce them before applying the data refinement.
Note that, since the types are Z data types, they do not need to be
declared in a process and so we do not need to introduce them by
refinement of $ObjMan$.

The state resulting from the data refinement on
line~\ref{algorithm-dro-data-refinement} is $StructManState$, shown
below, which uses the object struct types described above.
\begin{schema}{StructManState}
  BackingStoreManager \\
  objects : ObjectID \pfun ObjectStruct \\
  staticClassFields : StaticFields
\where
  backingStoreMap \partition \dom objects
\end{schema}
As mentioned in Section~\ref{cee-struct-manager-subsection},
$StructManState$ is very similar to $ObjManState$.
It contains $BackingStoreManager$, which is the same as in
$ObjManState$ since the management of backing stores is unaffected by
the compilation strategy.
The $objects$ map is similar to that in $ObjManState$, but it maps to
the $ObjectStruct$ type, rather than the $Object$ type.
The component $staticClassFields$, in $StructManState$, is of the
$StaticFields$ type, rather than the map from fields to their values
in $ObjManState$, since we have a known set of static fields, having
been supplied with the $cs$ map.

The data refinement is described by the coupling invariant $ObjectCI$,
shown in Figure~\ref{ObjectCI-figure}, which relates $ObjManState$ to
$StructManState$.
It is presented as a template to be instantiated by the identifiers of
the classes in $cs$ and their corresponding fields, in much the same
format as for the schemas in
Section~\ref{cee-struct-manager-subsection}.
This equates the $BackingStoreManager$ fields in $ObjManState$ with
those in $StructManState$, since management of backing stores is
unaffected by the data refinement.

\begin{figure}[tp!]
\begin{schema}{ObjectCI}
  ObjManState \\
  StructManState_1
\where
  backingStoreMap = backingStoreMap_1 \\
  backingStoreStacks = backingStoreStacks_1 \\
  rootBS = rootBS_1 \\
  \dom~objects = \dom objects_1 \\
  \forall x : \dom objects @ \\
  % \t1 thisClassID~((objects~x).class) \in \{{<}classID_1{>}, \ldots, {<}classID_n{>}\} \land \\
  \t1 (thisClassID~((objects~x).class) = {<}classID_1{>} \implies \\
  % \t2 (\dom (objects~x).fields = \{{<}fieldID_{1,1}{>}, \ldots, {<}fieldID_{1,m_1}{>}\} \land \\
  \t2 objects_1~x = {<}classID_1{>}Con~\lblot \\
  \t3 classID == {<}classID_1{>}, \\
  \t3 {<}fieldID_{1,1}{>} == (objects~x).fields~{<}fieldID_{1,1}{>}, \\
  \t4 \cdots \\
  \t3 {<}fieldID_{1,m_1}{>} == (objects~x).fields~{<}fieldID_{1,m_1}{>} \\
  \t2 \rblot) \land \\
  \t2 \cdots \\
  \t1 (thisClassID~((objects~x).class) = {<}classID_n{>} \implies \\
  % \t2 (\dom (objects~x).fields = \{{<}fieldID_{n,1}{>}, \ldots, {<}fieldID_{1,m_n}{>}\} \land \\
  \t2 objects_1~x = {<}classID_n{>}Con~\lblot \\
  \t3 classID == {<}classID_n{>}, \\
  \t3 {<}fieldID_{n,1}{>} == (objects~x).fields~{<}fieldID_{n,1}{>}, \\
  \t4 \cdots \\
  \t3 {<}fieldID_{n,m_n}{>} == (objects~x).fields~{<}fieldID_{n,m_n}{>} \\
  \t2 \rblot) \\
  staticClassFields_1.{<}classID_1{>}\_{<}staticfieldID_{1,1}{>} = \\
  \t1 staticClassFields~({<}classID_1{>}, {<}staticfieldID_{1,1}{>}) \\
  \t1 \cdots \\
  staticClassFields_1.{<}classID_1{>}\_{<}staticfieldID_{1,\ell_1}{>} = \\
  \t1 staticClassFields~({<}classID_1{>}, {<}staticfieldID_{1,\ell_1}{>}) \\
  \t1 \cdots \\
  staticClassFields_1.{<}classID_n{>}\_{<}staticfieldID_{n,1}{>} = \\
  \t1 staticClassFields~({<}classID_n{>}, {<}staticfieldID_{n,1}{>}) \\
  \t1 \cdots \\
  staticClassFields_1.{<}classID_n{>}\_{<}staticfieldID_{n,\ell_n}{>} = \\
  \t1 staticClassFields~({<}classID_n{>}, {<}staticfieldID_{n,\ell_n}{>}) \\
\end{schema}
\caption{The $ObjectCI$ schema, which is the coupling invariant between $ObjManState$ and $StructManState$}
\label{ObjectCI-figure}
\end{figure}

The functions $objects$ for both the abstract and concrete states have
the same domain, since the set of objects does not change, merely
their representation.
The representation of each object in $objects$ after the data
refinement is determined by the class identifier in its $class$
information before the refinement.
The object is of the struct type for the class identifier.
For example, $TPKClassID$ is the class identifier corresponding to the
class information of \texttt{TPK}, since that is the identifier in the
$constantPool$ of the $TPK$ structure
(Figure~\ref{example-model-figure}) that corresponds to its $this$
value.
Thus, any object of that type will have the correponding class
information stored and so is refined to an $ObjectStruct$ value using
the $TPKCon$ constructor and containing a value of the $TPKObj$ type
shown above.

The fields of the structure are taken from the values corresponding to
each field identifier in the object's information before the data
refinement.
These fields are guaranteed to correspond to the fields listed in the
object's class information (including the fields from its
superclasses) by the invariant of $ObjManState$.
Similarly, the fields in $staticClassFields$ map directly onto fields
in the $StaticFields$ structure, with the set of fields guaranteed to
be the same by the invariant of $StaticFieldsInfo$.

$ObjectCI$ is based on equating the information in the old model with
the fields of the object structs in the new model, and so it describes
a functional data refinement from which we can calculate the form of
the actions in the resultant model.
However, the direct application of this refinement yields actions in a
form that does not directly correspond to the representation of the
semantics of C structs that we desire.
We thus apply some additional compilation rules on
lines~\ref{algorithm-dro-refine-NewObject}
to~\ref{algorithm-dro-refine-PutStatic}, to refine the actions of the
process to the correct form.

An example of such a rule is Rule~[\nameref{refine-NewObject-rule}],
shown in Figure~\ref{refine-NewObject-rule-figure}.
It operates on the body of the $NewObject$ action, which has the form
on the left-hand side of the rule after the application of the data
refinement.
In this rule, we represent by the schema $StructManObjectInit$ the
result of applying the data refinement to the $ObjManObjectInit$
schema.
The rule splits this data operation into separate operations defined
by simpler schemas, which can be found in Appendix~B of the extended
version of this thesis~\cite{baxter2018-extended}, initialising each
of the object struct types, offering a choice over each class
identifier in $cs$ to determine which one should be used.
The $AllocateObject$ action that communicates with the memory manager
to allocate space for the object is also supplied with constants
indicating the size of each object struct type, rather than
determining it from the class information in $cs$.
This means the $GetObjectClassInfo$ schema, which determines the class
information, can also be removed, eliminating reliance on $cs$ in the
action.

\begin{figure}[t!]
  \begin{restatable}[refine-$NewObject$]{crule}{RefineNewObjectRule}
    \label{refine-NewObject-rule}
    \setlength{\zedtab}{0.6cm}
    \setlength{\zedindent}{0.6cm}
    \begin{circus}
      \begin{array}{l}
        \circvar thread : ThreadID; classID : ClassID \circspot \\
        \circvar objectID : ObjectID; class : Class \circspot \\
        newObject?t?c \then thread, classID := t, c \circseq \\
        \lschexpract GetObjectClassInfo \rschexpract \circseq \\
        AllocateObject(\\
        \t1 thread, sizeOfObject~class, objectID) \circseq \\
        \lschexpract StructManObjectInit \rschexpract \circseq \\
        newObjectRet!objectID \then \Skip
      \end{array}
      \circrefines_A
      \begin{array}{l}
        \circvar objectID : ObjectID \circspot \\
        newObject?thread?classID \then {} \\
        \circif classID = {<}classID_1{>} \circthen {} \\
        \t1 AllocateObject(\\
        \t2 thread, \\
        \t2 sizeof{<}classID_1{>}Obj, \\
        \t2 objectID) \circseq \\
        \t1 \lschexpract StructMan{<}classID_1{>}ObjInit \rschexpract \circseq \\
        \t1 {} \vdots {} \\
        {} \circelse classID = {<}classID_n{>} \circthen {} \\
        \t1 AllocateObject(\\
        \t2 thread, \\
        \t2 sizeof{<}classID_n{>}Obj, \\
        \t2 objectID) \circseq \\
        \t1 \lschexpract StructMan{<}classID_n{>}ObjInit \rschexpract \circseq \\
        \circfi \circseq newObjectRet!objectID \then \Skip
      \end{array}
    \end{circus}
    where, for all $k \in 1 \upto n$,
    \begin{circus} 
        \exists \Delta Class | \Xi Class \hide (fields, fields') @ \\
        \t1 \theta Class = cs~{<}classID_k{>} \land \\
        \t1 fields' = \bigcup \{ cid : \dom cs | ({<}classID_k{>},cid) \in subclassRel~cs @ (cs~cid).fields \} \land  \\
        %\t1 \exists class? == \theta Class~' @ ObjManObjectInit \\
        \t1 sizeof{<}classID_k{>}Obj = \theta Class~'\\
    \end{circus}
  \end{restatable}
  \caption{Rule~[\nameref{refine-NewObject-rule}]}
  \label{refine-NewObject-rule-figure}
\end{figure}

Rule~[\nameref{refine-GetField-rule}],
Rule~[\nameref{refine-PutField-rule}],
Rule~[\nameref{refine-GetStatic-rule}] and
Rule~[\nameref{refine-PutStatic-rule}], which refine the $GetField$,
$PutField$, $GetStatic$ and $PutStatic$ actions respectively, are
similar to Rule~[\nameref{refine-NewObject-rule}], refining the data
operations that result from the data refinement to choices over class
and field identifiers received on the channels for the operations.
These rules can be found in Appendix~\ref{compilation-rules-appendix}.
Note that, while the $Init$ action of $ObjMan$ is refined in this
stage, it does not require a separate rule, since the necessary
refinement is performed by the data refinement alone.

After the refinement has been performed, we can eliminate the $cs$
parameter of the process via an application of
Law~[\nameref{process-param-elim-law}] on
line~\ref{algorithm-dro-cs-elimination}.
This completes the transformation of $ObjMan(cs)$ into
$StructMan_{cs}$.
After this, the model corresponds completely to the C code.


\section{Proof of Main Theorem}
\label{main-theorem-proof-section}

The three stages of our strategy, taken together, refine the abstract
interpreting CEE described in Section~\ref{cee-interpreter-section} to
the concrete C CEE described in Section~\ref{cee-c-code-section}.
This can be seen from the proof of Theorem~\ref{main-theorem}, shown
below.

\begin{crproof}[Theorem~\ref{main-theorem}]
  \begin{argue}
    \begin{array}{l}
      CEE(bc,cs,instCS,sid,initOrder)
    \end{array}\\
    = & Definition of $CEE$ \\
    \begin{array}{l}
      ObjMan(cs) \parallel Interpreter(cs,bc,instCS) \parallel Launcher(sid,initOrder)
    \end{array}\\
    = & Definition of $Interpreter$ \\
    \begin{array}{l}
      ObjMan(cs) \parallel {} \\
      \t1 (\Parallel t : ThreadID \setminus \{ idle \} \lpar ThrChans(t) \rpar \circspot Thr(bc, cs, instCS, t)) \parallel {} \\
      \t1 Launcher(sid,initOrder)
    \end{array}\\
    \circrefines & Theorem~\ref{epc-thm} \\
    \begin{array}{l}
      ObjMan(cs) \parallel {} \\
      \t1 (\Parallel t : ThreadID \setminus \{ idle \} \lpar ThrChans(t) \rpar \circspot ThrCF_{bc,cs}(cs,t)) \parallel {} \\
      \t1 Launcher(sid,initOrder)
    \end{array}\\
    \circrefines & Theorem~\ref{efs-thm} \\
    \begin{array}{l}
      ObjMan(cs) \parallel {} \\
      \t1 (\Parallel t : ThreadID \setminus \{ idle \} \lpar ThrChans(t) \rpar \circspot CThr_{bc,cs}(t)) \parallel {} \\
      \t1 Launcher(sid,initOrder)
    \end{array}\\
    = & Definition of $CProg_{bc,cs}$ \\
    \begin{array}{l}
      ObjMan(cs) \parallel CProg_{bc,cs} \parallel Launcher(sid,initOrder)
    \end{array}\\
    \circrefines & Theorem~\ref{dro-thm} \\
    \begin{array}{l}
      StructMan_{cs} \parallel CProg_{bc,cs} \parallel Launcher(sid,initOrder)
    \end{array}\\
    % = & Definition of $CCEE_{bc, cs}$ \\
    % \begin{array}{l}
    %   CCEE_{bc,cs}(sid, initOrder)
    % \end{array}\\
  \end{argue}
\end{crproof}

The correctness of this proof rests on the correctness of theorems for
each stage of the strategy.
The compilation strategy forms the proofs of these theorems and it is
composed of applying compilation rules.
The correctness of the compilation rules is, in turn, ensured by their
proofs in terms of algebraic laws that are known to be correct.


\section{Final Considerations}
\label{compilation-final-considerations-section}

In this chapter, we have presented our compilation strategy from an
interpreting SCJVM to our model of C code. While our compilation
strategy proves the correctness of the compilation, there are further
optimisations that may be performed on the output of the strategy.

One example of such an optimisation is the removal of the unnecessary
choice offered in virtual method calls with only one possible target.
Such choices are made using the class identifier of the object, which,
in our model, is obtained via communication with the struct manager on
the $getClassIDOf$ channel. 
The removal of the choice requires the removal of this communication,
which is a refinement all the processes that participate in this
communication. 
It requires collapsing the parallelism between these processes and
using the fact that the communication is hidden to remove it. 
This is not performed in our strategy, since each stage of the
strategy operates only on a single process, but is a relatively
straightforward optimisation that could be added as an extension of
the strategy in future work.

A further consideration is that Z schemas bindings represent an
unordered collection of fields, whereas C structs define the order in
which their fields are stored in memory.
This means that, while our struct manager model defines what fields
must be in each object's struct type, it does not specify the order of
those fields and so is still somewhat more abstract than the C code
itself.
This can be addressed by a further data refinement to a representation
using Z sequences.
Field names for each struct type would then be associated with offsets
into these sequences, as field names in C are associated with offsets
in the struct's memory.
We have not performed such a data refinement as part of the strategy,
since we believe the form of the struct manager is sufficiently clear
to implementers, although there has to be a choice of ordering for the
C structs when implementing.

We expect other optimisations to be performed by the C compiler that
compiles the output of the strategy.
The correctness of such optimisations is part of verification of the C
compiler and thus outside the scope of our work.
However, some optimisations could be integrated into the strategy as
part of future work.
An example is the elimination of unnecessary assignments, such as on
line~\ref{C-code-unnecessary-assignment} of the code in
Figure~\ref{efs-introduce-variables-c-code-figure}.
There, \texttt{stack1} is used as an intermediate variable to set
\texttt{var4} and is not otherwise used before it is overwritten on
line~\ref{C-code-unnecessary-assignment-end}.
These assignments are removed by optimising C compilers, and so are
not removed by our strategy, or the icecap HVM, but could be removed
in order to produce clearer C code.

Other possible directions for future work extending the strategy
include weakening the assumptions described in
Section~\ref{compilation-assumptions-section}.
Our definition of a structured program is slightly stronger than the
structural requirements imposed by MISRA-C, which permits a single
exit from the middle of a loop in addition to the condition at the
start or end of the loop.
This means loops may have two exits in MISRA-C, whereas our strategy
only accounts for loops with a single exit point.
The strategy could be modified to allow for loops with two exit points
by adding new rules, similar to
Rule~[\nameref{while-introduction-rule1}], to introduce such loops,
having two conditionals to allow for exit from the loop.

We do not model and handle integer overflow in the strategy due to the
fact that it is not handled in icecap, instead requiring the SCJ
programmer to ensure that their code does not include overflows.
A possible extension of the strategy would be to model the overflow
behaviour of the JVM in the bytecode interpreter and refine it to C
code that enforces that overflow behaviour by checking if overflow
would occur before performing an operation.
This would create extra checks in many places where they would not be
necessary, but such checks could be removed in places where overflow
can be proved not to occur.

While we do not allow recursion due to its potential for stack
overflow, there may be a few cases in which recursion can be shown to
be bounded and hence safe.
The strategy could be extended to handle such cases, requiring rules
for introducing loops caused by method calls, and separating the
resulting recursions in recursive actions to represent recursive
methods.
Recursion with more than one method involves a similar approach,
introducing nested loops and creating mutually recursive actions.

In summary, there are many optimisations and extensions that are
possible. 
What we have achieved, however, is a formal account of a compilation
strategy that addresses all the central concerns involved in
transforming SCJ bytecode to a higher-level language like C. 
Correctness of the compiled code is established by construction.
In the next chapter, we discuss in more detail how the correctness of
the strategy is assured and evaluate it through consideration of some
examples.

