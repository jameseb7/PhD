\chapter{Compilation Strategy}
\label{strategy-chapter}

Our compilation strategy refines the $CEE(bc,cs,sid)$ process defined
in Section~\ref{model-section} to obtain a process that includes a
representation of C code as described in
Section~\ref{embedding-section}. 
The overall theorem for the strategy is as follows.
\begin{thm}[Compilation Strategy]\label{main-theorem}
  Given $bc$, $cs$ and $sid$, there are processes $StructMan_{cs}$ and
  $CProg_{bc,cs}$ such that,
  \begin{circus}
    CEE(bc,cs,sid) \circrefines StructMan_{cs} \parallel
    CProg_{bc,cs} \parallel Launcher(sid).
  \end{circus}
\end{thm}
$StructMan_{cs}$ manages objects represented by C structs that
incorporate the class information from $cs$, refining the process
$ObjMan$, which handles abstract objects.
$StructMan_{cs}$ has Z schemas representing struct types for objects
of each class.
%
% \begin{zed}
%   InputHandlerObj == [classid : ClassID; input, buffer : ObjectID]
% \end{zed}
%
These schemas contain the identifier $classid$ of the object's class, so
that polymorphic method calls can be made by choice over the object's
class. 
There are also components for each of the fields of the
object.

The schema types for each type of object are combined into a single
free type $ObjectStruct$.
% \begin{zed}
%   ObjectStruct ::= \dots | InputHandlerCon \ldata InputHandlerObj \rdata \dots
% \end{zed}
% We also define functions for casting between objects of different
% classes and for obtaining the class identifier of any object. 
% This matches the casting of object pointers in the C code that the
% icecap HVM generates.
$StructMan_{cs}$ contains a map from memory addresses managed by the
SCJVM to the $ObjectStruct$ type, representing the C structs in
memory, and provides access to the individual values in that map.

$CProg_{bc,cs}$ refines the $Interpreter$, with the $Thr$ processes
refined into the $CThr_{bc,cs}$ processes described in the previous
section.
This means that the threads from SCJ are mapped onto threads in C,
since we do not dictate a particular thread switch mechanism in either
the source or target models.

The compilation strategy is split into three stages, described in the
following section.
Each stage has a theorem describing it, for which the strategy acts as
a proof.
The proof of Theorem~\ref{main-theorem} is obtained by an application
of the theorems for each stage.

\subsection{Overview}

Each stage of the compilation strategy handles a different part of the
$Interpreter$ state:~the $pc$, the $frameStack$, and objects.
They operate over each of the $Thr$ processes, managed by the SCJVM
services.

The first stage introduces the control constructs of the C code.
This removes the use of $pc$ to determine the control flow of the
program.
The choice over $pc$ values is then replaced with a choice over method
identifiers pointing to sequences of operations representing method
bodies.

In the second stage, the information contained on the $frameStack$,
which is the local variable array and operand stack for each method,
is introduced in the C code.
This is done by introducing variables and parameters to represent each
method's local variables and operand stack slots.
A data refinement is then used to transform each operation over the
$frameStack$ to operate on the new variables.
The $frameStack$ is then eliminated from the state.

In the final stage, the class information from $cs$ is used to create
a representation of C structs.
This means that $ObjMan$, which has a very abstract representation of
objects, is transformed into $StructMan$.
The process for each thread is then made to access the structs for the
objects in a more concrete way that represents the way struct fields
are accessed in C code.

% This yields final method actions of a form similar to that of the
% example shown below, which is taken from the \texttt{InputHandler}
% presented in Section~\ref{model-section}.
% \begin{circusaction}
%   InputHandler\_HandleAsyncEvent \circdef \\
%   \t1 \circval var0 \circspot \circvar var1, stack0, stack1 : Word \circspot \\
%   \t1 stack0 := var0 \circseq Poll \circseq getObject!stack0 \then getObjectRet?struct \\
%   \t1 {} \then stack0 := (castInputHandler~struct).input \circseq \dots
% \end{circusaction}
% The \texttt{handleAsyncEvent()} method of \texttt{InputHandler} is
% compiled to the action $InputHandler\_HandleAsyncEvent$, with the
% implicit \texttt{this} parameter represented as a value parameter
% $var0$.
% The local variable ($var1$) and stack slots ($stack0$ and $stack1$)
% are represented as \Circus{} variables.
% The operations of the C code are composed in sequence, with an action
% named $Poll$ that polls for thread switches present at the points
% where thread switches may occur. 
% Stack operations are represented as assignments. 
% For instance, $stack0 := var0$ arises from the compilation to load a
% local variable into a stack slot.
% Access to objects is performed by communicating with $StructMan_{cs}$
% to obtain the struct for the object, then casting it to the correct
% type, and accessing the required value.
% Above, we obtain the value of the $input$ field from an $InputHandler$
% object.
% The communication with $StructMan_{cs}$ is performed via the
% $getObject$ channel and the function $castInputHandler$ is used to map
% the $ObjectStruct$ returned from the communication to a type
% representing an object of \texttt{InputHandler}.

We describe each of these stages in a separate section.
The first stage, which we call \emph{Elimination of Program Counter},
is described in Section~\ref{elimination-of-program-counter-section}.
The second stage, called \emph{Elimination of Frame Stack}, is
described in Section~\ref{elimination-of-frame-stack-section}.
Finally, third of the strategy, which is called \emph{Data Refinement
  of Objects}, is described in
Section~\ref{data-refinement-of-objects-section}.

\section{Elimination of Program Counter}
\label{elimination-of-program-counter-section}

This stage eliminates $pc$ from the state of each
thread's process, $Thr(bc,cs,t)$, introducing the control flow constructs of C in the
process. 
It may be summarised by the following theorem.
%
\begin{thm}[Elimination of Program Counter]\label{thread-splitting-thm}
  \begin{circus}
    Thr(bc,cs,t) \circrefines ThrCF_{bc,cs}(cs,t)
  \end{circus}%
\end{thm}
%
In this stage we act mainly upon the $Running$ action of $Thr$; its
loop is unrolled to introduce the control flow that follows each
bytecode instruction.
The aim is to get each method's bytecode instructions into a form in
which the control flow, but not the data operations, are described
using C constructs and, moreover, each path of execution (including
every branch of the conditionals) ends in a return instruction or a
loop.
We refer to a method in this form as a \emph{complete} method.

It is important to observe that it is possible to transform the
bytecode instructions of every method so that they become complete.
If we consider the control flow of a method beginning from that
method's entry point, each bytecode instruction reached must either be
a return instruction, or followed by another bytecode.
If another bytecode follows the bytecode's execution, then it must be
either a bytecode already considered, resulting in a loop, or one not
already considered.
Since there are finitely many bytecode instructions in a method, a
loop or return must eventually be reached.
Failure to do so would lead to an instruction beyond the end of the
method, which is forbidden by the structural restrictions on Java
bytecode~\cite{lindholm2014}. 
We assume bytecode input to our strategy will have undergone bytecode
verification so this cannot happen.

When a method is complete, it can be defined by a separate \Circus{}
action.
When the code for all the methods has been split in this way, the
choice of bytecode instruction using the program counter value can be
removed and replaced with a choice over method identifiers.
Thus dependency on the program counter can be completely removed,
allowing it to be eliminated from the state of $Thr$.

The overall strategy for transforming $Thr$ in this stage and
achieving this elimination is described by
Algorithm~\ref{epc-algorithm}.
It begins at line~\ref{algorithm-expand-bytecode} by expanding the
\Circus{} definitions of the bytecode instructions from the $bc$ map
into the $Running$ action, pulling out the program counter updates so
that they can be more easily manipulated by the strategy.
In line~\ref{algorithm-introduce-forward-sequence}, instructions that
are forward \texttt{goto}s or are simply followed by execution of the
bytecode at the next $pc$ value are sequenced with the instructions
following them.
After that, for each method, its loops and conditionals are introduced
in line~\ref{algorithm-introduce-loops-and-conditionals}. 
Afterwards, any complete methods are separated out, in
line~\ref{algorithm-separate-complete-methods}, and any method calls
involving completed methods are resolved by sequencing the method call
with the \Circus{} action representing the method, in
line~\ref{algorithm-resolve-method-calls}.

This is repeated until all methods have been separated out, as
indicated by the while loop in line~\ref{algorithm-method-loop}.
The $MainThread$ and $NotStarted$ actions are then refined in
line~\ref{algorithm-refine-main-actions} to provide a choice over
method identifiers, rather than $pc$ values, thus removing all uses of
$pc$ from the interpreter.
The $pc$ component is then removed from the state in
line~\ref{algorithm-remove-pc-from-state} of the algorithm.

\begin{algorithm}[t]
  \begin{algorithmic}[1]
    \State \Call{ExpandBytecode}{} \label{algorithm-expand-bytecode}
    \State \Call{IntroduceSequentialComposition}{} \label{algorithm-introduce-forward-sequence}
    \While{$\lnot$\Call{AllMethodsSeparated}{}} \label{algorithm-method-loop}
    \State \Call{IntroduceLoopsAndConditionals}{} \label{algorithm-introduce-loops-and-conditionals}
    \State \Call{SeparateCompleteMethods}{} \label{algorithm-separate-complete-methods}
    \State \Call{ResolveMethodCalls}{} \label{algorithm-resolve-method-calls}
    \EndWhile
    \State \Call{RefineMainActions}{} \label{algorithm-refine-main-actions}
    \State \Call{RemovePCFromState}{} \label{algorithm-remove-pc-from-state}
  \end{algorithmic}
  \caption{Elimination of Program Counter}
  \label{epc-algorithm}
\end{algorithm}

Each of the procedures used in Algorithm~\ref{epc-algorithm} is
defined in a separate section in the sequel.
Beforehand, we give a more detailed overview of the strategy.

\subsection{Overview}
\label{overview-subsection}

We explain the strategy with an example, the Java code for which is
shown in Figure~\ref{example-code-figure}.
\begin{figure}[t!]
  \begin{lstlisting}[language=Java,basicstyle=\ttfamily\footnotesize]
    import java.io.InputStream;
    import java.io.OutputStream;
    import javax.realtime.AperiodicParameters;
    import javax.realtime.ConfigurationParameters;
    import javax.realtime.PriorityParameters;
    import javax.safetycritical.AperiodicEventHandler;
    import javax.safetycritical.StorageParameters;
    import javax.safetycritical.io.ConsoleConnection;
    
    public class TPK extends AperiodicEventHandler {
      
      public TPK(PriorityParameters priority,
                 AperiodicParameters release,
                 StorageParameters storage,
                 ConfigurationParameters config) {
        super(priority, release, storage, config);
      }
      
      public void handleAsyncEvent() {
        ConsoleConnection console = new ConsoleConnection(null);
        
        InputStream input = console.openInputStream();
        OutputStream output = console.openOutputStream();
        
        for(int i = 0; i <= 10; i = i + 1) {
          int y = f(input.read());
          
          if (y > 400) {
            output.write(0);
          } else {
            output.write(y);
          }
        }
      }
      
      public static int f(int x){
        return x + x + x + 5;
      }
      
    }
  \end{lstlisting}
  \caption{Our example program}
  \label{example-code-figure}
\end{figure}
\begin{figure}[p]
  \scriptsize
  \setlength{\zedindent}{0cm}
  \setlength{\zedtab}{0.3cm}
  \setlength{\zedleftsep}{0.1cm}
  \setlength{\linewidth}{10cm}
  \begin{vwcol}[widths={0.7,0.3},rule=none]
    \begin{axdef}
      TPK : Class
    \where
      TPK = \lblot \\
      \t1 constantPool == \{ \\
      \t2 1 \mapsto ClassRef~TPKClassID, \\
      \t2 3 \mapsto ClassRef~AperiodicEventHandlerClassID, \\
      \t2 8 \mapsto MethodRef~AperiodicEventHandlerClassID~APEHinit, \\
      \t2 27 \mapsto ClassRef~ConsoleConnectionClassID, \\
      \t2 29 \mapsto  MethodRef~ConsoleConnectionClassID~CCinit, \\
      \t2 32 \mapsto MethodRef~ConsoleConnectionClassID~openInputStream, \\
      \t2 36 \mapsto MethodRef~ConsoleConnectionClassID~openOutputStream, \\
      \t2 40 \mapsto MethodRef~InputStreamClassID~read, \\
      \t2 41 \mapsto ClassRef~InputStreamClassID, \\
      \t2 46 \mapsto MethodRef~TPKClassID~f, \\
      \t2 50 \mapsto MethodRef~OutputStreamClassID~write, \\
      \t2 51 \mapsto ClassRef~OutputStreamClassID \\
      \t1 \}, \\
      \t1 this == 1, \\
      \t1 super == 3, \\
      \t1 interfaces == \{\}, \\
      \t1 methodEntry == \{ \\
      \t2 f \mapsto 43, \\
      \t2 handleAsyncEvent \mapsto 7, \\
      \t2 APEHinit \mapsto 0, \\
      \t1 \}, \\
      \t1 methodEnd == \{ \\
      \t2 f \mapsto 50, \\
      \t2 handleAsyncEvent \mapsto 42, \\
      \t2 APEHinit \mapsto 6 \\
      \t1 \}, \\
      \t1 methodLocals == \{ \\
      \t2 f \mapsto 1, \\
      \t2 handleAsyncEvent \mapsto 6, \\
      \t2 APEHinit \mapsto 5, \\
      \t1 \}, \\
      \t1 methodStackSize == \{ \\
      \t2 f \mapsto 2, \\
      \t2 handleAsyncEvent \mapsto 3, \\
      \t2 APEHinit \mapsto 5, \\
      \t1 \}, \\
      \t1 fields == \{\}, \\
      \t1 staticFields == \{\} \\
      \rblot
    \end{axdef}
    \begin{axdef}
      cs : ClassID \pfun Class
      \where
      cs = \{ \\
      \t1 TPKClassID \mapsto TPK \\
      \t1 \cdots \\
      \}
    \end{axdef}
    %\columnbreak
    \begin{axdef}
      bc : ProgramAddress \pfun Bytecode
      \where
      bc = \{ \\
      	\t1 0 \mapsto aload~0, \\
        \t1 1 \mapsto aload~1, \\
        \t1 2 \mapsto aload~2, \\
        \t1 3 \mapsto aload~3, \\
        \t1 4 \mapsto aload~4, \\
        \t1 5 \mapsto invokespecial~8, \\
        \t1 6 \mapsto return, \\
        \t1 7 \mapsto new~27, \\
        \t1 8 \mapsto dup, \\
        \t1 9 \mapsto aconst\_null, \\
        \t1 10 \mapsto invokespecial~29, \\
        \t1 11 \mapsto astore~1, \\
        \t1 12 \mapsto aload~1, \\
        \t1 13 \mapsto invokevirtual~32, \\
        \t1 14 \mapsto astore~2, \\
        \t1 15 \mapsto aload~1, \\
        \t1 16 \mapsto invokevirtual~36, \\
        \t1 17 \mapsto astore~3, \\
        \t1 18 \mapsto iconst~0, \\
        \t1 19 \mapsto astore~4, \\
        \t1 20 \mapsto goto~19, \\
        \t1 21 \mapsto aload~2, \\
        \t1 22 \mapsto invokevirtual~40, \\
        \t1 23 \mapsto invokestatic~46, \\
        \t1 24 \mapsto astore~5, \\
        \t1 25 \mapsto aload~5, \\
        \t1 26 \mapsto iconst~400, \\
        \t1 27 \mapsto if\_icmple~5, \\
        \t1 28 \mapsto aload~3, \\
        \t1 29 \mapsto iconst~0, \\
        \t1 30 \mapsto invokevirtual~50, \\
        \t1 31 \mapsto goto~4, \\
        \t1 32 \mapsto aload~3, \\
        \t1 33 \mapsto aload~5, \\
        \t1 34 \mapsto invokevirtual~50, \\
        \t1 35 \mapsto aload~4, \\
        \t1 36 \mapsto iconst~1, \\
        \t1 37 \mapsto iadd, \\
        \t1 38 \mapsto astore~4, \\
        \t1 39 \mapsto aload~4, \\
        \t1 40 \mapsto iconst~10, \\
        \t1 41 \mapsto if\_icmple~(\negate 20), \\
        \t1 42 \mapsto return, \\
        \t1 43 \mapsto aload~0, \\
        \t1 44 \mapsto aload~0, \\
        \t1 45 \mapsto iadd, \\
        \t1 46 \mapsto aload~0, \\
        \t1 47 \mapsto iadd, \\
        \t1 48 \mapsto iconst~5, \\
        \t1 49 \mapsto iadd, \\
        \t1 50 \mapsto areturn, \\
        \t1 {} \cdots {} \\
        \}
      \end{axdef}
      \vspace{0.1cm}
  \end{vwcol}
  \caption{The \Circus{} code corresponding to our example program}
  \label{example-model-figure}
\end{figure}%
Our example is based on the Trabb Pardo-Knuth
algorithm~\cite{knuth1980}, used for comparison of programming
languages, since it includes a variety of programming language
constructs that provide a good test of the strategy.
We have simplified the algorithm by removing the reading into an
array, since our bytecode subset does not include array operations.
Attempting to add arrays makes the example much longer, while not
giving any interesting insight into our compilation strategy.

We have also written the example as an SCJ program, with the algorithm
as the body of an aperiodic event handler, \texttt{TPK}, one or more
instances of which can be registered as part of a mission and released
during mission execution.
As already mentioned, each release of the handler causes its
\texttt{handleAsyncEvent()} method to be executed.
This method creates an instance of a \texttt{ConsoleConnection}, which
is the only standard input/output connection required by SCJ.
Instances of \texttt{InputStream} and \texttt{OutputStream} are then
obtained from the \texttt{ConsoleConnection}.

After the input and output streams have been obtained, we enter a for
loop in which an integer is read from the \texttt{InputStream}, a
static method \texttt{f()} is applied to it, and the result is output
if it is less than 400, otherwise 0 is output.
The method \texttt{f()} takes an integer as input, multiplies it by 3
and adds 5 to it.

The \texttt{TPK} class is part of a larger program that includes many
other classes, including a \texttt{Safelet}, a
\texttt{MissionSequencer}, a \texttt{Mission}, and the classes that
make up the SCJ API.
Considering these classes in our example would make the example much
larger and more complex, while not introducing any more interesting
aspects for the strategy to consider.
We, therefore, omit a presentation of these classes, though it should be
noted that they are part of the complete example.

Throughout the strategy we assume the extra classes have gone through
similar processing to that we illustrate for the $TPK$ class.
This adds little complexity to the strategy since the bytecode
instructions the strategy acts upon are placed in a contiguous array
that is acted upon consistently for all classes, and the current class
of a given bytecode instruction can always be determined from its
address in the array.

The Java code must be run through a Java compiler to generate the
corresponding bytecode, which then defines the $bc$ and $cs$ constants
of our model.
The $bc$ and $cs$ values for our example are shown in
Figure~\ref{example-model-figure}.


Applying the bytecode expansion on
line~\ref{algorithm-expand-bytecode} of Algorithm~\ref{epc-algorithm}
yields the $Running$ action shown in
Figure~\ref{bytecode-expansion-example-figure}.
\begin{figure}[t]
  \setlength{\zedindent}{0cm}
  \setlength{\zedtab}{0.3cm}
  \setlength{\zedleftsep}{0.1cm}
  \begin{circus}
    Running \circdef \\
    \t1 \circif frameStack = \emptyset \circthen \Skip \\
    \t1 {} \circelse frameStack \neq \emptyset \circthen {} \\
    \t2 \circif pc = 0 \circthen HandleAloadEPC(0) \circseq pc := 1 \\
    \t2 {} \circelse pc = 1 \circthen HandleAloadEPC(1) \circseq pc := 2 \\
    \t2 {} \circelse pc = 2 \circthen HandleAloadEPC(2) \circseq pc := 3 \\
    \t2 {} \circelse pc = 3 \circthen HandleAloadEPC(3) \circseq pc := 4 \\
    \t2 {} \circelse pc = 4 \circthen HandleAloadEPC(4) \circseq pc := 5 \\
    \t2 {} \circelse pc = 5 \circthen HandleInvokespecialEPC(8) \\
    \t2 {} \circelse pc = 6 \circthen HandleReturnEPC \\
    \t2 {} \circelse pc = 7 \circthen HandleNewEPC(27) \circseq pc := 8 \\
    \t2 {} \circelse pc = 8 \circthen HandleDupEPC \circseq pc := 9 \\
    \t2 {} \circelse pc = 9 \circthen HandleAconst\_nullEPC \circseq pc := 10 \\
    \t2 {} \circelse pc = 10 \circthen HandleInvokespecialEPC(29) \\
    \t2 {} \circelse pc = 11 \circthen HandleAstoreEPC(1) \circseq pc := 12 \\
    % \t2 {} \circelse pc = 12 \circthen HandleAloadEPC(1) \circseq pc := 13 \\
    % \t2 {} \circelse pc = 13 \circthen HandleInvokevirtualEPC(32) \\
    % \t2 {} \circelse pc = 14 \circthen HandleAstoreEPC(2) \circseq pc := 15 \\
    % \t2 {} \circelse pc = 15 \circthen HandleAloadEPC(1) \circseq pc := 16 \\
    % \t2 {} \circelse pc = 16 \circthen HandleInvokevirtualEPC(36) \\
    % \t2 {} \circelse pc = 17 \circthen HandleAstoreEPC(3) \circseq pc := 18 \\
    % \t2 {} \circelse pc = 18 \circthen HandleIconstEPC(0) \circseq pc := 19 \\
    % \t2 {} \circelse pc = 19 \circthen HandleAstoreEPC(4) \circseq pc := 20 \\
    % \t2 {} \circelse pc = 20 \circthen pc := 39 \\
    \t2 {} \cdots {} \\
    % \t2 {} \circelse pc = 21 \circthen HandleAloadEPC(2) \circseq pc := 22 \\
    % \t2 {} \circelse pc = 22 \circthen HandleInvokevirtualEPC(40) \\
    % \t2 {} \circelse pc = 23 \circthen HandleInvokestaticEPC(46) \\
    % \t2 {} \circelse pc = 24 \circthen HandleAstoreEPC(5) \circseq pc := 25 \\
    % \t2 {} \circelse pc = 25 \circthen HandleAloadEPC(5) \circseq pc := 26 \\
    % \t2 {} \circelse pc = 26 \circthen HandleIconstEPC(400) \circseq pc := 27 \\
    % \t2 {} \circelse pc = 27 \circthen \circvar value1, value2 : Word \circspot InterpreterPop2 \circseq \\
    % \t3 pc := \IF value1 \leq value2 \THEN 32 \ELSE 28 \\
    % \t2 {} \circelse pc = 28 \circthen HandleAloadEPC(3) \circseq pc := 29 \\
    % \t2 {} \circelse pc = 29 \circthen HandleIconstEPC(0) \circseq pc := 30 \\
    % \t2 {} \circelse pc = 30 \circthen HandleInvokevirtualEPC(50) \\
    % \t2 {} \circelse pc = 31 \circthen pc := 35 \\
    % \t2 {} \circelse pc = 32 \circthen HandleAloadEPC(3) \circseq pc := 33 \\
    % \t2 {} \circelse pc = 33 \circthen HandleAloadEPC(5) \circseq pc := 34 \\
    % \t2 {} \circelse pc = 34 \circthen HandleInvokevirtualEPC(50) \\
    % \t2 {} \circelse pc = 35 \circthen HandleAloadEPC(4) \circseq pc := 36 \\
    % \t2 {} \circelse pc = 36 \circthen HandleIconstEPC(1) \circseq pc := 37 \\
    % \t2 {} \circelse pc = 37 \circthen HandleIaddEPC \circseq pc := 38 \\
    % \t2 {} \circelse pc = 38 \circthen HandleAstoreEPC(4) \circseq pc := 39 \\
    % \t2 {} \circelse pc = 39 \circthen HandleAloadEPC(4) \circseq pc := 40 \\
    % \t2 {} \circelse pc = 40 \circthen HandleIconstEPC(10) \circseq pc := 41 \\
    % \t2 {} \circelse pc = 41 \circthen \circvar value1, value2 : Word \circspot InterpreterPop2 \circseq \\
    % \t3 pc := \IF value1 \leq value2 \THEN 21 \ELSE 42 \\
    % \t2 {} \circelse pc = 42 \circthen HandleReturnEPC \\
    % \t2 {} \circelse pc = 43 \circthen HandleAloadEPC(0) \circseq pc := 44 \\
    % \t2 {} \circelse pc = 44 \circthen HandleAloadEPC(0) \circseq pc := 45 \\
    % \t2 {} \circelse pc = 45 \circthen HandleIaddEPC \circseq pc := 46 \\
    % \t2 {} \circelse pc = 46 \circthen HandleAloadEPC(0) \circseq pc := 47 \\
    % \t2 {} \circelse pc = 47 \circthen HandleIaddEPC \circseq pc := 48 \\
    % \t2 {} \circelse pc = 48 \circthen HandleIconstEPC(5) \circseq pc := 49 \\
    % \t2 {} \circelse pc = 49 \circthen HandleIaddEPC \circseq pc := 50 \\
    % \t2 {} \circelse pc = 50 \circthen HandleAreturnEPC \\
    \t2 \circfi \circseq Poll \circseq Running \\
    \t1 \circfi
  \end{circus}
  \caption{The $Running$ action after bytecode expansion}
  \label{bytecode-expansion-example-figure}
\end{figure}
This step expands the bytecode instruction definitions, by copying
$HandleInstruction$ into $Running$, and converting it to a choice of
actions based on the value of the program counter, mirroring the
contents of the $bc$ map for each value.

The actions that make up $HandleInstruction$ are also replaced with
actions that incorporate instruction parameters from the $bc$ map and
have $pc$ updates separated from stack updates so they can be more
easily operated on in this stage of the strategy.
This can be seen in Figure~\ref{bytecode-expansion-example-figure},
where, in the $pc = 0$ case, $aload~0$ has been converted to
$HandleAloadEPC(0) \circseq pc := 1$, with the parameter, $0$, to the
bytecode instruction becoming a parameter of the new instruction
handling action $HandleAloadEPC$, and the update to $pc$ placed after
the data operation.

The reason for making parameters of the bytecode instructions into
parameters of the handling actions is to remove the need to reference
the bytecode instructions in the $bc$ map, as that involves use of the
$pc$ value, which we seek to remove in this stage.
This also has the benefit of fully incorporating $bc$ into the $Thr$
process, ensuring all the information required to introduce C code
constructs is available directly in \Circus{}, which makes stating
compilation laws simpler.
This is described in more detail in
Section~\ref{expand-bytecode-subsection}, where we define the
\Call{ExpandBytecode}{} procedure.

\begin{figure}[t!]
  \setlength{\zedindent}{0cm}
  \setlength{\zedtab}{0.3cm}
  \setlength{\zedleftsep}{0.1cm}
  \begin{circus}
    Running \circdef \\
    \t1 \circif frameStack = \emptyset \circthen \Skip \\
    \t1 {} \circelse frameStack \neq \emptyset \circthen {} \\
    \t2 \circif pc = 0 \circthen HandleAloadEPC(0) \circseq pc := 1 \circseq Poll \circseq HandleAloadEPC(1) \circseq \\
    \t3 pc := 2 \circseq Poll \circseq HandleAloadEPC(2) \circseq pc := 3 \circseq Poll \circseq HandleAloadEPC(4) \circseq \\
    \t3 pc := 5 \circseq Poll \circseq HandleInvokespecialEPC(8) \\
    \t2 {} \cdots {} \\
    % \t2 {} \circelse pc = 1 \circthen HandleAloadEPC(1) \circseq pc := 2 \\
    % \t2 {} \circelse pc = 2 \circthen HandleAloadEPC(2) \circseq pc := 3 \\
    % \t2 {} \circelse pc = 3 \circthen HandleAloadEPC(3) \circseq pc := 4 \\
    % \t2 {} \circelse pc = 4 \circthen HandleAloadEPC(4) \circseq pc := 5 \\
    % \t2 {} \circelse pc = 5 \circthen HandleInvokespecialEPC(8) \\
    \t2 {} \circelse pc = 6 \circthen HandleReturnEPC \\
    \t2 {} \circelse pc = 7 \circthen HandleNewEPC(27) \circseq pc := 8 \circseq Poll \circseq HandleDupEPC \circseq pc := 9 \circseq \\
    \t3 Poll \circseq HandleAconst\_nullEPC \circseq pc := 10 \circseq Poll \circseq HandleInvokespecialEPC(29) \\
    \t2 {} \cdots {} \\
    % \t2 {} \circelse pc = 8 \circthen HandleDupEPC \circseq pc := 9 \\
    % \t2 {} \circelse pc = 9 \circthen HandleAconst\_nullEPC \circseq pc := 10 \\
    % \t2 {} \circelse pc = 10 \circthen HandleInvokespecialEPC(29) \\
    \t2 {} \circelse pc = 11 \circthen HandleAstoreEPC(1) \circseq pc := 12 \circseq Poll \circseq HandleAloadEPC(1) \circseq \\
    \t3 pc := 13 \circseq Poll \circseq HandleInvokevirtualEPC(32) \\
    \t2 {} \cdots {} \\
    % \t2 {} \circelse pc = 12 \circthen HandleAloadEPC(1) \circseq pc := 13 \\
    % \t2 {} \circelse pc = 13 \circthen HandleInvokevirtualEPC(32) \\
    \t2 {} \circelse pc = 14 \circthen HandleAstoreEPC(2) \circseq pc := 15 \circseq Poll \circseq HandleAloadEPC(1) \circseq \\
    \t3 pc := 16 \circseq Poll \circseq HandleInvokevirtualEPC(36) \\
    \t2 {} \cdots {} \\
    % \t2 {} \circelse pc = 15 \circthen HandleAloadEPC(1) \circseq pc := 16 \\
    % \t2 {} \circelse pc = 16 \circthen HandleInvokevirtualEPC(36) \\
    \t2 {} \circelse pc = 17 \circthen HandleAstoreEPC(3) \circseq pc := 18 \circseq Poll \circseq HandleIconstEPC(0) \circseq \\
    \t3 pc := 19 \circseq Poll \circseq HandleAstoreEPC(4) \circseq pc := 20 \circseq Poll \circseq pc := 39 \\
    \t2 {} \cdots {} \\
    % \t2 {} \circelse pc = 18 \circthen HandleIconstEPC(0) \circseq pc := 19 \\
    % \t2 {} \circelse pc = 19 \circthen HandleAstoreEPC(4) \circseq pc := 20 \\
    % \t2 {} \circelse pc = 20 \circthen pc := 39 \\
    % \t2 {} \circelse pc = 21 \circthen HandleAloadEPC(2) \circseq pc := 22 \circseq Poll \circseq HandleInvokevirtualEPC(40) \\
    % \t2 {} \circelse pc = 22 \circthen HandleInvokevirtualEPC(40) \\
    % \t2 {} \circelse pc = 23 \circthen HandleInvokestaticEPC(46) \\
    % \t2 {} \circelse pc = 24 \circthen HandleAstoreEPC(5) \circseq pc := 25 \circseq Poll \circseq HandleAloadEPC(5) \circseq \\
    % \t3 pc := 26 \circseq Poll \circseq HandleIconstEPC(400) \circseq pc := 27 \circseq Poll \circseq \\
    % \t3 \circvar value1, value2 : Word \circspot InterpreterPop2 \circseq \\
    % \t3 pc := \IF value1 \leq value2 \THEN 32 \ELSE 28 \\
    % \t2 {} \circelse pc = 25 \circthen HandleAloadEPC(5) \circseq pc := 26 \\
    % \t2 {} \circelse pc = 26 \circthen HandleIconstEPC(400) \circseq pc := 27 \\
    % \t2 {} \circelse pc = 27 \circthen \circvar value1, value2 : Word \circspot InterpreterPop2 \circseq \\
    % \t3 pc := \IF value1 \leq value2 \THEN 32 \ELSE 28 \\
    % \t2 {} \circelse pc = 28 \circthen HandleAloadEPC(3) \circseq pc := 29 \circseq Poll \circseq HandleIconstEPC(0) \circseq \\
    % \t3 pc := 30 \circseq Poll \circseq HandleInvokevirtualEPC(50) \\
    % \t2 {} \circelse pc = 29 \circthen HandleIconstEPC(0) \circseq pc := 30 \\
    % \t2 {} \circelse pc = 30 \circthen HandleInvokevirtualEPC(50) \\
    % \t2 {} \circelse pc = 31 \circthen pc := 35 \\
    % \t2 {} \circelse pc = 32 \circthen HandleAloadEPC(3) \circseq pc := 33 \circseq Poll \circseq HandleAloadEPC(5) \circseq \\
    % \t3 pc := 34 \circseq Poll \circseq HandleInvokevirtualEPC(50) \\
    % \t2 {} \circelse pc = 33 \circthen HandleAloadEPC(5) \circseq pc := 34 \\
    % \t2 {} \circelse pc = 34 \circthen HandleInvokevirtualEPC(50) \\
    % \t2 {} \circelse pc = 35 \circthen HandleAloadEPC(4) \circseq pc := 36 \circseq Poll \circseq HandleIconstEPC(1) \circseq \\
    % \t3 pc := 37 \circseq Poll \circseq HandleIaddEPC \circseq pc := 38 \circseq Poll \circseq HandleAstoreEPC(4) \circseq \\
    % \t3 pc := 39 \\
    % \t2 {} \circelse pc = 36 \circthen HandleIconstEPC(1) \circseq pc := 37 \\
    % \t2 {} \circelse pc = 37 \circthen HandleIaddEPC \circseq pc := 38 \\
    % \t2 {} \circelse pc = 38 \circthen HandleAstoreEPC(4) \circseq pc := 39 \\
    \t2 {} \circelse pc = 39 \circthen HandleAloadEPC(4) \circseq pc := 40 \circseq Poll \circseq HandleIconstEPC(10) \circseq \\
    \t3 pc := 41 \circseq Poll \circseq \circvar value1, value2 : Word \circspot InterpreterPop2 \circseq \\
    \t3 pc := \IF value1 \leq value2 \THEN 21 \ELSE 42 \\
    \t2 {} \cdots {} \\
    % \t2 {} \circelse pc = 40 \circthen HandleIconstEPC(10) \circseq pc := 41 \\
    % \t2 {} \circelse pc = 41 \circthen \circvar value1, value2 : Word \circspot InterpreterPop2 \circseq \\
    % \t3 pc := \IF value1 \leq value2 \THEN 21 \ELSE 42 \\
    \t2 {} \circelse pc = 42 \circthen HandleReturnEPC \\
    \t2 {} \circelse pc = 43 \circthen HandleAloadEPC(0) \circseq pc := 44 \circseq Poll \circseq HandleAloadEPC(0) \circseq \\
    \t3 pc := 45 \circseq Poll \circseq HandleIaddEPC \circseq pc := 46 \circseq Poll \circseq HandleAloadEPC(0) \circseq \\
    \t3 pc := 47 \circseq Poll \circseq HandleIaddEPC \circseq pc := 48 \circseq Poll \circseq HandleIaddEPC \circseq \\
    \t3 pc := 50 \circseq Poll \circseq HandleAreturnEPC \\
    \t2 {} \cdots {} \\
    % \t2 {} \circelse pc = 44 \circthen HandleAloadEPC(0) \circseq pc := 45 \\
    % \t2 {} \circelse pc = 45 \circthen HandleIaddEPC \circseq pc := 46 \\
    % \t2 {} \circelse pc = 46 \circthen HandleAloadEPC(0) \circseq pc := 47 \\
    % \t2 {} \circelse pc = 47 \circthen HandleIaddEPC \circseq pc := 48 \\
    % \t2 {} \circelse pc = 48 \circthen HandleIconstEPC(5) \circseq pc := 49 \\
    % \t2 {} \circelse pc = 49 \circthen HandleIaddEPC \circseq pc := 50 \\
    % \t2 {} \circelse pc = 50 \circthen HandleAreturnEPC \\
    \t2 \circfi \circseq Poll \circseq Running \\
    \t1 \circfi
  \end{circus}
  \caption{The $Running$ action after forward sequence introduction}
  \label{forward-sequence-introduction-example-figure}
\end{figure}

On line~\ref{algorithm-introduce-forward-sequence} of the algorithm,
sequential composition is introduced for instructions that do not
affect the sequential flow of the program.
Such instructions are identified by considering the control flow graph
of the program and locating nodes with a single outgoing edge going to
target node exactly one incoming edge.
The introduction of sequential composition is performed by unrolling
the loop in $Running$ to introduce the control flow following each of
these instructions.
This causes the instruction to be sequentially composed with the next
instruction, with $Poll$ in between to allow for thread switches
between instructions.
This is performed exhaustively to get the code in the form shown in
Figure~\ref{forward-sequence-introduction-example-figure}, where the
choice over $pc$ has sequences of instructions collected together at
the point where they start, up to the point at which a more complex
control flow (such as a method call, conditional or a loop) occurs.
The introduction of sequential composition is described in more detail in
Section~\ref{introduce-forward-sequence-subsection}, where we define
the \Call{IntroduceSequentialComposition}{} procedure.

Handling the remaining constructs requires consideration of dependency
between methods to ensure method calls can be resolved correctly.
We say a method call is \emph{resolved} when the method invocation
bytecode has been placed in sequential composition with a call to a
\Circus{} action containing the body of the method being invoked,
which is then followed by the sequence of instructions that occur
after the invocation bytecode instruction in the calling method.
After a method call has been resolved, it no longer breaks up the
sequence of instructions it occurs in.

Since we have the bytecode instructions of all the methods needed, we
can always resolve the call of a complete method, provided that method
has already been split into its own \Circus{} action.
To ensure the method that a method call depends on is complete, we
first perform loop and conditional introduction upon it.
Since introducing loops and conditionals requires unbroken sequences
of instructions that form the bodies of loops and branches of
conditionals, introduction of loops and conditionals can only be
performed on methods that have no unresolved method calls.
For this reason, we perform method call resolution and loop and
conditional introduction repeatedly until all method calls are
resolved and the resulting complete methods have all been separated
out.
This is expressed in Algorithm~\ref{epc-algorithm} by the while loop
on line~\ref{algorithm-method-loop}.

Introduction of loops and conditionals to the body of a method with no
unresolved method calls occurs on
line~\ref{algorithm-introduce-loops-and-conditionals} of the
algorithm.
To introduce loops and conditionals we consider the control flow graph
of the method again, though it is now much simpler than the control
flow graph used for sequence introduction since straight sequences of
instructions have already been combined together.
Patterns representing conditionals and loops are then identified using
the control flow graph and the corresponding constructs are
introduced.
As loops and conditionals are introduced, nodes in the control flow
graph are merged until the graph consists of a single node, which is
the starting point of the method, containing the complete method body.

In our example, \texttt{handleAsyncEvent()} is the only method that
needs loops and conditionals introducing but, since it also contains
method calls that break up the body of a loop, we must wait until its
method calls have been resolved before introducing loops and
conditionals.
The result of introducing loops and conditionals after method calls
have been resolved in \texttt{handleAsyncEvent()} is shown in
Figure~\ref{loop-and-conditional-introduction-example-figure}.
The process of introducing loops and conditionals is described in more
detail in Section~\ref{introduce-loops-and-conditionals-subsection},
where we define the \Call{IntroduceLoopsAndConditionals}{} procedure.
\begin{figure}[t!]
  \setlength{\zedindent}{0cm}
  \setlength{\zedtab}{0.3cm}
  \setlength{\zedleftsep}{0.1cm}
  \begin{circus}
    Running \circdef \\
    \t1 \circif frameStack = \emptyset \circthen \Skip \\
    \t1 {} \circelse frameStack \neq \emptyset \circthen {} \\
    \t2 {} \circif pc = 0 \circthen HandleAloadEPC(0) \circseq pc := 1 \circseq Poll \circseq HandleAloadEPC(1) \circseq \\
    \t3 pc := 2 \circseq Poll \circseq HandleAloadEPC(2) \circseq pc := 3 \circseq Poll \circseq HandleAloadEPC(3) \circseq \\
    \t3 pc := 4 \circseq Poll \circseq HandleAloadEPC(4) \circseq pc := 5 \circseq Poll \circseq \\
    \t3 HandleInvokespecialEPC(8) \circseq Poll \circseq AperiodicEventHandler\_APEHInit \circseq \\
    \t3 Poll \circseq HandleReturnEPC \\
    \t2 {} \cdots {} \\
    \t2 {} \circelse pc = 7 \circthen HandleNewEPC(27) \circseq pc := 8 \circseq Poll \circseq HandleDupEPC \circseq pc := 9 \circseq \\
    \t3 {} \cdots {} \\
    % \t3 Poll \circseq HandleAconst\_nullEPC \circseq pc := 10 \circseq Poll \circseq HandleInvokespecialEPC(29) \circseq \\
    % \t3 Poll \circseq ConsoleConnection\_CCInit \circseq Poll \circseq HandleAstoreEPC(1) \circseq pc := 12 \circseq Poll \circseq \\
    % \t3 HandleAloadEPC(1) \circseq pc := 13 \circseq Poll \circseq HandleInvokevirtualEPC(32) \circseq Poll \circseq \\
    % \t3 OpenInputStream \circseq Poll \circseq HandleAstoreEPC(2) \circseq pc := 15 \circseq Poll \circseq \\
    % \t3 HandleAloadEPC(1) \circseq pc := 16 \circseq Poll \circseq HandleInvokevirtualEPC(36) \circseq Poll \circseq \\
    % \t3 OpenOutputStream \circseq Poll \circseq HandleAstoreEPC(3) \circseq pc := 18 \circseq Poll \circseq \\
    % \t3 HandleIconstEPC(0) \circseq pc := 19 \circseq Poll \circseq HandleAstoreEPC(4) \circseq pc := 20 \circseq \\
    \t3 Poll \circseq pc := 39 \circseq Poll \circseq \circmu Y \circspot \\
    \t4 HandleAloadEPC(4) \circseq pc := 40 \circseq Poll \circseq HandleIconstEPC(10) \circseq pc := 41 \circseq \\
    \t4 Poll \circseq \circvar value1, value2 : Word \circspot InterpreterPop2 \circseq \\
    \t4 pc := \IF value1 \leq value2 \THEN 21 \ELSE 42 \circseq Poll \circseq \\
    \t4 \circif value1 \leq value2 \circthen HandleAloadEPC(2) \circseq pc := 22 \circseq Poll \circseq \\
    \t5 {} \cdots {} \\
    \t5 HandleIconstEPC(400) \circseq pc := 27 \circseq Poll \circseq \circvar value1, value2 : Word \circspot \\
    \t5 InterpreterPop2 \circseq pc := \IF value1 \leq value2 \THEN 32 \ELSE 28 \circseq Poll \circseq \\
    \t5 \circif value1 \leq value2 \circthen HandleAloadEPC(3) \circseq pc := 33 \circseq Poll \circseq \\
    \t6 HandleAloadEPC(5) \circseq pc := 34 \circseq Poll \circseq HandleInvokevirtualEPC(50) \circseq \\
    \t6 Poll \circseq ConsoleOutput\_Write \\
    \t5 {} \circelse value1 > value2 \circthen HandleAloadEPC(3) \circseq pc := 29 \circseq Poll \circseq \\
    \t6  HandleIconstEPC(0) \circseq pc := 30 \circseq  Poll \circseq HandleInvokevirtualEPC(50) \circseq \\
    \t6 Poll \circseq ConsoleOutput\_Write \\
    \t5 \circfi \circseq pc := 35 \circseq Poll \circseq  HandleAloadEPC(4) \circseq pc := 36 \circseq Poll \circseq \\
    \t5 HandleIconstEPC(1) \circseq pc := 37 \circseq Poll \circseq HandleIaddEPC \circseq pc := 38 \circseq \\
    \t5 Poll \circseq HandleAstoreEPC(4) \circseq pc := 39 \circseq Poll \circseq Y \\
    \t5 {} \circelse value1 > value2 \circthen  HandleReturnEPC \\
    \t4 \circfi \\
    % \t2 {} \circelse pc = 8 \circthen HandleDupEPC \circseq pc := 9 \circseq Poll \circseq HandleAconst\_nullEPC \circseq pc := 10 \circseq \\
    % \t3 Poll \circseq HandleInvokespecialEPC(29) \circseq Poll \circseq CCInit \circseq Poll \circseq HandleAstoreEPC(1) \circseq \\
    % \t3 pc := 12 \circseq Poll \circseq HandleAloadEPC(1) \circseq pc := 13 \circseq Poll \circseq \\
    % \t3 HandleInvokevirtualEPC(32) \circseq Poll \circseq OpenInputStream \circseq Poll \circseq HandleAstoreEPC(2) \circseq \\
    % \t3 pc := 15 \circseq Poll \circseq HandleAloadEPC(1) \circseq pc := 16 \circseq Poll \circseq HandleInvokevirtualEPC(36) \circseq Poll \circseq OpenOutputStream \circseq Poll \circseq HandleAstoreEPC(3) \circseq pc := 18 \circseq Poll \circseq HandleIconstEPC(0) \circseq \\
    % \t3 pc := 19 \circseq Poll \circseq HandleAstoreEPC(4) \circseq pc := 20 \circseq Poll \circseq pc := 39 \circseq Poll \circseq \\
    % \t3 HandleAloadEPC(4) \circseq pc := 40 \circseq Poll \circseq HandleIconstEPC(10) \circseq pc := 41 \circseq \\
    % \t3 Poll \circseq \circvar value1, value2 : Word \circspot InterpreterPop2 \circseq \\
    % \t3 pc := \IF value1 \leq value2 \THEN 21 \ELSE 42 \\
    % \t2 {} \circelse pc = 9 \circthen HandleAconst\_nullEPC \circseq pc := 10 \circseq Poll \circseq HandleInvokespecialEPC(29) \circseq Poll \circseq CCInit \circseq Poll \circseq HandleAstoreEPC(1) \circseq pc := 12 \circseq Poll \circseq HandleAloadEPC(1) \circseq pc := 13 \circseq Poll \circseq HandleInvokevirtualEPC(32) \circseq Poll \circseq OpenInputStream \circseq Poll \circseq HandleAstoreEPC(2) \circseq pc := 15 \circseq Poll \circseq HandleAloadEPC(1) \circseq pc := 16 \circseq Poll \circseq HandleInvokevirtualEPC(36) \circseq Poll \circseq OpenOutputStream \circseq Poll \circseq HandleAstoreEPC(3) \circseq pc := 18 \circseq Poll \circseq HandleIconstEPC(0) \circseq \\
    % \t3 pc := 19 \circseq Poll \circseq HandleAstoreEPC(4) \circseq pc := 20 \circseq Poll \circseq pc := 39 \circseq Poll \circseq \\
    % \t3 HandleAloadEPC(4) \circseq pc := 40 \circseq Poll \circseq HandleIconstEPC(10) \circseq pc := 41 \circseq \\
    % \t3 Poll \circseq \circvar value1, value2 : Word \circspot InterpreterPop2 \circseq \\
    % \t3 pc := \IF value1 \leq value2 \THEN 21 \ELSE 42 \\
    % \t2 {} \circelse pc = 10 \circthen HandleInvokespecialEPC(29) \circseq Poll \circseq CCInit \circseq Poll \circseq HandleAstoreEPC(1) \circseq pc := 12 \circseq Poll \circseq HandleAloadEPC(1) \circseq pc := 13 \circseq Poll \circseq HandleInvokevirtualEPC(32) \circseq Poll \circseq OpenInputStream \circseq Poll \circseq HandleAstoreEPC(2) \circseq pc := 15 \circseq Poll \circseq HandleAloadEPC(1) \circseq pc := 16 \circseq Poll \circseq HandleInvokevirtualEPC(36) \circseq Poll \circseq OpenOutputStream \circseq Poll \circseq HandleAstoreEPC(3) \circseq pc := 18 \circseq Poll \circseq HandleIconstEPC(0) \circseq \\
    % \t3 pc := 19 \circseq Poll \circseq HandleAstoreEPC(4) \circseq pc := 20 \circseq Poll \circseq pc := 39 \circseq Poll \circseq \\
    % \t3 HandleAloadEPC(4) \circseq pc := 40 \circseq Poll \circseq HandleIconstEPC(10) \circseq pc := 41 \circseq \\
    % \t3 Poll \circseq \circvar value1, value2 : Word \circspot InterpreterPop2 \circseq \\
    % \t3 pc := \IF value1 \leq value2 \THEN 21 \ELSE 42 \\
    % \t2 {} \circelse pc = 11 \circthen HandleAstoreEPC(1) \circseq pc := 12 \circseq Poll \circseq HandleAloadEPC(1) \circseq \\
    % \t3 pc := 13 \circseq Poll \circseq HandleInvokevirtualEPC(32) \circseq Poll \circseq OpenInputStream \circseq Poll \circseq HandleAstoreEPC(2) \circseq pc := 15 \circseq Poll \circseq HandleAloadEPC(1) \circseq pc := 16 \circseq Poll \circseq HandleInvokevirtualEPC(36) \circseq Poll \circseq OpenOutputStream \circseq Poll \circseq HandleAstoreEPC(3) \circseq pc := 18 \circseq Poll \circseq HandleIconstEPC(0) \circseq \\
    % \t3 pc := 19 \circseq Poll \circseq HandleAstoreEPC(4) \circseq pc := 20 \circseq Poll \circseq pc := 39 \circseq Poll \circseq \\
    % \t3 HandleAloadEPC(4) \circseq pc := 40 \circseq Poll \circseq HandleIconstEPC(10) \circseq pc := 41 \circseq \\
    % \t3 Poll \circseq \circvar value1, value2 : Word \circspot InterpreterPop2 \circseq \\
    % \t3 pc := \IF value1 \leq value2 \THEN 21 \ELSE 42 \\
    % \t2 {} \circelse pc = 12 \circthen HandleAloadEPC(1) \circseq pc := 13 \circseq Poll \circseq HandleInvokevirtualEPC(32) \circseq Poll \circseq OpenInputStream \circseq Poll \circseq HandleAstoreEPC(2) \circseq pc := 15 \circseq Poll \circseq HandleAloadEPC(1) \circseq pc := 16 \circseq Poll \circseq HandleInvokevirtualEPC(36) \circseq Poll \circseq OpenOutputStream \circseq Poll \circseq HandleAstoreEPC(3) \circseq pc := 18 \circseq Poll \circseq HandleIconstEPC(0) \circseq \\
    % \t3 pc := 19 \circseq Poll \circseq HandleAstoreEPC(4) \circseq pc := 20 \circseq Poll \circseq pc := 39 \circseq Poll \circseq \\
    % \t3 HandleAloadEPC(4) \circseq pc := 40 \circseq Poll \circseq HandleIconstEPC(10) \circseq pc := 41 \circseq \\
    % \t3 Poll \circseq \circvar value1, value2 : Word \circspot InterpreterPop2 \circseq \\
    % \t3 pc := \IF value1 \leq value2 \THEN 21 \ELSE 42 \\
    % \t2 {} \circelse pc = 13 \circthen HandleInvokevirtualEPC(32) \circseq Poll \circseq OpenInputStream \circseq Poll \circseq HandleAstoreEPC(2) \circseq pc := 15 \circseq Poll \circseq HandleAloadEPC(1) \circseq pc := 16 \circseq Poll \circseq HandleInvokevirtualEPC(36) \circseq Poll \circseq OpenOutputStream \circseq Poll \circseq HandleAstoreEPC(3) \circseq pc := 18 \circseq Poll \circseq HandleIconstEPC(0) \circseq \\
    % \t3 pc := 19 \circseq Poll \circseq HandleAstoreEPC(4) \circseq pc := 20 \circseq Poll \circseq pc := 39 \circseq Poll \circseq \\
    % \t3 HandleAloadEPC(4) \circseq pc := 40 \circseq Poll \circseq HandleIconstEPC(10) \circseq pc := 41 \circseq \\
    % \t3 Poll \circseq \circvar value1, value2 : Word \circspot InterpreterPop2 \circseq \\
    % \t3 pc := \IF value1 \leq value2 \THEN 21 \ELSE 42 \\
    % \t2 {} \circelse pc = 14 \circthen HandleAstoreEPC(2) \circseq pc := 15 \circseq Poll \circseq HandleAloadEPC(1) \circseq \\
    % \t3 pc := 16 \circseq Poll \circseq HandleInvokevirtualEPC(36) \circseq Poll \circseq OpenOutputStream \circseq Poll \circseq HandleAstoreEPC(3) \circseq pc := 18 \circseq Poll \circseq HandleIconstEPC(0) \circseq \\
    % \t3 pc := 19 \circseq Poll \circseq HandleAstoreEPC(4) \circseq pc := 20 \circseq Poll \circseq pc := 39 \circseq Poll \circseq \\
    % \t3 HandleAloadEPC(4) \circseq pc := 40 \circseq Poll \circseq HandleIconstEPC(10) \circseq pc := 41 \circseq \\
    % \t3 Poll \circseq \circvar value1, value2 : Word \circspot InterpreterPop2 \circseq \\
    % \t3 pc := \IF value1 \leq value2 \THEN 21 \ELSE 42 \\
    % \t2 {} \circelse pc = 15 \circthen HandleAloadEPC(1) \circseq pc := 16 \circseq Poll \circseq HandleInvokevirtualEPC(36) \circseq Poll \circseq OpenOutputStream \circseq Poll \circseq HandleAstoreEPC(3) \circseq pc := 18 \circseq Poll \circseq HandleIconstEPC(0) \circseq \\
    % \t3 pc := 19 \circseq Poll \circseq HandleAstoreEPC(4) \circseq pc := 20 \circseq Poll \circseq pc := 39 \circseq Poll \circseq \\
    % \t3 HandleAloadEPC(4) \circseq pc := 40 \circseq Poll \circseq HandleIconstEPC(10) \circseq pc := 41 \circseq \\
    % \t3 Poll \circseq \circvar value1, value2 : Word \circspot InterpreterPop2 \circseq \\
    % \t3 pc := \IF value1 \leq value2 \THEN 21 \ELSE 42 \\
    % \t2 {} \circelse pc = 16 \circthen HandleInvokevirtualEPC(36) \circseq Poll \circseq OpenOutputStream \circseq Poll \circseq HandleAstoreEPC(3) \circseq pc := 18 \circseq Poll \circseq HandleIconstEPC(0) \circseq \\
    % \t3 pc := 19 \circseq Poll \circseq HandleAstoreEPC(4) \circseq pc := 20 \circseq Poll \circseq pc := 39 \circseq Poll \circseq \\
    % \t3 HandleAloadEPC(4) \circseq pc := 40 \circseq Poll \circseq HandleIconstEPC(10) \circseq pc := 41 \circseq \\
    % \t3 Poll \circseq \circvar value1, value2 : Word \circspot InterpreterPop2 \circseq \\
    % \t3 pc := \IF value1 \leq value2 \THEN 21 \ELSE 42 \\
    % \t2 {} \circelse pc = 17 \circthen HandleAstoreEPC(3) \circseq pc := 18 \circseq Poll \circseq HandleIconstEPC(0) \circseq \\
    % \t3 pc := 19 \circseq Poll \circseq HandleAstoreEPC(4) \circseq pc := 20 \circseq Poll \circseq pc := 39 \circseq Poll \circseq \\
    % \t3 HandleAloadEPC(4) \circseq pc := 40 \circseq Poll \circseq HandleIconstEPC(10) \circseq pc := 41 \circseq \\
    % \t3 Poll \circseq \circvar value1, value2 : Word \circspot InterpreterPop2 \circseq \\
    % \t3 pc := \IF value1 \leq value2 \THEN 21 \ELSE 42 \\
    % \t2 {} \circelse pc = 18 \circthen HandleIconstEPC(0) \circseq pc := 19 \circseq Poll \circseq HandleAstoreEPC(4) \circseq pc := 20 \circseq Poll \circseq pc := 39 \circseq Poll \circseq HandleAloadEPC(4) \circseq pc := 40 \circseq Poll \circseq HandleIconstEPC(10) \circseq pc := 41 \circseq Poll \circseq \circvar value1, value2 : Word \circspot InterpreterPop2 \circseq pc := \IF value1 \leq value2 \THEN 21 \ELSE 42 \\
    % \t2 {} \circelse pc = 19 \circthen HandleAstoreEPC(4) \circseq pc := 20 \circseq Poll \circseq pc := 39 \circseq Poll \circseq HandleAloadEPC(4) \circseq pc := 40 \circseq Poll \circseq HandleIconstEPC(10) \circseq pc := 41 \circseq Poll \circseq \circvar value1, value2 : Word \circspot InterpreterPop2 \circseq pc := \IF value1 \leq value2 \THEN 21 \ELSE 42 \\
    % \t2 {} \circelse pc = 20 \circthen pc := 39 \circseq Poll \circseq HandleAloadEPC(4) \circseq pc := 40 \circseq Poll \circseq HandleIconstEPC(10) \circseq pc := 41 \circseq Poll \circseq \circvar value1, value2 : Word \circspot InterpreterPop2 \circseq pc := \IF value1 \leq value2 \THEN 21 \ELSE 42 \\
    % \t2 {} \circelse pc = 21 \circthen HandleAloadEPC(2) \circseq pc := 22 \circseq Poll \circseq HandleInvokevirtualEPC(40) \circseq Poll \circseq Read \circseq Poll \circseq HandleInvokestaticEPC(46) \circseq Poll \circseq F \circseq Poll \circseq HandleAstoreEPC(5) \circseq pc := 25 \circseq Poll \circseq HandleAloadEPC(5) \circseq pc := 26 \circseq HandleIconstEPC(400) \circseq pc := 27 \circseq Poll \circseq \circvar value1, value2 : Word \circspot InterpreterPop2 \circseq pc := \IF value1 \leq value2 \THEN 32 \ELSE 28 \\
    % \t2 {} \circelse pc = 22 \circthen HandleInvokevirtualEPC(40) \circseq Poll \circseq Read \circseq Poll \circseq HandleInvokestaticEPC(46) \circseq Poll \circseq F \circseq Poll \circseq HandleAstoreEPC(5) \circseq pc := 25 \circseq Poll \circseq HandleAloadEPC(5) \circseq pc := 26 \circseq HandleIconstEPC(400) \circseq pc := 27 \circseq Poll \circseq \circvar value1, value2 : Word \circspot InterpreterPop2 \circseq pc := \IF value1 \leq value2 \THEN 32 \ELSE 28 \\
    % \t2 {} \circelse pc = 23 \circthen HandleInvokestaticEPC(46) \circseq Poll \circseq F \circseq Poll \circseq HandleAstoreEPC(5) \circseq pc := 25 \circseq Poll \circseq HandleAloadEPC(5) \circseq pc := 26 \circseq HandleIconstEPC(400) \circseq pc := 27 \circseq Poll \circseq \circvar value1, value2 : Word \circspot InterpreterPop2 \circseq pc := \IF value1 \leq value2 \THEN 32 \ELSE 28 \\
    % \t2 {} \circelse pc = 24 \circthen HandleAstoreEPC(5) \circseq pc := 25 \circseq Poll \circseq HandleAloadEPC(5) \circseq pc := 26 \circseq HandleIconstEPC(400) \circseq pc := 27 \circseq Poll \circseq \circvar value1, value2 : Word \circspot InterpreterPop2 \circseq pc := \IF value1 \leq value2 \THEN 32 \ELSE 28 \\
    % \t2 {} \circelse pc = 25 \circthen HandleAloadEPC(5) \circseq pc := 26 \circseq Poll \circseq HandleIconstEPC(400) \circseq pc := 27 \circseq Poll \circseq \circvar value1, value2 : Word \circspot InterpreterPop2 \circseq pc := \IF value1 \leq value2 \THEN 32 \ELSE 28 \\
    % \t2 {} \circelse pc = 26 \circthen HandleIconstEPC(400) \circseq pc := 27 \circseq Poll \circseq \circvar value1, value2 : Word \circspot InterpreterPop2 \circseq pc := \IF value1 \leq value2 \THEN 32 \ELSE 28 \\
    % \t2 {} \circelse pc = 27 \circthen \circvar value1, value2 : Word \circspot InterpreterPop2 \circseq \\
    % \t3 pc := \IF value1 \leq value2 \THEN 32 \ELSE 28 \\
    % \t2 {} \circelse pc = 28 \circthen HandleAloadEPC(3) \circseq pc := 29 \circseq Poll \circseq HandleIconstEPC(0) \circseq pc := 30 \circseq Poll \circseq HandleInvokevirtualEPC(50) \circseq Poll \circseq Write \circseq Poll \circseq pc := 38 \circseq Poll \circseq HandleAstoreEPC(4) \circseq pc := 39 \circseq Poll \circseq HandleAloadEPC(4) \circseq pc := 40 \circseq Poll \circseq HandleIconstEPC(10) \circseq pc := 41 \circseq Poll \circseq \circvar value1, value2 : Word \circspot InterpreterPop2 \circseq pc := \IF value1 \leq value2 \THEN 21 \ELSE 42 \\
    % \t2 {} \circelse pc = 29 \circthen HandleIconstEPC(0) \circseq pc := 30 \circseq Poll \circseq HandleInvokevirtualEPC(50) \circseq Poll \circseq Write \circseq Poll \circseq pc := 38 \circseq Poll \circseq HandleAstoreEPC(4) \circseq pc := 39 \circseq Poll \circseq HandleAloadEPC(4) \circseq pc := 40 \circseq Poll \circseq HandleIconstEPC(10) \circseq pc := 41 \circseq Poll \circseq \circvar value1, value2 : Word \circspot InterpreterPop2 \circseq pc := \IF value1 \leq value2 \THEN 21 \ELSE 42 \\
    % \t2 {} \circelse pc = 30 \circthen HandleInvokevirtualEPC(50) \circseq Poll \circseq Write \circseq Poll \circseq pc := 38 \circseq Poll \circseq HandleAstoreEPC(4) \circseq pc := 39 \circseq Poll \circseq HandleAloadEPC(4) \circseq pc := 40 \circseq Poll \circseq HandleIconstEPC(10) \circseq pc := 41 \circseq Poll \circseq \circvar value1, value2 : Word \circspot InterpreterPop2 \circseq pc := \IF value1 \leq value2 \THEN 21 \ELSE 42 \\
    % \t2 {} \circelse pc = 31 \circthen pc := 38 \circseq Poll \circseq HandleAstoreEPC(4) \circseq pc := 39 \circseq Poll \circseq HandleAloadEPC(4) \circseq pc := 40 \circseq Poll \circseq HandleIconstEPC(10) \circseq pc := 41 \circseq Poll \circseq \circvar value1, value2 : Word \circspot InterpreterPop2 \circseq pc := \IF value1 \leq value2 \THEN 21 \ELSE 42 \\
    % \t2 {} \circelse pc = 32 \circthen HandleAloadEPC(3) \circseq pc := 33 \circseq Poll \circseq HandleAloadEPC(5) \circseq pc := 34 \circseq Poll \circseq HandleInvokevirtualEPC(50) \circseq Poll \circseq Write \circseq Poll \circseq HandleAloadEPC(4) \circseq pc := 36 \circseq Poll \circseq HandleIconstEPC(1) \circseq pc := 37 \circseq Poll \circseq HandleIaddEPC \circseq pc := 38 \circseq Poll \circseq HandleAstoreEPC(4) \circseq pc := 39 \circseq Poll \circseq HandleAloadEPC(4) \circseq pc := 40 \circseq Poll \circseq HandleIconstEPC(10) \circseq pc := 41 \circseq Poll \circseq \circvar value1, value2 : Word \circspot InterpreterPop2 \circseq pc := \IF value1 \leq value2 \THEN 21 \ELSE 42 \\
    % \t2 {} \circelse pc = 33 \circthen HandleAloadEPC(5) \circseq pc := 34 \circseq Poll \circseq HandleInvokevirtualEPC(50) \circseq Poll \circseq Write \circseq Poll \circseq HandleAloadEPC(4) \circseq pc := 36 \circseq Poll \circseq HandleIconstEPC(1) \circseq pc := 37 \circseq Poll \circseq HandleIaddEPC \circseq pc := 38 \circseq Poll \circseq HandleAstoreEPC(4) \circseq pc := 39 \circseq Poll \circseq HandleAloadEPC(4) \circseq pc := 40 \circseq Poll \circseq HandleIconstEPC(10) \circseq pc := 41 \circseq Poll \circseq \circvar value1, value2 : Word \circspot InterpreterPop2 \circseq pc := \IF value1 \leq value2 \THEN 21 \ELSE 42 \\
    % \t2 {} \circelse pc = 34 \circthen HandleInvokevirtualEPC(50) \circseq Poll \circseq Write \circseq Poll \circseq HandleAloadEPC(4) \circseq pc := 36 \circseq Poll \circseq HandleIconstEPC(1) \circseq pc := 37 \circseq Poll \circseq HandleIaddEPC \circseq pc := 38 \circseq Poll \circseq HandleAstoreEPC(4) \circseq pc := 39 \circseq Poll \circseq HandleAloadEPC(4) \circseq pc := 40 \circseq Poll \circseq HandleIconstEPC(10) \circseq pc := 41 \circseq Poll \circseq \circvar value1, value2 : Word \circspot InterpreterPop2 \circseq pc := \IF value1 \leq value2 \THEN 21 \ELSE 42 \\
    % \t2 {} \circelse pc = 35 \circthen HandleAloadEPC(4) \circseq pc := 36 \circseq Poll \circseq HandleIconstEPC(1) \circseq pc := 37 \circseq Poll \circseq HandleIaddEPC \circseq pc := 38 \circseq Poll \circseq HandleAstoreEPC(4) \circseq pc := 39 \circseq Poll \circseq HandleAloadEPC(4) \circseq pc := 40 \circseq Poll \circseq HandleIconstEPC(10) \circseq pc := 41 \circseq Poll \circseq \circvar value1, value2 : Word \circspot InterpreterPop2 \circseq pc := \IF value1 \leq value2 \THEN 21 \ELSE 42 \\
    % \t2 {} \circelse pc = 36 \circthen HandleIconstEPC(1) \circseq pc := 37 \circseq Poll \circseq HandleIaddEPC \circseq pc := 38 \circseq Poll \circseq HandleAstoreEPC(4) \circseq pc := 39 \circseq Poll \circseq HandleAloadEPC(4) \circseq pc := 40 \circseq Poll \circseq HandleIconstEPC(10) \circseq pc := 41 \circseq Poll \circseq \circvar value1, value2 : Word \circspot InterpreterPop2 \circseq pc := \IF value1 \leq value2 \THEN 21 \ELSE 42 \\
    % \t2 {} \circelse pc = 37 \circthen HandleIaddEPC \circseq pc := 38 \circseq Poll \circseq HandleAstoreEPC(4) \circseq pc := 39 \circseq Poll \circseq HandleAloadEPC(4) \circseq pc := 40 \circseq Poll \circseq HandleIconstEPC(10) \circseq pc := 41 \circseq Poll \circseq \circvar value1, value2 : Word \circspot InterpreterPop2 \circseq pc := \IF value1 \leq value2 \THEN 21 \ELSE 42 \\
    % \t2 {} \circelse pc = 38 \circthen HandleAstoreEPC(4) \circseq pc := 39 \circseq Poll \circseq HandleAloadEPC(4) \circseq pc := 40 \circseq Poll \circseq HandleIconstEPC(10) \circseq pc := 41 \circseq Poll \circseq \circvar value1, value2 : Word \circspot InterpreterPop2 \circseq pc := \IF value1 \leq value2 \THEN 21 \ELSE 42 \\
    % \t2 {} \circelse pc = 39 \circthen HandleAloadEPC(4) \circseq pc := 40 \circseq Poll \circseq HandleIconstEPC(10) \circseq pc := 41 \circseq Poll \circseq \circvar value1, value2 : Word \circspot InterpreterPop2 \circseq pc := \IF value1 \leq value2 \THEN 21 \ELSE 42 \\
    % \t2 {} \circelse pc = 40 \circthen HandleIconstEPC(10) \circseq pc := 41 \circseq Poll \circseq \circvar value1, value2 : Word \circspot InterpreterPop2 \circseq pc := \IF value1 \leq value2 \THEN 21 \ELSE 42 \\
    % \t2 {} \circelse pc = 41 \circthen \circvar value1, value2 : Word \circspot InterpreterPop2 \circseq \\
    % \t3 pc := \IF value1 \leq value2 \THEN 21 \ELSE 42 \\
    % \t2 {} \circelse pc = 42 \circthen HandleReturnEPC \\
    \t2 {} \cdots {} \\
    \t2 {} \circelse pc = 43 \circthen TPK\_F \\
    \t2 {} \cdots {} \\
    \t2 \circfi \circseq Poll \circseq Running \\
    \t1 \circfi
  \end{circus}
  \caption{The $Running$ action after loop and conditional introduction}
  \label{loop-and-conditional-introduction-example-figure}
\end{figure}

After loops and conditionals have been introduced, methods that are
then complete can be copied into separate actions.
This occurs in line~\ref{algorithm-separate-complete-methods} of this
algorithm.
This is done with a simple application of the copy rule, replacing the
actions at the entry points of the split methods with references to
newly created method actions.
This can be seen in
Figure~\ref{method-call-resolution-example-figure}, where the $TPK\_F$
action has been created by splitting the sequence of actions for the
\texttt{f()} method of \texttt{TPK} from the $pc = 43$ case.
As this step is relatively simple, we do not explain it in a separate
section.
\begin{figure}[t]
  \setlength{\zedindent}{0cm}
  \setlength{\zedtab}{0.3cm}
  \setlength{\zedleftsep}{0cm}
  \begin{circus}
    Running \circdef \\
    \t1 \circif frameStack = \emptyset \circthen \Skip \\
    \t1 {} \circelse frameStack \neq \emptyset \circthen {} \\
    \t2 {} \circif pc = 0 \circthen HandleAloadEPC(0) \circseq pc := 1 \circseq Poll \circseq HandleAloadEPC(1) \circseq \\
    \t3 pc := 2 \circseq Poll \circseq HandleAloadEPC(2) \circseq pc := 3 \circseq Poll \circseq HandleAloadEPC(3) \circseq \\
    \t3 pc := 4 \circseq Poll \circseq HandleAloadEPC(4) \circseq pc := 5 \circseq Poll \circseq \\
    \t3 HandleInvokespecialEPC(8) \circseq Poll \circseq AperiodicEventHandler\_APEHInit \circseq \\
    \t3 Poll \circseq HandleReturnEPC \\
    \t2 {} \cdots {} \\
    % \t2 {} \circelse pc = 1 \circthen HandleAloadEPC(1) \circseq pc := 2 \circseq Poll \circseq HandleAloadEPC(2) \circseq pc := 3 \circseq \\
    % \t3 Poll \circseq HandleAloadEPC(3) \circseq pc := 4 \circseq Poll \circseq HandleAloadEPC(4) \circseq pc := 5 \circseq \\
    % \t3 Poll \circseq HandleInvokespecialEPC(8) \circseq Poll \circseq APEHInit \circseq Poll \circseq HandleReturnEPC \\
    % \t2 {} \circelse pc = 2 \circthen HandleAloadEPC(2) \circseq pc := 3 \circseq Poll \circseq HandleAloadEPC(3) \circseq pc := 4 \circseq \\
    % \t3 Poll \circseq HandleAloadEPC(4) \circseq pc := 5 \circseq Poll \circseq HandleInvokespecialEPC(8) \circseq \\
    % \t3 Poll \circseq APEHInit \circseq Poll \circseq HandleReturnEPC \\
    % \t2 {} \circelse pc = 3 \circthen HandleAloadEPC(3) \circseq pc := 4 \circseq Poll \circseq HandleAloadEPC(4) \circseq pc := 5 \circseq \\
    % \t3 Poll \circseq HandleInvokespecialEPC(8) \circseq Poll \circseq APEHInit \circseq Poll \circseq HandleReturnEPC \\
    % \t2 {} \circelse pc = 4 \circthen HandleAloadEPC(4) \circseq pc := 5 \circseq Poll \circseq HandleInvokespecialEPC(8) \circseq \\
    % \t3 Poll \circseq APEHInit \circseq Poll \circseq HandleReturnEPC \\
    % \t2 {} \circelse pc = 5 \circthen HandleInvokespecialEPC(8) \circseq Poll \circseq APEHInit \circseq Poll \circseq HandleReturnEPC \\
    % \t2 {} \circelse pc = 6 \circthen HandleReturnEPC \\
    % \t2 {} \circelse pc = 7 \circthen HandleNewEPC(27) \circseq pc := 8 \circseq Poll \circseq HandleDupEPC \circseq pc := 9 \circseq \\
    % \t3 Poll \circseq HandleAconst\_nullEPC \circseq pc := 10 \circseq Poll \circseq HandleInvokespecialEPC(29) \circseq \\
    % \t3 Poll \circseq ConsoleConnection\_CCInit \circseq Poll \circseq HandleAstoreEPC(1) \circseq pc := 12 \circseq Poll \circseq \\
    % \t3 HandleAloadEPC(1) \circseq pc := 13 \circseq Poll \circseq HandleInvokevirtualEPC(32) \circseq Poll \circseq \\
    % \t3 OpenInputStream \circseq Poll \circseq HandleAstoreEPC(2) \circseq pc := 15 \circseq Poll \circseq \\
    % \t3 HandleAloadEPC(1) \circseq pc := 16 \circseq Poll \circseq HandleInvokevirtualEPC(36) \circseq Poll \circseq \\
    % \t3 OpenOutputStream \circseq Poll \circseq HandleAstoreEPC(3) \circseq pc := 18 \circseq Poll \circseq \\
    % \t3 HandleIconstEPC(0) \circseq pc := 19 \circseq Poll \circseq HandleAstoreEPC(4) \circseq pc := 20 \circseq \\
    % \t3 Poll \circseq pc := 39 \\
    % \t2 {} \cdots {} \\
    % \t2 {} \circelse pc = 8 \circthen HandleDupEPC \circseq pc := 9 \circseq Poll \circseq HandleAconst\_nullEPC \circseq pc := 10 \circseq \\
    % \t3 Poll \circseq HandleInvokespecialEPC(29) \circseq Poll \circseq CCInit \circseq Poll \circseq HandleAstoreEPC(1) \circseq \\
    % \t3 pc := 12 \circseq Poll \circseq HandleAloadEPC(1) \circseq pc := 13 \circseq Poll \circseq \\
    % \t3 HandleInvokevirtualEPC(32) \circseq Poll \circseq OpenInputStream \circseq Poll \circseq HandleAstoreEPC(2) \circseq \\
    % \t3 pc := 15 \circseq Poll \circseq HandleAloadEPC(1) \circseq pc := 16 \circseq Poll \circseq HandleInvokevirtualEPC(36) \circseq Poll \circseq OpenOutputStream \circseq Poll \circseq HandleAstoreEPC(3) \circseq pc := 18 \circseq Poll \circseq HandleIconstEPC(0) \circseq \\
    % \t3 pc := 19 \circseq Poll \circseq HandleAstoreEPC(4) \circseq pc := 20 \circseq Poll \circseq pc := 39 \circseq Poll \circseq \\
    % \t3 HandleAloadEPC(4) \circseq pc := 40 \circseq Poll \circseq HandleIconstEPC(10) \circseq pc := 41 \circseq \\
    % \t3 Poll \circseq \circvar value1, value2 : Word \circspot InterpreterPop2 \circseq \\
    % \t3 pc := \IF value1 \leq value2 \THEN 21 \ELSE 42 \\
    % \t2 {} \circelse pc = 9 \circthen HandleAconst\_nullEPC \circseq pc := 10 \circseq Poll \circseq HandleInvokespecialEPC(29) \circseq Poll \circseq CCInit \circseq Poll \circseq HandleAstoreEPC(1) \circseq pc := 12 \circseq Poll \circseq HandleAloadEPC(1) \circseq pc := 13 \circseq Poll \circseq HandleInvokevirtualEPC(32) \circseq Poll \circseq OpenInputStream \circseq Poll \circseq HandleAstoreEPC(2) \circseq pc := 15 \circseq Poll \circseq HandleAloadEPC(1) \circseq pc := 16 \circseq Poll \circseq HandleInvokevirtualEPC(36) \circseq Poll \circseq OpenOutputStream \circseq Poll \circseq HandleAstoreEPC(3) \circseq pc := 18 \circseq Poll \circseq HandleIconstEPC(0) \circseq \\
    % \t3 pc := 19 \circseq Poll \circseq HandleAstoreEPC(4) \circseq pc := 20 \circseq Poll \circseq pc := 39 \circseq Poll \circseq \\
    % \t3 HandleAloadEPC(4) \circseq pc := 40 \circseq Poll \circseq HandleIconstEPC(10) \circseq pc := 41 \circseq \\
    % \t3 Poll \circseq \circvar value1, value2 : Word \circspot InterpreterPop2 \circseq \\
    % \t3 pc := \IF value1 \leq value2 \THEN 21 \ELSE 42 \\
    % \t2 {} \circelse pc = 10 \circthen HandleInvokespecialEPC(29) \circseq Poll \circseq CCInit \circseq Poll \circseq HandleAstoreEPC(1) \circseq pc := 12 \circseq Poll \circseq HandleAloadEPC(1) \circseq pc := 13 \circseq Poll \circseq HandleInvokevirtualEPC(32) \circseq Poll \circseq OpenInputStream \circseq Poll \circseq HandleAstoreEPC(2) \circseq pc := 15 \circseq Poll \circseq HandleAloadEPC(1) \circseq pc := 16 \circseq Poll \circseq HandleInvokevirtualEPC(36) \circseq Poll \circseq OpenOutputStream \circseq Poll \circseq HandleAstoreEPC(3) \circseq pc := 18 \circseq Poll \circseq HandleIconstEPC(0) \circseq \\
    % \t3 pc := 19 \circseq Poll \circseq HandleAstoreEPC(4) \circseq pc := 20 \circseq Poll \circseq pc := 39 \circseq Poll \circseq \\
    % \t3 HandleAloadEPC(4) \circseq pc := 40 \circseq Poll \circseq HandleIconstEPC(10) \circseq pc := 41 \circseq \\
    % \t3 Poll \circseq \circvar value1, value2 : Word \circspot InterpreterPop2 \circseq \\
    % \t3 pc := \IF value1 \leq value2 \THEN 21 \ELSE 42 \\
    % \t2 {} \circelse pc = 11 \circthen HandleAstoreEPC(1) \circseq pc := 12 \circseq Poll \circseq HandleAloadEPC(1) \circseq \\
    % \t3 pc := 13 \circseq Poll \circseq HandleInvokevirtualEPC(32) \circseq Poll \circseq OpenInputStream \circseq Poll \circseq HandleAstoreEPC(2) \circseq pc := 15 \circseq Poll \circseq HandleAloadEPC(1) \circseq pc := 16 \circseq Poll \circseq HandleInvokevirtualEPC(36) \circseq Poll \circseq OpenOutputStream \circseq Poll \circseq HandleAstoreEPC(3) \circseq pc := 18 \circseq Poll \circseq HandleIconstEPC(0) \circseq \\
    % \t3 pc := 19 \circseq Poll \circseq HandleAstoreEPC(4) \circseq pc := 20 \circseq Poll \circseq pc := 39 \circseq Poll \circseq \\
    % \t3 HandleAloadEPC(4) \circseq pc := 40 \circseq Poll \circseq HandleIconstEPC(10) \circseq pc := 41 \circseq \\
    % \t3 Poll \circseq \circvar value1, value2 : Word \circspot InterpreterPop2 \circseq \\
    % \t3 pc := \IF value1 \leq value2 \THEN 21 \ELSE 42 \\
    % \t2 {} \circelse pc = 12 \circthen HandleAloadEPC(1) \circseq pc := 13 \circseq Poll \circseq HandleInvokevirtualEPC(32) \circseq Poll \circseq OpenInputStream \circseq Poll \circseq HandleAstoreEPC(2) \circseq pc := 15 \circseq Poll \circseq HandleAloadEPC(1) \circseq pc := 16 \circseq Poll \circseq HandleInvokevirtualEPC(36) \circseq Poll \circseq OpenOutputStream \circseq Poll \circseq HandleAstoreEPC(3) \circseq pc := 18 \circseq Poll \circseq HandleIconstEPC(0) \circseq \\
    % \t3 pc := 19 \circseq Poll \circseq HandleAstoreEPC(4) \circseq pc := 20 \circseq Poll \circseq pc := 39 \circseq Poll \circseq \\
    % \t3 HandleAloadEPC(4) \circseq pc := 40 \circseq Poll \circseq HandleIconstEPC(10) \circseq pc := 41 \circseq \\
    % \t3 Poll \circseq \circvar value1, value2 : Word \circspot InterpreterPop2 \circseq \\
    % \t3 pc := \IF value1 \leq value2 \THEN 21 \ELSE 42 \\
    % \t2 {} \circelse pc = 13 \circthen HandleInvokevirtualEPC(32) \circseq Poll \circseq OpenInputStream \circseq Poll \circseq HandleAstoreEPC(2) \circseq pc := 15 \circseq Poll \circseq HandleAloadEPC(1) \circseq pc := 16 \circseq Poll \circseq HandleInvokevirtualEPC(36) \circseq Poll \circseq OpenOutputStream \circseq Poll \circseq HandleAstoreEPC(3) \circseq pc := 18 \circseq Poll \circseq HandleIconstEPC(0) \circseq \\
    % \t3 pc := 19 \circseq Poll \circseq HandleAstoreEPC(4) \circseq pc := 20 \circseq Poll \circseq pc := 39 \circseq Poll \circseq \\
    % \t3 HandleAloadEPC(4) \circseq pc := 40 \circseq Poll \circseq HandleIconstEPC(10) \circseq pc := 41 \circseq \\
    % \t3 Poll \circseq \circvar value1, value2 : Word \circspot InterpreterPop2 \circseq \\
    % \t3 pc := \IF value1 \leq value2 \THEN 21 \ELSE 42 \\
    % \t2 {} \circelse pc = 14 \circthen HandleAstoreEPC(2) \circseq pc := 15 \circseq Poll \circseq HandleAloadEPC(1) \circseq \\
    % \t3 pc := 16 \circseq Poll \circseq HandleInvokevirtualEPC(36) \circseq Poll \circseq OpenOutputStream \circseq Poll \circseq HandleAstoreEPC(3) \circseq pc := 18 \circseq Poll \circseq HandleIconstEPC(0) \circseq \\
    % \t3 pc := 19 \circseq Poll \circseq HandleAstoreEPC(4) \circseq pc := 20 \circseq Poll \circseq pc := 39 \circseq Poll \circseq \\
    % \t3 HandleAloadEPC(4) \circseq pc := 40 \circseq Poll \circseq HandleIconstEPC(10) \circseq pc := 41 \circseq \\
    % \t3 Poll \circseq \circvar value1, value2 : Word \circspot InterpreterPop2 \circseq \\
    % \t3 pc := \IF value1 \leq value2 \THEN 21 \ELSE 42 \\
    % \t2 {} \circelse pc = 15 \circthen HandleAloadEPC(1) \circseq pc := 16 \circseq Poll \circseq HandleInvokevirtualEPC(36) \circseq Poll \circseq OpenOutputStream \circseq Poll \circseq HandleAstoreEPC(3) \circseq pc := 18 \circseq Poll \circseq HandleIconstEPC(0) \circseq \\
    % \t3 pc := 19 \circseq Poll \circseq HandleAstoreEPC(4) \circseq pc := 20 \circseq Poll \circseq pc := 39 \circseq Poll \circseq \\
    % \t3 HandleAloadEPC(4) \circseq pc := 40 \circseq Poll \circseq HandleIconstEPC(10) \circseq pc := 41 \circseq \\
    % \t3 Poll \circseq \circvar value1, value2 : Word \circspot InterpreterPop2 \circseq \\
    % \t3 pc := \IF value1 \leq value2 \THEN 21 \ELSE 42 \\
    % \t2 {} \circelse pc = 16 \circthen HandleInvokevirtualEPC(36) \circseq Poll \circseq OpenOutputStream \circseq Poll \circseq HandleAstoreEPC(3) \circseq pc := 18 \circseq Poll \circseq HandleIconstEPC(0) \circseq \\
    % \t3 pc := 19 \circseq Poll \circseq HandleAstoreEPC(4) \circseq pc := 20 \circseq Poll \circseq pc := 39 \circseq Poll \circseq \\
    % \t3 HandleAloadEPC(4) \circseq pc := 40 \circseq Poll \circseq HandleIconstEPC(10) \circseq pc := 41 \circseq \\
    % \t3 Poll \circseq \circvar value1, value2 : Word \circspot InterpreterPop2 \circseq \\
    % \t3 pc := \IF value1 \leq value2 \THEN 21 \ELSE 42 \\
    % \t2 {} \circelse pc = 17 \circthen HandleAstoreEPC(3) \circseq pc := 18 \circseq Poll \circseq HandleIconstEPC(0) \circseq \\
    % \t3 pc := 19 \circseq Poll \circseq HandleAstoreEPC(4) \circseq pc := 20 \circseq Poll \circseq pc := 39 \circseq Poll \circseq \\
    % \t3 HandleAloadEPC(4) \circseq pc := 40 \circseq Poll \circseq HandleIconstEPC(10) \circseq pc := 41 \circseq \\
    % \t3 Poll \circseq \circvar value1, value2 : Word \circspot InterpreterPop2 \circseq \\
    % \t3 pc := \IF value1 \leq value2 \THEN 21 \ELSE 42 \\
    % \t2 {} \circelse pc = 18 \circthen HandleIconstEPC(0) \circseq pc := 19 \circseq Poll \circseq HandleAstoreEPC(4) \circseq pc := 20 \circseq Poll \circseq pc := 39 \circseq Poll \circseq HandleAloadEPC(4) \circseq pc := 40 \circseq Poll \circseq HandleIconstEPC(10) \circseq pc := 41 \circseq Poll \circseq \circvar value1, value2 : Word \circspot InterpreterPop2 \circseq pc := \IF value1 \leq value2 \THEN 21 \ELSE 42 \\
    % \t2 {} \circelse pc = 19 \circthen HandleAstoreEPC(4) \circseq pc := 20 \circseq Poll \circseq pc := 39 \circseq Poll \circseq HandleAloadEPC(4) \circseq pc := 40 \circseq Poll \circseq HandleIconstEPC(10) \circseq pc := 41 \circseq Poll \circseq \circvar value1, value2 : Word \circspot InterpreterPop2 \circseq pc := \IF value1 \leq value2 \THEN 21 \ELSE 42 \\
    % \t2 {} \circelse pc = 20 \circthen pc := 39 \circseq Poll \circseq HandleAloadEPC(4) \circseq pc := 40 \circseq Poll \circseq HandleIconstEPC(10) \circseq pc := 41 \circseq Poll \circseq \circvar value1, value2 : Word \circspot InterpreterPop2 \circseq pc := \IF value1 \leq value2 \THEN 21 \ELSE 42 \\
    \t2 {} \circelse pc = 21 \circthen HandleAloadEPC(2) \circseq pc := 22 \circseq Poll \circseq \\
    \t3 HandleInvokevirtualEPC(40) \circseq Poll \circseq ConsoleInput\_Read \circseq Poll \circseq \\
    \t3 HandleInvokestaticEPC(46) \circseq Poll \circseq TPK\_F \circseq Poll \circseq HandleAstoreEPC(5) \circseq \\
    \t3 pc := 25 \circseq Poll \circseq HandleAloadEPC(5) \circseq pc := 26 \circseq HandleIconstEPC(400) \circseq \\
    \t3 pc := 27 \circseq Poll \circseq \circvar value1, value2 : Word \circspot InterpreterPop2 \circseq \\
    \t3 pc := \IF value1 \leq value2 \THEN 32 \ELSE 28 \\
    % \t2 {} \circelse pc = 22 \circthen HandleInvokevirtualEPC(40) \circseq Poll \circseq Read \circseq Poll \circseq HandleInvokestaticEPC(46) \circseq Poll \circseq F \circseq Poll \circseq HandleAstoreEPC(5) \circseq pc := 25 \circseq Poll \circseq HandleAloadEPC(5) \circseq pc := 26 \circseq HandleIconstEPC(400) \circseq pc := 27 \circseq Poll \circseq \circvar value1, value2 : Word \circspot InterpreterPop2 \circseq pc := \IF value1 \leq value2 \THEN 32 \ELSE 28 \\
    % \t2 {} \circelse pc = 23 \circthen HandleInvokestaticEPC(46) \circseq Poll \circseq F \circseq Poll \circseq HandleAstoreEPC(5) \circseq pc := 25 \circseq Poll \circseq HandleAloadEPC(5) \circseq pc := 26 \circseq HandleIconstEPC(400) \circseq pc := 27 \circseq Poll \circseq \circvar value1, value2 : Word \circspot InterpreterPop2 \circseq pc := \IF value1 \leq value2 \THEN 32 \ELSE 28 \\
    % \t2 {} \circelse pc = 24 \circthen HandleAstoreEPC(5) \circseq pc := 25 \circseq Poll \circseq HandleAloadEPC(5) \circseq pc := 26 \circseq HandleIconstEPC(400) \circseq pc := 27 \circseq Poll \circseq \circvar value1, value2 : Word \circspot InterpreterPop2 \circseq pc := \IF value1 \leq value2 \THEN 32 \ELSE 28 \\
    % \t2 {} \circelse pc = 25 \circthen HandleAloadEPC(5) \circseq pc := 26 \circseq Poll \circseq HandleIconstEPC(400) \circseq pc := 27 \circseq Poll \circseq \circvar value1, value2 : Word \circspot InterpreterPop2 \circseq pc := \IF value1 \leq value2 \THEN 32 \ELSE 28 \\
    % \t2 {} \circelse pc = 26 \circthen HandleIconstEPC(400) \circseq pc := 27 \circseq Poll \circseq \circvar value1, value2 : Word \circspot InterpreterPop2 \circseq pc := \IF value1 \leq value2 \THEN 32 \ELSE 28 \\
    % \t2 {} \circelse pc = 27 \circthen \circvar value1, value2 : Word \circspot InterpreterPop2 \circseq \\
    % \t3 pc := \IF value1 \leq value2 \THEN 32 \ELSE 28 \\
    % \t2 {} \circelse pc = 28 \circthen HandleAloadEPC(3) \circseq pc := 29 \circseq Poll \circseq HandleIconstEPC(0) \circseq pc := 30 \circseq Poll \circseq HandleInvokevirtualEPC(50) \circseq Poll \circseq Write \circseq Poll \circseq pc := 38 \circseq Poll \circseq HandleAstoreEPC(4) \circseq pc := 39 \circseq Poll \circseq HandleAloadEPC(4) \circseq pc := 40 \circseq Poll \circseq HandleIconstEPC(10) \circseq pc := 41 \circseq Poll \circseq \circvar value1, value2 : Word \circspot InterpreterPop2 \circseq pc := \IF value1 \leq value2 \THEN 21 \ELSE 42 \\
    % \t2 {} \circelse pc = 29 \circthen HandleIconstEPC(0) \circseq pc := 30 \circseq Poll \circseq HandleInvokevirtualEPC(50) \circseq Poll \circseq Write \circseq Poll \circseq pc := 38 \circseq Poll \circseq HandleAstoreEPC(4) \circseq pc := 39 \circseq Poll \circseq HandleAloadEPC(4) \circseq pc := 40 \circseq Poll \circseq HandleIconstEPC(10) \circseq pc := 41 \circseq Poll \circseq \circvar value1, value2 : Word \circspot InterpreterPop2 \circseq pc := \IF value1 \leq value2 \THEN 21 \ELSE 42 \\
    % \t2 {} \circelse pc = 30 \circthen HandleInvokevirtualEPC(50) \circseq Poll \circseq Write \circseq Poll \circseq pc := 38 \circseq Poll \circseq HandleAstoreEPC(4) \circseq pc := 39 \circseq Poll \circseq HandleAloadEPC(4) \circseq pc := 40 \circseq Poll \circseq HandleIconstEPC(10) \circseq pc := 41 \circseq Poll \circseq \circvar value1, value2 : Word \circspot InterpreterPop2 \circseq pc := \IF value1 \leq value2 \THEN 21 \ELSE 42 \\
    % \t2 {} \circelse pc = 31 \circthen pc := 38 \circseq Poll \circseq HandleAstoreEPC(4) \circseq pc := 39 \circseq Poll \circseq HandleAloadEPC(4) \circseq pc := 40 \circseq Poll \circseq HandleIconstEPC(10) \circseq pc := 41 \circseq Poll \circseq \circvar value1, value2 : Word \circspot InterpreterPop2 \circseq pc := \IF value1 \leq value2 \THEN 21 \ELSE 42 \\
    % \t2 {} \circelse pc = 32 \circthen HandleAloadEPC(3) \circseq pc := 33 \circseq Poll \circseq HandleAloadEPC(5) \circseq pc := 34 \circseq Poll \circseq HandleInvokevirtualEPC(50) \circseq Poll \circseq Write \circseq Poll \circseq HandleAloadEPC(4) \circseq pc := 36 \circseq Poll \circseq HandleIconstEPC(1) \circseq pc := 37 \circseq Poll \circseq HandleIaddEPC \circseq pc := 38 \circseq Poll \circseq HandleAstoreEPC(4) \circseq pc := 39  \\
    % \t2 {} \circelse pc = 33 \circthen HandleAloadEPC(5) \circseq pc := 34 \\
    % \t2 {} \circelse pc = 34 \circthen HandleInvokevirtualEPC(50) \\
    % \t2 {} \circelse pc = 35 \circthen HandleAloadEPC(4) \circseq pc := 36 \circseq Poll \circseq HandleIconstEPC(1) \circseq \\
    % \t3 pc := 37 \circseq Poll \circseq HandleIaddEPC \circseq pc := 38 \circseq Poll \circseq HandleAstoreEPC(4) \circseq \\
    % \t3 pc := 39 \\
    % \t2 {} \circelse pc = 36 \circthen HandleIconstEPC(1) \circseq pc := 37 \\
    % \t2 {} \circelse pc = 37 \circthen HandleIaddEPC \circseq pc := 38 \\
    % \t2 {} \circelse pc = 38 \circthen HandleAstoreEPC(4) \circseq pc := 39 \\
    \t2 {} \circelse pc = 39 \circthen HandleAloadEPC(4) \circseq pc := 40 \circseq Poll \circseq HandleIconstEPC(10) \circseq \\
    \t3 pc := 41 \circseq Poll \circseq \circvar value1, value2 : Word \circspot InterpreterPop2 \circseq \\
    \t3 pc := \IF value1 \leq value2 \THEN 21 \ELSE 42 \\
    % \t2 \t2 {} \circelse pc = 40 \circthen HandleIconstEPC(10) \circseq pc := 41 \\
    % \t2 {} \circelse pc = 41 \circthen \circvar value1, value2 : Word \circspot InterpreterPop2 \circseq \\
    % \t3 pc := \IF value1 \leq value2 \THEN 21 \ELSE 42 \\
    % \t2 {} \circelse pc = 42 \circthen HandleReturnEPC \\
    \t2 {} \cdots {} \\
    \t2 {} \circelse pc = 43 \circthen TPK\_F \\
    \t2 {} \cdots {} \\
    % \t2 {} \circelse pc = 44 \circthen HandleAloadEPC(0) \circseq pc := 45 \\
    % \t2 {} \circelse pc = 45 \circthen HandleIaddEPC \circseq pc := 46 \\
    % \t2 {} \circelse pc = 46 \circthen HandleAloadEPC(0) \circseq pc := 47 \\
    % \t2 {} \circelse pc = 47 \circthen HandleIaddEPC \circseq pc := 48 \\
    % \t2 {} \circelse pc = 48 \circthen HandleIconstEPC(5) \circseq pc := 49 \\
    % \t2 {} \circelse pc = 49 \circthen HandleIaddEPC \circseq pc := 50 \\
    % \t2 {} \circelse pc = 50 \circthen HandleAreturnEPC \\
    \t2 \circfi \circseq Poll \circseq Running \\
    \t1 \circfi
  \end{circus}
  \begin{circus}
    TPK\_F \circdef HandleAloadEPC(0) \circseq pc := 44 \circseq Poll \circseq HandleAloadEPC(0) \circseq pc := 45 \circseq \\
    \t1 Poll \circseq HandleIaddEPC \circseq pc := 46 \circseq Poll \circseq HandleAloadEPC(0) \circseq pc := 47 \circseq \\
    \t1 Poll \circseq HandleIaddEPC \circseq pc := 48 \circseq Poll \circseq HandleIconstEPC(5) \circseq pc := 49 \circseq \\
    \t1 Poll \circseq HandleIaddEPC \circseq pc := 50 \circseq Poll \circseq HandleAreturnEPC
  \end{circus}
  \caption{The $Running$ action after method call resolution}
  \label{method-call-resolution-example-figure}
\end{figure}

Calls to those methods can then be resolved, sequencing the method
invocation instruction with a call to the \Circus{} action representing its body and the
instructions following the method call. 
This occurs on line~\ref{algorithm-resolve-method-calls} of the
algorithm, and can be seen in
Figure~\ref{method-call-resolution-example-figure}, which shows our
example after method call resolution has been applied.

The target of each method call can be determined from the parameter to
the method invocation instruction.
This parameter is an index into the constant pool of the current class
that points to a method reference for the method being called.
The correct current class for each bytecode instruction is always
known, since the information on the method entries and ends is
contained in the class information, and there is a one-to-one mapping
between classes and blocks of bytecode instructions that form methods.
After the target of the method call has been determined, the
invocation instruction can be sequenced with a call to the
corresponding \Circus{} action.

An example of this is the occurrence of $HandleInvokestaticEPC(46)$ in
the sequence of actions at $pc = 21$.
As can be seen from Figure~\ref{example-model-figure}, the constant
pool index $46$ corresponds to the method identifier for the method
\texttt{f()} of the \texttt{TPK}.
The sequence of instructions corresponding to this method is in an
action $TPK\_F$, created in the previous step, on
line~\ref{algorithm-separate-complete-methods}.
This action is sequenced with the invocation instruction, with the
$Poll$ action inbetween (to allow thread switches before the first
instruction of the called method).
The instructions following the method call are then sequenced after
it, with another $Poll$ action (to allow thread switches following the
return instruction).
Method call resolution is described in more detail in
Section~\ref{resolve-method-calls-subsection}, where we define the
\Call{SeparateCompleteMethods}{} and \Call{ResolveMethodCalls}{}
procedures.

\begin{figure}[p!]
  \setlength{\zedindent}{0cm}
  \setlength{\zedtab}{0.3cm}
  \setlength{\zedleftsep}{0.1cm}
  \begin{circus}
    Running \circdef \\
    \t1 \circif frameStack = \emptyset \circthen \Skip \\
    \t1 {} \circelse frameStack \neq \emptyset \circthen {} \\
    \t2 {} \circif pc = 0 \circthen TPK\_APEHInit \\
    \t2 {} \cdots {} \\
    \t2 {} \circelse pc = 7 \circthen TPK\_HandleAsyncEvent \\
    % \t2 {} \circelse pc = 8 \circthen HandleDupEPC \circseq pc := 9 \circseq Poll \circseq HandleAconst\_nullEPC \circseq pc := 10 \circseq \\
    % \t3 Poll \circseq HandleInvokespecialEPC(29) \circseq Poll \circseq CCInit \circseq Poll \circseq HandleAstoreEPC(1) \circseq \\
    % \t3 pc := 12 \circseq Poll \circseq HandleAloadEPC(1) \circseq pc := 13 \circseq Poll \circseq \\
    % \t3 HandleInvokevirtualEPC(32) \circseq Poll \circseq OpenInputStream \circseq Poll \circseq HandleAstoreEPC(2) \circseq \\
    % \t3 pc := 15 \circseq Poll \circseq HandleAloadEPC(1) \circseq pc := 16 \circseq Poll \circseq HandleInvokevirtualEPC(36) \circseq Poll \circseq OpenOutputStream \circseq Poll \circseq HandleAstoreEPC(3) \circseq pc := 18 \circseq Poll \circseq HandleIconstEPC(0) \circseq \\
    % \t3 pc := 19 \circseq Poll \circseq HandleAstoreEPC(4) \circseq pc := 20 \circseq Poll \circseq pc := 39 \circseq Poll \circseq \\
    % \t3 HandleAloadEPC(4) \circseq pc := 40 \circseq Poll \circseq HandleIconstEPC(10) \circseq pc := 41 \circseq \\
    % \t3 Poll \circseq \circvar value1, value2 : Word \circspot InterpreterPop2 \circseq \\
    % \t3 pc := \IF value1 \leq value2 \THEN 21 \ELSE 42 \\
    % \t2 {} \circelse pc = 9 \circthen HandleAconst\_nullEPC \circseq pc := 10 \circseq Poll \circseq HandleInvokespecialEPC(29) \circseq Poll \circseq CCInit \circseq Poll \circseq HandleAstoreEPC(1) \circseq pc := 12 \circseq Poll \circseq HandleAloadEPC(1) \circseq pc := 13 \circseq Poll \circseq HandleInvokevirtualEPC(32) \circseq Poll \circseq OpenInputStream \circseq Poll \circseq HandleAstoreEPC(2) \circseq pc := 15 \circseq Poll \circseq HandleAloadEPC(1) \circseq pc := 16 \circseq Poll \circseq HandleInvokevirtualEPC(36) \circseq Poll \circseq OpenOutputStream \circseq Poll \circseq HandleAstoreEPC(3) \circseq pc := 18 \circseq Poll \circseq HandleIconstEPC(0) \circseq \\
    % \t3 pc := 19 \circseq Poll \circseq HandleAstoreEPC(4) \circseq pc := 20 \circseq Poll \circseq pc := 39 \circseq Poll \circseq \\
    % \t3 HandleAloadEPC(4) \circseq pc := 40 \circseq Poll \circseq HandleIconstEPC(10) \circseq pc := 41 \circseq \\
    % \t3 Poll \circseq \circvar value1, value2 : Word \circspot InterpreterPop2 \circseq \\
    % \t3 pc := \IF value1 \leq value2 \THEN 21 \ELSE 42 \\
    % \t2 {} \circelse pc = 10 \circthen HandleInvokespecialEPC(29) \circseq Poll \circseq CCInit \circseq Poll \circseq HandleAstoreEPC(1) \circseq pc := 12 \circseq Poll \circseq HandleAloadEPC(1) \circseq pc := 13 \circseq Poll \circseq HandleInvokevirtualEPC(32) \circseq Poll \circseq OpenInputStream \circseq Poll \circseq HandleAstoreEPC(2) \circseq pc := 15 \circseq Poll \circseq HandleAloadEPC(1) \circseq pc := 16 \circseq Poll \circseq HandleInvokevirtualEPC(36) \circseq Poll \circseq OpenOutputStream \circseq Poll \circseq HandleAstoreEPC(3) \circseq pc := 18 \circseq Poll \circseq HandleIconstEPC(0) \circseq \\
    % \t3 pc := 19 \circseq Poll \circseq HandleAstoreEPC(4) \circseq pc := 20 \circseq Poll \circseq pc := 39 \circseq Poll \circseq \\
    % \t3 HandleAloadEPC(4) \circseq pc := 40 \circseq Poll \circseq HandleIconstEPC(10) \circseq pc := 41 \circseq \\
    % \t3 Poll \circseq \circvar value1, value2 : Word \circspot InterpreterPop2 \circseq \\
    % \t3 pc := \IF value1 \leq value2 \THEN 21 \ELSE 42 \\
    % \t2 {} \circelse pc = 11 \circthen HandleAstoreEPC(1) \circseq pc := 12 \circseq Poll \circseq HandleAloadEPC(1) \circseq \\
    % \t3 pc := 13 \circseq Poll \circseq HandleInvokevirtualEPC(32) \circseq Poll \circseq OpenInputStream \circseq Poll \circseq HandleAstoreEPC(2) \circseq pc := 15 \circseq Poll \circseq HandleAloadEPC(1) \circseq pc := 16 \circseq Poll \circseq HandleInvokevirtualEPC(36) \circseq Poll \circseq OpenOutputStream \circseq Poll \circseq HandleAstoreEPC(3) \circseq pc := 18 \circseq Poll \circseq HandleIconstEPC(0) \circseq \\
    % \t3 pc := 19 \circseq Poll \circseq HandleAstoreEPC(4) \circseq pc := 20 \circseq Poll \circseq pc := 39 \circseq Poll \circseq \\
    % \t3 HandleAloadEPC(4) \circseq pc := 40 \circseq Poll \circseq HandleIconstEPC(10) \circseq pc := 41 \circseq \\
    % \t3 Poll \circseq \circvar value1, value2 : Word \circspot InterpreterPop2 \circseq \\
    % \t3 pc := \IF value1 \leq value2 \THEN 21 \ELSE 42 \\
    % \t2 {} \circelse pc = 12 \circthen HandleAloadEPC(1) \circseq pc := 13 \circseq Poll \circseq HandleInvokevirtualEPC(32) \circseq Poll \circseq OpenInputStream \circseq Poll \circseq HandleAstoreEPC(2) \circseq pc := 15 \circseq Poll \circseq HandleAloadEPC(1) \circseq pc := 16 \circseq Poll \circseq HandleInvokevirtualEPC(36) \circseq Poll \circseq OpenOutputStream \circseq Poll \circseq HandleAstoreEPC(3) \circseq pc := 18 \circseq Poll \circseq HandleIconstEPC(0) \circseq \\
    % \t3 pc := 19 \circseq Poll \circseq HandleAstoreEPC(4) \circseq pc := 20 \circseq Poll \circseq pc := 39 \circseq Poll \circseq \\
    % \t3 HandleAloadEPC(4) \circseq pc := 40 \circseq Poll \circseq HandleIconstEPC(10) \circseq pc := 41 \circseq \\
    % \t3 Poll \circseq \circvar value1, value2 : Word \circspot InterpreterPop2 \circseq \\
    % \t3 pc := \IF value1 \leq value2 \THEN 21 \ELSE 42 \\
    % \t2 {} \circelse pc = 13 \circthen HandleInvokevirtualEPC(32) \circseq Poll \circseq OpenInputStream \circseq Poll \circseq HandleAstoreEPC(2) \circseq pc := 15 \circseq Poll \circseq HandleAloadEPC(1) \circseq pc := 16 \circseq Poll \circseq HandleInvokevirtualEPC(36) \circseq Poll \circseq OpenOutputStream \circseq Poll \circseq HandleAstoreEPC(3) \circseq pc := 18 \circseq Poll \circseq HandleIconstEPC(0) \circseq \\
    % \t3 pc := 19 \circseq Poll \circseq HandleAstoreEPC(4) \circseq pc := 20 \circseq Poll \circseq pc := 39 \circseq Poll \circseq \\
    % \t3 HandleAloadEPC(4) \circseq pc := 40 \circseq Poll \circseq HandleIconstEPC(10) \circseq pc := 41 \circseq \\
    % \t3 Poll \circseq \circvar value1, value2 : Word \circspot InterpreterPop2 \circseq \\
    % \t3 pc := \IF value1 \leq value2 \THEN 21 \ELSE 42 \\
    % \t2 {} \circelse pc = 14 \circthen HandleAstoreEPC(2) \circseq pc := 15 \circseq Poll \circseq HandleAloadEPC(1) \circseq \\
    % \t3 pc := 16 \circseq Poll \circseq HandleInvokevirtualEPC(36) \circseq Poll \circseq OpenOutputStream \circseq Poll \circseq HandleAstoreEPC(3) \circseq pc := 18 \circseq Poll \circseq HandleIconstEPC(0) \circseq \\
    % \t3 pc := 19 \circseq Poll \circseq HandleAstoreEPC(4) \circseq pc := 20 \circseq Poll \circseq pc := 39 \circseq Poll \circseq \\
    % \t3 HandleAloadEPC(4) \circseq pc := 40 \circseq Poll \circseq HandleIconstEPC(10) \circseq pc := 41 \circseq \\
    % \t3 Poll \circseq \circvar value1, value2 : Word \circspot InterpreterPop2 \circseq \\
    % \t3 pc := \IF value1 \leq value2 \THEN 21 \ELSE 42 \\
    % \t2 {} \circelse pc = 15 \circthen HandleAloadEPC(1) \circseq pc := 16 \circseq Poll \circseq HandleInvokevirtualEPC(36) \circseq Poll \circseq OpenOutputStream \circseq Poll \circseq HandleAstoreEPC(3) \circseq pc := 18 \circseq Poll \circseq HandleIconstEPC(0) \circseq \\
    % \t3 pc := 19 \circseq Poll \circseq HandleAstoreEPC(4) \circseq pc := 20 \circseq Poll \circseq pc := 39 \circseq Poll \circseq \\
    % \t3 HandleAloadEPC(4) \circseq pc := 40 \circseq Poll \circseq HandleIconstEPC(10) \circseq pc := 41 \circseq \\
    % \t3 Poll \circseq \circvar value1, value2 : Word \circspot InterpreterPop2 \circseq \\
    % \t3 pc := \IF value1 \leq value2 \THEN 21 \ELSE 42 \\
    % \t2 {} \circelse pc = 16 \circthen HandleInvokevirtualEPC(36) \circseq Poll \circseq OpenOutputStream \circseq Poll \circseq HandleAstoreEPC(3) \circseq pc := 18 \circseq Poll \circseq HandleIconstEPC(0) \circseq \\
    % \t3 pc := 19 \circseq Poll \circseq HandleAstoreEPC(4) \circseq pc := 20 \circseq Poll \circseq pc := 39 \circseq Poll \circseq \\
    % \t3 HandleAloadEPC(4) \circseq pc := 40 \circseq Poll \circseq HandleIconstEPC(10) \circseq pc := 41 \circseq \\
    % \t3 Poll \circseq \circvar value1, value2 : Word \circspot InterpreterPop2 \circseq \\
    % \t3 pc := \IF value1 \leq value2 \THEN 21 \ELSE 42 \\
    % \t2 {} \circelse pc = 17 \circthen HandleAstoreEPC(3) \circseq pc := 18 \circseq Poll \circseq HandleIconstEPC(0) \circseq \\
    % \t3 pc := 19 \circseq Poll \circseq HandleAstoreEPC(4) \circseq pc := 20 \circseq Poll \circseq pc := 39 \circseq Poll \circseq \\
    % \t3 HandleAloadEPC(4) \circseq pc := 40 \circseq Poll \circseq HandleIconstEPC(10) \circseq pc := 41 \circseq \\
    % \t3 Poll \circseq \circvar value1, value2 : Word \circspot InterpreterPop2 \circseq \\
    % \t3 pc := \IF value1 \leq value2 \THEN 21 \ELSE 42 \\
    % \t2 {} \circelse pc = 18 \circthen HandleIconstEPC(0) \circseq pc := 19 \circseq Poll \circseq HandleAstoreEPC(4) \circseq pc := 20 \circseq Poll \circseq pc := 39 \circseq Poll \circseq HandleAloadEPC(4) \circseq pc := 40 \circseq Poll \circseq HandleIconstEPC(10) \circseq pc := 41 \circseq Poll \circseq \circvar value1, value2 : Word \circspot InterpreterPop2 \circseq pc := \IF value1 \leq value2 \THEN 21 \ELSE 42 \\
    % \t2 {} \circelse pc = 19 \circthen HandleAstoreEPC(4) \circseq pc := 20 \circseq Poll \circseq pc := 39 \circseq Poll \circseq HandleAloadEPC(4) \circseq pc := 40 \circseq Poll \circseq HandleIconstEPC(10) \circseq pc := 41 \circseq Poll \circseq \circvar value1, value2 : Word \circspot InterpreterPop2 \circseq pc := \IF value1 \leq value2 \THEN 21 \ELSE 42 \\
    % \t2 {} \circelse pc = 20 \circthen pc := 39 \circseq Poll \circseq HandleAloadEPC(4) \circseq pc := 40 \circseq Poll \circseq HandleIconstEPC(10) \circseq pc := 41 \circseq Poll \circseq \circvar value1, value2 : Word \circspot InterpreterPop2 \circseq pc := \IF value1 \leq value2 \THEN 21 \ELSE 42 \\
    % \t2 {} \circelse pc = 21 \circthen HandleAloadEPC(2) \circseq pc := 22 \circseq Poll \circseq HandleInvokevirtualEPC(40) \circseq Poll \circseq Read \circseq Poll \circseq HandleInvokestaticEPC(46) \circseq Poll \circseq F \circseq Poll \circseq HandleAstoreEPC(5) \circseq pc := 25 \circseq Poll \circseq HandleAloadEPC(5) \circseq pc := 26 \circseq HandleIconstEPC(400) \circseq pc := 27 \circseq Poll \circseq \circvar value1, value2 : Word \circspot InterpreterPop2 \circseq pc := \IF value1 \leq value2 \THEN 32 \ELSE 28 \\
    % \t2 {} \circelse pc = 22 \circthen HandleInvokevirtualEPC(40) \circseq Poll \circseq Read \circseq Poll \circseq HandleInvokestaticEPC(46) \circseq Poll \circseq F \circseq Poll \circseq HandleAstoreEPC(5) \circseq pc := 25 \circseq Poll \circseq HandleAloadEPC(5) \circseq pc := 26 \circseq HandleIconstEPC(400) \circseq pc := 27 \circseq Poll \circseq \circvar value1, value2 : Word \circspot InterpreterPop2 \circseq pc := \IF value1 \leq value2 \THEN 32 \ELSE 28 \\
    % \t2 {} \circelse pc = 23 \circthen HandleInvokestaticEPC(46) \circseq Poll \circseq F \circseq Poll \circseq HandleAstoreEPC(5) \circseq pc := 25 \circseq Poll \circseq HandleAloadEPC(5) \circseq pc := 26 \circseq HandleIconstEPC(400) \circseq pc := 27 \circseq Poll \circseq \circvar value1, value2 : Word \circspot InterpreterPop2 \circseq pc := \IF value1 \leq value2 \THEN 32 \ELSE 28 \\
    % \t2 {} \circelse pc = 24 \circthen HandleAstoreEPC(5) \circseq pc := 25 \circseq Poll \circseq HandleAloadEPC(5) \circseq pc := 26 \circseq HandleIconstEPC(400) \circseq pc := 27 \circseq Poll \circseq \circvar value1, value2 : Word \circspot InterpreterPop2 \circseq pc := \IF value1 \leq value2 \THEN 32 \ELSE 28 \\
    % \t2 {} \circelse pc = 25 \circthen HandleAloadEPC(5) \circseq pc := 26 \circseq Poll \circseq HandleIconstEPC(400) \circseq pc := 27 \circseq Poll \circseq \circvar value1, value2 : Word \circspot InterpreterPop2 \circseq pc := \IF value1 \leq value2 \THEN 32 \ELSE 28 \\
    % \t2 {} \circelse pc = 26 \circthen HandleIconstEPC(400) \circseq pc := 27 \circseq Poll \circseq \circvar value1, value2 : Word \circspot InterpreterPop2 \circseq pc := \IF value1 \leq value2 \THEN 32 \ELSE 28 \\
    % \t2 {} \circelse pc = 27 \circthen \circvar value1, value2 : Word \circspot InterpreterPop2 \circseq \\
    % \t3 pc := \IF value1 \leq value2 \THEN 32 \ELSE 28 \\
    % \t2 {} \circelse pc = 28 \circthen HandleAloadEPC(3) \circseq pc := 29 \circseq Poll \circseq HandleIconstEPC(0) \circseq pc := 30 \circseq Poll \circseq HandleInvokevirtualEPC(50) \circseq Poll \circseq Write \circseq Poll \circseq pc := 38 \circseq Poll \circseq HandleAstoreEPC(4) \circseq pc := 39 \circseq Poll \circseq HandleAloadEPC(4) \circseq pc := 40 \circseq Poll \circseq HandleIconstEPC(10) \circseq pc := 41 \circseq Poll \circseq \circvar value1, value2 : Word \circspot InterpreterPop2 \circseq pc := \IF value1 \leq value2 \THEN 21 \ELSE 42 \\
    % \t2 {} \circelse pc = 29 \circthen HandleIconstEPC(0) \circseq pc := 30 \circseq Poll \circseq HandleInvokevirtualEPC(50) \circseq Poll \circseq Write \circseq Poll \circseq pc := 38 \circseq Poll \circseq HandleAstoreEPC(4) \circseq pc := 39 \circseq Poll \circseq HandleAloadEPC(4) \circseq pc := 40 \circseq Poll \circseq HandleIconstEPC(10) \circseq pc := 41 \circseq Poll \circseq \circvar value1, value2 : Word \circspot InterpreterPop2 \circseq pc := \IF value1 \leq value2 \THEN 21 \ELSE 42 \\
    % \t2 {} \circelse pc = 30 \circthen HandleInvokevirtualEPC(50) \circseq Poll \circseq Write \circseq Poll \circseq pc := 38 \circseq Poll \circseq HandleAstoreEPC(4) \circseq pc := 39 \circseq Poll \circseq HandleAloadEPC(4) \circseq pc := 40 \circseq Poll \circseq HandleIconstEPC(10) \circseq pc := 41 \circseq Poll \circseq \circvar value1, value2 : Word \circspot InterpreterPop2 \circseq pc := \IF value1 \leq value2 \THEN 21 \ELSE 42 \\
    % \t2 {} \circelse pc = 31 \circthen pc := 38 \circseq Poll \circseq HandleAstoreEPC(4) \circseq pc := 39 \circseq Poll \circseq HandleAloadEPC(4) \circseq pc := 40 \circseq Poll \circseq HandleIconstEPC(10) \circseq pc := 41 \circseq Poll \circseq \circvar value1, value2 : Word \circspot InterpreterPop2 \circseq pc := \IF value1 \leq value2 \THEN 21 \ELSE 42 \\
    % \t2 {} \circelse pc = 32 \circthen HandleAloadEPC(3) \circseq pc := 33 \circseq Poll \circseq HandleAloadEPC(5) \circseq pc := 34 \circseq Poll \circseq HandleInvokevirtualEPC(50) \circseq Poll \circseq Write \circseq Poll \circseq HandleAloadEPC(4) \circseq pc := 36 \circseq Poll \circseq HandleIconstEPC(1) \circseq pc := 37 \circseq Poll \circseq HandleIaddEPC \circseq pc := 38 \circseq Poll \circseq HandleAstoreEPC(4) \circseq pc := 39 \circseq Poll \circseq HandleAloadEPC(4) \circseq pc := 40 \circseq Poll \circseq HandleIconstEPC(10) \circseq pc := 41 \circseq Poll \circseq \circvar value1, value2 : Word \circspot InterpreterPop2 \circseq pc := \IF value1 \leq value2 \THEN 21 \ELSE 42 \\
    % \t2 {} \circelse pc = 33 \circthen HandleAloadEPC(5) \circseq pc := 34 \circseq Poll \circseq HandleInvokevirtualEPC(50) \circseq Poll \circseq Write \circseq Poll \circseq HandleAloadEPC(4) \circseq pc := 36 \circseq Poll \circseq HandleIconstEPC(1) \circseq pc := 37 \circseq Poll \circseq HandleIaddEPC \circseq pc := 38 \circseq Poll \circseq HandleAstoreEPC(4) \circseq pc := 39 \circseq Poll \circseq HandleAloadEPC(4) \circseq pc := 40 \circseq Poll \circseq HandleIconstEPC(10) \circseq pc := 41 \circseq Poll \circseq \circvar value1, value2 : Word \circspot InterpreterPop2 \circseq pc := \IF value1 \leq value2 \THEN 21 \ELSE 42 \\
    % \t2 {} \circelse pc = 34 \circthen HandleInvokevirtualEPC(50) \circseq Poll \circseq Write \circseq Poll \circseq HandleAloadEPC(4) \circseq pc := 36 \circseq Poll \circseq HandleIconstEPC(1) \circseq pc := 37 \circseq Poll \circseq HandleIaddEPC \circseq pc := 38 \circseq Poll \circseq HandleAstoreEPC(4) \circseq pc := 39 \circseq Poll \circseq HandleAloadEPC(4) \circseq pc := 40 \circseq Poll \circseq HandleIconstEPC(10) \circseq pc := 41 \circseq Poll \circseq \circvar value1, value2 : Word \circspot InterpreterPop2 \circseq pc := \IF value1 \leq value2 \THEN 21 \ELSE 42 \\
    % \t2 {} \circelse pc = 35 \circthen HandleAloadEPC(4) \circseq pc := 36 \circseq Poll \circseq HandleIconstEPC(1) \circseq pc := 37 \circseq Poll \circseq HandleIaddEPC \circseq pc := 38 \circseq Poll \circseq HandleAstoreEPC(4) \circseq pc := 39 \circseq Poll \circseq HandleAloadEPC(4) \circseq pc := 40 \circseq Poll \circseq HandleIconstEPC(10) \circseq pc := 41 \circseq Poll \circseq \circvar value1, value2 : Word \circspot InterpreterPop2 \circseq pc := \IF value1 \leq value2 \THEN 21 \ELSE 42 \\
    % \t2 {} \circelse pc = 36 \circthen HandleIconstEPC(1) \circseq pc := 37 \circseq Poll \circseq HandleIaddEPC \circseq pc := 38 \circseq Poll \circseq HandleAstoreEPC(4) \circseq pc := 39 \circseq Poll \circseq HandleAloadEPC(4) \circseq pc := 40 \circseq Poll \circseq HandleIconstEPC(10) \circseq pc := 41 \circseq Poll \circseq \circvar value1, value2 : Word \circspot InterpreterPop2 \circseq pc := \IF value1 \leq value2 \THEN 21 \ELSE 42 \\
    % \t2 {} \circelse pc = 37 \circthen HandleIaddEPC \circseq pc := 38 \circseq Poll \circseq HandleAstoreEPC(4) \circseq pc := 39 \circseq Poll \circseq HandleAloadEPC(4) \circseq pc := 40 \circseq Poll \circseq HandleIconstEPC(10) \circseq pc := 41 \circseq Poll \circseq \circvar value1, value2 : Word \circspot InterpreterPop2 \circseq pc := \IF value1 \leq value2 \THEN 21 \ELSE 42 \\
    % \t2 {} \circelse pc = 38 \circthen HandleAstoreEPC(4) \circseq pc := 39 \circseq Poll \circseq HandleAloadEPC(4) \circseq pc := 40 \circseq Poll \circseq HandleIconstEPC(10) \circseq pc := 41 \circseq Poll \circseq \circvar value1, value2 : Word \circspot InterpreterPop2 \circseq pc := \IF value1 \leq value2 \THEN 21 \ELSE 42 \\
    % \t2 {} \circelse pc = 39 \circthen HandleAloadEPC(4) \circseq pc := 40 \circseq Poll \circseq HandleIconstEPC(10) \circseq pc := 41 \circseq Poll \circseq \circvar value1, value2 : Word \circspot InterpreterPop2 \circseq pc := \IF value1 \leq value2 \THEN 21 \ELSE 42 \\
    % \t2 {} \circelse pc = 40 \circthen HandleIconstEPC(10) \circseq pc := 41 \circseq Poll \circseq \circvar value1, value2 : Word \circspot InterpreterPop2 \circseq pc := \IF value1 \leq value2 \THEN 21 \ELSE 42 \\
    % \t2 {} \circelse pc = 41 \circthen \circvar value1, value2 : Word \circspot InterpreterPop2 \circseq \\
    % \t3 pc := \IF value1 \leq value2 \THEN 21 \ELSE 42 \\
    % \t2 {} \circelse pc = 42 \circthen HandleReturnEPC \\
    \t2 {} \cdots {} \\
    \t2 {} \circelse pc = 43 \circthen F \\
    \t2 {} \cdots {} \\
    \t2 \circfi \circseq Poll \circseq Running \\
    \t1 \circfi
  \end{circus}
  \vspace{-1cm}
  \begin{circus}
    TPK\_APEHInit \circdef HandleAloadEPC(0) \circseq pc := 1 \circseq Poll \circseq HandleAloadEPC(1) \circseq pc := 2 \circseq \\
    \t3 Poll \circseq HandleAloadEPC(2) \circseq pc := 3 \circseq Poll \circseq HandleAloadEPC(3) \circseq pc := 4 \circseq \\
    \t3 Poll \circseq HandleAloadEPC(4) \circseq pc := 5 \circseq Poll \circseq HandleInvokespecialEPC(8) \circseq \\
    \t3 Poll \circseq AperiodicEventHandler\_APEHInit \circseq Poll \circseq HandleReturnEPC \\
  \end{circus}
  \vspace{-1.5cm}
  \begin{circus}
    TPK\_HandleAsyncEvent \circdef HandleNewEPC(27) \circseq pc := 8 \circseq Poll \circseq HandleDupEPC \circseq \\
    \t1 {} \cdots {} \\
    % \t3 Poll \circseq HandleAconst\_nullEPC \circseq pc := 10 \circseq Poll \circseq HandleInvokespecialEPC(29) \circseq \\
    % \t3 Poll \circseq ConsoleConnection\_CCInit \circseq Poll \circseq HandleAstoreEPC(1) \circseq pc := 12 \circseq Poll \circseq \\
    % \t3 HandleAloadEPC(1) \circseq pc := 13 \circseq Poll \circseq HandleInvokevirtualEPC(32) \circseq Poll \circseq \\
    % \t3 OpenInputStream \circseq Poll \circseq HandleAstoreEPC(2) \circseq pc := 15 \circseq Poll \circseq \\
    % \t3 HandleAloadEPC(1) \circseq pc := 16 \circseq Poll \circseq HandleInvokevirtualEPC(36) \circseq Poll \circseq \\
    % \t3 OpenOutputStream \circseq Poll \circseq HandleAstoreEPC(3) \circseq pc := 18 \circseq Poll \circseq \\
    % \t3 HandleIconstEPC(0) \circseq pc := 19 \circseq Poll \circseq HandleAstoreEPC(4) \circseq pc := 20 \circseq \\
    \t1 Poll \circseq pc := 39 \circseq Poll \circseq \circmu Y \circspot \\
    \t2 HandleAloadEPC(4) \circseq pc := 40 \circseq Poll \circseq HandleIconstEPC(10) \circseq pc := 41 \circseq \\
    \t2 Poll \circseq \circvar value1, value2 : Word \circspot InterpreterPop2 \circseq \\
    \t2 pc := \IF value1 \leq value2 \THEN 21 \ELSE 42 \circseq Poll \circseq \\
    \t2 \circif value1 \leq value2 \circthen HandleAloadEPC(2) \circseq pc := 22 \circseq Poll \circseq \\
    \t3 {} \cdots {} \\
    \t3 HandleIconstEPC(400) \circseq pc := 27 \circseq Poll \circseq \circvar value1, value2 : Word \circspot \\
    \t3 InterpreterPop2 \circseq pc := \IF value1 \leq value2 \THEN 32 \ELSE 28 \circseq Poll \circseq \\
    \t3 \circif value1 \leq value2 \circthen HandleAloadEPC(3) \circseq pc := 33 \circseq Poll \circseq \\
    \t4 HandleAloadEPC(5) \circseq pc := 34 \circseq Poll \circseq HandleInvokevirtualEPC(50) \circseq \\
    \t4 Poll \circseq ConsoleOutput\_Write \\
    \t3 {} \circelse value1 > value2 \circthen HandleAloadEPC(3) \circseq pc := 29 \circseq Poll \circseq \\
    \t4  HandleIconstEPC(0) \circseq pc := 30 \circseq  Poll \circseq HandleInvokevirtualEPC(50) \circseq \\
    \t4 Poll \circseq ConsoleOutput\_Write \\
    \t3 \circfi \circseq pc := 35 \circseq Poll \circseq  HandleAloadEPC(4) \circseq pc := 36 \circseq Poll \circseq \\
    \t3 HandleIconstEPC(1) \circseq pc := 37 \circseq Poll \circseq HandleIaddEPC \circseq pc := 38 \circseq \\
    \t3 Poll \circseq HandleAstoreEPC(4) \circseq pc := 39 \circseq Poll \circseq Y \\
    \t3 {} \circelse value1 > value2 \circthen  HandleReturnEPC \\
    \t2 \circfi \\
  \end{circus}
  \caption{The $Running$ action after all the methods are separated}
  \label{final-method-separation-example-figure}
\end{figure}

As mentioned previously, these steps are then repeated, in the loop
beginning at line~\ref{algorithm-method-loop} to introduce the loops
and conditionals in methods that would otherwise have unresolved
method calls in the middle of loops and conditionals.
Afterwards, those methods can be separated out and this loop, conditional and
method resolution repeated until every method has been separated out
in this way.

We do not allow recursion to ensure this terminates, since this
requires all the methods called by a given method to be resolved
before the method itself can be resolved.
This is a sensible restriction since recursion is not normally allowed
in safety-critical applications because of the potential for
unpredictable failure due to stack overflow.

The $Running$ action of our example at the end of the loop, when all
loops and conditionals have been introduced, all the methods have been
separated out, and all method calls have been resolved, is shown in
Figure~\ref{final-method-separation-example-figure}.
At this point, the choice over the $pc$ value maps entry points of
methods onto the actions representing those methods, with the other
$pc$ values now redundant.

\begin{figure}[t!]
  \setlength{\zedindent}{0cm}
  \setlength{\zedtab}{0.3cm}
  \setlength{\zedleftsep}{0.1cm}
  \begin{circusaction}
    ExecuteMethod \circdef \circval cid : ClassID; mid : MethodID \circspot \\
    \t1 \circif (cid, mid) = (TPKClassID, APEHInit) \circthen TPK\_APEHInit \\
    \t1 {} \circelse (cid, mid) = (TPKClassID, handleAsyncEvent) \circthen TPK\_HandleAsyncEvent \\
    \t1 {} \circelse (cid, mid) = (TPKClassID, f) \circthen TPK\_F \\
    \t1 {} \cdots {} \\
    \t1 \circfi
  \end{circusaction}
  \begin{circusaction}
    MainThread \circdef \\
    \t1 \circblockbegin
    \circvar cid : ClassID; method : MethodID; methodArgs : \seq Word \circspot \\
    interpreter?c?m?a \then cid, method, methodArgs := c, m, a \circseq  \\
    \lschexpract \exists class? == cs~cid; baseFrame? == \true @ InterpreterNewStackFrame \rschexpract \circseq \\
    ExecuteMethod(cid, method) \circseq MainThread \\
    \circblockend \\
    \t1 {} \extchoice {} \\
    \t1 CEEswitchThread?from?to \prefixcolon (from = main) \then Blocked \circseq MainThread
  \end{circusaction}
  \begin{circusaction}
    NotStarted \circdef \circvar cid : ClassID; method : MethodID; methodArgs : \seq Word \circspot \\
    \t1 CEEstartThread?toStart?bsid?c?m?a \prefixcolon (toStart = thread) \then {} \\
    \t2 cid, method, methodArgs := c, m, a \circseq \\
    \t1 \lschexpract \exists class? == cs~cid; baseFrame? == \true @ InterpreterNewStackFrame \rschexpract \circseq \\
    \t1 Blocked \circseq ExecuteMethod(cid, method) \circseq CEEremoveThread!thread \then NotStarted
  \end{circusaction}
  \caption{The $ExecuteMethod$, $NotStarted$, and $MainThread$ actions after main action refinement}
  \label{refine-main-actions-example-figure}
\end{figure}

\begin{figure}[p!]
  \begin{circus}
    HandleAsyncEvent \circdef HandleNewEPC(27) \circseq Poll \circseq HandleDupEPC \circseq \\
    \t1 Poll \circseq HandleAconst\_nullEPC \circseq Poll \circseq HandleInvokespecialEPC(29) \circseq \\
    \t1 Poll \circseq CCInit \circseq Poll \circseq HandleAstoreEPC(1) \circseq Poll \circseq \\
    \t1 HandleAloadEPC(1) \circseq Poll \circseq HandleInvokevirtualEPC(32) \circseq Poll \circseq \\
    \t1 OpenInputStream \circseq Poll \circseq HandleAstoreEPC(2) \circseq Poll \circseq \\
    \t1 HandleAloadEPC(1) \circseq Poll \circseq HandleInvokevirtualEPC(36) \circseq Poll \circseq \\
    \t1 OpenOutputStream \circseq Poll \circseq HandleAstoreEPC(3) \circseq Poll \circseq \\
    \t1 HandleIconstEPC(0) \circseq Poll \circseq HandleAstoreEPC(4) \circseq \\
    \t1 Poll \circseq Poll \circseq HandleAloadEPC(4) \circseq Poll \circseq \\
    \t1 HandleIconstEPC(10) \circseq Poll \circseq \circvar value1, value2 : Word \circspot \\
    \t1 InterpreterPop2 \circseq Poll \circseq \circmu Y \circspot \\
    \t1 \circif value1 \leq value2 \circthen HandleAloadEPC(2) \circseq Poll \circseq \\
    \t2 HandleInvokevirtualEPC(40) \circseq Poll \circseq Read \circseq Poll \circseq \\
    \t2 HandleInvokestaticEPC(46) \circseq Poll \circseq F \circseq Poll \circseq \\
    \t2 HandleAstoreEPC(5) \circseq Poll \circseq HandleAloadEPC(5) \circseq \\
    \t2 HandleIconstEPC(400) \circseq Poll \circseq \circvar value1, value2 : Word \circspot \\
    \t2 InterpreterPop2 \circseq Poll \circseq \\
    \t2 \circif value1 \leq value2 \circthen HandleAloadEPC(3) \circseq Poll \circseq \\
    \t3 HandleAloadEPC(5) \circseq Poll \circseq HandleInvokevirtualEPC(50) \circseq \\
    \t3 Poll \circseq Write \circseq Poll \circseq HandleAloadEPC(4) \circseq Poll \circseq \\
    \t3 HandleIconstEPC(1) \circseq Poll \circseq HandleIaddEPC \circseq Poll \circseq \\
    \t3 HandleAstoreEPC(4) \circseq Poll \circseq HandleAloadEPC(4) \circseq Poll \circseq \\
    \t3 HandleIconstEPC(10) \circseq Poll \circseq \circvar value1, value2 : Word \circspot \\
    \t3 InterpreterPop2 \circseq Poll \circseq \\
    \t3 \circif value1 \leq value2 \circthen Y \\
    \t3 {} \circelse value1 > value2 \circthen HandleReturnEPC \\
    \t3 \circfi \\
    \t2 {} \circelse value1 > value2 \circthen HandleAloadEPC(3) \circseq Poll \circseq \\
    \t3 HandleIconstEPC(0) \circseq Poll \circseq HandleInvokevirtualEPC(50) \circseq \\
    \t3 Poll \circseq Write \circseq Poll \circseq Poll \circseq HandleAstoreEPC(4) \circseq Poll \circseq \\
    \t3 HandleAloadEPC(4) \circseq Poll \circseq HandleIconstEPC(10) \circseq Poll \circseq \\
    \t3 \circvar value1, value2 : Word \circspot InterpreterPop2 \circseq Poll \circseq \\
    \t3 \circif value1 \leq value2 \circthen Y \\
    \t3 {} \circelse value1 > value2 \circthen  HandleReturnEPC \\
    \t3 \circfi \\
    \t2 \circfi \\
    \t1 {} \circelse value1 > value2 \circthen  HandleReturnEPC \\
    \t1 \circfi \\
  \end{circus}
  \caption{The $HandleAsyncEvent$ action after $pc$ has been eliminated from the state}
  \label{pc-elimination-HandleAsyncEvent-example-figure}
\end{figure}

The next step is then to eliminate these redundant paths and remove
the dependency on $pc$ to select the method action.
This occurs at line~\ref{algorithm-refine-main-actions} of the
algorithm, in which the $NotStarted$ and $MainThread$ actions are
refined to replace the $Running$ action with an $ExecuteMethod$ action
that contains a choice of method action based on the method and class
identifier of the method.
This can be seen in Figure~\ref{refine-main-actions-example-figure},
which shows the $ExecuteMethod$ action corresponding to our example,
and the refined $NotStarted$ and $MainThread$ actions that reference
it.
We describe this refinement in more detail in
Section~\ref{refine-main-actions-subsection}, where we define the
\Call{RefineMainActions}{} procedure.

When all of the previous steps are completed, reliance on $pc$ to
determine control flow has been completely removed.
The $pc$ state component can then be removed in a simple data
refinement that also removes all the assignments to $pc$, resulting in
the $HandleAsyncEvent$ action shown in
Figure~\ref{pc-elimination-HandleAsyncEvent-example-figure}.

The remaining instruction handling actions then only affect the stack,
the removal of which is the concern of the next stage of the
compilation strategy.
The data refinement to remove $pc$ is applied at the end of the
algorithm, on line~\ref{algorithm-remove-pc-from-state}, and is
described in more detail in
Section~\ref{remove-pc-from-state-subsection}, where we define the
\Call{RemovePCFromState}{} procedure.

We now proceed to describe each of the steps of program counter
elimination in more detail.

\FloatBarrier

\subsection{Expand Bytecode}
\label{expand-bytecode-subsection}

Before the control flow can be introduced, the bytecode instructions
provided in the $bc$ parameter to $Thr$ must be expanded to allow
consideration of their semantics.
This is achieved using the procedure shown in
Algorithm~\ref{expand-bytecode-algorithm}.
\begin{algorithm}
  \begin{algorithmic}[1]
    \Procedure{ExpandBytecode}{}
    \State \Call{IntroduceChoiceOverPC}{} \label{algorithm-introduce-choice-over-pc}
    \For{$pc \gets \dom bc$}
    \State \Call{CollapseHandleInstruction}{} \label{algorithm-collapse-handle-instruction}
    \State \Call{ExpandHandleAction}{} \label{algorithm-expand-handle-action}
    \EndFor
    \EndProcedure
  \end{algorithmic}
  \caption{Expand Bytecode}
  \label{expand-bytecode-algorithm}
\end{algorithm}
This begins on line~\ref{algorithm-introduce-choice-over-pc} by
introducing a choice over all the possible values of $pc$ associated
with the $HandleInstruction$ action in $Running$, replacing
$HandleInstruction$ with a choice of the form shown below.
\begin{circus}
  \circif pc = 0 \circthen HandleInstruction \\
  {} \circelse pc = 1 \circthen HandleInstruction \\
  {} \circelse pc = 2 \circthen HandleInstruction \\
  {} \cdots {} \\
  \circfi
\end{circus}
After that, we operate on the occurrence of $HandleInstruction$ at
each $pc$ value, replacing it with its definition, which is a choice
between actions to handle each type of instruction (e.g.\ $HandleDup$,
$HandleAload$ etc.).
For brevity, we refer to these handling actions that define
$HandleInstruction$ as $Handle^*$ actions.
Since the value of $bc$ at a given $pc$ value is known, we can
determine which of the $Handle^*$ actions is chosen for each
occurrence of $HandleInstruction$.
The occurrences of $HandleInstruction$ are therefore collapsed to the
appropriate $Handle^*$ actions on line~\ref{algorithm-collapse-handle-instruction}.
This produces the following choice for our example.
\begin{circus}
  \circif pc = 0 \circthen HandleAload \\
  {} \circelse pc = 1 \circthen HandleAload \\
  {} \circelse pc = 2 \circthen HandleAload \\
  {} \cdots {} \\
  {} \circelse pc = 7 \circthen HandleNew \\
  {} \circelse pc = 8 \circthen HandleDup \\
  {} \circelse pc = 9 \circthen HandleAconst\_null \\
  {} \cdots {} \\
  \circfi
\end{circus}
The $Handle^*$ actions are then replaced with new actions.
Those new actions are not guarded on the value of $bc$ at the
current $pc$ value, since the choice those guards mediate has already
been collapsed.
The parameters of the bytecode instructions are also transferred to
become parameters of the new actions.
Finally, the updates to $pc$ contained in the $Handle^*$ actions are
extracted in the form of assignments to $pc$.

This final transformation is not carried out in the case of the method
invocation and return instructions, where the $pc$ updates are closely
connected to the operations on the stack and require special handling.
The new actions' names are formed by appending $EPC$ to the names of
the $Handle^*$ actions.
The overall mapping from bytecode instructions to the new actions is
shown in Table~\ref{handle-action-table}.
\begin{table}
  \centering
  \begin{tabular}{lp{8.5cm}}
    \hline
    Bytecode ($bc~i$) & Action ($handleAction(bc~i)$) \\
    \hline
    $aconst\_null$ & $HandleAconst\_nullEPC \circseq pc := i+1$ \\
    $dup$ & $HandleDupEPC \circseq pc := i+1$ \\
    $aload~lvi$ & $HandleAloadEPC(lvi) \circseq pc := i+1$ \\
    $astore~lvi$ & $HandleAstoreEPC(lvi) \circseq pc := i+1$ \\
    $iadd$ & $HandleIaddEPC \circseq pc := i+1$ \\
    $iconst~n$ & $HandleIconstEPC(n) \circseq pc := i+1$ \\
    $ineg$ & $HandleInegEPC \circseq pc := i+1$ \\
    $goto~ofst$ & $pc := i+ofst$ \\
    $if\_icmple(ofst)$ & $\circvar value1, value2 : Word \circspot$ \endgraf
                         \t1 $InterpreterPop2 \circseq$ \endgraf
                         \t1 $\IF value1 \leq value2 \THEN i+ofst \ELSE i+1$ \\
    $areturn$ & $HandleAreturnEPC$ \\
    $return$ & $HandleReturnEPC$ \\
    $getfield~cpi$ & $HandleGetfieldEPC(cpi) \circseq pc := i+1$ \\
    $putfield~cpi$ & $HandlePutfieldEPC(cpi) \circseq pc := i+1$ \\
    $getstatic~cpi$ & $HandleGetstaticEPC(cpi) \circseq pc := i+1$ \\
    $putstatic~cpi$ & $HandlePutstaticEPC(cpi) \circseq pc := i+1$ \\
    $invokevirtual~cpi$ & $pc := i \circseq HandleInvokevirtualEPC(cpi)$ \\
    $invokespecial~cpi$ & $pc := i \circseq HandleInvokespecialEPC(cpi)$ \\
    $invokestatic~cpi$ & $pc := i \circseq HandleInvokestaticEPC(cpi)$ \\
    \hline
  \end{tabular}
  \caption{The syntactic function $handleAction$}
  \label{handle-action-table}
\end{table}
Note that the $Handle^*$ actions are eliminated completely in the case
of the \texttt{goto} and \texttt{if\_icmple} instructions, since the
$pc$ update is the main effect of these actions.

This replacement of the $Handle*$ actions with these new actions
occurs in line~\ref{algorithm-expand-handle-action}. 
After this transformation has been applied, the $Running$ action for
our example is as shown in
Figure~\ref{bytecode-expansion-example-figure}.

The overall transformation of the $HandleInstruction$ action in this
step is summarised by Lemma~\ref{bytecode-expansion-lemma}, which
makes use of a syntactic function $handleAction$ that maps bytecode
instructions onto \Circus{} actions as shown in
Table~\ref{handle-action-table}.
\begin{lem}[Bytecode Expansion]
  \label{bytecode-expansion-lemma}
  For a given $bc$
  \begin{circus}
    HandleInstruction_{bc} \circrefines_A
    \begin{array}{l}
      \circif {} \circelse_i pc = i \then handleAction(bc~i) \circfi
    \end{array}
  \end{circus}
  where $handleAction$ is a syntactic function defined by
  Table~\ref{handle-action-table}.
\end{lem}
After the bytecode semantics is expanded in the $Running$ action, the
control flow that corresponds to each $pc$ update can be introduced.

\subsection{Introduce Sequential Composition}
\label{introduce-forward-sequence-subsection}

\begin{algorithm}
  \begin{algorithmic}[1]
    \Procedure{IntroduceSequentialComposition}{}
    \State $cfg \gets$ \Call{MakeControlFlowGraph}{} \label{algorithm-make-control-flow-graph}
    \For{$node \gets cfg$} \label{algorithm-sequence-cfg-loop}
    \While{\Call{HasSimpleSequence}{$node$}} \label{algorithm-forward-sequence-condition}
    \State \Call{ApplyLemma\ref{sequence-introduction-lemma}}{$node$} \label{algorithm-forward-sequence-application}
    \EndWhile
    \EndFor
    \EndProcedure
  \end{algorithmic}
  \caption{Introduce Sequential Composition}
  \label{introduce-forward-sequence-algorithm}
\end{algorithm}
The simplest control flow to introduce is that of instructions where
execution continues at the next program counter value.
These control flows are introduced as shown in
Algorithm~\ref{introduce-forward-sequence-algorithm}.
The algorithm consists of constructing a control flow graph for each
method in the program, as specified on
line~\ref{algorithm-make-control-flow-graph}.
Since the introduction of sequential composition does not depend on
the relationships between methods, the control flow graph is
constructed as a disconnected control flow graph containing the
control flow of all the methods in the program.
Although method calls have not had their $pc$ updates formally made
explicit, we can assume a method call will be followed by the
instruction at the next program counter value.
The control flow graph for our example is shown in
Figure~\ref{example-control-flow-graph-figure}.

\begin{figure}
  \begin{center}
    \footnotesize
    \begin{tikzpicture}[every new ->/.style={-latex}]
      % \node (start) at (0,0) {};
      \foreach \x in {0,...,6}  { \node (\x) at (\x, 1cm) {\x}; }
      \foreach \x in {7,...,20}  { \node (\x) at (0.85*\x - 0.85*7,0) {\x}; }
      \foreach \x in {21,...,27} { \node (\x) at (0.85*\x - 0.85*21, -2cm) {\x}; }
      \foreach \x in {28,...,31} { \node (\x) at (0.85*\x - 0.85*21, -1cm) {\x}; }
      \foreach \x in {32,...,34} { \node (\x) at (0.85*\x - 0.85*25, -3cm) {\x}; }
      \foreach \x in {35,...,42} { \node (\x) at (0.85*\x - 0.85*24.3, -2cm) {\x}; }
      \foreach \x in {43,...,50} { \node (\x) at (-43+\x, -4.5cm) {\x}; }

      \graph{ (7) -> (8) -> (9) -> (10) -> (11) -> (12) ->
        (13) -> (14) -> (15) -> (16) -> (17) -> (18) -> (19) -> (20)
        -> (39) -> (40) -> (41) -> (42); (21) -> (22) -> (23) -> (24)
        -> (25) -> (26) -> (27) -> {
          (28) -> (29) -> (30) -> (31);
          (32) -> (33) -> (34);
        } -> (35) -> (36) -> (37) -> (38) -> (39);
      };

      \graph{(0) -> (1) -> (2) -> (3) -> (4) -> (5) -> (6)};
      \graph{(43) -> (44) -> (45) -> (46) -> (47) -> (48) -> (49) -> (50)};

      \draw[-latex] (41) edge[out=270,in=270,looseness=0.35] (21);
    \end{tikzpicture}
  \end{center}
  \caption{Control flow graph for our example program}
  \label{example-control-flow-graph-figure}
\end{figure}

After the control flow graph is constructed, we consider each node in
turn, as specified by the for loop on
line~\ref{algorithm-sequence-cfg-loop}.
As mentioned earlier, we require a node to have only a single outgoing
edge and its target to have only a single incoming edge in order to be
considered for the introduction of sequential composition.
The reason for this is that nodes with two outgoing edges are points
at which conditionals should be introduced, rather than sequential
compositions.
Such nodes in our example are the nodes for $pc$ values $27$ and $41$,
which represent the start of conditionals.
Likewise, nodes with multiple incoming edges represent points at which
a more complex control flows occur.
For our example, such nodes include $39$, which is the start of a
loop, and $35$, which is the end of a conditional.
These prevent introduction of sequential composition for the $pc$
values $20$, $31$, $34$, and $38$, since those the targets of those
nodes are nodes $35$ and $39$.

For a node that meets the above requirement and isn't a method call,
we can introduce sequential composition at that node by applying
Lemma~\ref{sequence-introduction-lemma}, on line
\ref{algorithm-forward-sequence-application} of the algorithm.
\begin{lem}[Sequence introduction]
  \label{sequence-introduction-lemma}
  \def\zedindent{0.25cm}
  If $i \neq j$ and
  \begin{circus}
    \{frameStack \neq \emptyset\} \circseq A \\
    {} = {} \\
    \{frameStack \neq \emptyset\} \circseq A \circseq \{frameStack \neq \emptyset\}
  \end{circus}
  then,
  \begin{circus}
    \begin{array}{l}
      \circmu X \circspot \\
      \t1 \circif frameStack = \emptyset \circthen \Skip \\
      \t1 {} \circelse frameStack \neq \emptyset \circthen {} \\
      \t2 \circif {} \cdots {} \\
      \t2 {} \circelse pc = i \circthen A \circseq pc := j \\
      \t2 {} \cdots {} \\
      \t2 {} \circelse pc = j \circthen B \\
      \t2 {} \cdots {} \\
      \t2 \circfi \circseq Poll \circseq X \\
      \t1 \circfi
    \end{array}
    \circrefines_A
    \begin{array}{l}
      \circmu X \circspot \\
      \t1 \circif frameStack = \emptyset \circthen \Skip \\
      \t1 {} \circelse frameStack \neq \emptyset \circthen {} \\
      \t2 \circif {} \cdots {} \\
      \t2 {} \circelse pc = i \circthen {} \\
      \t3 A \circseq pc := j \circseq Poll \circseq B \\
      \t2 {} \cdots {} \\
      \t2 {} \circelse pc = j \circthen B \\
      \t2 {} \cdots {} \\
      \t2 \circfi \circseq Poll \circseq X \\
      \t1 \circfi
    \end{array}
  \end{circus}
\end{lem}
This lemma works by unrolling the loop in $Running$ to sequence an
instruction with the instruction that is executed after it, inserting
$Poll$ inbetween.
It is required that the $pc$ value of the node's target, $j$, not be
the same as the $pc$ value of the node, $i$, since that would
introduce a loop, rather than a sequential composition.
Also, the sequence of instructions at the node, $A$, must not affect
the non-emptiness of the $frameStack$ to ensure that the choice at the
start of the main loop in $Running$ can be resolved.

Since Lemma~\ref{sequence-introduction-lemma} pulls two nodes
together, we can continue to introduce sequential composition at a
node after the first application of
Lemma~\ref{sequence-introduction-lemma}, until that node no longer
satisfies the conditions for introducing sequential composition.
This is specified by the while loop at
line~\ref{algorithm-forward-sequence-condition} of the algorithm.
This means the control flow graph is updated as
Lemma~\ref{sequence-introduction-lemma} is applied, to take into
account the merging of nodes.
The resulting control flow graph after introduction of sequential
composition has been performed at every point is shown in
Figure~\ref{example-control-flow-graph-after-sequence-introduction-figure}.
\begin{figure}
  \begin{center}
    \begin{tikzpicture}[every new ->/.style={-latex}]
      \path (0,0) node (0) {0} -- ++(1,0) node (6) {6};
      \path (0,-1cm) node (7) {7} -- ++(1,0) node (11) {11} -- ++(1,0) node (14) {14} -- ++(1,0) node (17) {17};
      \path (0,-3cm) node (21) {21} -- ++(1,0) node (23) {23} -- ++(1,0) node (24) {24};
      \path (3,-2.3cm) node (28) {28} -- ++(1,0) node (31) {31};
      \node at (3,-3.7cm) (32) {32};
      \path (5,-3cm) node (35) {35} -- ++(1,0) node (39) {39} -- ++(1,0) node (42) {42};
      
      \graph{
        (7) -> (11) -> (14) -> (17) -> (39) -> (42);
        (21) -> (23) -> (24) -> {
          (28) -> (31);
          (32);
        } -> (35) -> (39);
      };

      \graph[grow down]{(0) -> (6)};
      \node at (0,-5cm) {43};

      \draw[-latex] (39) edge[out=270,in=270,looseness=0.6] (21);
    \end{tikzpicture}
  \end{center}
  \caption{Control flow graph for our example after sequential composition introduction}
  \label{example-control-flow-graph-after-sequence-introduction-figure}
\end{figure}
The only remaining nodes in this graph are those where the sequence of
instructions ends with a method call or which represent a more complex
control flow.
In particular, the instructions for the \texttt{f()} method of
\texttt{TPK}, which begin at $pc = 43$, have been completely sequenced
together into a single node.
The code which corresponds to this control flow graph is that shown
earlier in Figure~\ref{forward-sequence-introduction-example-figure}

\subsection{Introduce Loops and Conditionals}
\label{introduce-loops-and-conditionals-subsection}

After sequential composition has been introduced for all methods, we
must begin considering each method separately to ensure method calls
are handled properly.
This means the strategy must loop, introducing loops and conditionals
to those methods that have no unresolved method calls and resolving
calls of methods that are then complete, until every method is
complete and has been separated into its own action.
Introducing loops and conditionals is performed as described by
Algorithm~\ref{introduce-loops-and-conditionals-algorithm}.
This considers each method individually, as specified by the for loop
on line~\ref{algorithm-introduce-loops-and-conditionals-method-loop}
of the algorithm. 
The condition on line~\ref{algorithm-no-unresolved-calls-condition}
ensures that only those methods where all method calls have already
been resolved undergo loop and conditional introduction.

\begin{algorithm}
  \begin{algorithmic}[1]
    \Procedure{IntroduceLoopsAndConditionals}{}
    \For{$m \gets methods$}
    \label{algorithm-introduce-loops-and-conditionals-method-loop}
    \If{\Call{HasNoUresolvedCalls}{$m$}}
    \label{algorithm-no-unresolved-calls-condition}
    \State $cfg \gets$
    \Call{MakeControlFlowGraph}{$m$}
    \label{algorithm-make-control-flow-graph2}
    \While{$\lnot$\Call{IsComplete}{$m$}}
    \label{algorithm-loop-until-complete}
    \State $innerLoopBodies \gets$ \Call{IdentifyInnermostLoops}{$cfg$}
    \label{algorithm-identify-innermost-loops}
    \For{$node \gets$ \Call{ReverseNodes}{$innerLoopBodies$}}
    %
    %
    %
    %
    \label{algorithm-node-checking-loop}
    \State \Call{ApplyLemma\ref{conditional-introduction-lemma}}{$node$}
    \label{algorithm-introduce-conditional}
    \If{\Call{HasSimpleSequence}{$node$}}
    \label{algorithm-conditional-introduce-sequence-start}
    \State \Call{ApplyLemma\ref{sequence-introduction-lemma}}{$node$}
    \EndIf \label{algorithm-conditional-introduce-sequence-end}
    \If{\Call{FirstBranchIsEndNode}{$node$}}
    \State \Call{ApplyLemma\ref{sequence-introduction-lemma}}{$node$}
    \EndIf
    \EndFor
    \State \Call{ApplyLemma\ref{loop-introduction-lemma}}{$innerLoops$}
    \label{algorithm-introduce-loop}
    \State \Call{ApplyLemma\ref{conditional-loop-introduction-lemma}}{$innerLoops$}
    \label{algorithm-introduce-conditional-loop}
    \State \Call{Move$Running$LoopInsideConditionals}{}
    \label{algorithm-move-running-loop-inside}
    \State \Call{ApplyLemma\ref{loop-introduction-in-arbitrary-context-lemma}}{$innerLoops$}
    \label{algorithm-introduce-nested-loop}
    \State \Call{ApplyLemma\ref{conditional-loop-introduction-in-arbitrary-context-lemma}}{$innerLoops$}
    \label{algorithm-introduce-nested-conditional-loop}
    \State \Call{Move$Running$LoopOutsideConditionals}{}
    \label{algorithm-move-running-loop-outside}
    \EndWhile
    \EndIf
    \EndFor
    \EndProcedure
  \end{algorithmic}
  \caption{Introduce Loops and Conditionals}
  \label{introduce-loops-and-conditionals-algorithm}
\end{algorithm}

For each method that undergoes loop and conditional introduction, we
must again consider the control flow graph of the method to ensure the
loops and conditionals are introduced in the correct order to properly
form the bodies of loops and conditionals.
This involves constructing a control flow graph for the method, at
line~\ref{algorithm-make-control-flow-graph2}, beginning at the entry
point of the method and following each \texttt{goto} and
\texttt{if\_icmple} instruction until a loop is detected or a
\texttt{return} or \texttt{areturn} instruction is reached.
The graph for the our example, beginning at $pc=7$ (the entry point of
the \texttt{handleAsyncEvent()} method), is shown in
Figure~\ref{example-simplified-control-flow-graph-figure}, alongside
the \Circus{} code obtained at the beginning of this stage for the
method.
\begin{figure}
  \begin{center}
    \begin{multicols}{4}
      \begin{tikzpicture}
        \node (7)  at (0,0)  {7};
        \node (39)  at (0,-1) {39};
        \node (42) at (-1,-2) {42};
        \node (21) at (1,-2) {21};
        \node (28) at (0.5,-3) {28};
        \node (32) at (1.5,-3) {32};
        \node (35) at (1,-4) {35};
        \draw[-latex] (7) to (39);
        \draw[-latex] (39) to (42);
        \draw[-latex] (39) to (21);
        \draw[-latex] (21) to (28);
        \draw[-latex] (21) to (32);
        \draw[-latex] (28) to (35);
        \draw[-latex] (32) to (35);
        % \draw[-latex,red!70!black,dashed,out=0,in=0,looseness=1.1] (35) to (39);
        \draw[-latex,dashed,out=0,in=0,looseness=1.1] (35) to (39);
      \end{tikzpicture}
      \columnbreak
      \scriptsize
      \setlength{\zedindent}{0cm}
      \begin{circus}
        Running \circdef \\
        \t1 \circif frameStack = \emptyset \circthen \Skip \\
        \t1 {} \circelse frameStack \neq \emptyset \circthen {} \\
        \t2 \circif pc = 0 \circthen {} \cdots {} \\
        \t2 {} \circelse pc = 7 \circthen HandleNewEPC(27) \circseq pc := 8 \circseq Poll \circseq \cdots \circseq pc := 39 \\
        \t2 {} \cdots {} \\
        \t2 {} \circelse pc = 21 \circthen HandleAloadEPC(2) \circseq pc := 22 \circseq Poll \circseq \cdots \circseq \\
        \t3 pc := \IF value1 \leq value2 \THEN 32 \ELSE 28 \\
        \t2 {} \cdots {} \\
        \t2 {} \circelse pc = 28 \circthen HandleAloadEPC(3) \circseq pc := 29 \circseq Poll \circseq \cdots \circseq pc := 35 \\
        \t2 {} \cdots {} \\
        \t2 {} \circelse pc = 32 \circthen HandleAloadEPC(3) \circseq pc := 33 \circseq Poll \circseq \cdots \circseq pc := 35 \\
        \t2 {} \cdots {} \\
        \t2 {} \circelse pc = 35 \circthen HandleAloadEPC(4) \circseq pc := 36 \circseq Poll \circseq \cdots \circseq pc := 39 \\
        \t2 {} \cdots {} \\
        \t2 {} \circelse pc = 39 \circthen HandleAloadEPC(4) \circseq pc := 36 \circseq Poll \circseq \cdots \circseq \\
        \t3 pc := \IF value1 \leq value2 \THEN 21 \ELSE 42 \\
        \t2 {} \cdots {} \\
        \t2 {} \circelse pc = 42 \circthen HandleReturnEPC \\
        \t2 \circfi \circseq Poll \circseq Running \\
        \t1 \circfi
      \end{circus}
    \end{multicols}
  \end{center}
  \caption{Simplified control flow graph and corresponding code for our example
    program}
  \label{example-simplified-control-flow-graph-figure}
\end{figure}

The loops must be introduced beginning with the innermost loop and
working outwards, with the constructs in the body of each loop
introduced before the loop itself is introduced.
We identify the innermost loops as specified in
Algorithm~\ref{identify-innermost-loops-algorithm}, which defines the
\Call{IdentifyInnermostLoops}{} function used to identify the
innermost loops on line~\ref{algorithm-identify-innermost-loops} of
Algorithm~\ref{introduce-loops-and-conditionals-algorithm}.
\begin{algorithm}[t]
  \begin{algorithmic}[1]
    \Procedure{IdentifyInnermostLoops}{$cfg$}
    % Find looping edges
    \State $loopingEdges \gets$ \Call{FindLoopingEdges}{$cfg$} \label{algorithm-find-looping-edges}
    % Merge loops with the same start
    \State $groupedLoopEdges \gets$ \Call{GroupEdgesWithSameTarget}{$loopingEdges$} \label{algorithm-group-loop-edges}
    % Check if there are no loops
    \If{$groupedloopEdges = \{\}$} \label{algorithm-check-for-no-loops}
    \State \textbf{return} \Call{Nodes}{$cfg$}
    \EndIf
    % Get sets of nodes in the loop body
    \State $loopBodies \gets \{\}$
    \For{$edgeGroup \gets groupedLoopEdges$} \label{algorithm-loop-group-loop}
    \State $loopBody \gets \{\}$
    \For{$edge \gets edgeGroup$} \label{algorithm-loop-body-loop}
    \State $loopBody \gets loopBody \cup {}$\Call{NodesBetweenTargetAndSource}{$edge$} \label{algorithm-collect-loop-nodes}
    \EndFor
    \State $loopBodies \gets loopBodies \cup \{loopBody\}$
    \EndFor
    % Find loop bodies that are disjoint or contained in all other loop bodies
    \State $innermostLoops \gets \{\}$
    \For{$loopBody \gets loopBodies$} \label{algorithm-loop-body-check-loop}
    \If{$\forall lb2 : loopBodies \circspot loopBody \cap lb2 = \{\} \lor loopBody \subseteq lb2$} \label{algorithm-loop-body-check}
    \State $innermostLoops \gets innermostLoops \cup \{loopBody\}$ \label{algorithm-add-innermost-loop}
    \EndIf
    \EndFor
    \If{$innermostLoops = \{\}$} \label{algorithm-empty-innermost-check}
    \State \textbf{fail} \label{algorithm-abort-on-no-innermost}
    \EndIf
    \State \textbf{return} $innermostLoops$ \label{algorithm-return-innermost-loops}
    \EndProcedure
  \end{algorithmic}
  \caption{Identify Innermost Loops}
  \label{identify-innermost-loops-algorithm}
\end{algorithm}

The finding of innermost loops begins by identifying the looping edges
in the control flow graph, on line~\ref{algorithm-find-looping-edges}.
Looping edges are easily identified as those that return to an earlier
node in the control flow graph.
However, a loop may have multiple points at which it loops.
Such loops can be identified by the fact that all the looping edges
have the same target, which is the first node in the loop.
An example of such a loop can be seen in the control flow graph below,
where the thick edges are the looping edges.
\begin{center}
\begin{tikzpicture}
  \node (1) at (0, 0.0) {$\bullet$};
  \node (2) at (2, 0.5) {$\bullet$};
  \node (3) at (2,-0.5) {$\bullet$};
  \node (4) at (4, 0.5) {$\cdots$};
  \node (5) at (4,-0.5) {$\cdots$};
  \node (6) at (6, 0.5) {$\bullet$};
  \node (7) at (6,-0.5) {$\bullet$};

  \draw[-latex] (-1,0) to (1);
  
  \draw[-latex] (1) to (2);
  \draw[-latex] (1) to (3);
  \draw         (2) to (4);
  \draw         (3) to (5);
  \draw[-latex] (4) to (6);
  \draw[-latex] (5) to (7);
  
  \draw[-latex, very thick, out=90, in=90, looseness=0.3] (6) to (1);
  \draw[-latex, very thick, out=270, in=270, looseness=0.3] (7) to (1);
\end{tikzpicture}
\end{center}
To ensure such loops are considered as a single loop (rather than a
loop per looping edge), we group together the looping edges that have
the same target.
This occurs on line~\ref{algorithm-group-loop-edges} of the algorithm.

There may be no loops in a method. 
In such cases we want to work over all the remaining nodes in the
method, treating them as if they were the body of a loop.
This is handled in the check on
line~\ref{algorithm-check-for-no-loops}.

We then obtain the nodes in the bodies of each of the loops.
This is done for each of the groups of looping edges, in the loop on
line~\ref{algorithm-loop-group-loop}.
For each of the looping edges in those groups, we collect the nodes
between the target and source of the edge.
That is, we take all the nodes in paths beginning at the target of the
looping edge and ending at the source node of the looping edge.
This occurs on line~\ref{algorithm-collect-loop-nodes} of the
algorithm, inside a loop over the looping edges on
line~\ref{algorithm-loop-body-loop}.

After all the edges in the loops are collected, we compare the loops
to find which are the innermost.
Each of the loop bodies is considered, in the loop on
line~\ref{algorithm-loop-body-check-loop}.
In order to be an innermost loop, a loop must either be contained in
or disjoint from every other loop.
This is checked in the conditional on
line~\ref{algorithm-loop-body-check}, and an innermost loop is added
to the set of innermost loops on
line~\ref{algorithm-add-innermost-loop}.

It is possible that this algorithm will generate an empty set of
innermost loops, if the input program contains overlapping loops.
Below is an example of the control flow graph of such a program.
\begin{center}
\begin{tikzpicture}
  \node (1) at (00, 0.0) {$\bullet$};
  \node (2) at (02, 0.0) {$\bullet$};
  \node (3) at (04, 0.0) {$\circ$};
  \node (4) at (06, 0.0) {$\circ$};
  \node (5) at (08, 0.0) {$\bullet$};

  \draw[-latex] (-1,0) to (1);
  
  \draw[-latex] (1) to (2);
  \draw[-latex] (2) to (3);
  \draw[-latex] (3) to (4);
  \draw[-latex] (4) to (5);
  
  \draw[-latex, very thick, out=90, in=90, looseness=0.4] (4) to (2);
  \draw[-latex, very thick, out=270, in=270, looseness=0.4] (5) to (3);
\end{tikzpicture}
\end{center}
In this graph, the nodes marked by white circles are counted as being
in both loops, but the loops are not completely contained within each
other. 
This means that the loops cannot be transformed into a well-structured
C loop.
Since such loops would not be generated in the bytecode resulting from
an SCJ program (though they are valid in Java class files in general),
we do not allow such loops in input to our strategy.
This is consistent with the restrictions imposed by MISRA-C.
So we must check if an empty set of innermost loops is generated, and
abort the strategy if it is.
This is specified in the conditional on
line~\ref{algorithm-empty-innermost-check} of the algorithm, where we
abort on line~\ref{algorithm-abort-on-no-innermost} if the set of
innermost loops is empty.

If there is a non-empty set of innermost loops, then that is returned
as the sets of nodes that are considered first in
Algorithm~\ref{introduce-loops-and-conditionals-algorithm}.
In our example there is only a single looping edge, from the $pc = 35$
node to the $pc = 39$ node, so the nodes $39$, $21$, $28$, $32$, and
$35$ are considered first as the interior of the loop formed by that
edge.
The remaining nodes in the method's control flow graph are considered
after the loop's nodes have been considered and the loop has been
introduced.

In general we iteratively work outwards from the innermost loop until
the method being considered is complete.
This is indicated by the loop beginning at
line~\ref{algorithm-loop-until-complete} of
Algorithm~\ref{introduce-loops-and-conditionals-algorithm}.
Within each loop, we introduce control flow constructs to collect the
body of the loop into a single node.
This is performed working backwards from the last node within the body
of a loop, as specified by the for loop on
line~\ref{algorithm-node-checking-loop} of the algorithm, to ensure
that the innermost nested conditionals are introduced first.

An example where this handling of innermost conditionals is necessary
is shown below.
\begin{center}
\begin{tikzpicture}
  \node (1) at (0, 0.0) {$\bullet$};
  \node (2) at (2, 0.5) {$\bullet$};
  \node (3) at (2,-0.5) {$\bullet$};
  \node (4) at (4, 1.0) {$\bullet$};
  \node (5) at (4,-1.0) {$\bullet$};
  \node (6) at (4, 0.1) {$\bullet$};
  \node (7) at (4,-0.1) {$\bullet$};
  \node (8) at (6, 0.0) {$\bullet$};

  \draw[-latex] (-1,0) to (1);
  
  \draw[-latex] (1) to (2);
  \draw[-latex] (1) to (3);
  \draw[-latex] (2) to (4);
  \draw[-latex] (3) to (5);
  \draw[-latex] (2) to (6);
  \draw[-latex] (3) to (7);
  \draw[-latex] (4) to (8);
  \draw[-latex] (5) to (8);
  \draw[-latex] (6) to (8);
  \draw[-latex] (7) to (8);
\end{tikzpicture}
\end{center}
This example contains two levels of nested conditionals, which then
end at the final node in the graph.
If the introduction of the conditionals were to start at the first
node, then the second level of conditionals would have to be
introduced from within the branches of the conditional.
While this is possible (and, as we shall see later, it is the method
used for introduction of loops in some cases), it complicates the
strategy and can easily be avoided by starting the introduction of
conditionals at the end of the graph.
Then the innermost conditionals are introduced first and are absorbed
into the body of the outer conditional, without the need to introduce
the conditionals within its body.


Within the body of a loop, conditionals can be identified as nodes
whose sequence of instructions ends with an assignment of the form
$pc := \IF b \THEN x \ELSE y$ and which have two target nodes, neither
of which are earlier in the method (so as to distinguish them from
loops).
The conditionals come in several different kinds, depending on whether
each branch of the conditional ends with a loop, a return or a
sequential composition.
The first kind of conditional we consider is one in which both
branches end with a sequential composition, such that the flow of
execution continues after the end of the conditional.
Such conditionals can be identified by the fact that both branches of
the conditional end in an assignment of the same value to the $pc$,
and they have control flow graphs of the following form.
\begin{center}
  \begin{tikzpicture}
    \useasboundingbox (-2,-0.5) rectangle (6,0.5);
    \node (0) at (-2,  0) {$\cdots$};
    \node (1) at (0, 0.0) {$\bullet$};
    \node (2) at (2, 0.5) {$\bullet$};
    \node (3) at (2,-0.5) {$\bullet$};
    \node (4) at (4, 0.0) {$\bullet$};
    \node (5) at (6, 0.0) {$\cdots$};

    \draw[-latex] (0) to (1);
    \draw[-latex] (1) to (2);
    \draw[-latex] (1) to (3);
    \draw[-latex] (2) to (4);
    \draw[-latex] (3) to (4);
    \draw         (4) to (5);
  \end{tikzpicture}
\end{center}
Such conditionals are introduced using
Lemma~\ref{conditional-introduction-lemma}, on
line~\ref{algorithm-introduce-conditional} of the algorithm.
This lemma introduces a conditional and distributes the $pc$
assignment at the end of both branches outside the conditional.
As in Lemma~\ref{sequence-introduction-lemma}, the sequence of actions
for the node must not affect the nonemptiness of the $frameStack$.
A similar condition is required for all the lemmas in this section.
We also require that the targets of the conditional are different from
the node at which the conditional is introduced, since that would
introduce a loop, which is not the purpose of this lemma.
\begin{lem}[Conditional introduction]
  \label{conditional-introduction-lemma}
  \setlength{\zedindent}{0.25cm}
  % \setlength{\abovedisplayskip}{0.1cm}
  % \setlength{\belowdisplayskip}{0.1cm}
  If $i \neq j$, $i \neq k$, and 
  \begin{circus}
    \{frameStack \neq \emptyset\} \circseq A \\
    {} = {} \\
    \{frameStack \neq \emptyset\} \circseq A \circseq \{frameStack \neq \emptyset\}
  \end{circus}
  then
  \begin{circus}
    \begin{array}{l}
      \circmu X \circspot \\
      \t1 \circif frameStack = \emptyset \circthen \Skip \\
      \t1 {} \circelse frameStack \neq \emptyset \circthen {} \\
      \t2 \circif \cdots \\
      \t2 {} \circelse pc = i \circthen A \circseq \\
      \t3 pc := \IF b \THEN j \ELSE k \\
      \t2 {} \cdots {} \\
      \t2 {} \circelse pc = j \circthen B \circseq pc := x \\
      \t2 {} \cdots {} \\
      \t2 {} \circelse pc = k \circthen C \circseq pc := x \\
      \t2 {} \cdots {} \\
      \t2 \circfi \circseq Poll \circseq X \\
      \t1 \circfi
    \end{array}
    \circrefines_A
    \begin{array}{l}
      \circmu X \circspot \\
      \t1 \circif frameStack = \emptyset \circthen \Skip \\
      \t1 {} \circelse frameStack \neq \emptyset \circthen {} \\
      \t2 \circif \cdots \\
      \t2 {} \circelse pc = i \circthen A \circseq Poll \circseq \\
      \t3 pc := \IF b \THEN j \ELSE k \circseq \\
      \t3 \circif b \circthen B \\
      \t3 {} \circelse \lnot b \circthen C \\
      \t3 \circfi \circseq pc := x \\
      \t2 {} \cdots {} \\
      \t2 {} \circelse pc = j \circthen B \circseq pc := x \\
      \t2 {} \cdots {} \\
      \t2 {} \circelse pc = k \circthen C \circseq pc := x \\
      \t2 {} \cdots {} \\
      \t2 \circfi \circseq Poll \circseq X \\
      \t1 \circfi 
    \end{array}
  \end{circus}
\end{lem}
After applying Lemma~\ref{conditional-introduction-lemma}, we may be
able to introduce sequential composition at a node that could not
previously be introduced.
Such sequential compositions could not be introduced earlier as the
node following the conditional has two incoming edges.
This is performed in the same manner as in
Section~\ref{introduce-forward-sequence-subsection}, checking for a
simple sequence and applying Lemma~\ref{sequence-introduction-lemma},
on lines~\ref{algorithm-conditional-introduce-sequence-start}
to~\ref{algorithm-conditional-introduce-sequence-end} of the
algorithm.
The check for simple sequence is still necessary as we must wait until
the outer conditionals have been introduced in the case of nested
conditionals.

The next kinds of conditional we consider are those in which only one
branch ends in a sequential composition with the instructions after
the conditional.
Such conditionals will only occur in the presence of other
conditionals or loops, as in the control flow graph below.
The node at the end of the graph below cannot be sequenced with the
end of the conditionals, since there are two edges leading to the
node.
If there were only a single conditional, then only one edge would lead
to the node after the conditional and sequential composition would
have been introduced in the previous section.
\begin{center}
  \begin{tikzpicture}
    \node (0) at (-2, 0.0) {$\cdots$};
    \node (1) at (0, 0.0) {$\bullet$};
    \node (2) at (2, 0.5) {$\bullet$};
    \node (3) at (2,-0.5) {$\bullet$};
    \node (4) at (4, 1.0) {$\bullet$};
    \node (5) at (4,-1.0) {$\bullet$};
    \node (6) at (4, 0.3) {$\bullet$};
    \node (7) at (4,-0.3) {$\bullet$};
    \node (8) at (6, 0.0) {$\bullet$};
    \node (9) at (8, 0.0) {$\cdots$};

    \draw[-latex] (0) to (1);
    \draw[-latex] (1) to (2);
    \draw[-latex] (1) to (3);
    \draw[-latex] (2) to (4);
    \draw[-latex] (3) to (5);
    \draw[-latex] (2) to (6);
    \draw[-latex] (3) to (7);
    \draw[-latex] (6) to (8);
    \draw[-latex] (7) to (8);
    \draw         (8) to (9);
  \end{tikzpicture}
\end{center}
The other branch of the conditional (that does not end in a sequential
composition), may end in a loop, a return, or another conditional.
If the branch ends in a return, then that means the return does not
occur at the end of the method.
Since we target C code that has only a single return at the end of a
function, we cannot translate such return instructions directly.
We must instead sequence the branch of the conditional without the
return with the instructions after the conditional.
In the case of the control flow graph above, this means that it will
be treated as if it had the form of the control flow graph below,
effectively duplicating the node after the conditional.
\begin{center}
  \begin{tikzpicture}
    \node (A) at (-2, 0.0) {$\cdots$};
    \node (B) at (0, 0.0) {$\bullet$};
    \node (C) at (2, 0.5) {$\bullet$};
    \node (D) at (2,-0.5) {$\bullet$};
    \node (E) at (4, 1.0) {$\bullet$};
    \node (F) at (4,-1.0) {$\bullet$};
    \node (G) at (4, 0.3) {$\bullet$};
    \node (H) at (4,-0.3) {$\bullet$};
    \node (I) at (6, 0.3) {$\bullet$};
    \node (J) at (8, 0.3) {$\cdots$};
    \node (K) at (6,-0.3) {$\bullet$};
    \node (L) at (8,-0.3) {$\cdots$};
    

    \draw[-latex] (A) to (B);
    \draw[-latex] (B) to (C);
    \draw[-latex] (B) to (D);
    \draw[-latex] (C) to (E);
    \draw[-latex] (D) to (F);
    \draw[-latex] (C) to (G);
    \draw[-latex] (D) to (H);
    \draw[-latex] (G) to (I);
    \draw[-latex] (H) to (K);
    \draw         (I) to (J);
    \draw         (K) to (L);
  \end{tikzpicture}
\end{center}
This pushes the conditional to the end of the method, slightly
breaking the form of the method in the original Java code but bringing
it form required for the C code.
Thus such conditionals are introduced using
Lemma~\ref{seq-ret-conditional-introduction-lemma}, which introduces
the sequential composition along with the conditional.
The preconditions for this lemma are the same as those for
Lemma~\ref{conditional-introduction-lemma}, but the second branch of
the conditional is required to end with an assignment to the $pc$.
% TODO: are conditions on x needed? e.g. x \neq i
\begin{lem}[Sequential composition--conditional introduction 1]
  \label{seq-ret-conditional-introduction-lemma}
  \setlength{\zedindent}{0.25cm}
  % \setlength{\abovedisplayskip}{0.1cm}
  % \setlength{\belowdisplayskip}{0.1cm}
  If $i \neq j$, $i \neq k$,  and 
  \begin{circus}
    \{frameStack \neq \emptyset\} \circseq A \\
    {} = {} \\
    \{frameStack \neq \emptyset\} \circseq A \circseq \{frameStack \neq \emptyset\}
  \end{circus}
  then
  \begin{circus}
    \begin{array}{l}
      \circmu X \circspot \\
      \t1 \circif frameStack = \emptyset \circthen \Skip \\
      \t1 {} \circelse frameStack \neq \emptyset \circthen {} \\
      \t2 \circif \cdots \\
      \t2 {} \circelse pc = i \circthen A \circseq \\
      \t3 pc := \IF b \THEN j \ELSE k \\
      \t2 {} \cdots {} \\
      \t2 {} \circelse pc = j \circthen B \\
      \t2 {} \cdots {} \\
      \t2 {} \circelse pc = k \circthen C \circseq pc := x \\
      \t2 {} \cdots {} \\
      \t2 {} \circelse pc = x \circthen D \\
      \t2 {} \cdots {} \\
      \t2 \circfi \circseq Poll \circseq X \\
      \t1 \circfi
    \end{array}
    \circrefines_A
    \begin{array}{l}
      \circmu X \circspot \\
      \t1 \circif frameStack = \emptyset \circthen \Skip \\
      \t1 {} \circelse frameStack \neq \emptyset \circthen {} \\
      \t2 \circif \cdots \\
      \t2 {} \circelse pc = i \circthen A \circseq Poll \circseq \\
      \t3 pc := \IF b \THEN j \ELSE k \circseq \\
      \t3 \circif b \circthen B \\
      \t3 {} \circelse \lnot b \circthen {} \\
      \t4 C \circseq pc := x \circseq Poll \circseq D \\
      \t3 \circfi \\
      \t2 {} \cdots {} \\
      \t2 {} \circelse pc = j \circthen B \\
      \t2 {} \cdots {} \\
      \t2 {} \circelse pc = k \circthen C \circseq pc := x \\
      \t2 {} \cdots {} \\
      \t2 {} \circelse pc = x \circthen D \\
      \t2 {} \cdots {} \\
      \t2 \circfi \circseq Poll \circseq X \\
      \t1 \circfi 
    \end{array}
  \end{circus}
\end{lem}
This lemma is applied on line~\ref{}

As mentioned above, we then repeat these steps, working backward
through the control flow graph for all the nodes in the loop being
considered.
In our example this means that we next introduce a conditional at the
$pc = 39$ node, bringing the body of the loop into a single node and
yielding the code below.
Note that the $pc = 42$ node is also included in the introduced
conditional, along with the body of the loop.
{
  \setlength{\zedindent}{0.7cm}
\begin{circus}
  Running \circdef \\
  \t1 \circif frameStack = \emptyset \circthen \Skip \\
  \t1 {} \circelse frameStack \neq \emptyset \circthen {} \\
  \t2 \circif pc = 0 \circthen {} \cdots {} \\
  \t2 {} \cdots {} \\
  \t2 {} \circelse pc = 39 \circthen HandleAloadEPC(4) \circseq pc := 40 \circseq Poll \circseq \cdots \circseq \\
  \t3 pc := \IF value1 \leq value2 \THEN 21 \ELSE 42 \circseq Poll \circseq \\
  \t3 \circif value1 \leq value2 \circthen HandleAloadEPC(2) \circseq pc := 22 \circseq Poll \circseq \cdots \circseq \\
  \t4 pc := \IF value1 \leq value2 \THEN 32 \ELSE 28 \circseq Poll \circseq \\
  \t4 \circif value1 \leq value2 \circthen HandleAloadEPC(3) \circseq pc := 33 \circseq \cdots \\
  \t4 {} \circelse value1 > value2 \circthen HandleAloadEPC(3) \circseq pc := 29 \circseq \cdots \\
  \t4 \circfi \circseq pc := 35 \circseq Poll \circseq HandleAloadEPC(4) \circseq \cdots \circseq pc := 39 \\
  \t3 {} \circelse value1 > value2 \circthen HandleReturnEPC \\
  \t3 \circfi \\
  \t2 {} \cdots {} \\
  \t2 \circfi \circseq Poll \circseq Running \\
  \t1 \circfi
\end{circus}
}

After the body of the loop has been reduced to a single node, we can
introduce the loop at that node.
If the loop is caused by a simple assignment of the form $pc := x$
then the loop introduction is performed using
Lemma~\ref{loop-introduction-lemma}, which introduces a recursion to
the beginning of the node that constitutes the body of the loop.
This occurs at line~\ref{algorithm-introduce-loop} of the algorithm.
\begin{lem}[Loop introduction]
  \label{loop-introduction-lemma}
  If
  \begin{circus}
    \{frameStack \neq \emptyset\} \circseq A \\
    {} = {} \\
    \{frameStack \neq \emptyset\} \circseq A \circseq \{frameStack \neq \emptyset\}
  \end{circus}
  then
  \def\zedindent{0.25cm}
  \begin{circus}
    \begin{array}{l}
      \circmu X \circspot \\
      \t1 \circif frameStack = \emptyset \circthen \Skip \\
      \t1 {} \circelse frameStack \neq \emptyset \circthen {} \\
      \t2 \circif \cdots \\
      \t2 {} \circelse pc = i \circthen {} \\
      \t3 A \circseq pc := i \\
      \t2 {} \cdots {} \\
      \t2 \circfi \circseq Poll \circseq X \\
      \t1 \circfi
    \end{array}
    \circrefines_A
    \begin{array}{l}
      \circmu X \circspot \\
      \t1 \circif frameStack = \emptyset \circthen \Skip \\
      \t1 {} \circelse frameStack \neq \emptyset \circthen {} \\
      \t2 \circif \cdots \\
      \t2 {} \circelse pc = i \circthen {} \\
      \t3 \circmu Y \circspot A \circseq pc := i \circseq Poll \circseq Y \\
      \t2 {} \cdots {} \\
      \t2 \circfi \circseq Poll \circseq X \\
      \t1 \circfi
    \end{array}
  \end{circus}
\end{lem}%

The other type of loop to be introduced is a loop that is caused by an
assignment of the form $pc := \IF b \THEN x \ELSE y$, which we refer
to as a conditional loop.
Note that the loop body will start at $pc = x$, since the conditional
instructions only branch if the condition is met, going to the next
instruction if it is not, so only the $pc = x$ branch can result in a
loop.
To introduce conditional loops, we use
Lemma~\ref{conditional-loop-introduction-lemma}, at
line~\ref{algorithm-introduce-conditional-loop} of the algorithm.
This lemma requires that the $pc$ value of the second branch of the
conditional must not be the same as that of the first branch (which is
the loop body), since that is impossible for Java bytecode
instructions, as discussed above.
\begin{lem}[Conditional loop introduction]
  \label{conditional-loop-introduction-lemma}
  \def\zedindent{0.25cm}
  If $i \neq j$,
  \begin{circus}
    \{frameStack \neq \emptyset\} \circseq A \\
    {} = {} \\
    \{frameStack \neq \emptyset\} \circseq A \circseq \{frameStack \neq \emptyset\}
  \end{circus}
  then
  \begin{circus}
    \begin{array}{l}
      \circmu X \circspot \\
      \t1 \circif frameStack = \emptyset \circthen \Skip \\
      \t1 {} \circelse frameStack \neq \emptyset \circthen {} \\
      \t2 \circif \cdots \\
      \t2 {} \circelse pc = i \circthen A \circseq \\
      \t3 pc := \IF b \THEN i \ELSE j \\
      \t2 \cdots \\
      \t2 {} \circelse pc = j \circthen B \\
      \t2 \cdots \\
      \t2 \circfi \circseq Poll \circseq X \\
      \t1 \circfi 
    \end{array}
    \circrefines_A
    \begin{array}{l}
      \circmu X \circspot \\
      \t1 \circif frameStack = \emptyset \circthen \Skip \\
      \t1 {} \circelse frameStack \neq \emptyset \circthen {} \\
      \t2 \circif \cdots \\
      \t2 {} \circelse pc = i \circthen \circmu Y \circspot A \circseq Poll \circseq \\
      \t3 \circif b \circthen Y \\
      \t3 {} \circelse \lnot b \circthen B \\
      \t3 \circfi \\
      \t2 \cdots \\
      \t2 {} \circelse pc = j \circthen B \\
      \t2 \cdots \\
      \t2 \circfi \circseq Poll \circseq X \\
      \t1 \circfi 
    \end{array}
  \end{circus}
\end{lem}%

In some cases the $pc$ update that causes a loop may be in a branch of
a conditional.
Such a case occurs our example, where the loop beginning at $pc = 39$
loops at the end of the first branch of a conditional.
The introduction of the conditional before the loop is introduced is
necessary to bring the body of the loop into a single node, so we must
handle the loop inside the conditional.
This is performed by first distributing the $Poll \circseq Running$ at
the end of the $Running$ action into the branches of the conditional,
on line~\ref{algorithm-move-running-loop-inside} of the algorithm.
This results in the code below for our example.
{ \setlength{\zedindent}{0.7cm}
\begin{circus}
  Running \circdef \\
  \t1 \circif frameStack = \emptyset \circthen \Skip \\
  \t1 {} \circelse frameStack \neq \emptyset \circthen {} \\
  \t2 \circif pc = 0 \circthen {} \cdots {} \\
  \t2 {} \cdots {} \\
  \t2 {} \circelse pc = 39 \circthen HandleAloadEPC(4) \circseq pc := 40 \circseq Poll \circseq \cdots \circseq \\
  \t3 pc := \IF value1 \leq value2 \THEN 21 \ELSE 42 \circseq Poll \circseq \\
  \t3 \circif value1 \leq value2 \circthen \cdots \circseq pc := 39 \circseq Poll \circseq Running \\
  \t3 {} \circelse value1 > value2 \circthen HandleReturnEPC \circseq Poll \circseq Running \\
  \t3 \circfi \\
  \t2 {} \cdots {} \\
  \t2 \circfi \\
  \t1 \circfi
\end{circus}}
Lemmas~\ref{loop-introduction-in-arbitrary-context-lemma}
and~\ref{conditional-loop-introduction-in-arbitrary-context-lemma} are
then used to introduce the loops at the points where they occur, on
lines~\ref{algorithm-introduce-nested-loop}
and~\ref{algorithm-introduce-nested-conditional-loop} of the
algorithm. 
These lemmas allow loops and conditional loops to be introduced in an
arbitrary context, loops and conditional loops to be introduced in an
arbitrary context.
These lemmas make use of context notation, in which $A[B]$ for a
particular action $B$ represents and action $A$ in which there is at
least one occurrence of $B$. 
Replacing $A[B]$ with an action $A[C]$ then means replacing all
occurrences of $B$ in $A$ with $C$.
These lemmas operate similarly to Lemmas~\ref{loop-introduction-lemma}
and~\ref{conditional-loop-introduction-lemma}.
\begin{lem}[Loop introduction in arbitrary context]
  If
  \begin{circus}
    \{ frameStack \neq \emptyset \} \circseq A[pc := i \circseq Poll \circseq X] \\
    {} = {} \\
    \{ frameStack \neq \emptyset \} \circseq A[\{ frameStack \neq \emptyset \} \circseq pc := i \circseq Poll \circseq X]
  \end{circus}
  then,
  \label{loop-introduction-in-arbitrary-context-lemma}
  \setlength{\zedindent}{0pt}
  \setlength{\zedtab}{1.3em}
  \begin{circus}
    \begin{array}{l}
      \circmu X \circspot \\
      \t1 \circif frameStack = \emptyset \circthen \Skip \\
      \t1 {} \circelse frameStack \neq \emptyset \circthen {} \\
      \t2 \circif \cdots \\
      \t2 {} \circelse pc = i \circthen {} \\
      \t3 A[pc := i \circseq Poll \circseq X] \\
      \t2 {} \cdots {} \\
      \t2 \circfi \\
      \t1 \circfi 
    \end{array}
    \circrefines_A
    \begin{array}{l}
      \circmu X \circspot \\
      \t1 \circif frameStack = \emptyset \circthen \Skip \\
      \t1 {} \circelse frameStack \neq \emptyset \circthen {} \\
      \t2 \circif \cdots \\
      \t2 {} \circelse pc = i \circthen \circmu Y \circspot \\
      \t3 A[pc := i \circseq Poll \circseq Y] \\
      \t2 {} \cdots {} \\
      \t2 \circfi \\
      \t1 \circfi  
    \end{array}
  \end{circus}
\end{lem}%
\begin{lem}[Conditional loop introduction in arbitrary context]
  \label{conditional-loop-introduction-in-arbitrary-context-lemma}
  If $i \neq j$ and
  \begin{circus}
    \{ frameStack \neq \emptyset \} \circseq A[pc := \IF b \THEN i \ELSE j \circseq  Poll \circseq X] \\
    {} = {} \\
    \{ frameStack \neq \emptyset \} \circseq A[\{ frameStack \neq \emptyset \} \circseq pc := \IF b \THEN i \ELSE j \circseq  Poll \circseq X]
  \end{circus}
  then,
  \setlength{\zedindent}{0pt}
  \setlength{\zedtab}{1.3em}
  \begin{circus}
    \begin{array}{l}
      \circmu X \circspot \\
      \t1 \circif frameStack = \emptyset \circthen \Skip \\
      \t1 {} \circelse frameStack \neq \emptyset \circthen {} \\
      \t2 \circif \cdots \\
      \t2 {} \circelse pc = i \circthen {} \\
      \t3 A[pc := \IF b \THEN i \ELSE j \circseq \\
      \t4 Poll \circseq X] \\
      \t2 {} \cdots {} \\
      \t2 {} \circelse pc = j \circthen B \circseq Poll \circseq X \\
      \t2 {} \cdots {} \\
      \t2 \circfi \\
      \t1 \circfi 
    \end{array}
    \circrefines_A
    \begin{array}{l}
      \circmu X \circspot \\
      \t1 \circif frameStack = \emptyset \circthen \Skip \\
      \t1 {} \circelse frameStack \neq \emptyset \circthen {} \\
      \t2 \circif \cdots \\
      \t2 {} \circelse pc = i \circthen \circmu Y \circspot \\
      \t3 A[pc := \IF b \THEN i \ELSE j \circseq Poll \circseq \\
      \t4 \circif b \circthen Y \\
      \t4 {} \circelse \lnot b \circthen B \circseq Poll \circseq X \\
      \t4 \circfi {} ] \\
      \t2 {} \cdots {} \\
      \t2 {} \circelse pc = j \circthen B \circseq Poll \circseq X \\
      \t2 {} \cdots {} \\
      \t2 \circfi \\
      \t1 \circfi  
    \end{array}
  \end{circus}
\end{lem}
After introducing the loops inside conditionals, we distribute the
$Poll \circseq Running$ actions back outside the conditionals, on
line~\ref{algorithm-move-running-loop-outside} of the algorithm.
For our example, this results in the following code after the loop at
$pc = 39$ has been introduced.
{ \setlength{\zedindent}{0.7cm}
\begin{circus}
  Running \circdef \\
  \t1 \circif frameStack = \emptyset \circthen \Skip \\
  \t1 {} \circelse frameStack \neq \emptyset \circthen {} \\
  \t2 \circif pc = 0 \circthen {} \cdots {} \\
  \t2 {} \cdots {} \\
  \t2 {} \circelse pc = 39 \circthen \circmu Y \circspot HandleAloadEPC(4) \circseq pc := 40 \circseq Poll \circseq \cdots \circseq \\
  \t3 pc := \IF value1 \leq value2 \THEN 21 \ELSE 42 \circseq Poll \circseq \\
  \t3 \circif value1 \leq value2 \circthen \cdots \circseq pc := 39 \circseq Poll \circseq Y \\
  \t3 {} \circelse value1 > value2 \circthen HandleReturnEPC \\
  \t3 \circfi \\
  \t2 {} \cdots {} \\
  \t2 \circfi \circseq Poll \circseq Running \\
  \t1 \circfi
\end{circus}}

Then, as discussed previously, we continue to introduce conditionals
in the next innermost loop, working outwards until all the loops and
conditionals in the whole method are introduced, making the method
complete as specified by the loop on
line~\ref{algorithm-loop-until-complete} of the algorithm.
In our example there are no further loops in the method so we consider
all the remaining nodes in the method's control flow graph.
All that remains to be done is to introduce a sequential composition
between the instructions at $pc = 7$ and the loop at $pc = 39$.
This results in the code shown for $pc = 7$ in
Figure~\ref{loop-and-conditional-introduction-example-figure} in
Section~\ref{overview-subsection}.

\subsection{Resolve Method Calls}
\label{resolve-method-calls-subsection}

When a method is complete, calls to that method can then be resolved.
This is performed after introduction of loops and conditionals,
ensuring methods with loops and conditionals are complete so that this
step can be applied.
% As mentioned previously, since this requires all the method calls in a
% given method to be resolved first, we do not allow recursion.

This step begins with the copying of the method into a separate
action, so that it can be referenced elsewhere.
This is performed by as described by
Algorithm~\ref{separate-complete-methods-algorithm}.
\begin{algorithm}
  \begin{algorithmic}[1]
    \Procedure{SeparateCompleteMethods}{}
    \For{$m \gets methods$} \label{algorithm-method-separation-loop}
    % \For{$mc \gets$ \Call{MethodsCalls}{$m$}}
    % \EndFor
    \If{\Call{MethodIsComplete}{$m$}} \label{algorithm-check-method-completeness}
    \State \Call{ApplyCopyRule}{$m$}
    \EndIf
    \EndFor
    \EndProcedure
  \end{algorithmic}
  \caption{Separate Complete Methods}
  \label{separate-complete-methods-algorithm}
\end{algorithm}

Algorithm~\ref{separate-complete-methods-algorithm} looks at each
method separately, as specified by the loop on
line~\ref{algorithm-method-separation-loop}, and determines if it is
complete, on line~\ref{algorithm-check-method-completeness}.
This involves a simple syntactic check that each conditional branch
ends in a return instruction or a recursion.
Those methods that are complete are moved into a separate action by an
application of the copy rule.

In our example, the method \texttt{f()} of the \texttt{TPK} class,
which starts at $pc = 43$, is complete on the first iteration of the
loop on line~\ref{algorithm-method-loop} of
Algorithm~\ref{epc-algorithm}, with the $Running$ action as shown
below.
The method is complete in this case because it consists of a straight
sequence of instructions ending with $HandleAreturnEPC$, which
represents the \texttt{areturn} instruction.
\begin{circus}
  Running \circdef \\
  \t1 \circif frameStack = \emptyset \circthen \Skip \\
  \t1 {} \circelse frameStack \neq \emptyset \circthen {} \\
  \t2 \circif pc = 0 \circthen {} \cdots {} \\
  \t2 {} \cdots {} \\
  \t2 {} \circelse pc = 43 \circthen HandleAloadEPC(0) \circseq Poll \circseq HandleAloadEPC(0) \circseq \\
  \t3 Poll \circseq HandleIaddEPC \circseq Poll \circseq HandleAloadEPC(0) \circseq Poll \circseq \\
  \t3 HandleIaddEPC \circseq Poll \circseq HandleIconstEPC(5) \circseq Poll \circseq \\
  \t3 HandleIaddEPC \circseq Poll \circseq HandleAreturnEPC \\
  \t2 {} \cdots {} \\
  \t2 \circfi \circseq Poll \circseq Running \\
  \t1 \circfi
\end{circus}
The sequence of instructions at $pc = 43$ can then be copied into a
separate action, shown below.
The name of this action contains the name of the class and method
identifier of the method it represents.
\begin{circus}
  TPK\_f \circdef HandleAloadEPC(0) \circseq Poll \circseq HandleAloadEPC(0) \circseq Poll \circseq \\
  \t1 HandleIaddEPC \circseq Poll \circseq HandleAloadEPC(0) \circseq Poll \circseq HandleIaddEPC \circseq \\
  \t1 Poll \circseq HandleIconstEPC(5) \circseq Poll \circseq  HandleIaddEPC \circseq Poll \circseq \\
  \t1 HandleAreturnEPC 
\end{circus}

After all the complete methods have been copied into separate actions,
calls to those methods are resolved.
This is performed as described by
Algorithm~\ref{resolve-method-calls-algorithm}.
\begin{algorithm}
  \begin{algorithmic}[1]
    \Procedure{ResolveMethodCalls}{}
    \For{$m \gets methods$}
    \For{$mc \gets$ \Call{UnresolvedMethodsCalls}{$m$}}
    \State $targets \gets$ \Call{DetermineMethodCallTargets}{$mc$}
    \If{$\# targets = 1$}
    \State \Call{ApplyLemma\ref{method-call-resolution-lemma}}{}
    \Else
    \State \Call{ApplyLemma\ref{dynamic-method-call-resolution-lemma}}{}
    \EndIf
    \EndFor
    \EndFor
    \EndProcedure
  \end{algorithmic}
  \caption{Resolve Method Calls}
  \label{resolve-method-calls-algorithm}
\end{algorithm}


%%%%

To resolve a method call, the type of method call must be considered.
Some methods are handled directly by the SCJVM as they relate to the
SCJVM services.
Such methods are treated specially by the interpreter, communicating
with the launcher to perform the behaviour of the method.
In cases where the method invocation is simply handled by
communication with the launcher and then followed by execution of the
next instruction, the control flow can be introduced using
Lemma~\ref{sequence-introduction-lemma} as for other instructions with
simple control flow.

% TODO: discuss special methods with nested calls here

For method calls that do not require special handling, the control
flow is that of the corresponding method's action, followed by
execution of the next instruction.
If the method is called with static dispatch (as is the case with the
\texttt{invokespecial} and \texttt{invokestatic} instructions), the
correct method, and hence the corresponding action can be easily
determined.
The method call is then resolved using
Lemma~\ref{method-call-resolution-lemma}.
We require as a precondition of the lemma that the method action
returns to the return address stored on the stack, to ensure that the
control flow is resolved correctly.
This will be the case for a method where every path of execution ends
in a return or loop, since returns establish the condition and loops
will either lead to a return eventually or form an infinite loop that
may be followed by any action (including the assumption we require).
%
\begin{lem}[Method call resolution]
  \label{method-call-resolution-lemma}
  If an action $M$ is such that
  \[\{ (head~frameStack).storedPC = i \land frameStack = fs \} \circseq M \\
    {} = {} \\
    \{ (head~frameStack).storedPC = i \land frameStack = fs \}
    \circseq M \circseq \\
    \t1 \{ pc = i \land frameStack = tail~fs \}\]
  and $i \neq j$,
  \setlength{\zedindent}{0.25cm}
  \begin{circus}
    \begin{array}{l}
      \circmu X \circspot \\
      \t1 \circif frameStack = \emptyset \circthen \Skip \\
      \t1 {} \circelse frameStack \neq \emptyset \circthen {} \\
      \t2 \circif \cdots \\
      \t2 {} \circelse pc = i \circthen A \circseq \\
      \t3 \{ pc = k \} \circseq \\
      \t3 \lschexpract \exists retAddr? == pc+1 @ \\
      \t4 SetReturnAddr \rschexpract \circseq \\
      \t3 pc := j \\
      \t2 {} \circelse pc = k+1 \circthen B \\
      \t2 {} \circelse pc = j \circthen M \\
      \t2 \cdots \\
      \t2 \circfi \circseq Poll \circseq X \\
      \t1 \circfi 
    \end{array}
    \circrefines_A
    \begin{array}{l}
      \circmu X \circspot \\
      \t1 \circif frameStack = \emptyset \circthen \Skip \\
      \t1 {} \circelse frameStack \neq \emptyset \circthen {} \\
      \t2 \circif \cdots \\
      \t2 {} \circelse pc = i \circthen A \circseq \\
      \t3 \{ pc = k \} \circseq \\
      \t3 \lschexpract \exists retAddr? == pc+1 @ \\
      \t4 SetReturnAddr \rschexpract \circseq \\
      \t3 pc := j \circseq Poll \circseq M \circseq Poll \circseq B \\
      \t2 {} \circelse pc = k+1 \circthen B \\
      \t2 {} \circelse pc = j \circthen M \\
      \t2 \cdots \\
      \t2 \circfi \circseq Poll \circseq X \\
      \t1 \circfi 
    \end{array}
  \end{circus}
  %TODO: elimination of pc assignment
\end{lem}
%
For method calls that have dynamic dispatch (as in the case of
\texttt{invokevirtual} instructions), we must determine what classes
the method may be called on.
The control flow is then a choice over the class of the object the
method is called on, with the method action corresponding to that
class chosen.
As with the static case, we require the method action to return to the
return address stored on the stack.
In the dynamic case, this requirement must be met by all the method
actions so that the next instruction can be executed after the choice
of methods.
Lemma~\ref{dynamic-method-call-resolution-lemma} is used to introduce
this choice.
%
\begin{lem}[Dynamic method call resolution]
  \label{dynamic-method-call-resolution-lemma}
  If actions $M_1, \dots, M_n$ are such that
  \[\{ returnAddress = i \land frameStack = fs \} \circseq M_k \\
    {} = {} \\
    \{ returnAddress = i \land frameStack = fs \}
    \circseq M_k \circseq \{ pc = i \land frameStack = tail~fs \}\]
   for $k in \{ 1, \dots, n \}$ and $i \neq j$,
  \setlength{\zedindent}{0.25cm}
  \begin{circus}
    \begin{array}{l}
      \circmu X \circspot \\
      \t1 \circif frameStack = \emptyset \circthen \Skip \\
      \t1 {} \circelse frameStack \neq \emptyset \circthen {} \\
      \t2 \circif \cdots \\
      \t2 {} \circelse pc = i \circthen A(class, method) \circseq \\
      \t3 \{ class \in \{ c_1, \dots, c_n \}\} \circseq \\
      \t3 \lschexpract \exists retAddr? == pc+1 @ \\
      \t4 SetReturnAddr \rschexpract \circseq \\
      \t3 pc := entry(class, method) \\
      \t2 {} \circelse pc = k+1 \circthen B \\
      \t2 {} \circelse pc = j_1 \circthen M_1 \\
      \t2 \cdots \\
      \t2 {} \circelse pc = j_n \circthen M_n \\
      \t2 \circfi \circseq Poll \circseq X \\
      \t1 \circfi 
    \end{array}
    \circrefines_A
    \begin{array}{l}
      \circmu X \circspot \\
      \t1 \circif frameStack = \emptyset \circthen \Skip \\
      \t1 {} \circelse frameStack \neq \emptyset \circthen {} \\
      \t2 \circif \cdots \\
      \t2 {} \circelse pc = i \circthen A(class, method) \circseq \\
      \t3 \{ class \in \{ c_1, \dots, c_n \} \circseq \\
      \t3 \lschexpract \exists retAddr? == pc+1 @ \\
      \t4 SetReturnAddr \rschexpract \circseq \\
      \t3 pc := entry(class, method) \circseq Poll \circseq \\
      \t3 \circif class = c_1 \circthen M_1 \\
      \t3 \cdots \\
      \t3 {} \circelse class = c_n \circthen M_n \\
      \t3 \circfi \circseq Poll \circseq B \\
      \t2 {} \circelse pc = k+1 \circthen B \\
      \t2 {} \circelse pc = j_1 \circthen M_1 \\
      \t2 \cdots \\
      \t2 {} \circelse pc = j_n \circthen M_n \\
      \t2 \circfi \circseq Poll \circseq X \\
      \t1 \circfi
    \end{array}
  \end{circus}
  %TODO: elimination of pc assignment
\end{lem}

The resolution of loops, conditionals and methods is performed in a
loop until all the methods have been separated into their own action.
The remaining use of the program counter in the main actions of $Thr$
can then be eliminated as described in the next section.

\subsection{Refine Main Actions}
\label{refine-main-actions-subsection}

At this stage of the strategy, the only place that the program counter
is used is when the first method is started, when it is used to select
the method action to execute, which will then proceed without any need
for the program counter value.
This can be eliminated by replacing it with a choice over the method
rather than the program counter.
This must be performed in the two places that the $Running$ action
occurs:~the $MainThread$ and $NotStarted$ actions.
Since the main action of $Thr$ is a guarded choice of these actions
depending on whether its thread parameter is the $main$ thread, these
actions may be thought of as two alternative main actions for $Thr$.

The context of the $Running$ action in both of these main actions is
the same:~the frame stack has only one frame and the program counter
is set to the entry point of a method.
This means that the same lemma can be used for both main actions by
introducing an assumption that states that context.
Additionally, we can use the fact that each method action will, when
started with a frame stack containing a single frame, cause the frame
stack to become empty.
This allows us to eliminate the loop in $Running$, reducing it
entirely to a choice of method action from the class and method
identifier.
The overall transformation of $Running$ in its context is described by
Lemma~\ref{main-action-refinement-lemma}.
\begin{lem}[Main Action Refinement]
  \label{main-action-refinement-lemma}
  If $entry(c_i,m_i) = j_i$ for $i \in \{1, \dots, n\}$ and
  \begin{circus}
    \{ \# frameStack = 1 \} \circseq M_i \\
    {} = {} \\
    \{ \# frameStack = 1 \} \circseq M_i \circseq \{ frameStack = \emptyset \}
  \end{circus}
  \setlength{\zedindent}{0.25cm}
  \begin{circus}
    \begin{array}{l}
      \{ \# frameStack = 1 \\
      \t1 {} \land pc = entry(cid, mid) \} \circseq \\
      \circmu X \circspot \\
      \t1 \circif frameStack = \emptyset \circthen \Skip \\
      \t1 {} \circelse framestack \neq \emptyset \circthen {}  \\
      \t2 \circif pc = j_1 \circthen M_1 \\
      \t2 \cdots \\
      \t2 {} \circelse pc = j_n \circthen M_n \\
      \t2 \circfi \circseq Poll \circseq X \\
      \t1 \circfi
    \end{array}
    \circrefines_A
    \begin{array}{l}
      \{ \# frameStack = 1 \} \circseq \\
      \circif (cid, mid) = (c_1, m_1) \circthen M_1 \\
      \cdots \\
      {} \circelse (cid, mid) = (c_n, m_n) \circthen M_n \\
      \circfi \\
    \end{array}
  \end{circus}
\end{lem}
Once this has been performed, the program counter value is no longer
used to determine the control flow of the program, so a trivial data
refinement can be performed to eliminate $pc$ from the state of the
$Thr$ process.

\subsection{Remove $pc$ From State}
\label{remove-pc-from-state-subsection}

\section{Elimination of Frame Stack}
\label{elimination-of-frame-stack-section}

\section{Data Refinement of Objects}
\label{data-refinement-of-objects-section}

