\documentclass{beamer}
\usetheme[compress]{Dresden}
\setbeamertemplate{navigation symbols}{}
\setbeamertemplate{footline}{
  \leavevmode%
  \hbox{%
  \begin{beamercolorbox}[wd=.333333\paperwidth,ht=2.25ex,dp=1ex,center]{author in head/foot}%
    \usebeamerfont{author in head/foot}\insertshortauthor~~(\insertshortinstitute)
  \end{beamercolorbox}%
  \begin{beamercolorbox}[wd=.333333\paperwidth,ht=2.25ex,dp=1ex,center]{title in head/foot}%
    \usebeamerfont{title in head/foot}\insertshorttitle
  \end{beamercolorbox}%
  \begin{beamercolorbox}[wd=.333333\paperwidth,ht=2.25ex,dp=1ex,right]{date in head/foot}%
    \usebeamerfont{date in head/foot}\insertshortdate{}\hspace*{2em}
    \insertframenumber{} \hspace*{2ex}
  \end{beamercolorbox}}%
  \vskip0pt%
} 
\usecolortheme{rose}

\renewcommand{\insertsectionnumber}{}
\renewcommand{\sectionname}{}

\newcommand{\beginbackup}{
   \newcounter{framenumbervorappendix}
   \setcounter{framenumbervorappendix}{\value{framenumber}}
}
\newcommand{\backupend}{
   \addtocounter{framenumbervorappendix}{-\value{framenumber}}
   \addtocounter{framenumber}{\value{framenumbervorappendix}} 
}


%\setbeamertemplate{headline}
%{%
%  \begin{beamercolorbox}[ht=3.5ex,dp=1.125ex,%
%      leftskip=.3cm,rightskip=.3cm plus1fil]{section in head/foot}
%    \usebeamerfont{section in head/foot}\usebeamercolor[fg]{section in head/foot}%
%  \insertsectionhead
% \end{beamercolorbox}%
%  \begin{beamercolorbox}[colsep=1.5pt]{middle separation line head}
%  \end{beamercolorbox}
%  \begin{beamercolorbox}[ht=2.5ex,dp=1.125ex,%
%    leftskip=.3cm,rightskip=.3cm plus1fil]{subsection in head/foot}
%    \usebeamerfont{subsection in head/foot}\insertsubsectionhead
%  \end{beamercolorbox}%
%  \begin{beamercolorbox}[colsep=1.5pt]{lower separation line head}
%  \end{beamercolorbox}
%}
%{}%remove navigation symbols
%\useoutertheme[subsection=false]{miniframes}

\usepackage{verbatim}
\usepackage[absolute,overlay]{textpos}
\usepackage{amsmath,listings}
\usepackage[backend=bibtex,style=numeric-comp,sorting=none,sortcites=true,isbn=false,url=false,doi=false,block=space,defernumbers=true]{biblatex}
\definecolor{darkblue}{rgb}{0,0,0.5}
\definecolor{darkgreen}{rgb}{0,0.5,0}
\lstset{language=[LaTeX]TeX,
  basicstyle={\color{red}},
  morekeywords={%
    part,chapter,subsection,subsubsection,%
    paragraph,subparagraph,%
    appendix}}

\usepackage{tikz}

\title{Correctness of Compilers for Java-like Languages}
\author{James Baxter}
\date[2014-12-9]{Literature Review Seminar \\ 9 December 2014}

\renewcommand{\footnotesize}{\tiny}
\renewcommand{\printbibheading}{}
\defbibheading{bibliography}{}

\setbeamertemplate{bibliography item}[text]
\setbeamerfont{bibliography entry author}{shape=\scshape,size=\tiny}
\setbeamerfont{bibliography entry title}{shape=\scshape,size=\tiny}
\setbeamerfont{bibliography entry journal}{shape=\scshape,size=\tiny}
\setbeamerfont{bibliography entry note}{shape=\scshape,size=\tiny}
\setbeamerfont{bibliography item}{shape=\scshape,size=\tiny}
\renewcommand{\bibfont}{\tiny}

\setbeamertemplate{footnote}{\insertfootnotetext}

\setlength{\abovecaptionskip}{-2ex} 
\setlength{\belowcaptionskip}{-2ex} 
\newcommand{\tab}{\hspace*{2em}}
%\bibliographystyle{apalike}  
\bibliography{literature} 
\setbeamertemplate{footline}[frame number]

%\newenvironment{slide}[1]%
%{\begin{refsegment}
%\begin{frame}[fragile,environment=slide]
%\frametitle{#1}
%}%
%{
%\footnotetext{\printbibliography[segment=1]}
%\end{frame}
%\end{refsegment}
%}
\usepackage{xifthen}% provides \isempty test
\newcommand{\footmake}[1]{
\ifthenelse{\equal{#1}{}}%
	{}%
	{\footnotetext{#1}}%
}

%\defbibenvironment{bibliography}
%{\begin{minipage}[b]{\linewidth}}
%{\end{minipage}}
%{}

\defbibenvironment{bibliography}
{
 \list{\printtext[labelnumberwidth]{%
    \printfield{prefixnumber}%
    \printfield{labelnumber}}}
{\setlength{\bibhang}{0pt}%
\setlength{\leftmargin}{\bibhang}%
\setlength{\itemindent}{-\leftmargin}%
\setlength{\itemsep}{\bibitemsep}%
\setlength{\parsep}{\bibparsep}}}
{\endlist}
{\item}

\newsavebox{\ptbib}

\newenvironment{slide}[2][fragile,environment=slide]
{\begin{frame}[#1]
	\frametitle{#2}\begin{refsegment}}
{\footmake{\printbibliography[segment=\therefsegment]}\end{refsegment}\end{frame}}

\begin{document}

\frame[plain]{\titlepage}

\begin{frame}
\frametitle{Outline}
\tableofcontents
\end{frame}

\section{Motivation}

\begin{slide}{Motivation}
\begin{itemize}
\item Popularity of Java\cite{gosling2013}
\item Variants of Java: JavaCard\cite{chen2000}, RTSJ\cite{gosling2000}, SCJ\cite{locke2013}, Java ME\cite{oracle2014}
\item Program correctness relies on compiler/interpreter correctness.
\item Testing is usually not sufficient to ensure correctness.
\end{itemize}
\end{slide}

\section{Compiler Correctness}

\frame{\sectionpage}

\subsection{Early Work}

\begin{slide}{Early Work}
  \begin{itemize}
  \item McCarthy and Painter\cite{mccarthy1967}
    \begin{itemize}
    \item Source: simple single-operator expression language
    \item Target: simple four-instruction single-register machine
    \item Definition of Correctness: partial equality of machine states
    \end{itemize}
  \item Burstall and Landin\cite{burstall1969}
    \begin{itemize}
    \item Source: expression language similar to McCarthy and Painter's, but allowing for more operators.
    \item Target: several intermediate machines, including a stack machine and a machine similar to McCarthy and Painter's.
    \item Definition of Correctness: construction of homomorphisms between algebras
    \end{itemize}
  \end{itemize}
\end{slide}

\begin{slide}{Early Work}
  \begin{itemize}
  \item Milner and Weyhrauch\cite{milner1972}
    \begin{itemize}
    \item Source: simple ALGOL-like language
    \item Target: stack machine that allows jumps
    \item Definition of correctness: equality between the source semantics and the composition of the compilation function, the target semantics and a function to extract the source state of the machine
    \item Partially mechanised proof using LCF
    \end{itemize}
  \end{itemize}
\end{slide}

\begin{slide}{Early Work}
  In general, the traditional approach to compilation is based on showing that a diagram of the following form commutes\cite{morris1973, thatcher1979}.
  \begin{center}
    \begin{tikzpicture}
      \node[align=center] (S) at (0cm,3cm) {source\\language};
      \node[align=center] (T) at (4cm,3cm) {target\\language};
      \node[align=center] (M) at (0cm,0cm) {source\\meanings};
      \node[align=center] (U) at (4cm,0cm) {target\\meanings};
      
      \path (S) edge[->] node[align=center, above] {compile}           (T);
      \path (S) edge[->] node[align=right, left]   {source\\semantics} (M);
      \path (T) edge[->] node[align=left, right]   {target\\semantics} (U);
      \path (M) edge[->] node[align=center, above] {encode}            (U);
    \end{tikzpicture}
  \end{center}
\end{slide}

\subsection{Algebraic Approach}

\begin{slide}{Algebraic Approach}
  \begin{itemize}
  \item Proposed by Hoare in 1991\cite{hoare1991}
  \item Source and target in the same specification language with laws for reasoning about them \cite{hoare1987}.
  \item Refinement relation between programs: $P \sqsubseteq Q$ means $Q$ is at least as good as $P$ \cite{back1981, morris1987, morgan1990}.
  \item Refines source program to a normal form that resembles an interpreter for the target machine.
  \end{itemize}
\end{slide}

\begin{frame}
  \frametitle{Algebraic Approach}
  \begin{tikzpicture}
    \draw (0cm,0cm) |- (10cm,5cm) |- (0cm,0cm);
    \draw (5cm,4.2cm) node[align=center] {Source Language\\+\\Specification Constructs};

    \draw (2cm,2cm)   circle[radius=1.5cm]               node[align=center] {Source Program\\(in Source\\Language)};
    \draw (8cm,2cm)   circle[radius=1.5cm] ++(0cm,0.7cm) node[align=center] {Target\\Machine};
    \draw (8cm,1.5cm) circle[radius=0.7]                 node[align=center] {Target\\Code};

    \path (3.5cm,2cm) edge[->] 
              node[align=center, above] {\LARGE $\sqsubseteq$} 
              node[align=center, below] {Refinement\\Strategy\\(Compilation)}  
          (6.5cm,2cm);
  \end{tikzpicture}
\end{frame}

\begin{slide}{Algebraic Approach}
  \begin{itemize}
  \item Sampaio\cite{hoare1993, sampaio1993, sampaio1997}
    \begin{itemize}
    \item Source: basic imperative language, including procedures and recursion
    \item Target: Simple single-register machine with jumps
    \item Compilation using rewrite rules proved from basic laws.
    \item Mechanisation in the OBJ3 term rewriting system\cite{goguen1988}.
    \end{itemize}
  \end{itemize}
\end{slide}

\begin{slide}{Algebraic Approach}
  \begin{itemize}
  \item Perna\cite{perna2010, perna2011} --- Hardware compilation
    \begin{itemize}
    \item Source: Handel-C
    \item Target: Hardware components, FPGAs
    \item Handles timed structures and parallelism with shared variables
    \item Refines complex structures to simple ones for implementation in hardware
    \item Laws proved from a denotational semantics expressed in the Unifying Theories of Programming\cite{hoare1998}.
    \end{itemize}
    % \item \cite{hoare1984}\cite{morgan1990}
  \end{itemize}
\end{slide}

\subsection{Recent Work}

\begin{slide}{Recent Work}
  \begin{itemize}
  \item CompCert\cite{leroy2012}
    \begin{itemize}
    \item Project to develop a realistic formally verified compiler
    \item Formally verified compiler for a large subset of C
    \item Mechanised in the Coq proof assistant
    \end{itemize}
  \item Wang, Cuellar and Chipala\cite{wang2014}
    \begin{itemize}
    \item Verifying compilers that allow linking with other languages
    \item Combined algebraic and operational semantics of the source language
    \item Algebraic reasoning about calls to programs in other languages
    \item Mechanised in Coq 
    \end{itemize}
  \end{itemize}
\end{slide}

\section{Compiler Correctness for Java-like Languages}

\frame{\sectionpage}

\subsection{Compilation of Java}

\begin{frame}
  \frametitle{Compilation of Java}
  \begin{itemize}
  \item Java is usually compiled to Java bytecode, which is run on the JVM.
  \item The JVM may either interpret the bytecode or compile it to native code.
  \item Both the initial compilation and the JVM must be proved correct
  \end{itemize}
  \begin{center}
    \begin{tikzpicture}
      \draw (3cm,0cm)   circle[radius=1cm]               node[align=center] {Java code};
      \draw (4cm,0cm) edge[->] node[above] {compile} (7cm,0cm);
      \draw (8cm,0.3cm) circle[radius=1.5cm] ++(0cm,1cm) node[align=center] {JVM};
      \draw (8cm,0cm)   circle[radius=1cm]               node[align=center] {Java\\bytecode};
    \end{tikzpicture}
  \end{center}
\end{frame}

\subsection{Compiler Correctness for the whole of Java}

\begin{slide}{ASM approach}
  St\"{a}rk, Schmid and B\"orger\cite{stark2001}
  \begin{itemize}
  \item Source: the whole of Java
  \item Target: JVM with a few instructions not supported
  \item Definition of Correctness:
    \begin{itemize}
    \item Defines Java and the JVM in terms of abstract state machines (ASMs)
    \item Splits Java into sublanguages: imperative, procedural, object-oriented, exception-handling, concurrent
    \item Requires equivalences between the Java ASM and the JVM ASM to be satisfied for the compilation to be correct
    \item Proves correctness by induction over the structure of Java code
    \end{itemize}
  \end{itemize}
\end{slide}

\subsection{Compiler Correctness for subsets of Java}

\begin{slide}{ROOL}
%borba2000, borba2004, cavalcanti2000
  Duran\cite{duran2005, duran2010}
  \begin{itemize}
  \item Source: ROOL (Refinement Object-Oriented Language)\cite{cavalcanti2000}
    \begin{itemize}
    \item Subset of Java with specification features
    \item Laws developed for reasoning about object-oriented languages\cite{borba2000}
    \end{itemize}
  \item Target: RVM (ROOL Virtual Machine)
    \begin{itemize}
    \item Similar to a JVM running a subset of Java bytecode
    \end{itemize}
  \item Definition of Correctness: algebraic approach
  \end{itemize}
\end{slide}

\begin{slide}{Java Compilers in Isabelle/HOL}
  \begin{itemize}
  \item Strecker\cite{strecker2002}
    \begin{itemize}
    \item $\mu$Java --- contains many core features of Java but not interfaces, arrays, access modifiers and concurrency
    \item Definition of Correctness: ``commuting diagram'' argument
    \end{itemize}
  \item Klein and Nipkow\cite{klein2006}
    \begin{itemize}
    \item Slightly larger subset of Java called Jinja
    \end{itemize}
  \item Lochbihler\cite{lochbihler2010}
    \begin{itemize}
    \item Adds support for concurrency
    \end{itemize}
  \end{itemize}
\end{slide}

\begin{slide}{Embedded Systems}
  \begin{itemize}
  \item Schultz\cite{schultz2003} --- compiling Java to native code
  \item Varma and Bhattacharyya\cite{varma2004} --- compiling Java to C
  \item Icecap Hardware Virtual Machine (HVM)\cite{sondergaard2012, korsholm2014} --- compiling SCJ to C
  \item No formal verification work done
  \end{itemize}
%schultz2003, varma2004
\end{slide}

\section{Conclusion}

\begin{frame}
\frametitle{Conclusion}
  \begin{itemize}
  \item There are two main approaches to compilation:
    \begin{itemize}
    \item algebraic approach
    \item traditional approach based on commuting diagrams
    \end{itemize}
  \item Work has been done on verifying compilers for various languages, including Java
  \item There seems to be little work on verifying Java compilers for embedded systems
  \end{itemize}
\end{frame}

\begin{frame}[fragile]
\frametitle{}
\begin{center}
{\huge Any Questions?}
\end{center}
\end{frame}

\beginbackup

\backupend

\end{document}
