\documentclass[a4paper,12pt]{article}

\usepackage{circus}
\usepackage{fullpage}


\title{An Approach to Verification of Safety-Critical Java Virtual
  Machines with Ahead-of-time Compilation
  \\[1cm]
  \Large Thesis Audit}
\author{James Baxter}

\begin{document}
\maketitle

This document describes the progress on our work and on writing the
thesis. 
The structure of the thesis has not changed substantially since the
Thesis Outline document so we just provide an update on the work for
each of the chapters proposed there. 
Each of the following sections corresponds to a chapter of the thesis
and provides an overview of the work carried out for that chapter and
the work that remains to be done.

\section{Introduction}

This chapter just introduces the content of the thesis and describes
its structure. 
Because of this, there is no specific research work associated with
this chapter. 
The content of the chapter has already been written and will only
require very minor revision for the final thesis.

\section{Compilers and Virtual Machines for Java-like languages in the
  Safety-critical Domain}

This chapter is a literature review chapter, so the only work required
for it is that of reviewing the literature, which has already been
performed.
The content of this chapter has already been written and will only
require minor changes before the submission of the final thesis,
taking into account new developments in the literature.

\section{Safety-Critical Java Virtual Machine Services}

This chapter describes the first part of our model of Safety-Critical
Java Virtual Machines (SCJVMs). 
All of the work on the model itself has been completed, and it is
expected that it will not require any further changes.
An explanation of the model, which forms the content of this chapter
of the thesis, has been written but will need to be updated to reflect
changes in the model.
This should take no longer than a month.

\section{Core Execution Environment}

This chapter describes the second part of our SCJVM model, which forms
the starting point for our compilation strategy, and our shallow
embedding of C in \Circus{}, which forms the target of our compilation
strategy.
As for the first part of the model, the work on the second part of the
SCJVM model has already been completed and an explanation of the model
has been written but the explanation will need to be revised to tske
into account changes to the model.
However, the changes required for this should be simpler than for the
previous chapter as the changes tothe model make it easier to present.
The work on our shallow embedding of C is also complete and a
description of it has been written as part of preparing a paper for
submission to a conference.
However, this description is yet to be expanded and incorporated into
the thesis. 
Work on the content of this chapter (both the model explanation and
the shallow embedding description) is expected to take no more than a
few weeks.

\section{Compilation Strategy}

This chapter describes our compilation strategy to translate Java
bytecode to C.
The overall structure of this strategy has already been decided and we
have a plan for how each stage of the strategy should proceed.
An overview of the strategy has been written as part of the conference
paper mentioned in the previous section of this document, but needs adjusting to fit
with changes to the model and incorporating into the thesis.
The changes to the model that affect the compilation strategy all work
to simplify the presentation of the strategy, so the adjustments to
the overview should not take long.

For the remaining sections of this thesis chapter, writing of the
thesis is being performed alongside working out the details of each
stage of the strategy, so that the thesis forms the main account of
the compilation strategy.
Writing of the section for the first stages of the strategy,
\emph{Elimination of Program Counter}, is underway and this section is
already well developed.
The sections for remaining stages of the strategy, \emph{Elimination
  of Stack} and \emph{Data Refinement of Memory}, though they mainly
comprise data refinements that should be relatively simple to present.
The \emph{Collapsing of Intepreter and Launcher} stage of the
strategy, which was mentioned in the Thesis Outline, was deemed to not
be necessary so it no longer requires a section in the thesis.
Overall, we expect the work on this chapter to take no longer than 2
months, based on the time taken for work so far on the Elimination of
Program Counter section.

\section{Evaluation}

\section{Conclusions}

\end{document}